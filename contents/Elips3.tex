% 椭圆
% 极坐标系|直角坐标系|圆锥曲线|椭圆

\pentry{圆锥曲线的极坐标方程\upref{Cone}}

我们已经知道用焦点和准线如何定义椭圆\upref{Cone}, 下面介绍另外三种。 其中 “圆锥截面定义” 揭示了 “圆锥曲线” 一词的由来。

\subsection{用直角坐标方程定义椭圆}
从椭圆的极坐标公式难以看出椭圆的对称性, 另一种定义椭圆的方法是直接在直角坐标系中给出椭圆的方程
\begin{equation}\label{eq_Elips3_3}
\frac{x^2}{a^2} + \frac{y^2}{b^2} = 1
\end{equation}
这相当于把一个单位圆(方程 $x^2 + y^2 = 1$)在 $x$ 轴和 $y$ 轴分别拉长了 $a$ 倍和 $b$ 倍。 我们这里用焦点和准线的定义来推导出上式, 以证明它们等价。 我们不妨先以一个焦点为原点定义直角坐标系, 且令 $x$ 轴指向另一个焦点, 则有
\begin{equation}
r = \sqrt{x^2 + y^2}~, \qquad \cos\theta = \frac{x}{\sqrt{x^2 + y^2}}
\end{equation}
代入椭圆的极坐标方程\autoref{eq_Cone_5}~\upref{Cone} 得
\begin{equation}
\sqrt{x^2 + y^2} = p + ex
\end{equation}
两边平方并整理得
\begin{equation}\label{eq_Elips3_2}
(1 - e^2) \qty( x - \frac{ep}{1 - e^2} )^2 + y^2 = \frac{p^2}{1 - e^2}
\end{equation}
由此可见,如果我们把椭圆左移 $ep/(1 - e^2)$,椭圆将具有\autoref{eq_Elips3_3} 的形式。 其中 $a$ 为\textbf{半长轴}, $b$ 为\textbf{半短轴}。这就是椭圆的第二种定义, 即把单位圆沿两个垂直方向分别均匀拉长 $a$ 和 $b$。 所以也可以表示为参数方程
\begin{equation}\label{eq_Elips3_1}
\leftgroup{
&x(t) = a\cos t\\
&y(t) = b\sin t
} \qquad
(a > b > 0)
\end{equation}

下面来看系数的关系。首先定义椭圆的焦距为焦点到椭圆中心的距离(即以上左移的距离)为
\begin{equation}\label{eq_Elips3_5}
c = \frac{ep}{1 - e^2}
\end{equation}
\autoref{eq_Elips3_2} 和\autoref{eq_Elips3_3} 对比系数得
\begin{equation}\label{eq_Elips3_6}
a = \frac{p}{1 - e^2}~, \qquad b = \frac{p}{\sqrt {1 - e^2} }
\end{equation}
以上两式可以将椭圆的极坐标方程转为直角坐标方程。 另外易证
\begin{equation}\label{eq_Elips3_7}
a^2 = b^2 + c^2
\end{equation}
若要从直角坐标方程变回极坐标方程, 将\autoref{eq_Elips3_5} \autoref{eq_Elips3_6} 逆转得
\begin{equation}\label{eq_Elips3_8}
e = \frac{c}{a}~,\qquad
p = \frac{b^2}{a}
\end{equation}

\subsection{用焦点距离之和定义椭圆}
椭圆的另一种定义是, 椭圆上任意一点到两焦点的距离之和等于长轴 $2a$。 现在我们来证明前两种定义下的椭圆满足这个条件。 由直角坐标方程可知对称性,可在椭圆的两边做两条准线,令椭圆上任意一点到两焦点的距离分别为 $r_1$ 和 $r_2$,到两准线的距离分别为 $d_1$ 和 $d_2$,则有
\begin{equation}
e = \frac{r_1}{d_1} = \frac{r_2}{d_2} = \frac{r_1 + r_2}{d_1 + d_2}
\end{equation}
所以
\begin{equation}\label{eq_Elips3_9}
r_1 + r_2 = e(d_1+d_2) = 2e(c + h) = 2\frac{c}{a} \qty( c + \frac{b^2}{c} ) = 2a
\end{equation}
证毕。

\subsection{用圆锥截面定义椭圆}
椭圆之所以叫做圆锥曲线, 是因为它们可以由平面截取圆锥面得到, 详见“圆锥曲线和圆锥\upref{ConSec}”。
