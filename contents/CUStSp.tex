% 闭弦与无向弦的谱
% license Usr
% type Tutor


\subsection{闭弦的光锥量子化}
仍然施加开弦的规范条件,不过闭弦有额外自由度:
\begin{equation}
	\sigma' = \sigma + s(\tau) \mod l ~,
\end{equation}
而 $\sigma = 0$ 可以选为任意一点,故额外令 $\gamma_{\tau \sigma}(\tau, 0) = 0$。除了一个整体的平移以外,完全决定了 $\sigma = 0$ 的情况,现在除了 $\sigma$ 的与 $\tau$ 无关的平移外,固定了所有的规范自由度。

完全类比开弦可以得到闭弦的情况:
\begin{equation}\begin{aligned}
	X^i(\tau, \sigma) = x^i + \frac{p^i}{p^+}&\tau + \mathbf{i} \sqrt{\frac{\alpha'}{2}} \times \\
	&\sum\limits_{n = -\infty, n\neq 0}^{\infty} \left[ \frac{\alpha_n^i}{n} \exp\left(-\frac{2\pi \mathbf{i} n (\sigma + c\tau)}{l}\right) + \frac{\widetilde{\alpha}_n^i}{n} \exp\left(\frac{2\pi \mathbf{i} n (\sigma - c\tau)}{l}\right) \right] ~.
\end{aligned}\end{equation}
现在有两组独立的振动 $\alpha_n^i$ 和 $\widetilde{\alpha}_n^i$,对应弦上向左和向右的波。独立的自由度是:$\alpha_n^i, \widetilde{\alpha}_n^i, x^i, p^i, x^-, p^+$,赋予对易子 
\begin{equation}
	[x^-, p^+] = -\mathbf{i}, [x^i p^j] = \mathbf{i}\delta^{ij}, [\alpha_m^i, \alpha_n^j] = [\widetilde{\alpha}_m^i, \widetilde{\alpha}_n^j]  = m \delta^{ij} \delta_{m, -n} ~.
\end{equation}

仍是从态 $\ket{0, 0; k}$ 开始,有质心动量 $k^\mu$,可以被 $m>0$ 的 $\alpha_m^i$ 或 $\widetilde{\alpha}_m^i$ 湮灭。一般的态是 
\begin{equation}
	\ket{N, \widetilde{N}; k} = \left[\prod_{i=2}^{D-1} \prod_{n=1}^\infty \frac{(\alpha_{-n}^i)^{N_{i, n}}(\widetilde{\alpha}_{-n}^i)^{\widetilde{N}_{i, n}}}{\sqrt{n^{N_{i, n}} N_{i, n}! n^{\widetilde{N}_{i, n}} \widetilde{N}_{i, n}!}}\right] \ket{0, 0; k} ~.
\end{equation}
质量是 $m^2 = 2 p^+ H - p^i p^i = 2(N + \widetilde{N} + A + \widetilde{A})/\alpha'$。此外,求和仍给出 
\begin{equation}
	A = \widetilde{A} = \frac{2-D}{24} ~.
\end{equation}

对于剩下的规范自由度,即 $\sigma$ 的平移,进行进一步约束。生成 $\sigma$ 平移的算符是:
\begin{equation}
	\begin{aligned}
		P &= -\int_0^l \dd \sigma \Pi^i \partial_\sigma X^i \\
		&= -\frac{2\pi}{l} \left[\sum_{n=1}^\infty (\alpha_{-n}^i \alpha_n^i - \widetilde{\alpha}_{-n}^i \widetilde{\alpha}_n^i) + A - \widetilde{A}\right] \\
		&= -\frac{2\pi}{l} (N - \widetilde{N}) ~.
	\end{aligned}
\end{equation}
故需要 $N = \widetilde{N}$。

最轻的仍是 $\ket{0, 0; k}$,对应 $m^2$ 在 $D>2$ 时小于 $0$,是快子。而对于第一激发态,$N = \widetilde{N} = 1$,故是 
\begin{equation}
	\alpha_{-1}^i \widetilde{\alpha}_{-1}^j \ket{0, 0; k}, ~~ m^2 = \frac{26- D}{6 \alpha'} ~.
\end{equation}
仍不够一个 $\text{SO}(D-1)$ 的完全表示,故仍必须是无质量的,$A = \widetilde{A} = -1, D = 26$。

关注到,任何能级下,除 $N = \widetilde{N}$ 的约束外,对 $N_{i, n}$ 和 $\widetilde{N}_{i, n}$ 并无约束,故 $m^2 = 4(N-1)/\alpha'$ 处闭弦的谱其实是两对开弦能级 $m^2= (N-1)/\alpha'$ 的乘积。

$2$ 维 Diff-不变性从谱中移除了两类简正模。在尝试没有这个不变性的前提下构建协变理论,将推广横向对易子为 $[\alpha_m^\mu, \alpha_n^\nu] = m \eta^{\mu\nu} \delta_{m, -n}$,Lorentz 不变性使得类时振子有\textbf{负号对易子}(wrong-sign)。

\subsection{非定向弦}
若考虑坐标变换 $\sigma' = l-\sigma, \tau' = \tau$,这仅改变世界面的方向(手性)。可以考虑是由宇称算符 $\Omega$ 生成。变换两次将不变,故 $\Omega$ 的本征值是 $\pm 1$。可以看到,开弦中,
\begin{equation}
	\Omega \alpha_n^i \Omega^{-1} = (-1)^n \alpha_n^i ~.
\end{equation}
同时,闭弦中,
\begin{equation}
	\begin{aligned}
		\Omega \alpha_n^i \Omega^{-1} &= \widetilde{\alpha}_n^i ~,\\
		\Omega \widetilde{\alpha}_n^i \Omega^{-1} &= \alpha_n^o ~.
	\end{aligned}
\end{equation}

通过对于基态 $\ket{0; k}$、$\ket{0, 0; k}$,固定 $\Omega = +1$,而 
\begin{equation}
	\Omega \ket{N; k} = (-1)^N \ket{N; k} ~,
\end{equation}
\begin{equation}
	\Omega \ket{N, \widetilde{N}; k} = \ket{\widetilde{N}, N; k} ~.
\end{equation}
故必须对后者 $\Omega = + 1$。