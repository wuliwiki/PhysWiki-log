% 相对性原理
% license Xiao
% type Tutor

\begin{issues}
\issueDraft
\end{issues}


牛顿-伽利略相对性原理认为,一切力学定律在任意惯性参考系中均成立,后爱因斯坦将其适用范围推广至所有物理定律,并以此作为\enref{狭义相对论}{SpeRel}的基本假设之一。

进一步说,在不同参考系中尽管可以观察到物理量的不同(例如,测得物体的速度不同),但不会观察到物理定律的不同(物体的运动始终遵循$\bvec F=m \bvec a$)。

相对性原理意味着不存在一个“绝对静止的”的特殊惯性参考系;或者说,所有的参考系都是平等的。
