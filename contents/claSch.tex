% 薛定谔经典场
% keys 经典场论|薛定谔方程|理论力学
% license Usr
% type Tutor

\begin{issues}
\issueDraft
\end{issues}

\pentry{经典场论基础\nref{nod_classi}}{nod_29e2}
通常的场论都是相对论性的,满足洛伦兹协变,但是在很多情况下我们也要处理非相对论的场,例如在凝聚态系统中,我们需要处理诸如Gross-Pitaevskii方程等。因此在这里我们讨论一些从场论角度看非相对论量子力学的问题。
\subsection{薛定谔方程的场论表述}
我们知道对于单粒子,\enref{薛定谔方程为}{TDSE11}
\begin{equation}
i\hbar \frac{\partial}{\partial t} \Psi = \hat{H} \Psi = -\frac{\hbar^2 \nabla^2}{2m} \Psi + V(x)\Psi ~.
\end{equation}
拉氏量的构造可以有很多种,这里我们选取一种拉式量的构造方法,给出薛定谔方程。我们定义
$$
L =  \Psi^\star [i\hbar \frac{\partial}{\partial t}  - V(x)]\Psi - \frac{\hbar^2}{2m}\nabla \Psi \nabla \Psi^\star~,
$$
对应的作用量为
$$
S = \int dt \int d^3 x L~.
$$
根据\enref{欧拉-拉格朗日方程}{classi}
$$
\partial_u \frac{\partial L}{\partial (\partial_u \Psi)} - \frac{\partial L}{\partial \Psi}=0 ~,
$$
我们有
$$
i\hbar \partial_t \Psi^\star - \frac{\hbar^2}{2m}\nabla^2 \Psi^\star + V(x)\Psi^\star=0 ~,
$$
即薛定谔方程的共轭方程,对整个方程取复共轭,即得到薛定谔方程
$$
i\hbar \partial_t \Psi = -\frac{\hbar^2}{2m}\nabla^2 \Psi + V(x)\Psi ~.
$$
在这里场变量为$\Psi$,对应的正则动量为
$$
\pi \equiv \frac{\partial L}{\partial (\partial_t \Psi)} = i\hbar \Psi^\star~.
$$
做正则变换,给出哈密顿量
$$
H = \pi \frac{\partial}{\partial t}\Psi - L =  \frac{\hbar^2}{2m} \nabla\Psi \nabla \Psi^\star+ V(x)\Psi^\star\Psi~.
$$
这里哈密顿方程给出
$$
\frac{d}{d t}\Psi = \{H,\Psi\} = -\frac{\partial \Psi}{\partial \Psi}\frac{\partial H}{i\hbar \partial \Psi^\star}~.
$$
\subsection{场的定域性}
在一般的场论中,我们构造的拉格朗日量都是定域的,这也就是说,在拉氏量中,场之间的耦合(两个以上场算符乘在一起时称为耦合)都是在时空的邻域上。例如,我们不会写$\phi(\vec{x},t)\phi(\vec{y},t)$这样的项在拉氏量里。场论中只会通过梯度项$(\nabla \psi)^2$将临近的时空点上的场耦合起来。我们没有理由说明场为什么一定是定域的,但定域性却作为我们构造一切场论的基本前提,出现在科学实践中。\footnote{Quantum Field Theory,David Tong}