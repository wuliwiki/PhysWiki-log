% 可积函数
% 可积函数|性质|种类

\begin{issues}
\issueOther{子节2待证明}
\end{issues}

\pentry{定积分存在条件\upref{Rieman}}
可积函数的定义已经在\autoref{def_DInt_3}~\upref{DInt}给出,严格来说,这里的可积函数称为黎曼可积函数。本词条将给出一些常见的可积函数和可积函数的一些性质。
\subsection{一些可积函数}
\begin{theorem}{连续函数必可积}
定义在区间 $[a,b]$ 上的连续函数必在区间 $[a,b]$ 上可积。
\end{theorem}
\textbf{证明:}
由\autoref{sub_conff_1}~\upref{conff}的康托尔定理,连续函数在闭区间上必一致连续(\autoref{sub_conff_2}~\upref{conff})由此可证明:即对任意 $\epsilon>0$ 可找到 $\delta>0$,使得当 $[a,b]$ 分成长度 $\Delta x_i<\delta$ 的若干部分时,所有的 $\omega_i<\epsilon/(b-a)$( $\omega_i$ 是对应区间上的振幅(\autoref{def_Rieman_1}~\upref{Rieman}))。由此
\begin{equation}
\sum_{i=0}^{n-1}\omega_i\Delta x_i<\epsilon/(b-a)\cdot\sum_{i=0}^{n-1}\Delta x_i=\epsilon~.
\end{equation}
由\autoref{def_DInt_1}~\upref{DInt},
\begin{equation}\label{eq_InFun_1}
\lim\sum_{i=0}^{n-1}\omega_i\Delta x_i=0~
\end{equation}
由定积分存在条件\upref{Rieman}便证得定理。

\textbf{证毕!}

\begin{theorem}{有限间断点的连续函数必可积}
若 $f(x)$ 是区间 $[a,b]$ 上除在有限个点外都连续的函数,则 $f(x)$ 可积。
\end{theorem}
这里只给出该定理的一个理解方式:由维尔斯特拉斯第一定理,定义在闭区间上的函数必有界,于是在每个间断点 $x_{0i}$ 附近做个小区间 $(x_{0i}-\epsilon_i,x_{0i}+\epsilon_i)$,那么它们对\autoref{eq_InFun_1} 的贡献相当于 $\sum_{i=0}^{m}\omega_{0i}2\epsilon_i\leq2m\Omega\epsilon $,其中 $\Omega,\epsilon$ 是函数在区间 $[a,b]$ 的振幅和 $\epsilon_{0i}$ 中的最大者,显然这个极限就是0,而其它地方的函数是连续的,所以满足可积条件,所以最后的\autoref{eq_InFun_1} 就应为0。 

\begin{theorem}{单调有界函数必可积}
若 $f(x)$ 是区间 $[a,b]$ 上的单调有界函数,则 $f(x)$ 可积。
\end{theorem}
\textbf{证明:}
给定任一 $\epsilon>0$,令 
\begin{equation}
\delta=\frac{\epsilon}{f(b)-f(a)}~,
\end{equation}
于是当所有的 $\Delta x_i<\delta$ 时,立即有
\begin{equation}
\sum_{i=0}^{n-1}\omega_i\Delta x_i<\delta\sum_{i=0}^{n-1}(f(x_{i+1})-f(x_i))=\epsilon~
\end{equation}
即 
\begin{equation}
\lim_{\lambda\rightarrow0}\sum_{i=0}^{n-1}\omega_i\Delta x_i=0~.
\end{equation}

\textbf{证毕!}

\subsection{可积函数的性质}

\begin{theorem}{}\label{the_InFun_2}
如果 $f(x)$ 在 $[a,b]$ 上可积,则 $\abs{f(x)},kf(x)$ ($k$为常数)也在该区间上可积。
\end{theorem}
\begin{theorem}{}
如果 $f(x),g(x)$ 在 $[a,b]$ 上可积,则它们的和、差、积都可积。
\end{theorem}

\begin{theorem}{}\label{the_InFun_1}
函数 $f(x)$ 在区间 $[a,b]$ 上可积,当且仅当对于区间的分划,$f(x)$ 在每一部分区间上可积。
\end{theorem}

\begin{theorem}{}
改变可积函数在有限个点上的值,并不会破坏它的可积性。
\end{theorem}

\begin{example}{不可积函数举例}
Dirichlet函数
\begin{equation}
D(x)=
\leftgroup{
    1, x\in\mathbb{Q}\\
    0, x\not\in\mathbb{Q}
}~
\end{equation}
不是可积的:对于任何一个子区间$[a, b]$,其上最大值为$1$,最小值为$0$,因此上黎曼积分为 $1$,下黎曼积分为 $0$。
\end{example}