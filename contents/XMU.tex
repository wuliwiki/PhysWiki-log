% 厦门大学 2000 年 考研 量子力学
% license Usr
% type Note

\textbf{声明}:“该内容来源于网络公开资料,不保证真实性,如有侵权请联系管理员”

\subsection{(30分)回答和计算下列问题}
\begin{enumerate}
\item 由费密子组成的全同粒子体系,对于交换两个费密子,体系的波函数是对称的还是反对称的?
\item 什么是简并度?一维自由粒子激发态所对应的能级是几重简并?
\item 定义
\[
[\hat{A}, \hat{B}] = \hat{A} \hat{B} - \hat{B} \hat{A} ~
\]
试计算
\[
[z, [\hat{L}_x, \hat{P}_y]] = ?~
\]
其中 $z, \hat{L}_x, \hat{P}_y$ 分别是坐标 $\vec r$ 的 $Z$ 分量、角动量 $\vec{L}$ 的 $X$ 分量、动量 $\vec{P}$ 的 $Y$ 分量。
\item 粒子在中心力场中运动,问:$\hat P_z$ 是否守恒量?为什么?
\item 什么是宇称守恒?宇称算符是厄米算符吗?为什么?
\end{enumerate}
\subsection{(20分)}
粒子在一维势场
\[U(x) =  \begin{cases}   0, & |x| < a \\\\  \infty, & |x| \geq a   \end{cases}~\]
中运动,如图所示:
\begin{figure}[ht]
\centering
\includegraphics[width=6cm]{./figures/63a23b7042c65409.png}
\caption{} \label{fig_XMU_1}
\end{figure}
\begin{enumerate}
    \item 求该粒子的能级和对应的归一化波函数。
    \item 设粒子处于第一激发态,试求动量算符 $P$ 在此态上的平均值。
\end{enumerate}