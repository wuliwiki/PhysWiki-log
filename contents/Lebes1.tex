% Lebesgue 积分
% keys 实变函数|勒贝格积分
% license Xiao
% type Tutor

\pentry{可测函数的结构\nref{nod_MsbFSt}}{nod_f322}

Lebesgue积分的思路是对函数的值域进行分划,以相应值域的逆映射作为“柱底”。归根到底,Lebesuge积分还是要对定义域作分划的,但相比Riemann积分的直接对定义域作分划,Lebesgue积分的分划方式更任意。对于可测函数,Lebesgue积分的分划得到的“柱底”都是可测集。

我们就从将可测集划分为两两不交的可测子集入手,先研究这种分划的性质。

\subsection{可测集的分划}



\begin{definition}{可测分划}
设 $E\in\mathbb{R}^n$ 是可测集。如果有限\textbf{族}$\{E_1, E_2, \cdots, E_n\}$ 中各 $E_i$\textbf{两两不交}、都是 $E$ 的子集、\textbf{可测},且 $E=\bigcup^n_{i=1}E_i$,那么称集族 $\{E_i\}_{i=1}^n$ 为可测集 $E$ 的一个\textbf{分划},或者\textbf{可测分划}。
\end{definition}

如果 $A=\{E_i\}_{i=1}^n$ 和 $B=\{F_i\}_{i=1}^m$ 都是 $E$ 的分划,那么易证 $C=\{E_i\cap F_j|E_i\in A, F_j\in B\}$ 也是 $E$ 的分划。称 $C$ 是分划 $A$ 和 $B$ 的\textbf{合并}。

容易看到,$C$ 中存在每一个 $E_i$ 的分划 $\{E_i\cap F_j\}_{j=1}^m$,类似地也存在每一个 $F_j$ 的分划,像是更细一层地进行分划。因此,如果分划 $C$ 是 $A$ 和另一个分划的合并,我们就称 $C$ 是比 $A$\textbf{更细}的分划,反过来 $A$ 比 $C$\textbf{更粗}。

\begin{definition}{上和与下和}

设 $f$ 是 $E$ 上的非负可测函数,$D=\{E_i\}_{i=1}^n$ 是 $E$ 的一个可测分划。定义 $a_i=\inf_{x\in E_i}f(x)$,$A_i=\sup_{x\in E_i}f(x)$,则称
\begin{equation}
s_D=\sum_{i=1}^n a_i \opn{m}E_i~
\end{equation}
为 $f$ 在 $E$ 上关于分划 $D$ 的\textbf{下和},而称
\begin{equation}
S_D=\sum_{i=1}^n A_i \opn{m}E_i~
\end{equation}
为 $f$ 在 $E$ 上关于分划 $D$ 的\textbf{上和}。

\end{definition}

如果 $A\subseteq B\subseteq E$,那么显然 $f$ 在 $A$ 上的上确界要小于等于在 $B$ 上的上确界,在 $A$ 上的下确界要大于等于在 $B$ 上的下确界,因此容易得出以下引理:

\begin{lemma}{}\label{lem_Lebes1_1}

设 $f$ 是可测集 $E$ 上的可测函数,$A$ 和 $B$ 是 $E$ 的可测分划,且 $A$ 比 $B$ 更细。那么

\begin{equation}
s_B\leq s_A\leq S_A\leq S_B~.
\end{equation}


\end{lemma}

由此可得一个有用的推论:

\begin{corollary}{}\label{cor_Lebes1_1}
设 $f$ 是可测集 $E$ 上的可测函数,$D_1$ 和 $D_2$ 是 $E$ 的可测分划。那么

\begin{equation}
s_{D_i}\leq S_{D_j}~
\end{equation}
对任意 $i, j\in\{1, 2\}$ 成立。

\end{corollary}

就是说,不管怎么求分划,任意两个分划之间,上和一定大于等于下和,不会出现一个分划的下和大于另一个分划的上和这种情况。






\subsection{积分}

\subsubsection{上积分与下积分}

有了分划和上下和的概念,我们描述起积分就方便多了。

\begin{definition}{}

设 $f$ 是可测集 $E$ 上的可测函数,$\Lambda$ 是 $E$ 的一切可能的分划之集合。那么称
\begin{equation}
\overline{\int_E} f(x) \dd x=\inf_{D\in \lambda} \{S_D\}~
\end{equation}
为 $f$ 在 $E$ 上的\textbf{上积分},称
\begin{equation}
\underline{\int_E} f(x) \dd x=\sup_{D\in \lambda} \{s_D\}~
\end{equation}
为 $f$ 在 $E$ 上的\textbf{下积分},



\end{definition}

简而言之,上积分就是上和的下确界,下积分就是下和的上确界。

显然,简单函数的上积分和下积分总是相等,也就是\enref{可测函数的结构}{MsbFSt}中\autoref{eq_MsbFSt_1}~\upref{MsbFSt}所定义的简单函数的Lebesgue积分。

函数的上下积分有以下重要的性质:

\begin{theorem}{}\label{the_Lebes1_1}

设 $f$、$g$ 是可测集 $E$ 上的可测函数,$\{E_1, E_2\}$ 是 $E$ 的某个分划。

\begin{enumerate}
  \item 若 $f\leq g$ 几乎处处成立,则有
  \begin{equation}
  \underline{\int_E} f(x) \dd x \leq \underline{\int_E} g(x) \dd x \leq \overline{\int_E} f(x) \dd x \leq \overline{\int_E} g(x) \dd x~.
  \end{equation}
  \item 若 $\{E_1, E_2\}$ 是 $E$ 的某个分划,则
  \begin{equation}\label{eq_Lebes1_1}
  \underline{\int_E} f(x) \dd x=\underline{\int_{E_1}} f(x) \dd x+\underline{\int_{E_2}} f(x) \dd x~,
  \end{equation}
  且
  \begin{equation}\label{eq_Lebes1_4}
  \overline{\int_E} f(x) \dd x=\overline{\int_{E_1}} f(x) \dd x+\overline{\int_{E_2}} f(x) \dd x~.
  \end{equation}
  \item 恒有
  \begin{equation}\label{eq_Lebes1_5}
  \underline{\int_E} \qty(f(x)+g(x)) \dd x \geq \underline{\int_E} f(x) \dd x+\underline{\int_E} g(x) \dd x~
  \end{equation}
  和
  \begin{equation}\label{eq_Lebes1_6}
  \overline{\int_E} \qty(f(x)+g(x)) \dd x \leq \overline{\int_E} f(x) \dd x+\overline{\int_E} g(x) \dd x~.
  \end{equation}
\end{enumerate}



\end{theorem}

\textbf{证明}:

1。太过显然,在此从略\footnote{若不是那么显然,可留言,笔者视情况补充证明细节。}。

2. 为方便,记 $A$ 为对 $E$ 进行任意分划后求上和的结果的集合;$A'$ 为先将 $E$ 作分划 $\{E_1, E_2\}$ 后,再分别对这两个 $E_i$ 作分划后求上和,将两个上和相加后,所得值的集合。这样,按定义,\autoref{eq_Lebes1_4} 的左边就是 $A$ 的下确界,右边就是 $A'$ 的下确界。

\autoref{eq_Lebes1_4} 左边是对 $E$ 作分划,右边则是先将 $E$ 作分划 $\{E_1, E_2\}$ 后,再分别对这两个 $E_i$ 作分划,因此可知 $A'\subseteq A$,故必有\footnote{子集的下确界大于等于母集的下确界。}

 \begin{equation}\label{eq_Lebes1_2}
  \underline{\int_E} f(x) \dd x \leq \underline{\int_{E_1}} f(x) \dd x+\underline{\int_{E_2}} f(x) \dd x~.
  \end{equation}

另一方面,由\autoref{lem_Lebes1_1} ,可知任取 $A$ 中一个数字 $a$,都必有 $A'$ 中的数字 $a'$,使得 $a'\leq a$。因此又有

\begin{equation}\label{eq_Lebes1_3}
\underline{\int_E} f(x) \dd x \geq \underline{\int_{E_1}} f(x) \dd x+\underline{\int_{E_2}} f(x) \dd x~,
\end{equation}

综合\autoref{eq_Lebes1_2} 和\autoref{eq_Lebes1_3} 即得\autoref{eq_Lebes1_4} 。

由上下和与上下积分定义的对偶性,可直接推得\autoref{eq_Lebes1_1} 。由此得证。

3. 只需证明\autoref{eq_Lebes1_5} 即可,之后可由对偶性直接推知\autoref{eq_Lebes1_6} .

考虑任意可测集 $E_i\subseteq E$ 上的 $f$ 和 $g$,则由加法和下确界的定义直接可得“$f$ 的下确界加 $g$ 的下确界\textbf{小于等于}$f+g$ 的下确界”。

于是,对于 $E$ 的任意分划 $D$,总存在两个分划 $D_1$ 和 $D_2$\footnote{直接取 $D_1=D_2=D$ 就行,更细当然更好。},使得“$f$ 对于 $D_1$ 计算出来的下和加上 $g$ 对于 $D_2$ 计算出来的下和”,\textbf{大于等于}“$f+g$ 对于 $D$ 计算出来的下和”。因此,前者的上确界大于等于后者的下确界,也即\autoref{eq_Lebes1_5} .

\textbf{证毕}。







\subsubsection{测度有限的可测集上,非负有界函数的积分}

\autoref{the_Lebes1_1} 所描述的性质是非常符合直觉的。类比Riemann积分的定义过程,我们也希望上下积分相等,从而成为新的积分定义。事实上,可测函数就具有这样优良的性质。

\begin{theorem}{}\label{the_Lebes1_2}
设 $E\subseteq \mathbb{R}^n$ 是\textbf{测度有限}的可测集,$f$ 是其上\textbf{非负有界}函数,那么
\begin{equation}\label{eq_Lebes1_7}
\overline{\int_E} f(x) \dd x = \underline{\int_E} f(x) \dd x~
\end{equation}
的\textbf{充要条件}是 $f$ 为\textbf{可测函数}。
\end{theorem}



\textbf{证明}:

\textbf{充分性}:

设 $f$ 是\textbf{非负有界}的\textbf{可测}函数。

由\autoref{cor_Lebes1_1} ,必有
\begin{equation}
\overline{\int_E} f(x) \dd x \geq \underline{\int_E} f(x) \dd x~.
\end{equation}

因此,接下来只需要证明:对于任意 $\epsilon>0$,总存在一个分划 $D$,使得 $s_D\geq S_D-\epsilon$。

设 $\opn{m}E=c$,由题设知 $c<+\infty$。又因为 $f$ 非负有界,不妨设 $f(x)\in [0, s)$。

将 $[0, s)$ 拆分为一系列区间 $A_{k, i}=[\frac{i}{k}s, \frac{i+1}{k}s)$ 的不交并,其中 $k$ 是任意给定的正整数,$i$ 是取值范围为 $[0, k)$ 的整数。

利用区间 $A_{k, i}$ 来对 $E$ 进行分划:$E_{k, i}=\{x\in E|f(x)\in A_{k, i}\}$。显然,固定 $k$ 时,各 $E_{k, i}$ 构成 $E$ 的一组分划。

对于任意固定的 $k$,在每个 $E_{k, i}$ 上,$f$ 的上确界和下确界之差\textbf{小于等于}$s/k$,而各 $E_{k, i}$ 的外测度之和为 $c$。因此,该固定的 $k$ 按上述方式决定的分划下,$f$ 在 $E$ 上的上和与下和之差\textbf{小于等于}$sc/k$。

因此,只需要取 $k>sc/\epsilon$,所得分划就是所要的 $D$。

\textbf{必要性}:

设 $f$\textbf{非负有界},且\autoref{eq_Lebes1_7} 式成立。

\addTODO{笔者不明白这里为什么需要必要性。按理说只有可测函数才能定义上和与下和、进而有上下极限的概念啊?没有可测条件谈什么\autoref{eq_Lebes1_7} 的存在性?更不用说成立了。}

% 参考定理的位置:江泽坚《实变函数论》P122 定理2


\textbf{证毕}。



\autoref{the_Lebes1_2} 告诉我们,对于测度有限的 $E$ 上的非负可测函数 $f$,其上下积分是相等的,于是我们就可以把它们统一称为“积分”,记为
\begin{equation}
\int_E f(x) \dd x~.
\end{equation}
进一步,由\autoref{eq_Lebes1_5} 和\autoref{eq_Lebes1_6} 可知,对于可测函数 $f$,有
\begin{equation}
\int_E [f(x)+g(x)] \dd x=\int_E f(x) \dd x+\int_E g(x) \dd x~.
\end{equation}






\subsubsection{测度有限的可测集上,任意非负可测函数的积分}

\autoref{the_Lebes1_2} 讨论的是“有界”的可测函数,颇有限制。任意的非负函数有没有类似的性质呢?我们没法套用\autoref{the_Lebes1_2} 的证明方式,因为失去了有界性就无法用\textbf{同样的方法}对 $E$ 进行\textbf{有限}划分了。不过,回想一下\autoref{the_MsbFSt_1}~\upref{MsbFSt}是怎么证明的,你会发现我们可以用同样的思路来从有界推广到无界。

设 $f$ 是测度有限的可测集 $E$ 上的非负可测函数。对于任意正整数 $k$,定义一个 $E$ 上的新函数 $f_k$ 如下:$f_k(x)=\min \{k, f(x)\}$。直观来说,$f_k$ 就像是用一根长棍子去“压”$f$,把 $k$ 以上的部分全都压平到 $k$ 的高度。这样,每个 $f_k$ 都是非负有界的可测函数,它们都是有积分的了。于是,我们可以定义 $f$ 的积分为:
\begin{equation}
\int_E f(x) \dd x = \lim\limits_{k\to\infty} \int_E f_k(x) \dd x~.
\end{equation}

上述非负可测函数的积分具有积分应有的性质,我们写为以下习题:

\begin{exercise}{}\label{exe_Lebes1_2}
设 $f$、$g$ 都是测度有限的可测集 $E$ 上的非负可测函数,$\{E_1, E_2\}$ 是 $E$ 的一个分划。证明以下性质:
\begin{enumerate}
\item 当 $f(x)\leq g(x)$ 时,有
\begin{equation}
\int_E f(x) \dd x\leq \int_E g(x) \dd x~.
\end{equation}
\item 
\begin{equation}
\int_E f(x) \dd x=\int_{E_1} f(x) \dd x+\int_{E_2} f(x) \dd x~.
\end{equation}
\item 
\begin{equation}
\int_E [f(x)+g(x)] \dd x = \int_E f(x) \dd x+\int_E g(x) \dd x~.
\end{equation}
\end{enumerate}

\end{exercise}




\subsubsection{任意可测集上,任意非负可测函数的积分}


到此为止,我们已经讨论清楚了测度有限可测集 $E$ 上任意非负可测函数的积分了。接下来讨论的是 $\opn{m}E=+\infty$、或者说任意可测集 $E$ 的情况。

\begin{definition}{任意可测集上任意非负可测函数的积分}

设 $f$ 是可测集 $E$ 上的非负可测函数。

对于任意正整数 $k$,定义 $E_k=E\cap [-k, k]$。那么 $f$ 在各 $E_k$ 上都有上述定义的积分。于是可以定义 $f$ 在 $E$ 上的积分为
\begin{equation}
\int_E f(x) \dd x = \lim\limits_{k\to\infty}\int_{E_k} f(x) \dd x~.
\end{equation}

\end{definition}

利用极限的知识,结合\autoref{exe_Lebes1_2} ,我们很容易得到以下推论:

\begin{corollary}{}
设 $f$、$g$ 都是可测集 $E$ 上的非负可测函数,$\{E_1, E_2\}$ 是 $E$ 的一个分划。证明以下性质:
\begin{enumerate}
\item 当 $f(x)\leq g(x)$ 时,有
\begin{equation}
\int_E f(x) \dd x\leq \int_E g(x) \dd x~.
\end{equation}
\item 
\begin{equation}
\int_E f(x) \dd x=\int_{E_1} f(x) \dd x+\int_{E_2} f(x) \dd x~.
\end{equation}
\item 
\begin{equation}
\int_E [f(x)+g(x)] \dd x = \int_E f(x) \dd x+\int_E g(x) \dd x~.
\end{equation}
\end{enumerate}
\end{corollary}

另外,由于Lebesgue积分是基于测度定义的,而“零测”在测度论意义下相当于不存在,因此也容易得到以下定理:

\begin{theorem}{}
如果 $f$ 和 $g$ 都是可测集 $E$ 上的非负可测函数,且彼此几乎处处相等,那么
\begin{equation}
\int_E f(x) \dd x = \int_E g(x) \dd x~.
\end{equation}
\end{theorem}






\subsubsection{Lebesgue积分}

本节讨论的全部都是非负函数的情况,但结论很容易推广到任意函数上。

\begin{definition}{正部与负部}
考虑可测集 $E\subseteq\mathbb{R}^n$ 上的可测函数 $f$,定义如下两个新函数:
\begin{equation}
f^+(x) = \max\{f(x), 0\}~,
\end{equation}
\begin{equation}\label{eq_Lebes1_8}
f^-(x) = -\max\{f(x), 0\}~.
\end{equation}
称 $f^+$ 为 $f$ 的\textbf{正部},$f^-$ 为 $f$ 的\textbf{负部}。

\end{definition}

注意定义中\autoref{eq_Lebes1_8} 右边的负号。可测函数的正部与负部都是非负可测函数,这有些像复数的实部与虚部都是实数一样的逻辑。因此,我们之前讨论的“非负可测函数的积分”可以完全适用于任意可测函数的正部与负部。

再考虑到任意可测集上恒等于 $0$ 的简单函数的积分都是 $0$,我们就可以得到任意可测函数的Lebesgue积分了:

\begin{definition}{Lebesgue积分}\label{def_Lebes1_1}
设 $f$ 是可测集 $E\subseteq\mathbb{R}^n$ 上的可测函数,$f^+$ 和 $f^-$ 分别是其正部与负部。如果 $\int_E f^+(x) \dd x$ 和 $\int_E f^-(x) \dd x$ 中\textbf{至少有一个是有限的},则可定义其Lebesgue积分为:
\begin{equation}
\int_E f(x) \dd x = \int_E f^+(x) \dd x - \int_E f^-(x) \dd x~.
\end{equation}
\end{definition}














\subsection{Lebesgue积分的几何直观}


回忆\autoref{def_MsbFun_2}~\upref{MsbFun}所述,$G(E; f)$ 是函数 $f$ 在可测集 $E$ 上的\textbf{下方图形}。在简明的微积分课程中,常把(Riemann)积分解释为“求函数图像下方图形的面积”。实际上,这一点也适用于我们现在讨论的Lebesgue积分。

\begin{exercise}{}\label{exe_Lebes1_1}
证明以下命题:

\begin{enumerate}
\item 由\autoref{the_MsbFSt_1}~\upref{MsbFSt},可测集 $E$ 上的任意非负可测函数 $f$ 都可以表示为一列非负简单函数 $f_k$ 的极限,则必有
\begin{equation}
\int_E f(x) \dd x = \lim\limits_{k\to\infty} \int_{E} f_x(x) \dd x~.
\end{equation}
\item 同上,设可测集 $E$ 上的非负可测函数 $f$ 表示为一列非负简单函数 $f_k$ 的极限,则必有
\begin{equation}
\opn{m}G(E; f)=\lim\limits_{k\to \infty} \opn{m}G(E; f_k)~.
\end{equation}
\item 若 $g$ 是 $E$ 上的简单函数,则
\begin{equation}
\int_E g(x) \dd x = G(E; g)~.
\end{equation}
\end{enumerate}

如果前两条的证明有困难,可以从 $f_k$ 处处关于 $k$ 单调不减的情况入手,也可以干脆弱化命题,只证明 $f_k$ 处处关于 $k$ 单调不减的情况。弱化的命题不影响接下来的讨论。
\end{exercise}

有了\autoref{exe_Lebes1_1} 的结论,我们就可以得到一个非常符合直觉的定理:

\begin{theorem}{}\label{the_Lebes1_3}
设 $f$ 是可测集 $E$ 上的非负可测函数,则
\begin{equation}
\int_E f(x) \dd x = G(E; f)~.
\end{equation}
\end{theorem}

\autoref{the_Lebes1_3} 有力地说明Lebesgue积分定义的合理性,并且可以用于推论出,当函数Riemann可积时,其Riemann积分和Lebesgue积分相等。由此可知,Lebesgue积分是Riemann积分的推广。













