% 托卡马克
% license CCBYSA3
% type Wiki

(本文根据 CC-BY-SA 协议转载自原搜狗科学百科对英文维基百科的翻译)

\begin{figure}[ht]
\centering
\includegraphics[width=8cm]{./figures/c8a05ad14934a09d.png}
\caption{托卡马克中的磁场} \label{fig_TKMK_11}
\end{figure}

\textbf{托卡马克}(俄语:Токамáк)是一种使用强磁场将热的等离子体限制为圆环形状的装置,是为了产生受控核聚变而开发的几种磁约束装置之一。[1]

托卡马克最初是在20世纪50年代由苏联物理学家伊戈尔·塔姆(Igor Tamm)和安德烈·萨哈罗夫(Andrei Sakharov)根据奥列格·拉夫伦蒂耶夫(Oleg Lavrentiev)的一封信提出的概念。同时,托卡马克的第一个工作被认为是Natan Yavlinskii在T-1上的工作。已经证明,稳定的等离子体平衡需要绕在环面上呈螺旋状的磁力线。像z-pinch和stellarator这样的设备已经尝试过这样做,但显示出严重的不稳定性。现在被称为安全系数(数学标记为q)的概念的发展引导了托卡马克的发展;通过将反应堆的关键因素q设置为大于1,托卡马克装置有效地抑制了早期设计的不稳定性。

第一台托卡马克T-1于1958年开始运行。到20世纪60年代中期,托卡马克的设计开始显示出性能的极大提高。最初的结果于1965年公布,但被忽略了;莱曼·斯皮策在注意到他们测量温度的系统存在潜在问题后,立即将其驳回。第二组结果发表于1968年,这一次声称性能远远领先于任何其他机器,同样被认为是不可靠的。这导致了联合王国的一个代表团被邀请来进行他们自己的测量。这些结果证实了苏联的结论,并在1969年发表,导致各国疯狂建设托卡马克装置。

到20世纪70年代中期,世界各地使用了数十台托卡马克。到20世纪70年代末,这些机器已经达到了实际聚变所需的所有条件,尽管不是同时进行,也不是在一个反应堆中进行。随着盈亏平衡的目标在望,一系列新的机器被设计出来,它们将使用氘和氚的聚变燃料运行。这些机器,特别是欧洲联合环流器(喷气)、托卡马克聚变试验堆 (TFTR)和JT-60 ,有明确的目标达到盈亏平衡。

相反,这些机器显现出了限制其性能的新问题。解决这些问题需要更大,更昂贵的机器,超出了任何一个国家的能力。在1985年11月罗纳德·里根和米哈伊尔·戈尔巴乔夫达成初步协议后,国际热核聚变实验堆计划 (ITER)项目应运而生,并且仍然是发展实际聚变发电的主要国际合作。许多较小的设计和分支,如球形托卡马克,继续用于研究性能参数和其他问题。

\subsection{语源}
托卡马克这个词是俄语单词токамак的音译,是以下两者的首字母缩写:
\begin{itemize}
\item "тороидальнаясасесассагнитнымикатушками"(带磁性线圈的环形室;
\end{itemize}
或者
\begin{itemize}
\item "тороидальнаясамесасаксиальным магнитным полем"(具有轴向磁场的环形腔体。[2]
\end{itemize}
这个术语是由伊戈尔·戈洛温在1957年提出的,他是科学院测量仪器实验室的副主任,现在的库尔恰托夫研究所。类似的术语“tokamag”也曾被提出过一段时间。

\subsection{历史}
\begin{figure}[ht]
\centering
\includegraphics[width=6cm]{./figures/a022bdad1bb07f4d.png}
\caption{苏联邮票,1987:托卡马克热核系统} \label{fig_TKMK_1}
\end{figure}
\subsubsection{2.1 第一步}
1934年,马克·奥列芬特、保罗·哈特克和欧内斯特·卢瑟福是第一个在地球上实现聚变的人,他们使用粒子加速器将氘原子核射入含有氘或其他原子的金属箔中。[3]这使他们能够测量各种聚变反应的核截面,并确定氘-氘反应发生的能量低于其他反应,峰值约为10万电子伏特(100 keV)。[4]

基于加速器的聚变是不现实的,因为反应截面很小;加速器里的大部分粒子会从燃料中散开,而不是与燃料融合。这些散射使粒子失去能量,达到不再发生聚变的程度。因此,注入这些粒子的能量就消失了,很容易证明这比由此产生的聚变反应释放的能量要多得多。[5]

为了维持聚变和产生净能量输出,燃料的大部分必须提高到高温,这样它的原子就会不断高速碰撞;这就产生了“热核”这个名字,因为产生它需要很高的温度。1944年,恩里科·费米(Enrico Fermi)计算出,在大约5000万K的情况下,这个反应是可以自我维持的;在那个温度下,反应释放能量的速率足够高,使周围的燃料迅速升温,以保持温度不受环境损失,继续进行反应。[5]

在曼哈顿计划期间,第一个达到这些温度的实际方法是用原子弹创造的。1944年,费米在当时假设的氢弹背景下发表了一场关于核聚变物理学的演讲。然而,人们已经想到了一种可控核聚变装置,吉姆·塔克和斯坦尼斯拉夫·乌兰姆曾尝试使用聚能炸药驱动注入氘的金属箔,尽管没有成功。[6]

第一次尝试制造实用的核聚变机器是在英国,乔治·佩吉特·汤姆森在1945年选择了箍缩效应作为一种有前途的技术。在几次试图获得资助失败后,他放弃了,并要求两名研究生斯坦·卡曾斯(Stan Cousins)和艾伦·韦尔(Alan Ware)用剩余的雷达设备制造一种设备。该系统于1948年成功运行,但没有明确的证据表明核聚变,未能获得原子能研究机构的兴趣。[7]
\subsubsection{2.2 奥列格·拉夫罗夫(Oleg Lavrentiev)的信}
1950年,时驻扎在库页岛的一名无事可做的红军中士奥列格·拉夫罗夫给苏联共产党中央委员会写了一封信。信中概述了用原子弹点燃聚变燃料的想法,然后描述了一个利用静电场来控制稳定状态的热等离子体以产生能量的系统。[8][9]

这封信被送往安德烈·德米特里耶维奇·萨哈罗夫那里征求意见。萨哈罗夫指出“作者提出了一个非常重要但不一定是无望的问题”,并发现他在构想中主要关心的是等离子会撞击电极线,“宽网格和薄载流部分将几乎所有入射的原子核反射回到反应堆中。这一要求很可能与设备的机械强度不相容。”[8]

从处理的速度可以看出拉夫罗夫的信的重要性;这封信在7月29日被中央委员会收到,萨哈罗夫在8月18日发出了他的评论,到10月,萨哈罗夫和伊戈尔·塔姆已经完成了第一个关于核聚变反应堆的详细研究,他们在1951年1月申请了建造资金。[10]
\subsubsection{2.3 磁约束}
当加热到熔化温度时,原子中的电子电离,形成一种原子核和电子的流体,称为等离子体。与电中性原子不同,等离子体是导电的,因此可以被电场或磁场控制。[11]

萨哈罗夫对电极的担心导致他考虑使用磁约束代替静电。在磁场的情况下,粒子将围绕磁力线旋转。[11]当粒子高速运动时,它们产生的路径看起来像螺旋。如果排列的磁场使得磁力线平行且靠近,围绕相邻磁力线的粒子可能会碰撞并融合。[12]

这种磁场可以在螺线管中产生,螺线管是外部缠绕有磁体的圆柱体。磁体的组合磁场产生了一组沿着圆柱体长度延伸的平行磁力线。这种布置防止粒子侧向移动到圆柱体的壁上,但并不防止它们从末端跑出。这种排列方式可以防止颗粒向圆筒壁面的侧面移动,但却不能防止颗粒跑出圆筒的末端。解决这个问题的一个明显的办法是把圆柱体弯曲成一个甜甜圈的形状,这样这些线条就形成了一系列连续的圆环。在这种排列中,粒子无休止地循环。[12]

萨哈罗夫与伊戈尔·塔姆(Igor Tamm)讨论了这个概念,到1950年10月底,两人写了一份提案,并把它寄给苏联原子弹项目主管伊戈尔·库尔恰托夫(Igor Kurchatov)及其副手伊戈尔·戈洛文(Igor Golovin)。[12]但是,这一初步建议忽略了一个基本问题;当沿直线螺线管排列时,外部磁铁的间隔是均匀的,但当它们弯曲成环形时,它们在环内比在环外靠得更近。这导致不均匀的力量,导致粒子漂移远离他们的磁力线。[13][13]

萨哈罗夫在访问苏联核研究中心苏联科学院测量仪器实验室期间,提出了解决这一问题的两个可能办法。一种方法是在环面中心悬挂载流环。环中的电流会产生一个磁场,这个磁场会与外部磁铁产生的磁场混合。由此产生的磁场将被扭曲成螺旋状,因此任何给定的粒子都会发现自己反复出现在环面内外。由不均匀场引起的漂移在内外方向相反,所以在环绕环面长轴的多次轨道运行过程中,相反的漂移会相互抵消。另外,他建议使用外部磁铁在等离子体中产生电流,而不是使用单独的金属环,这样也会产生同样的效果。[13]

1951年1月,库尔恰托夫在利潘安排了一次会议,讨论萨哈罗夫的概念。他们获得了广泛的兴趣和支持,2月,一份关于这一主题的报告被转交给了监督苏联原子能工作的拉夫连季·帕夫洛维奇·贝利亚。有一段时间,没有任何回音。[13]
\subsubsection{2.4 里克特和融合研究的诞生}
1951年3月25日,阿根廷总统胡安·裴隆宣布,前德国科学家罗纳德·里克特已经成功地在实验室规模上生产聚变,这是现在称为休穆尔项目的一部分。世界各地的科学家都对这一消息感到兴奋,但很快得出结论:这不是真的;简单的计算表明,他的实验装置不能产生足够的能量将聚变燃料加热到所需的温度。[14]

尽管遭到核研究人员的驳斥,但广泛的新闻报道意味着政治家们突然意识到并接受了聚变研究。在英国,屡次遭到拒绝的汤姆森突然获得了大量的研究资金。接下来的几个月里,有两个基于pinch系统的项目开始运行。[15]在美国,莱曼·斯必泽读了休穆尔的故事,意识到它是错误的,并着手设计一台可以工作的机器。[16]5月,他获得了5万美元,开始研究他的仿星器概念。[17]吉姆•塔克(Jim Tuck)曾短暂返回英国,参观了汤姆森的缩放机。当他回到洛斯阿拉莫斯的时候,他也和斯皮策同时申请了资金,但是被拒绝了。但直接从洛斯阿拉莫斯的预算中得到了5万美元。[18]

苏联也发生了类似的事件。4月中旬,电物理仪器科学研究所的德米特里·埃夫雷莫夫带着一本杂志闯入库尔恰托夫的研究,杂志上有一篇关于里克特工作的报道,要求知道他们为什么被阿根廷人打败。库尔恰托夫立即联系了贝利亚,提议建立一个独立的聚变研究实验室,由列夫·阿特莫维奇(Lev Artsimovich)担任主任。仅仅几天之后,即5月5日,约瑟夫·斯大林签署了这项建议。[13]
\subsubsection{2.5 新想法}
到10月份,萨哈罗夫和塔姆已经完成了对他们最初提议的更详细的考虑,要求一个整个圆环的外半径为12米,内半径为2米的设备。该系统每天可以生产100克(3.5盎司)氚,或每天生产10公斤(22磅)U233[13]

随着这一想法的进一步发展,人们认识到等离子体中的电流可以产生一个足够强的磁场来限制等离子体,从而消除了对外部磁体的需求。在这一点上,苏联研究人员重新发明了英国正在开发的箍缩系统,[6]尽管他们是从一个截然不同的起点开始设计的。

一旦提出使用箍缩效应,一个简单得多的解决方案就显而易见了。人们可以简单地将电流导入一个线性管中,而不是一个大的环形管,这将导致管内的等离子体坍缩成一条细丝。这有一个巨大的优势;等离子体中的电流会通过正常的电阻加热使其升温,但不会将等离子体加热到熔化温度。然而,当等离子体崩溃时,绝热过程将导致温度急剧上升,远远超过聚变所需的温度。随着这一发展,只有戈洛温和纳坦·亚夫林斯基继续考虑更静态的环形布置。
\subsubsection{2.6 不稳定性}
1952年7月4日,尼古拉·菲利波夫(Nikolai Filippov)的团队测量了一台线性箍缩机释放的中子。列夫·阿齐莫维奇要求他们在得出核聚变发生的结论之前检查所有的东西,在这些检查中,他们发现中子根本不是聚变产生的。英国和美国的研究人员也经历了同样的线性排列,他们的机器表现出了同样的行为。但是这项研究的高度保密意味着没有一个小组意识到其他小组正在研究,更不用说有同样的问题了。[19]

经过大量研究,发现中子是由等离子体的不稳定性引起的。有两种常见的不稳定性“香肠”主要出现在线性机器中,“扭结”在环面机器中最常见。[19]这三个国家的团队都开始研究这些不稳定性的形成以及解决这些不稳定性的潜在方法。[20]美国的马丁·大卫·克鲁斯卡尔、马丁·史瓦西和苏联的沙夫拉诺夫对该领域做出了重要贡献。[21]

来自这些研究的一个想法被称为“稳定夹点”。这一概念在腔室的外部增加了额外的磁体,将出现在等离子体之前的缩放放电。在大多数概念中,外场相对较弱,因为等离子体是抗磁性的,所以它只穿透等离子体的外部区域。[19]当收缩放电发生时,等离子体迅速收缩,这个场被“冻结”在产生的细丝上,在其外层产生一个强场。在美国,这被称为“给等离子体一个主干”[22]

萨哈罗夫重新审视了他最初的环形概念,并对如何稳定等离子体得出了略有不同的结论。布局将与稳定收缩概念相同,但两个场的作用将相反。在新的布局中,外部磁体将更强大,以提供大部分限制,而不是提供稳定的弱外部场和负责限制的强收缩电流,而电流将更小,并负责稳定效果。
\subsubsection{2.7 步骤解密}
1955年,随着线性方法仍然受到不稳定性的影响,苏联制造了第一个环形装置。TMP是一种经典的箍缩机,与英美同时期的机型相似。真空室由陶瓷制成,放电光谱显示是二氧化硅,这意味着等离子体没有完全被磁场限制,并撞击了室壁。随后出现了两台使用铜质外壳的小型机器。[23]这种导电外壳原本是用来稳定等离子体的,但在任何一台尝试过的机器上都没有完全成功。[23]

随着进展明显停滞,库尔恰托夫于1955年召开了苏联研究人员全联盟会议,其最终目标是在苏联内部开展核聚变研究。[24]1956年4月,作为尼古拉·赫鲁晓夫和尼古拉·布尔加宁广泛宣传访问的一部分,库尔恰托夫前往英国。他主动提出要在前英国皇家空军(RAF)哈维尔(Harwell)的原子能研究机构(Atomic Energy Research institution)做一次演讲。在那里,他详细介绍了苏联核聚变努力的历史概况,震惊了主办方。[30]他花了一些时间,特别注意到早期机器中出现的中子,并警告说,中子并不意味着聚变。[25]

库尔恰托夫不知道,英国的 ZETA 稳定箍缩机正在原跑道的远端建造。当时,ZETA是最大、最强的聚变机器。ZETA得到了早期设计实验的支持,这些实验已经被修改为包括稳定性,旨在产生低水平的聚变反应。这显然是一个巨大的成功,1958年1月,他们宣布基于中子释放和等离子体温度的测量,在ZETA中实现了聚变。[26]

维塔利·沙夫拉诺夫和斯坦尼斯拉夫·布拉金斯基研究了新闻报道,并试图弄清楚其中的原理。他们考虑的一种可能性是使用弱“冻结”磁场,但最终拒绝了这种可能性,因为他们认为磁场不会持续足够长的时间。然后,他们得出结论,ZETA与他们研究的设备基本相同,具有强大的外磁场。[27]
\subsubsection{2.8 第一台托卡马克装置}
那时,苏联研究人员已经决定按照萨哈罗夫建议的路线建造一台更大的环形机器。特别是,他们的设计考虑了“克鲁斯卡尔算法-沙夫拉诺夫极限”中的一个重要观点;如果粒子的螺旋路径使它们围绕等离子体圆周运动的速度比环绕环面长轴运动的速度快,则扭结不稳定性将被强烈地抑制。[20]

今天,这个基本概念被称为安全系数。表示粒子绕长轴运行的次数与短轴运行的次数之比$q$,和“克鲁斯卡尔算法-沙夫拉诺夫极限”声明一样,只要$q>1$,扭结将被抑制。这条路径由外部磁铁相对于内部电流产生的磁场的相对强度控制。为了获得$q>1$,外部磁铁必须更强大,否则,必须降低内部电流。[20]

遵循这一标准,一个新的反应堆T-1开始设计,今天被称为第一个真正的托卡马克。[23]与ZETA相比,T-1使用了更强的外部磁体和更小的电流。T-1的成功使它被认为是第一个工作的托卡马克。[28][29][30][31]由于在“气体中强大的脉冲放电,以获得热核过程所需的异常高温”方面的工作,亚夫林斯基在1958年获得了的列宁奖和斯大林奖。Yavlinskii已经在准备设计一个更大的模型,后来被命名为T-3。随着ZETA的发布,Yavlinskii的概念得到了很好的评价。[27][31]

ZETA的详细信息在1999年在《自然》杂志1月下旬的一系列文章中公之于众。。令沙夫拉诺夫惊讶的是,该系统确实使用了“冻结”领域的概念。[27]但他仍然持怀疑态度,而位于圣彼得斯贝格 Ioffe Institute 的一个团队开始计划建造一种类似的机器,名为阿尔法。仅仅几个月后,在5月份,ZETA团队发布了一份声明,称他们还没有实现聚变,并且他们被等离子体温度的错误测量误导了。[32]

T-1于1958年底开始运行。[33]它通过辐射显示了非常高的能量损失。这是由于真空系统引起的等离子体中的杂质从容器材料中逸出。这是一种内部用波纹金属制成的内衬,在550°C(1022°F)的温度下烘烤,用来燃烧被困住的气体。[33]
\subsubsection{2.9 “原子促进和平和萧条”}
作为1958年9月在日内瓦举行的第二次原子促进和平会议的一部分,苏联代表团发表了许多论文,涵盖了他们的聚变研究。其中有一组关于他们的环形机器的初步结果,当时没有显示任何值得注意的东西。[34]

该会议的“明星”是斯皮策的仿星器的一个大模型,它立即引起了苏联人的注意。与他们的设计相反,仿星器在没有电流通过的情况下在等离子体中产生了所需的扭曲路径,使用了一系列可以在稳定状态下工作的磁铁,而不是感应系统的脉冲。库尔恰托夫开始要求亚夫林斯基将他们的T-3设计改为仿星器,但他们说服他,电流在加热方面提供了一个有用的第二个效果,这是仿星器所缺乏的。[34]

在进展中,仿星器遭遇了一长串的小问题,而这些小问题刚刚被解决。结果表明,等离子体的扩散速度远快于理论预测。由于这样或那样的原因,类似的问题出现在所有的当代设计中。星状器、各种各样的夹点概念以及美苏两国的磁镜机器都证明了限制其限制时间的问题。[33]

从最初对受控聚变的研究来看,就有一个隐藏在背后的问题。在曼哈顿计划期间,戴维·玻姆是研究铀同位素分离的团队的一员。战后,他继续研究磁场中的等离子体。根据基本理论,人们可以预期等离子体将以与场强平方成反比的速度沿力线扩散,这意味着力的微小增加将大大改善约束。但根据他们的实验,玻姆发展了一个经验公式,现在被称为玻姆扩散,表明速度是线性的磁力,而不是它的平方。[35]

如果玻姆的公式是正确的,人们就没有希望建造基于磁约束的聚变反应堆。为了将等离子体限制在聚变所需的温度范围内,磁场必须比任何已知的磁体大几个数量级。斯皮策将玻色子与经典扩散速率的差异归因于等离子体中的湍流,[36]并相信仿星器的稳定长不会受到这个问题的影响。当时的各种实验表明玻姆速率不适用,经典公式是正确的。[35]

但是到了20世纪60年代初,各种各样的设计以惊人的速度泄漏等离子体,斯皮策自己得出结论:玻姆尺度是等离子体固有的性质,磁约束不起作用。[33]整个领域陷入了后来被称为“萧条”的境地,[37]一段极度悲观的时期。
\subsubsection{2.10 托卡马克在20世纪60年代的进展}
与其他设计相比,实验托卡马克似乎进展得很顺利,以至于一个小小的理论问题现在成了一个真正的问题。在重力的存在下,等离子体中有一个小的压力梯度,以前小到可以忽略,但现在变成了必须解决的问题。这导致1962年又增加了一组磁体,产生的垂直磁场抵消了这些影响。这些工作都是成功的,到20世纪60年代中期,这些机器开始显示出超越玻姆极限的迹象。[38]

在1965年的第二届国际原子能机构核聚变会议上,阿特西莫维奇报告说,他们的核聚变系统超过了玻姆极限的10倍。斯皮策回顾了这些报告,认为玻姆极限仍然适用;这些结果在星型仪的实验误差范围内,而基于磁场的温度测量结果根本不可靠。[38]

下一次重大国际聚变会议于1968年8月举行新西伯利亚。到那时,另外两个托卡马克设计也已经完成,TM-2在1965年,T-4在1968年。T-3反应堆的测试结果持续改善,新反应堆的早期测试也得出了类似的结果。在会议上,苏联代表团宣布,T-3产生的电子温度为1000 eV(相当于1000万摄氏度),限制时间至少是玻姆极限的50倍。[39]

这些结果至少是任何其他机器的10倍。如果正确的话,它们代表了聚变领域的巨大飞跃。斯皮策仍然持怀疑态度,他指出温度测量仍然基于等离子体磁性的间接计算。许多人认为这是由于一种被称为“逃逸电子”的效应,苏联人只测量了那些能量极高的电子,而没有测量体积温度。苏联人反驳说,他们测量的温度是麦克斯韦式的,争论很激烈。[40]

\textbf{“文化五人组”}

ZETA之后,英国团队开始开发新的等离子体诊断工具,以提供更精确的测量。其中包括激光器:使用汤姆孙散射直接测量体电子的温度。这项技术在聚变领域中众所周知且令人钦佩;[41]阿齐莫维奇曾公开称之为“辉煌”。阿特西莫维奇邀请了库勒姆的负责人巴斯·皮斯在苏联的反应堆上使用他们的设备。在冷战最激烈的时期,英国物理学家被允许参观库尔恰托夫研究所(Kurchatov Institute),这是苏联核弹计划的核心。[42]

这支绰号为“文化五人组”的英国小队[43]于1968年底抵达伦敦。经过漫长的安装和校准过程,该团队在多次实验运行期间测量了温度。1969年8月获得初步结果;苏联人是正确的,他们的结果是准确的。该小组将结果打电话给卡勒姆,卡勒姆随后通过秘密电话将结果传递给华盛顿。[44]最终结果发表在1969年11月的《自然》杂志上。[45]这项宣布的结果被描述为世界各地托卡马克建设的“名副其实的成功”。[46]

还有一个严重的问题。因为等离子体中的电流比箍缩机低得多,产生的压缩也小得多,这意味着等离子体的温度受限于电流的电阻加热速率。斯皮策电阻率在1950年首次提出,它指出等离子体的电阻随着温度的升高而减小,这意味着随着设备的改进和温度的升高,等离子体的升温速度会减慢。[47]这意味着随着设备的改进和温度的升高,等离子体的升温速度会减慢。计算表明,在q > 1范围内的最高温度将被限制在数百万度以下。阿特西莫维奇很快在新西伯利亚指出了这一点,称未来的进展将需要开发新的加热方法。[48]

\textbf{美国动荡}

参加1968年新西伯利亚会议的人之一是阿玛萨·斯通,美国聚变计划的领导人之一,该计划是迄今为止世界上最大的计划。在当时,有明确证据证明可以超越玻姆极限的其他设备不多,多极概念就是其中之一。劳伦斯·利弗莫尔和斯皮策仿星器的发源地普林斯顿等离子体物理实验室 (PPPL)都在多极设计上进行了改进。T-3的表现大大优于两者;毕晓普担心多极是多余的,认为美国应该考虑自己的托卡马克。[49]

当他在1968年12月的一次会议上提出这个问题时,实验室的主管们拒绝考虑。梅尔文·戈特利布普林斯顿大学的教授很恼火,问“你认为这个委员会能比科学家们想得更远吗?”[50]由于主要实验室要求他们控制自己的研究,一个实验室发现自己被遗漏了。橡树岭最初进入聚变领域是为了研究反应堆燃料系统,但后来扩展到自己的镜像计划。到了20世纪60年代中期,他们在DCX的设计已经没有创意了,提供的任何东西都比不上更有声望和政治影响力的利弗莫尔的类似项目。这使他们成为美国唯一一个高度接受新概念的主要实验室。[51]

经过相当长的内部辩论,赫尔曼·波斯特在1969年初成立了一个小组来考虑托卡马克。[51]他们想出了一个新的设计,后来命名为 Ormak ,它有几个新颖的特征。其中最主要的是外部磁场在单个大型铜块中产生的方式,它由圆环下方的大型变压器供电。这与在外部使用磁体绕组的传统设计相反。他们觉得单个块会产生更均匀的场。它还有一个优点,就是允许环面有一个更小的主半径,不需要把电缆从甜甜圈孔中穿过,从而降低了纵横比,而苏联人已经建议过这样会产生更好的结果。[52]

\textbf{美国托卡马克蓬勃发展事件}

1969年初,阿齐莫维奇访问了麻省理工学院,在那里他被那些对核聚变感兴趣的人追逐。他最终同意在4月份举办几场讲座,[48]然后允许进行冗长的问答环节。随着时间的推移,麻省理工学院本身也对托卡马克产生了兴趣,之前由于种种原因,它一直没有涉足核聚变领域。布鲁诺·科皮当时在麻省理工学院,他和波斯特马的团队遵循同样的概念,提出了自己的低宽比概念——阿尔卡特。阿尔卡特使用了传统的环形磁铁,而不是Ormak的环形变压器,但要求它们比现有的设计小得多。麻省理工学院的Francis Bitter磁铁实验室是世界上磁铁设计的领导者,他们很有信心能够建造出这样的磁铁。[48]

1969年,又有两个团体进入该领域。在通用原子公司,Tihiro Ohkawa一直在开发多极反应堆,并提出了一个基于这些想法的概念。这是一个托卡马克它有一个非圆形的等离子体截面;同样的计算表明,较低的宽高比可以提高性能,同样的计算也表明,C形或d形等离子也可以提高性能。他称新设计为双重线。[53]与此同时,德克萨斯大学奥斯汀分校的一个研究小组提出了一个相对简单的托卡马克模型,用来探索通过故意诱导的湍流——德克萨斯湍流托卡马克来加热等离子体。[54]

1969年6月,原子能委员会核聚变指导委员会的成员再次开会时,他们“听到了托卡马克提议”。[54]普林斯顿大学(Princeton)是唯一一个没有提出托卡马克(tokamak)设计方案的主要实验室,尽管他们的C星状推进器(C stellarator)模型几乎可以完美地完成这种转换,但普林斯顿拒绝考虑托卡马克。他们继续提出了一长串理由,说明为什么不应转换C型。当这些被质疑时,一场关于苏联结果是否可靠的激烈辩论爆发了。[54]

看着辩论的进行,戈特利布改变了主意。如果苏联的电子温度测量不准确,托卡马克的研究就没有意义了,所以他制定了一个计划来证明或者反驳他们的结论。午休时间在游泳池游泳时,他告诉哈罗德·弗斯他的计划,弗斯回答说:“嗯,也许你是对的。”[44]午餐后,不同的团队展示了他们的设计,这时Gottlieb提出了基于C模型的“星月星-托卡马克”的想法。[44]

常设委员会指出,这一系统可以在六个月内完成,而Ormak将需要一年时间。[44]“文化五人组”的机密结果不久后才被公布。当他们在10月再次开会时,常务委员会公布了所有这些提案的资金来源。模型C的新结构,很快被命名为对称托卡马克,目的是简单地验证苏联的结果,而其他的将探索远远超过T-3的方法[55]
\subsubsection{2.11 加热方法:美国领先}
\begin{figure}[ht]
\centering
\includegraphics[width=10cm]{./figures/6cca1c58577b8468.png}
\caption{普林斯顿大圆环面的俯视图,摄于1975年。PLT是一个非常成功的托卡马克聚变装置,它创造了无数的记录,证明了聚变所需的温度是可能的。} \label{fig_TKMK_2}
\end{figure}
对称托卡马克实验始于1970年5月,到第二年年初,他们证实了苏联的实验结果。现在,PPPL将其相当多的专业知识用于解决等离子体加热的问题。有两个概念似乎很有希望。PPPL提出了使用磁压缩的方法,这是一种类似于捏的技术,通过压缩温暖的等离子体来提高它的温度,但是通过磁铁而不是电流来提供这种压缩。[56]橡树岭实验室提出了中性束注入,一种小型粒子加速器,它可以通过周围的磁场发射燃料原子,使它们与等离子体发生碰撞并使其升温。[57]

PPPL公司的绝热环形压缩机(ATC)在1972年5月开始运行,不久之后,配备了中性点光束的Ormak也开始运行。两者都存在明显的问题,但PPPL通过在ATC上安装喷射器,超越了橡树岭,并在1973年提供了成功加热的明确证据。这一成功“抢先”了橡树岭,而橡树岭在华盛顿指导委员会中已失宠。。[58]

这时,一个基于中性束加热的更大的设计正在建造中普林斯顿大圆环,或PLT。PLT是专门设计来“明确指出托卡马克概念加上辅助加热是否可以构成未来聚变反应堆的基础”。[59]PLT是一个巨大的成功,不断提高它的内部温度,直到它达到6000万摄氏度(8000 eV,是T-3纪录的8倍)。这是托卡马克发展的一个关键点;PLT证明,在5000万至1亿摄氏度的温度下,核聚变反应可以自我维持。[59]

这些实验,尤其是PLT,使美国在托卡马克研究中遥遥领先。这主要是由于预算问题;托卡马克的成本约为50万美元,当时美国聚变预算约为2500万美元。[39]他们有能力探索所有有前途的加热方法,最终发现中性束是最有效的方法之一。[60]

在此期间,罗伯特·希尔施接管了美国原子能委员会聚变发展局。赫希认为,如果没有明显的结果,该计划就无法维持目前的资助水平。他开始重新制定整个计划。一度主要是科学探索的实验室主导的努力,现在变成了华盛顿主导的建造工作中的发电反应堆的努力。[60]这得到了第一次石油危机的推动,导致对替代能源系统的研究大大增加。[61]
\subsubsection{2.12 80年代:希望越大,失望越大}
\begin{figure}[ht]
\centering
\includegraphics[width=10cm]{./figures/b7834c47030c963e.png}
\caption{欧洲联合环流器 (JET),目前运行的最大托卡马克,自1983年开始运行} \label{fig_TKMK_3}
\end{figure}
到20世纪70年代末,托卡马克已经达到了一个实用的聚变反应堆所需的所有条件;1978年PLT演示了点火温度,次年苏联的T-7首次成功地使用了超导磁体,双重态被证明是成功的,并导致几乎所有未来的设计采用这种“形状等离子体”的方法[62]似乎建造一个发电反应堆所需要做的就是把所有这些设计理念整合到一台机器中,一台能够在燃料混合物中运行放射性氚的机器。[63]

比赛开始了。在20世纪70年代,全球资助了四项主要的第二代提案。苏联继续发展T-15,[62]而泛欧和日本发展欧洲联合环流器,开始了JT-60工作(最初称为“盈亏平衡等离子体测试设施”)。在美国,赫希开始为类似的设计制定计划,跳过了另一个直接燃烧氚的踏脚石设计的提议。这就是托卡马克聚变试验反应堆(TFTR),它直接由华盛顿运行,不与任何特定的实验室连接。[63]赫希最初支持橡树岭作为东道主,但在其他人说服他他们会尽最大努力做这件事,因为他们损失最大之后,他把它搬到了PPPL。[64]

这种兴奋是如此普遍,以至于大约在这个时候开始了几家生产商用托卡马克的商业企业。其中最著名的是1978年,《阁楼》杂志的出版人Bob Guccione遇到了Robert Bussard,成为世界上最大和最坚定的核聚变技术私人投资者,最终将自己的2000万美元投资到Bussard的托卡马克上。里格斯银行的资助导致了这项被称为里格特龙。[65]

TFTR赢得了建造比赛,并于1982年开始运营,随后是1983年的JET和1985年的JT-60。JET很快在关键的实验中取得了领先地位,从测试气体到氘以及越来越强大的“射击”。但很快就发现,这些新系统都没有达到预期的效果。出现了许多新的不稳定因素,同时还出现了一些更实际的问题,这些问题继续干扰它们的性能。除此之外,在TFTR和JET中,等离子体撞击反应堆壁的危险“短途旅行”是显而易见的。即使在工作完美的情况下,等离子体限制在聚变温度下,即所谓的“聚变三重产物”,仍然远远低于实际反应堆设计所需的水平。

到了20世纪80年代中期,许多这些问题的原因变得很清楚,并提供了各种解决方案。然而,这将显著增加机器的尺寸和复杂性。如果在后续设计中加入这些变化,将是巨大的,而且比JET或TFTR都要昂贵得多。一个新的悲观时期降临在核聚变领域。
\subsubsection{2.13 ITER}
\begin{figure}[ht]
\centering
\includegraphics[width=8cm]{./figures/6226e6ad5b688877.png}
\caption{世界上最大的托卡马克国际热核聚变实验堆计划 (ITER)剖面图,该托卡马克于2013年开始建设,预计2035年开始运行。它的目的是作为一个示范,一个实用的聚变反应堆是可能的,并将产生500兆瓦的电力。底部的蓝色人体图显示的是比例。} \label{fig_TKMK_4}
\end{figure}
与此同时,这些实验证明了一些问题,美国巨额资金投入的动力大部分消失了;1986年,罗纳德•里根(Ronald Reagan)宣布上世纪70年代的能源危机已经结束,[66]20世纪80年代初,对先进能源的资金大幅削减。

自1973年6月以来,国际托卡马克反应堆一直在以INTOR的名义进行国际反应堆设计。这最初是由理查德·尼克松和列昂尼德·勃列日涅夫之间的一项协议开始的,但自1978年11月23日第一次真正的会晤以来进展缓慢。[67]

在1985年11月的日内瓦超级大国峰会期间,里根向米哈伊尔·戈尔巴乔夫提出了这个问题,并提议改革该组织。"...两位领导人强调了旨在为和平目的利用受控热核聚变的工作的潜在重要性,并在这方面倡导在获取这一基本上取之不尽、用之不竭的能源方面尽可能广泛地开展国际合作,为全人类造福。”[68]

第二年,美国、苏联、欧盟和日本签署了一项协议,成立了国际热核聚变实验堆计划组织。[69][70]

设计工作始于1988年,从那时起,ITER反应堆一直是全世界主要的托卡马克设计工作。

\subsection{托卡马克装置设计}
\begin{figure}[ht]
\centering
\includegraphics[width=6cm]{./figures/5b313d9b3dd98d34.png}
\caption{托卡马克磁场和电流。图中显示了环形场和产生环形场的线圈(蓝色)、等离子体电流(红色)和由环形场产生的极向场,以及叠加后产生的扭曲场。} \label{fig_TKMK_5}
\end{figure}
\subsubsection{3.1 基本问题}
在聚变等离子体中,带正电荷的离子和带负电荷的电子处于非常高的温度,相应的速度也很大。为了维持聚变过程,来自热等离子体的粒子必须被限制在中心区域,否则等离子体将迅速冷却。磁约束聚变装置利用了这样一个事实,即在磁场中带电粒子受到洛伦兹力,并沿磁场线沿螺旋路径运动。[71]

最简单的磁约束系统是一个螺线管。螺线管中的等离子体将围绕沿其中心向下的场线盘旋,阻止其向两侧运动。然而,这并不妨碍运动的终点。显而易见的解决办法是把螺线管弯曲成一个圆,形成一个圆环。但是,这种安排并不统一;由于纯粹的几何原因,环面的外边缘的磁场比内边缘的低。这种不对称性导致电子和离子在磁场中漂移,最终击中环面壁。[72]

解决的办法是把这些线条塑造得不只是围绕着环面,而是像理发杆或糖果棒上的条纹那样扭曲。在这样的磁场中,任何单个粒子都会发现自己在外部边缘,它会向一个方向漂移,比如说向上,然后当它沿着环绕环面的磁力线运动时,它会发现自己在内部边缘,它会向另一个方向漂移。这种取消并不完美,但计算表明,这足以让燃料在反应堆中保留一段有用的时间。[71]
\subsubsection{3.2 托卡马克装置的解决方案}
第一个设计需要扭转的解决方案是星状分离器,它通过一个机械装置来扭转整个环面;另一个是z箍缩设计,它让电流通过等离子体,在同一末端产生第二个磁场。与简单的环面相比,两者都显示出了更好的封闭时间,但也都显示出了各种各样的影响,导致反应堆中的等离子体以不可持续的速度流失。

托卡马克在物理布局上基本上与z箍缩概念相同。[72]它的关键创新是认识到导致收缩失去等离子体的不稳定性是可以控制的。问题是地区有多“扭曲”;在绕长轴环形轨道的轨道上,使粒子多次从内到外运动的磁场要比扭转较小的装置稳定得多。这个扭转轨道的比率被称为安全系数,记作$q$。以前的设备在问关于⅓,托卡马克在问> > 1。这增加了数量级的稳定性。

当更仔细地考虑这个问题时,需要磁场的垂直(平行于旋转轴)分量。垂直场中环形等离子体电流的洛伦兹力提供了保持等离子体圆环平衡的向内的力。
\subsubsection{3.3 其他问题}
虽然托卡马克从总体上解决了等离子体稳定性问题,但等离子体也受到许多动态不稳定性的影响。其中之一,扭结不稳定性被托卡马克布局强烈抑制,这是托卡马克高安全系数的副作用。扭结的缺乏使得托卡马克能够在比以前的机器高得多的温度下运行,这使得许多新现象出现。

其中之一是香蕉轨道,是由托卡马克中广泛的粒子能量引起的——大部分燃料是热的,但有一定比例的燃料要冷得多。由于托卡马克中磁场的高度扭曲,粒子沿着它们的磁力线迅速向内边缘移动,然后向外移动。当它们向内移动时,由于集中磁场的半径较小,它们受到磁场增加的影响。燃料中的低能粒子将显示离开这个不断增加的磁场,开始向后通过燃料,与高能量的原子核碰撞,将它们散射出等离子体。这一过程导致燃料从反应堆中流失,尽管这一过程相当缓慢,一个实用的反应堆仍在可及的范围内。[73]

任何受控聚变装置的首要目标之一是达到盈亏平衡聚变反应释放的能量等于维持反应所用的能量。输入能量与输出能量之比表示如下$Q$,盈亏平衡对应于$Q=1$。反应器产生净能量至少需要$Q=1$,但是出于实际原因,希望它高得多。

一旦达到收支平衡,限制条件的进一步改善通常会导致快速增长的Q值。这是因为最常见的聚变燃料(氘和氚各占50\%的比例)的聚变反应所释放出的部分能量是以粒子的形式存在的。这些粒子可以与等离子体中的燃料核发生碰撞并对其进行加热,从而减少所需的外部热量。在某一时刻,即所谓的点火,内部的自热足以使反应在没有任何外部加热的情况下继续进行,对应的是无穷大的$Q$。

在托卡马克的装置中,如果$\alpha$粒子在燃料中停留足够长的时间以保证它们会与燃料碰撞,这种自加热过程将被最大化。当$\alpha$粒子带电时,它们受到的电场与限制燃料等离子体的电场相同。他们在燃料上花费的时间可以通过确保他们在磁场中的轨道保持在等离子体中而最大化。可以证明,当等离子体中的电流约为3毫安时,就会发生这种现象。[74]
\subsubsection{3.4 先进托卡马克装置}
20世纪70年代初,普林斯顿大学对未来托卡马克设计中使用大功率超导磁体的研究考察了磁体的布局。他们注意到,主环形线圈的布置意味着曲率内侧的磁体之间存在着更大的张力,因为它们更靠近。考虑到这一点,他们指出,如果磁体的形状像$D$形而不是$\circ$形,磁体内部的张力将会均匀。这被称为“普林斯顿$D$形线圈”。[75]

这不是第一次考虑这种安排,尽管原因完全不同。安全系数在机器的轴线上变化;纯粹因为几何上的原因,等离子体内部最靠近机器中心的边缘总是较小的,因为那里的长轴较短。这意味着一台平均$q= 2$在某些区域可能仍小于1。在20世纪70年代,有人提出一种方法来抵消这种影响,并生产出平均值更高的设计$q$等离子体只能填充圆环的外半部分,当从侧面观察时,形状像$D$或$C$,而不是通常的圆形截面。

JET 是第一批集成$D$形等离子体的机器之一,它于1973年开始设计工作。这一决定既是出于理论原因,也是出于实践原因;因为环面的内缘的力更大,所以有一个很大的合力向内压在整个反应堆上。$D$形还具有减小合力的优点,并且使支撑的内边缘更平,因此更容易支撑。[76]探索总体布局的代码注意到非圆形会慢慢垂直漂移,这导致添加了一个主动反馈系统来将其保持在中心。[77]一旦JET选择了这种布局,通用原子公司的Doublet III团队将该机器重新设计成具有$D$形横截面的D-IIID,它也被选为日本的JT-60设计。从那时起,这种布局基本上已经普及。

所有聚变反应堆都存在的一个问题是,重元素的存在会导致能量以更快的速度流失,从而冷却等离子体。在核聚变能量发展的最早期,就找到了解决这个问题的方法,即分流器,它本质上是一个大型质谱仪,可以将较重的元素抛出反应堆。这最初是星转子设计的一部分,很容易集成到磁性线圈中。然而,为托卡马克设计转向器是一个非常困难的设计问题。

所有聚变设计中出现的另一个问题是等离子体施加在约束容器壁上的热负荷。有些材料可以承受这种负荷,但它们通常是不受欢迎的昂贵重金属。当这些材料在与热离子的碰撞中溅射时,它们的原子与燃料混合并迅速使其冷却。托卡马克设计中使用的一种解决方案是限制器,这是一种由轻金属制成的小环,投射到燃烧室中,这样等离子体就可以在击中墙壁之前击中它。这侵蚀了限制器,并导致其原子与燃料混合,但这些较轻的材料造成的破坏比墙材料少。

当反应堆移动到$D$形等离子体时,人们很快注意到等离子体逃逸的粒子通量也可以被成形。随着时间的推移,这导致了使用磁场来创建内部分流器的想法,该分流器将较重的元素从燃料中排出,通常朝向反应器底部。在那里,液态锂金属池被用作一种限制器;粒子撞击它并迅速冷却,保留在锂中。由于其所处的位置,这个内部池更容易冷却,尽管一些锂原子被释放到等离子体中,但其非常低的质量使得它比以前使用的最轻金属的问题要小得多。

当机器开始探索这种新的成形等离子体时,他们注意到某些场的排列和等离子体参数有时会进入现在所谓的高约束模式,即$H$模式,这种模式在更高的温度和压力下稳定运行。在$H$模式下工作,这也可以在星状器中看到,现在是托卡马克设计的主要设计目标。

最后指出,当等离子体密度不均匀时,会产生内部电流。这就是所谓的引导电流。这使得一个设计合理的反应堆可以产生一些内部电流来扭曲磁力线,而不需要从外部电源供电。这有许多优点,现代设计都试图通过引导过程产生尽可能多的总电流。

到20世纪90年代初,这些特征和其他特征的结合共同产生了“先进托卡马克”概念。这为包括ITER在内的现代研究奠定了基础。
\subsubsection{3.5 等离子中断}
托卡马克会受到被称为“中断”的事件的影响,这种中断会导致限制在几毫秒内消失。有两个主要的机制。第一种是“垂直位移事件”(VDE),整个等离子体垂直移动,直到它接触到真空室的上部或下部。另一种是“大破裂”,即长波长、非轴对称的磁铃动力学不稳定性导致等离子体被迫形成非对称形状,常常挤压到腔体的顶部和底部。[78]

当等离子体接触容器壁时,它经历快速冷却,即“热淬火”。在大量的中断情况下,这通常伴随着等离子体浓缩时等离子体电流的短暂增加。淬火最终导致等离子体约束解体。在主要中断的情况下,电流再次下降,即“电流抑制”。在VDE中没有看到电流的初始增加,热淬火和电流淬火同时发生。[78]在这两种情况下,等离子体的热负荷和电负荷都会迅速沉积在反应堆容器上,而反应堆容器必须能够承受这些负荷。ITER的设计目标是在其生命周期内处理2600个这样的事件。[79]

对于现代高能设备,其中等离子体电流在ITER 约为15兆安培,在一次大的中断期间中,电流的短暂增加可能会超过临界阈值。当电流对电子产生的力大于等离子体中粒子之间碰撞的摩擦力时,就会发生这种现象。在这种情况下,电子可以被迅速加速到相对论速度,在相对论的逃逸电子雪崩中产生所谓的“逃逸电子”。即使在大部分等离子体发生电流猝灭时,它们仍然保持能量。[79]

当当约束最终被打破时,这些逃逸电子会沿着阻力最小的路径运动,并撞击反应堆的一侧。这些可以达到12兆安培的电流沉积在一个小区域,远远超过任何机械解决方案的能。[78]在一个著名的例子中,“方特奈玫瑰”托卡马克发生了重大破坏,逃逸电子在真空室中烧了一个洞。[79]

托卡马克运行中的重大中断发生率一直相当高,约为总发射次数的百分之几。在目前运行的托卡马克中,损害通常很大,但很少引人注目。在ITER托卡马克中,如果发生有限数量的重大破坏,则肯定会破坏反应堆的燃烧室,使设备无法恢复。[80][81][82]发展系统来对抗失控电子的影响被认为是操作级ITER必须具备的一项技术。[79]

中心电流密度的大振幅也可能导致内部中断、或者锯齿状的电流,而这些通常不会导致放电的终止。[83]

\subsection{等离子加热}
在一个运行的聚变反应堆中,当引入新的氘和氚时,产生的部分能量将用来维持等离子体的温度。然而,在启动反应堆时,无论是最初还是在临时关闭后,等离子体都必须加热到其工作温度超过10 keV(超过1亿摄氏度)。在目前的托卡马克(和其他)磁聚变实验中,产生的聚变能不足以维持等离子体的温度,必须提供恒定的外部加热。中国研究人员建立实验先进超导托卡马克(东)在2006年被认为是维持1亿摄氏度等离子体(太阳有1500万摄氏度温度),需要启动氢原子之间的聚变,根据最新的测试进行的东部(在2018年11月)进行测试。
\subsubsection{4.1 欧姆加热~感应模式}
因为等离子体是电导体,所以可以通过感应电流来加热等离子体;提供极向场大部分的感应电流也是初始加热的主要来源。

由感应电流引起的加热被称为欧姆(或电阻)加热;这与电灯泡或电加热器中的加热方式相同。产生的热量取决于等离子体的电阻和流过它的电流量。但是随着加热等离子体温度的升高,电阻降低,欧姆加热效果降低。托卡马克中欧姆加热可以达到的最高等离子体温度似乎是2000万-3000万摄氏度。为了获得更高的温度,必须使用额外的加热方法。

电流是通过不断增加通过与等离子体环相连的电磁绕组的电流来感应的:等离子体可以被视为变压器的次级绕组。这本质上是一个脉冲过程,因为通过初级线圈的电流是有限制的(长脉冲也有其他限制)。因此,托卡马克必须要么短时间运行,要么依赖其他加热和电流驱动手段。
\subsubsection{4.2 磁压缩}
气体可以通过突然压缩而加热。同样地,如果通过增加限制磁场来快速压缩等离子体,等离子体的温度也会升高。在托卡马克中,这种压缩是简单地通过移动等离子体到一个更高的磁场区域(即径向向内)。由于等离子体压缩使离子更接近,该方法还有另外一个好处,即有利于达到聚变反应器所需的密度。

磁压缩是早期“托卡马克踩踏事件”的研究领域,也是ATC主要设计的目的之一。尽管一个有点类似的概念是General Fusion 设计的一部分,但这个概念从那时起就没有被广泛使用。
\subsubsection{4.3 中性束注入}
中性束注入涉及将高能(快速移动)原子或分子引入托卡马克内的欧姆加热、磁约束等离子体。

高能原子以离子的形式在电弧腔中产生,然后通过高压栅极提取。“离子源”一词一般指由一组发射电子的灯丝、一段电弧腔体和一组提取栅极组成的装置。第二个装置,在概念上类似,是用来分别加速电子到相同的能量。电子的质量轻得多,使得这个器件比它的离子对应器件小得多。然后,两束粒子束相交,在那里离子和电子重新结合成中性原子,使它们能够穿过磁场。

一旦中性束进入托卡马克,就会与主要等离子体离子发生相互作用。这有两个影响。一个是注入的原子重新电离并带电,从而被困在反应堆内,增加了燃料质量。另一个是电离的过程是通过与其余燃料的撞击发生的,这些撞击将能量沉积在燃料中,使其升温。

与欧姆法相比,这种加热形式没有固有的能量(温度)限制,但其速率受限于喷射器中的电流。离子源提取电压通常约为50-100千伏,正在为ITER开发高压负离子源(-1 MV)。帕多瓦的ITER中性束测试设施将是第一个开始运行的ITER设施。[84]

虽然中性束注入主要用于等离子体加热,但它也可以用作诊断工具,并通过产生由一串短的2-10 ms束信号组成的脉冲束来进行反馈控制。氘是中性束加热系统的主要燃料,氢和氦有时用于选定的实验。
\subsubsection{4.4 射频加热}
\begin{figure}[ht]
\centering
\includegraphics[width=8cm]{./figures/331ad2df8378feca.png}
\caption{通过托卡马克à配置变量 (TCV)上的电子回旋波为等离子体加热提供的一组超高频管(84千兆赫和118千兆赫)。由EPFL太平洋共同体秘书处提供} \label{fig_TKMK_6}
\end{figure}
高频电磁波是由环外的振荡器(通常是由回旋管或速调管)产生的。如果这些波具有正确的频率(或波长)和极化,它们的能量就可以转移到等离子体中的带电粒子上,这些带电粒子又与其他等离子体粒子发生碰撞,从而提高了体等离子体的温度。电子回旋共振加热(ECRH)和离子回旋共振加热是目前存在的多种加热技术。这种能量通常是通过微波传递的。

\subsection{托卡马克粒子剩余}
托卡马克真空室中的等离子体放电由带电离子和原子组成,这些粒子的能量通过辐射、碰撞或不受约束最终到达真空室的内壁。燃烧室内壁水冷,颗粒的热量通过壁传导到水,加热水对流到外部冷却系统。

涡轮分子泵或扩散泵允许粒子从大体积中排出,由液氦冷却表面组成的低温泵通过为发生冷凝提供能量汇来有效控制整个排放过程中的密度。如果操作正确,聚变反应会产生大量高能中子。中子是电中性的,相对较小,不受磁场影响,也不会被周围的真空室阻止。

在托卡马克周围所有方向的特制中子屏蔽边界处,中子通量显著降低。屏蔽材料各不相同,但通常是由接近中子大小的原子制成的材料,因为这些材料最能吸收中子及其能量。良好的候选材料包括那些氢含量高的材料,如水和塑料。硼原子也是中子的良好吸收剂。因此,掺硼的混凝土和聚乙烯可以制成廉价的中子屏蔽材料。

一旦释放,中子的半衰期相对较短,约为10分钟,然后随着能量的发射衰变为质子和电子。当真正尝试从托卡马克反应堆发电时,聚变过程中产生的一些中子将被液态金属覆盖层吸收,它们的动能将被用于传热过程,最终驱动发电机。

\subsection{托卡马克实验}
\subsubsection{6.1 目前正在运行}
(按开始运行的时间顺序)
\begin{figure}[ht]
\centering
\includegraphics[width=6cm]{./figures/b21d4f49ff11d113.png}
\caption{Alcator C-Mod} \label{fig_TKMK_7}
\end{figure}
\begin{itemize}
\item 20世纪60年代:TM1-MH(1977年起称为Castor;自2007年称为Golem)在布拉格,捷克共和国。从20世纪60年代早期开始在库尔恰托夫研究所运行,但在1977年更名为Castor,并迁移到布拉格的IPP CAS。[85]布拉格。2007年迁至位于布拉格的捷克技术大学FNSPE,更名为Golem。[86]
\item 1975年:T-10,俄罗斯莫斯科库尔恰托夫研究所(前苏联);2兆瓦
\item 1983年:欧洲联合环流器(JET),在英国的卡勒姆
\item 20世纪86年代:DIII-D,[87]在圣地亚哥,通用原子公司从20世纪80年代末开始运营
\item 1987年:STOR-M,萨斯喀彻温大学;加拿大;首次在托卡马克上演示交流电。
\item 1988年:Tore Supra,[88]在法国卡达拉舍 CEA
\item 1989年:Aditya,等离子研究所在印度古吉拉特邦
\item 1989年:指南针,[85]在捷克共和国的布拉格;自2008年开始运行,之前于1989年至1999年在英国库勒姆运行
\item 1990年: FTU ,[89]在意大利的弗拉斯卡蒂
\item 1991年: ISTOK ,[90]在葡萄牙里斯本里斯本的等离子体和扶桑核研究所;
\end{itemize}
\begin{figure}[ht]
\centering
\includegraphics[width=8cm]{./figures/9ffe9edfa49ae917.png}
\caption{NSTX 反应堆的外部视图} \label{fig_TKMK_8}
\end{figure}
\begin{itemize}
\item 1991年:ASDEX升级,在德国的Garching。
\item 1992年:H-1NF(H-1国家等离子聚变研究设施)[91]该设施基于澳大利亚国立大学等离子体物理小组建造的H-1 Heliac装置,自1992年开始运行
\item 1992年:托卡马克配置变量(TCV),在瑞士的EPFL
\item 1994年:TCABR,在巴西圣保罗圣保罗大学;这台托卡马克是从瑞士等离子中心
\item 1995年:HT-7 ,中国合肥等离子体物理研究所
\item 1996年:威斯康星大学麦迪逊分校飞马星座实验[92]在威斯康星大学麦迪逊分校;从20世纪90年代末开始运作
\item 1999年:在普林斯顿 (新泽西州)NSTX
\item 1999年:在俄罗斯Ioffe Institute圣彼得堡的Globus-M
\item 2002年:HL-2A,中国成都
\item 2006年:东方(HT-7U),·        合肥,中国合肥物理科学研究院(ITER成员)
\item 2008年:韩国大田KSTAR (ITER成员)
\item 2010年:日本那卡JT-60SA (ITER成员);从JT-60升级。
\item 2012年:Medusa CR,在哥斯达黎加理工学院
\item 2012年:印度甘地那加等离子体研究所SST-1 (ITER成员)
\item 2012年:伊朗德黑兰伊斯兰阿扎德大学科学与研究部IR-T1[93]
\item 2015年:英国库勒姆托卡马克能源有限公司ST25-HTS
\item 2017年:KTM (奥地利公司)——这是一个实验性热核设施,用于在能源负载条件下研究和测试材料,靠近哈萨克斯坦的ITER和未来的能源聚变反应堆。
\item 2018年:英国库勒姆托卡马克能源有限公司ST40
\end{itemize}
\subsubsection{6.2 之前的实验}
\begin{figure}[ht]
\centering
\includegraphics[width=6cm]{./figures/862930ffcd55bafd.png}
\caption{麻省理工学院等离子体科学与聚变中心阿尔卡特C托卡马克的控制室,大约在1982-1983年。} \label{fig_TKMK_9}
\end{figure}
\begin{itemize}
\item 1960年代:T-3和T-4,俄罗斯莫斯科库尔恰托夫研究所(前苏联);T-4在1968年投入使用。
\item 1963年:LT-1,澳大利亚国立大学的等离子体物理小组建立了一个设备来探索环形结构,独立发现托卡马克布局
\item 1970年5月:仿星器C在PPPL重新开放对称托卡马克
\item 1971-1980年:美国德克萨斯大学奥斯汀分校,德州湍流托卡马克
\item 1972年:绝热环形压缩机在PPPL开始运行
\item 1973-1976年:东方玫瑰托卡马克(TFR),靠近法国巴黎
\item 1973-1979年:美国麻省理工学院Alcator A
\item 1975年:普林斯顿大圆环开始在PPPL运行
\item 1978-1987年:美国麻省理工学院Alcator C
\item 1978-2013年:德国Julich的TEXTOR
\item 1979-1998年:匈牙利布达佩斯MT-1托卡马克(建于俄罗斯库尔恰托夫研究所,1979年运到匈牙利,1991年重建为MT-1M)
\item 1980-1990年:南非原子能委员会托卡马克[94]
\item 1980-2004年:美国德克萨斯大学奥斯汀分校TEXT/TEXT-U
\item 1982-1997年:美国普林斯顿大学TFTR
\item 1983-2000年:诺维略·托卡马克,[95]在墨西哥墨西哥城的国家核研究所
\item 1984-1992年:中国成都HL-1托卡马克
\item 1985-2010年:JT-60,日本茨城县那卡市;(2015-2018升级至超级先进机型)
\item 1987-1999年:托卡马克·德瓦雷纳;瓦雷纳,加拿大;由魁北克水电公司操作,由来自魁北克电力研究所(IREQ)和国家科学研究所(INRS)的研究人员使用
\item 1988-2005年:T-15,俄罗斯莫斯科库尔恰托夫研究所(前苏联);10兆瓦
\item 1991-1998年:开始在英国运行的库勒姆
\item 20世纪90年代-2001年:在英国的库勒姆,指南针
\item 1994-2001年:中国成都托卡马克(HL-1M Tokamak)
\item 1999-2006年:美国洛杉矶加州大学洛杉矶分校电子托卡马克
\item 1999-2014年:桅杆,在英国的库勒姆
\item 1992-2016年:Alcator C-Mod ,[96] 美国麻省理工学院,和英国剑桥
\end{itemize}
\subsubsection{6.3 计划的项目}
\begin{figure}[ht]
\centering
\includegraphics[width=8cm]{./figures/0135e20a1b683539.png}
\caption{目前正在建设中的ITER将是目前最大的托卡马克。} \label{fig_TKMK_10}
\end{figure}
\begin{itemize}
\item ITER ,卡达拉舍国际项目;500兆瓦;建造始于2010年,第一批等离子体预计在2025年。预计到2035年全面运行。[97]
\item DEMO ;2000兆瓦,连续运行,接入电网。ITER的计划继承人;根据初步时间表,工程将于2024年开工。
\item CFETR ,也称为“\textbf{中国聚变工程实验堆}”;200兆瓦;下一代中国聚变堆,是一种新型托卡马克装置。[98][99][100][101]
\item 韩国的K-DEMO;2200-3000兆瓦,计划净发电量约为500兆瓦;建设目标是到2037年。[102]
\end{itemize}

\subsection{注解}
\begin{enumerate}
\item Shafranov also states the term was used "after 1958".[103]
\item D-T fusion occurs at even lower energies, but tritium was unknown at the time. Their work created tritium, but they did not separate it chemically to demonstrate its existence. This was performed by Luis Alvarez and Robert Cornog in 1939.[103]
\item The system Lavrentiev described is very similar to the concept now known as the fusor.
\item Although one source says "late 1957".[104]
\end{enumerate}

\subsection{参考文献}
[1]
^Greenwald, John (24 August 2016). "Major next steps for fusion energy based on the spherical tokamak design". Princeton Plasma Physics Laboratory. United States Department of Energy. Retrieved 16 May 2018..

[2]
^"Tokamak - Definition of tokamak by Merriam-Webster". merriam-webster.com..

[3]
^Oliphant, Mark; Harteck, Paul; Rutherford, Ernest (1934). "Transmutation Effects Observed with Heavy Hydrogen". Proceedings of the Royal Society. 144 (853): 692–703. Bibcode:1934RSPSA.144..692O. doi:10.1098/rspa.1934.0077..

[4]
^McCracken & Stott 2012, p. 35..

[5]
^McCracken & Stott 2012, pp. 36–38..

[6]
^Bromberg 1982, p. 18..

[7]
^Herman, Robin (1990). Fusion: the search for endless energy. Cambridge University Press. p. 40. ISBN 978-0-521-38373-8..

[8]
^Shafranov 2001, p. 873..

[9]
^Bondarenko, B.D. (2001). "Role played by O. A. Lavrent'ev in the formulation of the problem and the initiation of research into controlled nuclear fusion in the USSR" (PDF). Phys. Usp. 44 (8): 844. doi:10.1070/PU2001v044n08ABEH000910..

[10]
^Shafranov 2001, p. 837..

[11]
^Bromberg 1982, p. 15..

[12]
^Shafranov 2001, p. 838..

[13]
^Shafranov 2001, p. 839..

[14]
^Arnoux, Robert (26 October 2011). "'Proyecto Huemul': the prank that started it all". iter..

[15]
^Bromberg 1982, p. 75..

[16]
^Bromberg 1982, p. 14..

[17]
^Bromberg 1982, p. 21..

[18]
^Bromberg 1982, p. 25..

[19]
^Adams, John (31 January 1963). "Can we master the thermonuclear plasma?". New Scientist. pp. 222–225..

[20]
^Cowley, Steve. "Introduction to Kink Modes – the Kruskal- Shafranov Limit" (PDF). UCLA..

[21]
^Kadomtsev 1966..

[22]
^Clery, Daniel (2014). A Piece of the Sun: The Quest for Fusion Energy. MIT Press. p. 48. ISBN 978-1-4683-1041-2..

[23]
^Bromberg 1982, p. 70..

[24]
^Shafranov 2001, p. 240..

[25]
^Kurchatov, Igor (26 April 1956). The possibility of producing thermonuclear reactions in a gaseous discharge (PDF). UKAEA Harwell..

[26]
^McCracken & Stott 2012, p. 5..

[27]
^Shafranov 2001, p. 841..

[28]
^столетию·о·днярождения。а。явлинского.

[29]
^в。д。шафранов·ксосиииииилееоспппоуправляемомутермоядерному·ссеет.

[30]
^Shafranov, Vitaly (2001). "On the history of the research into controlled thermonuclear fusion" (PDF). Journal of the Russian Academy of Sciences. 44 (8): 835–865..

[31]
^Arnoux, Robert. "Which was the first "tokamak"—or was it "tokomag"?". ITER. Retrieved 6 November 2018..

[32]
^Herman 1990, p. 53..

[33]
^Smirnov 2009, p. 2..

[34]
^Shafranov 2001, p. 842..

[35]
^Bromberg 1982, p. 66..

[36]
^Spitzer, L. (1960). "Particle Diffusion across a Magnetic Field". Physics of Fluids. 3 (4): 659. Bibcode:1960PhFl....3..659S. doi:10.1063/1.1706104..

[37]
^Bromberg 1982, p. 130..

[38]
^Bromberg 1982, p. 153..

[39]
^Bromberg 1982, p. 151..

[40]
^Bromberg 1982, p. 166..

[41]
^Bromberg 1982, p. 172..

[42]
^"The Valleys boy who broached the Iron Curtain to convince the USA that Russian Cold War nuclear fusion claims were true". WalesOnline. 3 November 2011..

[43]
^Arnoux, Robert (9 October 2009). "Off to Russia with a thermometer". ITER Newsline. No. 102..

[44]
^Bromberg 1982, p. 167..

[45]
^Peacock, N. J.; Robinson, D. C.; Forrest, M. J.; Wilcock, P. D.; Sannikov, V. V. (1969). "Measurement of the Electron Temperature by Thomson Scattering in Tokamak T3". Nature. 224 (5218): 488–490. Bibcode:1969Natur.224..488P. doi:10.1038/224488a0..

[46]
^Kenward, Michael (24 May 1979). "Fusion research - the temperature rises". New Scientist..

[47]
^Cohen, Robert S.; Spitzer, Jr., Lyman; McR. Routly, Paul (October 1950). "The Electrical Conductivity of an Ionized Gas" (PDF). Physical Review. 80 (2): 230–238. Bibcode:1950PhRv...80..230C. doi:10.1103/PhysRev.80.230..

[48]
^Bromberg 1982, p. 161..

[49]
^Bromberg 1982, p. 152..

[50]
^Bromberg 1982, p. 154..

[51]
^Bromberg 1982, p. 158..

[52]
^Bromberg 1982, p. 159..

[53]
^Bromberg 1982, p. 164..

[54]
^Bromberg 1982, p. 165..

[55]
^Bromberg 1982, p. 168..

[56]
^Bromberg 1982, p. 169..

[57]
^Bromberg 1982, p. 171..

[58]
^Bromberg 1982, p. 212..

[59]
^"Timeline". PPPL..

[60]
^Bromberg 1982, p. 173..

[61]
^Bromberg 1982, p. 175..

[62]
^Smirnov 2009, p. 5..

[63]
^Bromberg 1982, p. 10..

[64]
^Bromberg 1982, p. 215..

[65]
^Arnoux, Robert (25 October 2010). "Penthouse founder had invested his fortune in fusion". ITER..

[66]
^Reagan, Ronald (19 April 1986). "Radio Address to the Nation on Oil Prices". The American Presidency Project..

[67]
^Arnoux, Robert (15 December 2008). "INTOR: The international fusion reactor that never was". ITER..

[68]
^《苏美关于日内瓦首脑会议的联合声明》·罗纳德·里根。1985年11月21日.

[69]
^Educational Foundation for Nuclear Science, Inc. (October 1992). Bulletin of the Atomic Scientists. Educational Foundation for Nuclear Science, Inc. pp. 9–. ISSN 0096-3402..

[70]
^Braams, C.M.; Stott, P.E. (2010). Nuclear Fusion: Half a Century of Magnetic Confinement Fusion Research. Taylor & Francis. pp. 250–. ISBN 978-0-7503-0705-5..

[71]
^Wesson 1999, p. 13..

[72]
^Bromberg 1982, p. 16..

[73]
^Wesson 1999, pp. 15-18..

[74]
^Wesson, John (November 1999). The Science of JET (PDF). JET Joint Undertaking. p. 20..

[75]
^Gray, W.H.; Stoddart, W.C.T. (1977). (Technical report). Oak Ridge National Laboratory https://www.osti.gov/servlets/purl/5233082. Missing or empty |title= (help).

[76]
^Wesson 1999, p. 22..

[77]
^Wesson 1999, p. 26..

[78]
^克鲁格,东南大学;Schnack,D. D .Sovinec,中华人民共和国(2005年)。“DIII-D等离子体主要破坏的动力学”。物理等离子体12,056113。doi:10.1063/1.1873872。.

[79]
^托卡马克中的逃逸电子及其在ITER的缓解ITER组织普京斯基.

[80]
^Wurden, G. A. (9 September 2011). Dealing with the Risk and Consequences of Disruptions in Large Tokamaks (PDF). MFE Roadmapping in the ITER Era. Archived from the original (PDF) on 5 November 2015..

[81]
^洛杉矶贝勒;库姆斯,S. K中华人民共和国第四;新泽西州杰尼根;迈特纳;巴黎公园;考曼,法学学士;费林,D. T .丸山;Qualls,A. L .拉斯穆森特区;Thomas,c . e .(2009年)。ITER的颗粒燃料、ELM起搏和中断缓解技术开发“”。努克。聚变49 085013。doi:10.1088/0029-5515/49/8/085013。>.

[82]
^桑顿,美国法学会;吉布森,K. J哈里索纳;Kirka,a;s . w . lis goc;Lehnend,m;马丁纳;纳罗拉,g;斯坎内拉;Cullena和a . MAST Team Thornton(2011年)。“百万安培球形托卡马克中断缓解研究”。努克尔杂志。马特。415,1,补编,1,S836-S840。doi:10.1016/j.jnucmat.2010.10.029。.

[83]
^Goeler,v .等等。(1974年)。用软x射线技术研究托卡马克放电中的内部干扰和m= 1振荡物理评论快报,第20卷,第1201页。.

[84]
^Neutral Beam Test Facility (PDF) (Technical report)..

[85]
^"Tokamak Department, Institute of Plasma Physics". cas.cz. Archived from the original on 2015-09-01..

[86]
^假人历史.

[87]
^DIII-D (视频).

[88]
^Tore Supra Archived 11月 15, 2012 at the Wayback Machine.

[89]
^EMazzitelli, Giuseppe. "ENEA-Fusion: FTU". www.fusione.enea.it..

[90]
^"Centro de Fusão Nuclear". utl.pt..

[91]
^融合研究:澳大利亚的联系、过去和未来B.布莱克韦尔,(1)霍尔、霍华德和奥康纳.

[92]
^"Pegasus Toroidal Experiment". wisc.edu..

[93]
^"Tokamak". Pprc.srbiau.ac.ir. Retrieved 2012-06-28..

[94]
^De Villiers, J. A. M; Hayzen, A. J; Omahony, J. R; Roberts, D. E; Sherwell, D (1979). "Tokoloshe - the South African Tokamak". South African Journal of Science. 75: 155. Bibcode:1979SAJSc..75..155D..

[95]
^Ramos, J.; Meléndez, L.; et al. (1983). "Diseño del Tokamak Novillo" (PDF). Rev. Mex. Fís. 29 (4): 551–592..

[96]
^"MIT Plasma Science & Fusion Center: research>alcator>". mit.edu. Archived from the original on 2015-07-09..

[97]
^"ITER & Beyond. The Phases of ITER". Archived from the original on 22 September 2012. Retrieved 12 September 2012..

[98]
^https://web.archive.org/web/20221028213927/http://www-naweb . IAEA . org/napc/physics/meetings/TM45256/talks/Gao . pdf.

[99]
^Zheng, Jinxing; Liu, Xufeng; Song, Yuntao; Wan, Yuanxi; Li, Jiangang; Wu, Sontao; Wan, Baonian; Ye, Minyou; Wei, Jianghua; Xu, Weiwei; Liu, Sumei; Weng, Peide; Lu, Kun; Luo, Zhengping (2013). "Concept design of CFETR superconducting magnet system based on different maintenance ports". Fusion Engineering and Design. 88 (11): 2960–2966. doi:10.1016/j.fusengdes.2013.06.008..

[100]
^Song, Yun Tao; et al. (2014). "Concept Design of CFETR Tokamak Machine". IEEE Transactions on Plasma Science. 42 (3): 503–509. Bibcode:2014ITPS...42..503S. doi:10.1109/TPS.2014.2299277..

[101]
^Ye, Minyou (26 March 2013). "Status of design and strategy for CFETR" (PDF)..

[102]
^Kim, K.; Im, K.; Kim, H.C.; Oh, S.; Park, J.S.; Kwon, S.; Lee, Y.S.; Yeom, J.H.; Lee, C.; Lee, G-S.; Neilson, G.; Kessel, C.; Brown, T.; Titus, P.; Mikkelsen, D.; Zhai, Y. (2015). "Design concept of K-DEMO for near-term implementation". Nuclear Fusion. 55 (5): 053027. doi:10.1088/0029-5515/55/5/053027. ISSN 0029-5515..

[103]
^Shafranov 2001, p. 840..

[104]
^Arnoux, Robert (27 October 2008). "Which was the first "tokamak" — or was it "tokomag"?". ITER..