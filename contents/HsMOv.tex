% 【导航】高中数学
% keys 高中|数学|概述
% license Xiao
% type Map

\begin{issues}
\issueDraft
\end{issues}

% Giacomo:首先要确定目标

本节为高中数学章节,参考高中数学课本,同时增加一定的深度,帮助高中生更好的学习备考,祝各位备考顺利!

\subsection{数列与函数}

\subsection{三角函数}

\subsection{排列、组合和概率与统计}

\subsection{解方程}

\subsection{几何向量}

线性代数的研究对象是向量和矩阵,而我们最早认识的向量就是\textbf{几何向量},这里我们回顾几何向量的相关概念。

几何向量的存在与坐标系无关,它是一些有长度有方向的箭头。我们把(二维)平面中的向量称为平面向量,(三维)空间中的向量称为空间向量;在高中数学的语境下,我们把(一维)直线上的向量称为标量,但这是不严谨的。

% 对于讨论问题的不同,我们有时仅需要处于同一平面(\textbf{二维空间})的所有几何向量,有时需要\textbf{三维空间}中的所有几何向量,最简单的情况下只需要沿某条线(\textbf{一维空间})的所有几何向量(这时我们可以规定一个正方向,且仅使用几何向量的模长加正负号来表示几何向量以简化书写)。

\addTODO{高中数学中平面向量和空间向量的链接}
% Giacomo:是不是应该把高中数学/物理,改成中学数学/物理?

几何向量有起点(箭尾)和终点(箭头),但我们对几何向量的绝对位置不感兴趣,我们只在乎起点和终点的相对位置,即两个几何向量如果有相同的方向和长度就被视为同一个向量。

几何向量的一些基本运算\upref{GVec} 同样不需要有任何坐标系的概念,\textbf{几何向量相加}按照三角形法则或平行四边形法则即可。
\textbf{几何向量数乘}就是把几何向量的模长乘以一个实数,若乘以正数,方向不变,若乘以负数,取相反方向。 \textbf{几何向量的线性组合}是把若干几何向量分别乘以一个实数再相加得到新的几何向量。

几何向量的\textbf{内积}\upref{Dot}等于一个几何向量在另一个几何向量上的投影长度乘以另一个几何向量的模长得到一个实数,几何向量的\textbf{模长}等于几何向量与自身内积再开方,把几何向量除以自身模长使模长变为单位长度的过程叫做\textbf{归一化}。若两几何向量内积为零,这两个几何向量相互\textbf{正交}\footnote{对于几何向量,正交就是方向垂直,不加区分。}。

三维欧几里得空间中,两几何向量\textbf{叉乘}\upref{Cross}得到的几何向量垂直于两几何向量,模长为一个几何向量在另一个几何向量垂直方向的投影长度乘以另一个几何向量的模长。

为了方便描述几何向量之间的关系,我们选取一些\textbf{线性无关}的几何向量作为所有几何向量的\textbf{基底},使空间中的任何几何向量可以用这些基底的唯一一种线性组合来表示,$N$ 维空间需要 $N$ 个基底向量。一般来说,基底不必互相正交。我们先把这些基底排序,任意几何向量表示成它们的线性组合时,把式中的 $N$ 个系数按照顺序排列,就是该几何向量的\textbf{坐标},通常用列几何向量表示。由于线性组合的唯一性,每个几何向量的坐标是唯一的。

为了方便计算任意几何向量的坐标,往往取\textbf{正交归一}的基底\upref{OrNrB}(所有基底模长为1,任意两基底互相正交)。这样,任意向量的坐标都可以通过与基底的内积得到。

\addTODO{添加相关文章的链接}

\subsubsection{几何向量的线性变换}
\addTODO{科普版本的线性映射}

我们可以设计一种规则把某个空间的任意几何向量\textbf{变换}(\textbf{映射})到另一个空间的几何向量;如果任意几何向量线性组合的变换等于这些几何向量分别变换再线性组合,这个变换就被称为\textbf{线性变换}\upref{LTrans}。

\addTODO{具体例子旋转变换}



\addTODO{链接待处理,文本待处理}

\subsection{平面几何与立体几何}
