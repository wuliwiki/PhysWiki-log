% 南京航空航天大学 2007 量子真题答案
% license Usr
% type Note

\textbf{声明}:“该内容来源于网络公开资料,不保证真实性,如有侵权请联系管理员”

\subsection{一}
\subsubsection{1.}
解:设厄米函数$\hat{F}$的本征值为$\lambda$,$\hat{F}\psi = \lambda \psi$\\
在厄米算符定义式$$\int \psi^* \hat{F} \phi d\tau = \int (\hat{F} \psi)^* \phi d\tau~$$中,令$\phi=\psi$,则$$\lambda \int \psi^* \varphi d\tau = \lambda^* \int \psi^* \psi d\tau~$$\\
$\therefore \lambda=\lambda^*$得证。
\subsubsection{2.}
解:它取极小值的条件为
$$\frac{\partial E}{\partial \overline{(\Delta x)^2}} = 0~$$
由此得出
$$\overline{(\Delta x)^2} = \frac{\hbar}{2 m \omega}~$$
用此值代入(3)式, 可知
$$E \geq \frac{1}{2} \hbar \omega~$$
所以讲振子基态能量
$$E = \frac{1}{2} \hbar \omega~$$
由于一维谐振子势具有对坐标原点的反射对称性,我们有
$$ x = 0, \quad  p = 0~$$
因而
$$ \overline{\Delta x^2} =\overline{x^2} -\overline{x}^2 =\overline{x^2}~ $$
$$ \overline{\Delta p^2} =\overline{p^2} - \overline{p}^2 =\overline{p^2}~ $$
所以在能量本征态下
$$ E = \frac{\overline{p^2}}{2m} + \frac{1}{2} m \omega^2\overline{x^2}= \frac{\overline{\Delta p^2}}{2m} + \frac{1}{2} m \omega^2 \overline{\Delta x^2}~$$
按不确定性关系
$$ (\overline{\Delta x)^2}.\overline{(\Delta p)^2}\geq \frac{\hbar^2}{4}~$$
所以
$$ E \geq \frac{\hbar^2}{8m (\Delta x)^2} + \frac{1}{2} m \omega^2 \overline{(\Delta x)^2}~$$

