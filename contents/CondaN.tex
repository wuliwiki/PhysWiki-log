% Conda 笔记

\begin{itemize}
\item Anaconda 太大了, 推荐用 miniconda, google miniconde 可以下载 win 安装包或者 linux/mac 的 sh 安装脚本(\verb|bash xxx.sh|).
\item windows 中安装 miniconda/anaconda 以后使用 anaconda prompt, linux/mac 使用系统命令行即可.
\item 更新 \verb|conda update conda|; 卸载 \verb|rm -rf ~/miniconda|
\item \verb|conda info -e| 进入 conda(linux/mac)并查看虚拟环境
\item \verb|conda deactivate| 退出 conda
\item \verb|conda create -n 新环境名| 创建新环境
\item 【conda 4.14】\verb|conda rename -n 旧环境名 -d 新环境名| (旧版本只能用 clone)
\item \verb|conda activate 环境名| 切换到某个虚拟环境.
\item \verb|conda remove --name 环境名 --all| 删除虚拟环境(不能是当前环境).
\item \verb|conda install anaconda-navigator| 安装完以后, 就可以在开始菜单搜到 Anaconda Navigator 了.
\item \verb|conda install -n 环境名 包名称|. 如果没有 activate 任何环境, 那么默认安装到 \verb|base| 环境中.
\item \verb|conda create --name 新环境名 --clone 环境名| 复制环境
\item 常见包如 \verb|conda install numpy scipy matplotlib|
\item \verb|conda list| 列出当前环境的所有包 \verb|conda list 包名| 给出包的详细信息.
\item \verb|conda remove 包名| 卸载包
\item \verb|conda search 包名| 搜索包
\item \verb|conda install 安装包 -c 频道|, 频道如 \verb|conda-forge|, 如果不指定频道可能会找不到包.
\item 如果要指定包的版本, 用 \verb|conda install 包名=0.13.1|
\item \verb|import module名字| 以后, 可以用 \verb|module名字.__file__| 查看 module 的文件路径.
\item conda 包的默认安装路径是 \verb|/home/用户/miniconda3/lib/python3.9/site-packages/|
\end{itemize}

\subsection{一些常用的包}
\begin{itemize}
\item \verb|-c conda-forge| 是一个很常用的 channel.
\item \verb|anaconda-navigator|. 启动: \verb|anaconda-navigator|. 可以直接在里面安装一些东西.
\item \verb|notebook| (jupyter notebook)
\item 
\end{itemize}
