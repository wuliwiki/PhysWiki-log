% 引力原子
% license Usr
% type Tutor

我们的世界由原子构成:即核和电子通过电磁力结合在一起。同时,尽管并非所有粒子都带有电荷,但它们都有引力“电荷”(即它们的质量)。而且引力是普遍存在的且具有吸引力。因此很自然地会问:为什么我们没有观察到任何由引力束缚的“原子”?

答案是引力很弱:一个小磁铁就能提起一个钉子,对抗整个地球的引力。引力的弱小意味着一个引力“氢原子”的半径将大于可观测宇宙的大小。因此要制造一个引力原子,我们需要寻找引力强的地方——比如黑洞周围!

想象一个非常轻的粒子,其质量为$\mu$,康普顿波长$\lambda = h / (\mu c)$,与黑洞事件视界的尺寸相当(或者简称为“视界”)。这个粒子可以与黑洞形成一个束缚态:本质上,这是一个以黑洞“核”和轻粒子作为“电子”的引力“原子”。

\begin{figure}[ht]
\centering
\includegraphics[width=6cm]{./figures/b414726427066c58.png}
\caption{一个“引力原子”的示意图:一个旋转的黑洞核周围环绕着紫色的轴子概率云。} \label{fig_GAtom_1}
\end{figure}

将库仑力定律$F = e^2 / 4\pi r^2 = \alpha / r^2$与牛顿引力定律$F = G_N m_1 m_2 / r^2$进行比较,其中$\alpha$是电磁耦合常数,通过简单地定义引力“精细结构常数”为$\alpha_{grav} = G_N m_1 m_2$,我们可以将我们对氢原子的所有了解映射到引力原子上。(在黑洞附近,牛顿定律有修正,但在我们的近似中这些修正很小)。例如,引力原子具有量子化的能级,能级量子数为$(n, l, m)$,能量与典型氢原子的能级相当。

与氢原子类似,引力原子的大小由“玻尔半径”$r = h / (\mu c \alpha_{grav})$给出。我们需要一个非常轻的粒子来形成我们的引力原子:对于更重的粒子,原子的大小将在黑洞视界内部。

太阳质量黑洞的视界大约是$10$公里,这与质量大约为$10^{-11} eV/c^2$的粒子的康普顿波长相当—比电子轻10¹⁷倍!然而,有充分的理由相信这种超轻粒子的存在。最好的例子是QCD轴子,它在大约40年前被提出作为解决标准模型中强CP问题的一种方案,并且仍然是我们宇宙暗物质的一个极好且完全可行的候选者。QCD轴子的另一个方面使得这些引力原子变得有趣,那就是它是一个玻色子,因此不受泡利不相容原理的限制。这意味着与氢原子不同,引力原子原则上可以在每个能级上拥有无限数量的轴子。

\subsection{黑洞“超辐射”}

与黑洞束缚的光粒子是一个美妙的想法,但它能被观测到吗?这样的引力原子是如何形成的?

黑洞超辐射是当一个粒子(或光)波通过一个旋转黑洞附近时,它以比进入时更大的振幅离开黑洞环境,通过从黑洞提取能量和动量。这个过程只有在粒子满足超辐射条件$E < m\Omega_{BH}$时才会发生。其中$E$是束缚态的总能量,$m$是磁量子数,$\Omega_{BH}$是黑洞视界的角速度。如果粒子的角速度小于黑洞的角速度,就会发生能量和动量的增益。

虽然这听起来很神秘,但黑洞超辐射只是出现在各种系统中的一种现象的表现。最著名的是“惯性运动超辐射”—也就是切伦科夫辐射。在切伦科夫辐射中,一个非加速的带电粒子如果比介质中的光速移动得更快,就会自发地发射辐射,而在粒子后面的锥形区域内散射的辐射会被放大。类似地,对于一个以恒定角速度旋转的轴对称导体(Zel’dovich圆柱体),光波的超辐射会发生,当物体的角速度比光的角相位速度更快时,光波会被放大。黑洞超辐射是类似的效应,其中圆柱体的旋转速度被黑洞视界的角速度所取代,电磁相互作用被引力所取代。

超辐射过程填充了所有满足超辐射条件的能级,最接近黑洞的超辐射能级增长最快。这个过程是自发发生的。对于玻色子,原子能级中的占据数将从零开始指数增长,形成一个围绕黑洞的宏观“云”,非常迅速地累积到黑洞质量的几个百分点!当黑洞的旋转速度降低到不再满足超辐射条件时,一个能级就会停止增长—此时,该能级可以填充超过10⁷⁵个粒子!

一个能级中的粒子数量翻倍所需的时间通常是黑洞光穿越时间的10⁷倍或更长;对于一个太阳质量的黑洞,这可以短至100秒。这与黑洞由于吸积等天体物理过程而发生变化的时间尺度相比非常快,后者需要大约10⁸年的顺序:如果满足超辐射条件,超辐射可以主导黑洞的演化。

黑洞超辐射的概念已经存在了几十年,但直到最近人们才开始考虑太阳质量黑洞的超辐射—这需要超出标准模型的新粒子,但可以有可观测的效果。

\subsection{我们如何看到黑洞“原子”?}

天体物理黑洞是检测QCD轴子或任何其他超轻玻色子的完美探测器,这些粒子与重力的相互作用几乎一样弱,而且在实验室中非常难以观测。

在我们的星系中可能存在多达十亿个太阳质量的黑洞。由于黑洞是由更大物体的坍塌或碰撞形成的,它们通常旋转得非常快:为了保持角动量,物体在坍塌时必须加速旋转。一旦形成黑洞,如果它旋转得足够快,并且存在具有适当质量的轻玻色子粒子,超辐射就会自动开始用这些粒子填充“原子”的能级。黑洞会旋转减速,直到不再满足超辐射条件。

黑洞通过这个过程失去相当一部分自旋的事实已经对非常轻的粒子设定了限制。

\begin{figure}[ht]
\centering
\includegraphics[width=6cm]{./figures/6a673bc0f82c4c86.png}
\caption{蓝色阴影区域(每个对应不同的角动量水平)在存在质量为10⁻¹¹ eV的QCD轴子时会受到超辐射的影响。y轴:黑洞的自旋,范围从零(非旋转)到1(极端,其中视界速度接近光速),x轴:以太阳质量为单位的黑洞质量。} \label{fig_GAtom_3}
\end{figure}

如果一个黑洞以蓝色区域内的自旋和质量被创造出来,它会在短短几年内迅速旋转减速——这与爱丁顿吸积时间1亿年(黑洞增长一倍的典型时间)相比非常快。点是恒星黑洞测量值,误差条告诉我们,这样的粒子质量不太可能存在(否则今天在蓝色阴影区域就不会有旧黑洞)。这提供了对这种轻粒子的首个限制,仅依赖于它们与黑洞的引力相互作用—即,仅它们的质量。

当“云”中有很多粒子时,会发生两个可能的可观测过程。就像在原子中一样,粒子可以从高能级向低能级跃迁,但与带电电子发射光子不同,这些粒子发射引力子。与电子不同,如果粒子是它们自己的反粒子,它们也可以在黑洞的引力场中成对湮灭成引力子。由于云中的占据数非常庞大,发射的引力子数量可以非常大,并且由于所有粒子都处于同一状态,引力辐射是相干的和单色的,这使其成为使用即将到来的引力波天文台(如激光干涉引力波天文台LIGO—由其两个站点的L形激光干涉仪组成,因此在标题中提到了“半钻石”)进行探测的完美候选者。虽然这些过程在天体物理时间尺度上很短—10年的跃迁和数千年的湮灭—但它们在人类时间尺度上非常长。高级LIGO可以在实验运行期间观察到同一个“原子”。

要了解我们可能观察到多少这样的原子,我们估计了我们宇宙中被粒子云包围的黑洞候选者的数量。对于给定的粒子质量,有一系列黑洞质量和自旋,这些信号可以从我们银河系的黑洞中被观测到。跃迁信号更少见,因为它们需要同时有两个能级被填充,并且持续时间更短。高级LIGO可以乐观地希望看到一个这样的跃迁事件。

湮灭信号更有希望—目标高级LIGO灵敏度下,可以观察到数十万次事件!

\begin{figure}[ht]
\centering
\includegraphics[width=6cm]{./figures/6f7784b3c23c8c81.png}
\caption{预期在aLIGO观察到的事件数量作为粒子质量的函数的估计。每个事件持续数千年或更长时间。垂直的灰色阴影区域目前由黑洞自旋测量所不赞成。蓝色带的宽度对应于我们对天体物理不确定性的估计,包括银河系中黑洞的质量和自旋分布。两个湮灭轴子的能量直接转换为引力波,所以信号的频率大约是轴子质量的两倍(在顶部的刻度上显示)。} \label{fig_GAtom_4}
\end{figure}


一旦观察到信号,就可以更详细地研究它——例如,信号功率的变化和小频率漂移可能有助于确定它来自引力原子。

计划中的低频探测器(AGIS,eLISA)将能够看到围绕星系中心超大质量黑洞的更轻粒子的引力原子的签名。

高级LIGO将在一年内上线,可能是第一个观测到引力波的实验。在黑洞超辐射的帮助下,LIGO也可能在这个过程中发现一种新粒子(也许是长期寻找的QCD轴子)!

那将是一个相当重大的发现。





