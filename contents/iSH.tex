% iSH 笔记

\pentry{Linux 基础\upref{Linux}}

\begin{issues}
\issueDraft
\end{issues}

iSH 是 iOS 设备上一个 shell 程序, 可以用 \verb|apk add| 安装各种常用包, 相当于 \verb|atp install|

\begin{itemize}
\item 兼容性还不错, Linux 大部分命令行工具都能使用,
\item 已成功安装的包: \verb|g++, gdb, git, make, vim, cmake, octave, py3-pip (numpy, sympy, venv)|。
\item 但性能很慢, 一个简单的 \verb|std::vector<int>| 的打印程序需要编译一分钟, \verb|git clone| 和 \verb|pip3 install numpy| 几乎卡住不动。 所以还是比较适合作为 ssh 客户端使用(可以设置 \verb|ssh-copy-id| 之类的, 很方便)。
\item \verb|sysbench| 跑分和我的台式机的 WSL1 对比: \verb|cpu: 30%|, \verb|mem: 5.6%|, \verb|io: 200%|。 可见瓶颈在内存速度。
\item apk 的包可以在\href{https://pkgs.alpinelinux.org/}{这里}查找。
\item 所有文件都可以在 iOS 的 files app 里面看到。
\end{itemize}
