% 排列
% keys 排列|集合|序列|阶乘
% license Usr
% type Tutor

\pentry{映射\nref{nod_map}, 阶乘\nref{nod_factor}}{nod_4292}

\subsection{定义}
\footnote{参考 Wikipedia \href{https://en.wikipedia.org/wiki/Permutation}{相关页面}。}我们讨论含有 $N$ 个元素的任意集合, 由于集合中元素的名称不重要, 我们以下将它记为 $\qty{1,2,\dots, N}$。 注意集合的是没有顺序的, 例如 $\qty{1,2,3}$ 和 $\qty{1,3,2}$ 是同一个集合。 当我们把集合 $S$ 中的的元素按照某种顺序排列成一个序列时, 就称为它是集合 $S$ 的一种\textbf{排列(permutation)}。 每个排列可以看作一个映射 $f:\qty{1,\dots,N}\to\qty{1,\dots,N}$。

一般地,从 $n$ 个不同元素中任取 $m$ ($m \leq n$)个元素,按照一定顺序排列成一列,叫做从 $n$ 个不同元素中取出 $m$ 个元素的\textbf{排列(arrangement)}

那么 $N$ 个元素的集合一共有几种不同的排列呢? 第 1 个位置有 $N$ 种不同的可能, 确定之后第 2 个位置有 $N-1$ 种不同的可能, 第 3 个位置有 $N-2$ 种…… 最后一个位置只有 1 种。 所以可能性的种数可以用\enref{阶乘}{factor}表示, 记为 $A_N$
\begin{equation}
A_N = N! = N(N-1)(N-2)\dots 1~.
\end{equation}

我们可以把第 $i$ 种排列记为 $p_i$, 该排列的元素按照顺序分别记为 $p_{i,1},\ p_{i,2},\ \dots, \ p_{i,N}$。

根据一个排列的定义,两个排列相同的含义为:组成排列的元素相同,并且元素的排列顺序也相同。


从 $n$ 个不同元素中取出 $m$ ($m \leq n$)个元素所有排列的个数,叫做从 $n$ 个不同元素中取出 $m$ 个元素的\textbf{排列数(number of arrangement)},用符号 $A_n^m$ 表示。

根据\textbf{分布乘法计数原理}可得排列数公式\begin{equation}\label{eq_HsPm_1}
A_n^m = n (n - 1)(n - 2) \cdots (n - m + 1)~.
\end{equation}
\addTODO{分布乘法计数原理是什么}

\subsection{变形}
我们对 \autoref{eq_HsPm_1} 进行变形,\begin{equation}\label{eq_HsPm_2}
\begin{aligned}
A_n^m &= \frac{n(n - 1)(n - 2) \cdots 2 \cdot 1}{(n - m)(n - m - 1) \cdots 2 \cdot 1}\\
&= \frac{n!}{(n - m)!}~.
\end{aligned}
\end{equation}

\autoref{eq_HsPm_2} 为排列数的另一种表达形式

\addTODO{例题,补充更多内容}
