% 单色光
% 光学|谱线|频率|波长|电磁波|光源

\pentry{平面波\upref{PWave}}

可见光是波长在 $400-760\mathrm{nm}$,亦即频率在 $4.3\e{14} \sim 7.5\e{14} \mathrm{Hz}$ 之间的电磁波。具有单一频率的光波称为单色光(monochromatic light)。然而,严格的单色光是不存在的。任何光源所发出的光波都有一定的频率(或波长)范围,在此范围内,各种频率(或波长)所对应的强度是不同的,以波长为横坐标,强度为纵坐标,可以直观地表示出这种强度与波长间的关系,称为\textbf{光谱曲线},或称\textbf{谱线(spectrum)},如\autoref{fig_MonoLi_1} 所示,谱线所对应的波长范围越窄,则称光的单色性越好。
\begin{figure}[ht]
\centering
\includegraphics[width=5cm]{./figures/ebcd9b3e3b3c85ce.png}
\caption{谱线及其宽度} \label{fig_MonoLi_1}
\end{figure}

设谱线中心处的波长为 $\lambda$,强度为 $I_0$,通常用最大强度的一半所包围的波长范围 $\Delta\lambda$ 当作\textbf{谱线宽度}(line width),它是标志谱线单色性好坏的物理量。普通单色光源,如钠光灯、镉灯、汞灯等,谱线宽度的数量级为 $0.1 \sim 10^{-3} \mathrm{nm}$,而激光的谱线宽度只有 $10^{-9}\mathrm{nm}$,甚至更小。
