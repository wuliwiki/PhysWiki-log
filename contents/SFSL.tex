% 索菲斯·李(综述)
% license CCBYSA3
% type Wiki

本文根据 CC-BY-SA 协议转载翻译自维基百科 \href{https://en.wikipedia.org/wiki/Sophus_Lie}{相关文章}。

\begin{figure}[ht]
\centering
\includegraphics[width=6cm]{./figures/a7728e1c8b169d85.png}
\caption{李在1896年} \label{fig_SFSL_1}
\end{figure}
马留斯·索福斯·李(Marius Sophus Lie,/liː/,挪威语:[liː];1842年12月17日-1899年2月18日)是一位挪威数学家。他在连续对称性理论方面做出了奠基性的贡献,并将其应用于几何和微分方程的研究中。他还在代数学的发展中做出了重要贡献。
\subsection{生平与职业生涯}
马留斯·索福斯·李于1842年12月17日出生在挪威小镇诺尔德菲尤尔。他是路德教牧师约翰·赫尔曼·李与其妻子所生的六个孩子中最小的一个,母亲出身于特隆赫姆的一个知名家族。\(^\text{[1]}\)

他在挪威东南部的莫斯接受了初等教育,之后在奥斯陆(当时称为克里斯蒂安尼亚)读高中。高中毕业后,他原本希望投身军事事业,但由于视力不佳而被军队拒绝,于是改而进入皇家弗雷德里克大学(即今日奥斯陆大学)就读。

索福斯·李的第一篇数学论文《平面几何中虚数的表示》于1869年由克里斯蒂安尼亚科学院与《克雷勒期刊》共同发表。同年,他获得了一项奖学金前往柏林,自9月起在当地停留至1870年2月。在那里,他结识了费利克斯·克莱因,两人迅速成为挚友。离开柏林后,李前往巴黎,克莱因也于两个月后与他会合。在巴黎,他们结识了卡米耶·若尔当与加斯顿·达布。然而,1870年7月19日,普法战争爆发,克莱因因身为普鲁士人而不得不迅速离开法国。李则前往枫丹白露,却被误认为是德国间谍而遭到逮捕,这在挪威为他带来了一定的名声。在达布的干预下,李被关押一个月后获释。\(^\text{[2]}\)

李在1871年于皇家弗雷德里克大学(今奥斯陆大学)获得博士学位,其博士论文题为《一类几何变换》(挪威语:Over en Classe geometriske Transformationer,英文:On a Class of Geometric Transformations)。\(^\text{[3]}\)这篇论文后来被达布称为“现代几何中最优美的发现之一”。次年,挪威议会为他特别设立了一个教授职位。同年,李拜访了正在埃尔朗根任教的克莱因,后者当时正在发展著名的“埃尔朗根纲领”。

1872年,李与彼得·路德维希·迈德尔·西洛一起合作,花了八个月时间编辑并出版了挪威同胞尼尔斯·亨里克·阿贝尔的数学著作。

1872年底,索福斯·李向当时年仅18岁的安娜·比奇求婚,并于1874年结婚。这对夫妇育有三个孩子:玛丽(Marie,生于1877年)、达格妮(Dagny,生于1880年)和赫尔曼(Herman,生于1884年)。

从1876年起,他与医生雅各布·沃姆-穆勒以及生物学家乔治·奥西安·萨尔斯共同编辑《数学与自然科学档案》期刊。

1884年,在克莱因和阿道夫·迈耶(当时二人均为莱比锡大学教授)的支持下,弗里德里希·恩格尔来到克里斯蒂安尼亚(今奥斯陆)协助李。恩格尔帮助李撰写了他最重要的著作《变换群理论》,该书分三卷于1888年至1893年间在莱比锡出版。数十年后,恩格尔也成为李全集的两位编辑之一。

1886年,李接替前往哥廷根任教的克莱因,成为莱比锡大学的教授。1889年11月,李突发精神崩溃,被迫住院治疗,直到1890年6月才出院。出院后他虽然重返岗位,但随着贫血症状的加重,他最终不得不返回故乡。1898年5月,他递交辞呈,并于当年9月离开莱比锡回国。次年,即1899年,李因恶性贫血去世,享年56岁。这种疾病是由于维生素B12吸收障碍所致。

李在其生前获得多项荣誉:1878年被授予伦敦数学学会名誉会员,1892年成为法国科学院通讯院士,1895年被选为英国皇家学会外籍会员,同年亦当选为美国国家科学院外籍院士。
\begin{figure}[ht]
\centering
\includegraphics[width=6cm]{./figures/a2c3f983d47e5724.png}
\caption{《变换群理论》第一卷,1888年版。} \label{fig_SFSL_2}
\end{figure}
\begin{figure}[ht]
\centering
\includegraphics[width=6cm]{./figures/2929c736d07ed25e.png}
\caption{《变换群理论》标题页} \label{fig_SFSL_3}
\end{figure}
\begin{figure}[ht]
\centering
\includegraphics[width=6cm]{./figures/fe4fa35fd71ead25.png}
\caption{《变换群理论》前言} \label{fig_SFSL_4}
\end{figure}
\subsection{遗产}
Lie 的主要工具,也是他最伟大的成就之一,是他发现连续变换群(后来以他的名字命名为 Lie 群)可以通过“线性化”来更好地理解,即研究对应的生成向量场(即所谓的无穷小生成元)。这些生成元服从群律的线性化形式,即现在所称的对易括号,从而构成了我们今天所说的 Lie 代数的结构。\(^\text{[4][5]}\)

Hermann Weyl 在 1922 年和 1923 年的论文中使用了 Lie 的群论研究成果,而 Lie 群今天在量子力学中发挥着重要作用。\(^\text{[5]}\)然而,今天对 Lie 群的研究与 Sophus Lie 当年的研究方向已有很大不同,正如有人所说:“在 19 世纪的数学大师中,Lie 的工作无疑是今天最不为人所熟知的。”\(^\text{[6]}\)

Sophus Lie 积极倡导设立阿贝尔奖。他受到以 Fridtjof Nansen 命名的南森基金启发,并注意到诺贝尔奖未设立数学奖项,因此他致力于推动设立一个表彰纯数学杰出工作的奖项。\(^\text{[7]}\)

Lie 指导了许多后来成为著名数学家的博士生。Élie Cartan 被广泛认为是 20 世纪最伟大的数学家之一。Kazimierz Żorawski 的工作在多个领域被证明具有重要意义。Hans Frederick Blichfeldt 对多个数学领域做出了贡献。
\subsection{书籍}
\begin{itemize}
\item Lie, Sophus(1888年),《变换群理论 I》(德语),莱比锡:B. G. Teubner 出版社。由 Friedrich Engel 协助撰写。有英文翻译版:Joël Merker 编辑并从德语翻译并撰写前言,详见 ISBN 978-3-662-46210-2 及 arXiv:1003.3202。
\item Lie, Sophus(1890年),《变换群理论 II》(德语),莱比锡:B. G. Teubner 出版社。由 Friedrich Engel 协助撰写。
\item Lie, Sophus(1891年),《已知无穷小变换的微分方程讲义》(德语),莱比锡:B. G. Teubner 出版社。由 Georg Scheffers 协助撰写。\(^\text{[8]}\)
\item Lie, Sophus(1893年),《连续群讲义》(德语),莱比锡:B. G. Teubner 出版社。由 Georg Scheffers 协助撰写。\(^\text{[9]}\)
\item Lie, Sophus(1893年),《变换群理论 III》(德语),莱比锡:B. G. Teubner 出版社。由 Friedrich Engel 协助撰写。
\item Lie, Sophus(1896年),《接触变换几何学》(德语),莱比锡:B. G. Teubner 出版社。由 Georg Scheffers 协助撰写。\(^\text{[10]}\)
\item Lie, Sophus;Engel, Friedrich;Heegaard, Poul(编辑),《Lie 论文集》(德语),莱比锡:Teubner 出版社;共7卷,出版时间为1922–1960年。\(^\text{[11][12]}\)
\end{itemize}
\subsection{另见}
\begin{itemize}
\item Lie 导数
\item 单纯 Lie 群列表
\item 以 Sophus Lie 命名的事物列表
\end{itemize}
\subsection{注释翻译}
\begin{enumerate}
\item James, Ioan(2002年)。《杰出的数学家》。剑桥大学出版社,第201页。ISBN 978-0-521-52094-2。
\item Darboux, Gaston(1899年)。“Sophus Lie”。《美国数学会通报》5(7): 367–370。doi:10.1090/s0002-9904-1899-00628-1。
\item Lie, Sophus(1871年)。《论一类几何变换》(博士论文)。克里斯蒂安尼亚大学(今奥斯陆大学)。
\item Helgason, Sigurdur(1994年),“数学家 Sophus Lie”(PDF),载于《Sophus Lie 纪念会议论文集》,1992年8月,奥斯陆:斯堪的纳维亚大学出版社,第3–21页。
\item Gale, Thomson。《Marius Sophus Lie 传记》,《数学世界》。2009年1月23日访问。
\item Hermann, Robert(编)(1975年)。《Sophus Lie 1880 年变换群论文》,《Lie 群:历史、前沿与应用》,第1卷,Math Sci Press,第iii页。ISBN 0-915692-10-4。
\item “阿贝尔奖的历史”。[www.abelprize.no。2018年3月16日存档。2021年2月4日访问。](http://www.abelprize.no。2018年3月16日存档。2021年2月4日访问。)
\item Lovett, E. O.(1898年)。“书评:已知无穷小变换的微分方程讲义”。《美国数学会通报》4(4): 155–167。doi:10.1090/s0002-9904-1898-00476-7。
\item Brooks, J. M.(1895年)。“书评:带有几何和其他应用的连续群讲义”。《美国数学会通报》1(10): 241–248。doi:10.1090/s0002-9904-1895-00283-9。
\item Lovett, E. O.(1897年)。“书评:接触变换几何学”。《美国数学会通报》3(9): 321–350。doi:10.1090/s0002-9904-1897-00430-x。
\item Schilling, O. F. G.(1939年)。“书评:Sophus Lie 的论文集。几何论文,第 I、II 卷”。《美国数学会通报》45(7): 513–514。doi:10.1090/S0002-9904-1939-07032-8。ISSN 0002-9904。
\item Carmichael, R. D.(1930年)。“书评:Sophus Lie 论文集(挪威文版《Samlede Avhandlinger》,由 Aschehoug 出版)第 IV 卷”。《美国数学会通报》36(5): 337–338。doi:10.1090/S0002-9904-1930-04950-2。ISSN 0002-9904。(附:1923年第III卷、1925年第V卷、1928年第VI卷的书评链接)
\end{enumerate}
\subsection{参考文献}
\begin{itemize}
\item Fritzsche, Bernd(1999),《Sophus Lie:其生平与工作的概述》,发表于《Lie理论杂志》,第9卷,第1期,第1–38页,ISSN 0949-5932,MR 1680023,Zbl 0927.01029,检索日期:2010年12月2日。
\item Freudenthal, Hans(1970–1980),《Lie, Marius Sophus》,收录于《科学传记词典》,Charles Scribner’s Sons 出版。
\item Stubhaug, Arild(2002),《数学家 Sophus Lie:是我思想的大胆造就了我》,施普林格出版社,ISBN 3-540-42137-8。
\item Yaglom, Isaak Moiseevich(1988),Grant, Hardy;Shenitzer, Abe(编),《Felix Klein 与 Sophus Lie:19 世纪对对称性思想的演化》,Birkhäuser出版社,ISBN 3-7643-3316-2。
\end{itemize}
\subsection{外部链接}
\begin{itemize}
\item Chisholm, Hugh(编)(1911),“Lie, Marius Sophus”,收录于《大英百科全书》(第11版),剑桥大学出版社。
\item O'Connor, John J.;Robertson, Edmund F.(2000年2月),“Sophus Lie”,收录于圣安德鲁斯大学的《MacTutor 数学史档案》。
\item [Sophus Lie 的作品在 Project Gutenberg 上](https://www.gutenberg.org/author/Sophus+Lie)
\item [关于 Sophus Lie 的作品或资料在 Internet Archive 上](https://archive.org/search.php?query=creator%3A%22Sophus+Lie%22)
\item 《无限连续变换群理论的基础 – 第一部分》:Lie 一篇重要论文的英文翻译(第一部分)
\item 《无限连续变换群理论的基础 – 第二部分》:Lie 一篇重要论文的英文翻译(第二部分)
\item 《关于复合体——特别是直线与球面复合体——在偏微分方程理论中的应用》:Lie 一篇重要论文的英文翻译
\item 《接触变换不变量理论的基础》:Lie 一篇重要论文的英文翻译
\item 《力学中的无穷小接触变换》:Lie 一篇重要论文的英文翻译
\item U. Amaldi,《自 Sophus Lie 逝世以来连续群理论所取得的主要成果(1898–1907)》:一篇关于 Lie 逝世后发展成果的英文综述论文。
\end{itemize}
