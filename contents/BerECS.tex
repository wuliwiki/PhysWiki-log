% Berkeley-ECS 方法
% license Xiao
% type Tutor

\begin{issues}
\issueDraft
\end{issues}

\pentry{Lippmann-Schwinger 方程\upref{LipSch}}

\subsection{推导}
\footnote{参考 Renate Pazourek Thesis Eq 3.36}要把最后的波函数 $\ket{\Psi(t_0)}$ 投影到精确散射态 $\ket{\varphi_{\alpha,E}}$ 上, 而避免计算精确散射态, 可以先解
\begin{equation}\label{eq_BerECS_1}
(E-H)\ket{\Psi_{SC}(E)} = \ket{\Psi(t_0)}~.
\end{equation}
类比\autoref{eq_LipSch_2}~\upref{LipSch}, 形式上这相当于计算
\begin{equation}
\ket{\Psi_{SC}(E)} = G^+(E)\ket{\Psi(t_0)}~.
\end{equation}
但这并不重要, 数值上仍然直接解非齐次线性方程组\autoref{eq_BerECS_1} 即可。

下一步, 把末态投影到精确散射态上得
\begin{equation}\label{eq_BerECS_2}
\begin{aligned}
&\quad \braket{\varphi_{\alpha,E}}{\Psi(t_0)} = \mel{\varphi_{\alpha,E}}{E-H}{\Psi_{SC}(E)}\\
&= \mel*{\varphi_{\alpha,E}}{\frac{\laplacian}{2}}{\Psi_{SC}(E)} + \mel*{\varphi_{\alpha,E}}{E-V}{\Psi_{SC}(E)}~.
\end{aligned}
\end{equation}
由于 $(-\laplacian/2+V)\varphi_{\alpha,E} = E\varphi_{\alpha,E}$
\begin{equation}
\mel*{\varphi_{\alpha,E}}{E-V}{\Psi_{SC}(E)} = -\braket*{\frac{\laplacian}{2} \varphi_{\alpha,E}}{\Psi_{SC}(E)}~.
\end{equation}


根据格林第二恒等式\footnote{推导: $\div(u\grad v - v\grad u) = u\laplacian v - v\laplacian u$, 再使用高斯定理}
\begin{equation}
\int (u\laplacian v - v\laplacian u)\dd{V} = \oint (u\grad v - v\grad u) \dd{\bvec S}~.
\end{equation}
可以把\autoref{eq_BerECS_2} 写成面积分(注意 $\varphi_{\alpha,E}$ 是一个实函数)
\begin{equation}\label{eq_BerECS_3}
\braket{\varphi_{\alpha,E}}{\Psi(t_0)} = \frac{1}{2} \oint [\varphi_{\alpha,E} \grad \Psi_{SC} - \Psi_{SC} \grad \varphi_{\alpha,E}] \dd{\bvec S}~.
\end{equation}
注意这里的矢量都是 6 维的。

另外由于这个面积分距离原子核很远, 可以用 $\mathrm{He}^+$ 束缚态和库仑平面波的对称化乘积来代替。 注意 $\ket{n_1,l_1,m_1,l_2,k_2}$ 和 $\ket{n_1,l_1,m_1,\bvec k_2}$ 在无穷远处都是没有额外相位的。

\subsection{高维曲面积分}
6 维高斯定理为
\begin{equation}
\oint \bvec A\vdot \dd{\bvec S} = \int \div \bvec A \dd{V}~.
\end{equation}
令六维矢量场 $\bvec A$ 为
\begin{equation}
\bvec A = (f_{r1} \uvec r_1 + f_{\theta1}\uvec \theta_1 + f_{\phi1}\uvec \phi_1, \quad f_{r2} \uvec r_2 + f_{\theta2}\uvec \theta_2 + f_{\phi2}\uvec \phi_2)~,
\end{equation}
其中 $f$ 都是 $r_1,\theta_1,\phi_1,r_2,\theta_2,\phi_2$ 的函数。

右边的体积分的区域为两个球体 $r_1<R_1$ 和 $r_2<R_2$ 的张量积。 那么左边的面积分为
\begin{equation}
\begin{aligned}
\oint \bvec A\vdot \dd{\bvec S} =&\int f_{r1}\cdot  R_1^2\sin\theta_1\dd{\theta_1}\dd{\phi_1}\cdot r_2^2\sin\theta_2 \dd{r_2}\dd{\theta_2}\dd{\phi_2}\\
&+\int f_{r2}\cdot  R_2^2\sin\theta_2\dd{\theta_2}\dd{\phi_2}\cdot r_1^2\sin\theta_1 \dd{r_1}\dd{\theta_1}\dd{\phi_1}~.
\end{aligned}
\end{equation}


令两个径向波函数分别为 $\varphi_{l_1,l_2}^{L,M}$ 和 $\psi_{l_1,l_2}^{L,M}$, 那么\autoref{eq_BerECS_3} 中的第一项的积分为
\begin{equation}
\begin{aligned}
\oint \varphi_{\alpha,E} \grad \Psi_{SC} \dd{\bvec S} &= \sum_{l_1,l_2,L,M}\int_0^{R_2} \varphi_{l_1,l_2}^{L,M}(R_1,r_2) \pdv{r_1} \psi_{l_1,l_2}^{L,M}(R_1,r_2)\dd{r_2}\\
&+\sum_{l_1,l_2,L,M}\int_0^{R_1} \varphi_{l_1,l_2}^{L,M}(r_1,R_2) \pdv{r_2} \psi_{l_1,l_2}^{L,M}(r_1,R_2)\dd{r_1}~.
\end{aligned}
\end{equation}
假设 $\varphi_{l_1,l_2}^{L,M}(r_1,r_2)$ 中 $r_1$ 是束缚态, 那么上面第一个求和为零。 第二个求和中, $r_1 \ll R_2$, 所以 $\varphi_{l_1,l_2}^{L,M}(r_1,R_2)$ 可以用乘积来替代。 \autoref{eq_BerECS_3} 中第二项的推导类似。
