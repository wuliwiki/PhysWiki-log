% 量子纠缠 2
% keys 纠缠|entanglement|施密特秩
% license Xiao
% type Tutor

\pentry{约化密度矩阵\nref{nod_partra}}{nod_31fd}

我们研究一个二元纯态 $\ket{\psi}_{AB}$ 的子系统 $A$,假设整个大的孤立系统的 Hilbert 空间可以表示为两个子 Hilbert 空间的张量积:$\mathcal{H}_A\otimes \mathcal{H}_B$,其中 $\mathcal A$ 是待研究的子系统的 Hilbert 空间,设约化密度算符为 $\rho_A=\text{tr}_B \ket{\psi}\bra{\psi}$。在\enref{约化密度矩阵}{partra}文章中,我们证明了 $A,B$ 处于纠缠态的一个判据 \autoref{eq_partra_2}。

\begin{equation}
\text{tr} \rho^2 <1~.
\end{equation}

它意味着约化密度算符在某个正交完备基下对角化以后,表现为 $\mathcal{H}_A$ 中若干个纯态(大于一个)组成的系综,系综中每个纯态有 $p_a<1$ 的概率出现。这也意味着\textbf{当且仅当} $\rho_A$ 的\textbf{施密特秩}\footnote{类似于\enref{矩阵的秩}{MatRnk},可以将这一概念推广到任意线性算符。}大于 $1$, $A,B$ 处于纠缠态。在此处我们考察的约化密度算符是正定算符,因此施密特秩等于正的本征值的个数。若本征值个数 $>1$,体系处于纠缠态,我们称 $A,B$ 之间具有\textbf{量子相关性}。

如果施密特秩为 $1$,那么约化密度算符可以表示为 ${}_A\ket{\varphi}\bra{\varphi}_A,\varphi_A\in \mathcal{H}$,此时 $A$ 与 $B$ 之间是不纠缠的,或者被称为\textbf{可分的}(seperable)。此时二元纯态 $\ket{\psi}$ 实际上可以表示为两个子系统的量子态的张量积:
\begin{equation}
\ket{\psi}_{AB}=\ket{\varphi}_A\otimes \ket{\varphi}_B~.
\end{equation}

为了量化纠缠的程度,我们引入纠缠熵(entanglement entropy)。纠缠熵是通过约化密度算符的谱(即其特征值)来定义的。通常使用冯·诺依曼熵(von Neumann entropy)来表示纠缠熵。如\autoref{def_vonNE_1}:
\begin{equation}
S(\rho_A) = - \text{tr}(\rho_A \log \rho_A)~,
\end{equation}
其中 $\rho_A$ 是子系统 $A$ 的约化密度算符。

冯·诺依曼熵具有以下性质:
\begin{enumerate}
    \item \textbf{非负性}:$S(\rho_A) \geq 0$。
    \item \textbf{纯态的熵为零}:如果 $\rho_A$ 对应一个纯态,则 $S(\rho_A) = 0$。
    \item \textbf{最大熵}:对于最大混合态,$S(\rho_A)$ 达到最大值。
\end{enumerate}
对于一个二元纯态 $\ket{\psi}_{AB}$,其约化密度矩阵 $\rho_A$ 和 $\rho_B$ 具有相同的谱(特征值),因此 $S(\rho_A) = S(\rho_B)$。纠缠熵可以被解释为子系统 $A$ 和 $B$ 之间量子纠缠的度量。当 $S(\rho_A) = 0$ 时,表示两个子系统是可分的;当 $S(\rho_A)$ 较大时,表示两个子系统之间具有较强的量子纠缠。

总结来说,纠缠熵提供了一种定量描述系统内部量子纠缠程度的方法,它在量子信息理论和量子计算中有广泛的应用。通过计算子系统的冯·诺依曼熵,我们可以确定并量化量子系统中的纠缠程度。这是理解量子系统内部复杂关联的重要工具。我们经常用这个来看模型处于哪一个相,并以此作为一个序参量来判定相变。但是注意不是所有相变都可以用冯·诺依曼熵当序参量,需要参考相变的类型。
