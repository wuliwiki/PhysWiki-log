% Euler-Mascheroni 常数
% keys 定积分|收敛|调和级数
% license Xiao
% type Tutor

\begin{issues}
\issueDraft
\end{issues}

\pentry{定积分\nref{nod_DefInt}}{nod_997c}

\begin{definition}{Euler-Mascheroni常数}
\begin{equation}\label{eq_Masche_1}
\gamma = \lim_{n\to\infty} \qty(\sum_{m=1}^n \frac{1}{m} - \ln n) = 0.5772156619\dots~
\end{equation}
\end{definition}


其中可以把 $1/m$ 看成区间 $[m, m+1]$ 内高为 $1/m$ 矩形的面积, 而 $\ln n$ 是函数 $1/x$ 在区间 $[1,n]$ 的定积分(函数曲线下方的面积), 如\autoref{fig_Masche_1}。

\begin{figure}[ht]
\centering
\includegraphics[width=6cm]{./figures/92811bb90d24b2bb.pdf}
\caption{Euler-Mascheroni常数} \label{fig_Masche_1}
\end{figure}

\autoref{eq_Masche_1} 的收敛也可以用于证明调和级数 $\sum_\limits{n=1}^\infty \frac{1}{n}$ 不收敛: 因为极限 $\ln n(n\to\infty)$ 不收敛。

下证$\lim_\limits{n\to\infty}(\sum_\limits{k=1}^n \frac{1}{k}- lnk)$收敛:\\
由\enref{Euler-Maclaurin 求和公式}{EMSum},
$$\sum_\limits{k=1}^n \frac{1}{k}=
\int_1^n \frac{1}{x}dx+\frac{1+\frac{1}{n}}{2}-\int_1^n \psi(x) \frac{1}{x^2}dx~$$\\
显然$\int_1^n \psi(x) \frac{1}{x^2}dx$收敛,得$\sum_\limits{k=1}^n \frac{1}{k}=lnn=\gamma$\\