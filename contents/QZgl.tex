% 乔治·格林(综述)
% license CCBYSA3
% type Wiki

本文根据 CC-BY-SA 协议转载翻译自维基百科\href{https://en.wikipedia.org/wiki/George_Green_(mathematician)}{相关文章}。

乔治·格林(George Green,1793年7月14日-1841年5月31日)是一位英国数学物理学家,他于1828年撰写了《数学分析在电学与磁学理论中的应用论文》\(^\text{[2][3]}\)。这篇论文引入了多个重要概念,包括一个类似于现代“格林公式”的定理、如今在物理学中广泛使用的势函数思想,以及现在被称为“格林函数”的概念。格林是第一个构建电学与磁学数学理论的人,他的理论为詹姆斯·克拉克·麦克斯韦、威廉·汤姆逊(开尔文勋爵)等科学家的研究奠定了基础。他在势理论方面的工作与卡尔·弗里德里希·高斯的研究是并行发展的。

格林的人生经历极为非凡,因为他几乎完全是自学成才。他在童年时期仅接受过大约一年的正规教育,年龄在8岁到9岁之间。
\subsection{早年生活}
\begin{figure}[ht]
\centering
\includegraphics[width=6cm]{./figures/e426677496c6ac27.png}
\caption{斯尼顿的格林风车,由格林的父亲所拥有。该风车于1986年翻修,如今是一座科学中心。} \label{fig_QZgl_1}
\end{figure}
格林出生并大部分时间生活在英格兰诺丁汉郡的斯尼顿镇,如今该地已划入诺丁汉市。他的父亲也叫乔治,是一位面包师,还建造并拥有一座砖结构的风车,用于碾磨谷物\(^\text{[1]}\)。

在年轻时,格林被形容为体质虚弱,不喜欢在父亲的面包房里干活。然而,他并没有选择的余地;就像那个时代许多孩子一样,他很可能从五岁起便开始每天劳作谋生。
\subsubsection{罗伯特·古达克学院}
在那个时代,诺丁汉只有大约25\%至50\%的儿童能接受任何形式的教育【需要引用】。大多数学校是由教会运营的主日学校,孩子们通常只会在其中学习一到两年。格林的父亲由于面包店生意成功,经济状况良好,注意到年幼的格林智力超群,于是在1801年3月将他送入位于上议会街的罗伯特·古达克学院就读。罗伯特·古达克是当时著名的科普作家和教育家,他出版了《论青少年教育》一书,其中写道,他关注的不是“男孩的利益,而是未来成年人的塑造”。对外行人来说,他似乎在科学与数学方面知识渊博,但细读他的论文和教学大纲便会发现,他的数学教学范围仅限于代数、三角和对数。因此,格林日后在数学领域所作出的那些展现出极其先进数学知识的贡献,不可能是在古达克学院期间获得的。他仅在该校待了四个学期(相当于一个学年),他的同时代人曾猜测,他之所以离开,是因为学院已无可教之物。
\subsubsection{从诺丁汉搬到斯尼顿}
1773年,格林的父亲搬到了诺丁汉,当时该地以环境宜人、道路宽敞著称。然而到了1831年,受工业革命初期的影响,城市人口已增长近五倍,诺丁汉成了英格兰最糟糕的贫民区之一。饥饿的工人经常爆发骚乱,面包师和磨坊主则因被怀疑囤粮抬高物价而成为众矢之的。

基于这些原因,1807年,乔治·格林的父亲在斯尼顿购置了一块土地,并在那里建造了一座“砖结构风力磨坊”,如今被称为“格林风车”。这座风车在当时的技术水平上相当先进,但却需要几乎全天候的维护,而这成为了格林此后二十年的负担。
\subsection{成年生活}
\subsubsection{磨坊主}
与面包烘焙一样,格林也觉得操控风车的工作烦人而单调。农田的谷物源源不断地运到磨坊门口,而风车的帆必须不断根据风速调整,以防强风损毁,同时在微风中尽可能提高转速。用于碾磨的磨石会持续互相摩擦,若缺粮则可能磨损过快甚至引发火灾。每月,重达一吨的磨石还需更换或修复一次。
\subsubsection{家庭生活}
1823年,格林与简·史密斯建立了关系。简是威廉·史密斯的女儿,后者被格林的父亲聘为磨坊经理。尽管格林与简·史密斯从未正式结婚,但简最终被称为简·格林,两人共同育有七个孩子,除第一个孩子外,其余在受洗时均使用了“格林”作为姓氏。最小的孩子出生于格林去世前13个月。格林在遗嘱中为其事实婚姻妻子和孩子们留有遗产\(^\text{[4]}\)。
\subsubsection{诺丁汉订阅图书馆}
格林三十岁时,成为了诺丁汉订阅图书馆的一员。该图书馆至今仍然存在,很可能是格林获得高等数学知识的主要来源。与传统图书馆不同,订阅图书馆仅对大约一百名订阅者开放,订阅者名单上的首位便是纽卡斯尔公爵。该图书馆根据订阅者的具体兴趣,提供他们所需的专业书籍与期刊。
\subsubsection{1828年论文}
\begin{figure}[ht]
\centering
\includegraphics[width=6cm]{./figures/659dbb6b00e6f629.png}
\caption{格林原始论文的标题页,内容涉及后来被称为“格林公式”的理论。} \label{fig_QZgl_2}
\end{figure}
1828年,格林发表了《数学分析在电与磁理论中的应用论文》,这是他如今最为人所知的一篇论文。该文由作者自费私下出版,因为他认为自己这样一个没有正式数学教育背景的人,若将论文投至正规期刊会显得自不量力。当格林出版这篇论文时,它是以认购的方式售出,共售出给51人,其中大多数是他的朋友,而这些朋友很可能根本无法理解这篇论文的内容。

一位富有的地主兼数学家爱德华·布罗姆黑德购得一册后,鼓励格林继续从事数学研究。但格林起初并不相信这份邀请是真诚的,因此两年都没有与布罗姆黑德联系。
\subsection{数学家生涯}
到1829年格林的父亲去世时,这位老格林因其积累的财富和所拥有的大量土地,已跻身乡绅阶层。他将大约一半财产留给了儿子,另一半则留给了女儿。年仅三十六岁的乔治·格林因此得以利用这笔财富,摆脱磨坊工的职责,专心从事数学研究。
\subsubsection{剑桥大学}
诺丁汉订阅图书馆的一些会员知道格林的才华后,多次坚持劝他接受正规的大学教育。尤其是该图书馆最著名的订阅者之一爱德华·布罗姆黑德爵士,他与格林有过多次通信,并坚决主张格林进入剑桥大学深造。

1832年,年近四十的格林被剑桥大学冈维尔与凯斯学院录取为本科生\(^\text{[5]}\)。他对自己缺乏希腊语和拉丁语的知识非常不安,因为这是入学的前提条件。但后来他发现这些语言其实并不像他想象的那样难掌握,因为学校对语言掌握的要求并不高。他在数学考试中表现出色,获得了一年级数学奖。在1838年,他以第四名优等生(Fourth Wrangler,毕业班中数学成绩第四名)的成绩获得学士学位,仅次于获得第二名的詹姆斯·约瑟夫·西尔维斯特\(^\text{[5]}\)。
\subsubsection{学院研究员}
毕业后,格林被选为剑桥哲学学会成员。即使没有他在学业上的优异表现,该学会此前也已关注并审阅了他那篇《论文》及其另外三篇发表作品,因此他受到了热情欢迎。

接下来的两年里,格林拥有了前所未有的机会来阅读、写作并讨论自己的科学思想。在这段短暂但高产的时期,他又发表了六篇论文,内容涉及流体力学、声学与光学的应用。
\subsection{晚年与身后声誉}
\begin{figure}[ht]
\centering
\includegraphics[width=6cm]{./figures/3aa026e02c7397f1.png}
\caption{《已故乔治·格林数学论文集》1871年版本的扉页。} \label{fig_QZgl_3}
\end{figure}
在剑桥的最后几年,格林身体状况每况愈下,于1840年返回斯尼顿,并于一年后去世。有传言称,格林在剑桥时“沉溺于酒精”,一些早期支持者,如爱德华·布罗姆黑德爵士,试图与他保持距离。
\begin{figure}[ht]
\centering
\includegraphics[width=6cm]{./figures/1d46bf250195fe9f.png}
\caption{格林的墓地位于教堂的庭院内,离他的风车不远。} \label{fig_QZgl_4}
\end{figure}
格林的研究在他生前并未在数学界广为人知。除格林本人之外,第一个引用他1828年论文的数学家是英国人罗伯特·墨菲(Robert Murphy,1806–1843),他在1833年的著作中提到该论文\(^\text{[6]}\)。1845年,即格林去世四年后,年轻的威廉·汤姆森(当时年仅21岁,后被封为开尔文勋爵)重新发现了格林的工作,并使其在后来广泛传播,造福后来的数学家。据D.M. 卡内尔所著《乔治·格林》一书所述,汤姆森注意到墨菲对格林1828年论文的引用,但最初难以找到论文原件;直到1845年,他才从威廉·霍普金斯那里获得了一些格林1828年论文的副本。

1871年,诺曼·麦克劳德·费勒斯将《已故乔治·格林的数学论文》整理出版\(^\text{[7]}\)。
\begin{figure}[ht]
\centering
\includegraphics[width=6cm]{./figures/82cc90cd09d67773.png}
\caption{数学家乔治·格林的父母——乔治·格林与凯瑟琳·格林的墓碑。} \label{fig_QZgl_6}
\end{figure}
格林关于运河中波动运动的研究(即“格林定律”)预示了量子力学中的 WKB 近似;他关于光波与以太性质的研究则催生了今天所称的“柯西–格林张量”。格林定理与格林函数曾是经典力学中的重要工具,后来在朱利安·施温格于 1948 年的电动力学研究中得到了重大发展,并促成其与理查德·费曼及朝永振一郎共同获得 1965 年诺贝尔物理学奖。格林函数后来在超导性分析中也发挥了关键作用。1930 年,阿尔伯特·爱因斯坦在访问诺丁汉时曾评论道,格林“领先时代二十年”。理论物理学家朱利安·施温格在其突破性工作中广泛使用格林函数,并于 1993 年发表纪念文章《量子场论的绿化:我与乔治》\(^\text{[8]}\)。

诺丁汉大学的“乔治·格林图书馆”以他命名,收藏了该校绝大多数科学与工程类藏书。该校工程学院的“乔治·格林电磁研究所”研究团队也以他命名\(^\text{[9]}\)。1986 年,位于诺丁汉斯尼顿的格林风车被修复恢复运转,如今既是一个19世纪风车的工作样本,又是一个纪念格林的博物馆与科学中心。

在威斯敏斯特教堂中,格林的纪念石碑位于艾萨克·牛顿与开尔文勋爵墓地旁的中殿\(^\text{[10]}\)。

他的研究及其对19世纪应用物理学的影响一度被遗忘,直到玛丽·卡内尔于1993年出版了他的传记才重新引起关注。
\begin{figure}[ht]
\centering
\includegraphics[width=6cm]{./figures/890bf984cc0144de.png}
\caption{数学家乔治·格林的墓碑位于圣斯蒂芬公墓中,位置比他父母的墓碑更靠近东侧围墙。} \label{fig_QZgl_5}
\end{figure}
\subsection{知识来源}
近期的历史研究表明,[11] 格林数学教育中的关键人物可能是约翰·托普利斯(John Toplis,约1774–1857)。他曾在剑桥大学数学专业毕业,成绩为第十一名 Wrangler,后于1806–1819 年间担任诺丁汉高级中学前身的校长,并与格林及其家人生活在同一个社区。托普利斯是大陆数学学派的倡导者,精通法语,并曾翻译过皮埃尔-西蒙·拉普拉斯关于天体力学的著作。

托普利斯可能在格林的数学教育中扮演了重要角色,这一观点可以解答长期以来关于格林数学知识来源的诸多疑问。例如,格林在其研究中使用了“数学分析法”(the Mathematical Analysis),这是一种源自莱布尼茨的微积分形式,在当时的英格兰几乎无人知晓,甚至遭到排斥(因为莱布尼茨与牛顿为同时代人,而英国更推崇牛顿自己的方法)。这种微积分形式,以及法国数学家如拉普拉斯、拉克鲁瓦和泊松的发展成果,在当时甚至连剑桥都未教授,更不用说诺丁汉了。然而,格林不仅熟知这些成果,还加以改进。[12]
\subsection{出版作品列表}
\begin{itemize}
\item 《数学分析在电与磁理论中的应用》(,乔治·格林著,诺丁汉:T. Wheelhouse 私人印刷,1828年。(四开本,页码:vii + 72)
\item Green, George (1835). “关于类比电流体的流体平衡定律的数学研究,以及其他类似研究”。《剑桥哲学会会刊》,第五卷第一部分,页 1–63。1832年11月12日提交。
\item Green, George (1835). “关于具有可变密度椭球体的内外引力的确定”。《剑桥哲学会会刊》,第五卷第三部分,页 395–429。1833年5月6日提交。
\item Green, George (1836). “关于流体介质中钟摆振动的研究”。《爱丁堡皇家学会会刊》,第十三卷第一期,页 54–62。DOI:10.1017/S0080456800022183。1833年12月16日提交。
\item Green, George (1838). “关于声波的反射与折射”。《剑桥哲学会会刊》,第六卷第三部分,页 403–413。1837年12月11日提交。
\item Green, George (1838). “关于浅窄可变运河中波浪运动的研究”。《剑桥哲学会会刊》,第六卷第三部分,页 457–462。1837年5月15日提交。
\item Green, George (1842). “关于两种非晶介质公共界面处光的反射与折射规律”。《剑桥哲学会会刊》,第七卷第一部分,页 1–24。1837年12月11日提交。
\item Green, George (1842). “关于运河中波浪运动的简要说明”。《剑桥哲学会会刊》,第七卷第一部分,页 87–95。1839年2月18日提交。
\item Green, George (1842). “关于光的反射与折射研究论文的补遗”。《剑桥哲学会会刊》,第七卷第一部分,页 113–120。1839年5月6日提交。
\item Green, George (1842). “关于光在晶体介质中传播的研究”。《剑桥哲学会会刊》,第七卷第二部分,页 121–140。1839年5月20日提交。
\end{itemize}
\subsection{注释}
\begin{enumerate}
\item O'Connor, John J.; Robertson, Edmund F.,“乔治·格林(数学家)”,圣安德鲁斯大学 MacTutor 数学史档案。
\item 这篇 1828 年的论文可见于《已故乔治·格林的数学论文集》,由 N. M. Ferrers 编辑,相关网站链接见下方。
\item Cannell, D.M.(1999)。“乔治·格林:一位神秘的数学家”,《美国数学月刊》,106 (2): 136–151。doi:10.2307/2589050。JSTOR: 2589050。
\item Cannel, D. M.; Lord, N. J.; Lord, N. J(1993)。“乔治·格林,数学家与物理学家(1793–1841)”,《数学公报》,77 (478): 26–51。doi:10.2307/3619259。JSTOR: 3619259。S2CID: 238490315。
\item “Green, George (GRN832G)”,剑桥大学校友数据库。
\item Murphy, R.(1833)。“关于定积分逆方法及其物理应用”,《剑桥哲学会会刊》,第4卷,第353–408页。格林在第357页脚注中被提及。
\item N. M. Ferrers 编辑(1871),《已故乔治·格林的数学论文集》,麦克米伦出版社,密歇根大学历史数学藏书链接。
\item  Schwinger, Julian(1996年1月)。“量子场论的绿化:乔治与我”,收录于 Ng, Yee Jack 编辑的《Julian Schwinger: The Physicist, the Teacher, and the Man》,世界科技出版社,第13–27页。arXiv\:hep-ph/9310283。doi:10.1142/9789812830449\_0003。ISBN: 9789812830449。重印于《George Green: Mathematician and Physicist 1793–1841: The Background to his Life and Work》(SIAM, 2001),第220–231页,doi:10.1137/1.9780898718102.appvia。
\item “乔治·格林电磁学研究所”。存档于2014年1月17日。访问时间:2014年2月17日。
\item  乔治·格林,来自《威斯敏斯特教堂》。
\item Harding, R., Harding, M. “违禁数学:对1823–1828年乔治·格林在诺丁汉图书馆可用资源的文献回顾”,《数学通识》41, 44–55(2019),[https://doi.org/10.1007/s00283-018-09871-7。](https://doi.org/10.1007/s00283-018-09871-7。)
\item Cannell, D.M.(1999)。“乔治·格林:一位神秘的数学家”,《美国数学月刊》,106 (2): 137, 140。CiteSeerX: 10.1.1.383.6824。doi:10.1080/00029890.1999.12005020。
\end{enumerate}
\subsection{参考文献}
\begin{itemize}
\item Ivor Grattan-Guinness,《乔治·格林(1793–1841)》,《牛津国家人物传记词典》,牛津大学出版社,2004年,2009年5月26日访问。
\item D. M. Cannell,《乔治·格林:数学家与物理学家 1793–1841》,伦敦:亚特隆出版社(The Athlone Press),1993年。
\item Robert Murphy(1833)。“关于定积分的逆方法”,《剑桥哲学会会刊》,第4卷,第353–408页。(注:这是除格林本人之外第一次引用其1828年论文的文献)。
\item John J. O'Connor 和 Edmund F. Robertson,“乔治·格林(数学家)”,圣安德鲁斯大学 MacTutor 数学史档案。
\item “George Green”,存档于2010年12月26日——一份极佳的乔治·格林在线信息资源。
\item George Green(1828),“《数学分析在电磁理论中的应用论文》”,arXiv:0807.0088 [physics.hist-ph]。
\item D.M. Cannell 与 N.J. Lord(1993年3月),“乔治·格林:数学家与物理学家(1793–1841)”,《数学公报》77(478),第26–51页。doi:10.2307/3619259。JSTOR: 3619259。S2CID: 238490315。
\item Lawrie Challis 与 Fred Sheard(2003年12月),“格林函数的‘格林’”,《今日物理》(Physics Today),56(12): 41–46。Bibcode:2003PhT....56l..41C。doi:10.1063/1.1650227。S2CID: 17977976。
\item “格林风车与科学中心”(Green's Mill and Science Centre)网页。访问时间:2005年11月22日。
\end{itemize}

\subsection{外部链接}
\begin{itemize}
\item 乔治·格林参考资料列表
\item 维基语录上与乔治·格林相关的名言
\item Roger Bowley,“乔治·格林与格林函数”,《Sixty Symbols》视频,由诺丁汉大学的 Brady Haran 制作。
\end{itemize}
