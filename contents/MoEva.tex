% 模型评估
% keys 模型 评估

\begin{issues}
\issueDraft
\end{issues}

\pentry{数据\upref{datast}}

当机器学习算法从某个数据集上学习到一个模型之后,需要评估该模型的好坏,也就是所获得的模型对于数据的拟合程度,此评价过程就称为\textbf{模型评估}。

对于一个数据集而言,通常须要该数据集的一部分用于模型训练,称为训练集(training set),另一部分用于模型评估,称为测试集(test set)。因此,就需要一种划分数据集的方案。划分训练集和测试集的一个重要原则是:测试集要与训练集互斥。也就是说,一条数据样本要么在训练集中,要么在测试集中,不能同时出现在训练集和测试集中。最常用的模型评估方法主要有三种,分别是\textbf{留出法}\upref{holdou}(Hold-out)、\textbf{交叉验证}\upref{CroVal}(Cross validation)和\textbf{自助法}(Bootstrap)。