% 滑块和运动斜面问题
% keys 滑块|斜面|人船模型|加速度|动量守恒
% license Xiao
% type Tutor

%\begin{issues}
%\issueTODO
%\end{issues}

%\pentry{人船模型}{nod_a52d} % \addTODO{链接}

\begin{figure}[ht]
\centering
\includegraphics[width=10cm]{./figures/6171636efc88dcad.pdf}
\caption{受力分析} \label{fig_blkSlp_1}
\end{figure}

在滑块斜面问题的基础上, 如果我们假设斜面质量为 $M$, 滑块质量为 $m$ ,滑块、斜面、地面三者之间均无摩擦, 那么滑块在斜面上自由下滑时,相对斜面的加速度是多少呢?

令 $x, y$ 为滑块水平方向和竖直方向的位移。 $X$ 为斜面水平方向的位移, $l$ 为滑块相对斜面的位移大小。对滑块与斜面组成得系统而言,在水平方向不受力,动量守恒,质心在水平方向速度 $\bvec{v_{cx}}$ 不变。以系统质心所在竖直方向为 $y$ 轴(向上为其正方向)地面为 $x$ 轴(向右为其正方向)建立直角坐标系,则有
\begin{equation}
mx+MX=0 \qquad x-X=l\cos\theta~,
\end{equation}
解得
\begin{equation}\label{eq_blkSlp_2}
x = \frac{M}{M + m}l\cos\theta \qquad X = -\frac{m}{M + m}l\cos\theta~.
\end{equation}
另外竖直方向有
\begin{equation}
y = -l\sin\theta~.
\end{equation}

以下介绍三种方法, 都可以解得滑块相对斜面的加速度为
\begin{equation}\label{eq_blkSlp_1}
a = \ddot l = \frac{g\sin\theta(M+m)}{M + m\sin^2\theta}~.
\end{equation}

\subsection{受力分析法}
以地面为固定坐标系,斜面为平动坐标系,初始时刻两坐标系重合。

由运动学关系式:
\begin{equation}
\bvec{r}=\bvec{r_{O}}+\bvec{r_{e}}~.
\end{equation}
其中,$\bvec{r}$ 为物体相对固定坐标系中的位矢、$\bvec{r_{O}}$ 为平动坐标系基点 $O$ 的位矢、$\bvec{r_{e}}$ 为物体相对平动坐标系的位矢。\\
带入
\begin{equation}
\bvec{r}=(x,y)\qquad \bvec{r_O}=(X,0)\qquad \bvec{r_e}=(l\cos\theta,-l\sin\theta)~,
\end{equation}
得
\begin{equation}\label{eq_blkSlp_3}
x=X+l\cos\theta \qquad y=-l\sin\theta~.
\end{equation}
对物块受力分析,如\autoref{fig_blkSlp_1} ,由牛顿第二定律\autoref{eq_New3_1}~\upref{New3}
\begin{equation}
m\ddot{\bvec{r}}=m\bvec g+\bvec{N}~.
\end{equation}
带入 $\bvec g=(0,-g)$、$N=(N\sin\theta,N\cos\theta)$,得
\begin{equation}\label{eq_blkSlp_4}
m(\ddot x,\ddot y)=(N\sin\theta,-mg+N\cos\theta)~.
\end{equation}

对斜面,其合外力沿 $x$ 负方向,大小为 $N\sin\theta$,对其运用牛顿第二定律
\begin{equation}\label{eq_blkSlp_5}
M\ddot X=-N\sin \theta~.
\end{equation}
联立\autoref{eq_blkSlp_3} 、\autoref{eq_blkSlp_4} 、\autoref{eq_blkSlp_5} ,得
\begin{equation}
m\qty(\frac{-N\sin\theta}{M}+\ddot l\cos\theta,-\ddot l\sin\theta)=(N\sin\theta,-mg+N\cos\theta)~,
\end{equation}
解得\autoref{eq_blkSlp_1}。

\subsection{非惯性系法}
\pentry{惯性力\nref{nod_Iner}}{nod_adba}
这是最简单的方法。 在斜面的参考系, 滑块会受到向右的惯性力 $-m\ddot X$, 所以沿斜面向下使用牛顿第二定律\upref{New3}得
\begin{equation}
-m\ddot X\cos\theta + mg\sin\theta = m\ddot l~,
\end{equation}
把\autoref{eq_blkSlp_2} 代入解得\autoref{eq_blkSlp_1}。

\subsection{拉格朗日方程法}
\pentry{拉格朗日方程\nref{nod_Lagrng}}{nod_5b81}
考虑动量守恒, 这个系统只有一个自由度, 即一个广义坐标 $l$。 拉格朗日量等于
\begin{equation}
\begin{aligned}
L = T - V &= \frac12 m(\dot x^2 + \dot y^2) + \frac12 M \dot X^2 - mgy\\
&= \frac{1}{2}m \qty( \frac{M\cos^2\theta}{M + m}+\sin^2\theta) \dot l^2 + mg\sin\theta \cdot l\\
&=\frac{1}{2}m\frac{M+m\sin^2\theta}{m+M}\dot{l}^2+mg\sin\theta\cdot l~.
\end{aligned}
\end{equation}
代入拉格朗日方程(\autoref{eq_Lagrng_1}~\upref{Lagrng})
\begin{equation}
\dv{t} \pdv{L}{\dot l} = \pdv{L}{l}~,
\end{equation}
有
\begin{equation}
m\frac{M+m\sin^2\theta}{m+M}\ddot{l}=mg\sin\theta~,
\end{equation}
解得\autoref{eq_blkSlp_1}。
