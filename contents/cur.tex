% 平面曲线的曲率(古典微分几何)

\begin{issues}
\issueDraft
\end{issues}

\pentry{欧几里得空间中的曲线\upref{eucur},平面曲线的曲率和曲率半径(简明微积分)\upref{curvat}}

我们平时所说的平面指的是二维欧几里得空间 $\mathbb{R}^2$,本节讨论的是平面中的二阶连续可导曲线($C^2$ 曲线) $C \subseteq \mathbb{R}^2$,以及参数曲线 $f: I = (a, b) \to \mathbb{R}^2$ 的曲率。

\subsection{参数曲线的曲率}

参数曲线的曲率可以定义方向,我们规定向逆时针方向弯曲(左转弯)为正曲率,顺时针方向弯曲(右转弯)为负曲率。\footnote{显然这个定义依赖于平面本身的定向(可以理解成“正反面”):假设我们将平面翻转,那么顺时针和逆时针将会互换。因此参数曲线的曲率的正负性只能在可定向的曲面上定义,不过局部上所有曲面都是欧几里得空间,因此是可定向的。}

\subsection{曲线的曲率}

曲线的曲率被用来衡量曲线的弯曲程度,直线的曲率恒为零;由于曲线的曲率可以被视作密切圆形半径的导数\upref{curvat},曲线的曲率是一个非负数。