% 用梯度求曲线和曲面的法向量
% license Xiao
% type Tutor

\begin{issues}
\issueDraft
\end{issues}

\pentry{梯度 梯度定理\upref{Grad}}{nod_ef76}

先以 $xy$ 平面的曲线为例,任意曲线可以用函数 $F(x, y) = 0$ 表示。 的曲线和 $F(x, y, z)$ 表示的曲面在某点的法向量就是他们在该点的梯度。

\subsection{推导}

平面曲线可以表示为 $F(x, y) = 0$。 即 $x, y$ 在变化的过程中始终满足这一条件。 根据微分定理, 一点 $(x, y)$ 在曲线上移动的过程中, 显然有
\begin{equation}
\dd{F} = \grad F \vdot \dd{\bvec r} = \pdv{F}{x} \dd{x} + \pdv{F}{y} \dd{y} = 0~.
\end{equation}
其中 $\dd{\bvec r}$ 表示曲线上的一段微小位移, 延曲线的切向。

上式表示, 这两个矢量的点乘为零, 即 $\grad F$ 就是就是曲线在 $(x,y)$ 点的法向量。

空间直角坐标系中的曲面同样也可以用 $F(x, y, z)$ 来表示, 从曲面上 $P_0 = (x_0, y_0)$ 点出发, 延曲面的任意微小位移 $\dd{\bvec r}$ 都满足微分关系
\begin{equation}
\dd{F} = \grad F \vdot \dd{\bvec r} = \pdv{F}{x} \dd{x} + \pdv{F}{y} \dd{y} + \pdv{F}{z} \dd{z} = 0~,
\end{equation}
既然 $\grad F$ 同时垂直于曲面内过 $P_0$ 的任意微小位移 $\dd{\bvec r}$, $\grad F$ 就是曲面在 $P_0$ 点的法向量。

======= 回收 ==============

若从曲线上的某点出发,沿曲线的切线方向取一个微位移 $\dd{\bvec r} = \dd{x}\uvec x + \dd{y} \uvec y$,由于 $(x+\dd{x}, y+\dd{y})$ 仍然在等值线上,函数增量 $\dd{f} = 0$。 代入\autoref{eq_Grad_5} 得
\begin{equation}
\grad f \vdot \dd{\bvec r} = 0~,
\end{equation}
即 $f(x,y)$ 的梯度与 $\dd{\bvec r}$ 垂直。 所以 $\grad f(x,y)$ 必定是 $(x,y)$ 点所在等值线的法向量,且指向函数值 $C$ 更大的等值线(因为函数值在梯度方向增加最快)。
% 未完成:图,画若干条等值线,其中一条上画一个法向量

=============================
