% 广义球谐函数

有了 CG 系数的相位约定和球谐函数的相位约定, 就可以定义广义球谐函数(Generalized Spherical Harmonics)
\begin{equation}
\mathcal{Y}_{l_1,l_2}^{L,M}(\uvec r_1, \uvec r_2) = \sum_{m_1, m_2} C_{l_1 m_1 l_2 m_2}^{L,M} Y_{l_1 m_1}(\uvec r_1) Y_{l_2 m_2} (\uvec r_2)
\end{equation}

\subsection{宇称}
宇称(parity)算符 $\Pi$ 的作用是把所有自变量乘以 $-1$ 得到的函数. 本征函数是所有中心对称或反对车的函数, 本征值为分别为 $\pm 1$.

球谐函数是宇称算符的本征矢, 本征值为 $(-1)^l$ (\autoref{SphHar_eq6}\upref{SphHar}), 易得广义球谐函数也是宇称算符的本征矢, 本征值为 $(-1)^{l_1+l_2}$
\begin{equation}
\Pi \mathcal{Y}_{l_1,l_2}^{L,M}(\uvec r_1, \uvec r_2) =  \mathcal{Y}_{l_1,l_2}^{L,M}(-\uvec r_1, -\uvec r_2) = (-1)^{l_1+l_2} \mathcal{Y}_{l_1,l_2}^{L,M}(\uvec r_1, \uvec r_2)
\end{equation}

另一个性质是
\begin{equation}\label{GenYlm_eq3}
y_{l_1,l_2}^{L,-M}(\uvec r_1, \uvec r_2) = (-1)^{l_1 + l_2 + L + M} y_{l_1,l_2}^{L,M}(\uvec r_1, \uvec r_2)^*
\end{equation}
推导如下,% 引用未完成 CG 系数的对称性未完成
\begin{equation}
\ali{
y_{l_1,l_2}^{L,-M}(\uvec r_1, \uvec r_2) &= \sum_{m_1+m_2 = M} \bmat{l_1 & l_2 & L \\ -m_1 & -m_2 & -M} Y_{l_1,-m_1}(\uvec r_1) Y_{l_2, -m_2} (\uvec r_2)\\
&=  (-1)^{l_1+l_2+L} \sum_{m_1+m_2 = M} \bmat{l_1 & l_2 & L \\ m_1 & m_2 & M} Y_{l_1,-m_1}(\uvec r_1) Y_{l_2, -m_2} (\uvec r_2)\\
&=  (-1)^{l_1+l_2+L+M} \sum_{m_1+m_2 = M} \bmat{l_1 & l_2 & L \\ m_1 & m_2 & M} Y_{l_1,m_1}^*(\uvec r_1) Y_{l_2, m_2}^* (\uvec r_2)\\
&= (-1)^{l_1+l_2+L+M} y_{l_1,l_2}^{L,M}(\uvec r_1, \uvec r_2)^*
}\end{equation}
