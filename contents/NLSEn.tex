% 非线性薛定谔方程的数值解法
% keys 非线性|薛定谔方程|数值
% license Usr
% type Tutor

\begin{issues}
\issueTODO
\end{issues}

\subsection{非线性薛定谔方程}
在实际某些科研方向的问题中我们会遇到需要求解非线性薛定谔方程的例子,或者著名的Gross-Pitaevskii方程,不同于薛定谔方程,非线性方程的求解在数值上更有难度。方程的一般形式是
\begin{equation}
i \frac{\partial}{\partial t} \psi = -\frac{1}{2}\nabla^2 \psi + g |\psi|^2 \psi + V(r)\psi~,
\end{equation}
其中的$V(r)$是外势。注意到这一方程通常描述多粒子体系的性质,例如Landau-Ginburg理论中描述超导序参量或者超流等,所以波函数并不总是满足归一化,即$\int \psi(x)^*\psi(x) dx \neq 1$。而且由于非线性项的出现,破坏了线性叠加原理,所以直接的虚时演化也不显然地适用与这个方程的求解。读者可能会考虑诸如有限差分法等数值方法求解,我们在这里介绍一些其它的数值计算方法,通常运算速度要优于有限差分法,且精度也更好。
\subsection{求解基态:梯度下降法}
寻找基态的过程等价于对能量函数进行优化的过程,找到能量的极小值,这里的能量就是损失函数,见梯度下降\upref{GraDec} 。能量表达式为
\begin{equation}
E = \int \frac{1}{2}|\frac{d \psi}{dx}|^2 + V(x)|\psi|^2 + \frac{g}{2}|\psi|^4~.
\end{equation}
在势阱中或自由体系,一般取$V(x)\ge 0$. 实际上在这一表达式下,基态可以一眼看出来是$\psi(x) = 0$。因为三项的贡献全部为正。但我们并不想要这样的基态,更一般情形下,我们想要给定一个有限粒子数时的基态
