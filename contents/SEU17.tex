% 东南大学 2017 年 考研 量子力学
% license Usr
% type Note

\textbf{声明}:“该内容来源于网络公开资料,不保证真实性,如有侵权请联系管理员”

\subsection{(15 分)}
\begin{enumerate}
    \item 李萨如轨迹是定常轨迹,无节点或过零。
    \item 线形物体必须是刚体。
    \item 时间反演对称性违反了物理规律。
    \item 自然单位制是一种常用于高能物理的单位制,它能够简化方程。
    \item 在经典力学中没有对应的力学量。
\end{enumerate}
\subsection{(共 30分,每小题3分)选样题}
\begin{enumerate}
    \item $\{x_i, p_j\}$ 的共同本征函数为 
(a) $\delta(x-a)\delta(y-b)\quad  (b) \delta(x-a)\delta(y-b) \quad (c) \delta(x-a)\delta(p-b) \quad (d) \delta(p-a)\delta(p-b)$
    \item 无自旋单粒子在 $xy$ 平面内运动,力学量完全集可选为 
(a) $\{p_x, p_y\} \quad (b) \{r_x, r_y\}\quad (c) \{r_x, r_y, p_x, p_y\}\quad (d) \{L_z, p_x, p_y\}$
    \item $A$ 和 $\beta$ 均为线性算符,$[\alpha, \beta]$ 可为 
(a) $A\beta + \beta A\quad (b) A\beta\quad (c) \beta A + \beta^2\quad (d) A\beta - \beta A$
    \item 设 $\sigma = |\psi\rangle \langle\phi|$,则 $\sigma$ 可为\\ 
(a)$|\phi\rangle \langle\psi| \quad (b) -\langle\phi|\psi\rangle \quad (c) \langle\phi|\psi\rangle\quad d) |\psi\rangle \langle\phi|$
     \item 设 $S$ 为电子自旋向上态的最小投影数,视为:\\
(a) $3\hbar/4 \quad (b) 1/4 \quad  (c) \hbar/4 \quad (d) 1/2 $
     \item 角动量算符的对易式 $[\hat{L}_x, \hat{L}_z]$ 等于:\\
(a) $i\hbar\hat{L}_z \quad (b) \hat{L}_z  \quad  (c) 0 \quad (d) i\hbar $
\item 设一粒自由粒子的能级 $ E = 0$,则其能级的简并度为:\\
(a) $1 \quad (b) 2  \quad  (c) 3 \quad (d) 4 $
\item 体系由2个全同玻色子组成,每个粒子可处于两个单粒子态中的任何一个,则体系可能的量子态数目为:\\
(a) $1 \quad (b) 2  \quad  (c) 3 \quad (d) 4 $
\item 以下哪一个量子体系存在散射态:\\
(a) $ \text{一维谐振子;} \quad (b) \text{三维各向同性谐振子;}  \quad  (c) \text{一维无限深方势井;} \quad (d)\text{一维有限深方势井} $
\item 一下哪种现场必须要用电子自旋的概念才能解释:\\
(a) $ \text{反常 Zeeman 效应;} \quad (b) \text{正常 Zeeman 效应;}  \quad  (c) \text{Landau 能级;} \quad (d)\text{AB 效应} $
\end{enumerate}
\subsection{共30分,每小题3分)填空题}
\begin{enumerate}
    \item 对于中心力场中的无自旋粒子,能级 $E_{nl}$ 的简并度为 $\underline{\hspace{3cm}}$ 。
    \item 设 $t=0$ 时刻的量子态 $\left|\psi(0)\right\rangle$ 恰好为能量本征态,即 $\hat{H}\left|\psi(0)\right\rangle = E\left|\psi(0)\right\rangle$ ,则 $t$ 时刻的量子态可表为 $\left|\psi(t)\right\rangle = \underline{\hspace{3cm}}$ 。
    \item 带电粒子在电磁场中的哈密顿算符为 $\hat{H} = \left[(\hat{p} + qB/c)^2 + \hat{p}_y^2 + \hat{p}_z^2\right]/ 2\mu - q \varepsilon \hat{y}$ ,除去,并列举两个守恒量: $\underline{\hspace{3cm}}$ 。
    \item 设 $\left| x \right\rangle$ 和 $\left| x' \right\rangle$ 为位置算符 $\hat{x}$ 的本征态,则 $\left\langle x \left| x' \right\rangle\right. = \underline{\hspace{3cm}}$ 。
    \item 设 $\dot{A}$ 为湮灭算符,$\dot{A} |a\rangle = A_a |a\rangle , \quad \dot{A} |n\rangle = A_n |n\rangle , \quad A_a \neq A_n$ ,则 $\left\langle m \left| n \right\rangle\right. = \underline{\hspace{3cm}}$ 。
    \item 设一维谐振子的摄动频率为 $\omega$ ,则能量本征值为 $\underline{\hspace{3cm}}$ 。
    \item 设态矢 $\left|\phi\right\rangle =  \Lambda\left|\psi\right\rangle$ ,则 $\left\langle \phi|\right. = \underline{\hspace{3cm}}$ 。
    \item 设球谐函数 $Y_{lm}(\theta, \phi)$ 表示 $\{j^2, j_z\}$ 的共同本征函数为$\psi = (1/\sqrt{2}) Y_{1,-1}(\theta, \phi) + (1/\sqrt{2})Y_{2,1}(\theta, \phi), \text{则} \hat{l}^2 \text{的平均值为}\underline{\hspace{3cm}}$ 。
    \item 设量子体系某个表象的基矢量为 $|k\rangle, (k = 1, 2, \ldots)$,它具有正交归一性和完备性。若系的任意一个量子态可表示为 $|\psi\rangle = \sum a_k|k\rangle$,则 $a_k$ 满足:$\underline{\hspace{3cm}}$
    \item 在一维散射问题中,反射系数为 $r$,透射系数为 $t$,则 $r + t = $
\end{enumerate}
\subsection{(10 分)}质量为 $m$ 的粒子处于频率为 $\omega$ 的一维谐振子势中。基态 $|0\rangle$ 满足 $\langle 0| 0\rangle = 0$。其中下降算符 $a = \frac{1}{\sqrt{2}} (\alpha x + ip/\hbar a)$, $\alpha = \sqrt{m\omega/\hbar}$。试求坐标表示中的基态波函数 $\psi_0(x) = \langle x| 0\rangle$。

\subsection{(10 分)}两个电子的交换算符为 $\hat{P}_{12} = (\hat{\sigma}_1\cdot \hat{\sigma}_2+1)/2$,自旋单态和三重态分别记为 $|00\rangle$ 和 $|11\rangle$,$(M = 0, \pm1)$。试求 $\hat{P}_{12}|00\rangle$, $\hat{P}_{12}|11\rangle$。

\subsection{(10 分)}设氢原子在 $t<0$ 时处于基态 $|0\rangle$,且 $\langle 0| \hat{x}^2 | 0\rangle = \sigma^2$。此后受到微扰 $H' = \lambda \dot{x} \delta(t)$ 的作用,在一级近似下,求 $t>0$ 时氢原子仍然保持在基态的几率。提示:量子跃迁几率公式为

$$C_{jk}(t) = \frac{1}{i\hbar} \int_0^t dt' \, H_{jk}(t') \exp(i\omega_{jk} t')~$$

\subsection{(10 分)}质量为 $m$ 的电子被限制在面积为 $S=L_x L_y$ 的平面内运动,外加一个 $z$ 方向的均匀磁场 $B$。电子的哈密顿算符为

$$\hat{H} = p_z^2/2m + 1/2m \omega^2\left(\hat{y}-c\hat{p}_x/eB\right)^2, \quad (\omega = eB/mc)~$$
试求iindaū 能级及其简并度

\subsection{(10 分)}设某微扰体系的某个能级 $E$ 是 4 重简并的。相应的正规归一的能量本征态为 $\ket{i} (i=1,2,3,4)$。在这 4 个简并态所张成的子空间中,微扰哈密顿量的非零矩阵元为

$$H_{12} - H_{21} = \lambda,~$$其它矩阵元 $H_{ij}$ 为零,试求能级 $E$ 的一级修正 $\epsilon$。

\subsection{(10 分)}考虑某个量子态空间,$F$表象的正交归一基失组为 $\ket{k}$,$G$ 表象的正交归一基失组为 $\ket{a}$。 这两个表象之间的变换矩阵元定义为 $S_{lk} = \langle a | k \rangle$。 证明变换矩阵 $S$ 为幺正矩阵,即 $S^\dagger S = S S^\dagger = I$ (单位矩阵)。
