% 双共轭梯度法解线性方程组

\begin{issues}
\issueDraft
\end{issues}

\pentry{多元函数的极值\upref{MulPlo}}

\footnote{本文参考 \cite{NR3}.}要求解对称正定矩阵(SPD) $\mat A$, 的线性方程组
\begin{equation}\label{ConGra_eq1}
\mat A \bvec x = \bvec b
\end{equation}
只需要令
\begin{equation}
f(\bvec x) = \frac{1}{2}\bvec x\Tr \mat A\bvec x - \bvec b\Tr \vdot \bvec x
\end{equation}
容易证明 $\grad f = \mat A \bvec x - \bvec b$. 注意 $f(x)$ 是一个凹二次函数, 所以取最小值当且仅当梯度为零. 这样, 解方程组的问题就转化为求函数极小值问题. 我们可以用梯度法(链接未完成)来求最小值, 即从出发点 $\bvec x_0$ 开始, 在梯度方向搜索函数最小值的位置 $\bvec x_1$, 再在其梯度方向搜索最小值的位置 $\bvec x_2$……

梯度法可以拓展为\textbf{共轭梯度法}, 以适用于任意线性方程组. 该方法的优势在于用户只需要向解算器提供矩阵乘矢量的函数, 而不需要提供矩阵本身. 这样, 矩阵可以具有任意的数据结构, 例如各种稀疏矩阵\upref{SprMat}.

当矩阵 $\mat A$ 接近于单位矩阵时, 该方法收敛更快, 因此, 我们可以选择不直接求解\autoref{ConGra_eq1} 而是求解
\begin{equation}
(\tilde {\mat A}^{-1}\mat A) \bvec x = \tilde {\mat A}^{-1}\bvec b
\end{equation}
其中 $\tilde {\mat A}$ 和 $\mat A$ 接近, 但更易于求解. 这样就有 $\tilde {\mat A}^{-1}\mat A \approx \mat I$. 这里 $\tilde {\mat A}$ 通常称为 preconditioner. 该方法称为 \textbf{preconditioned biconjugate gradient method (PBCG)}. 如果你找不到更好的 preconditioner, 通常可以用 $\mat A$ 的对角线充当. 若选择不使用 preconditioner, 也可以直接令 $\tilde {\mat A}$ 为单位矩阵.

PBCG 的 C++ 代码见 \cite{NR3} 的 \verb|linbcg.h|, 其中 \verb|asolve()|. 另外 C++ 的 Eigen 库也提供 \verb|Eigen::BiCGSTAB| 算法. \verb|Matlab| 也有 \verb|x = bicgstab(A,b)| 函数(支持复数). 其中 \verb|STAB| 意思是 \verb|stablized|, 单纯的 \verb|BiCG| 据说并不稳定, 见 Wikipedia 相关页面.
