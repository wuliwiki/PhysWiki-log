% 氢原子的含时薛定谔方程(球坐标)
% keys 薛定谔方程|球坐标|本征矢|张量积|球谐函数
% license Xiao
% type Tutor

\begin{issues}
\issueAbstract
\issueOther{这篇大概是用来数值解 TDSE 的。}
\end{issues}

\pentry{氢原子的定态薛定谔方程(球坐标)\nref{nod_RadSE}, 张量积空间\nref{nod_DirPro},Wigner 3j 符号\nref{nod_ThreeJ}}{nod_4ca7}

本文使用\enref{原子单位制}{AU}。 我们希望求解氢原子在电场中的\enref{薛定谔方程}{TDSE}
\begin{equation}\label{eq_HyTDSE_7}
-\frac{1}{2m}\laplacian \Psi -\frac{Z}{r}+ H_F(\bvec r, t)\Psi = \I \pdv{t}\Psi~.
\end{equation}
$H_F(\bvec r, t)$ 是电磁波对电子的作用项, \enref{长度规范}{LenGau}下
\begin{equation}\label{eq_HyTDSE_8}
H_F(\bvec r, t) =  - q\bvec{\mathcal E}(t) \vdot \bvec r~,
\end{equation}
速度规范下(见\autoref{sub_HyTDSE_1})
\begin{equation}
H_F(\bvec r, t) =  -\frac{q}{m} \bvec A\vdot\bvec p~.
\end{equation}
在原子单位中电子电荷 $q = -1$。 我们近似认为原子的长度远小于电磁波的波长, 所以电磁场不随位置变化。

虽然求解方程最直观的方法是使用直角坐标, 但计算效率较低。 实际中一般使用\enref{球坐标系}{Sph}, 用\enref{球谐函数}{SphHar}展开波函数(参考球坐标系中的\enref{定态薛定谔方程}{RadSE})。
\begin{equation}\label{eq_HyTDSE_3}
\Psi(\bvec r, t) = \sum_{l,m} R_l (r, t) Y_{l,m} (\uvec r) = \frac{1}{r}\sum_{l,m} \psi_{l,m}(r, t) Y_{l,m}(\bvec r)~.
\end{equation}
其中 $\psi_{l,m}(r, t)$ 是\textbf{约化径向波函数(scaled radial wave function)}, 式中每一项叫做一个\textbf{分波(partial wave)}。 如果哈密顿算符是关于 $z$ 轴对称的(例如线偏振电场), 且初始波函数也轴对称, 那么波函数将始终保持轴对称。 这时只需要 $m = 0$ 的球谐函数, 即勒让德多项式(见\autoref{eq_SphHar_11})。

\subsection{线偏振光}
若我们取电场极化方向为 $\uvec z$, 则角动量 $L_z$ 是一个守恒量。 假设初始波函数关于 $\uvec z$ 轴对称, 那么在波函数的整个演化(propagation)过程中, 我们只需要 $m=0$ 的球谐函数展开波函数, 即
\begin{equation}
\Psi(\bvec r, t) = \frac{1}{r}\sum_{l'} \psi_{l'}(r, t) Y_{l', 0}(\uvec r)~.
\end{equation}
另外薛定谔方程中 $H_F(\bvec r, t) = \bvec{\mathcal E}(t) \vdot \bvec r = \mathcal E(t) z$, 进而可以用球谐函数表示(\autoref{eq_SphHar_4})
\begin{equation}
H_F(\bvec r, t) = \mathcal E(t) r \cos\theta = \sqrt{\frac{4\pi}{3}} \mathcal E(t) r \cdot Y_{1,0}(\uvec r)~.
\end{equation}
以上两式代入\autoref{eq_HyTDSE_7}, 再把每一项与 $Y_{l,0}(\uvec r)$ 做内积(放入积分 $\int Y_{1,0}^*(\uvec r) \dd{\Omega}$ 中)可得一系列\textbf{耦合方程(coupled equations)}
\begin{equation}\label{eq_HyTDSE_4}
H_0 \psi_{l} + \mathcal E(t)r\sum_{l' = 0}^{\infty} F_{l, l'} \psi_{l'} = \I \pdv{\psi_{l}}{t} \quad (l=0,1,\dots)~.
\end{equation}
其中无场哈密顿算符为
\begin{equation}
H_0 = -\frac{1}{2m} \dv[2]{r} -\frac{Z}{r} + \frac{l(l+1)}{2mr^2}~.
\end{equation}
矩阵 $\mat F$ 就是跃迁偶极子矩阵的角向积分(\autoref{eq_SphHar_5})
\begin{equation}
\begin{aligned}
F_{l,l'} &= \mel{Y_{l,0}}{\cos\theta}{Y_{l',0}} = \sqrt{\frac{4\pi}{3}} \mel{Y_{l,0}}{Y_{1,0}}{Y_{l',0}}\\
&= \sqrt{(2l+1)(2l'+1)} \pmat{l & 1 & l'\\ 0 & 0 & 0}^2\\
&= \sqrt{\frac{2l+1}{2l'+1}} \bmat{l & 1 & l'\\ 0 & 0 & 0}^2~.
\end{aligned}
\end{equation}
可见, 当没有外场的时候每一个项(即每一个分波)都可以独立演化, 而电场将不同的分波耦合起来。 根据氢原子的选择定则 $\Delta l = \pm 1$(\autoref{eq_SelRul_2}), 矩阵 $\mat F$ 中除了两条副对角线上的元都为零。 另外由\autoref{eq_ThreeJ_2}  易得 $\mat F$ 是一个对称矩阵。

\subsection{任意偏振光}
将所有 $(l,m)$ 按某种顺序排列,例如
\begin{equation}
(0, 0),\ (1,-1),\ (1,0),\ (1,1),\ (2,-2), \dots~
\end{equation}
并将他们编号为 $\lambda = 1,2, \dots$, 那么可以把\autoref{eq_HyTDSE_3} 记为
\begin{equation}
\Psi(\bvec r, t) = \frac{1}{r}\sum_\lambda \psi_\lambda(r) Y_\lambda(\bvec r)~.
\end{equation}
电场作用为 $H_F(\bvec r, t) = \bvec{\mathcal E}(t) \vdot \bvec r  = E_x x + E_y y + E_z z$。 其中 $x,y,z$ 可以用球谐函数表示为(\autoref{eq_SphHar_4})
\begin{equation}
x = \sqrt{\frac{2\pi}{3}} r (Y_{1,-1} - Y_{1,1})~, \qquad
y = \I \sqrt{\frac{2\pi}{3}} r (Y_{1,-1}+Y_{1,1})~, \qquad
z = \sqrt{\frac{4\pi}{3}} rY_{1,0}~.
\end{equation}
\autoref{eq_HyTDSE_4} 的耦合方程拓展为
\begin{equation}
\begin{aligned}
H_0 \psi_{\lambda}(r) + r \sum_{\lambda'} \qty[E_x(t) F_{\lambda, \lambda'}^{x} + E_y(t) F_{\lambda, \lambda'}^{y} + E_z(t) F_{\lambda, \lambda'}^{z}] \psi_{\lambda'}(r) \\
= \I \pdv{\psi_{\lambda}}{t} \quad (\lambda=0,1,\dots)~.
\end{aligned}
\end{equation}
三个耦合矩阵分别为
\begin{equation}
\begin{aligned}
F_{\lambda,\lambda'}^{(x)} = \sqrt{\frac{2\pi}{3}} \qty(\mel{Y_{l,m}}{Y_{1,-1}}{Y_{l',m'}} - \mel{Y_{l,m}}{Y_{1,1}}{Y_{l',m'}})~,
\end{aligned}
\end{equation}
\begin{equation}
\begin{aligned}
F_{\lambda,\lambda'}^{(y)} = \I\sqrt{\frac{2\pi}{3}} \qty(\mel{Y_{l,m}}{Y_{1,-1}}{Y_{l',m'}} + \mel{Y_{l,m}}{Y_{1,1}}{Y_{l',m'}})~,
\end{aligned}
\end{equation}
\begin{equation}\label{eq_HyTDSE_10}
\begin{aligned}
F_{\lambda,\lambda'}^{(z)} = \mel{Y_{l,m}}{\cos\theta}{Y_{l',m'}}
= \sqrt{\frac{4\pi}{3}} \mel{Y_{l,m}}{Y_{1,0}}{Y_{l',m'}}~.
\end{aligned}
\end{equation}
其中(\autoref{eq_SphHar_5})
\begin{equation}
\mel{Y_{l,m}}{Y_{1,m_1}}{Y_{l',m'}} = (-1)^m\sqrt{\frac{3(2l+1)(2l'+1)}{4\pi}} \pmat{l & 1 & l'\\ 0 & 0 & 0}\pmat{l & 1 & l'\\ -m & m_1 & m'}~.
\end{equation}
这在 “\enref{氢原子的跃迁偶极子}{SelRul}” 也有出现。 当 $m_1=0$ 时,使用(\autoref{eq_SphHar_20})令\autoref{eq_HyTDSE_10} 为
\begin{equation}\label{eq_HyTDSE_11}
\mathcal C_{l,m} = \mel{Y_{l,m}}{\cos\theta}{Y_{l+1,m}} = \sqrt{\frac{(l+1)^2-m^2}{(2l+1)(2l+3)}}~.
\end{equation}
则
\begin{equation}
\mat F^{(z)}
=\pmat{
0 & \mathcal C_{0m} & 0 & 0 & \dots\\
\mathcal C_{0m} & 0 & \mathcal C_{1m} & 0 &\dots\\
0 & \mathcal C_{1m} & 0 & \mathcal C_{2m} &\dots\\
0 & 0 & \mathcal C_{2m} & 0 & \dots\\
\vdots & \vdots & \vdots & \vdots & \ddots}~.
\end{equation}
也就是
\begin{equation}\label{eq_HyTDSE_14}
\mel{Y_{l,m}}{\cos\theta}{Y_{l',m'}} = \delta_{m,m'}(\delta_{l+1,l'}\mathcal C_{l,m} + \delta_{l,l'+1}C_{l',m'})~.
\end{equation}

\subsection{速度规范}\label{sub_HyTDSE_1}
\pentry{速度规范\nref{nod_LVgaug}}{nod_e7b3}
在速度规范下, 当矢势不为零时, 长度规范波函数乘以 $\exp(\I \bvec A \vdot \bvec r)$ 就是速度规范波函数(\autoref{eq_LVgaug_6})。 这导致不同分波的概率不同。 考虑到强场下矢势就是波包的速度, 这个相位因子有助于让波函数的波长变长, 使所需的分波大大减少(频率高的平面波需要更多\enref{球谐函数展开}{Pl2Ylm})。

\textbf{特别注意}:在速度规范下即使只考虑从基态的单光子电离,也需要好几个分波,因为电矢势不为零时,波函数比起长度规范叠乘了一个平面波,而这个平面波需要更多分波才能展开。

要使用速度规范(注意仍然是位置表象而不是动量表象), $H_0$ 算符的计算是一样的, 唯一不同的是把 $\bvec{\mathcal E}\vdot \bvec r$ 换成了 $\bvec A\vdot \bvec p$。

先把电场限制在 $z$ 方向, 所以场的作用主要就是(\autoref{eq_SphNab_5})
\begin{equation}\label{eq_HyTDSE_25}
\pdv{z} = \cos\theta\pdv{r} - \frac{\sin\theta}{r}\pdv{\theta}~.
\end{equation}
第二项只耦合不同的分波。 但第一项要更为复杂,它耦合不同分波中同一有限元中的不同基底。 所以 $\exp(-\I \lambda \pdv{z})$ 需要把 $\pdv*{z}$ 作用在整个波函数上面, 然后用 lanczos 这样的整体方法来演化。

在 FEDVR 中, $\pdv*{r}$ 可以用矩阵 $\mat D$ 精确表达, $\cos\theta$ 也在上文中可以表示为分波耦合矩阵。 所以 $\pdv*{z}$ 就是把这两个矩阵相乘即可。

要注意第一项的角向并不是 $\mel{Y_{l,m}}{\cos\theta}{Y_{l',m}}$,而是要同时考虑径向
\begin{equation}
\begin{aligned}
&\quad r\mel{Y_{l,m}}{\cos\theta\pdv{r}}{\frac{\psi_{l',m}}{r}Y_{l',m}}\\
&= \dv{r}\psi_{l',m}\mel{Y_{l,m}}{\cos\theta}{Y_{l',m}}
-\frac{\psi_{l',m}}{r}\mel{Y_{l,m}}{\cos\theta}{Y_{l',m}}~.
\end{aligned}
\end{equation}
所以\autoref{eq_HyTDSE_25} 的第一项实际上需要拆分成两项, 没有导数的那个合并到第二项中去。 另外由\autoref{eq_SphHar_18}  得
\begin{equation}\label{eq_HyTDSE_13} % 和 Qprop [Baue06] 论文中一样
F^{(vz)}_{l,l'} = -\mel{Y_{l,m}}{\cos\theta + \sin\theta\pdv{\theta}}{Y_{l',m}}
= \delta_{l',l+1} l'\mathcal C_{l,m} - \delta_{l,l'+1} l \mathcal C_{l',m}~.
\end{equation}
\begin{equation}
\mat F^{(v2)}
=\pmat{
0 & \mathcal C_{0m} & 0 & 0 & \dots\\
-\mathcal C_{0m} & 0 & 2\mathcal C_{1m} & 0 &\dots\\
0 & -2\mathcal C_{1m} & 0 & 3\mathcal C_{2m} &\dots\\
\vdots & 0 & -3\mathcal C_{2m} & 0 & \dots}~.
\end{equation}
所以\autoref{eq_HyTDSE_25} 的矩阵元为
\begin{equation}\label{eq_HyTDSE_12}
\mel{Y_{l,m}}{\pdv{z}}{Y_{l',m}} = F_{l,l'}^{(z)}\pdv{r} + \frac{F_{l,l'}^{(vz)}}{r}~.
\end{equation}

\addTODO{上面这些都与数值无关,下面才是数值解特有的}

\subsection{任意含时势能}
如果要给\autoref{eq_HyTDSE_8} 加上一个额外的势能项 $V'(\bvec r, t)$, 首先需要用球谐函数进行分解
\begin{equation}\label{eq_HyTDSE_6}
V'(\bvec r, t) = \sum_{l,m} V'_{l,m}(r, t) Y_{l,m}(\uvec r)~.
\end{equation}
那么耦合矩阵元为(\autoref{eq_SphHar_5})
\begin{equation}\label{eq_HyTDSE_5}
F'_{\lambda,\lambda'}(r, t) = \mel{Y_{l'',m''}}{V'(\bvec r, t)}{Y_{l',m'}} = \sum_{l,m} V'_{l,m}(r, t) \mel{Y_{l'',m''}}{Y_{l,m}}{Y_{l',m'}}~,
\end{equation}
\begin{equation}
\mel{Y_{l'',m''}}{Y_{l,m}}{Y_{l',m'}} = (-1)^{m''}\sqrt{\frac{(2l''+1)(2l+1)(2l'+1)}{4\pi}} \pmat{l''& l& l'\\ 0 & 0 & 0}\pmat{l'' & l & l'\\  -m'' & m & m'}~.
\end{equation}
在程序中, 可以把 $\mel{Y_{l'',m''}}{Y_{l,m}}{Y_{l',m'}}$ 表示为三维数组 \verb|F1(λ'', λ', λ)|, 然后在每个 $r$ 格点对 \verb|λ| 加权求和得到二维方阵。

对 $m = 0$ 的对称情况, $\mel{Y_{l'',m''}}{Y_{l,m}}{Y_{l',m'}}$ 在 $l = 0$ 时是一个对角矩阵, $l = 1$ 时只有两个 1-副对角线不为零, $l = 2$ 时只有对角线和两个 2-副对角线不为零, 以此类推。 左上角的三角形也会等于零(见\autoref{fig_AMAdd_2})。 耦合薛定谔方程变为
\begin{equation}
H_0 \psi_{l} + \mathcal E(t)r\sum_{l' = 0}^{\infty} F_{l, l'} \psi_{l'} + \sum_{l'=0}^\infty F'_{l,l'}(r, t)\psi_{l'} = \I \pdv{\psi_{l}}{t} \quad (l=0,1,\dots)~.
\end{equation}

在下面介绍的算符拆分中, 若把 $F'$ 矩阵对角线上的元合并到 $H_0$ 中很可能会减小误差。


\subsection{回收内容}
在球坐标系中用球谐函数表示波函数是常用的做法。

我们可以将 $\psi_l (r, t) Y_{l,m} (\uvec r)$ 看做径向空间和角向空间中态矢的张量积。 我们将 $l, m$ 的组合进行排序并给每个组合一个全局下标 $i$ 或 $j$。
\begin{equation}
\Psi(\bvec r, t) = \sum_j R_j (r, t) Y_j (\uvec r) = \sum_j \frac{1}{r} \psi_j (r, t) Y_j (\uvec r)~.
\end{equation}

将波函数代入含时薛定谔方程
\begin{equation}
H \sum_j\ket{R_j}\ket{Y_j} = \I \pdv{t}  \sum_j\ket{R_j}\ket{Y_j}~.
\end{equation}
左乘 $\bra{Y_i}$, 可以将角向坐标积去, 得到一组径向函数的 coupled equation。 这不完全是 TDSE 的矩阵形式, 因为我们没有在径向选取基底\footnote{另一种理解是在径向选取 $\delta(r - r_0)$ 作为基底, 但本征值连续的基底比较复杂, 就不这么想吧。}。
\begin{equation}
\sum_j \mel{Y_i}{H}{Y_j} \ket{R_j} = \I \pdv{t} \ket{R_j}~.
\end{equation}
如果 TDSE 可以分离变量, $\ket{Y_j}$ 是 $H$ 的本征矢, 那各个径向波函数将会是独立的 (uncoupled)。
