% 分子平均碰壁数
% keys 热力学|分子|碰撞|碰壁数
% license Xiao
% type Tutor

\pentry{理想气体状态方程\upref{PVnRT}}

我们来考虑理想气体, 即气体分子之间不发生相互作用。 当容器中的气体分子平均速度为 $\bar v$, 分子数密度(单位体积内的分子个数)为 $n$ 时,单位容器面积单位时间受到分子碰撞的平均次数为
\begin{equation}
\frac14 n\bar v~,
\end{equation}
这个结论与容器的形状无关。

\subsection{简单的推导}

假设所有分子的速度都是 $v$,分子数密度为 $n$ (单位体积内的分子数)。假设分子之间不发生碰撞。如果所有的分子都向同一个方向运动,那么单位时间通过面积为 $a$ 的垂直截面的分子数为 $nva$。 如果容器是一个球壳,那么球壳的一半会受到粒子的撞击,单位时间的撞击次数(碰撞率)等于单位时间粒子通过容器最大截面的个数(如图1),即 $nv \qty(\pi R^2)$。 
\begin{figure}[ht]
\centering
\includegraphics[width=4cm]{./figures/f667d3c572f67c61.pdf}
\caption{分子同向运动的情况} \label{fig_AvgHit_1}
\end{figure}
如果有一半的分子向右移动,一半向上移动(如图2),那么每个方向的分子数密度变为原来的一半,总的碰撞率仍为
\begin{equation}
\frac12 nv \qty(\pi R^2) + \frac12 nv \qty(\pi R^2) = nv \qty(\pi R^2)~.
\end{equation}


\begin{figure}[ht]
\centering
\includegraphics[width=4cm]{./figures/46038544d5755cd3.pdf}
\caption{分子向两个方向运动的情况} \label{fig_AvgHit_2}
\end{figure}
依此类推,如果分子运动的方向被均匀分布在空间的各个方向上,单位时间碰撞数仍然是 $nv \qty(\pi R^2)$。
由于球形容器的表面积为 $4\pi R^2$, 所以单位容器壁面积单位时间的碰撞数就是 $nv/4$。 

接下来如果把球形容器改成任意形状的容器,由于分子运动在各个方向都是一样,所以结论不变。 另外,一般情况下并不是每个分子都具有相同的速度,所以速度取平均值 $v$ 即可。

\subsection{积分推导}

利用立体角\upref{SolAng} 对不同运动方向的分子数积分。设运动方向与截面法向量夹角为 $\theta$,那么在 $\dd\omega=\sin\dd\theta\dd\phi$ 所对应的方向上,单位时间内通过面积为 $a$ 的截面的分子数为 $n\bar v\cos\theta \frac{\dd\Omega}{4\pi}$。对 $\theta$ 从 $0$ 到 $\pi/2$ 积分,得到
\begin{equation}
\frac{1}{4\pi}\int_0^{2\pi}\dd\phi\int_0^{\pi/2}\sin\theta\dd\theta \cdot (n \bar v\cos\theta)=\frac{n\bar v}{2} \int_0^{\pi/2}\sin\theta\cos\theta \dd\theta = \frac{1}{4} n\bar v ~.
\end{equation}

特别地,如果考虑一个各向同性的光子气体,也就是说辐射场。每个光子的运动速度都为 $c$。假设频率为 $\nu$ 的辐射场能量密度(具体的定义见黑体辐射定律\upref{BBdLaw})为 $S_\nu(\nu)$,那么单位面积上的能流辐射强度就是
\begin{equation}
J(\nu)=\frac{1}{4} S_\nu(\nu) c ~.
\end{equation}
这就是黑体辐射的 Stefan-Boltzmann 定律。
