% 折射
% license CCBYSA3
% type Wiki

(本文根据 CC-BY-SA 协议转载自原搜狗科学百科对英文维基百科的翻译)

\begin{figure}[ht]
\centering
\includegraphics[width=6cm]{./figures/f21bf66165bd4191.png}
\caption{一束光线在塑料块中被折射。} \label{fig_ZS_1}
\end{figure}

在物理学中,\textbf{折射}是波从一种介质传播到另一种介质传播方向的变化,或者是在介质中的传播方向逐渐变化。 光的折射是最常见的折射现象,但声波和水波等其他波也会经历折射。波被折射的程度取决于波速的变化以及波传播相对于速度变化方向的初始方向。

对于光,折射遵循斯涅尔定律,该定律指出,对于给定的一对介质,入射角 $\theta_1$ 和折射角 $\theta_2$ 的正弦之比等于两种介质中的相速度之比 $(v_1 / v_2)$,或者等同于两种介质的折射率之比 $(n_2 / n_1)$。

$\frac{\sin \theta_1}{\sin \theta_2} = \frac{v_1}{v_2} = \frac{n_2}{n_1}$

光学棱镜和透镜利用折射来改变光线的方向,人眼也是如此。材料的折射率随着光的波长而变化,[3] 因此折射角也相应地变化。这被称为色散,并导致棱镜和彩虹将白光分成其组成光谱颜色。[4]

\subsection{常规解释}

\subsection{光}

\subsubsection{2.1 折射定律}

\subsubsection{2.2 水面折射}

\subsubsection{2.3 色散}

\subsubsection{2.4 大气折射}

\subsection{水波}

\subsection{临床意义}

\subsection{声学}

\subsection{隧道效应}

\subsection{参考文献}