% 代数拓扑(综述)
% license CCBYSA3
% type Wiki

本文根据 CC-BY-SA 协议转载翻译自维基百科\href{https://en.wikipedia.org/wiki/Algebraic_topology}{相关文章}

\begin{figure}[ht]
\centering
\includegraphics[width=6cm]{./figures/1e3777d61cc0bb5a.png}
\caption{环面,代数拓扑中最常被研究的对象之一。} \label{fig_DSTP_1}
\end{figure}
代数拓扑是数学的一个分支,它利用抽象代数的工具研究拓扑空间。其基本目标是寻找能够刻画拓扑空间的代数不变量,从而将这些空间按同胚分类,尽管在实际研究中,大多数情况下是按同伦等价进行分类的。

虽然代数拓扑主要是用代数方法研究拓扑问题,但有时也可以利用拓扑方法来解决代数问题。例如,代数拓扑就能给出一个简便的证明:自由群的任意子群仍然是自由群。
\subsection{主要分支}
以下是代数拓扑中研究的一些主要方向:
\subsubsection{同伦群}
在数学中,同伦群用于代数拓扑中对拓扑空间进行分类。第一个也是最简单的同伦群是基本群,它记录了一个空间中环路的信息。直观地说,同伦群记录了拓扑空间的基本形状或“洞”的信息。
\subsubsection{同调}
在代数拓扑和抽象代数中,同调(部分源自希腊语“ὁμός homos”,意为“相同”)是一种通用方法,用于将一个给定的数学对象(例如拓扑空间或群)关联到一个阿贝尔群或模的序列中。
\subsubsection{上同调}
在同调理论和代数拓扑中,上同调是一个总称,指的是由上链复形定义的一系列阿贝尔群。换句话说,上同调是一种对上链、上循环和上边界的抽象研究。上同调可以看作是一种赋予拓扑空间代数不变量的方法,这些不变量具有比同调更精细的代数结构。上同调起源于同调构造的代数对偶化过程。用不那么抽象的语言来说,上链从根本意义上讲,应该是为同调理论中的链赋予某种“量”的方法。
\subsubsection{流形}
流形是一种拓扑空间,它在每个点附近都与欧几里得空间相似。常见的例子有平面、球面和环面,这些都可以在三维空间中实现;还有克莱因瓶和实射影平面,它们无法嵌入三维空间,但可以嵌入四维空间。通常,代数拓扑中的研究结果聚焦于流形的整体性、不可微分的性质;例如庞加莱对偶性。
\subsubsection{结理论}
结理论研究的是数学中的“结”。虽然它的灵感来源于日常生活中鞋带或绳索打结的现象,但数学上的“结”有所不同,其两端是连在一起的,因此无法简单地解开。用精确的数学语言表述,一个结是将一个圆嵌入三维欧几里得空间$\mathbb{R}^3$中的映射。如果两个数学结可以通过对 $\mathbb{R}^3$ 本身进行连续变形(称为“环境同胚”或 ambient isotopy)而互相转化,则它们被视为等价的。这些变形对应于我们操作一根绳子时,绳子没有被剪断或穿过自身的过程。
\subsubsection{复形}
\begin{figure}[ht]
\centering
\includegraphics[width=6cm]{./figures/e922151cd4e6f24e.png}
\caption{一个三维单纯复形。} \label{fig_DSTP_2}
\end{figure}
\textbf{单纯复形}是一类特殊的拓扑空间,通过将点、线段、三角形以及它们的 $n$ 维对应物“拼接”在一起构造而成(见示意图)。需要注意的是,单纯复形不应与现代单纯同伦理论中出现的更抽象的“单纯集”混淆。与单纯复形对应的纯组合论对象被称为“抽象单纯复形”。

\textbf{CW复形}是一类由 J. H. C. Whitehead 引入的拓扑空间,用于满足同伦理论的研究需求。这类空间比单纯复形更为广泛,并且在范畴性质上有一些更好的表现,但它依然保留了组合论的特性,使得计算仍然可行(并且通常可以使用规模更小的复形来完成计算)。
\subsection{代数不变量方法}
该学科早期的名称是“组合拓扑”,强调研究空间 $X$ 如何通过更简单的空间构造出来\(^\text{[2]}\)(现代用于此类构造的标准工具是CW复形)。在20世纪20至30年代,研究重心逐渐转向通过寻找拓扑空间与代数群之间的对应关系来研究空间,这促成了名称从“组合拓扑”向“代数拓扑”的转变\(^\text{[3]}\)。如今,“组合拓扑”这一名称有时仍被使用,用来强调基于空间分解的算法化研究方法\(^\text{[4]}\)。

在代数方法中,人们寻找空间与群之间的对应关系,这种对应关系尊重空间的同胚或更一般的同伦关系。这样,关于拓扑空间的命题就可以转化为关于群的命题,而群具有丰富且易于处理的代数结构,使得证明这些命题通常更为容易。实现这种转化的两种主要方法是通过**基本群或更一般的同伦理论,以及通过同调群和上同调群。基本群提供了关于拓扑空间结构的基础信息,但它们往往是非阿贝尔群,处理起来较为复杂。不过,有限单纯复形的基本群是可以用有限表示描述的。

另一方面,同调群和上同调群是阿贝尔群,并且在许多重要情况下是有限生成的。有限生成的阿贝尔群已经被完全分类,因此在理论和计算中都特别容易处理。
\subsection{范畴论中的框架}
总体而言,代数拓扑中的所有构造都是函子性的;范畴、函子以及自然变换的概念都起源于此。基本群、同调群和上同调群不仅是基础拓扑空间的不变量(即两个同胚的拓扑空间具有相同的相关群),而且它们的态射也彼此对应——拓扑空间之间的连续映射会诱导出相应群之间的群同态。这些群同态可以用来证明某些映射的不存在性,或者更深层次地,证明某些映射的存在性。

最早研究不同类型上同调的数学家之一是乔治·德拉姆。通过德拉姆上同调,可以利用光滑流形的微分结构,或借助Čech 上同调或层上同调,去研究定义在该流形上的微分方程的可解性。德拉姆证明了这些方法是相互关联的,并且对于一个封闭且定向的流形,通过单纯同调得出的贝蒂数与通过德拉姆上同调得到的贝蒂数完全一致。这种思想在20世纪50年代得到拓展。塞缪尔·埃伦伯格和诺曼·斯廷罗德推广了这一方法,将同调和上同调定义为带有自然变换的函子,并满足一系列公理(例如,空间的弱等价会诱导同调群的同构)。他们验证了所有已有的(上)同调理论都满足这些公理,并进一步证明这种公理化刻画唯一地确定了该理论。
\subsection{应用}
代数拓扑的一些经典应用包括:
\begin{itemize}
\item 利用圆的基本群是 $\mathbb{Z}$可以较容易地证明代数学基本定理。这一证明背后依托了直观的几何思想。
\item 布劳威尔不动点定理,单位 $n$-维圆盘到自身的任意连续映射都存在至少一个不动点。
\item 同调群与欧拉–庞加莱特征数,单纯复形的第 $n$ 个同调群的自由秩就是其第 $n$ 个贝蒂数(Betti number),这使得我们可以计算其欧拉–庞加莱特征数。
\item 微分方程的可解性研究,可以通过光滑流形的微分结构,结合德拉姆上同调、Čech 上同调或层上同调,研究定义在流形上的微分方程是否有解。
\item 流形的可定向性,如果一个流形的最高维整系数同调群为整数群 $\mathbb{Z}$,则该流形是可定向的;如果该同调群为 $0$,则该流形是不可定向的。
\item 球面上的连续单位向量场,当且仅当 $n$ 是奇数时,$n$-维球面上存在一个处处不为零的连续单位向量场。(当 $n=2$ 时,这一结果常被称为“毛球定理”。)
\item 域不变性与维数不变性,空间嵌入与映射性质的基本结果。
\item 博苏克–乌拉姆定理,从 $n$-维球面到 $n$-维欧几里得空间的任意连续映射,至少会使一对对跖点被映射到相同的点。
\item 约旦曲线定理及其推广,研究平面曲线的封闭性和分割性质的基础定理。
\item 尼尔森–施赖尔定理,自由群的任意子群仍然是自由群,并且其指数与生成元数量之间存在明确关系。这一结果非常有趣,因为其表述完全是代数的,但目前已知的最简证明却是拓扑性的:任意自由群 $G$ 都可以看作某个图 $X$ 的基本群。覆盖空间理论的主要定理表明,$G$ 的任意子群 $H$ 都是 $X$ 的某个覆盖空间 $Y$ 的基本群;而每个这样的 $Y$ 本身也是一个图,因此其基本群 $H$ 也是自由群。另一方面,利用群胚的覆盖态射可以更直接地处理这种应用,该方法还得出了代数拓扑方法尚未证明的子群定理(见 Higgins, 1971)。
\item 将拓扑方法应用于组合数学问题的研究领域。
\end{itemize}
\subsection{著名人物}
\begin{itemize}
\item 弗兰克·亚当斯
\item 迈克尔·阿蒂亚
\item 恩里科·贝蒂
\item 阿尔芒·博雷尔
\item 卡罗尔·博鲁苏克
\item 劳尔·博特
\item 吕岑·埃格伯图斯·扬·布劳威尔
\item 威廉·布劳德
\item 罗纳德·布朗
\item 亨利·卡当
\item 陈省身
\item 阿尔布雷希特·多尔德
\item 夏尔·埃雷斯曼
\item 塞缪尔·艾伦伯格
\item 汉斯·弗罗因登塔尔
\item 彼得·弗赖德
\item 皮埃尔·加布里埃尔
\item 以色列·盖尔范德
\item 亚历山大·格罗滕迪克
\item 艾伦·哈彻
\item 弗里德里希·希尔泽布鲁赫
\item 海因茨·霍普夫
\item 迈克尔·J·霍普金斯
\item 维托尔德·胡雷维奇
\item 埃格伯特·范·坎彭
\item 丹尼尔·坎
\item 赫尔曼·库内特
\item 露丝·劳伦斯
\item 所罗门·莱夫谢茨
\item 让·勒雷
\item 桑德斯·麦克莱恩
\item 马克·马霍瓦尔德
\item J·彼得·梅
\item 巴里·马祖尔
\item 约翰·米尔诺
\item 约翰·科尔曼·摩尔
\item 杰克·莫拉瓦
\item 约瑟夫·内森多费尔
\item 埃米·诺特
\item 谢尔盖·诺维科夫
\item 格里高利·佩雷尔曼
\item 亨利·庞加莱
\item 列夫·庞特里亚金
\item 尼古拉·波佩斯库
\item 米哈伊尔·波斯特尼科夫
\item 丹尼尔·奎伦
\item 让-皮埃尔·塞尔
\item 伊萨多·辛格
\item 斯蒂芬·斯梅尔
\item 埃德温·斯帕尼尔
\item 诺曼·斯廷罗德
\item 丹尼斯·沙利文
\item 勒内·托姆
\item 户田浩
\item 利奥波德·维托里斯
\item 哈斯勒·惠特尼
\item J. H. C. 怀特海德
\item 戈登·托马斯·怀本
\end{itemize}
\subsection{重要定理}
\begin{itemize}
\item 布莱克斯–马西定理
\item 博苏克–乌拉姆定理
\item 布劳威尔不动点定理
\item 细胞逼近定理
\item 多尔德–托姆定理
\item 艾伦伯格–加尼亚定理
\item 艾伦伯格–齐尔伯定理
\item 弗罗因登塔尔悬挂定理
\item 胡雷维奇定理
\item 库内特定理
\item 勒夫谢茨不动点定理
\item 勒雷–希尔施定理
\item 庞加莱对偶定理
\item 塞弗特–范坎彭定理
\item 通用系数定理
\item 怀特海德定理
\end{itemize}
\subsection{参见}
\begin{itemize}
\item 代数K理论
\item 正合列
\item 代数拓扑术语表
\item 格罗滕迪克拓扑
\item 高阶范畴理论
\item 高维代数
\item 同调代数
\item K理论
\item 李代数丛
\item 李群胚
\item 塞尔谱序列
\item 层
\item 拓扑量子场论
\end{itemize}
\subsection{注释}
\begin{enumerate}
\item Fraleigh (1976, p. 163)
\item Fréchet, Maurice; Fan, Ky (2012), Invitation to Combinatorial Topology, Courier Dover Publications, p. 101, ISBN 9780486147888.
\item Henle, Michael (1994), A Combinatorial Introduction to Topology, Courier Dover Publications, p. 221, ISBN 9780486679662.
\item Spreer, Jonathan (2011), Blowups, slicings and permutation groups in combinatorial topology, Logos Verlag Berlin GmbH, p. 23, ISBN 9783832529833.
\end{enumerate}
\subsection{参考文献}
\begin{itemize}
\item Allegretti, Dylan G. L. (2008), Simplicial Sets and van Kampen's Theorem(讨论范坎彭定理推广到拓扑空间和单纯集的版本)。
\item ABredon, Glen E. (1993), Topology and Geometry, Graduate Texts in Mathematics, vol. 139, Springer, ISBN 0-387-97926-3.
\item ABrown, R. (2007), Higher Dimensional Group Theory,原文存档于2016-05-14,2022-08-17检索(广泛介绍了涉及多重群胚的高维范坎彭定理)。
\item ABrown, R.; Razak, A. (1984), "A van Kampen theorem for unions of non-connected spaces", Arch. Math., 42: 85–88, doi:10.1007/BF01198133, S2CID 122228464.(给出了针对由开集并集构成的空间,带有基点集的基本群胚的一般定理。)
\item ABrown, R.; Hardie, K.; Kamps, H.; Porter, T. (2002), "The homotopy double groupoid of a Hausdorff space", Theory Appl. Categories, 10 (2): 71–93.
\item ABrown, R.; Higgins, P.J. (1978), "On the connection between the second relative homotopy groups of some related spaces", Proc. London Math. Soc.* S3-36 (2): 193–212, doi:10.1112/plms/s3-36.2.193.(首次提出范坎彭定理的二维版本。)
\item ABrown, Ronald; Higgins, Philip J.; Sivera, Rafael (2011), *Nonabelian Algebraic Topology: Filtered Spaces, Crossed Complexes, Cubical Homotopy Groupoids, European Mathematical Society Tracts in Mathematics, vol. 15, European Mathematical Society, arXiv\:math/0407275, ISBN 978-3-03719-083-8,原文存档于2009-06-04(提供了不依赖奇异同调或单纯逼近方法的同伦论视角下的基础代数拓扑研究,并包含了大量关于交叉模的内容。)
\item AFraleigh, John B. (1976), A First Course In Abstract Algebra (2nd ed.), Reading: Addison-Wesley, ISBN 0-201-01984-1.
\item AGreenberg, Marvin J.; Harper, John R. (1981), Algebraic Topology: A First Course, Revised edition, Mathematics Lecture Note Series, Westview/Perseus, ISBN 9780805335576.(由Greenberg提出函子化的代数方法,Harper补充了几何风格。)
\item AHatcher, Allen (2002), Algebraic Topology, Cambridge: Cambridge University Press, ISBN 0-521-79540-0.(现代代数拓扑的几何化入门。)
\item AHiggins, Philip J. (1971), Notes on Categories and Groupoids, Van Nostrand Reinhold, ISBN 9780442034061.
\item AMaunder, C. R. F. (1970), "Algebraic Topology", Nature, 227 (5259), London: Van Nostrand Reinhold: 756, Bibcode:1970Natur.227..756F, doi:10.1038/227756a0, ISBN 0-486-69131-4.
\item Atom Dieck, Tammo (2008), Algebraic Topology, EMS Textbooks in Mathematics, European Mathematical Society, ISBN 978-3-03719-048-7.
\item Avan Kampen, Egbert (1933), "On the connection between the fundamental groups of some related spaces", American Journal of Mathematics, 55 (1): 261–7, JSTOR 51000091.
\end{itemize}
\subsection{延伸阅读}
\begin{itemize}
\item Hatcher, Allen (2002). *Algebraic Topology*. Cambridge University Press. ISBN 0-521-79160-X 和 ISBN 0-521-79540-0.
\item "Algebraic topology", *Encyclopedia of Mathematics*, EMS Press, 2001 [1994].
\item May, J. P. (1999). *A Concise Course in Algebraic Topology* (PDF). University of Chicago Press. 2022-10-09 从原始 PDF 存档。2008-09-27 检索。第 2.7 节提供了该定理在群胚范畴中作为余极限(colimit)的范畴论表
\end{itemize}述。
