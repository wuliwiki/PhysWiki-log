% 洛伦兹变换
\pentry{时间的变换\upref{SRtime}}
\subsection{洛伦兹变换}

在时间的变换与钟慢效应\upref{SRtime}一节中,我们自然而然地推导出了一维空间中的洛伦兹变换:

\begin{equation}\label{SRLrtz_eq1}
\leftgroup{
&x_2 = \frac{x_1 - vt_1}{\sqrt{1 - v^2/c^2}}\\
&t_2 = \frac{t_1 - vx_1/c^2}{\sqrt{1 - v^2/c^2}}
}
\end{equation}

当然,由于没有哪个惯性系更特殊,考虑到$K_1$相对$K_2$的速度是$-v$,我们有:

\begin{equation}\label{SRLrtz_eq2}
\leftgroup{
&x_1 = \frac{x_2 + vt_2}{\sqrt{1 - v^2/c^2}}\\
&t_1 = \frac{t_2 + vx_2/c^2}{\sqrt{1 - v^2/c^2}}
}
\end{equation}

当然了,这一逆变换也是可以手动从\autoref{SRLrtz_eq1}中解出来的.

\begin{example}{垂直方向上的洛伦兹变换}\label{SRLrtz_ex1}

依然考虑火车模型.现在在火车和铁轨的原点都树一根杆子,从铁轨系看来两根杆子高度相同.请说明为什么从火车系看来两根杆子高度依然相同.(提示:竖直的杆子和水平的杆子本质区别是什么?利用同时性的相对性说明一下.你可能需要考虑“两个事件的同时性不是绝对的”以及“同一个事件在任何参考系都是同一个事件”.)

\end{example}

\autoref{SRLrtz_ex1}的结果说明,尺缩效应和同时性的相对性只发生在火车运动的方向上.这就是说,垂直于火车的坐标轴不会因为火车的运动而发生变化.这就引出了完整版的三维空间中——或者说四维时空中的洛伦兹变换:

\begin{equation}\label{SRLrtz_eq3}
\leftgroup{
&x' = \frac{x - vt}{\sqrt{1 - v^2/c^2}}\\
&y'= y\\
&z' = z\\
&t' = \frac{t - vx/c^2}{\sqrt{1 - v^2/c^2}}
}
\qquad
\leftgroup{
&x = \frac{x' + vt'}{\sqrt{1 - v^2/c^2}}\\
&y = y'\\
&z = z'\\
&t = \frac{t' + vx'/c^2}{\sqrt{1 - v^2/c^2}}
}
\end{equation}

这里我们应用了多数资料中使用的字母习惯,避免读者造成混淆.在上式中,$x_1=x$,$t_1=t$,$x_2=x'$,$t_2=t'$.

\subsection{矩阵表示}

洛伦兹变换也可以用矩阵表示为:
\begin{equation}
L=
\left[\begin{matrix}
\frac{1}{\sqrt{1-v^2/c^2}}& -\frac{v/c^2}{\sqrt{1-v^2/c^2}}& 0& 0\\
-\frac{v}{\sqrt{1-v^2/c^2}}& \frac{1}{\sqrt{1-v^2/c^2}}& 0& 0\\
0&0&1&0\\
0&0&0&1


\end{matrix}\right]
\end{equation}

将$L$称为\textbf{洛伦兹(变换)矩阵}.

如果一个事件在$K_1$和$K_2$中的坐标分别是$\left(\begin{matrix}   t\\x\\y\\z    \end{matrix}\right)$和$\left(\begin{matrix}   t'\\x'\\y'\\z'    \end{matrix}\right)$,那么有

\begin{equation}
\left(\begin{matrix}   t'\\x'\\y'\\z'    \end{matrix}\right)
=
L
\left(\begin{matrix}   t\\x\\y\\z    \end{matrix}\right)
\end{equation}
