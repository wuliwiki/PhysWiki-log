% 柯西积分定理(综述)
% license CCBYSA3
% type Wiki

本文根据 CC-BY-SA 协议转载翻译自维基百科\href{https://en.wikipedia.org/wiki/Cauchy\%27s_integral_theorem}{相关文章}。

在数学中,柯西积分定理(也称为柯西–古尔萨定理)是复分析中的一个重要结论,以 奥古斯丁–路易·柯西(和爱德华·古尔萨的名字命名。该定理描述了复平面上全纯函数的路径积分性质。其核心内容是:如果函数$f(z)$在一个单连通域$\Omega$ 内是全纯的,那么对于 $\Omega$ 内的任何闭合路径$C$,沿着该路径的积分都为零:
$$
\int_{C} f(z)\, dz = 0.~
$$
\subsection{命题}
\subsubsection{复线积分的基本定理}
如果函数 $f(z)$ 在某个开区域 $U$ 上是全纯函数,且曲线 $\gamma$ 位于该区域内,从点 $z_0$ 延伸到点 $z_1$,则有:
$$
\int_{\gamma} f'(z)\,dz = f(z_1) - f(z_0).~
$$
此外,如果 $f(z)$ 在开区域 $U$ 内存在一个单值原函数,那么在该区域内,路径积分$\int_{\gamma} f(z)\,dz$对于所有路径来说都是路径无关的。

\textbf{在单连通区域上的表述}

设 $U \subseteq \mathbb{C}$ 是一个单连通开集,并且 $f: U \to \mathbb{C}$ 是一个全纯函数。如果 $\gamma: [a, b] \to U$ 是一条光滑的闭曲线,则有:
$$
\int_{\gamma} f(z)\,dz = 0.~
$$
其中,$U$ 是单连通集意味着它没有“洞”,换句话说,$U$ 的基本群是平凡的。

\textbf{一般形式}

设 $U \subseteq \mathbb{C}$ 是一个开集,且 $f: U \to \mathbb{C}$ 是一个全纯函数。如果 $\gamma: [a, b] \to U$ 是一条光滑的闭曲线,并且 $\gamma$同伦于一条常值曲线,则有:
$$
\int_{\gamma} f(z)\,dz = 0,~
$$
其中 $z \in U$。

一条曲线如果能通过在 $U$ 内的平滑同伦逐渐收缩到某一点(即常值曲线),则称这条曲线同伦于常值曲线。直观地说,这意味着可以在不离开区域 $U$ 的情况下,把闭合曲线“缩成一个点”。第一种表述是这一一般情况的特殊情形,因为在单连通区域中,任意闭曲线都可以同伦收缩为一点。

\textbf{主要示例}

在两种情形下,都需要注意:曲线$\gamma$不能环绕定义域中的“洞”,否则该定理不再适用。一个著名的例子如下:
$$
\gamma(t) = e^{it}, \quad t \in [0, 2\pi],~
$$
这条曲线描绘的是单位圆。

在这种情况下,积分:
$$
\int_{\gamma} \frac{1}{z} \, dz = 2\pi i \neq 0~
$$
结果不为零。这是因为在此情形下,函数:$f(z) = 1/z$在 $z = 0$没有定义。直观地看,曲线 $\gamma$ 围绕了定义域中的一个“洞”,因此无法在不离开区域的情况下把曲线缩成一个点。因此,柯西积分定理在该情形下不适用。
\subsection{讨论}
正如爱德华·古尔萨所示,柯西积分定理只需要假设函数 $f'(z)$ 在区域 $U$ 内处处存在即可成立。这一点非常重要,因为这意味着可以进一步证明这些函数满足柯西积分公式,并由此推导出这些函数是无限可微的。

条件 $U$ 是单连通区域意味着该区域内没有“洞”;从同伦的角度来看,这意味着 $U$ 的基本群是平凡的。例如,对于任意 $z_0 \in \mathbb{C}$,开圆盘$U_{z_0} = \{z : |z - z_0| < r \}$都满足单连通的条件。

这个条件非常关键。考虑以下情形:
$$
\gamma(t) = e^{it}, \quad t \in [0, 2\pi],~
$$
这条曲线描绘的是单位圆。对应的路径积分为:
$$
\oint_{\gamma} \frac{1}{z}\,dz
= \int_0^{2\pi} \frac{1}{e^{it}} \big(i e^{it} dt\big)
= \int_0^{2\pi} i\, dt
= 2\pi i.~
$$
结果不为零。这是因为:$f(z) = 1/z$在 $z = 0$没有定义(并且当然也不是全纯的)。因此,柯西积分定理在该情形下不适用。

该定理的一个重要推论是:在单连通区域内,全纯函数的路径积分可以用类似于微积分基本定理的方式来计算。设 $U$ 是复平面 $\mathbb{C}$ 中的一个单连通开子集,函数$f: U \to \mathbb{C}$在 $U$ 上是全纯的,$\gamma$ 是 $U$ 中一条分段连续可微的路径,起点为 $a$,终点为 $b$。如果 $F$ 是 $f$ 的一个复原函数(复反导函数),那么有:
$$
\int_{\gamma} f(z)\,dz = F(b) - F(a).~
$$
柯西积分定理在比上述更弱的假设下同样成立。例如,如果 $U$ 是复平面中的一个单连通开子集,则只需假设:$f$ 在 $U$ 内全纯,且在闭包 $\overline{U}$ 上连续,并且 $\gamma$ 是 $\overline{U}$ 内一条可求长的简单闭曲线,该定理依然成立$1$。

柯西积分定理是推导柯西积分公式以及留数定理的重要基础。
\subsection{证明}
如果假设一个全纯函数的偏导数是连续的,那么柯西积分定理可以直接由格林公式推导出来。这是因为函数$f = u + iv$的实部 $u$ 和虚部 $v$ 在曲线 $\gamma$ 所围成的区域内(以及该区域的开邻域 $U$ 中)必须满足柯西–黎曼方程。柯西最早给出了这种证明方法,但后来古尔萨在不使用向量分析技术、也不要求偏导数连续的条件下,提供了一个更为一般的证明。

我们可以将被积函数 $f$ 和微分 $dz$ 拆分为实部和虚部:
$$
f = u + iv~
$$
$$
dz = dx + i\,dy~
$$
在这种情况下,沿闭曲线 $\gamma$ 的积分为:
$$
\oint_{\gamma} f(z)\,dz
= \oint_{\gamma} (u + iv)(dx + i\,dy)
= \oint_{\gamma} (u\,dx - v\,dy)
+ i \oint_{\gamma} (v\,dx + u\,dy)~
$$
根据格林公式,可以将沿闭合曲线 $\gamma$ 的积分,转化为其所围平面区域 $D$ 内的面积积分:
$$
\oint_{\gamma} (u\,dx - v\,dy)
= \iint_{D} 
\left(
-\frac{\partial v}{\partial x} 
-\frac{\partial u}{\partial y}
\right)
dx\,dy~
$$
$$
\oint_{\gamma} (v\,dx + u\,dy)
= \iint_{D} 
\left(
\frac{\partial u}{\partial x} 
-\frac{\partial v}{\partial y}
\right)
dx\,dy~
$$
然而,在区域 $D$ 内,函数 $f(z) = u + iv$ 是全纯函数,其实部 $u$ 和虚部 $v$ 必须满足柯西–黎曼方程:
$$
\frac{\partial u}{\partial x} = \frac{\partial v}{\partial y},~
$$
$$
\frac{\partial u}{\partial y} = -\frac{\partial v}{\partial x}.~
$$
可以发现,这两个面积积分(以及它们对应的路径积分)都为零:
$$
\iint_{D}
\left(
-\frac{\partial v}{\partial x}
-\frac{\partial u}{\partial y}
\right)
dx\,dy
=
\iint_{D}
\left(
\frac{\partial u}{\partial y}
-\frac{\partial u}{\partial y}
\right)
dx\,dy
= 0~
$$
$$
\iint_{D}
\left(
\frac{\partial u}{\partial x}
-\frac{\partial v}{\partial y}
\right)
dx\,dy
=
\iint_{D}
\left(
\frac{\partial u}{\partial x}
-\frac{\partial u}{\partial x}
\right)
dx\,dy
= 0~
$$
最终得到所需结论:$\oint_{\gamma} f(z)\,dz = 0$
\subsection{参见}
\begin{itemize}
\item 莫雷拉定理
\item 路径积分方法
\item 星形域
\end{itemize}
\subsection{参考文献}
\begin{itemize}
\item Walsh, J. L. (1933-05-01). "The Cauchy-Goursat Theorem for Rectifiable Jordan Curves". Proceedings of the National Academy of Sciences. 19 (5): 540–541. Bibcode:1933PNAS...19..540W. doi:10.1073/pnas.19.5.540. ISSN 0027-8424. PMC 1086062. PMID 16587781.
\item Kodaira, Kunihiko (2007). Complex Analysis. Cambridge Studies in Advanced Mathematics, 107. Cambridge University Press. ISBN 978-0-521-80937-5.
\item Ahlfors, Lars (2000). Complex Analysis. McGraw-Hill Series in Mathematics. McGraw-Hill. ISBN 0-07-000657-1.
\item Lang, Serge (2003). Complex Analysis. Springer Verlag GTM. Springer Verlag.
\item Rudin, Walter (2000). Real and Complex Analysis. McGraw-Hill Series in Mathematics. McGraw-Hill.
\end{itemize}
\subsection{外部链接}
\begin{itemize}
\item "Cauchy integral theorem", Encyclopedia of Mathematics, EMS Press, 2001 [1994]
\item Weisstein, Eric W. "Cauchy Integral Theorem". MathWorld.
\item Jeremy Orloff, 18.04 Complex Variables with Applications Spring 2018, Massachusetts Institute of Technology: MIT OpenCourseWare, Creative Commons.
\end{itemize}
