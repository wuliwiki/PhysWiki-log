% C++ 解释器 Cling 笔记
% license Xiao
% type Note

\begin{issues}
\issueDraft
\end{issues}

\pentry{Conda 笔记\nref{nod_CondaN}}{nod_fd76}

\begin{itemize}
\item \href{https://root.cern/cling/}{Cling} 是一个 C++ 解释器, 可以直接像动态语言那样逐行运行 C++ 代码无需预先编译。
你可以在 \href{https://jupyter.org/try}{jupyter notebook 的 C++ 页面}在线使用 cling(具体来说是 \href{https://xeus-cling.readthedocs.io/en/latest/index.html}{Xeus-Cling})。
\item 如果想在本地使用 Jupyter Notebook 中的 Cling (推荐), 用 \verb|conda|\upref{CondaN} 安装即可。 Xeus-Cling 不支持 Windows 版的 conda(2023/8/13)。可以用 WSL2 安装 conda 再装 Cling。 WSL1 不行,安装成功,但启动 jupyter notebook 失败(2023/8/13)。
\item 新建环境(可选)\verb|conda create -n cling|; 安装 Jupyter Notebooko:\verb|conda install notebook|; 安装 Xeus Cling: \verb|conda install xeus-cling -c conda-forge|; 初始化: \verb|conda activate cling|。 运行: \verb|jupyter notebook|。
\item 在 \verb|conda| 中安装 Cling 后, 也可以在命令行运行: 可执行文件的路径如 \verb|~/miniconda3/pkgs/cling-0.8-hf817b99_1/bin/cling|, 在 \verb|/usr/bin| 里面创建一个 symlink 即可。
\item Cling 基于 \enref{LLVM}{LLVMnt}, \enref{Clang}{clangp} 能编译的 c++ 代码 Cling 都支持。
\item linux 暂时不支持 \verb|apt| 安装, 安装方法参考\href{https://kaustubh13.medium.com/how-to-install-cling-on-linux-or-wsl-8125798ed9b9}{这里}。 推荐直接 \href{https://root.cern/download/cling/}{download binary}。 解压: \verb|bzip2 -d ???.bz2|, \verb|tar -xvf ???.tar|。 解压以后可以直接进入到 \verb|bin| 文件夹运行 \verb|./cling| 看看是否成功。 如果成功, 可以添加一个软链 \verb|sudo ln -s /abs/path/to/cling /usr/bin/|。
\item 文件夹解压后不到 1GB, 如果想节约空间可以删掉所有 \verb|.a| 文件以及体积最大的几个没有后缀名的二进制文件, 剩下 100MB 左右, 貌似不影响 Cling 运行。
\item CLion 可以使用 Cling, 但如果在 WSL 中安装貌似会出问题。
\item 用 \verb|cling| 进入 Cling 命令行, 用 \verb|.q| 退出。或者 \verb|cling '命令1' '命令2'| 单独执行几个命令。 也可以用 \verb|cling < test.cpp| 执行一个文件。
\item 在打开 notebook 以前设置 \verb|CPATH| 环境变量, 可以添加头文件搜索路径。 但貌似添加了某些目录以后, 会导致 Cling kernel 无法启动。
\item 也可以在代码中使用 \verb|#pragma cling add_include_path("路径")| 添加头文件路径。
\item \verb|#include| 的相对路径是相对于当前路径的。
\item 要设置当前路径, 用 \verb|#include <unistd.h>|, 然后 \verb|chdir("路径");| 即可。 要查看当前路径, 可以用 \verb|char *getcwd(char *buf, size_t size);| 返回 \verb|buf|。
\item \verb|#pragma cling add_library_path("路径")| 可以添加动态链接库的搜索路径。
\item \verb|#pragma cling load("动态库")| 可以加载动态库\upref{gppLib}。
\end{itemize}
