% 潮汐力
% license Xiao
% type Tutor

\begin{issues}
\issueDraft
\issueNeedCite
\end{issues}
\pentry{万有引力、引力势能\upref{Gravty}, 离心力\upref{Centri}}
\begin{figure}[ht]
\centering
\includegraphics[width=8cm]{./figures/b0ba171561a1773a.pdf}
\caption{潮汐力分布示意图(地球参考系)} \label{fig_Tidal_1}
\end{figure}

我们先以月球对地球的潮汐力为例,并忽略地球自转、公转等因素。

\begin{figure}[ht]
\centering
\includegraphics[width=12cm]{./figures/14cb7436228abb83.pdf}
\caption{月球对地球各点的引力不同} \label{fig_Tidal_2}
\end{figure}

众所周知,引力\upref{Gravty} 与两点距离有关:
$$\bvec F_\text{引力}= - \frac{GMm}{R^2}\hat R~.$$

如\autoref{fig_Tidal_2}  所示,由于月球到地球各点的距离不同,所以准确的说,月球对地球各点的引力也不同,而潮汐现象正是由于引力的这种细微差异。地球表面上正对月球$A$点、背对月球的$B$点的引力分别为:
\begin{equation}
\begin{aligned}
\bvec F_A &= - \frac{GM_\text{月}\Delta m}{(R-r)^2} \hat R~,\\
\bvec F_B &= -\frac{GM_\text{月} \Delta m}{(R+r)^2}\hat R~.\\
\end{aligned}
\end{equation}

\begin{figure}[ht]
\centering
\includegraphics[width=8 cm]{./figures/458e98151982211a.pdf}
\caption{地球与月球正相互环绕} \label{fig_Tidal_5}
\end{figure}
同时别忘了,地球与月球正相互环绕,因此地球整体还有向心加速度 
\begin{equation}
\bvec a_\text{地球} = -\frac{GM_\text{月}}{R^2}\hat R~.
\end{equation}
以上二式中,$r$是地球半径,$R$是地月距离,$\hat R$是从月球指向地球的单位方向向量,负号表示引力方向是从地球指向月球的。

\begin{figure}[ht]
\centering
\includegraphics[width=12cm]{./figures/e9372f8d8c657546.pdf}
\caption{地球表面上,正对与背对月球的两点都被拉起} \label{fig_Tidal_3}
\end{figure}

如\autoref{fig_Tidal_3} 所示,从地球这一非惯性参考系上看,地球整体有加速度意味着各点都还受一个离心力(不是地球自转的离心惯性力,而是地球与月球转动的):
$$f_\text{惯} = -\bvec a_\text{地球} \Delta m =\frac{GM_\text{月}\Delta m}{R^2}\hat R~.$$考虑到这个因素后,有趣的事情发生了:在一个地球人看来,地球表面上正对与背对月球的两点似乎都正被拉起,这就是\textbf{潮汐力}。某种意义上,潮汐力也是由于处于地球这一非惯性参考系的我们的错觉。

具体而言,A点处的潮汐力为
$$
\bvec F_A' =\bvec F_A - a_\text{地球}\Delta m =  -\frac{GM_\text{月}\Delta m}{(R-r)^2}\hat R + \frac{GM_\text{月}\Delta m}{R^2} \hat R
\approx -\frac{2rGM_\text{月}\Delta m}{R^3} \hat R~.
$$
同理,B点处的潮汐力为
$$
\bvec F_B' = \bvec F_B - a_\text{地球}\Delta m
\approx \frac{2rGM_\text{月}\Delta m}{R^3} \hat R~.
$$

从\autoref{fig_Tidal_3} 可以发现,无论是正对月球的一面,还是背对月球的一面,潮汐力的方向都是离开地表的。因此,潮汐力不能简单地理解为“因为地球的一面离月球更近,所以那一侧的海洋被吸起了”。由于潮汐力的这一非同寻常的性质(再加上之前被忽略的地球自转),因此一天之内,地球出现两次潮汐现象。
\begin{figure}[ht]
\centering
\includegraphics[width=12cm]{./figures/42872539f4bfe63e.png}
\caption{一天内,地球出现两次潮汐现象} \label{fig_Tidal_4}
\end{figure}

同时,我们发现潮汐力是与距离的三次方成反比的,这意味着感性上说“潮汐力比引力衰减得更快”。
\begin{exercise}{谁是弄潮儿?}
你可能觉得太阳的质量远比月球大,因此太阳对地球产生的潮汐作用更明显?是这样的吗?\footnote{月球产生的潮汐力应该大概是太阳的二倍多。但是,这不代表太阳对潮汐现象没有影响。}
\end{exercise}

绘制地球表面潮汐力分布的octave/matlab代码:
\begin{lstlisting}[language=matlab]
clc
clear

[x y z] = sphere(20);
xs = -60; %地月距离大概是地球半径的60倍
scatter3(xs,0,0);

u = x - xs;
v = y;
w = z;

r = sqrt(u.^2+v.^2+w.^2);
mag = 10^4./(r.^2);

u = -mag.*u./r;
v = -mag.*v./r;
w = -mag.*w./r;
u = u + 10^4/(xs^2); %地球参考系这一非惯性参考系中的离心力。注释掉这一行即可得到月球引力在地球表面的分布。

hold on
axis equal
axis off
surf(x,y,z,'FaceColor','none','EdgeColor',[0.8 0.8 0.8]);
axis([-1.2 1.2 -1.2 1.2 -1.2 1.2])
quiver3(x,y,z,u,v,w);
view(-30,30);
hold off

\end{lstlisting}
