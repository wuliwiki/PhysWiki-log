% 函数的极限(简明微积分)

\pentry{数列的极限(简明微积分)\upref{Lim0}, 充分必要条件\upref{SufCnd}, 函数}

\subsection{引入}
我们先看一些简单的例子,以初步了解“极限”的概念.
\begin{example}{}\label{FunLim_ex1}

\begin{figure}[ht]
\centering
\includegraphics[width=10cm]{./figures/FunLim_10.png}
\caption{$f(x)=\frac{\sin(x)}{x}$的图像} \label{FunLim_fig1}
\end{figure}
思考一下$f(x)=\frac{\sin(x)}{x}$这一经典函数在原点附近的值.

众所周知,当$x=0$时,由于分母为$0$,该分数没有意义;但当$x$趋近于$0$($x=0.1,x=0.01,...$)而不等于$0$时,有趣的事情发生了: 如\autoref{FunLim_fig1} 所示,此时分数的值似乎趋于一个确定的值$1$.

\begin{table}[ht]
\centering
\caption{x与f(x)}\label{FunLim_tab1}
\begin{tabular}{|c|c|c|c|}
\hline
$x$ & $0.1$ & $0.01$ & $0.001$ \\
\hline
$f(x)$ & $0.9983$ & $0.99998$ & $0.9999998$ \\
\hline
\end{tabular}
\end{table}
看起来,尽管我们不能定义$f(x)$在零点处的值,但是我们知道,当$x$趋近于$0$时,$f(x)$趋近于$1$. 因此,我们说$\lim_{x\to0}\frac{sin(x)}{x}=1$
\end{example}

\begin{example}{}
\begin{figure}[ht]
\centering
\includegraphics[width=10cm]{./figures/FunLim_2.png}
\caption{$f(x)=\frac{1}{x}$的图像 (x>0)} \label{FunLim_fig2}
\end{figure}
然后,我们再看看$f(x)=\frac{1}{x}$另一经典函数的图像.我们还是知道,两个正数的商始终大于零;但当$x$\textsl{足够大}时,$\frac{1}{x}$会\textsl{足够小}以至趋于$0$. 因此,我们说$\lim_{x\to+\infty}\frac{1}{x}=0$.
\end{example}

\subsection{自变量趋于无穷的极限}
实函数 $f(x)$ 可以看成是一种 “连续” 的数列, 只不过把元素编号从离散的 $n$ 改为连续的 $x$. 类比数列的极限, 我们也可以定义\textbf{函数趋于正无穷的极限}.

\begin{definition}{函数趋于正无穷的极限}\label{FunLim_def1}
考虑实函数 $f(x)$. 若无论要求 $f(x)$ 和一确定实数 $A$ 的距离 $\epsilon$ 有多小(但 $\epsilon>0$), 都存在 实数 $X$ ,使得所有 $x>X$ 都满足 $\abs{f(x)-A}<\epsilon$, 那么我们说 $A$ 是函数 $f(x)$ 在 $x$ 趋于正无穷时的极限, 记为
\begin{equation}
\lim\limits_{x\to +\infty} f(x) = A
\end{equation}
\end{definition}

可以看到该定义和数列极限的定义(\autoref{Lim0_def2}~\upref{Lim0})非常相似, 只是简单做了替换.不过,函数并不是简单地把数列的概念拓展到连续的情况. 数列的编号只能朝着一个方向增大, 但函数的自变量 $x$ 既可以趋近正无穷也可以奔向负无穷, 

%\addTODO{画图, 画出函数曲线, 距离要求就是两条直线之间的范围, 等等}
\begin{figure}[ht]
\centering
\includegraphics[width=12cm]{./figures/FunLim_5.pdf}
\caption{对于任意一个$\epsilon$,都存在对应的$X$.仿自\cite{Thomas}} \label{FunLim_fig5}
\end{figure}

\begin{exercise}{}
请仿照\autoref{FunLim_def1} 给出函数趋于负无穷时极限的定义
\begin{equation}
\lim\limits_{x\to -\infty} f(x) = A
\end{equation}
\end{exercise}

\subsection{自变量趋于一点的的极限}
另外, 由于 $x$ 是连续取值的, 也可以考察自变量 $x$ 不断趋近某一点 $x_0$ 的极限, 即 $x\to x_0$.如何描述 “自变量趋于一个给定的实数 $x_0$” 呢? 只需要取自变量 $x$ 使得二者间的距离 $\abs{x-x_0}$ 越来越接近 $0$ 即可.
\begin{definition}{函数在某点的极限}\label{FunLim_def3}
考虑实函数 $f(x)$. 若无论要求 $f(x)$ 和确定实数 $A$ 的距离 $\epsilon>0$ 有多小, 都存在一个自变量的取值半径 $\delta>0$,使得只要 $\abs{x-x_0} < \delta$,就有 $\abs{f(x)-\delta}<\epsilon$,
% 能通过不等式 $\abs{x-x_0} < \delta$ ($\delta$ 是一确定实数)使要求成立, 
那么我们说 $A$ 是函数 $f(x)$ 在 $x$ 趋于 $x_0$ 时的极限, 记为
\begin{equation}
\lim\limits_{x\to x_0}f(x)=A
\end{equation}
\end{definition}

\begin{figure}[ht]
\centering
\includegraphics[width=12cm]{./figures/FunLim_8.pdf}
\caption{对于任意一个$\epsilon$,都存在对应的$\delta$.仿自\cite{Thomas}} \label{FunLim_fig8}
\end{figure}

\begin{example}{}
求一些简单的函数在某个值处的极限时, 通常可以直接代入数值计算, 如
\begin{equation}
\lim_{x\to 1} 2x + 1 = 3 \qquad \lim_{x\to 2}\frac{x + 1}{x + 2} = \frac34
\end{equation}

当无穷大与常数相加时, 可以忽略常数, 如
\begin{equation}
\lim_{x\to +\infty} \frac{x + 1}{2x + 2} = \lim_{x\to +\infty} \frac{x}{2x} = \frac12
\end{equation}

如果你想要处理更复杂的极限问题,那你可以参考求极限的一些方法\upref{ChaLim}.不过,其中部分方法已经超出了初学者的(以及“简明微积分”所意图介绍的)知识范围.
\end{example}

\subsubsection{左、右极限}
我们还可以区分函数在某点的\textbf{左极限(left limit)}和\textbf{右极限(right limit)}. 简而言之就是 $x$ 分别从左边和右边两个方向趋近 $x_0$ 时的极限, 具体定义留做思考. 左右极限记为
\begin{equation}
\lim_{x\to x_0^-} f(x) = A_- \qquad \lim_{x\to x_0^+} f(x) = A_+
\end{equation}

\begin{theorem}{}
函数在某点存在极限的充分必要条件是它左右极限都存在并相等.
$$\lim_{x\to x_0} f(x) = A \Leftrightarrow \lim_{x\to x_0^-} f(x) = \lim_{x\to x_0^+} f(x) = A $$

也就是说,若左(或右)极限不存在,或者左右极限存在但不相等,那此处的极限就不存在.
\end{theorem}

\begin{example}{}
\begin{figure}[ht]
\centering
\includegraphics[width=8cm]{./figures/FunLim_3.pdf}
\caption{函数$\theta(x)$的图像} \label{FunLim_fig3}
\end{figure}
函数
\begin{equation}
\theta(x) = \leftgroup{
0 \qquad (x < 0)\\
1 \qquad (x \ge 0)
}\end{equation}

计算左极限$\lim_{x\to x_0^-} \theta(x)$时,假定x从左侧不断接近$0$($x=-0.1,x=-0.01,...$),但从不超过(也不等于)$0$.此时总有$x<0$,因此$\lim_{x\to x_0^-} \theta(x) = 0$. 

同理,$\lim_{x\to x_0^+} \theta(x) = 1$

由于左右极限不相同,因此$\theta(x)$在$x=0$处的\textsl{极限}不存在.
\end{example}

\subsubsection{某点处函数的极限值与函数值}
新手最常犯的错误莫过于\textsl{过度纠结}某处的函数极限值$\lim_{x\to x_0} f(x)$与函数值$f(x_0)$的联系.事实上,这两者之间没有\textsl{必要的关联}\footnote{对于连续函数,才有$\lim_{x\to x_0} f(x)=f(x_0)$.然而,大多数常见的函数都是连续函数,这使得这个问题更具迷惑性}.$f(x_0)$可以不等于$\lim_{x\to x_0} f(x)$,$f(x)$甚至可以在$x_0$处没有定义.总之,某处的函数极限值并不依赖于该点处的函数值.

这是因为定义中只考虑 $x$ 慢慢接近 $x_0$ 的过程, 而不考虑 $x = x_0$ 的情况. 即使我们把这点从函数定义域中挖去, 极限是否存在, 以及极限值是多少都不会被改变. 例如在\autoref{FunLim_ex1} 与在 “小角正弦极限\upref{LimArc}” 中会看到, 虽然 $\sin x/ x$ 在 $x = 0$ 处没有定义, 但其极限却等于 $1$.

\begin{example}{可去间断点}
\begin{figure}[ht]
\centering
\includegraphics[width=10cm]{./figures/FunLim_4.png}
\caption{函数f(x)的图像} \label{FunLim_fig4}
\end{figure}
函数
\begin{equation}
f(x) = \leftgroup{
x \qquad (x \ne 1)\\
1.5 \qquad (x = 1)
}\end{equation}
计算$\lim_{x\to 1} f(x)$时,由于只考虑$x=1$附近的情况、而不考虑$x=1$本身的情况,因此$\lim_{x\to 1} f(x)$的结果与$f(1)$的值无关.在本例中,$\lim_{x\to 1} f(x)=1$, 而$f(1)=1.5$.
\end{example}

\subsection{极限的基本性质与运算法则}
下面列出一些函数极限的定理, 从直觉上来看它们是显然的, 证明略, 感兴趣的读者可以尝试自己证.
\begin{theorem}{极限的四则运算}
若两个函数分别存在极限 $\lim_{x\to a} f(x)$ 和 $\lim_{x\to a} g(x)$ ($a$ 可取 $\pm \infty$), 那么有
\begin{equation}
\lim_{x\to a} [f(x) \pm g(x)] = \lim_{x\to a}f(x) \pm  \lim_{x\to a} g(x)
\end{equation}
\begin{equation}
\lim_{x\to a} [f(x) g(x)] = \lim_{x\to a}f(x) \lim_{x\to a} g(x)
\end{equation}
\begin{equation}
\lim_{x\to a} [f(x)/g(x)] = \lim_{x\to a}f(x)/\lim_{x\to a} g(x) \qquad (\lim_{x\to a} g(x) \ne 0)
\end{equation}
注意,可以四则运算的前提是参与运算的各个极限均存在.
\end{theorem}

\begin{theorem}{局部保号性、保序性}
%我对着我的笔记写的,定理的表述可能有一点不严谨;不过对于“入门”启发,应该是够了(
局部保号性:
\begin{equation}
\lim_{x\to x_0}f(x)=A>0\Rightarrow \exists \mathring{U} (x_0), \forall x \in U(x_0), f(x)>0
\end{equation}

局部保序性:
\begin{equation}
\lim_{x\to x_0}f(x)=A>0 \Rightarrow \exists \mathring{U} (x_0), \forall x \in U(x_0), f(x)>A/2
\end{equation}

$\mathring{U} (x_0)$指$x_0$附近的一个小区间,但不包括$x_0$自身,也称去心区间.\textsl{“去心”这个词让我想到“比干挖心”的传说故事}

这是一组简单的、但却\textsl{有点难以理解}的结论.在处理一些刁钻的问题时,局部保号、保序性偶尔会派上用场.通俗地说,这意味着若$\lim_{x\to x_0}f(x)=A$,则$x$在$x_0$附近时,$f(x)$的函数值也会收缩到$A$附近.

\begin{figure}[ht]
\centering
\includegraphics[width=10cm]{./figures/FunLim_9.pdf}
\caption{在$(x_0-\delta, x_0+\delta)$区间内,$f(x)>0$. 仿自\cite{Thomas}} \label{FunLim_fig9}
\end{figure}
局部保号性的一个幼稚“证明”:如\autoref{FunLim_fig9} 所示(其实就是\autoref{FunLim_fig8} ),我们总能取一个$\varepsilon_1 \in (0,A)$,极限的定义保证了我们总能找到对应的$\delta_1$.显然,在$(x_0-\delta_1, x_0+\delta_1)$这个小去心区间\footnote{严格来说,或许应该写为$(x_0-\delta_1,x_0)\cup(x_0, x_0+\delta_1)$,不过在正文中这么写实在太繁琐了,也不利于把握重点}(即$\mathring{U} ({x_0})$)内,有$f(x)>0$.

同理,将选取$\varepsilon$的区间改为$(0,A/2)$,我们就能找到相应的、使$f(x)>A/2$的区间,即说明了局部保序性.原则上,由此可导出更广义的局部保序性,即总存在$f(x)>A/3,A/4,A/5,...$的区间.

\end{theorem}