% 随机存储器
% keys 随机存储器|随机存取存储器|内存
% license Xiao
% type Wiki

\begin{issues}
\issueDraft
\end{issues}\begin{align}
\end{align}

\subsection{随机存储器——RAM}

随机存储器(Random Access Memory, RAM)是计算机的主要存储介质之一,具有可读写的特性。其“随机存取”(Random Access)的能力意味着数据的读取或写入速度是均匀的,不依赖于数据在存储器中的物理位置。这种特性区别于例如磁带存储设备的顺序访问方式,后者的访问速度会因数据的位置不同而有很大差异。
\subsubsection{主要功能和应用}
RAM的主要功能是临时存储计算机运行中的程序和数据。它是执行中程序的“工作区”,存储着程序执行的指令和处理的数据。当您开启一个应用程序比如文字处理软件或者一个游戏时,程序的代码和所需的数据就被加载到RAM中,因为相比硬盘、SSD或其他形式的永久存储,RAM的数据访问速度更快,延迟更低。

由于其易失性的特点,RAM不适合用于长期数据存储。一旦电源断开,存储在RAM中的信息就会丢失。
\subsubsection{类型}

RAM主要分为两种类型:静态随机存取存储器(SRAM)和动态随机存取存储器(DRAM)。


\subsection{静态随机访问存储器——SRAM}

静态随机访问存储器(Static random-access memory,SRAM)其中的Static意味着:只要保持通电,存储的数据就可以恒常保持。

SRAM保持数据的方式依靠持续供电的稳定性。具体地,一个SRAM基本单元有0 和 1两个电平稳定状态。SRAM使用六个晶体管存储单元的状态构成一个存储单元来存储一个数据位,通常使用六个MOSFETs。因为不需要像DRAM那样定期刷新, SRAM的速度比DRAM快,功耗更低。但造价更高,常用于处理器的缓存。

具体地,如下图所示,SRAM基本单元由两个CMOS反相器组成。两个反相器的输入、输出交叉连接,即第一个反相器的输出连接第二个反相器的输入,第二个反相器的输出连接第一个反相器的输入。这就能实现两个反相器的输出状态的锁定、保存,即储存了1个比特的状态。

\begin{figure}[ht]
\centering
\includegraphics[width=6cm]{./figures/87070ace2adbda7b.png}
\caption{六管SRAM存储单元示意图} \label{fig_RAM_3}
\end{figure}


\subsection{动态随机访问存储器——DRAM}

动态随机访问存储器(Dynamic Random-Access Memory, DRAM)存储每位数据依赖于电容和晶体管。

动态随机存取存储器(Dynamic Random-Access Memory, DRAM)是一种常用于计算机主存的存储技术,每个数据位存储在由一个晶体管和一个电容组成的单元中。因为电容会随时间自然放电,DRAM需要周期性的电荷刷新来保持信息,这也是其名称中“动态”一词的由来。


由于其结构原因,DRAM在制造成本和存储密度方面优于静态随机访问存储器(SRAM)。这些特点使得DRAM适合作为成本效益高的大容量主存解决方案,虽然它在存取速度上不如SRAM。通常我们俗称的"内存条"一般是由DRAM制成的。

\subsubsection{内存的技术演进}
DRAM技术自诞生以来已发展多个版本,包括单数据速率SDRAM及其后续版本双数据速率DRAM(DDR、DDR2、DDR3、DDR4、DDR5)。当前,DDR4是主流的内存技术,提供比其前代DDR3更高的传输速率和更低的能耗。标准内存条主流容量为8GB或16GB,一般的主板都支持通过双通道配置可以提高处理速度和整体系统性能。


\subsection{总结}
从体系结构的角度来看,DRAM存在的意义在于弥补高速运行的CPU对数据访问的需求与低速的外部存储设备之间的差距。试想,如果没有DRAM(当然,在现实的操作系统是不允许的),CPU就要一直等待数据从外部持久设备(硬盘 )中读入CPU内,导致运行堵塞。

RAM是现代计算技术中不可或缺的组成部分,广泛应用于各种设备中,从个人计算机到服务器和移动设备等。其快速的数据处理能力对于实时运算和高效数据处理至关重要,尤其是在需要处理大量数据的服务器和大型计算机系统中。

通过不断的技术发展,RAM已经在速度、容量和功耗方面取得了显著进步,使得现代计算机能够支持高性能的应用程序和复杂的操作系统。

