% 电磁场张量
% 场张量|相对论|电磁场|张量变换|麦克斯韦张量|法拉第张量|麦克斯韦双矢量


\pentry{张量的分类\upref{CatTns}, 洛伦兹规范\upref{LoGaug}, 闵可夫斯基空间\upref{MinSpa},抽象指标\upref{AbsInd}}
我们继续使用自然单位制,令 $\mu_0=\epsilon_0=c=1$ 来简化表达。依照习惯,上下标使用希腊字母如 $\mu, \nu$ 时,取值范围为 $\{0, 1, 2, 3\}$;使用拉丁字母如 $i, j$ 时,取值范围为 $\{1, 2, 3\}$。约定闵氏时空度规为 $(-1,1,1,1)$。

一个参考系中的电磁场需要用六个实函数数来刻画,三个用来刻画电场,三个用来刻画磁场。六个实数太过复杂,我们希望寻求一种简单的方式来简化表达。把六个实数合成一个对象的方法,最直接的当然是使用一个六维向量——不过这样并不能带来实质上的简化。实践中我们使用的其实是一个反对称张量场,用它来表示电磁场。

在狭义相对论里,时空是一个线性空间,事件的时空坐标随着基的不同而不同,而不同的基就代表不同的观察者,事件的坐标分量就是观察者的测量值。

和向量一样,任何张量只有给定了空间的基,才有“坐标分量”的概念。换句话说,只有有了观察者,才有观察者的测量值。张量本身不随基的选择而改变,改变的只是坐标分量。对于电磁场张量来说,其坐标分量,或称观察者的测量值,就是电场强度和磁场强度的空间分量,一共六个实函数。

\subsection{电磁场张量的定义}
\begin{definition}{电磁场张量}
$F^{\mu\nu}=\partial^{\mu}A^{\nu}-\partial^{\nu}A^{\mu}$,其中 $\partial^{\mu}=(-\frac{\partial}{\partial t},\frac{\partial}{\partial x},\frac{\partial}{\partial y},\frac{\partial}{\partial z})$,$A^{\mu}=(\phi,A_x,A_y,A_z)$,其中 $\phi,\bvec A$ 满足洛伦兹规范\upref{LoGaug} $\frac{\partial \phi}{\partial t}+\nabla \cdot \bvec A=0$\footnote{在 SI 单位制下,洛伦兹规范应当写为 $\frac{1}{c^2}\frac{\partial \phi}{\partial t}+\nabla \cdot \bvec A=0$。而这里用的是自然单位制,将 $1/c^2$ 略去。},洛伦兹规范在这里可以简写为 $\partial_\mu A^{\mu}=0$。
\end{definition}
从上面的定义式可以看出,电磁场张量 $F^{\mu\nu}$ 是个反对称张量\footnote{“反对称”意味着 $F^{\mu\nu}=-F^{\nu\mu}$。},且具有洛伦兹协变性,即满足
\begin{equation}\label{EMFT_eq4}
\begin{aligned}
&F'^{\mu\nu}={L^\mu}_\alpha {L^\nu}_\beta F^{\alpha\beta}\\
&F'_{\mu\nu}={L_\mu}^\alpha {L_\nu}^\beta F_{\alpha\beta}
\end{aligned}
\end{equation}

电磁场张量是一个二阶张量,为了方便可以写成矩阵形式(假定 $F^{\mu\nu}$ 的第一个指标为行数,第二个指标为列数)。矩阵的元素是关于 $\phi$ 和 $\bvec A$ 的表达式,所以它反映了电磁场的性质,进一步地\autoref{EMFT_eq4} 反映了电磁场的变化规律。

根据 $\phi,\bvec A$ 的定义,我们有
\begin{equation}
\begin{aligned}
&\bvec E=-\nabla \phi-\frac{ \partial \bvec A}{\partial t}
\\
&\bvec B=\nabla\times \bvec A
\end{aligned}
\end{equation}
所以对任意的 $i=1,2,3$(不妨记 $x_1=x,x_2=y,x_3=z$)。
\begin{equation}
\begin{aligned}
&F^{0i}=-\frac{\partial A_i}{\partial t}-\frac{\partial \phi}{\partial x_i}=E_i\\
&F^{ij}=\frac{\partial A_j}{\partial i}-\frac{\partial A_i}{\partial j}=\epsilon_{ijk}B_k
\end{aligned}
\end{equation}

于是我们有下面的定义,描述了电磁场张量的坐标分量与坐标系对应的电磁场分量的联系。

\begin{definition}{电磁场张量}
一个伪黎曼流形上的电磁场是一个二阶\textbf{反对称}张量 $F^{\mu\nu}$。若在某基下其分量为 $F^{01}=E_x, F^{02}=E_y, F^{03}=E_z, F^{23}=B_x, F^{31}=B_y, F^{12}=B_z$,那么在这个基对应的观察者所观察到的电场就是 $\pmat{E_x, E_y, E_z}\Tr$,磁场就是 $\pmat{B_x, B_y, B_z}\Tr$。
电磁场张量可以写成以下矩阵形式:
\begin{equation}\label{EMFT_eq1}
F^{\mu\nu}=\pmat{
&0, &E_x, &E_y, &E_z\\
&-E_x, &0, &B_z, &-B_y\\
&-E_y, &-B_z, &0, &B_x\\
&-E_z, &B_y, &-B_x, &0
}_{\mu\nu}
\end{equation}
或者用列维—奇维塔符号\upref{LeviCi} 来表示:
\begin{equation}
F^{0 i}=-F^{i 0}=E_i,\ F^{ij}=\epsilon_{ijk}B_k
\end{equation}
\end{definition}
即假定 $F^{\mu\nu}$ 的第一个指标为行数,第二个指标为列数\footnote{为了方便,我们常借用二阶矩阵来表示电磁张量,但是要记住,这是一个不规范的表达。}。

用反对称张量来表示电磁场的方式不止一种,还可以使用以下定义的对偶张量。

\begin{definition}{电磁张量的对偶张量}
电磁场张量 $F^{\mu\nu}$ 的\textbf{对偶张量},记为 $G^{\mu\nu}$。如果 $F^{\mu\nu}$ 在某基下的分量为 $F^{01}=E_x, F^{02}=E_y, F^{03}=E_z, F^{23}=B_x, F^{31}=B_y, F^{12}=B_z$,那么有 $G^{01}=-B_x, G^{02}=-B_y, G^{03}=-B_z, G^{23}=E_x, G^{31}=E_y, G^{12}=E_z$。
对偶张量有一个简单的记法:
\begin{equation}
G_{\mu\nu}=\frac{1}{2}\epsilon_{\mu\nu\rho\sigma}F^{\rho\sigma}
\end{equation}
其中 $\epsilon_{\mu\nu\rho\sigma}$ 是一个四维反对称张量,$(\mu\nu\rho\sigma)$ 为偶排列的时候 $\epsilon$ 取 $1$,如果是奇排列则 $\epsilon$ 取 $-1$,其他情况(有重复指标的时候) $\epsilon$ 取 $0$。由于爱因斯坦求和约定,这里对重复指标 $\rho,\sigma$ 进行求和,所以要乘以系数 $\frac{1}{2}$。

也可以用列维—奇维塔符号\upref{LeviCi} 来表示:
\begin{equation}
G^{0i}=-G^{i0}=-B_i,\ G^{ij}=\epsilon_{ijk}E_k
\end{equation}
\end{definition}

对偶张量,用类似\autoref{EMFT_eq1} 的方式写出来,就是

\begin{equation}
\begin{aligned}
G^{\mu\nu}=\eta^{\mu\mu'}\eta^{\nu\nu'}G_{\mu'\nu'}=
\pmat{
&0, &-B_x, &-B_y, &-B_z\\ 
&B_x, &0, &E_z, &-E_y\\ 
&B_y, &-E_z, &0, &E_x\\
&B_z, &E_y, &-E_x, &0
}_{\mu\nu}
\end{aligned}
\end{equation}

\subsection{电磁场张量的参考系变换}
电磁场张量是二阶张量,所以它满足四维时空下的变换规律。
只需要将洛伦兹变换的矩阵形式\autoref{SRLrtz_eq6}~\upref{SRLrtz} 代入\autoref{EMFT_eq4} 就可以得到电磁场在参考系变换下的结果。

在自然单位制下,$c=1$,所以 $\bvec \beta=\bvec v/c=\bvec v$,$\gamma=1/\sqrt{1-v^2/c^2}=1/\sqrt{1-v^2}$,电磁场的变换公式为
\begin{equation}
\begin{aligned}
\bvec E'=\gamma(\bvec E+\bvec v\times \bvec B)-\frac{\gamma^2}{1+\gamma}(\bvec v \cdot \bvec E)\bvec v\\
\bvec B'=\gamma(\bvec B-\bvec v\times \bvec E)-\frac{\gamma^2}{1+\gamma}(\bvec v \cdot \bvec B)\bvec v
\end{aligned}
\end{equation}
这里我们先推导电场的变换公式。根据电磁场张量的定义和变换公式, $E'_i=F'^{0i}={L^0}_\mu {L^i}_{\nu} F^{\mu\nu}$,所以
\begin{equation}
\begin{aligned}
E'_i&={L^0}_\mu {L^i}_\nu F^{\mu\nu}\\
&={L^0}_0 {L^i}_j F^{0j}+{L^0}_j {L^i}_0 F^{j0} + {L^0}_j {L^i}_k F^{jk} \\
&=\gamma[\delta_{ij}+(\gamma-1) v_iv_j/v^2]E_j+\gamma v_j\gamma v_i  (-E_j)\\
&\ \ \ -\gamma v_j [\delta_{ik}+(\gamma-1)v_i v_k/v^2]\epsilon_{jkl}B_l\\
&=\gamma E_i+(\gamma^2-\gamma)\frac{v_iv_jE_j}{v^2}-\gamma^2v_iv_jE_j-\gamma v_j\epsilon_{jil}B_l -(\gamma^2-\gamma)\frac{v_iB_l}{v^2}\epsilon_{jkl}v_jv_k \\
&=\gamma E_i+(1-\gamma)\frac{v_iv_jE_j}{v^2}-\gamma\epsilon_{ijk}B_jv_k
\end{aligned}
\end{equation}
改写成矢量形式,就可以得到电场的变换公式
\begin{equation}
\begin{aligned}
\bvec E'&=\gamma\bvec E+\frac{1-\gamma}{v^2}\bvec v(\bvec v\cdot \bvec E)-\gamma \bvec B\times \bvec v\\
&=\gamma(\bvec E+\bvec v\times \bvec B)+\frac{1-\gamma}{1-1/\gamma^2}\bvec v(\bvec v \cdot \bvec E)\\
&=\gamma(\bvec E+\bvec v\times \bvec B)-\frac{\gamma^2}{1+\gamma}(\bvec v\cdot \bvec E)\bvec v
\end{aligned} 
\end{equation}
如果想得到磁场的变换公式,根据麦克斯韦方程组的对称性,我们只需要将上面的 $\bvec E$ 替换为 $\bvec B$,$\bvec B$ 替换为 $-\bvec E$,就可以得到
\begin{equation}
\bvec B'=\gamma (\bvec B-\bvec v\times \bvec E)-\frac{\gamma^2}{1+\gamma}(\bvec v\times B)\bvec v
\end{equation}
\subsection{闵可夫斯基时空中的电动力学}

\subsubsection{场源的描述}

\begin{definition}{电流密度4-矢量}
在闵可夫斯基空间中,电流密度被表示为一个4-矢量 $J^\mu$。如果在某观察者看来,电荷密度的分布是 $\rho$,而电流密度的分布是 $\pmat{J_x, J_y, J_z}\Tr$,那么在这个观察者的坐标系下,电流密度4-矢量的坐标就是 $\pmat{\rho, J_x, J_y, J_z}$。
\end{definition}

电流密度 4-矢量应当满足连续性条件(或者说电荷守恒):
\begin{equation}
\nabla\cdot\bvec{J}=-\frac{\partial\rho}{\partial t}
\end{equation}
即
\begin{equation}
\frac{\partial}{\partial x^i}J^i+\frac{\partial}{\partial t}\rho=0
\end{equation}
或者可以写成协变形式:
\begin{equation}
\partial_\mu J^\mu=0
\end{equation}
\subsubsection{四维形式的麦克斯韦方程组}

利用四维形式的电磁场张量和电流密度4-矢量,我们可以将麦克斯韦方程组表示为以下两个方程
\begin{equation}\label{EMFT_eq2}
\leftgroup{
    &\partial_\mu F^{\mu\nu}=-4\pi J^\nu\\
    &\partial^\mu G_{\mu\nu}=0
}
\end{equation}

注意\autoref{EMFT_eq2} 中的真指标是 $\nu$,赝指标是 $\mu$(根据爱因斯坦求和约定,对重复指标进行求和)。第二个方程也可以用 $\partial^\mu F^{\nu\rho}+\partial^\nu F^{\rho\mu}+\partial^\rho F^{\mu\nu}=0$ 代替。

\autoref{EMFT_eq2} 的第一个方程是一个四维矢量方程,可以拆成一个标量方程和一个三维矢量方程,其中标量方程就是 $\nabla\cdot\bvec{E}=-4\pi \rho$,而矢量方程就是 $\nabla\times\bvec{B}=4\pi \bvec{J}+\partial\bvec{E}/\partial t$。 

类似地,\autoref{EMFT_eq2} 的第二个方程分别代表了 $\nabla\cdot\bvec{B}=0$ 和 $\nabla\times\bvec{E}=-\partial\bvec{B}/\partial t$。

详细推导:
\begin{equation}
\begin{aligned}
&\partial_\mu F^{\mu 0}=-4\pi J^0=-4\pi \rho\\
\Rightarrow &\nabla \cdot \bvec E=4\pi \rho 
\end{aligned}
\end{equation}

\begin{equation}
\begin{aligned}
&\partial_\mu F^{\mu i}=-4\pi J^i=-4\pi \bvec J_i\\
\Rightarrow & \frac{\partial E_i}{\partial t} +\partial_k\epsilon_{kij}B_j=-4\pi J_i\\
\Rightarrow &\frac{\partial \bvec E}{\partial t}-\nabla\times \bvec B=-4\pi \bvec J\\
\Rightarrow &\nabla\times \bvec B=4\pi \bvec J+\frac{\partial \bvec E}{\partial t}
\end{aligned}
\end{equation}

\begin{equation}
\begin{aligned}
&\partial^\mu G_{\mu 0} = 0\\
\Rightarrow &\nabla \cdot \bvec B = 0
\end{aligned}
\end{equation}

\begin{equation}
\begin{aligned}
&\partial^\mu G_{\mu i} = 0\\
\Rightarrow &-\frac{\partial B_i}{\partial t}+\partial_k \epsilon_{kij} \bvec E_j = 0\\
\Rightarrow &\nabla \times \bvec E=-\frac{\partial \bvec B}{\partial t}
\end{aligned}
\end{equation}
\subsubsection{电荷受力}

对于一个带电荷 $q$ 的粒子,如果它在某参考系下的4-速度为 $U^\mu=(\gamma,\gamma v_x,\gamma v_y,\gamma v_z)$,那么在电磁场 $F^{\mu\nu}$ 中,该电荷受到的\textbf{闵可夫斯基力}为
\begin{equation}
K^{\mu}=q {F^{\mu}}_\nu U^\nu
\end{equation}
注意,其中 $U_\mu=\eta_{\mu\nu}U^\nu$,而 $\eta_{\mu\nu}$ 是闵可夫斯基度规。换句话说,如果粒子在该参考系下的3-速度为 $\pmat{v_x, v_y, v_z}\Tr$,那么有 $U_\mu=\gamma\pmat{-1, v_x, v_y, v_z}$。$K^\mu$ 是一个四矢量,也具有洛伦兹协变性,事实上它正比于4-速度关于原时的导数,比例系数为 $m$,即粒子的静质量。
\begin{equation}\label{EMFT_eq3}
m\frac{\dd U^\mu}{\dd \tau}=K^\mu=q {F^{\mu}}_\nu U^\nu
\end{equation}

我们把\autoref{EMFT_eq3} 分为一个标量方程和一个三维向量方程。

标量方程展开来就是
\begin{equation}
\begin{aligned}
m\frac{\dd \gamma}{\dd\tau}&=q \bvec{E}\cdot \gamma\bvec{v}\\
\Rightarrow\frac{\dd (\gamma m)}{\dd \tau}&=\gamma q \bvec v\cdot \bvec E\\
\Rightarrow \frac{\dd (\gamma m)}{\dd t}&=q\bvec v\cdot \bvec E
\end{aligned}
\end{equation}
注意到 $\gamma m$ 是粒子的\textbf{动质量},$q \bvec E$ 对应的是电场对粒子做的功(磁场对粒子不做功)。所以该方程对应经典物理中的\textbf{功能原理}:$\dd E/\dd t=\bvec{v}\cdot\bvec{F}$。

向量方程展开则得到的是
\begin{equation}
\begin{aligned}
&m\frac{\dd (\gamma v_i)}{\dd \tau} =K^i=q{F^i}_0U^0+q{F^i}_jU^j=q\gamma E_i+q\epsilon_{ijk} \gamma v_j B_k
\\
\Rightarrow &\frac{\dd P_i}{\dd \tau}=\bvec{K}=q\gamma(\bvec{E}+\bvec{v}\times\bvec{B})\\
\Rightarrow &\frac{\dd P_i}{\dd t}=\bvec{F}=q(\bvec{E}+\bvec{v}\times\bvec{B})
\end{aligned}
\end{equation}
它对应经典物理中的\textbf{动量原理},其中力是洛伦兹力。




