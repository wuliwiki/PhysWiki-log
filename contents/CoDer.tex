% 协变导数
% keys Covariant Derivative|微分几何|differential geometry|christoffel symbol|克氏符|流形
% license Xiao
% type Tutor

\addTODO{应当移动到\enref{仿射联络}{affcon}条目下}


\pentry{仿射联络(切丛)\nref{nod_affcon}}{nod_aba9}

定义联络时,我们讲联络看成是向量场之间的映射,$\nabla_{\bvec{X}}\bvec{Y}$ 中的 $\bvec{X}$ 和 $\bvec{Y}$ 都是光滑向量场。也就是说,我们着眼于场的整体,而没有关注局部的性质,比如某个点或者某个轨迹上的切向量场如何变化。

不过,在\enref{Christoffel符号}{CrstfS}文章中我们看到,在具体图中计算联络时,只用到了几个定义在欧几里得空间的函数,也就是向量场的坐标值函数和Christoffel符号。欧几里得空间上函数的求导是可以考虑局部的,也就是我们可以计算一个点上函数的导数,而不必像联络的定义那样要考虑整个场的变换。这就意味着我们有可能局部地计算联络。

\subsection{协变导数的概念}

考虑一个带联络的 $n$ 维实流形 $(M, \nabla)$。令 $c:I\to M$ 是一个从区间 $[0, 1]$ 到流形 $M$ 上的连续映射,我们称之为一条\textbf{道路}。沿着 $c(t)|_{t\in I}$ 定义一个光滑切向量场 $\bvec{X}(t)$。也就是说,各 $\bvec{X}(t)$ 都是 $c(t)$ 处的切向量,但不一定是沿着 $c'(t)$ 方向的\footnote{当然,我们可以把 $\opn{Im}c(t)$ 本身看成一个一维的流形,那么此时 $\bvec{X}$ 就不是其切向量。我们可以考虑在这个一维流形上的一个二维\enref{向量丛}{TanBun},这样 $\bvec{X}$ 就是这个丛上的一个截面。}。

比如说,令 $M$ 二维球面 $S^2$,嵌入为三维欧几里得空间中圆心在原点的单位球面。取 $c(t)=\pmat{\cos t&\sin t&0}\Tr$。如果令 $\bvec{X}(t)=\pmat{0&0&\E^t}$,那么它处处是 $S^2$ 上的切向量,且沿着 $c(t)$ 的各坐标分量都是关于 $t$ 的光滑函数,因此是沿着 $c(t)$ 的光滑向量场,但它和 $c'(t)$ 处处都不平行。

% 现在我们尝试导出 $\dd\bvec{X}(t)/\dd t$。

% 任取一个图 $\phi:\mathbb{R}^n\to M$,记 $\Im\phi=U$,使得 $\forall t\in I, c(t)\in U$。在给定图中,$\phi^{-1}(\bvec{X}(t))$ 表示为坐标 $\pmat{x^1(t)&x^2(t)&\cdots&x^n(t)}\Tr$,点 $c(t)$ 表示为坐标 $\pmat{c^1(t)&c^2(t)&\cdots&c^n(t)}\Tr$\footnote{即 $\phi\qty(\pmat{x^1(t)&x^2(t)&\cdots&x^n(t)}\Tr)=\bvec{X}(t)$,$\phi\qty(\pmat{c^1(t)&c^2(t)&\cdots&c^n(t)}\Tr)=c(t)$。}。特别地,为简化考虑,令 $c(0)=\pmat{0&0&\cdots&0}\Tr$。

% 记 $c'(t)=\frac{\dd}{\dd t}c(t)\in T_{c(t)}M$,即 $c'(t)$ 是道路 $c(t)$ 所代表的切向量。于是,道路 $c(t)$ 各处的切向量 $c'(t)$ 就可以表示为 $\pmat{\frac{\dd}{\dd t}c^1(t)&\frac{\dd}{\dd t}c^2(t)&\cdots&\frac{\dd}{\dd t}c^n(t)}\Tr$。

% 根据\autoref{eq_CrstfS_4},誊抄如下:
% \begin{equation}
% \nabla_{x^i}y^j=x^i(\partial_iy^s)+x^iy^j\Gamma^s_{ij}
% \end{equation}
% 我们可以写出
% \begin{equation}\label{eq_CoDer_1}
% \nabla_{\frac{\dd}{\dd t}c^i(t)}x^j(t)=\frac{\dd}{\dd t}c^i(t)\qty(\partial_ix^s(t)+x^j(t)\Gamma^s_{ij})
% \end{equation}

% \autoref{eq_CoDer_1} 右边是只有一个上标 $s$ 的量,可以用来表示图 $\phi^{-1}(U)$ 上的向量。问题是,它对应 $M$ 上的切向量吗?


% 观察下式:
% \begin{equation}\label{eq_CoDer_2}
% \frac{\dd x^i}{\dd t}\partial_i+x^i\nabla_{c'(t)}\partial_i
% \end{equation}
% 其中 $\partial_i$ 是 $\phi^{-1}(U)$ 上的偏微分算子。由于 $\paritial_i$ 和 $\nabla_{c'(t)}\partial_i$

% 考虑到 $\phi^{-1}(c'(t))=\pmat{\frac{\dd}{\dd t}c^1(t)&\frac{\dd}{\dd t}c^2(t)&\cdots&\frac{\dd}{\dd t}c^n(t)}\Tr$,可知
% \begin{equation}
% c'(t)=\frac{\dd}{\dd t}c^1(t)\partial_1+\frac{\dd}{\dd t}c^2(t)\partial_2+\cdots+\frac{\dd}{\dd t}c^n(t)\partial_n
% \end{equation}
% 因此
% \begin{equation}\label{eq_CoDer_3}
% \begin{aligned}
% \nabla_{c'(t)}\partial_i&=\nabla_{\frac{\dd}{\dd t}c^j(t)\partial_j}\partial_i\\
% &=\frac{\dd}{\dd t}c^j(t)\nabla_{\partial_j}\partial_i\\
% &=\frac{\dd}{\dd t}c^j(t)\Gamma^k_{ji}\partial_k
% \end{aligned}
% \end{equation}

% 将\autoref{eq_CoDer_3} 代入\autoref{eq_CoDer_2},可见\autoref{eq_CoDer_2} 在 $\phi^{-1}$ 上的坐标为

如果 $M$ 是\textbf{欧几里得空间},那么我们已经知道该怎么求 $\frac{\dd}{\dd t}\bvec{X}(t)$ 了,因为 $\bvec{X}(t)$ 可以自然地表示成坐标形式,我们对每个坐标求导就可以了。具体来说,如果 $\bvec{X}(t)=x^i(t)\partial_i$,那么 $\frac{\dd}{\dd t}\bvec{X}(t)=\dot{x}^i(t)\partial_i$。

欧几里得空间中,沿给定道路求导的过程满足以下性质:
\begin{enumerate}
\item $\frac{\dd }{\dd t}\bvec{X}$ 对 $\bvec{X}$ 满足 $\mathbb{R}-$ 线性性\footnote{即如果在道路上定义了两个向量场 $\bvec{X}(t)$ 和 $\bvec{Y}(t)$,那么任取实数 $a, b$,都有 $\frac{\dd}{\dd t}(\bvec{a\bvec{X}+b\bvec{Y}})=a\frac{\dd}{\dd t}\bvec{X}+b\frac{\dd}{\dd t}\bvec{Y}$。}。
\item 任取 $M=$ 上的光滑函数 $f$,则
\begin{equation}
\frac{\dd}{\dd t}\qty(f\bvec{X})=f\frac{\dd}{\dd t}\bvec{X}+\bvec{X}\frac{\dd}{\dd t}f~.
\end{equation}
\item 如果 $\bvec{X}$ 是 $\mathbb{R}^n$ 上光滑切向量场 $\tilde{\bvec{X}}$ 的一部分,且记 $D$ 是 $\mathbb{R}^n$ 上的方向导数,那么有
\begin{equation}\label{eq_CoDer_4}
\frac{\dd}{\dd t}\bvec{X}=D_{c'(t)}\tilde{\bvec{X}}~.
\end{equation}

\end{enumerate}

\autoref{eq_CoDer_4} 很重要,它意味着至少在欧几里得空间中,沿着道路对切向量求导,可以由联络(方向导数)导出。顺着这个思想,我们可以尝试将沿道路求导的概念拓展到任意的流形 $M$ 上。 

\subsubsection{切向量场从局部拓展到整体}

在 $M$ 上局部区域定义的光滑切向量场,比如道路 $c(t)$ 上的 $\bvec{X}(t)$,能不能拓展到整个流形 $M$ 呢?也就是说,是否存在一个 $M$ 上的光滑切向量场 $\widetilde{\bvec{X}}$,使得 $\bvec{X}(t)=\widetilde{\bvec{X}}|_{c(t)}$ 呢?

答案是肯定的。这是因为任意 $n$ 维实流形 $M$,都可以嵌入到 $\mathbb{R}^{2n}$ 上。换句话说,存在一个 $\mathbb{R}^{2n}$ 上的光滑函数 $f$,使得 $\mathbb{R}^{2n}$ 的子流形 $\{p\in\mathbb{R}^{2n}|f(p)=0\}$,和 $M$ 微分同胚(或者说就是 $M$ 本身)。使用 $\mathbb{R}^{2n}$ 天然的坐标,可以把任何切向量场表示成 $\mathbb{R}^{2n}$ 上光滑函数的组合\footnote{比如三维欧几里得空间中的切向量场,可以表示为三个光滑函数的组合,每个光滑函数是一个坐标分量。}。我们总可以把局部的光滑函数拓展为整个 $\mathbb{R}^{2n}$ 上的光滑函数,从而把局部的光滑切向量场拓展为整个 $\mathbb{R}^{2n}$ 上的光滑切向量场。对这个拓展的切场取在 $M$ 上的限制,再投影到 $M$ 上,即得到在 $M$ 上的拓展。
\addTODO{投影是为了得到 $M$ 上的切向量。}

\begin{example}{切向量的整体拓展}
令 $M$ 为三维欧几里得空间中的单位球 $M=\{\pmat{x&y&z}\Tr|x^2+y^2+z^2=0\}$。在 $M$ 的“赤道”$\{\pmat{x&y&z}|x^2+y^2=1, z=0\}$ 上定义了一个光滑切向量场:
\begin{equation}
\bvec{V}\pmat{x&y&z}=\pmat{-y&x&0}~.
\end{equation}

在 $\mathbb{R}^3$ 上定义
\begin{equation}
\widehat{\bvec{V}}\pmat{x&y&z}=\pmat{-y&x&0}~,
\end{equation}

那么显然 $\bvec{V}$ 是 $\widehat{\bvec{V}}$ 的限制。

但是对于单位球 $M$ 上的各点 $p=\pmat{x&y&z}\Tr$,$\widehat{\bvec{V}}(p)$ 却不一定是 $T_pM$ 中的切向量。这时候我们就需要投影:
\begin{equation}
\widetilde{\bvec{V}}=\widehat{\bvec{V}}-\widehat{\bvec{V}}\vdot\bvec{N}~.
\end{equation}
其中 $\bvec{N}\pmat{x&y&z}\Tr=\pmat{x&y&z}\Tr$,是单位球在 $p$ 处的单位法向量。

这样就能保证 $\widetilde{\bvec{V}}|_p$ 一定是 $T_pM$ 的元素了。自此,我们就得到了 $\bvec{V}$ 在整个 $M$ 上的拓展:$\widetilde{\bvec{V}}|_{M}$。

\end{example}

\subsubsection{协变导数}

继续之前的讨论,令 $\bvec{X}(t)$ 是沿着 $c(t)$ 的光滑切向量场。将 $\bvec{X}(t)$ 拓展为整个 $M$ 上的光滑切场 $\widetilde{\bvec{X}}$,$c'(t)$ 拓展为 $\bvec{T}$。

$\nabla_{\bvec{T}}\widetilde{\bvec{X}}$ 是可以计算出来的。考虑到欧几里得空间中沿道路求导的第3条性质,也就是和方向导数(联络)的相容性,我们可以沿着 $c(t)$ 定义一个算子 $\frac{D}{\dd t}$,使得:
\begin{equation}
\frac{D}{\dd t}\bvec{X}(t)=\nabla_{\bvec{T}}\widetilde{\bvec{X}}|_{c(t)}~.
\end{equation}

我们称如上定义的 $\frac{D}{\dd t}\bvec{X}(t)$ 为 $\bvec{X}$ 沿着道路 $c(t)$ 的\textbf{协变导数(covariant derivative)}。

欧几里得空间上沿着 $c(t)$ 求 $\bvec{X}$ 的方向导数,就是协变导数的一个特例。一般的协变导数,和这个特例一样,有三个性质:

\begin{enumerate}
\item $\frac{D}{\dd t}\bvec{X}$ 对 $\bvec{X}$ 满足 $\mathbb{R}-$ 线性性\footnote{即如果在道路上定义了两个向量场 $\bvec{X}(t)$ 和 $\bvec{Y}(t)$,那么任取实数 $a, b$,都有 $\frac{D}{\dd t}(\bvec{a\bvec{X}+b\bvec{Y}})=a\frac{D}{\dd t}\bvec{X}+b\frac{D}{\dd t}\bvec{Y}$。}。
\item 任取 $M=$ 上的光滑函数 $f$,则
\begin{equation}
\frac{D}{\dd t}\qty(f\bvec{X})=f\frac{D}{\dd t}\bvec{X}+\bvec{X}\frac{D}{\dd t}f~.
\end{equation}
\item 如果 $\bvec{X}$ 是 $\mathbb{R}^n$ 上光滑切向量场 $\tilde{\bvec{X}}$ 的一部分,且记 $D$ 是 $\mathbb{R}^n$ 上的方向导数,那么有
\begin{equation}
\frac{D}{\dd t}\bvec{X}=D_{c'(t)}\tilde{\bvec{X}}~.
\end{equation}

\end{enumerate}

在Loring W. Tu的课本\textsl{Differential Geometry: Connections, Curvature, and Characteristic Classes}\cite{GTM275}的第13.1节,导出协变导数的思路是先给出以上三个性质,然后证明存在且唯一存在满足这三条性质的算子,将其定义为 $\frac{D}{\dd t}$。如果你使用GTM 275作为参考书,请注意这里思路的差别。



\subsection{协变导数的计算}

取 $M$ 上的一个局部坐标系\footnote{$M$ 上并非总有整体坐标系,比如说单位球面上就不可能存在处处非零的光滑切向量场,进而任何一组光滑切向量场都会有零点,进而任何一组光滑切场都不可能是整体坐标系。但是局部是可以的,因为流形是局部同胚于欧几里得空间的,取欧几里得空间里的坐标系,映射回流形上就可以。}$\{\bvec{e}_1, \bvec{e}_2, \cdots, \bvec{e}_n\}$。给定沿着 $c(t)$ 的一个光滑切向量场 $\bvec{X}=x^i(t)\bvec{e}_i$,其中各 $x^i(t)$ 是区间 $I$ 上的光滑函数。将光滑切场 $\bvec{e}_i$ 拓展为光滑切场 $\widetilde{\bvec{e}}_i$,光滑函数 $x^i$ 拓展为光滑函数 $\widetilde{x}^i$\footnote{即将 $\bvec{X}$ 拓展为 $\widetilde{x}^i\widetilde{\bvec{e}}_i$。}。此时再将 $c'(t)$ 拓展为整个 $M$ 上的光滑切场 $\bvec{T}=T^i\widetilde{\bvec{e}}_i$。

则按定义有:

\begin{equation}\label{eq_CoDer_5}
\begin{aligned}
 \frac{D}{\dd{t}} \bvec{X}(t) &= \nabla_{\bvec{T}} \left(\widetilde{x}^i \widetilde
 {\bvec{e}}_i\right) |_{c(t)} \\
&=\widetilde{x}^i\nabla_{T^j\widetilde{\bvec{e}}_j}\left(\widetilde{\bvec{e}}_i\right) |_{c(t)} + (\bvec{T}\widetilde{x^i})\widetilde{\bvec{e}}_i|_{c(t)}\\
&=T^j\widetilde{x}^i\nabla_{\widetilde{\bvec{e}}_j}\left(\widetilde{\bvec{e}}_i\right)|_{c(t)}+\frac{\dd x^i(t)}{\dd t}\bvec{e}_i~.
\end{aligned}
\end{equation}

按照GTM 275中式(13.1)的写法,有
\begin{equation}
\frac{D}{\dd t}\bvec{X}(t)=\frac{\dd x^i(t)}{\dd t}\bvec{e}_i+{x}^i(t)\nabla_{\bvec{T}}\bvec{e}_i~
\end{equation}

和\autoref{eq_CoDer_5} 是一样的。

将\autoref{eq_CoDer_5} 与\enref{Christoffel符号}{CrstfS}的概念结合,我们还可以进一步写出给定图中协变导数的计算公式:

\begin{equation}\label{eq_CoDer_6}
\frac{D}{\dd t}\bvec{X}(t)=T^jx^i\Gamma^k_{ji}+\frac{\dd x^k(t)}{\dd t}~.
\end{equation}
注意\autoref{eq_CoDer_6} 只写了坐标,省略了 $\bvec{e}_i$。

这样,只要知道了一个图中的Christoffel符号,就可以用\autoref{eq_CoDer_6} 在这个图中计算出协变导数了。

\subsection{与度量的相容性}

相关概念参见\enref{黎曼联络}{RieCon}。

\begin{theorem}{}
给定\textbf{黎曼流形}$M$ 上的一条道路 $c(t)$,沿着道路有两个光滑切场 ${\bvec{X}}(t)$ 和 ${\bvec{Y}}(t)$。那么内积 $g({\bvec{X}}(t), {\bvec{Y}}(t))$ 就是 $c(t)$ 上的光滑函数,因此可以直接对 $t$ 求导。

此时我们有:
\begin{equation}
\frac{\dd}{\dd t}g({\bvec{X}}(t), {\bvec{Y}}(t))=g(\frac{D}{\dd t}{\bvec{X}}(t), {\bvec{Y}}(t))+g({\bvec{X}}(t), \frac{D}{\dd t}{\bvec{Y}}(t))~.
\end{equation}
\end{theorem}



这是由“度量相容性(见\autoref{def_RieCon_1} 的第二条)”推论得出的。

\subsection{张量场的协变导数}
\subsubsection{定义}
设$TM$是$M$上的切丛,利用切丛上的联络$\nabla$,我们可以诱导作用在张量场上的协变导数。
\begin{theorem}{}
设$M$上的仿射联络为$\nabla$;$X$与$Y$是光滑切场。
\begin{enumerate}
\item 如果$\omega$是$M$上的光滑1形式,那么
\begin{equation}\label{eq_CoDer_9}
\left(\nabla_X \omega\right)(Y):=X(\omega(Y))-\omega\left(\nabla_X Y\right)~,
\end{equation}
可以证明$\nabla_X \omega$也是光滑1形式。
\end{enumerate}
\end{theorem}
若$T$是光滑的$(a,b)$型张量场,那么
\begin{equation}\label{eq_CoDer_7}
\begin{aligned}
\left(\nabla_X T\right)\left(\omega_1, \ldots, \omega_a, Y_1, \ldots, Y_b\right):= & X\left(T\left(\omega_1, \ldots, \omega_a, Y_1, \ldots, Y_b\right)\right) \\
& -\sum_{i=1}^a T\left(\omega_1, \ldots, \nabla_X \omega_i, \ldots \omega_a, Y_1, \ldots, Y_b\right) \\
& -\sum_{j=1}^b T\left(\omega_1, \ldots, \omega_a, Y_1, \ldots, \nabla_X Y_j, \ldots Y_b\right)~,
\end{aligned}
\end{equation}
可以证明$\nabla_X T$也是光滑的$(a,b)$型张量场。

上述定义显然是为了呼应“仿射联络把光滑切场映射为光滑切场”。接下来我们证明这两种定义的合理性。

\textbf{证明:}

从定义来看,联络对张量场的作用结果必然是光滑的。要证明结果依然是张量场,只需要证明对括号内的切场(余切场)满足光滑函数线性。

设$f$为光滑函数,我们有

\begin{equation}
\begin{aligned}
\nabla_X\omega(fY)&=X(\omega(fY))-\omega(\nabla_XfY)\\
& =(X f) \omega(Y)+f X(\omega(Y))-\omega((X f) Y)-f \omega\left(\nabla_X Y\right) \\
& =f\left(\nabla_X \omega\right)(Y)~.
\end{aligned}
\end{equation}
同理可证,用$f\omega_i$替换\autoref{eq_CoDer_7} 中的$\omega_i$,右式第一项为
\begin{equation}\label{eq_CoDer_8}
\begin{aligned}
X\left(T\left(\omega_1, \ldots, f\omega_i,\ldots,\omega_a, Y_1, \ldots, Y_b\right)\right)&=(Xf)\left(T\left(\omega_1, \ldots,\omega_a, Y_1, \ldots, Y_b\right)\right)\\&+fX\left(T\left(\omega_1, \ldots,\omega_a, Y_1, \ldots, Y_b\right)\right)~.
\end{aligned}
\end{equation}
第三项不变,第二项利用莱布尼兹律展开,其中一项与\autoref{eq_CoDer_8} 右式第一项抵消,然后得到对任意切场是光滑函数线性的性质。对余切场亦同理可证。于是我们便证明了,协变导数对光滑张量场作用,结果依然是光滑张量场。
\subsubsection{坐标表示}
从定义式\autoref{eq_CoDer_7} 可知 ,只要分别求出联络对切向(协变矢量),余切向量(逆变矢量)basis作用的表达式,代入此式后我们便能得到对张量场求协变导数的坐标表示。在该节中,我们对上下标采取爱因斯坦求和约定。

Christoffel符号的定义式已表明,对于任意切矢量$V$我们有
\begin{equation}
\begin{aligned}
\nabla_{\partial_i}\partial_j&=\Gamma^k_{ij}\p artial_k\\
\nabla_{\partial_i}V&=(\nabla_{\partial_i} V)^j\partial_j\\
&=(\partial_iV^j+\Gamma^j_{ik}V^k)\partial_j
\end{aligned}~.
\end{equation}
所以我们只需要求出对余切向量求协变导数的坐标表示即可。为表示方便,接下来我们用$\nabla_i$表示$\nabla_{\partial_i}$。

光滑1形式是余切向量的线性组合,设$\omega=\omega_i\theta^i$,现在我们利用\autoref{eq_CoDer_9} ,求出$(\nabla_i\omega)_j$即可。用$\partial_j$代替$Y$,右式为
\begin{equation}
\begin{aligned}
X(\omega(Y))-\omega\left(\nabla_X Y\right)&=\nabla_i[\omega(\partial_j)]-\omega[\nabla_i\partial_j]\\
&=\partial_i\omega_j-\omega_k\Gamma^k_{ij}~.
\end{aligned}
\end{equation}
因此
\begin{equation}
\begin{aligned}
(\nabla_i\omega)_j&=\partial_i\omega_j-\omega_k\Gamma^k_{ij}\\
\nabla_i\theta^j&=-\Gamma^j_{ik}\theta^k~.
\end{aligned}
\end{equation}
现在我们欲求$(\nabla_iT)^{r_1r_2...r_a}_{s_1s_2...s_b}$,利用定义得
\begin{equation}
\begin{aligned}
(\nabla_i T)^{r_1 r_2 \ldots r_a}_{s_1 s_2 \ldots s_b} &= (\nabla_i T)(\theta^{r_1}, \theta^{r_2}, \ldots, \theta^{r_a}, \partial_{s_1}, \partial_{s_2}, \ldots, \partial_{s_b}) \\
&= \partial_i [T(\theta^{r_1}, \theta^{r_2}, \ldots, \theta^{r_a}, \partial_{s_1}, \partial_{s_2}, \ldots, \partial_{s_b}) ] \\
&\quad - \sum_{r_k=r_1}^{r_a} T(\theta^{r_1}, \ldots, \nabla_i \theta^{r_k}, \ldots, \theta^{r_a}, \partial_{s_1}, \partial_{s_2}, \ldots, \partial_{s_b}) \\
&\quad - \sum_{s_j=s_1}^{s_b} T(\theta^{r_1}, \theta^{r_2}, \ldots, \theta^{r_a}, \partial_{s_1}, \ldots, \nabla_i \partial_{s_j}, \ldots, \partial_{s_b})\\
&=\partial_iT^{r_1 r_2 \ldots r_a}_{s_1 s_2 \ldots s_b}\\
&\quad+\Gamma_{i \sigma}^{r_1} T^{\sigma r_2 \ldots r_a}{ }_{s_1 \ldots s_b}+\ldots+\Gamma_{i \sigma}^{r_a} T^{r_1 \ldots r_{a-1} \sigma}{ }_{s_1 \ldots s_b}\\
&\quad-\Gamma^\sigma _{i s_1} T^{r_1 \ldots r_a}{ }_{\sigma s_2 \ldots s_b}-\ldots-\Gamma_{i s_b}^\sigma T^{r_1 \ldots r_a}{ }_{s_1 \ldots s_{b-1} \sigma}~.
\end{aligned}
\end{equation}









