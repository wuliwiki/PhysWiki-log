% QR 分解
% license Xiao
% type Tutor

\pentry{正交矩阵、酉矩阵\upref{UniMat}, 矩阵的秩\upref{MatRnk}}{nod_860a}

\footnote{参考 Wikipedia \href{https://en.wikipedia.org/wiki/QR_decomposition}{相关页面}。}一个复数矩阵 $\mat A$ 可以分解为一个酉矩阵 $\mat Q$ 和上半三角矩阵 $\mat R$ 的乘积
\begin{equation}
\mat A = \mat Q\mat R~.
\end{equation}
该分解叫做 \textbf{QR 分解(QR decomposition)}, 通常用于求解线性方程组。 用于解本征方程的 QR 算法也基于此。

在 QR 分解中, 如果 $\mat A$ 的秩为 $R$, 那么 $\mat Q$ 的前 $R$ 列张成的空间和 $\mat A$ 的各列张成的空间相同。
