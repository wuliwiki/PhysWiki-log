% 稠密性与完备性
% keys 序|稠密|完备
% license Usr
% type Tutor

\begin{issues}
\issueDraft
\issueTODO
\end{issues}

\pentry{连续性\nref{nod_Topo3}, 紧致性\nref{nod_Topo2}}{nod_e550}

我们从最常见的两个全序集开始,即$(\mathbb{Q};\leq),(\mathbb{R};\leq)$。我们知道它们可以表示在数轴上,如下图。

\addTODO{缺少图。}

直观上,有理数集密集在数轴上,实数集则填满了整个数轴。在数学上,我们分别称之为稠密性和完备性。

\subsection{(序)稠密性}

如何严格定义稠密性呢?不妨以自然数和有理数为例。

我们说自然数集是离散的,因为两个相邻的自然数之间不能有其他自然数,所以是有“空隙”的。比如不存在一个自然数既大于3又小于4。从数轴上看,自然数集是一系列零散的点。

那么对于有理数呢?有理数集应该是稠密的,这符合我们的直观。当我们想进行类似于自然数的讨论时,一个问题就出现了:不存在两个相邻的有理数。比如对于0和1,$\frac{1}{2}$就在它们之间,因此0和1不相邻。实际上,对任何两个有理数都是如此。仔细一想,这不就反映了稠密的本质吗?

因此,我们有如下的定义。

\begin{definition}{(序)稠密性}
如果偏序集$(A;\leq)$满足对任意$a,b\in A$,存在$c\in A$使$a\leq c\leq b$,则称$(A;\leq)$是\textbf{(序)稠密的}。

如果偏序集$(A;\leq)$的一个非空子集$B$满足对任意$a,b\in A$,且$a\neq b$,存在$c\in B$使$a\leq c\leq b$,则称$B$在$A$中稠密,或称$B$是$A$的\textbf{稠密子集}。
\end{definition}

\begin{example}{}
$(\mathbb{Q};\leq),(\mathbb{R};\leq)$是稠密的,且$\mathbb{Q}$是$\mathbb{R}$的一个稠密子集。
\end{example}

注意对于$\mathbb{Q}$和$\mathbb{R}$而言,当$a\leq b$时总有$a\leq\frac{a+b}{2}\leq b$。

此外由实数的阿基米德性可推出$\mathbb{Q}$在$\mathbb{R}$中稠密。\footnote{实数的阿基米德性是指对任意正实数$a,b$,总存在正整数$n$使$na>b$.}

\begin{example}{}
$\mathbb{R}\times\mathbb{R}$按字典序稠密。一般地,对于两稠密的偏序集,其积按字典序稠密。
\end{example}

\begin{example}{}
$\mathbb{C}$按如下的偏序稠密:

$a+b\I\leq c+d\I\Longleftrightarrow a\leq c,b\leq d.$

通常我们只讨论全序集的稠密性。
\end{example}

\subsection{上、下界与上、下确界}

实际上,有理数与实数集是有区别。如果仔细思考,会发现有理数也是有“空隙”的,比如,$\sqrt{2}$不是无理数。如果学过数列极限,这种”空隙”就是所谓的柯西序列不收敛,但这其实有更本质的原因。

为此,我们引入下面的概念。
\addTODO{与“上确界和下确界”的词条进行整合。}

\begin{definition}{上、下界}
在全序集$(A;\leq)$中,$B$是$A$的非空子集,
\begin{itemize}
\item 元素$M\in A$称为$B$的一个\textbf{上界},如果对任意$b\in B$,都有$b\leq M$。
\item 元素$m\in A$称为$B$的一个\textbf{下界},如果对任意$b\in B$,都有$m\leq b$。
\end{itemize}
\end{definition}

\begin{definition}{上、下确界}
在全序集$(A;\leq)$中,$B$是$A$的非空子集,分别记$B$所有上界和所有下界构成的集合为$S$和$I$。
\begin{itemize}
\item $S$的极小元(若存在)称为$B$的\textbf{上确界},记为$\sup B$。
\item $I$的极大元(若存在)称为$B$的\textbf{下确界},记为$\inf B$。
\end{itemize}
\end{definition}
\begin{exercise}{}
证明:对全序集$(A;\leq)$的任意非空子集$B$,$\sup B\geq\inf B$。
\end{exercise}
\begin{exercise}{}
证明在$(\mathbb{R};\leq)$中,$a$是某个数集$B$的上确界等价于$a$是$B$的上界且对任意正实数$\eta$,均存在$b\in B$使$a-\eta<b$.
\end{exercise}
\subsection{(序)完备性}

下来我们给出完备性的定义。

\begin{definition}{(序)完备性}
全序集$(A;\leq)$称为\textbf{完备的},如果对任意非空集合$B\subseteq A$,$\sup B$存在。
\end{definition}

根据对偶原理,说$\inf B$存在是一样的。

对于实数集而言,$\mathbb{R}$上的闭区间完备应视作实数集应满足的一条公理。而实际上,$\mathbb{R}$本身不是序完备的\footnote{但是$\mathbb{R}$是完备度量空间,这就是序完备性和度量空间完备性的区别。}。

\begin{example}{有理数集的不完备性}

考虑集合$\{x\in\mathbb{Q}|1<x<2,x^2>2\}$,它没有下确界。

这并不代表完备性的定义有问题,因为对应的$\{x\in\mathbb{Q}|1<x<2,x^2<2\}$没有上确界。
\end{example}

\subsection{完备性的拓扑视角}

在此先了解序拓扑。

\begin{definition}{序拓扑}
对于偏序集$(A;\leq)$,由开区间为基生成的拓扑称为\textbf{序拓扑}。
\end{definition}

\addTODO{序拓扑下区间的连续性和闭区间的紧致性的证明。}
