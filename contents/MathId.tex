% 数学归纳法(高中)
% keys 数学归纳法|递推|归纳法
% license Usr
% type Tutor

\begin{issues}
\issueDraft
\end{issues}
\pentry{数列\nref{nod_HsSeFu}}{nod_aa25}

数学归纳法是一种证明数学命题的重要方法,主要用于证明某些关于正整数的命题。它的思想类似于“多米诺骨牌效应”:如果能够证明第一个骨牌倒下(初始成立),并且任意一个骨牌倒下都会带动下一个骨牌倒下(归纳成立),那么可以推断所有的骨牌都会倒下。

数学归纳法的过程分为两步:
\begin{enumerate}
\item 基础步骤(证明初始情况):验证命题在最小的整数(通常为 $n = 1$)时成立。
\item 归纳步骤(假设和推导):假设命题在某个整数 $k$ 成立,即假设 $n = k$ 时命题成立,这被称为“归纳假设”。然后证明命题在 $n = k+1$ 时也成立,即由归纳假设推导出命题对 $n = k+1$ 成立。
\end{enumerate}

如果以上两步都完成,就可以得出结论:该命题对所有正整数 $n$ 都成立。

示例
\begin{example}{证明:对于任意正整数 $n$,数列 $1 + 2 + 3 + \cdots + n$ 的和为 $S_n = \frac{n(n+1)}{2}$。}
证明:

基础步骤:当 $n = 1$ 时,$S_1 = 1$。公式 $\frac{1(1+1)}{2} = 1$ 成立。

归纳步骤:
假设对于 $n = k$,命题成立,即:
\begin{equation}
S_k = 1 + 2 + 3 + \cdots + k = \frac{k(k+1)}{2}~.
\end{equation}
需要证明 $n = k+1$ 时命题也成立:
\begin{equation}
S_{k+1} = S_k + (k+1)~.
\end{equation}

根据归纳假设,将 $S_k$ 代入:
\begin{equation}
S_{k+1} = \frac{k(k+1)}{2} + (k+1)~.
\end{equation}

提取公因式 $(k+1)$,化简得:
\begin{equation}
S_{k+1} = \frac{(k+1)(k+2)}{2}~.
\end{equation}

这与命题的形式一致,因此归纳步骤成立。

综上,利用数学归纳法,可以证明命题对所有正整数 $n$ 都成立。
\end{example}

数学归纳法是一种逻辑性强、步骤清晰的方法,不仅在数列中有广泛应用,还可以用于证明多项式、几何问题等。

数学归纳法把复杂的证明过程转化为了对最终结果的猜想。由此,也使得在高中阶段绕过复杂困难的证明,直接使用其它方法得到结果并使用数学归纳法给出证明成为可能。