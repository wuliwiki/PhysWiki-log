% 木块堆叠问题(里拉斜塔)
% license Xiao
% type Tutor

\begin{issues}
\issueDraft
\end{issues}

\pentry{重心\nref{nod_CenG}}{nod_6e72}

\footnote{参考 Wikipedia \href{https://en.wikipedia.org/wiki/Block-stacking_problem}{相关页面}。}木块推叠问题也称为里拉斜塔, 如\autoref{fig_LireTo_1} 所示。 如何对方可以使最顶端的木块伸出最多。

可以证明从上到下第 $n$ 块($n=1,2,\dots$)最多可以伸出的长度为 $Ln/2$。 故 $N$ 个木块最多伸出的总长度为
\begin{equation}
d_N = \frac{L}{2}\qty(1 + \frac{1}{2} + \frac{1}{3} + \dots + \frac{1}{N})~.
\end{equation}
当 $N\to\infty$ 时, 括号中的级数称为调和级数,%(\addTODO{链接})
可以证明它不收敛, 也就是 $d_N$ 会趋近于无穷。

\begin{figure}[ht]
\centering
\includegraphics[width=10cm]{./figures/713da6ab1145176f.png}
\caption{里拉斜塔(来自 Wikipedia)} \label{fig_LireTo_1}
\end{figure}

\subsection{推导}
我们先来证明,   可以用递归法证明。  一个关键的思路在于, 上方 $n$ 个木块看成一个整体, 然后在下方添加一个木块后, $L/2$ 长度的两端质量比例为 $n:1$, 所以质心位置应该在 $1/(n+1)$ 处。

这里用到一个定理: 两个质点系的质心位置就是两个质点系各自的质心的质心, 见\autoref{sub_CM_1}~\upref{CM}。


