% 机械波
% license CCBYSA3
% type Wiki

(本文根据 CC-BY-SA 协议转载自原搜狗科学百科对英文维基百科的翻译)

\begin{figure}[ht]
\centering
\includegraphics[width=6cm]{./figures/1dac8da5bb90f516.png}
\caption{水中的波纹是一种表面波。} \label{fig_JXB_1}
\end{figure}

机械波是物质振荡的波,因此可以通过介质传递能量。[1]虽然波可以长距离移动,但传输的介质即材料,它的移动是有限的,因此振荡材料不会远离其初始平衡位置。机械波传输能量,这种能量与波的传播方向相同,任何种类的波(机械波或电磁波)都有一定的能量,机械波只能在具有弹性和惯性的介质中产生。

机械波需要初始能量输入。一旦这个初始能量被加入,波就会穿过介质,直到它所有的能量都被转移走了。相比之下,电磁波则不需要介质,但它仍然可以通过介质传播。

机械波的一个重要特性,是它们的振幅的测量方式不同寻常:位移除以(规约的)波长。当它可以与单位尺度比较时,可能会出现显著的非线性效应,如产生谐波,如果它足够大,可能会导致混沌效应。例如,当无量纲振幅超过1时,水体表面的波浪就会破碎,导致表面泡沫和湍流混合。机械波最常见的例子有水波、声波和地震波。

机械波有三种类型:横波、纵波和表面波。电磁波不同于机械波。
\subsection{横波}
横波是波的一种形式,在这种形式中,介质颗粒沿着垂直于波运动方向、在平均位置附近振动。

例如,将长弹簧的一端(另一端固定)左右晃动,而不是前后晃动。[2]光也具有横波的特性,尽管它是电磁波。[3]
\subsection{纵波}
纵波使介质沿平行于波的方向振动。它由多个压缩和稀疏的部分组成。纵波中稀疏部分的距离最远,压缩部分的距离最近。由于被压缩介质中的原子更接近,纵波的速度在更高的压缩率下增加。声音是一种纵波。
\subsection{表面波}
表面波沿着两种介质之间的表面或界面传播。表面波的一个例子是水池中的波,或者海洋、湖泊或任何其他类型的水体中的波。表面波有两种类型,即瑞利波和洛夫波。

瑞利波,也称为地滚波,是以波纹形式传播的波,其运动类似于水面上的波。对于典型的均匀弹性介质,瑞利波比体波慢得多,大约是体波速度的90\%。瑞利波只有二维能量损失,因此在地震中比传统的体波更具破坏性,而纵波和横波则在所有三个方向都损失能量。

洛夫波是一种表面波,其水平波被剪切或横向于传播方向。它们的传播速度通常比瑞利波稍快,约为体波速度的90\%,振幅最大。
\subsection{例子}
\begin{itemize}
\item 地震波
\item 声波
\item 海洋和湖泊上的风浪
\end{itemize}
\subsection{参考文献}
[1]
^Giancoli, D. C. (2009) Physics for scientists & engineers with modern physics (4th ed.). Upper Saddle River, N.J.: Pearson Prentice Hall..

[2]
^Giordano, Nicholas (2009). College Physics: Reasoning and Relationships (illustrated ed.). Cengage Learning. p. 387. ISBN 978-0-534-42471-8. Extract of page 387.

[3]
^Towne, Dudley H. (2014). Wave Phenomena (illustrated ed.). Courier Dover Publications. p. 139. ISBN 978-0-486-14515-0. Extract of page 139