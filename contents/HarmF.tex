% 调和场(无散无旋场)
% keys 散度|旋度|调和场|拉普拉斯
% license Xiao
% type Tutor

% 调和场(无散无旋场)
% 散度|旋度|调和场|拉普拉斯

\pentry{拉普拉斯方程与调和函数\nref{nod_LapEq}}{nod_5537}

我们把散度和旋度都为零的场称为\textbf{无散无旋场}或\textbf{调和场}。 注意后者并不是一个很常用的数学名词, 笔者只在个别中文教材中见过。 如果这只在一个空间的一定区域内成立, 那么就说它在这个区域内是调和场。

由于调和场旋度为零, 线积分与路径无关, 必定可以定义势函数 $u(\bvec r)$ (包含一个任意常数项), 而调和场就是其梯度
\begin{equation}\label{eq_HarmF_1}
\bvec f(\bvec r) = \grad u~.
\end{equation}
要保证散度 $\div \bvec f$ 为零, \autoref{eq_HarmF_1} 就要求 $u$ 是一个调和函数:
\begin{equation}
\laplacian u = 0~.
\end{equation}
所以调和场的充分必要条件是它可以表示为一个调和函数的梯度。 因为只有常数的梯度处处为零, 当且仅当给调和函数加一个任意常数时, $\bvec f(\bvec r)$ 不会改变。

调和场在电磁学中经常出现, 若在空间选取中一个不含电荷的区域, 那么该区域外的静止电荷以及恒定电流在该区域中产生的电场和磁场都是调和场——由\enref{麦克斯韦方程组}{MWEq}可知它们的散度和旋度在该区域都为零。

\begin{theorem}{}\label{the_HarmF_2}
调和场的各个分量都是调和函数。
\end{theorem}
证明: 使用定义, 注意偏微分的顺序可以任意改变, 证毕。 注意各个分量都是调和函数的矢量场未必是调和场, 反例: $x\uvec x + y \uvec y$ 的分量都是调和函数, 但散度不为零。

\begin{corollary}{有界调和场}
如果 $\mathbb R^N$ 上的调和场是有界的(模长为有限值), 那么它是一个常矢量场。
\end{corollary}
证明: 有界调和场对调和场的各个分量也是有界的, 根据\autoref{the_HarmF_2} 它们都是调和函数, 分别使用刘维尔定理(\autoref{the_HarFun_1})得每个分量都是常数, 所以矢量场也是常矢量场。 证毕。

与调和函数类似, (非零)调和场的一个显著特点是其在无穷远处不为零(\autoref{cor_HarmF_1}), 由于无散无旋, 它的 “场线” (例如电场线)没有起点也没有终点且不闭合, 而是从无穷远来, 到无穷远去。 我们可以把它想象为某种不可压缩流体的速度场或\enref{流密度}{CrnDen}场。

\begin{theorem}{最大值定理}\label{the_HarmF_1}
$\mathbb R^N$ 的有限区域内的调和场的模长最大值必定出现在该区域的边界处。
\end{theorem}
证明: 令调和场为 $\bvec f = \div u$, $u$ 是一个调和函数。 易证 $\abs{\bvec f}^2$ 也是一个调和函数。 对其使用刘维尔定理(\autoref{the_HarFun_1})即可, 证毕。

\begin{corollary}{}\label{cor_HarmF_1}
如果 $\mathbb R^N$ 上的调和场 $\bvec f(\bvec r)$ 满足 $\lim_{\abs{\bvec r}\to \infty} \abs{\bvec f(\bvec r)}  = 0$, 那么 $\bvec f(\bvec r) \equiv \bvec 0$。
\end{corollary}
证明: 可以先选择一个半径为 $r$ 的圆/球作为\autoref{the_HarmF_1} 的区域, 然后令半径区域无穷即可。
