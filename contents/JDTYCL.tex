% 经典统一场论(综述)
% license CCBYSA3
% type Wiki

(本文根据 CC-BY-SA 协议转载自原搜狗科学百科对英文维基百科的翻译)

自19世纪以来,一些物理学家,特别是阿尔伯特·爱因斯坦,一直试图创建一个单一的理论框架来解释自然的所有基本作用力——一个统一的场论。经典统一场论是在经典物理基础上建立统一场论的尝试。特别是,在两次世界大战之间的几年里,几位物理学家和数学家积极追求引力和电磁学的统一。这项工作刺激了微分几何的纯数学发展。

\subsection{概观}

创建统一场论的早期尝试始于广义相对论的黎曼几何,并试图将电磁场纳入更一般的几何,因为普通的黎曼几何似乎无法表述电磁场的性质。爱因斯坦并不是唯一一个试图统一电磁和引力的人;包括赫尔曼·韦勒、阿瑟·爱丁顿和西奥多·卡鲁扎在内的大量数学家和物理学家也试图开发能够统一这些相互作用的方法。[1][2] 这些科学家探索了几种推广途径,包括扩展几何基础和增加额外的空间维度。

\subsection{早期工作}

米氏在1912年和恩斯特·赖欣巴赫在1916年首次尝试提供统一的理论。[3][4] 然而,这些理论并不令人满意,因为它们没有包含广义相对论,因为广义相对论尚未形成。连同鲁道夫·福斯特的这些尝试都未成功,包括把度规张量(以前被认为是对称的和实值的)变成不对称的和/或复值的张量,他们还试图为物质创造场论。

\subsection{微分几何和场论}

从1918年到1923年,场论有三种截然不同的方法:韦勒的规范理论、卡鲁扎的五维理论和爱丁顿对仿射几何的发展。爱因斯坦与这些研究人员通信,并与卡鲁扎合作,但尚未完全参与统一的工作。

\subsection{韦勒的微分几何}

为了将电磁学纳入广义相对论的几何范畴,赫尔曼·韦勒致力于推广广义相对论所依据的黎曼几何。他的想法是创造一个更一般的微分几何。他指出,除了度量场之外,在流形中两点之间的路径上还可能有额外的自由度,他试图通过引入一种基本方法来利用这一点,依据规范场来比较沿着这一路径的局部尺寸测量值。这个几何推广了黎曼几何,因为除了公制$g$之外,还有一个向量场$Q$,它们共同产生电磁场和引力场。这个理论尽管很复杂,在数学上却是合理的,但却导致了困难的高阶场方程。这个理论中的关键数学成分,拉格朗日元和曲率张量,是由韦勒和他的同事们计算出来的。然后韦勒就其物理有效性与爱因斯坦和其他人进行了广泛的通信,最终发现该理论在物理上是不合理的。然而,韦勒的规范不变性原理后来以一种修正的形式应用于量子场论。

\subsection{卡鲁扎的第五维度}

卡鲁扎统一的方法是将时空嵌入一个由四维空间和一维时间组成的五维圆柱世界。与韦勒的方法不同,黎曼几何被保留下来,额外的维度允许电磁场矢量结合到几何中。尽管这种方法在数学上相对优雅,但在爱因斯坦和他的助手格罗姆的合作下,我们确定了这种理论不允许非奇异的静态球对称解。这一理论确实对爱因斯坦后来的工作产生了一些影响,后来克莱因进一步发展了这一理论,试图将相对论纳入量子理论,即现在所说的卡鲁扎-克莱因理论。

\subsection{爱丁顿的仿射几何}

亚瑟·斯坦利·爱丁顿爵士是一位著名的天文学家,他成为爱因斯坦广义相对论的狂热而有影响力的倡导者。他是最早提出引力理论扩展的人之一,该理论以仿射连接作为基本结构场,而不是作为广义相对论最初焦点的度规张量。仿射连接是矢量从一个时空点并行传输到另一个时空点的基础;爱丁顿假设仿射连接在其协变指数中是对称的,因为沿着另一个平行的方向传输一个无穷小向量的结果应该产生与沿着第一个方向传输相同的结果似乎是合理的。(后来的工作者们重新审视了这一假设。)

爱丁顿强调了他所考虑的是认识论上的考虑;例如,他认为宇宙常数版本的广义相对论场方程表达了宇宙具有“自我测量”的性质。由于求解该方程的最简单的宇宙模型(德西特宇宙)是一个球对称的、静止的、封闭的宇宙(表现出宇宙红移,通常被解释为由于膨胀),它似乎解释了宇宙的整体形式。

像许多其他经典的统一场理论家一样,爱丁顿认为,在广义相对论的爱因斯坦场方程中, 应力-能量张量$T_{\mu \nu}$,代表物质/能量的应力-能量张量仅仅是暂时的,在真正统一的理论中,源项将作为自由空间场方程的某个方面自动出现。他也希望存在一个改进的基础理论能解释为什么当时已知的两种基本粒子(质子和电子)质量完全不同。

相对论量子电子的狄拉克方程导致爱丁顿重新思考他以前的信念,即基础物理理论必须以张量为基础。他后来致力于发展一种主要基于代数概念(他称之为“E-框架”)的“基础理论”。不幸的是,他对这个理论的描述很粗略,很难理解,所以很少有物理学家跟进他的工作。[5]

\subsection{爱因斯坦的几何方法}

当在爱因斯坦广义相对论的框架内建立等效的麦克斯韦电磁方程时,电磁场能量(相当于从爱因斯坦著名方程E=mc2中预期的质量)贡献于应力张量,从而贡献于由引力场广义相对论表示的时空曲率;或者换句话说,弯曲时空的某些结构包含了电磁场的影响。这表明,纯几何理论应该把这两个领域视为同一基本现象的不同方面。然而,普通的黎曼几何无法将电磁场的性质描述为纯粹的几何现象。

爱因斯坦试图构建一个广义引力理论,该理论将引力和电磁力(或许还有其他力)统一起来,其指导思想是对整套物理定律的单一起源的信念。这些尝试最初集中在其他的几何概念上,如vierbeins和“远平行度”,但最终集中在将度规张量和仿射连接都视为基本场。(因为它们不是独立的,度量仿射理论有些复杂。)在广义相对论中,这些场是对称的(在矩阵意义上),但是由于反对称对于电磁学来说似乎是必不可少的,所以对一个或两个场的对称要求被放宽了。爱因斯坦提出的统一场方程(物理学的基本定律)一般是从假设的时空流形的黎曼曲率张量表示的变分原理推导出来的。[6]

在这类场论中,粒子在时空上表现为场强或能量密度特别高的有限区域。爱因斯坦和他的同事利奥波德·英费尔德设法证明,在爱因斯坦统一场的最终理论中,场的真正奇点确实有类似点粒子的轨迹。然而,奇点是方程出现问题的地方,爱因斯坦认为,在最终的理论中,定律应该适用于任何地方,粒子是(高度非线性的)场方程的类孤子解。此外,宇宙的大规模拓扑应该对解施加限制,例如量化或离散对称。

理论的抽象程度,加上相对缺乏分析非线性方程系统的好的数学工具,使得很难将这些理论与它们可能描述的物理现象联系起来。例如,有人提出扭转(仿射连接的反对称部分)可能与同位旋而不是与电磁有关;这与爱因斯坦称为“位移场对偶性”的离散(或“内部”)对称性有关。

爱因斯坦在广义引力理论的研究中变得越来越孤独,大多数物理学家认为他的尝试最终不会成功。特别是,他对基本力统一的追求忽略了量子物理学的发展(反之亦然),尤其是发现了强核力和弱核力。[7]

\subsection{薛定谔的纯仿射理论}

受爱因斯坦统一场论方法和爱丁顿将仿射连接作为时空微分几何结构唯一基础的思想的启发,埃尔温·薛定谔从1940年到1951年深入地研究了广义引力理论的纯仿射公式。虽然他最初假设对称仿射连接,但后来他像爱因斯坦一样考虑了非对称场。

薛定谔在这项工作中最惊人的发现是度规张量是通过黎曼曲率张量的简单构造在流形上诱导出来的,而黎曼曲率张量又完全由仿射连接形成。此外,以最简单可行的变分原理为基础,采用这种方法得到了一个场方程,其形式为爱因斯坦的广义相对论场方程,并自动产生一个宇宙论的项。[8]

爱因斯坦的怀疑和其他物理学家发表的批评使薛定谔气馁,他在这一领域的工作在很大程度上被忽视了。

\subsection{以后的工作}

20世纪30年代后,由于对自然界非引力基本力的量子理论描述的持续发展,以及发展引力量子理论遇到的困难,从事经典统一研究的科学家逐渐减少。爱因斯坦继续努力从理论上统一引力和电磁学,但他在这项研究中变得越来越孤独,他一直坚持到去世。爱因斯坦的名人身份引起了人们对他最终探索的极大关注,最终他的探索获得了有限的成功。

另一方面,大多数物理学家最终放弃了经典的统一理论。目前对统一场论的主流研究集中在创建引力量子理论以及与物理学中其他基本理论统一的问题上,所有这些理论都是量子场论。(有些程序,如弦理论,试图同时解决这两个问题。)在已知的四种基本力中,万有引力仍然是一种与其他力的统一证明存在问题的力。

尽管新的“经典”统一场论不断被提出,通常涉及自旋子等非传统元素,但没有一个被物理学家普遍接受。

\subsection{参考文献}

\begin{enumerate}
\item Weyl, H. (1918). "Gravitation und Elektrizität". Sitz. Preuss. Akad. Wiss.: 465..
\item Eddington, A. S. (1924). The Mathematical Theory of Relativity, 2nd ed. Cambridge Univ. Press..
\item Mie, G. (1912). "Grundlagen einer Theorie der Materie". Ann. Phys. 37 (3): 511–534. Bibcode:1912AnP...342..511M. doi:10.1002/andp.19123420306..
\item Reichenbächer, E. (1917). "Grundzüge zu einer Theorie der Elektrizität und der Gravitation". Ann. Phys. 52 (2): 134–173. Bibcode:1917AnP...357..134R. doi:10.1002/andp.19173570203..
\item Kilmister, C. W. (1994). Eddington's search for a fundamental theory. Cambridge Univ. Press..
\item Einstein, A. (1956). The Meaning of Relativity. 5th ed. Princeton Univ. Press..
\item Gönner, Hubert F. M. "On the History of Unified Field Theories". Living Reviews in Relativity. Retrieved August 10, 2005..
\item Schrödinger, E. (1950). Space-Time Structure. Cambridge Univ. Press..
\end{enumerate}
