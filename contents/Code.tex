% 跨平台源代码编辑器安装
% 编辑器|跨平台

\begin{issues}
\issueTODO
\end{issues}

\subsection{VS code}

VSCode(全称:Visual Studio Code)是一款由微软开发且跨平台的免费源代码编辑器。

优点:开发环境易用、安装插件简单方便、可中文、打开速度快

缺点:代码提示、纠错相对没那么好、运行速度不快

访问 \href{https://code.visualstudio.com/}{VS code 官方下载地址} 即可下载

\begin{figure}[ht]
\centering
\includegraphics[width=14.25cm]{./figures/Code_1.png}
\caption{VS code 官网} \label{Code_fig1}
\end{figure}

\subsection{Sublime Text}

Sublime Text是由程序员Jon Skinner于2008年1月份所开发出来,它最初被设计为一个具有丰富扩展功能的Vim。

优点:体积较小,运行速度快、文本功能强大

缺点:插件不方便

访问 \href{https://www.sublimetext.com/}{Sublime Text 官方下载地址} 即可下载

\begin{figure}[ht]
\centering
\includegraphics[width=14.25cm]{./figures/Code_2.png}
\caption{Sublime Text 官网} \label{Code_fig2}
\end{figure}

\subsubsection{Sublime Text的汉化}

访问 \href{https://github.com/rexdf/ChineseLocalization}{汉化插件的GitHub地址} 下载插件

找到 preferences——browse packages,这是插件安装目录。下载插件的压缩包,解压,我们仅仅只需要其中的 \verb|ZH_CN| 目录里的文件,将里面的文件全部复制,然后到上面的插件安装目录下,新建文件夹,叫 Default(自动覆盖自带的语言包),在粘贴进去。

%\subsection{VS}

%\subsection{Atom}

%\subsection{Komodo Edit}
