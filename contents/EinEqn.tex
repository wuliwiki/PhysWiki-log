% 爱因斯坦场方程
% 引力场|引力|gravity|场方程|Ricci张量|测地线|geodesic|广义相对论|相对论|relativity|时空|spacetime|弯曲|曲率

\pentry{测地线\upref{geodes},尘埃云的能动张量\upref{SRFld}}

\subsection{能动张量}

爱因斯坦场方程的引出过程中,我们考虑的是最简单的宏观模型,即\textbf{尘埃云的能动张量}\upref{SRFld}中所介绍的\textbf{理想流体}.对于任何物质,其四动量的各分量,都会随着参考系的不同而有不同取值,因此这些量只能是流形上某种量在具体坐标系中的坐标分量而已.这样一来,在流形上描述能量质量分布的量,就不能是简单的光滑函数,或者说标量场,而只能是更高阶的张量场.能描述理想流体四动量分布的张量,可以是四动量本身,也可以是能动张量,而我们会选择能动张量,这样才能和Ricci曲率张量的阶数吻合.

能动张量能如何描述物质分布的性质呢?

\begin{definition}{能量密度}
令$T_{ab}$为一理想流体在时空流形上的能动张量场,$u^a$是该流体在各时空点处局部质心系的四速度,则$T_{ab}u^au^b$是该流体在该处的\textbf{能量密度}.
\end{definition}

这一定义是合理的.考虑到局部质心系中$u^a=\pmat{1&0&0&0}\Tr$,根据理想流体的能动张量的定义,可以计算出$T_{ab}u^au^b$就是能量密度分布,而这是张量场的抽象指标分布,不依赖于参考系的选择,所以是合理的推广表达.














