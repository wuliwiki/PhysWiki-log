% 波动和粒子(转载)
% license Usr
% type Art

(本文根据 CC-BY 协议转载自季燕江的《量子序曲》, 进行了重新排版和少量修改)

我们给自己设定一个任务,就是努力把事情说清楚,把事情说清楚就是使听众认同,如果没有听众就是使假想的听众——另一个自己——认同。这听起来有点分裂,但确实要求这另一个自己有一点“较真”和“不断置疑”的精神。

说服,或者使用图像,或者使用语言。而使用图像更有说服力。

图像就是图景,是Picture,但要让它动起来,这需要一点想象力。有时候我们会说动起来的图像比静止的图像更真实,这是由于人在大多数情况已经适应了动起来的图像,或赋予动起来的图像以更高的审美价值。

比如拍照,我们在任一瞬间拍下的照片都是“真实”的,但拿来看却往往惹人笑,表情的瞬间也许很丑,我们从来都没有看过小于$0.01$秒的瞬间,我们看的是“时间”的绵延,是处理过的富于动感的图像,我们是借助动感的图像建立起我们的审美,对人神情的,人动态的……审美。我们觉得眨眼很迷人,但我们觉得眼睛完全闭合的瞬间丑死了。

\begin{figure}[ht]
\centering
\includegraphics[width=6cm]{./figures/356af97ab00da9ae.png}
\caption{米洛的维纳斯像:看起来有动感的雕像都有不符合解剖学比例的地方。} \label{fig_QMPre5_1}
\end{figure}

在我们的想象中要让比如一个点动起来,一个三角形变大,旋转……是需要想象力的,而我们确实有这样的能力,有时我们在瞬间看到万丈光芒,极度具有结构特征的几何图形,色彩的斑驳变化,伴随着节奏、音乐的节奏,或我们思绪的内在节奏,在自己的脑海里轮番更替。

在这样的瞬间我们仿佛透彻了宇宙间所有的真理,但却没有合适的语言把它们描述下来,我们只能试一试,努力凭记忆把它们画下来,借着篝火,用颜料涂抹在山洞的顶上。

一些极富想象力的抽象作品。

据说这种抽象风格的绘画和极度写实的绘画同时出现。“写实”是对“可见”事物的模仿,而“抽象”是对“可以想见”事物的模仿。它们都是有力量的存在。

今天的人会设想,也许是出于交流或记录的目的,我们的祖先才开始绘画。但也有人说最早的绘画应该没有任何教育或交流的目的,它纯粹是出于精神的需要,是人灵性的发泄,是与神亲近的方式,所以它们才被创作于黑黢黢的山洞里面。

\begin{figure}[ht]
\centering
\includegraphics[width=6cm]{./figures/38011ba9c33d6a3b.png}
\caption{肖维岩洞(Chauvet Cave)中的绘画已经有三万多年的历史了,几乎与人类本身的历史一样悠久,图中奔跑的野牛被表现为有很多条腿的野牛。} \label{fig_QMPre5_2}
\end{figure}

\subsection{粒子的图像}

我们现在就需要通过想象而非观瞧来建立关于粒子的图像,和关于波动的图像。这是我们谈论波粒二象性的前提。

何谓粒子?

它是一个点,理想的点,只有方位,而无部分。在我的想象里它居于一个三维的空间,我可以让它上下稍微动一动,而完全不会影响其在左右方向上的位置,也不会影响其在前后方向上的位置。

假如世界只有这一个粒子,这是多么的空洞。

它可以任意地上下、左右、前后地改变位置,但对我来说是完全一样的。它都在我想象的空洞空间的中央,位置的改变无从说起。

我们必须有个参照,有了参照,我们就能说出粒子的位置了。

这个参照就是三把想象的尺子,它们互相垂直构成一个参照。或我们需要第二个点,作为原点,粒子是参照原点运动变化的。我们还需要三个箭头,标明三维空间的三个方向。

但“三”,为什么是“三”呢?

这是我们对我们所在空间的合理分类。

分类是理性的活动,是智慧的活动,谁善于分类谁就是聪明人!

“亚当证明自己是世界上第一位且最伟大的哲学家:他能根据物种的真正的本质和差异恰如其分地对它们加以区分。”

合理分类的标准是:既不重复,也不遗漏。

把我们所在的空间分解为三个不同方向是既不重复也不遗漏的。

我们也可以设想二维或一维的世界,这或者是出于限制,比如我们人类的活动就长期被限制在二维的空间上,当然是在球面上。物理学家现在也喜欢谈量子限制效应(quantum confinement effect),所谓限制就是出于某种原因,粒子(比如电子)就仅仅在二维或一维空间里运动。

想象低维空间的好处是想起来比较容易。比如落体运动,其实就是粒子在引力的作用下以越来越快的速度下落,描述这个运动只需要想象一维空间。比如炮弹的运动,就是粒子一方面以初始速度往斜刺里飞,要想飞的最远就需要以$45^o$的角度往斜刺里飞,另一方面粒子仍然受引力的作用在以越来越快的速度往地面落。

所以这是一个想象中的两个运动的叠加,先往斜刺里飞一段,再往下掉一段,然后再往斜刺里飞一段,然后再以更大速度往下掉一段。最后在我们的想象中再让这一段段锯齿缩小,让它看起来圆顺光滑一些,这就是炮弹的运动——抛物线了。

我们还可以想象很多,比如坐过山车,呼啸而下,越来越快,心悸的感觉,然后在极度的空虚中,我们向上,越来越高,在最高处,时间仿佛停止了,其实是此时速度最小,最后向下,加速,新的循环开始。

这里的窍门,是把我们自己想象成粒子,把自己的心替换为粒子的心,让我们进入粒子的世界。我们会感到有风迎面吹来,感到阻力,……

阻力是阻碍粒子运动的。我们喜欢引力,只要速度合适我们能围绕一个引力的中心(比如地球)循环往复地运动起来,一个椭圆:当我们如过山车一般冲向地球的时候,我们的速度最快,因为速度,我们从离地球最近的地方呼啸而过,然后摆脱地球,离它越来越远,向上,弯曲着向上,依靠惯性,或依靠动能($K = \frac{mv^2}{2}$)反抗地球的吸引,直到冲到离地球最远的地方,空虚地失去了太多动能,然后引力又占了上风,拉着我们加速下降,如此循环。

我们在飞,我们努力想控制飞行的轨迹,但很可惜,我们没有办法,就像梦境中的人想努力控制自己的飞行一样,徒劳和无能为力,我们在虚空中飞行,引力是唯一的外部原因,它严格地按$F = \frac{G M m}{r^2}$行为,椭圆轨道由我们的初始冲动决定,即我们在距离地球多远的地方,决定以一个什么样的速度,什么样的角度,开始运动。

(对一个二阶常微分方程$F = m a$来说,只要给定两个初始条件,初始的位置$x_0$和初始的动量$p_0$,粒子的运动就完全决定了。换句话说我们就能求解出粒子运动的轨迹。)

粒子是我们现在思维的基本单元,每个粒子都可以用质量,位置,和动量来描述。

质量就是粒子的质量。我们假设万事万物都有质量,但光子(光的粒子)除外,我们暂时先不讨论它。位置就是一个三维矢量,我们一般把它记为$\vec r$,它分解为三个固定方向上向量的叠加:

\begin{equation}
\vec r = \vec i x + \vec j y + \vec k z~
\end{equation}

动量的定义是质量乘以速度,速度定义为对位置的微分:

\begin{align}
\vec v & = & \frac{d \vec r}{d t} \\
\vec p  & = & m \vec v~
\end{align}

位置,速度,动量都是矢量,还有力,这给我们的想象力带来极大的挑战,为了思维的轻松,我们往往把它们想象为二维的或一维的。