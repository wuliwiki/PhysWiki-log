% 块对角厄米矩阵的本征问题

\pentry{厄米矩阵的本征问题\upref{HerEig}, 块对角矩阵\upref{BlDiag}}

\begin{theorem}{}
若一个厄米矩阵 $\mat H$ 是块对角的, 那么每个对角块 $\mat H_i$ 也显然是厄米矩阵。 只需要分别解每个 $\mat H_i$ 的本征方程, 得到相同大小的本征矢列矩阵 $\mat P_i$, 然后按相同顺序拼成块对角矩阵 $\mat P$, 就是 $\mat H$ 的本征矢列矩阵。
\end{theorem}

从线性算符的角度来理解, 就是
\begin{theorem}{}
若厄米算符 $H:V\to V$ 在 $V$ 的若干子空间 $V_i$ 中闭合, 且
\begin{equation}
V = V_1\oplus V_2\oplus \dots \oplus V_N~.
\end{equation}
那么 $V_i$ 两两正交, 且每个 $H:V_i\to V_i$ 的本征矢和本征值就是 $H:V\to V$ 的本征矢和本征值。
\end{theorem}

\addTODO{举例}

由于映射 $H$ 在每个子空间中都闭合, 只需要在每个子空间解 $H$ 的本征矢, 即第 $n$ 个对角块矩阵元为 $\mel{v_{n,i}}{H}{v_{n,j}}$……
