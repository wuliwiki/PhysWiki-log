% 库仑定律
% 库仑力|库仑定律|平方反比力|电介质常数

\begin{issues}
\issueDraft
\end{issues}

\pentry{万有引力\upref{Gravty}}

学习万有引力以后, 库仑力就可以轻而易举地类比过来, 所以我们这里不再做重复的推导.

先来定义\textbf{点电荷}, 点电荷就是在质点的基础上(忽略物体的大小与形状), 增加了一个总电荷量的属性. 与万有引力的\autoref{Gravty_eq1}~\upref{Gravty}类似, 两个点电荷间的库仑力的矢量表达式为
\begin{equation}\label{ClbFrc_eq1}
\bvec F_{12} = k\frac{q_1 q_2}{r_{12}^2} \uvec r_{12} = \frac{1}{4\pi\epsilon_0} \frac{q_1 q_2}{\abs{\bvec r_2 - \bvec r_1}^3}(\bvec r_2 - \bvec r_1)
\end{equation}
其中 $q_1, q_2$ 分别是两个点电荷的电荷量. $k$ 是库仑常数, 但在大学物理中我们几乎不会见到这个符号, 它通常被替换为
\begin{equation}
k = \frac{1}{4\pi\epsilon_0} \approx 8.9876\e9 \Si{N m^2/C^2}
\end{equation}
$\epsilon_0$ 是\textbf{真空中的电介质常数(vacuum permittivity)}, 我们以后会在 “高斯定律\upref{EGauss}” 中见到.

注意比起万有引力的\autoref{Gravty_eq1}~\upref{Gravty}, \autoref{ClbFrc_eq1} 没有负号, $q_1, q_2$ 可以是负数(代表负电荷). 我们容易看出两电荷同号相吸, 异号相斥.
