% 一般线性变换的 Jordan(若尔当)标准形
% license Xiao
% type Tutor


\begin{issues}
\issueTODO
\end{issues}

利用幂零变换的结论(\autoref{the_Jordan_1}),本节进一步证明,任意线性变换都能化为类似的Jordan形,并且Jordan形是唯一的。
\subsection{线性变换的Jordan形}
由\autoref{the_nullpl_1} 可知,任意线性空间都可以分解为任意线性变换的不变子空间之直和,例如对于线性变换$A$,我们有:
\begin{equation}
V=\opn{ker}f(A)=\opn{ker}f_1(A)\oplus\opn{ker}f_2(A)...\oplus\opn{ker}f_m(A)~,
\end{equation}
其中$f(A)$是$A$的特征多项式,$f_i(A)$是本征值相关的互素项:$f_i(A)=(A-\lambda_iI)^{k_i}$。为方便计,设$\opn{ker}f_i(A)=W_i,A-\lambda_iI=B_i$,则$A=A|_{W_1}\oplus A|_{W_2}\oplus...\oplus A|_{W_n}=\bigoplus^n_{i=1}(B_i+\lambda_iI)$

由\autoref{the_Jordan_1} 可知,幂零变换意味着$W_i$存在某组基,使得$B_i$在该基下都是Jordan形矩阵,称对应该基的$A|_{W_i}$为一般线性变换的Jordan形。归纳此段讨论为下述定理:
\begin{theorem}{}\label{the_ltrJor_1}
设$A$为域$\mathbb F$上$n$维线性空间$V$内的一个线性变换,其特征多项式的所有根$\lambda\in\mathbb F$,则在$V$内存在一组基,使线性变换$A$在这组基下可以表示为如下块对角矩阵:
\begin{equation}
J=\left(\begin{array}{cccc}
J_1 & & & 0 \\
& J_2 & & \\
& & \ddots & \\
0 & & & J_s
\end{array}\right), \quad J_i=\left(\begin{array}{cccc}
\lambda_i & 0 & & 0 \\
1& \lambda_i &  & \\
\vdots& 1& \ddots & 0 \\
0 &\cdots &1 & \lambda_i
\end{array}\right)~.
\end{equation}
\end{theorem}
$A|_{W_i}$由若干个Jordan块直和而成,每块对应一个本征向量,不同块的本征数量线性无关,因此块数就是几何重数。实数域上不能保证根都在域内,因而线性变换未必有Jordan形,而复数域上的线性变换总能表示为该形式。

\subsubsection{另一种证明思路}
对于具有本征值 $\lambda$ 的线性变换$A$而言,我们有如下基本结论:
\begin{equation}\label{eq_ltrJor_1}
\begin{aligned}
&\opn{ker}B\subseteq\opn{ker}B^2\subseteq...\subseteq\opn{ker}B^k\\
&\opn{Im}B\supseteq\opn{Im}B^2\supseteq...\supseteq\opn{Im}B^k~,
\end{aligned}
\end{equation}
其中$B=A-\lambda I$。\textbf{在这两个序列中,一旦有两个连续的子空间相等,后续序列中的子空间都相等。}
\begin{lemma}{}
设$\opn{ker}B^k=\opn{ker}B^{k+1}$,对于任意$m>k$,有\begin{equation}
\opn{ker}B^{k+m}=\opn{ker}B^{k+m+1},\quad\opn{Im}B^{k+m}=\opn{Im}B^{k+m+1}~.
\end{equation}
\end{lemma}
Proof.\footnote{参考蓝以中《高等代数简明教程》}

由\autoref{eq_ltrJor_1} 知,我们只需要证明$\opn{ker}B^{k+m+1}\supseteq \opn{ker}B^{k+m}$即可。设任意$\bvec x\in \opn{ker}B^{k+m+1}$,则
\begin{equation}
B^{k+m+1}\bvec x=B^{k+1}B^m\bvec x=\bvec 0~,
\end{equation}
因此可知$B^m\bvec x\in\opn{ker}B^{k+1}$。由题设得:$B^{k}B^m\bvec x=\bvec 0$,遂$\bvec x\in\opn{ker}B^{k+m}$,得证。

由于$V$有限维,象空间或者核空间不可能无限“扩大”,因而满足上述条件的$k$总是存在的。

接下来我们证明:$V=\opn{ker}B^k\oplus\opn{Im}B^k$,由于$\opn{dim}V=\opn{dim}\opn{ker}B^k+\opn{dim}\opn{Im}B^k$,因此只需要证明两个子空间无交集即可。

设任意$\bvec v\in \opn{ker}B^k\cap\opn{Im}B^k$,可令$\bvec v=B^k\bvec u$。则有$B^k\bvec v=B^2k\bvec u=\bvec 0$,则$\bvec u\in \opn{ker}B^{2k}=\opn{B^k}$,那么我们有$\bvec v=B^k\bvec u=\bvec 0$,得证。

由于$B^k$与$A$可交换,所以$\opn{ker}B^k,\opn{Im}B^k$是$A$的不变子空间。设$K=\opn{ker}B^k,I=\opn{Im}B^k$,则有:
\begin{equation}
A=A|_K\oplus A|_I~,
\end{equation}

接下来我们就可以用归纳法证明\autoref{the_ltrJor_1} 了。依然设$\lambda$为$A$的一个本征值,且$B=A-\lambda I,n=\opn{dim}V$。

维度为$1$时$A$是一阶方阵所以该定理成立。假设定理对维度$<n$的线性空间都成立,只要我们推出对$n$维线性空间$V$成立则得证。由于$B|_K$是幂零变换,因此$A|_K$可化为Jordan形,$A|_I$由题设也有Jordan形,定理得证。
\subsection{Jordan形的唯一性}
当$A$有Jordan形时,其对应$V$上的一个幂等变换。因此,只要证明$B$的Jordan形是唯一的\footnote{该证明引用Jier Peter的《代数学基础》},则$A$自然也有唯一的Jordan表示。(只改变以不同$\lambda$区分的Jordan矩阵顺序视为一种表示。)

$B$的Jordan形是唯一的,等价于说$V$关于$B$的循环子空间分解方式唯一。用归纳法证明,并设定理对于$Im B$成立。反设不唯一,即:
\begin{equation}
V=V_1\oplus V_2\oplus...\oplus V_k=W_1\oplus W_2\oplus...\oplus W_s~.
\end{equation}
用$B$作用两端后得:
\begin{equation}
\opn{Im}B=B(V_1)\oplus B(V_2)\oplus...\oplus B(V_k)=B(W_1)\oplus B(W_2)\oplus...\oplus B(W_s)~.
\end{equation}
由于在像上分解唯一,则上述两种分解之间必然有指定同构。比如:$B(V_r)=B(W_s)$。由维度关系知:$V_r=W_s$
\subsection{Jordan矩阵的计算方法}
