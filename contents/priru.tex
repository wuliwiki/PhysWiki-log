% 单位根与本原单位根(数论)
% keys 本原单位根
% license Usr
% type Tutor

\pentry{数论三角和\nref{nod_ntrtre},最大公约数与最小公倍数\nref{nod_gcdlcm}}{nod_4de4}

\begin{definition}{单位根}
若 $p^q=1$,就说 $p$ 是 \textbf{$q$ 次单位根}。
\end{definition}

\begin{definition}{本原单位根}
若 $p^q = 1$,就称 $p$ 是\textbf{本原 $q$ 次单位根(primitive $q$-th root of unity)}。但对于任何小于 $q$ 的正 $r$,$q^r$ 都不等于 $1$。
\end{definition}
我们会注意到 $p$ 几乎总是复数。

\begin{theorem}{}
任何 $q$ 次单位根都是对 $q$ 的某个因子 $r$ 而言的一个本原 $r$ 次单位根。
\end{theorem}
\textbf{证明}:假设 $p^q=1$,而 $r$ 是使 $p^r=1$ 的最小正整数,则对于 $q = kr+s$,其中 $0\le s < r$,将有
\begin{equation}
p^s = p^{q - kr} = 1~,
\end{equation}
故 $s=0$ 且 $r | q$。

\begin{theorem}{单位根的取值}
$q$ 次单位根是下列各数,
\begin{equation}
e\left( h/q \right), ~ (h = 0, 1, \dots, q-1) ~.
\end{equation}

\end{theorem}

\begin{theorem}{单位根与本原单位根的关系}
对于所有 $q$ 次单位根
\begin{equation}
e(h/q), ~(h = 0, 1, \dots, q-1) ~,
\end{equation}
其是本原 $q$ 次单位根当且仅当 $h$ 与 $q$ 互素。
\end{theorem}
