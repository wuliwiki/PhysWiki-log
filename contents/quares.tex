% 二次剩余
% keys 二次剩余
% license Usr
% type Tutor

\pentry{素数与合数\nref{nod_prmnt},整除\nref{nod_divisb}}{nod_9b3c}
\begin{definition}{伴随数}
若 $p$ 是一个奇素数,且 $p \not{\mid}~ a$,而 $x$ 是 $1, 2, \dots, (p-1)$ 中的一个,则根据\autoref{the_EulFun_1}~\upref{EulFun},在
\begin{equation}
1 \cdot x, 2 \cdot x, \dots, (p-1)\cdot x ~~
\end{equation}
这些数中,必有且仅有一个与 $a$ 模 $p$ 同余,这就说明,存在唯一的 $x'$ 使得
\begin{equation}
x x' \equiv a \pmod p ~,
\end{equation}
称 $x'$ 是 $x$ 关于 $a$ 的\textbf{伴随数(associate)}。
\end{definition}
我们会发现,由于伴随数的定义,要么会至少有一个 $x$ 与自己相伴,要么没有这样的 $x$。由此引出了 $x^2 \equiv a \pmod p$ 的解的情况,就是二次剩余。
\begin{definition}{二次剩余}
若 $x_1$ 与自己关于 $a$ 在模奇素数 $p$ 意义下相伴,此时同余方程
\begin{equation}
x^2 \equiv a \pmod p ~~
\end{equation}
有解 $x = x_1$,就说 $a$ 是 $p$ 的\textbf{二次剩余(quadratic residue)},或简称为 $p$ 的\textbf{剩余},并记作 $a \opn{R} p$。
\end{definition}

\begin{definition}{二次非剩余}
若不存在任何的 $x$ 与自己相伴,此时称 $a$ 是 $p$ 的\textbf{二次非剩余(quadratic non-residue)},或简称 $p$ 的\textbf{非剩余},记作 $a \opn N p$。
\end{definition}
