% 质点系的动量
% keys 动量|动量定理|质心|质点系
% license Xiao
% type Tutor

\pentry{动量、动量定理(单个质点)\nref{nod_PLaw1}, 质心 质心系\nref{nod_CM}}{nod_df69}

质点系的总动量为
\begin{equation}\label{eq_SysMom_1}
\bvec p = \sum_i m_i \bvec v_i = \sum_i m_i \dot{\bvec r}_i = \dv{t}  \sum_i m_i \bvec r_i~.
\end{equation}
由质心的定义(\autoref{eq_CM_1}~\upref{CM}) 
\begin{equation}
\sum_i m_i \bvec r_i = M \bvec r_c~.
\end{equation}
其中 $\bvec r_c$ 为质心的位置, $M = \sum_i m_i$ 为质点系的总质量。 两边对时间求导并代入\autoref{eq_SysMom_1} 得
\begin{equation}\label{eq_SysMom_2}
\bvec p = M\bvec v_c~,
\end{equation}
其中 $\bvec v_c = \dot{\bvec r}_c$ 是质心的速度。

\autoref{eq_SysMom_2} 告诉我们一个重要的结论: \textbf{在求一个系统的总动量时, 我们可以把它等效为其质心处具有相同质量的质点}。

\begin{example}{滚动的圆盘}
一个质量为 $M$ 圆盘在地面延直线滚动, 圆心的速度为 $\bvec v$。 若将其分割为许多小份, 使用 $\sum_i m_i \bvec v_i$ (或者用积分形式)求总动量会比较麻烦。 但如果直接用\autoref{eq_SysMom_2}, 我们可以马上写出它的总动量为 $\bvec p = M\bvec v$, 甚至不需要知道它的半径和角速度, 也不需要知道它和地面是否存在打滑。
\end{example}

通过\autoref{eq_SysMom_2} 也可以直观地得出: 质心参考系中系统的总动量为零。
