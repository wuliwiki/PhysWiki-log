% 理想气体单粒子能级密度
% 理想气体|能级密度|相空间|量子力学|量子态

\subsection{相空间法}
这里只考虑单粒子共 $6$ 个自由度构成的相空间。满足能量 $<\epsilon$ 的状态数为

\begin{equation}\label{eq_IdED1_1}
\Omega_0 = \frac{1}{h^3}\int\limits_{\sum {p^2}  \leqslant 2m\epsilon} \dd[3]{q} \dd[3]{p}
 = \frac{V}{h^3}\frac43 \pi {p^3}
 = \frac{V}{h^3}\frac43 \pi (2m\varepsilon)^{3/2}
\end{equation}
相空间中的能量密度为 $(\Omega_0(E+\Delta E)-\Omega_0(E))/\Delta E(\Delta E\rightarrow 0)$,即
\begin{equation}\label{eq_IdED1_2}
a(\varepsilon) = \dv{\Omega_0}{\varepsilon} = \frac{2\pi V(2m)^{3/2}}{h^3} \varepsilon^{1/2}
\end{equation}
对于多粒子体系,也有类似公式\autoref{eq_IdSDp_2}~\upref{IdSDp}。

在这种计算方法中,我们假定了每个状态点占据相空间的体积为 $h^3$(对于单粒子而言)。然而为什么是 $h^3$ 我们不清楚,只是从量子力学的不确定原理给出了一个“说法”,没有给出证明。下面我们将从量子力学的角度来解释这件事情。
\subsection{量子力学法}
在量子力学中,束缚态的能级是分立的。盒子对粒子的波函数有一定束缚,可看作是三维的无限深方势阱\footnote{在一些教材中,会用周期性边界条件进行推导,读者不妨尝试一下,两种边界条件退出来的相格的大小都是相同的。}。单粒子的能级为
\begin{equation}
\varepsilon = \frac{\hbar ^2}{2m} \qty[\qty(\frac{\pi n_x}{L_x})^2 + \qty(\frac{\pi n_y}{L_y})^2 + \qty(\frac{\pi n_z}{L_z})^2] = \frac{\hbar ^2}{2m} (k_x^2 + k_y^2 + k_z^2)
\end{equation}
在 $k$ 空间中, 每个能级所占的体积为
\begin{equation}
V_1 = \frac{\pi^3}{L_x L_y L_z} = \frac{\pi^3}{V}
\end{equation}
$k$ 空间中, 能量小于 $E$ 的量子态数为(注意 $n$ 为正值, 所以只求一个卦限的体积, 要乘 $1/8$)
\begin{equation}
\Omega_0 = \left. \frac18\cdot \frac{\hbar^2}{2m}\frac43 \pi k^3 \middle/ \frac{\pi^2}{V} \right. = \frac{V}{h^3}\frac43 \pi(2m\varepsilon)^{3/2}
\end{equation}
这恰好就是\autoref{eq_IdED1_1} ,对能量求导得\autoref{eq_IdED1_2} 。于是我们能明白为什么要取相格的体积为 $h^3$,这样推导出的微观状态数才与量子态数相符。