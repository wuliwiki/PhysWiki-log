% 热传导定律
% keys 热传导|温度梯度|热导率|分子动能
% license Xiao
% type Tutor

\pentry{梯度 梯度定理\nref{nod_Grad}}{nod_dbab}
如果气体内各部分的温度不同,从温度较高处向温度较低处,将有热量的传递,这一现象叫做\textbf{热传导(heat conduction)}现象。

\begin{figure}[ht]
\centering
\includegraphics[width=8.5cm]{./figures/36e0ab71602be50d.pdf}
\caption{热传导现象} \label{fig_Heatco_1}
\end{figure}

如\autoref{fig_Heatco_1} 所示,$Ox$ 轴是气体温度变化最大的方向,在这个方向上气体温度的空间变化率 $\mathrm dT/\mathrm dx$,叫做\textbf{温度梯度}。设 $\Delta S$ 为垂直于 $Ox $ 轴的某指定平面的面积。实验证明,在单位时间内,从温度较高的一侧,通过这一平面,向温度较低的一侧所传递的热量,与这一平面所在处的温度梯度成正比,同时也与面积 $\Delta S$ 成正比,即得\textbf{热传导定律}:
\begin{equation}
\frac{\Delta Q}{\Delta t}=-\kappa \frac{\mathrm{d} T}{\mathrm{d} x} \Delta S~,
\end{equation}
比例系数 $\kappa$ 叫做\textbf{热导率(thermal conductivity)}。式中负号表示热量传递的方向是从高温处传到低温处,和温度梯度的方向是相反的热导率
的单位是 $\rm W /(m \cdot K)$。

实验测得,在 $0^{\circ} \mathrm{C}$ 时,氢的热导率为 $0.168 \mathrm{W} /(\mathrm{m} \cdot \mathrm{K})$,氧气 $2.42\e{-1} \mathrm{W} /(\mathrm{m} \cdot \mathrm{K})$, 空气为 $\text { 2. } 23\e{-1} \mathrm{W} /(\mathrm{m} \cdot \mathrm{K})$。在 $100^{\circ} \mathrm{C}$ 时,水汽的热导率为 $2. 18\times  10{-1}\mathrm{W} /(\mathrm{m} \cdot \mathrm{K})$。显然,气体的热导率都很小,所以,当气体中不存在对流时,气体可用作很好的绝热材料。

在气体动理论中,对气体热传导现象给出这样的解释:在温度较高的热层中,分子平均动能较大;而在温度较低的冷层中,分子平均动能较小。由于冷热两层分子的互相掺和与相互碰撞,从热层到冷层出现热运动能量的净输运。输运的热运动能量,对单原子气体来说,只是分子的平动动能;而对多原子气体来说,还包含转动和振动的能量在内。
