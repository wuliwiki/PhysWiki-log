% 薄透镜
% keys 透镜|折射|球面镜|成像|像距|物距|焦距
% license Xiao
% type Tutor

% 球形界面
% 未完成: 改名为透镜

\pentry{折射定律\nref{nod_Snel}}{nod_e37a}

\subsection{单个球面的成像公式}
\begin{figure}[ht]
\centering
\includegraphics[width=12.5cm]{./figures/6d2130f19ead8248.pdf}
\caption{单球面成像} \label{fig_ThnLen_1}
\end{figure}

如\autoref{fig_ThnLen_1}, 我们考虑两种折射率分别为 $n_1$ 和 $n_2$ 得介质被一个球面划分为左右两部分。 当光线从左边入射时, 经过界面反射。

\textbf{傍轴条件}: 图中所有角度都很小\footnote{即 $\sin\beta \approx \tan\beta \approx \beta$, 见 “\enref{小角极限}{LimArc}”}。 球面近似是平面, 球面上任意一点横坐标相同。
\begin{equation}
l_1 \alpha_1 = l_2 \alpha_2 = R\theta~.
\end{equation}
由三角形性质得
\begin{equation}
\alpha_1 = \theta_1 - \theta ~,\qquad
\alpha_2 = \theta - \theta_2~,
\end{equation}

\begin{equation}
n_1 \theta_1 = n_2 \theta_2~,
\end{equation}
消去 $\theta_1$ 和 $\theta_2~,$ 得
\begin{equation}
\frac{n_1}{l_1} + \frac{n_2}{l_2} = \frac{n_2 - n_1}{R}~,
\end{equation}
这已经比较接近凸透镜成像公式了。 注意当 $l_1$ 或者 $l_2$ 取负数时同样成立, 这意味着物或者像在透镜的另一侧(即虚物或者虚像)。 若透镜的圆心在左侧, 将式中 $R$ 也改为负数即可(请读者自行证明这两个结论)。 最后注意我们并不要求式中 $n_1, n_2$ 哪个更大。

\subsection{薄透镜}
%图未完成

如果一个透镜的两个面可以近似为球面, 且它们之间的距离比起物距和像距来可以忽略不计, 那么我们就称它为\textbf{薄透镜}。 我们假设透镜外介质的折射率为 $n_1$, 透镜内折射率为 $n_2$。 我们另透镜一侧的半径为 $R_1$, 物距为 $u$, 像距为 $v_1$, 另一侧半径为 $R_2$(另焦点在另一边为正), 物距为 $u_2$, 像距为 $v$。 于是有
\begin{equation}\label{eq_ThnLen_1}
\frac{n_1}{u} + \frac{n_2}{v_1} = \frac{n_2 - n_1}{R_1}~,
\end{equation}
\begin{equation}\label{eq_ThnLen_2}
\frac{n_2}{u_2} + \frac{n_1}{v} = \frac{n_2 - n_1}{R_2}~.
\end{equation}
由于透镜厚度可以忽略, 任何情况下都有
\begin{equation}\label{eq_ThnLen_3}
v_1 = -u_2~.
\end{equation}
将\autoref{eq_ThnLen_1} 与\autoref{eq_ThnLen_2} 相加再用\autoref{eq_ThnLen_3} 消去含有 $v_1$ 和 $u_2$ 的两项, 就得到了熟悉的\textbf{薄透镜成像公式}
\begin{equation}
\frac{1}{u} + \frac{1}{v} = \frac{1}{f}~.
\end{equation}
其中
\begin{equation}
\frac{1}{f} = \qty(\frac{n_2}{n_1} - 1) \qty(\frac{1}{R_1} + \frac{1}{R_2})~.
\end{equation}
再次声明这里使用的正负号规范: 实物距离为正, 实像距离为正, 凸透镜半径为正; 反之为负。

% 例题未完成。

\subsection{薄透镜叠加}
与上一节同理, 若两个焦距分别为 $f_1$ 和 $f_2$ 的薄透镜叠加, 当其间距远小于物距和像距时, 合成后的透镜焦距 $f$ 满足
\begin{equation}\label{eq_ThnLen_4}
\frac{1}{f} = \frac{1}{f_1} + \frac{1}{f_2}~.
\end{equation}

推导: 另第一个透镜的物距为 $u$, 像距为 $v_1$, 第二个透镜物距为 $u_2$, 像距为 $v$, 由于两透镜距离可忽略, 第一个透镜的实像像距等于第二个透镜的虚物物距, 所以分别可以列出
\begin{equation}
\frac{1}{u} + \frac{1}{v_1} = \frac{1}{f_1}~,
\end{equation}
\begin{equation}
\frac{1}{u_2} + \frac{1}{v} = \frac{1}{f_2}~,
\end{equation}
\begin{equation}
v_1 = -u_2~,
\end{equation}
联立得
\begin{equation}
\frac{1}{u} + \frac{1}{v} = \frac{1}{f_1} + \frac{1}{f_2}~.
\end{equation}
令组合透镜的等效焦距为 $f$, 即等式右边为 $1/f$, 得到\autoref{eq_ThnLen_4}。
