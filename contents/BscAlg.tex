% 代数学基本定理
% keys 复变函数|代数拓扑|基本群|连续映射|同伦|多项式
% license Xiao
% type Tutor

\begin{issues}
\issueTODO
\end{issues}

\pentry{复变函数\nref{nod_Cplx},一元多项式\nref{nod_OnePol}}{nod_1ce7}
%陆续补充不同角度的证明吧

\subsection{代数学基本定理的表述}

\begin{definition}{代数学基本定理}\label{def_BscAlg_1}
设 $f(z)=\sum\limits_{i=0}^n a_iz^i$ 为复数域上的多项式,那么必存在 $z_0\in\mathbb{C}$,使得 $f(z_0)=0$。
\end{definition}

由\autoref{cor_DivAlg_2},如果 $f(z_0)=0$,那么必然存在一个 $n-1$ 阶多项式 $g(z)$,使得 $f(z)=g(z)\cdot(z-z_0)$。对 $g(z)$ 重复此步骤,再对接下来得到的每一个次数大于 $1$ 的多项式都重复此步骤,最终结果是存在一系列 $z_i\in\mathbb{C}$,使得 $f(z)=\prod\limits_{i=0}^{n}(z-z_i)$。

\subsection{代数学基本定理的证明}

本小节将用不同思路证明代数学基本定理。

\subsubsection{代数拓扑方法}
\pentry{基本群\nref{nod_HomT3}}{nod_ea74}

考虑 $\mathbb{C}$ 的度量拓扑,其中度量 $d(z_1, z_2)=\abs{z_1-z_2}$;换句话说,把 $\mathbb{C}$ 看成二维欧氏空间。假设存在多项式 $f(z)$ 使得任意复数 $z$ 都满足 $f(z)\not=0$,那么 $\abs{f(z)}\not=0$,于是可以引入映射 $p:\mathbb{C}\rightarrow S^1$,其中 $p(z)=f(z)/\abs{f(z)}$;换句话说,$p$ 将每个 $z\in\mathbb{C}$ 先映射到 $f(z)$,再保持辐角不变并移动到单位圆上。

由于 $f(z)$ 是连续映射,且 $f(z)/\abs{f(z)}$ 没有奇点,于是 $p(z)=f(z)/\abs{f(z)}$ 也是连续映射。

在 $\mathbb{C}$ 上选择任意一点 $z_0$ 作为基点来构造基本群。设任意回路 $r:S^1\rightarrow\mathbb{C}$,以及零回路 $n:S^1\rightarrow\mathbb{C}$,其中 $n(t)=z_0$ 对所有 $t$ 成立。由于 $\mathbb{C}$ 是可缩空间,因此必有 $r\cong n$。由\autoref{the_HomT1_2},$p\circ r\cong p\circ n$,因此单位圆上的回路也彼此同伦。

然而我们知道 $S^1$ 的基本群是 $\mathbb{Z}$,意味着并非所有回路都彼此同伦,因此出现了矛盾。

矛盾意味着假设错误,也就是说$f(z)$ 必然有零点。


