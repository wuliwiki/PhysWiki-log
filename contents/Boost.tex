% C++ Boost 库笔记
% license Xiao
% type Note

\pentry{C++ 基础\nref{nod_Cpp0}}{nod_1b46}

Boost 是一些 C++ 库, 是 C++ 最常用的库之一。

\subsubsection{安装}
Ubuntu/Debian 可以直接 \verb`sudo apt install libboost-all-dev` 或者指定版本号如 \verb`libboost1.65-all-dev`。 注意这个安装非常大, 建议安装某个具体的 lib, 如 \verb`libboost-filesystem-dev` (输入 \verb`sudo apt install libboost-` 然后按 Tab 就可以显示所有子 lib)。

但如果想获取最新版本可以直接\href{https://www.boost.org/users/history/version_1_76_0.html}{下载}。 Boost 绝大部分库都可以是 header only 的。 用 \verb`tar -xzvf boost_1_76_0.tar.gz` 解压以后, 在解压文件夹中运行 \verb`sudo ./bootstrap.sh` (可以添加一些选项, 见\href{https://www.boost.org/doc/libs/1_66_0/more/getting_started/unix-variants.html#easy-build-and-install}{这里}。), 然后 \verb`sudo ./b2 install [--with-libraries=filesystem] [-j12]` 即可(默认需要 \verb`gcc` 编译器, 注意解压和编译时间可能较长)。 如果只需要头文件, 用 \verb`./b2 headers`。 然后把 \verb`boost` 文件夹复制到任何目录, 编译时 \verb`-I 目录/boost` 即可。

boost 的一些模块需要依赖第三方库, 如果没有这些库, \verb`b2` 就会跳过, 最后显示 \verb`..failed updating 54 targets... skipped 10 targets... updated 18065 targets...` 然后返回 exit code 1。 但这并不影响其他库, 详见\href{https://stackoverflow.com/questions/12906829/failed-updating-58-targets-when-trying-to-build-boost-what-happened}{这里}。

头文件默认安装路径是 \verb`/usr/local/include/boost/`, 二进制文件在子目录 \verb`/usr/local/include/lib/` 如果没有 sudo 权限也可以指定安装目录, 详情参考\href{https://www.boost.org/doc/libs/1_76_0/more/getting_started/unix-variants.html#easy-build-and-install}{安装说明}。

\subsubsection{数学物理相关}
\begin{itemize}
\item \href{https://www.boost.org/doc/libs/1_81_0/libs/safe_numerics/doc/html/index.html}{Safe Numerics} (带有溢出检测的整数以及运算)
\item \href{https://www.boost.org/doc/libs/1_81_0/libs/rational/}{Rational} (有理数)
\item \href{https://www.boost.org/doc/libs/1_81_0/libs/math/doc/html/quaternions.html}{Math Quaternion} (\enref{四元数}{Quat})
\item \href{https://www.boost.org/doc/libs/1_81_0/libs/math/doc/html/special.html}{Math/Special Functions}
\item \href{https://www.boost.org/doc/libs/1_81_0/libs/geometry/doc/html/index.html}{Geometry} (多边形,平面几何,坐标旋转,非欧几何等)
\item \href{https://www.boost.org/doc/libs/1_81_0/libs/numeric/ublas/doc/index.html}{uBLAS} (线性代数库,定义容器类)
\item \href{https://www.boost.org/doc/libs/1_81_0/doc/html/boost_random.html}{Random} (随机数)
\item \href{https://www.boost.org/doc/libs/1_72_0/libs/multiprecision/doc/html/index.html}{Multiprecision} (任意精度计算)
\item \href{https://www.boost.org/doc/libs/1_81_0/libs/math/doc/html/dist.html}{Statistical Distributions} (统计)
\item \href{https://www.boost.org/doc/libs/1_81_0/libs/numeric/odeint/doc/html/index.html}{Odeint} (常微分方程, 支持任意精度)
\item Units (量纲分析)
\end{itemize}

\subsubsection{其他}
\begin{itemize}
\item \href{https://www.boost.org/doc/libs/1_81_0/libs/sort/doc/html/index.html}{Sort} (排序)
\item \href{https://www.boost.org/doc/libs/1_81_0/doc/html/chrono.html}{Chrono} (时间工具)
\item \href{https://www.boost.org/doc/libs/1_81_0/doc/html/date_time.html}{Date Time} (时间工具)
\item \href{https://www.boost.org/doc/libs/1_81_0/libs/gil/doc/html/index.html}{Generic Image Library}(读写图片,基本色彩,变形, 像素处理)
\item \href{https://www.boost.org/doc/libs/1_81_0/libs/python/doc/html/index.html}{Python} (C++ 和 Python 之间的接口)
\item \href{https://www.boost.org/doc/libs/1_81_0/libs/regex/doc/html/index.html}{Regex} (正则表达式)
\item \href{https://www.boost.org/doc/libs/1_81_0/libs/serialization/doc/index.html}{Serialization} (把数据类型保存文件)
\item \href{https://www.boost.org/doc/libs/master/doc/html/stacktrace.html}{Stacktrace} (记录函数调用顺序)
\item \href{https://www.boost.org/doc/libs/1_81_0/doc/html/mpi.html}{MPI} (多机器并行计算)
\item \href{https://www.boost.org/doc/libs/1_81_0/libs/json/doc/html/index.html}{JSON} (读写 json 格式文件)
\item \href{https://www.boost.org/doc/libs/1_81_0/libs/filesystem/doc/index.htm}{Filesystem} (文件处理)
\item \href{https://www.boost.org/doc/libs/1_81_0/doc/html/thread.html}{Thread}
\item \href{https://www.boost.org/doc/libs/1_81_0/libs/compute/doc/html/index.html}{Compute} (GPU 并行运算库, 基于 OpenCL)
\end{itemize}


\subsection{filesystem 3}
一个获取文件大小的程序(同时显示 boost 版本号)
\begin{lstlisting}[language=cpp, caption=test\_filesystem3.cpp]
#include <iostream>
#include <boost/filesystem.hpp>
using namespace std;
using namespace boost::filesystem;

int main(int argc, char* argv[])
{
	int major_ver = BOOST_VERSION / 100000,
        minor_ver = (BOOST_VERSION / 100) % 1000,
		sub_minor_ver = BOOST_VERSION % 100;
	cout << "boost version: " << major_ver << "." << minor_ver << "."
        << sub_minor_ver << endl;
	cout << "file size: " << file_size(argv[1]) << " bytes" << endl;
	return 0;
}
\end{lstlisting}

编译 \verb`g++ test_filesystem3.cpp -o filesize -lboost_system -lboost_filesystem`, 运行 \verb`./filesize filesize`。 运行结果:
\begin{lstlisting}[language=none]
boost version: 1.65.1
file size: 26800 bytes
\end{lstlisting}

\begin{itemize}
\item \verb`boost::filesystem::path`
\end{itemize}
