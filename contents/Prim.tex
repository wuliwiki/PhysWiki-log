% 最小生成树
% Pirm|最小生成树|算法|C++

\pentry{并查集\upref{DSU}, Dijkstra\upref{SSSP}}

生成树的定义:是指在一个带权的无向联通图中选择 $n$ 个点和 $n - 1$ 条边构成的无向联通子图。

最小生成树的定义即为边权最小的生成树。

求最小生成树最常用的两种算法为:Prim 和 Kruskal。Prim 常用于稠密图,Kruskal 则相反。

\subsection{Prim 算法}

Prim 算法的思路与 Dijkstra 算法非常相似。
定义 $S$ 为当前已经确定了属于最小生成树的结点,$T$ 为集合外的结点。使用 \verb|dist| 数组存储每个结点到 $S$ 集合的距离,距离定义为如果有多个结点指向 $S$ 集合,则距离最短的边为这个结点到 $S$ 集合的距离。最开始初始化所有结点到 $S$ 集合的距离为 $+\infty$,$1$ 号点到 $S$ 集合的距离为 $0$。一共进行 $n$ 次迭代,每次找到 $T$ 集合中距离 $S$ 集合距离最短的结点 $t$,然后用 $t$ 结点更新其他点(与 $t$ 相连的结点)到 $S$ 集合的距离,然后把 $t$ 从 $T$ 集合中删去,加入到 $S$ 集合中,则 $t$ 结点为当前已经确定了属于最小生成树的结点。

具体的做法是用一个布尔数组来标记一个点是否属于 $S$ 集合,\verb|st[i]| 为 \verb|true| 则结点 $i$ 属于 $S$ 集合,反之不属于。每次从未标记的结点中选择一个 \verb|dist| 值最小的结点,把这个结点加入到 $S$ 集合中,并把这个结点的权值加到答案里,然后更新所有出边。

朴素 Prim 的时间复杂度为:$\mathcal{O}(n^2)$,使用堆优化可以达到 $\mathcal{O}(m \log_2 n)$,但是使用堆优化的 Prim 算法代码太长,不如直接用 Kruskal。

\subsubsection{Prim 算法的正确性证明}

\textbf{使用数学归纳法证明:}

证明对于 $\forall k < n$,存在一棵最小生成树包含算法前 $k$ 选择的不在 $S$ 集合且距离最近的边。

初始 $k = 1$,存在一棵最小生成树数包含第一步选择的最短的边,证明显然成立。

假设算法前 $k$ 步选择的边能包含在最小生成树中,那么第 $k + 1$ 步选择的边也包含在最小生成树中。

\begin{figure}[ht]
\centering
\includegraphics[width=10cm]{./figures/42cefb1f86b45ded.png}
\caption{图示} \label{fig_Prim_1}
\end{figure}

第 $k + 1$ 步算法选择的边为 $e_{k + 1}$,并且权值最小。同样用反证法证明,假设没有选择 $e_{k + 1}$ 这条边,选择了另一条 $e_{k + 1}'$ 这条边,如果把 $e_{k + 1}$ 选择上,会构成一条回路,但如果把 $e_{k + 1}'$ 删去,构成了一棵树,并且总权值之和不会变大。

所以算法第 $k+1$ 步仍可得到最小生成树. 

\begin{lstlisting}[language=cpp]
const int N = 510, M = 1e5 + 10, INF = 0x3f3f3f3f;
int n, m, dist[M], st[M], g[N][N];  // g 数组用邻接矩阵存储图

int prim()
{
    memset(dist, 0x3f, sizeof dist);
    dist[1] = 0;
    
    int res = 0;
    for (int i = 0; i < n; i ++ )
    {
        int t = -1; // t 表示不在 S 集合中距离 S 集合距离最短的结点,-1 表示还没找到
        for (int j = 1; j <= n; j ++ )
            if (!st[j] && (t == -1 || dist[j] < dist[t]))  
                t = j;  
        
        st[t] = true;
        if (dist[t] == INF) return INF; // 找到一个点与图是不联通的
        res += dist[t]; // 这个结点到 S 集合的边权为答案
        
        for (int j = 1; j <= n; j ++ )
            dist[j] = min(dist[j], g[t][j]); // 用 t 更新其他点的距离
    }
     
    return res;
}
\end{lstlisting}

\subsection{Kruskal 算法}

Kruskal 算法的思想较为简单,思想是基于贪心算法。具体做法是:把每条边的权值从小到大排序,然后依次枚举每条边 $(a,b,w)$,如果 $a$ 和 $b$ 不联通,就合并这两个点所在的集合,并把答案加上这条边的权值,不连通就忽略这条边。初始化每个结点都是一个集合。维护不相交森林可以用并查集,该算法的时间复杂度的瓶颈在排序上,所有 Kruskal 的时间复杂度为 $\mathcal{O}(m \log_2 m)$。最后如果最小生成树中加的边的个数小于 $n - 1$,则最小生成树构建失败。

\begin{lstlisting}[language=cpp]
int n, m, p[N];
struct Node
{
    int a, b, c;
}e[M];

int find(int x)  // 并查集
{
    p[x] = p[x] != x ? find(p[x]) : p[x];
}

int kruskal()
{
    sort(e, e + m, [&](Node &a, Node &b) {  // 将每条边的权值从小到大排序,这里使用了 Lambda
        return a.c < b.c;
    });
    
    int res = 0, cnt = 0;
    for (int i = 0; i < m; i ++ )
    {
        int a = e[i].a, b = e[i].b, c = e[i].c;
        a = find(a), b = find(b);
        if (a != b)  // 如果 a b 不连通
        {
            p[a] = b;
            res += c;
            cnt ++ ;
        }
    }
    
    if (cnt < n - 1) return INT_MAX;
    return res;
}


\end{lstlisting}