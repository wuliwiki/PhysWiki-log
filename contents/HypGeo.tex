% 超几何函数
% keys 广义超几何函数|Pochhammer符号|合流超几何函数|级数展开
% license Xiao
% type Tutor

\pentry{幂级数\nref{nod_powerS}, 连分数\nref{nod_ConFra}}{nod_3e2a}

在微积分中接触过,\textbf{几何级数}定义如下:

\begin{equation}
\sum_{n=0}^\infty z^n~.
\end{equation}

研究几何级数推广得到的\textbf{超几何函数}(或称\textbf{普通超几何函数}、\textbf{高斯超几何函数})是一个级数。很多特殊函数都是它的特例或极限。定义如下:

\begin{equation}
F(a,b;c;z) = \sum_{n=0}^\infty \frac{(a)_n(b)_n}{(c)_n} \frac{z^n}{n!}~.
\end{equation}
其中 $(a)_n = a(a+1)\dots(a+n-1)$, 叫做 \textbf{Pochhammer 符号}。

若$a,b$为非负整数,则此时$F$为有限级数和;$c$为非负整数,则$F$为无穷。

由于$(1)_n=n!$,可以发现,几何级数是超几何函数在$a=1,b=c$时的一个特例,即:
\begin{equation}
\sum_{n=0}^\infty z^n=F(1,b;b;z)~.
\end{equation}

\subsection{广义超几何函数}

超几何函数进一步推广得到的\textbf{广义超几何函数}表示为\footnote{但据说当 $p > q+1$ 时级数不收敛}
\begin{equation}
{_pF_q}(a_1,\dots, a_p; b_1, \dots, b_q; z) = \sum_{n=0}^\infty \frac{(a_1)_n\dots (a_p)_n}{(b_1)_n\dots(b_q)_n} \frac{z^n}{n!}~.
\end{equation}
其中 $(a)_n$同上,为Pochhammer 符号。

对比可以发现,超几何函数是广义超几何函数在$p=2,q=1$时的特例,即:
\begin{equation}
F(a,b;c;z)={_2F_1}(a_1,a_2; b_1; z)~.
\end{equation}
因此,一般也会将超几何函数记作${_2F_1}(a,b;c; z)$,用以明晰它和广义超几何函数的关系。同时,几何级数也是广义超几何函数在$p=1,q=0,a=1$时的特例,即:
\begin{equation}
\sum_{n=0}^\infty z^n={_1F_0}(1;; z)~.
\end{equation}
根据指数函数$e^z$的泰勒展开可以看出,它是广义超几何函数在$p=0,q=0$时的特例,即:
\begin{equation}
e^z=\sum_{n=0}^\infty\frac{z^n}{n!}={_0F_0}(;; z)~.
\end{equation}
在$p=1,q=0$时,广义超几何函数称为合流超几何函数,记作$_1F_1(a; b; z)$。一般在求平面库仑波函数使用,其级数展开为
\begin{equation}
_1F_1(a; b; z) = \sum_{n=0}^\infty \frac{(a)_n}{(b)_n} \frac{z^n}{n!}~,
\end{equation}
渐进展开为
\begin{equation}\ali{
{_1F_1}(a; b; z) &= \frac{(-1)^a\Gamma(b)}{\Gamma(b-a)} \sum_{n=0}^\infty  (-1)^n\frac{(a)_n (a-b+1)_n}{n!} z^{-n-a}\\
&+ \frac{\Gamma(b)}{\Gamma(a)} \sum_{n=0}^\infty \frac{(b-a)_n (1-a)_n}{n!} z^{-n+a-b}\E^z~,
}\end{equation}
连分数展开为
\begin{equation}
{_1F_1}(a; b; z) = 1 + \frac{az/b}{1+\dots}\ \frac{-c_1 z}{1 + c_1 z + \dots}\ \frac{-c_2 z}{1 + c_2 z +\dots}~,
\end{equation}
\begin{equation}
c_n = \frac{a + n}{(n+1)(b + n)}~.
\end{equation}
