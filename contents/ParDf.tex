% 偏导与差分
% license Xiao
% type Tutor

\begin{issues}
\issueDraft
\end{issues}

\pentry{多元泰勒展开\nref{nod_NDtalr}, 导数与差分\nref{nod_DerDif}}{nod_5d77}

对于可微的二元函数 $f(x,y)$, 可以证明
\begin{equation}
\begin{aligned}
\pdv{}{x}{y} f(x_0,y_0) = \lim_{a\to 0}\lim_{b\to 0}\frac{1}{4ab}[-& f(x_0-a,\ y_0+b) + f(x_0+a,\  y_0+b)\\
& f(x_0-a,\  y_0-b) - f(x_0+a,\  y_0-b)]~,
\end{aligned}
\end{equation}
对每个 $f$ 使用多元泰勒展开\upref{NDtalr}即可证明。

\subsubsection{Matlab 代码}

计算过程中, 由于出现了大数加小数, 所以实际上 \verb|x(i) + h| 增加的值可能比 \verb|h| 有较大不同, 可以用 \verb|(x(i) + h) - x(i)| 来计算实际的增量, 这样会使结果更精确。

注意该代码的有效数字估计只包括计算减法的截断误差, 不包括用差分代替微分的误差。
\begin{lstlisting}[language=matlab, caption=D2\_ij.m]
% 数值二阶偏导
% digi 是结果的有效数字
% D^2 f / D(x_i)D(x_j)
function [ret, digi] = D2_ij(f, i, j, x, h)
if i == j
    h = (x(i) + h) - x(i);
    if h == 0
        error('h 太小');
    end
    f2 = f(x);
    x(i) = x(i) + h;
    f3 = f(x);
    x(i) = x(i) - 2*h;
    f1 = f(x);
    val = f1 + f3;
    dif = val - 2*f2;
    ret = dif/h^2;
else % i ~= j
    x(i) = x(i) - 0.5*h; x(j) = x(j) - 0.5*h;
    hi = (x(i) + h) - x(i);
    hj = (x(j) + h) - x(j);
    if hi == 0 || hj == 0
        error('h 太小');
    end
    f1 = f(x);
    x(i) = x(i) + hi;
    f2 = f(x);
    x(j) = x(j) + hj;
    f3 = f(x);
    x(i) = x(i) - hi;
    f4 = f(x);
    val = f1+f3;
    dif = val-(f2+f4);
    ret = dif/(hi*hj);
end
if nargout == 2
    digi = max(0, 15.6536 + log10(abs(dif/val)));
end
end
\end{lstlisting}

另外给出一个变精度计算版本, 可以使截断误差精确到最后一位有效数字。 显然函数 \verb|f| 需要支持变精度 \verb|sym| 类型的变量。
\begin{lstlisting}[language=matlab, caption=D2\_ij\_vpa.m]
% 数值二阶偏导 (变精度)
% D^2 f / D(x_i)D(x_j)
% 精确到最后一位有效数字
function ret = D2_ij_vpa(f, i, j, x, h)
[ret, digi] = D2_ij(f, i, j, x, h);
if digi >= 15
    return;
end
old_digi = digits;
digits(2*15.6536 - digi - 8);
ret = D2_ij(f, i, j, arrayfun(@num2vpa, x), h);
ret = double(ret);
digits(old_digi);
end
\end{lstlisting}
