% 位置表象和动量表象
% keys 动量表象|薛定谔方程

\pentry{薛定谔方程\upref{TDSE}, 傅里叶变换与矢量空间\upref{FTvec}}

本文使用原子单位制\upref{AU}。 量子力学常见的薛定谔方程\upref{TDSE}使用的是\textbf{位置表象(position representation)}的波函数, 即波函数是位置和时间的函数。 先来看一维的情况:
\begin{equation}\label{eq_moTDSE_1}
-\frac{1}{2m} \pdv[2]{x}\psi(x, t) + V(x,t)\psi(x,t) = \I \pdv{t} \psi(x,t)~,
\end{equation}
下面要讨论另一种等效的波函数和薛定谔方程叫\textbf{动量表象(momentum representation)}。 在 “傅里叶变换与矢量空间\upref{FTvec}” 中, 我们提到可以把函数也就是这里的波函数 $\ket{\psi}$ 看作某个无穷维空间中的矢量, 在正交归一的狄拉克 $\delta$ 函数\upref{Delta}基底以及平面波\upref{PWave}基底上的展开系数(即坐标\upref{Gvec2})。

在位置表象中, 狄拉克 $\delta$ 函数 $\delta(x-x_0)$ 是位置算符 $\hat x$ 的本征函数, 而 $\exp(\I kx)/\sqrt{2\pi}$ 是动量算符 $\hat p$ 的本征函数。 若使用 $\delta(x-x_0)$ 函数基底展开波函数 $\psi(x, t)$, 系数同样为 $\psi(x, t)$。 若使用 $\exp(\I kx)/\sqrt{2\pi}$ 基底展开态矢, 坐标记为 $\varphi(k, t)$
\begin{equation}\label{eq_moTDSE_2}
\psi(x,t) = \mathcal{F}^{-1} \qty[\varphi(k,t)] = \frac{1}{\sqrt{2\pi}}\int_{-\infty}^{+\infty} \varphi(k,t)\E^{\I kx}\dd{k} ~,
\end{equation}
那么我们就使用了动量表象, $\varphi(k, t)$ 就是动量表象下的波函数。 式中 $\mathcal{F}^{-1}$ 表示反傅里叶变换\upref{FTExp}。 即\textbf{动量表象的波函数是位置表象波函数的傅里叶变换}
\begin{equation}
\varphi(k,t) = \frac{1}{\sqrt{2\pi}}\int_{-\infty}^{+\infty} \psi(x,t)\E^{-\I kx}\dd{x} = \mathcal{F}\qty[\psi(x, t)]~.
\end{equation}

可以证明, 如果 $V(x,t)$ 可以表示为关于 $x$ 的幂级数\upref{powerS}, 那么\textbf{动量表象的薛定谔方程}为
\begin{equation}\label{eq_moTDSE_3}
\frac{k^2}{2m}\varphi(k, t) + V\qty(\I \pdv{k}, t)\varphi(k, t) = \I \pdv{t} \varphi(k,t)~.
\end{equation}
这和\autoref{eq_moTDSE_1} 是等效的。
\addTODO{什么样的函数可以展开为幂级数? 除了泰勒展开。 例如方势垒可以吗?}

从直接构造哈密顿算符的角度来理解,表象无关的薛定谔方程为
\begin{equation}
\Q H\ket{\psi} = \qty(\frac{\Q p^2}{2m} + V(\Q x, t))\ket{\psi} = \I \pdv{t}\ket{\psi}~.
\end{equation}
而动量表象中 $\Q p = p$, $\Q x  = \I \pdv*{k}$ (未完成: 为什么?), 代入即可。
\addTODO{例子}

\autoref{eq_moTDSE_3} 中的势能项也可以记为卷积的形式, 动量表象薛定谔方程变为一个积分—微分方程
\begin{equation}\label{eq_moTDSE_4}
\frac{k^2}{2m}\varphi(k, t) + \int_{-\infty}^{+\infty} \mathcal{V}(k-k', t) \varphi(k', t) \dd{k'} = \I \pdv{t} \varphi(k,t)~.
\end{equation}
其中
\begin{equation}
\mathcal{V}(k, t) = \frac{1}{2\pi}\int_{-\infty}^{+\infty} V(x,t)\E^{-\I kx}\dd{x}~,
\end{equation}
是势能函数关于 $x$ 的傅里叶变换。

类似地, 动量表象得三维薛定谔方程为
\begin{equation}
\frac{\bvec k^2}{2m}\varphi(\bvec k, t) + V\qty(\I \grad_k)\varphi(\bvec k, t) = \I \pdv{t} \varphi(\bvec k,t)~.
\end{equation}
或者
\begin{equation}
\frac{\bvec k^2}{2m}\varphi(\bvec k, t) + \int \mathcal{V}(\bvec k-\bvec k', t) \varphi(\bvec k', t)\dd[3]{k'} = \I \pdv{t} \varphi(\bvec k,t)~,
\end{equation}
其中
\begin{equation}
\mathcal{V}(\bvec k, t) = \frac{1}{(2\pi)^{3}} \int V(\bvec r, t) \E^{-\I \bvec k \vdot \bvec r}\dd[3]{r}~
\end{equation}
是三维势能函数 $V(\bvec r, t)$ 的三维傅里叶变换。 %(连接未完成)

\subsection{证明}
要证明位置表象和动量表象的一维薛定谔(\autoref{eq_moTDSE_1} 和\autoref{eq_moTDSE_3})等效, \autoref{eq_moTDSE_2} 代入\autoref{eq_moTDSE_1} 得
\begin{equation}\label{eq_moTDSE_5}
\frac{1}{2m}\qty(-\I\pdv{x})^2 \mathcal{F}^{-1}\qty[\varphi(k, t)] + V(x, t)\mathcal{F}^{-1}\qty[\varphi(k, t)] = \I \pdv{t}\mathcal{F}^{-1}\qty[\varphi(k, t)]~.
\end{equation}
对第一项使用\autoref{eq_FTExp_9}~\upref{FTExp}(令 $g_1(k) = k^2/2m$, $g_2(k) = \varphi(k, t)$), 第二项使用\autoref{eq_FTExp_8}~\upref{FTExp} (令 $f_1(x) = V(x, t)$, $g_2(k) = \varphi(k, t)$, 然后两边取反傅里叶变换), 等式右边的时间偏导可以移到方括号内, 得
\begin{equation}
\mathcal{F}^{-1}\qty[\frac{k^2}{2m}\varphi(k,t)] + \mathcal{F}^{-1}\qty[V\qty(\I \pdv{k}, t)\varphi(k,t)] = \mathcal{F}^{-1}\qty[\I \pdv{t}\varphi(k,t)]~,
\end{equation}
两边取反傅里叶变换就得到\autoref{eq_moTDSE_3}。 证毕。

事实上, 对微分方程要求解的函数做傅里叶变换是解微分方程的常用技巧。 % 相关词条未完成

要证明\autoref{eq_moTDSE_4}, 可以在以上推导中不对\autoref{eq_moTDSE_5} 的势能项做处理, 最后变为
\begin{equation}
\begin{aligned}
&\quad\mathcal{F}[V(x, t)\mathcal{F}^{-1}\qty[\varphi(k, t)]]\\
&= \int_{-\infty}^{+\infty}V(x, t) \qty[\int_{-\infty}^{+\infty} \varphi(k', t) \frac{\E^{\I k'x}}{\sqrt{2\pi}} \dd{k'}] \frac{\E^{-\I kx}}{\sqrt{2\pi}} \dd{x}\\
&= \int_{-\infty}^{+\infty} \varphi(k', t)\qty[\frac{1}{2\pi}\int_{-\infty}^{+\infty} V(x, t)\E^{-\I (k-k')x} \dd{x}] \dd{k'}\\
&= \int_{-\infty}^{+\infty} \varphi(k', t) \mathcal{V}(k-k')  \dd{k'}~.
\end{aligned}
\end{equation}
这就得到了\autoref{eq_moTDSE_4}, 证毕。
