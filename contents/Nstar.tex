% 中子星
% license CCBY4
% type Wiki


中子星,顾名思义是由中子组成的天体,这是简单的理解。现在的天文研究仍然没有完全理解中子星的内部组成和结构(因为这涉及到极其复杂的广义相对论和QCD过程),这里我们归纳一下目前比较可靠的共识。

\subsection{形成和演化}
大质量恒星(> 8 倍太阳质量 )演化末期会发生核坍缩型超新星爆炸,绝大部分的恒星物质都被抛射到宇宙空间中形成超新星遗迹,在其中心剩余一个富含铁的致密的核(见主序星演化)。在极强的引力作用下,原子核都被压碎,核外电子会与原子核里的质子结合成中子,形成一个只有中子的巨大的天体。如果这个天体的质量小于TOV极限[1],中子简并压可以与引力达到平衡,就会演化为中子星,否则没有任何压力可以抵挡引力坍缩,结局就是变成恒星级黑洞。

中子星形成后会高速自转,也叫脉冲星(下面我们可能会混用中子星和脉冲星)。诞生初期主要通过中微子辐射在短时间内损失大量能量,然后在较长的时间内以坡印亭流(辐射)和脉冲星风的形式损失能量,也叫自转减慢演化(spin down evolution)。对于孤立的中子星,其自转会逐渐减慢,最终无法产生辐射和星风,成为一个“死”中子星。而如果中子星处于双星或多星系统中,它可以通过吸积伴星物质来补充能量,演化过程会更加复杂。

\subsection{密度和磁场}
众所周知是中子星一种密度极高而体积极小的天体,典型中子星的的质量约为1个太阳质量($\sim 2\times10^{30}$千克),而半径只有约10公里,使其平均密度达到了$\rm 10^{14} \, g \, cm^{-3}$(不要纠结算的是否精确,天文只关注量级,QAQ)。也就是说中子星平均密度是水的一百万亿倍,相比较一下,太阳的密度其实跟水差不多,也就是只有中子星密度的一百万亿分之一。另外,我们可以计算一下它的史瓦西半径:
\begin{equation}
R_{s} = \frac{2GM}{c^2} \approx 3 \, {\rm km} ~
\end{equation}
也就是其表面半径的三分之一,说明中子星是极其接近坍缩成黑洞的。如果未来有宇宙飞船登陆中子星表面的话,那真正是“踩在视界面的边上”。

中子星表面有极强的磁场,这个磁场来源于其前身星(即发生超新星爆炸的主序星)。前身星的表面磁场一般是100高斯左右,而半径一般是$10^6$千米。坍缩成中子星后半径缩小为了原来的$1/10^5$,根据磁通量守恒,中子星表面磁场应该是前身星的$10^{10}$倍,即$10^{12}$高斯,这也是典型的中子星表面磁场强度。一些中子星表面磁场甚至可以达到$10^{15}$高斯,相比之下,地球磁场只有不到1高斯,而最强的人造磁场大约$10^5$高斯,并且维持不了多久。

中子星的磁场结构类似于地球磁场,也有自己的北磁极、南磁极和磁轴,并且中子星的磁轴和自转轴一般也不重合。这就导致中子星的磁场会随着中子星的自转而高速转动,从而产生辐射。虽然我们一般会近似地认为脉冲星的辐射是磁偶极辐射,但磁偶极辐射集中在垂直磁轴方向,而脉冲星辐射集中在磁轴方向,这是因为脉冲星辐射主要还是磁场中带电粒子加速产生的同步辐射或曲率辐射,这些粒子的加速主要发生在磁极附近,因此脉冲星的辐射主要沿磁极成双锥形分布(见\autoref{fig_Nstar_1} )。磁极随脉冲星自转而转动,每扫过地球一次就形成一个脉冲,因此我们在地球上能观测到极为规律的脉冲信号。
\begin{figure}[ht]
\centering
\includegraphics[width=10cm]{./figures/3307e7b4e14efdc7.png}
\caption{脉冲星磁场以及辐射} \label{fig_Nstar_1}
\end{figure}

[1] Oppenheimer J R, Volkoff G M. On Massive Neutron Cores[J/OL]. Physical Review, 1939, 55(4):374-381. DOI: 10.1103/PhysRev.55.374.

[2] 图片来自于  Jake Parks, Atmospheres are key to pulsar-proofing exoplanets。链接:\href{https://www.astronomy.com/science/atmospheres-are-key-to-pulsar-proofing-exoplanets/}{图片网址}