% Julia 容器:数组(上)
% 容器:数组(上)

本文授权转载自郝林的 《Julia 编程基础》. 原文链接:\href{https://github.com/hyper0x/JuliaBasics/blob/master/book/ch09.md}{第 9 章 容器:数组(上)}.


\subsection{第 9 章 容器:数组(上)}

数组(array)也是一种容器.与元组相比,它最显著的特点有这么几个:

1. 数组是可变的对象.关于这一点,我们在前面已经见识过了.
2. 同一个数组中的所有元素值都必须有着相同的类型.虽然这个元素类型也可以是抽象类型,从而让元素值的具体类型多样化,但这样做在很多时候都会给基于它的计算带来不必要的负担.
3. 数组可以是多维(度)的.也就是说,它不只可以代表一列车队,还可以代表一个停车场、一座停车楼,以及拥有更多维度的结构.而且,数组的维数(即维度的数量)与元素类型一样,也会被写入到其类型的字面量中.

\begin{figure}[ht]
\centering
\includegraphics[width=12.5cm]{./figures/JuC9S0_1.png}
\caption{数组类型的示意} \label{JuC9S0_fig1}
\end{figure}

从这些区别上,我们可以看得出来,数组擅长的不是承载函数参数值的列表,而是存储表达形式一致的数据.它的特点非常有利于科学计算和数据分析.下面,我们就从数组的类型、值的表示和构造、常见的操作等几个方面去详细地了解一下这种容器.