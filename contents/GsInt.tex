% 高斯积分
% keys 多元微积分|极坐标|积分|二重积分|面积分|换元积分法|高斯积分
% license Xiao
% type Tutor

\pentry{极坐标中的二重积分\upref{CrIntN}}

\subsection{第一个高斯积分}

我们先来看一个重要的例子,再讲一般的高斯积分。

以下定积分被定义为一个\textbf{高斯积分}
\begin{equation}
I = \int_{-\infty}^{+\infty} \E^{-x^2}\dd{x}~.
\end{equation}

求解高斯积分最简单的方法是在极坐标中求解以下面积分
\begin{equation}
I^2 = \int_{-\infty}^{+\infty} \E^{-x^2}\dd{x} \int_{-\infty}^{+\infty} \E^{-y^2}\dd{y}
=\int_{-\infty}^{+\infty}\int_{-\infty}^{+\infty} \E^{-(x^2 + y^2)} \dd{x}\dd{y}~.
\end{equation}
在极坐标系中, $r^2 = x^2 + y^2$, 上式变为
\begin{equation}
I^2 = \int_0^{2\pi} \int_0^{+\infty} \E^{-r^2} r\dd{r}\dd{\theta}
= 2\pi \int_0^{+\infty} r\E^{-r^2} \dd{r}~.
\end{equation}
用换元积分法, 令 $t = r^2$, $\dd{t} = 2r\dd{r}$, 得
\begin{equation}
I^2 = \pi \int_0^{+\infty} \E^{-t} \dd{t} = \pi \eval{(-\E^{-t})}_0^{+\infty} = \pi~,
\end{equation}
最后开方即可得到高斯积分为
\begin{equation}\label{eq_GsInt_1}
I = \int_{-\infty}^{+\infty} \E^{-x^2}\dd{x} = \sqrt{\pi}~.
\end{equation}

更一般地, 由换元积分法\autoref{eq_IntCV_5}~\upref{IntCV} 可得
\begin{equation}\label{eq_GsInt_5}
\int_{-\infty}^{+\infty} \E^{-a x^2}\dd{x} = \sqrt{\frac{\pi}{a}} \qquad (a > 0)~.
\end{equation}

如果拓展到复数域中, 还可以证明(未完成)
\begin{equation}\label{eq_GsInt_4}
\int_{-\infty}^{+\infty} \exp(-ax^2 + bx) \dd{x} = \sqrt{\frac{\pi}{a}} \exp(\frac{b^2}{4a}) \qquad (\Re[a] > 0)~.
\end{equation}


\subsection{一般的高斯积分}

一般来说,高斯积分指的是形如
\begin{equation}
G(n)=\int_{-\infty}^{+\infty} x^n\E^{-x^2}\dd x~
\end{equation}
的定积分。

考虑到
\begin{equation}\label{eq_GsInt_3}
\int x\E^{-x^2}\dd x=-\frac{1}{2}\E^{-x^2}~.
\end{equation}
我们可以利用\textbf{分部积分法}\upref{IntBP}给 $G(n)$ 降阶:
\begin{equation}\label{eq_GsInt_2}
\begin{aligned}
G(n)&=\int_{-\infty}^{+\infty} x^{n-1}\cdot x\E^{-x^2}\dd x\\
&=x^{n-1}\cdot\left.\left(-\frac{1}{2}\E^{-x^2}\right)\right|_{-\infty}^{+\infty}+\frac{n-1}{2}\int_{-\infty}^{+\infty} x^{n-2}\E^{-x^2}\dd x\\
&=\frac{n-1}{2}\int_{-\infty}^{+\infty} x^{n-2}\E^{-x^2}\dd x\\
&=\frac{n-1}{2}G(n-2)~.
\end{aligned}
\end{equation}

因此,我们只要能计算出 $G(0)$ 和 $G(1)$,就可以利用\autoref{eq_GsInt_2} 推出所有的高斯积分了。

$G(0)$ 就是上一小节所说的 $I=\sqrt{\pi}$。$G(1)$ 就是\autoref{eq_GsInt_3} 所计算的 $-\frac{1}{2}\E^{-x^2}$。









