% 渐近展开
% 泰勒级数|收敛|发散|渐近级数

\pentry{泰勒级数\upref{Taylor}}

\subsection{定义辨析}\label{sub_Asympt_1}
在数学中, \textbf{渐近展开 (asymptotic expansion)} 是用一列较简单的函数来逐次逼近给定的函数的办法。 它的形式定义如下:

\begin{definition}{渐近展开}
设自变量 $x$ 趋于某点 $a$ (有限或无限) 时, 函数序列 $\{\phi_{n}(x)\}$ 满足
$$
\phi_{n+1}(x)=o(\phi_n(x)),\,x\to a~.
$$
则对于给定的函数 $f$, 称 $f(x)$ 在 $x\to a$ 时有渐近展开式
\begin{equation}\label{eq_Asympt_1}
f(x)\simeq c_0\phi_0(x)+c_1\phi_1(x)+...+c_n\phi_n(x)+...,\,x\to a~
\end{equation}
是指:任何给定的 $n$, 皆有
\begin{equation}\label{eq_Asympt_2}
f(x)=c_0\phi_0(x)+c_1\phi_1(x)+...+c_n\phi_n(x)+o(\phi_n(x)),\,x\to a~.
\end{equation}
\autoref{eq_Asympt_1} 中的级数称为 $x\to a$ 时 $f(x)$ 的渐近级数。
\end{definition}

一般常用的序列是单项式序列或者一般的幂函数序列。 渐近展开的四则运算性质是容易验证的。

有如下注意事项:
\begin{enumerate}
\item 一般来说, 从渐近展开不能将函数还原: 当 $x\to0$ 时, 函数 $f(x)\equiv 0$ 和函数 $g(x)=\E^{-1/x^2}$ 都有渐近展开
$$
0+0x+0x^2+...~.
$$

\item 渐近级数可以收敛也可以发散。 \autoref{eq_Asympt_1} 只是一个形式等式, 它的真正含义是\autoref{eq_Asympt_2},  而\autoref{eq_Asympt_2} 只表示对于\textbf{固定的 $n$}, 当 $x\to a$ 时, $f(x)$ 与渐近级数的第 $n$ 项部分和相差一个高阶无穷小。 

\item 通过直接计算, 可以看出渐近展开式可以逐项积分。 但一般来说渐近展开式不可以逐项微分。 例如, $f(x)=e^{-x}\sin(e^{2x})$ 当 $x\to+\infty$ 时趋于零, 但它的导数却根本没有极限。
\end{enumerate}

\subsection{基本例子}
\begin{example}{泰勒展开}
泰勒展开\upref{Taylor}就是渐近展开的例子。 泰勒级数不必收敛, 即便收敛也不必收敛到函数本身。 在函数 $\E^{-1/x^2}$ 的例子中, 原点处的泰勒级数与函数本身的差永远是函数本身, 但显然这个差在 $x\to0$ 时衰减得比 $x$ 的任意正幂都快。
\end{example}

\begin{example}{发散级数部分和}
如果 $f(x)$ 在 $x\geq1$ 时是单调不减函数, 那么有$$
\sum_{k=1}^nf(k)=\int_1^n f(x)dx+O(f(n))+O(1),\,n\to\infty~.
$$
例如, 如果取 $f(x)=1/x$, 那么有熟知的公式
$$
\sum_{k=1}^n\frac{1}{k}=\log n+O(1)~.
$$
实际上当然还可以借助更复杂的分析技巧写得再精确些, 例如我们知道 $-\log n+\sum_{k=1}^n1/k$ 的极限是存在的, 也就是欧拉常数 $\gamma=0.57721566...$。 更精细的渐近展开式可以由欧拉-麦克劳林公式给出。
\end{example}

\subsection{欧拉的例子}
考察非初等的函数
$$
f(x)=\int_0^\infty\frac{e^{-t}}{x+t}dt~,
$$
当 $x\to+\infty$ 时的行为。 欧拉将 $1/(x+t)$ 展开为几何级数
$\sum_{k=0}^\infty {(-1)^kt^k}/{x^{k+1}}$, 
代入并计算得到如下的形式等式:
\begin{equation}\label{eq_Asympt_3}
f(x)=\sum_{k=0}^\infty\frac{(-1)^kk!}{x^{k+1}}~.
\end{equation}
当然, 欧拉的时代还没有收敛性的观念。 \autoref{eq_Asympt_3} 右边的级数对于任何 $x$ 都不收敛, 之所以出现这样的问题是因为 $t$ 的几何级数收敛半径是有限的, 于是将几何级数逐项积分的计算违反分析学的准则。 

然而, 从今天的观点看, 这个等式仍然在渐近展开的意义下成立。 实际上, 通过换元可得
$$
f(x)=e^{x}\int_x^\infty \frac{e^{-t}}{t}dt~.
$$
反复进行分部积分, 得到
$$
f(x)=\sum_{k=0}^n\frac{(-1)^kk!}{x^{k+1}}
+(n+1)!e^x\int_x^\infty\frac{e^{-t}}{t^{n+2}}dt~.
$$
最后的这个积分可估算如下: 当 $t>x$ 时 $1/t^{n+2}<1/x^{n+2}$, 于是
\begin{equation}\label{eq_Asympt_4}
\int_x^\infty\frac{e^{-t}}{t^{n+2}}dt
\leq\frac{1}{x^{n+2}}\int_x^\infty e^{-t}dt
=\frac{e^{-x}}{x^{n+2}}~.
\end{equation}
这样就有
$$
f(x)=\sum_{k=0}^n\frac{(-1)^kk!}{x^{k+1}}
+O\left(\frac{1}{x^{n+2}}\right),\,x\to+\infty~.
$$
这表示\autoref{eq_Asympt_3} 在渐近展开的意义下成立。 

特别地, 这个渐近级数尽管发散, 但对于大的 $x$ 却可以很好地计算 $f(x)$ 的值, 因为根据估计\autoref{eq_Asympt_4}, $f(x)$ 同渐近级数 $n$ 项部分和之差的绝对值不超过
$$
\frac{(n+1)!}{x^{n+2}}~,
$$
而对于大的 $x$, 当 $n<x$ 时这个误差项随着 $n$ 的增大而减小, 只有当 $n$ 大致超过 $x$ 时误差才会重新开始增大。 因此, 如果将渐近级数截断到大约 $x$ 项, 则部分和与 $f(x)$ 之差将不超过 $1/2^x$。 对于大的 $x$ 来说, 这足以给出相当精确的近似值。 

由这个例子可以看出渐近展开的意义: 可以用一个发散级数去很好地逼近一个收敛的对象。