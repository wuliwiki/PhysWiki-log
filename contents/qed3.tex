% 洛伦兹群覆盖群 SL(2,C) 的不可约表示

\pentry{洛伦兹群\upref{qed1},不可约表示,李群}
\subsection{李代数的重新推导}
洛伦兹群的李代数是
\begin{equation}
\begin{aligned}
\left[J_{i}, J_{j}\right] &=i \epsilon_{i j k} J_{k}~, \\
\left[J_{i}, K_{j}\right] &=i \epsilon_{i j k} K_{k} ~,\\
\left[K_{i}, K_{j}\right] &=-i \epsilon_{i j k} J_{k}~.
\end{aligned}
\end{equation}
洛伦兹群元表达为
\begin{equation}\label{eq_qed3_1}
\Lambda=e^{-i \mathbf{J} \cdot \theta+i \mathbf{K} \cdot \phi}~.
\end{equation}
引入新算符
\begin{equation}
\mathbf{J}^{\pm}=\frac{\mathbf{J} \pm i \mathbf{K}}{2}~.
\end{equation}
经过计算,新的生成元的对易关系如下所示
\begin{equation}
\begin{aligned}
\left[J^{+, i}, J^{+, j}\right] &=i \epsilon^{i j k} J^{+, k}~, \\
\left[J^{-, i}, J^{-, j}\right] &=i \epsilon^{i j k} J^{-, k}~, \\
\left[J^{+, i}, J^{-, j}\right] &=0~.
\end{aligned}
\end{equation}
显然,这与 $SU(2)$ 群的李代数一致,也就是说洛伦兹群包含了两份SU(2)的李代数。
然而洛伦兹群不是单连通群。在数学上,给定李代数,它可对应许多李群,而其中唯一的单连通李群可以视为其他群的“母体”,因为总存在从这个单连通群到其他群的映射,反之则不一定,这个母群被数学家们称为“覆盖群”,并且称该群覆盖了同一李代数对应的其他群。通过推导洛伦兹群李代数的不可约表示,可以导出洛伦兹群的覆盖群的表示。(对于非单
连通群,不存在 Lie 代数的不可约表示和群的表示之间的一一映射)可以证明,其覆盖群为 $SL(2,C)$。$SU(2)$ 的每一不可约表示都可以用 $SU(2)$ 的Casimir元对应的标量\textbf{j}来标记。(回忆一下,Casimir算符是群元中与群的所有生成元都对易的算符。本征值在群元所作变换下不变所以可以拿来标记群表示。)设这两份 $SU(2)$ 的 $j$ 为 $j_-,j_+$,则SL(2,C)可以以($j_-,j_+$)作为群表示的标记,维数为 $j=(2j_-+1)(2j_++1)$。

\subsection{SL(2,C)与SO(3,1)关系的简单证明}
令坐标矢量为 $(x_0,x_1,x_2,x_3)$,$SO(3,1)$ 的群元为 $\Lambda$.$x'^{\nu}=\Lambda_{v}^{\mu} x^{\nu}$,洛伦兹变换使得该时空距离不变,即 $x^\nu x^\nu=x'^\nu x'^\nu$.

一般的厄米矩阵可以表示为
\begin{equation}
\chi=\left[\begin{array}{cc}
x_0+x_3 & x_1-\mathrm{i} x_2 \\
x+\mathrm{i}x_2 & x_0-x_3
\end{array}\right]~.
\end{equation}
显然,该矩阵行列式即为时空距离。对于 $\chi'=M \chi M^{*}$,该线性变换同样不会改变时空距离,即矩阵的行列式。由于 $\pm M$ 都对应同一个 $\Lambda$,所以 $SL(2,C)$ 是 $SO(3,1)$ 的双覆盖。

\subsection{SL(2,C)的几个表示}
我们通常省略洛伦兹群双覆盖的说法,而把以下几个表示称之为洛伦兹群表示。但我们依然需要明晰,它们是洛伦兹群双覆盖的表示。

\subsubsection{($0,0$)表示}
该表示维数为1。$\mathbf{J}^{\pm}=0$,则 $\mathbf{J},\mathbf{K}$ 都是0。所以是作用在标量上的简单表示。

\subsubsection{( $\frac{1}{2},0$)表示与($0,\frac{1}{2}$)表示}
这两个表示都是二维表示,且自旋为 $\frac{1}{2}$,所以它们都是作用在旋量上的表示。对于( $\frac{1}{2},0$)作用的旋量,我们标记为 $\left(\psi_{L}\right)_{\alpha},\alpha=1,2$,这就是左手的Wely旋量。($0,\frac{1}{2}$)作用的旋量,我们标记为 $\left(\psi_{R}\right)_{\alpha},\alpha=1,2$。这就是右手的Wely旋量。
\begin{equation}
\text { Weyl spinors: } \quad \psi_{L} \in\left(\frac{1}{2}, 0\right)~, \quad \psi_{R} \in\left(0, \frac{1}{2}\right)~,
\end{equation}
接下来我们来推导这两个表示的具体形式。通过定义,我们知晓,( $\frac{1}{2},0$)表示意味着 $\mathbf{J}^{-}$ 为2×2矩阵。同时,$\mathbf{J}^{+}=0$.符合这一点,也符合李代数的解是 $\mathbf{J}^{-}=\boldsymbol{\sigma} / 2$,所以
\begin{equation}
\begin{aligned}
\mathbf{J} &=\mathbf{J}^{+}+\mathbf{J}^{-}=\frac{\sigma}{2} ~,\\
\mathbf{K} &=-i\left(\mathbf{J}^{+}-\mathbf{J}^{-}\right)=i \frac{\sigma}{2}~.
\end{aligned}
\end{equation}
代入\autoref{eq_qed3_1} ,左手旋量的洛伦兹变换为
\begin{equation}
\psi_{L} \rightarrow \Lambda_{L} \psi_{L}=\exp \left\{(-i \boldsymbol{\theta}-\boldsymbol{\phi}) \cdot \frac{\boldsymbol{\sigma}}{2}\right\} \psi_{L}~.
\end{equation}
对于旋转变换我们有
\begin{equation}
\begin{aligned}
R_{x}\left(\theta_{x}\right) &=e^{-i \theta_{x} J_{x}}=e^{-i \theta_{x} \frac{1}{2} \sigma_{x}} \\
&=1-\frac{i}{2} \theta_{x} \sigma_{x}+\frac{1}{2}\left(-\frac{i}{2} \theta_{x} \sigma_{x}\right)^{2}+\cdots \\
&=\left(\begin{array}{cc}
1 & 0 \\
0 & 1
\end{array}\right)-\frac{i}{2} \theta_{x}\left(\begin{array}{cc}
0 & 1 \\
1 & 0
\end{array}\right)-\frac{1}{2}\left(\frac{-\theta_{x}}{2}\right)^{2}\left(\begin{array}{cc}
1 & 0 \\
0 & 1
\end{array}\right)+\cdots \\
&=\left(\begin{array}{cc}
1-\frac{1}{2}\left(-\frac{\theta_{x}}{2}\right)^{2}+\cdots & i \frac{-\theta_{x}}{2}+\cdots \\
i \frac{-\theta_{x}}{2}+\cdots & 1-\frac{1}{2}\left(\frac{-\theta_{x}}{2}\right)^{2}+\cdots
\end{array}\right)+\cdots \\
&=\left(\begin{array}{cc}
\cos \frac{\theta_{x}}{2} & -i \sin \frac{\theta_{x}}{2} \\
-i \sin \frac{\theta_{x}}{2} & \cos \frac{\theta_{x}}{2}
\end{array}\right)~.
\end{aligned}
\end{equation}
同样可得,
\begin{equation}
R_{y}\left(\theta_{y}\right)=\left(\begin{array}{cc}
\cos \frac{\theta_{y}}{2} & -\sin \frac{\theta_{y}}{2} \\
\sin \frac{\theta_{y}}{2} & \cos \frac{\theta_{y}}{2}
\end{array}\right)~.
\end{equation}
以及,
\begin{equation}
R_{z}\left(\theta_{z}\right)=\left(\begin{array}{cc}
\cos \frac{\theta_{z}}{2}-i \sin \frac{\theta_{z}}{2} & 0 \\
0 & \cos \frac{\theta_{z}}{2}+i \sin \frac{\theta_{z}}{2}
\end{array}\right)=\left(\begin{array}{cc}
e^{-i \frac{\theta_{z}}{2}} & 0 \\
0 & e^{i \frac{\theta_{z}}{2}}
\end{array}\right)~.
\end{equation}
对于boost(推动),我们有
\begin{equation}
\begin{array}{c}
B_{x}\left(\phi_{x}\right)=\left(\begin{array}{cc}
\cosh \frac{\phi_{x}}{2} & \sinh \frac{\phi_{x}}{2} \\
\sinh \frac{\phi_{x}}{2} & \cosh \frac{\phi_{x}}{2}
\end{array}\right) \\
B_{y}\left(\phi_{y}\right)=\left(\begin{array}{cc}
\cosh \frac{\phi_{y}}{2} & -i \sinh \frac{\phi_{y}}{2} \\
i \sinh \frac{\phi_{y}}{2} & \cosh \frac{\phi_{y}}{2}
\end{array}\right) \\
B_{z}\left(\phi_{z}\right)=\left(\begin{array}{cc}
\cosh \frac{\phi_{z}}{2}+\sinh \frac{\phi_{z}}{2} & 0 \\
0 & \cosh \frac{\phi_{z}}{2}-\sinh \frac{\phi_{z}}{2}
\end{array}\right)=\left(\begin{array}{cc}
e^{\frac{\phi_{z}}{2}} & 0 \\
0 & e^{-\frac{\phi_{z}}{2}}
\end{array}\right)~.
\end{array}
\end{equation}
显然这与矢量的群表示相差甚远。
对于($0,\frac{1}{2}$),我们可以重复以上过程。从定义中解得 $\mathbf{J}=\sigma / 2 ,\mathbf{K}=-i \sigma / 2$。所以右手旋量的变换为
\begin{equation}
\psi_{R} \rightarrow \Lambda_{R} \psi_{R}=\exp \left\{(-i \boldsymbol{\theta}+\boldsymbol{\phi}) \cdot \frac{\sigma}{2}\right\} \psi_{R}~.
\end{equation}

\subsubsection{( $\frac{1}{2},\frac{1}{2}$)表示}
根据定义,我们知道此时 $SL(2,C)$ 包含了两份独立变换的 $SU(2)$ 表示(因为 $[J^{+, i}, J^{-, j}]=0$,所以是互相独立的),维数为4维。那么它的作用对象必定也具有两个独立变量,代表在不同的SU(2)群下进行变换。
\begin{equation}
\left(\frac{1}{2}, \frac{1}{2}\right)=\left(\frac{1}{2}, 0\right) \otimes\left(0, \frac{1}{2}\right)~.
\end{equation}
从维数上看,作用对象可能是一个2×2矩阵。我们希望定义一个表示变换,使得该变换等价于常见的 $SO(3,1)$ 群作用在四维矢量的效果。这种效果上的等价性类似于二维旋转操作可以用单位复数作用到向量对应的复数上来表示,在定义了复数与实数矩阵间的映射后,旋转也可以用实矩阵作用于列向量来表示。
那么,这个2×2矩阵是什么类型的呢?首先考虑复数矩阵,一般的复数矩阵有8个自由参数,由前文知,我们的( $\frac{1}{2},\frac{1}{2}$)表示作用对象是四维的,具有4个独立参数。可以验证,任意的复数矩阵 $M$ 可以由厄米矩阵 $H(\left.H^{\dagger}=H\right)$ 和反厄米矩阵 $A(\left.A^{\dagger}=-A\right)$ 组成:$M=H+A$,而厄米矩阵和反厄米矩阵都具有四个独立参数。那么我们可以将变换限制在厄米矩阵上。即寻求( $\frac{1}{2},\frac{1}{2}$)的不可约表示,使得厄米矩阵在变换后依然是厄米矩阵(即厄米矩阵构成四维复数矩阵的不变子空间,该表示是厄米矩阵上的不可约表示),且等价于四矢量变换。(即时空距离不变)
一般的厄米矩阵以泡利矩阵为基
\begin{equation}
\sigma^{1}=\left(\begin{array}{ll}
0 & 1 \\
1 & 0
\end{array}\right)~, \quad \sigma^{2}=\left(\begin{array}{cc}
0 & -i \\
i & 0
\end{array}\right)~, \quad \sigma^{3}=\left(\begin{array}{cc}
1 & 0 \\
0 & -1
\end{array}\right)~.
\end{equation}
\begin{equation}
\sigma^{\mu}=\left(1, \sigma^{i}\right), \quad \bar{\sigma}^{\mu}=\left(1,-\sigma^{i}\right)~.
\end{equation}
\begin{equation}
v=\left(\begin{array}{cc}
v_{0}+v_{3} & v_{1}-i v_{2} \\
v_{1}+i v_{2} & v_{0}-v_{3}
\end{array}\right)=v_{0}\left(\begin{array}{cc}
1 & 0 \\
0 & 1
\end{array}\right)+v_{1}\left(\begin{array}{cc}
0 & 1 \\
1 & 0
\end{array}\right)+v_{2}\left(\begin{array}{cc}
0 & -i \\
i & 0
\end{array}\right)+v_{3}\left(\begin{array}{cc}
1 & 0 \\
0 & -1
\end{array}\right)~.
\end{equation}
在该表示下
\begin{equation}
v\rightarrow\Lambda_{L}v \bar{\Lambda}_{R}~,
\end{equation}
\begin{equation}
\bar{\Lambda}_{R}\in\Lambda_{R}=e^{\frac{1}{2}(i \theta-\phi) \cdot \sigma}~.
\end{equation}
我们不妨展开表示到一阶
那么
\begin{equation}
\begin{aligned}
v'=\Lambda_{L} v \Lambda_{R} &=\left(1-\frac{i}{2} \theta_{i} \sigma^{i}-\frac{1}{2} \phi_{i} \sigma^{i}\right)\left(v^{0}\sigma^0-v^{i} \sigma^{i}\right)\left(1+\frac{i}{2} \theta_{i} \sigma^{i}-\frac{1}{2} \phi_{i} \sigma^{i}\right) \\
&=v^{0}-v^{i} \sigma^{i}-\phi_{i} \sigma^{i} v^{0}+\frac{i}{2} \theta_{i} v^{j}\left[\sigma^{i}, \sigma^{j}\right]+\frac{1}{2} \phi_{i} v^{j}\left\{\sigma^{i}, \sigma^{j}\right\} \\
&=v^{0}-v^{i} \sigma^{i}-\phi_{i} \sigma^{i} v^{0}-\epsilon^{i j k} \theta_{i} v^{j} \sigma^{k}+\phi_{i} v^{i} \\
&=\left(v^{0}+\phi_{i} v^{i}\right)-\left(v^{i}+\phi_{i} v^{0}+\epsilon^{i j k} \theta_{j} v^{k}\right) \sigma^{i}~,
\end{aligned}
\end{equation}
这正是洛伦兹变换的无穷小形式。
\begin{equation}
\left(\begin{array}{cccc}1 & \phi_{1} & \phi_{2} & \phi_{3} \\ \phi_{1} & 1 & -\theta_{3} & \theta_{2} \\ \phi_{2} & \theta_{3} & 1 & \theta_{1} \\ \phi_{3} & -\theta_{2} & \theta_{1} & 1\end{array}\right)\left(\begin{array}{c}v^{0} \\ v^{1} \\ v^{2} \\ v^{3}\end{array}\right)=\left(\begin{array}{c}v^{0}+\phi_{1} v^{1}+\phi_{2} v^{2}+\phi_{3} v^{3} \\ v^{1}+\phi_{1} v^{0}+\theta_{2} v^{3}-\theta_{3} v^{2} \\ v^{2}+\phi_{2} v^{0}+\theta_{3} v^{1}-\theta_{1} v^{3} \\ v^{3}+\phi_{3} v^{0}+\theta_{1} v^{2}-\theta_{2} v^{1}\end{array}\right)~.
\end{equation}
所以厄米矩阵是广义的四维协变矢量,洛伦兹变换最基本的作用对象是旋量。
可以证明 $\xi_{R}^{\dagger} \sigma^{\mu} \psi_{R}$ 和 $\xi_{L}^{\dagger} \bar{\sigma}^{\mu} \psi_{L}$ 也是协变的四维矢量。
