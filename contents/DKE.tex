% 勒内·笛卡尔(综述)
% license CCBYSA3
% type Wiki

本文根据 CC-BY-SA 协议转载翻译自维基百科\href{https://en.wikipedia.org/wiki/Ren\%C3\%A9_Descartes}{相关文章}。

勒内·笛卡尔(/deɪˈkɑːrt/ day-KART 或英国发音:/ˈdeɪkɑːrt/ DAY-kart;法语:[ʁəne dekaʁt] ⓘ;[注3][11] 1596年3月31日 – 1650年2月11日)[12][13]: 58  是法国哲学家、科学家和数学家,被广泛认为是现代哲学和科学兴起的奠基人物之一。数学在他的研究方法中至关重要,他将几何与代数相结合,创立了解析几何。笛卡尔的职业生涯大部分时间是在荷兰共和国度过的,最初在荷兰国军服役,后来成为荷兰黄金时代的核心知识分子。[14] 尽管他服务于一个新教国家,且后来被批评者视为自然神论者,笛卡尔实际上是罗马天主教徒。[15][16]

笛卡尔哲学的许多元素可以在晚期的亚里士多德主义、16世纪复兴的斯多葛主义或更早的哲学家如奥古斯丁的思想中找到前例。在他的自然哲学中,他在两个主要方面不同于当时的学派。首先,他拒绝将有形实质划分为质料和形式;其次,他拒绝在解释自然现象时诉诸于神或自然的终极目的。[17] 在神学中,他坚持神创造行为的绝对自由。笛卡尔拒绝接受前人哲学家的权威,常常将自己的观点与之前的哲学家区分开来。在《灵魂的激情》开篇中,这部早期现代情感论著中,笛卡尔甚至声称他将“仿佛从未有人写过这些问题一样”来论述该主题。他最著名的哲学陈述是“我思故我在”(拉丁语:cogito, ergo sum;法语:Je pense, donc je suis),出现在《方法谈》(1637年,以法语和拉丁语写成,1644年)和《哲学原理》(1644年拉丁语版,1647年法语版)中。[注4] 这一陈述要么被解释为逻辑三段论,要么被视为一种直觉思想。[18]

笛卡尔常被称为现代哲学之父,被广泛认为是17世纪对认识论关注增加的主要推动者。[19][注5] 他为17世纪大陆理性主义奠定了基础,后由斯宾诺莎和莱布尼茨提倡,但后来受到霍布斯、洛克、贝克莱和休谟组成的经验主义学派的反对。早期现代理性主义的兴起——作为历史上首次独立的系统性哲学学派——对现代西方思想产生了广泛影响,诞生了笛卡尔(笛卡尔主义)和斯宾诺莎(斯宾诺莎主义)两个理性主义哲学体系。正是17世纪的理性主义大师如笛卡尔、斯宾诺莎和莱布尼茨赋予了“理性时代”其名称和历史地位。莱布尼茨、斯宾诺莎[20] 和笛卡尔都在数学和哲学方面造诣颇深,笛卡尔和莱布尼茨还在多个科学领域有所贡献。[21] 尽管只有莱布尼茨被广泛认可为博学家,这三位理性主义者都在各自的著作中整合了不同的知识领域。[22]

笛卡尔的《第一哲学沉思》(1641年)至今仍是大多数大学哲学系的标准教材。笛卡尔在数学方面的影响同样显著,以他的名字命名了笛卡尔坐标系。他被誉为解析几何之父,这一数学分支后来用于微积分和数学分析的发现。笛卡尔也是科学革命的关键人物之一。
\subsection{生平}
\subsubsection{早年生活}

笛卡尔出生的房子位于图赖讷的拉艾

勒内·笛卡尔于1596年3月31日出生在法国图赖讷省的拉艾(现今法国安德尔-卢瓦尔省的笛卡尔镇)。[23] 1597年5月,他的母亲让娜·布罗沙在生下一个死胎后几天去世。[24][23] 笛卡尔的父亲乔阿希姆是雷恩议会的成员。[25]: 22 勒内与祖母和叔祖一起生活。尽管笛卡尔一家是罗马天主教徒,但普瓦图地区当时由新教徒胡格诺派控制。[26] 1607年,由于体弱多病,笛卡尔晚入学,进入位于拉弗莱什的耶稣会皇家亨利-勒格朗学院。[27][28] 在那里,他接触了数学和物理学,包括伽利略的作品。[29][30] 在此期间,笛卡尔首次接触到赫尔墨斯神秘主义。1614年毕业后,他在普瓦捷大学学习了两年(1615-1616年),于1616年获得教会法和民法的学士和执照学位,[29] 这符合其父亲希望他成为律师的愿望。[31] 随后,他前往巴黎。

笛卡尔在普瓦捷大学的毕业注册记录,1616年

在《方法谈》中,笛卡尔回忆道:[32]: 20–21

我完全放弃了对书本的学习,决定不再追求任何除了可以在自己内心或伟大的‘世界之书’中找到的知识。我将余下的青春岁月用于旅行,拜访宫廷和军队,与不同性格和地位的人交往,积累各种经验,在命运带给我的各种情境中考验自己,并始终反思遇到的一切,以从中获得一些收获。