% YAML 笔记
% license Xiao
% type Note

\begin{issues}
\issueDraft
\end{issues}

\pentry{JSON 笔记\nref{nod_json}}{nod_81b9}

\begin{itemize}
\item \textbf{YAML Ain't Markup Language}
\item 许多设置文件的格式都是 yaml 的(后缀名 \verb`.yaml` 或 \verb`.yml`)
\item yaml 禁止 tab, 用空格缩进(数量任意)
\item yaml 几乎和 json 一样, 有 dictionary (key-value 对), value 也可以是别的 dictionary 或者 array (list)
\item dic 可以写成一个 pair 一行, 也可以写成类似 json 的 \verb`{key : val, key : val}`
\item array 的元素可以每行用一个 \verb`-` 开头, 也可以写成类似 json 的 \verb`[val, val, val]`
\item 一个转换器见\href{https://onlineyamltools.com/convert-yaml-to-json}{这里}
\end{itemize}

一个转换例子
\begin{lstlisting}[language=none]
key1:
    key1.1: val1.1
    key1.2: val1.2
    key1.3:
        key1: 1
        key2: 2
    key1.4: {key1: 1, key2: 2}
    key1.5:
        - 1
        - 2
        - 3
    key1.6: [1, 2, 3]
    key1.7:
        - a: b
          c: d
        - e: f
          g: h
\end{lstlisting}
对应的 JSON
\begin{lstlisting}[language=json]
{
  "key1": {
    "key1.1": "val1.1",
    "key1.2": "val1.2",
    "key1.3": {"key1": 1, "key2": 2},
    "key1.4": {"key1": 1, "key2": 2},
    "key1.5": [1, 2, 3],
    "key1.6": [1, 2, 3],
    "key1.7": [{"a": "b", "c": "d"}, {"e": "f", "g": "h"}]
  }
}
\end{lstlisting}
