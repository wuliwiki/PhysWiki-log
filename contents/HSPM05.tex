% 圆周运动(高中)
% keys 圆周运动|线速度|角速度|周期|向心力|向心加速度

\pentry{曲线运动\upref{HSPM04}}

质点的运动轨迹是圆(或圆弧的一部分)的运动叫做\textbf{圆周运动}。

\subsection{线速度}

\begin{itemize}
\item 质点做圆周运动时,为了描述它经过某个点 $M$ 附近时运动的快慢,我们可以取一段很短的时间 $\Delta t$ 以及从点 $M$ 开始运动的弧长 $\Delta l$,$\Delta l$ 与 $\Delta t$ 之比就反映了质点在点 $M$ 附近的运动快慢。

\item 当 $\Delta t$ 极短时,可以认为 $\Delta l$ 与 $\Delta t$ 之比表示质点在点 $M$ 的运动快慢,将其称为线速度,其大小为
\begin{equation}
v=\frac{\Delta l}{\Delta t}~.
\end{equation}
此时,质点运动的弧可近似看成线段,$\Delta l$ 近似为质点的位移大小,则线速度实际上就是直线运动中的瞬时速度。

\item 线速度是一个矢量,既有大小,也有方向。质点在某一点的线速度方向沿着圆周在该点的切线方向。因为圆周运动中运动方向(线速度方向)时刻在变,所以圆周运动是变速曲线运动。

\item 线速度的定义也适用于圆周运动外的其他曲线运动。

\item 当质点做圆周运动且线速度的大小处处相等时,物体所做的运动叫做\textbf{匀速圆周运动},我们可以用质点通过的弧长 $l$ 与通过这段弧长所用时间 $t$ 之比来表示线速度的大小
\begin{equation}
v=\frac{l}{t}~.
\end{equation}
要注意的是,匀速圆周运动的“匀速”只是描述速率不变,匀速圆周运动依然是变速运动。
\end{itemize}

\subsection{角速度}

质点做圆周运动时,质点所在的半径在转过的角度 $\Delta \theta$ 与所用时间 $\Delta t$ 之比叫做角速度,用 $\omega$ 表示,即
\begin{equation}
\omega = \frac{\Delta \theta}{\Delta t}~.
\end{equation}

此处的 $\Delta \theta$ 以弧度为单位,因此角速度的单位为 $\mathrm{rad/s}$。

设圆周运动轨迹的半径为 $r$,对于相同的时间内,有 $\Delta l=r \cdot \Delta \theta$,由此可得
\begin{equation}\label{eq_HSPM05_3}
v=\omega r~.
\end{equation}

\subsection{匀速圆周运动的周期性}

\subsubsection{周期}

匀速圆周运动是周期性运动,做圆周运动的质点运动一周所用的时间叫做\textbf{周期},用 $T$ 表示。质点在做匀速圆周运动时,在圆周上的任意一点每经过周期的整数倍周期的时间,都会回到原来的位置,且线速度的大小和方向与原来的一致。

做匀速圆周运动的质点在一个周期 $T$ 内经过的弧长为 $2\pi r$,转过的角度为 $2\pi$,则线速度的大小与周期的关系为
\begin{equation}\label{eq_HSPM05_1}
v=\frac{2\pi r}{T}~.
\end{equation}
角速度的大小与周期的关系为
\begin{equation}\label{eq_HSPM05_2}
\omega = \frac{2\pi}{T}~.
\end{equation}
联立\autoref{eq_HSPM05_1} 和\autoref{eq_HSPM05_2} 也可得\autoref{eq_HSPM05_3} .

\subsubsection{转速}

物体转动的圈数与所用时间之比叫做\textbf{转速},用符号 $n$ 表示,常用单位是转每秒($\mathrm{r/s}$)或转每分($\mathrm{r/min}$),实际计算时一般把转速换算成弧度每秒($\mathrm{rad/s}$)。

\subsection{向心力}

做匀速圆周运动的物体所受的合力总是指向圆心,这个指向圆心的力叫做\textbf{向心力}。

质点做匀速圆周运动时,向心力的方向时刻在变化,且始终与线速度的方向垂直。向心力只改变线速度的方向,不改变线速度的大小。

向心力的大小:
\begin{equation}\label{eq_HSPM05_4}
F=\frac{mv^2}{r}=m\omega^2r=m\omega v=m\frac{4\pi^2}{T^2}r~.
\end{equation}

向心力是根据力的作用效果来命名的,向心力可以是某种性质的力如重力、弹力等,也可以是多个力的合力,也可以是某个力的分力。

向心力的大小和方向适用于所有圆周运动。当质点做匀速圆周运动时,合力提供向心力;当质点做变速圆周运动时,合力指向圆心的分力提供向心力。

\subsubsection{质点做匀速圆周运动的条件}

质点绕某一个点做匀速圆周运动时,需要具有一定的初速度,以及受到方向总是垂直于速度方向的合力 $\bvec F$,$\bvec F$ 的大小应等于质点做圆周运动所需的向心力大小,即满足\autoref{eq_HSPM05_4} 。

当 $F>mv^2/r$ 时,质点做半径变小的向心运动;当 $F<mv^2/r$ 时,合力不足以提供维持圆周运动所需的向心力,质点做离心运动。

\subsection{向心加速度}

质点做匀速圆周运动时,其加速度总是指向圆心,这个加速度叫做\textbf{向心加速度}。

向心加速度的大小:
\begin{equation}
a=\frac{v^2}{r}=\omega^2r=\omega v=\frac{4\pi^2r}{T^2}~.
\end{equation}

向心加速度是由向心力产生的,其大小和方向同样适用于所有圆周运动。
