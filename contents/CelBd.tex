% 开普勒问题
% keys 平方反比|开普勒问题|圆锥曲线轨道
% license Xiao
% type Tutor

\pentry{万有引力\nref{nod_Gravty}, 角动量\nref{nod_AMLaw1}, 椭圆\nref{nod_Elips3}, 双曲线\nref{nod_Hypb3}, 抛物线\nref{nod_Para3}}{nod_36b1}

在\enref{中心力场问题}{CenFrc} 中,若 $F(r)$ 是平方反比的力(斥力为正引力为负), 即
\begin{equation}\label{eq_CelBd_1}
F(r) = \frac{k}{r^2}  \qquad V(r) = \frac{k}{r}~,
\end{equation}
则该问题被称为\textbf{开普勒问题}。 其中 $k$ 为非零实数。 例如对于万有引力 $k = -GMm$, 对于异种电荷间的\enref{库仑力}{ClbFrc},有\footnote{高中所学的库仑定律的系数 $k$ 在大学物理中通常记为 $1/(4\pi\epsilon_0)$, 其中 $\epsilon_0$ 为真空中的电介质常数, 详见 “\enref{库仑定律}{ClbFrc}”。} $k = Qq/(4\pi\epsilon_0)$。

在开普勒问题中, 可以\enref{证明}{Keple1}质点的运动轨道是圆锥曲线的一种, 力心处于焦点。 质点的机械能(动能加势能) $E$ 和\enref{角动量}{AngMo} $L$ 可以唯一地确定轨道的形状和大小。 轨道的形状一般由离心率 $e$ 描述, 大小由半通径 $p$ 描述(\autoref{eq_Cone_5}~\upref{Cone})。 $E < 0$ 对应椭圆轨道, $E = 0$ 对应抛物线轨道\footnote{显然只有引力($k < 0$)可以产生非正的机械能, 即椭圆轨或抛物线轨道。}, $E > 0$ 对应双曲线轨道。 注意双曲线轨道有两支, 当 $k < 0$ (引力)时取离中心天体较近的一支, $k > 0$ (斥力)时取较远的一支。
\begin{equation}\label{eq_CelBd_2}
e = \sqrt{1 + \frac{2EL^2}{mk^2}}~,
\end{equation}
\begin{equation}\label{eq_CelBd_3}
p = \frac{L^2}{m\abs{k}}~,
\end{equation}
椭圆或双曲线的大小和形状也可以由参数 $a,b$ 描述。 $a,b$ 与 $e,p$ 的对应关系见“\enref{椭圆}{Elips3}”和“\enref{双曲线}{Hypb3}”。
\begin{equation}\label{eq_CelBd_7}
a = \frac{\abs{k}}{2\abs{E}}~,
\end{equation}
\begin{equation}\label{eq_CelBd_8}
b = \frac{L}{\sqrt{2m\abs{E}}}~,
\end{equation}
证明见下文。 若要求位置和时间的关系, 见 “\enref{开普勒问题的运动方程}{EqMoKp}”。

\begin{example}{}
由\autoref{eq_CelBd_7} 可见对同一个中心天体, 半长轴 $a$ 仅和物体的机械能 $E$ 有关。 例如在地球表面同一位置, 以同一初速度向不同角度发射弹道导弹, 它的半长轴 $a$ 将始终相同但离心率 $e$ 或半短轴 $b$ 将不同。

另外注意除了水平发射, 其他任何角度以小于第二宇宙速度发射都会重新撞向地面。 我们在 “\enref{开普勒问题的数值计算(Matlab)}{KepNum}” 中给出了一个数值计算和画图程序。
\end{example}

\subsection{证明}
如果我们已知质点轨道为圆锥曲线, 只需要简单的代数方法就可以得到上述关系。 而证明轨道是圆锥曲线则要复杂得多, 见 “\enref{开普勒第一定律的证明}{Keple1}”。

\subsubsection{椭圆轨道}
令椭圆轨道($k<0$)距离焦点的最近和最远距离分别为 $r_1$ 和 $r_2$,列出总能量(动能加势能)守恒
\begin{equation}\label{eq_CelBd_4}
\frac12 m v_1^2 + \frac{k}{r_1} = \frac12 mv_2^2 + \frac{k}{r_2}~,
\end{equation}
以及角动量守恒
\begin{equation}\label{eq_CelBd_5}
mv_1 r_1 = mv_2 r_2~.
\end{equation}
把\autoref{eq_CelBd_5} 中的 $v_2$ 代入\autoref{eq_CelBd_4},可得
\begin{equation}\label{eq_CelBd_6}
v_1^2 = \frac{-2k/m}{r_1 + r_2} \frac{r_2}{r_1}~.
\end{equation}
代入\autoref{eq_CelBd_4} 的左边,并使用 $r_1+r_2=2a$ (\autoref{eq_Elips3_9}~\upref{Elips3})得到总能量
\begin{equation}\label{eq_CelBd_9}
E = \frac{k}{2a}~.
\end{equation}
把\autoref{eq_CelBd_6} 代入\autoref{eq_CelBd_5} 的左边,并使用 $r_1 r_2 = (a+c)(a-c) =b^2$ % \addTODO{此处该引用公式}
得角动量
\begin{equation}\label{eq_CelBd_10}
L = b\sqrt{\frac{-mk}{a}}~,
\end{equation}
将\autoref{eq_CelBd_9} 和\autoref{eq_CelBd_10} 逆转即可得到\autoref{eq_CelBd_7} 和\autoref{eq_CelBd_8}。 要得到\autoref{eq_CelBd_2} \autoref{eq_CelBd_3}, 只需使用\autoref{eq_Elips3_7}~\upref{Elips3} 和\autoref{eq_Elips3_8}~\upref{Elips3} 即可。

\subsubsection{抛物线轨道}
已知抛物线轨道($k<0$)的总能量为零, 抛物线轨道离焦点的最近距离为焦距 $p/2$, 该点处, 角动量和能量为
\begin{equation}
L = mv_0 \frac p2~,
\end{equation}
\begin{equation}
0 = E = \frac 12 mv_0^2 + \frac{k}{p/2}~,
\end{equation}
两式消去 $v_0$ 得角动量为 $L = \sqrt{mkp}$。 证毕。

\subsubsection{双曲线轨道}
无论 $k$ 的正负如何, 令双曲线轨道离焦点最近的距离为 $r_1$, 可列出总能量守恒
\begin{equation}\label{eq_CelBd_11}
\frac12 mv_0^2 = \frac12 mv_1^2 + \frac{k}{r_1}~.
\end{equation}
该式左边表示质点在无穷远处的总能量, 此时势能为 $0$, 总能量等于动能。再来看角动量守恒
\begin{equation}\label{eq_CelBd_12}
m v_0 b = m v_1 r_1~,
\end{equation}
该式左边为无穷远处的角动量。 由\autoref{eq_Hypb3_11}~\upref{Hypb3} 可知, 在无穷远处, 双曲线的渐近线(无论哪一支)与焦点的距离为 $b$。

用以上两式消去 $v_1$, 再利用 $r_1 = a - c$, 得
\begin{equation}\label{eq_CelBd_13}
E = \frac 12 m v_0^2 = \frac{\abs{k}}{2a}~,
\end{equation}
再将该式的 $v_0$ 代入\autoref{eq_CelBd_12} 左边得
\begin{equation}
L = b\sqrt{\frac{m\abs{k}}{a}}~.
\end{equation}
