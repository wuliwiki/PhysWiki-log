% 隐函数
% 隐函数|显函数|长方体

\subsection{一元隐函数}
假定二变元 $x$ 及 $y$ 的值用方程联系着,若把这方程的一切项移到左边,则得到一般的形式
\begin{equation}\label{eq_ImpFun_1}
F(x,y)=0~.
\end{equation}
此处 $F(x,y)$ 是在某一区域给定的二元函数。

若函数 $y=f(x)$ 由\autoref{eq_ImpFun_1} 给定,但未解出,则称其为\textbf{隐函数};若 $y$ 对 $x$ 的关系 $y=f(x)$ 被(直接)解出来,就称为\textbf{显函数}。
\subsection{多元隐函数}
同一元隐函数一样,若多个变元的方程
\begin{equation}\label{eq_ImpFun_5}
F(x_1,\cdots,x_n,y)=0~,
\end{equation}
确定 $y$ 为 $n$ 个变元 $x_1,\cdots,x_n$ 的函数 $y=f(x_1,\cdots,x_n)$,若未解出具体的表达式,则 $y=f(x_1,\cdots,x_n)$ 称为\textbf{$n$ 元隐函数}。 

在一般的情形,若 $m$ 个方程组
\begin{equation}\label{eq_ImpFun_6}
\begin{aligned}
F_1(x_1,\cdots,x_n;y_1\cdots,y_m)&=0~,\\
&\ \vdots\\
F_m(x_1,\cdots,x_n;y_1\cdots,y_m)&=0~.
\end{aligned}
\end{equation}
确定 $m$ 个变元 $y_1,\cdots,y_m$ 为另 $n$ 个变元 $x_1,\cdots,x_n$ 的函数
\begin{equation}\label{eq_ImpFun_2}
y_1=f_1(x_1,\cdots,x_n);\cdots;y_m=f_m(x_1,\cdots,x_n)~
\end{equation}
但未解出,则\autoref{eq_ImpFun_2} 称\textbf{ $m$ 个变元 $y_1,\cdots,y_m$ 为 $n$ 个变元 $x_1,\cdots,x_n$ 的隐函数}。 
\subsection{隐函数的单值性}
方程\autoref{eq_ImpFun_1} 确定的函数 $y=f(x)$ 可以是多值函数,即存在某一个 $x$,使得对于几个 $y$ 值,它们与 $x$ 同时满足方程\autoref{eq_ImpFun_1} 。

我们关心的是隐函数的存在及单值性问题。先来解释这一问题的几何意义。方程\autoref{eq_ImpFun_1} 在某种场合下表示为平面曲线,在这种情形,它称为\textbf{隐示方程},问题归结为:曲线(或它的一部分)\autoref{eq_ImpFun_1} 能否用右边单值的函数的普通方程 $y=f(x)$ 来表示。几何意义是,曲线(或它的一部分)与平行于 $y$ 轴的直线仅相交于一点。例如,方程 
\begin{equation}\label{eq_ImpFun_3}
\frac{x^2}{a^2}+\frac{y^2}{b^2}-1=0~
\end{equation}
确定一个椭圆(\autoref{fig_ImpFun_1} )。它在 $[-a,a]$ 内显然为双值函数。
\begin{figure}[ht]
\centering
\includegraphics[width=5cm]{./figures/b0f37da79d2e03dd.pdf}
\caption{隐示方程\autoref{eq_ImpFun_3} 确定的椭圆曲线} \label{fig_ImpFun_1}
\end{figure}
通常我们只关心\autoref{eq_ImpFun_1} 的曲线上某一点附近的领域 $y$ 对 $x$ 的单值性。例如在椭圆(\autoref{fig_ImpFun_1} )的情形,显然可以判定,方程\autoref{eq_ImpFun_3} ,除椭圆长轴上的顶点 $A,A''$ 外,在椭圆上任一点的充分小的领域内确定纵标 $y$ 为横标 $x$ 的单值函数。为了描述某一点附近的邻域(在直角坐标中通常为长方形邻域,比如\autoref{fig_ImpFun_1} 中椭圆曲线上的小长方形邻域),需先了解一般高维空间中的长方体的表示方式。
\begin{definition}{n维“长方体”}\label{def_ImpFun_2}
$n$ 维空间中,坐标各自相互独立地满足于不等式
\begin{equation}\label{eq_ImpFun_4}
a_1\leq x_1\leq b_1,\cdots a_n\leq x_n\leq b_n~
\end{equation}
的一切点 $M(x_1,\cdots,x_n)$ 所成的集,称为\textbf{ $n$ 维闭的长方体}。记为
\begin{equation}
[a_1,b_1;\cdots;a_n,b_n]~
\end{equation}
若在不等式\autoref{eq_ImpFun_4} 中去掉等号,就可定义\textbf{$n$ 维开的长方体},记作
\begin{equation}
(a_1,b_1;\cdots;a_n,b_n)~
\end{equation}

\end{definition}
1维长方体通常称\textbf{线段},2维长方体称为\textbf{长方形(或矩形)},通常空间的长方体则对应三维长方体。
\begin{definition}{一元隐函数在某一点附近的单值性}\label{def_ImpFun_1}
在矩形 $(a,b;c,d)$ 内方程\autoref{eq_ImpFun_1} $F(x,y)=0$ 确定 $y$ 为 $x$ 的\textbf{单值函数},若在区间 $(a,b)$ 内的 $x$ 的每一值,在区间 $(c,d)$ 内方程\autoref{eq_ImpFun_1} 有且仅有一个根 $y$。
\end{definition}

\begin{definition}{多元隐函数在一点附近的单值性}
在 $n+1$ 维长方体 $(a_1,b_1;\cdots;a_n,b_n;c,d)$ 中,方程\autoref{eq_ImpFun_5} $F(x_1,\cdots,x_n,y)=0$ 确定 $y$ 为 $x_1,\cdots,x_n$ 的\textbf{单值函数},若在 $n$ 维长方体 $(a_1,b_1;\cdots;a_n,b_n)$ 中任一点 $(x_1,\cdots,x_n)$,方程\autoref{eq_ImpFun_5} 有且仅有一个根 $y$ 位于区间 $(c,d)$ 内。
\end{definition}

\begin{definition}{一般隐函数在一点附近的单值性}
在 $n+m$ 维长方体 $(a_1,b_1;\cdots;a_n,b_n;c_1,d_1;\cdots;c_m,d_m)$ 中,方程组\autoref{eq_ImpFun_6} 
\begin{equation}
\begin{aligned}
&F_1(x_1,\cdots,x_n;y_1\cdots,y_m)=0\\
\vdots\\
&F_m(x_1,\cdots,x_n;y_1\cdots,y_m)=0
\end{aligned}~
\end{equation}
确定 $y_1,\cdots,y_m$ 为 $x_1,\cdots,x_n$ 的\textbf{单值函数},若在 $n$ 维长方体 $(a_1,b_1;\cdots;a_n,b_n)$ 中任一点 $(x_1,\cdots,x_n)$,方程组\autoref{eq_ImpFun_6} 有且仅有一组属于 $m$ 维长方体 $(c_1,d_1;\cdots;c_m,d_m)$ 的解 $y_1,\cdots,y_m$。
\end{definition}