% 双缝干涉中一个重要极限
% 干涉|泰勒级数|极限|几何光学

\pentry{泰勒级数\upref{Taylor}}

该近似最常用于波动理论, 如光的远场干涉衍射。
\begin{equation}
\abs{\bvec r - \bvec r_0} = r - \uvec r \vdot \bvec r_0 = r - r_0 \cos\theta~,
\end{equation}
$\theta$ 为 $\bvec r$ 与 $\bvec r_0$ 的夹角。 这个近似的误差远小于 $r_0$。

该近似的几何意义是, 若三角形的两条边远长于第三条, 那么这两条边的长度差等于第三条在任意一条临边上的投影。 % (图未完成)

\subsection{推导}

令 $\varepsilon = r_0 / r$, $\uvec r \vdot \uvec r_0 = \cos\theta$, 该命题变为证明
\begin{equation}
\lim_{\varepsilon \to 0} \frac{1}{r_0} \qty[\abs{\bvec r - \bvec r_0} - (r - r_0 \cos\theta)] = 0~,
\end{equation}
或
\begin{equation}
\lim_{\varepsilon \to 0} \frac{1}{r_0} \qty[\sqrt{r^2 + r_0^2 - 2rr_0\cos\theta} - (r - r_0 \cos\theta)] = 0~.
\end{equation}
把一个 $r$ 提出中括号外, 得
\begin{equation}
\lim_{\varepsilon \to 0} \frac{1}{\varepsilon} \qty[\sqrt{1 + \varepsilon^2 - 2\varepsilon\cos\theta} - (1 - \varepsilon \cos\theta)] = 0~.
\end{equation}
现在把根号对 $\varepsilon$ 进行泰勒展开, 保留其一阶无穷小, 得
\begin{equation}
\sqrt{1 + \varepsilon^2 - 2\varepsilon\cos\theta} = 1 - \varepsilon \cos\theta + \order{\varepsilon^2}~,
\end{equation}
代入得
\begin{equation}
\lim_{\varepsilon \to 0} \frac1\varepsilon \order{\varepsilon^2} = \lim_{\varepsilon \to 0} \order{\varepsilon} = 0~.
\end{equation}
证毕。
