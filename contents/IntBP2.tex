% 分部积分的高维拓展
% keys 分部积分|散度|体积分|面积分
% license Xiao
% type Tutor

\pentry{矢量算符常用公式\upref{VopEq}, 牛顿—莱布尼兹公式的高维拓展\upref{NLext}}{nod_ea72}

以下同一公式中所有体积分 $\int \dots \dd{V}$ 都是在某个有限区域 $\mathcal V$ 进行, 所有的曲面积分\upref{SurInt} $\int \dots \dd{\bvec s}$ 都在 $\mathcal V$ 的边界 $\mathcal S$ 上进行, 正方向向外。

\subsection{标量分部积分}
这是最接近一元函数分部积分的高维拓展
\begin{equation}\label{eq_IntBP2_2}
\int (\grad f) g \dd{V} = \oint fg \dd{\bvec s} - \int f (\grad g) \dd{V}~.
\end{equation}
对于二维情况, 面积分变为线积分, 方向沿逆时针。 一维情况就一元函数分部积分。

\subsubsection{证明}
把\autoref{eq_VopEq_2}~\upref{VopEq} 两边积分, 移项得
\begin{equation}
\int (\grad f) g \dd{V} = \int \grad (fg) \dd{V} - \int f (\grad g) \dd{V}~.
\end{equation}
现在只需证明
\begin{equation}
\int \grad (fg) \dd{V} = \oint fg \dd{\bvec s}~,
\end{equation}
这由\autoref{eq_NLext_3}~\upref{NLext} 可证。 证毕。

\subsection{矢量分部积分}
\begin{equation}\label{eq_IntBP2_1}
\int f (\div \bvec A)\dd{V} =  \oint f \bvec A \vdot \dd{\bvec s} - \int\bvec A \vdot (\grad f)\dd{V}~.
\end{equation}
参考\cite{GriffE}, 在电动力学中有应用(见“电场的能量\upref{EEng}”)。

\subsubsection{证明}
把\autoref{eq_VopEq_1}~\upref{VopEq} 两边做体积分
\begin{equation}
\int\div (f \bvec A) \dd{V} = \int f (\div \bvec A)\dd{V} + \int\bvec A \vdot (\grad f)\dd{V}~.
\end{equation}
由散度定理, 左边的体积分变为面积分 $\oint f \bvec A \vdot \dd{\bvec s}$, 移项得\autoref{eq_IntBP2_1} 。 证毕。
