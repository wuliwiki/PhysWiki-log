% 约翰·考克饶夫(综述)
% license CCBYSA3
% type Wiki

本文根据 CC-BY-SA 协议转载翻译自维基百科\href{https://en.wikipedia.org/wiki/John_Cockcroft}{相关文章}。

\begin{figure}[ht]
\centering
\includegraphics[width=6cm]{./figures/ed431b9628bc8c4d.png}
\caption{1961年的考克饶夫} \label{fig_YHkrf_1}
\end{figure}
约翰·道格拉斯·考克饶夫爵士(Sir John Douglas Cockcroft,1897年5月27日-1967年9月18日)是英国核物理学家,因与欧内斯特·沃尔顿共同实现原子核裂变而获1951年诺贝尔物理学奖,这一成就对核能的发展起到了关键作用。

在第一次世界大战期间,考克饶夫曾在西线担任皇家野战炮兵服役。战后,他在曼彻斯特市立工艺学院学习电气工程,同时在大都会维克斯特拉福德园区担任学徒,并成为该公司研究部门的一员。随后,他获得奖学金进入剑桥大学圣约翰学院,并于1924年6月参加三一试,成为Wrangler(剑桥数学优等生)。欧内斯特·卢瑟福接纳考克饶夫在卡文迪许实验室攻读研究生,考克饶夫于1928年在卢瑟福的指导下完成博士学位。在沃尔顿和马克·奥利芬特的合作下,他建造了后来被称为考克饶夫–沃尔顿发生器的装置。考克饶夫和沃尔顿利用这一装置首次实现了对原子核的人造裂变,这一壮举被大众称为“劈开原子”。

在第二次世界大战期间,考克饶夫担任英国供应部科研助理主任,负责雷达相关工作。他还是处理弗里施–佩尔斯备忘录(该备忘录计算出原子弹在技术上可行)相关问题的委员会成员,并参与了随后成立的MAUD委员会。1940年,作为提泽德代表团的一员,他将英国技术与美国同行共享。战争后期,提泽德代表团成果以SCR-584雷达和近炸引信的形式返回英国,并被用于协助击落V-1飞弹。

1944年5月,他出任蒙特利尔实验室主任,负责监督ZEEP和NRX反应堆的开发,以及乔克河实验室的创建。

战后,考克饶夫出任哈韦尔原子能研究机构(AERE)主任,1947年8月15日,低功率、石墨慢化的GLEEP反应堆在哈韦尔启动,成为西欧首座投入运行的核反应堆。随后在1948年又建成了英国实验堆0号(BEPO)。哈韦尔参与了温斯凯尔反应堆和化学分离工厂的设计。在他的领导下,哈韦尔还参与了前沿聚变研究,包括ZETA计划。他坚持要求在温斯凯尔反应堆的排气烟囱上安装过滤器,这一做法曾被讥讽为“考克饶夫的愚行”,但在1957年温斯凯尔火灾导致其中一座反应堆堆芯燃烧并释放放射性物质后,这一措施证明了其重要性。

1959年至1967年,他出任剑桥大学丘吉尔学院首任院长。1961年至1965年,他还担任堪培拉澳大利亚国立大学校监。
\subsection{早年经历}
约翰·道格拉斯·考克饶夫,也被称作“Johnny W.”,于1897年5月27日出生在英格兰约克郡西区托德莫登,是纺织厂主约翰·阿瑟·考克饶夫和妻子安妮·莫德(娘家姓菲尔登,Annie Maude née Fielden)的长子。他有四个弟弟:埃里克、菲利普、基思和莱昂内尔。1901年至1908年,他在沃尔斯登的英格兰教会学校接受早期教育,1908年至1909年就读于托德莫登小学,1909年至1914年就读于托德莫登中学,在校期间,他参加了足球和板球运动。在这所学校就读的女生中,有他未来的妻子尤尼斯·伊丽莎白·克拉布特里。1914年,他获得了约克郡西区的郡优秀奖学金,进入曼彻斯特维多利亚大学学习数学。

1914年8月,第一次世界大战爆发。考克饶夫于1915年6月完成在曼彻斯特的第一学年。他加入了校内的军官训练团,但并不希望成为军官。在暑假期间,他在威尔士金梅尔军营的基督教青年会食堂工作。1915年11月24日,他参军入伍。1916年3月29日,他加入了皇家野战炮兵第59训练旅,在此接受通信兵训练。随后,他被分配到西线战场第20(轻型)师所属的第92野战炮兵旅B炮兵连服役。

考克饶夫曾参与了向兴登堡防线的推进战役和第三次伊普尔战役。他申请转任军官并获批准。1918年2月,他被派往布莱顿学习炮兵知识,1918年4月前往北安普敦郡威登贝克的候补军官学校,接受野战炮兵军官培训。1918年10月17日,他被任命为皇家野战炮兵中尉。

战争结束后,考克饶夫于1919年1月从军队退役。他选择不返回曼彻斯特维多利亚大学,而是在曼彻斯特市立工艺学院学习电气工程。由于他已在曼彻斯特维多利亚大学完成一年学业,因此获准跳过课程的第一年。他于1920年6月获得理学士学位。该校电气工程教授迈尔斯·沃克(Miles Walker)说服他在大都会维克斯公司进行学徒训练。他获得了英国1851年博览会皇家委员会颁发的“1851年博览会奖学金”,并于1922年6月提交了硕士论文《交流电的谐波分析》。

随后,沃克建议考克饶夫申请剑桥大学圣约翰学院(沃克的母校)的奖学金。考克饶夫申请成功,获得了30英镑的奖学金和20英镑的助学金(发放给经济条件有限的本科生)。大都会维克斯同意向他提供50英镑,条件是他完成学业后返回公司任职。沃克和考克饶夫的一位姑妈帮助他凑齐了总计316英镑的学费。作为其他大学的毕业生,他获准跳过三一试第一年的课程。他于1924年6月参加了三一试考试,取得了B*等级并成为Wrangler(剑桥数学优等生),并获得了学士学位。

1925年8月26日,考克饶夫在托德莫登桥街联合卫理公会教堂与伊丽莎白·克拉布特里结婚。他们共育有六个孩子,第一个孩子是男孩,名为蒂莫西,不幸在婴儿期夭折。此后,他们育有四个女儿:琼·多萝西娅(Joan Dorothea,昵称Thea)、乔斯林、伊丽莎白·菲尔登、凯瑟琳·海伦娜,以及另一位儿子克里斯托弗·休·约翰。
\subsection{核研究}
\begin{figure}[ht]
\centering
\includegraphics[width=8cm]{./figures/966e76c528980952.png}
\caption{约翰·考克饶夫从两岁起直到28岁所居住的西约克郡沃尔斯登的住宅} \label{fig_YHkrf_2}
\end{figure}
在大都会维克斯公司研究主管和迈尔斯·沃克的推荐下,欧内斯特·卢瑟福同意接收考克饶夫到卡文迪许实验室担任研究生。1924年,考克饶夫以剑桥圣约翰学院基金奖学金和国家奖学金的资助身份,正式注册为博士研究生。在卢瑟福的指导下,他撰写了博士论文《分子流在表面凝结时出现的现象》,并发表在《皇家学会会刊》上。1928年9月6日,他获得了博士学位。在此期间,他曾担任俄罗斯物理学家彼得·卡皮察的助手,协助其进行极低温下磁场物理的研究工作,并帮助设计和建造了液化氦设备。

1919年,卢瑟福利用衰变镭原子释放的α粒子成功实现了氮原子的裂变。这项实验及后续实验为探索原子核结构提供了线索。为了进一步研究这一领域,卢瑟福需要一种能够以足够高的速度克服原子核电荷排斥力的人造粒子加速手段,这为卡文迪许实验室开辟了一条新的研究方向。他将这一课题分配给考克饶夫、托马斯·阿利博恩和欧内斯特·沃尔顿研究。他们随后建造了后来被称为“考克饶夫–沃尔顿加速器”的装置。马克·奥利芬特为他们设计了质子源。

一个关键时刻是考克饶夫阅读了乔治·伽莫夫关于量子隧穿效应的论文后,意识到由于这一现象,所需的加速电压比最初设想的要低得多。实际上,他计算出只需能量为30万电子伏特的质子即可穿透硼原子核。随后,考克饶夫和沃尔顿花了两年时间继续改进他们的加速器。卢瑟福从剑桥大学为他们申请到一笔1000英镑的经费,用于购买变压器和其他所需设备。

1928年11月5日,考克饶夫当选为剑桥大学圣约翰学院院士。1932年3月,他和沃尔顿开始操作他们的加速器,用高能质子轰击锂和铍。他们原本预期会观测到此前法国科学家报道过的伽马射线,但并未发现。1932年2月,詹姆斯·查德威克证明此前观测到的实际上是中子。随后,考克饶夫和沃尔顿转而寻找α粒子。1932年4月14日,沃尔顿用质子轰击锂靶时,注意到可能出现了α粒子。考克饶夫和随后赶来的卢瑟福确认了这一发现。当晚,考克饶夫和沃尔顿在卢瑟福家中撰写了致《自然》杂志的简报,宣布了他们的实验结果,即首次实现了对原子核的人造裂变,其反应式可描述如下:
$$
_3^7\text{Li} + p \rightarrow 2\,_2^4\text{He} + 17.2\,\text{MeV}~
$$
这一壮举被大众称作“劈开原子”。凭借这一成就,考克饶夫和沃尔顿于1938年获得休斯奖章,1951年获得诺贝尔物理学奖。他们随后继续利用质子、氘核和α粒子实现对碳、氮和氧的裂变,并证明他们已成功制备放射性同位素,包括碳-11和氮-13。
\begin{figure}[ht]
\centering
\includegraphics[width=8cm]{./figures/b8bbad429132038f.png}
\caption{考克饶夫–沃尔顿倍压电路} \label{fig_YHkrf_3}
\end{figure}
1929年,考克饶夫被任命为剑桥大学圣约翰学院机械科学导师。1931年被任命为物理导师,1933年成为初级财务主管,负责学院建筑物的维护,当时许多建筑因年久失修而状况不佳。学院的大门楼因死亡钟甲虫造成的破坏需要部分拆除修复,考克饶夫还监督了电力线路的重新布线工作。1935年,卢瑟福在卡皮察返回苏联后,任命他为蒙德实验室研究主任。他负责监督安装新的低温设备,并指导低温领域的研究工作。1936年,他当选为皇家学会院士;1939年当选为剑桥大学自然哲学杰克逊讲座教授,自1939年10月1日起生效。

考克饶夫和沃尔顿清楚他们的加速器存在局限性。在美国,欧内斯特·劳伦斯研发出一种更先进的设计,即回旋加速器。尽管卡文迪许实验室使用的加速器性能不如美国的先进设备,但凭借巧妙的实验物理,依然保持了领先。然而,考克饶夫敦促卢瑟福为卡文迪许实验室购置回旋加速器。卢瑟福因价格昂贵而犹豫不决,但奥斯汀勋爵捐赠的25万英镑,使得基于劳伦斯设计的36英寸(910毫米)回旋加速器得以建造,并修建了用于安置该设备的新楼翼。考克饶夫监督了这项工作。回旋加速器于1938年10月投入运行,新楼翼于1940年竣工。奥利芬特认为36英寸的回旋加速器规模不够大,于是在伯明翰大学开始建造更大型的60英寸回旋加速器。然而,1939年欧洲爆发第二次世界大战导致建设延迟,并且该设备在战后建成时已接近过时。
\subsection{第二次世界大战}
\begin{figure}[ht]
\centering
\includegraphics[width=8cm]{./figures/5becb6c17b46182e.png}
\caption{GL Mk. III 型火控雷达} \label{fig_YHkrf_4}
\end{figure}
第二次世界大战爆发时,考克饶夫出任英国供应部科研助理主任,负责雷达相关工作。1938年,亨利·提泽德爵士(Sir Henry Tizard)向考克饶夫展示了“链家”(Chain Home)系统,即由英国皇家空军(RAF)建设的一系列沿海预警雷达站,用于探测和追踪飞机。此后,他帮助调配科学家力量,使该系统全面投入运行。

1940年,他成为科研与技术发展咨询委员会成员。同年4月,他成为空战科学研究委员会成员,该委员会成立旨在处理“弗里施–佩尔斯备忘录”提出的相关问题(该备忘录计算出制造原子弹在技术上可行)。该委员会于1940年6月被“MAUD委员会”取代,考克饶夫也成为其成员之一。该委员会领导了英国早期开创性的原子能研究工作。1940年8月,考克饶夫作为“提泽德代表团”成员前往美国。由于英国虽然研发出许多新技术,但缺乏充足的工业能力加以充分利用,因此决定将这些技术与尚未参战的美国分享。提泽德代表团提供的信息包含了战时最重大的科学进展。共享的技术包括雷达技术,尤其是伯明翰大学奥利芬特团队设计的大幅改进的空腔磁控管(美国历史学家詹姆斯·巴克斯特三世称其为“有史以来运往美国最有价值的货物”),近炸引信设计、弗兰克·惠特尔喷气发动机的细节,以及描述原子弹可行性的“弗里施–佩尔斯备忘录”。除这些最重要的技术外,还包括火箭、增压器、瞄准具和潜艇探测设备等设计图纸。1940年12月,他返回英国。

回国后不久,考克饶夫被任命为汉普郡克赖斯特彻奇空防研究发展机构(ADRDE)主任。他的重点工作是利用雷达击落敌机。当时,GL Mk. III雷达被开发用于目标追踪和预测射击,但到1942年,美国开发用于同一目的的SCR-584雷达已可供使用,考克饶夫建议通过租借法案引进该设备。他主动购入SCR-584进行测试,并于1943年10月在谢佩岛进行的试验中,明确证明SCR-584性能更优。这使他在供应部内非常不得人心,但他掌握的情报显示,德国正计划部署V-1飞弹。1944年1月1日,罗纳德·维克斯中将紧急致电华盛顿,请求提供134套SCR-584雷达设备。
\begin{figure}[ht]
\centering
\includegraphics[width=8cm]{./figures/dc0d38e6ebff2f36.png}
\caption{近炸引信} \label{fig_YHkrf_5}
\end{figure}
近炸引信最早由艾伦·比尤特曼(Alan Butement)提出。其原理是,如果炮弹能在接近敌机时引爆,那么近距离爆炸几乎可以与直接命中产生同等效果。技术难点在于如何将雷达装置小型化,并且坚固到足以承受从炮管中发射时的加速度冲击。第二个问题已被德国人解决:英军回收了一枚哑弹,发现其中的电子管可以承受发射时的加速度。提泽德代表团将近炸引信的设计方案交给了美国人,但英国方面仍在继续研发。1942年2月,查尔斯·德拉蒙德·埃利斯在克赖斯特彻奇组建了一个团队推进该项目。然而研发进展缓慢,到1943年时预计量产仍需两年时间。

1943年11月,考克饶夫访问美国,与默尔·图夫讨论了将美制近炸引信改装为适用于英军使用的方案。结果,1944年1月16日,英方订购了15万个用于QF 3.7英寸高射炮的引信。这些引信及时到达,并在1944年8月用于拦截V-1飞弹,击落了其中97\%的飞弹。因其贡献,考克饶夫于1944年6月被授予大英帝国司令勋章(CBE)。1943年8月,《魁北克协定》将英国“管合金”\核计划并入美国“曼哈顿计划”,并成立了“联合政策委员会”以管理曼哈顿计划。1944年5月20日,最终协议明确:美方将协助在加拿大建造一座重水慢化核反应堆,并在腐蚀和辐射对材料影响等技术问题上提供支持,但不会提供钚的化学与冶金细节,尽管美方提供了辐照过的铀块供英方自行研究。

一个关键障碍是蒙特利尔实验室主任汉斯·冯·哈尔班。他管理能力不足,与加方合作不顺畅,并且被美方视为安全隐患。1944年4月,“联合政策委员会”在华盛顿会议上同意,让非英籍的蒙特利尔实验室科学家离开,并决定由考克饶夫接任蒙特利尔实验室主任,时间是1944年5月。
\begin{figure}[ht]
\centering
\includegraphics[width=8cm]{./figures/36be1c0be50152f8.png}
\caption{1954年2月的ZEEP反应堆,背景中可见NRX反应堆和在建的NRU反应堆} \label{fig_YHkrf_6}
\end{figure}
1944年8月24日,决定建造一座小型反应堆,用于验证蒙特利尔实验室在晶格尺寸、包壳材料和控制棒等方面的计算结果,以便在此基础上继续建造全规模的NRX反应堆。这座小型反应堆被命名为ZEEP,即“零能量实验堆”。在蒙特利尔市中心建造反应堆显然是不可能的;加拿大方面选定了位于安大略省乔克河、渥太华河北岸约110英里(180公里)西北处的选址,并获得格罗夫斯批准。乔克河实验室于1944年启用,蒙特利尔实验室于1946年7月关闭。1945年9月5日,ZEEP实现临界,成为美国以外首座投入运行的核反应堆。更大型的NRX反应堆于1947年7月21日实现临界。其中子通量是当时其他任何反应堆的五倍,使其成为当时世界上最强大的研究反应堆。

NRX反应堆最初在1944年7月设计时计划输出功率为8兆瓦,但通过将不锈钢包壳、重水冷却的铀棒更换为铝包壳、轻水冷却的铀棒等设计改进,功率提升到了10兆瓦。1945年9月10日,当考克饶夫得知在乔克河实验室工作的英国物理学家艾伦·纳恩·梅是苏联间谍时感到震惊。1947年8月,考克饶夫曾与其他科学家一起签署请愿书,呼吁缩减纳恩·梅的十年监禁刑期,但他后来对这一行为表示后悔。
\subsection{战后}
1945年4月,考克饶夫与奥利芬特在英国勘察选址,用于建立类似的核研究机构,最终选定了哈韦尔皇家空军基地。1945年11月9日,考克饶夫被正式邀请出任哈韦尔原子能研究机构(AERE)主任。官方公告于1946年1月29日发布,但消息在发布前两个月就已泄露,加拿大政府也未提前获知,导致了一场外交风波。最终达成一致,在找到继任者前,考克饶夫不会离职,他于1946年9月30日才从乔克河前往哈韦尔。在此期间,他为新实验室招募了科研人员。来自“曼哈顿计划”洛斯阿拉莫斯实验室的克劳斯·福克斯出任理论物理部门主管;蒙特利尔实验室副主任罗伯特·斯彭斯出任化学部门主管;H.W.B. 斯金纳主管普通物理;奥托·弗里施主管核物理;约翰·邓沃思主管反应堆物理。福克斯后来于1950年2月3日因苏联间谍案被捕。

由蒙特利尔实验室设计、低功率石墨慢化的GLEEP于1947年8月15日启动,成为西欧首座投入运行的核反应堆。随后,1948年7月3日,由AERE设计的6兆瓦研究反应堆BEPO投入运行。由于英国当时无法获得重水,BEPO被设计和建造为石墨慢化反应堆。哈韦尔还参与了温斯凯尔反应堆和化学分离厂的设计工作。

1946年8月,美国通过《1946年原子能法》(麦克马洪法案),明确禁止英国继续接触美国的核研究成果,这在一定程度上与1946年2月艾伦·纳恩·梅因间谍案被捕有关。考克饶夫帮助谈判达成了一项新的、较为非正式且未签署的协议,该协议于1948年1月7日公布,被称为“共处模式”,但他希望在此框架下恢复的合作最终未能实现。英国独立核威慑力量的发展促使《原子能法》于1958年修订,并根据《1958年美英相互防务协定》,美英之间恢复了核领域的特殊关系。

在考克饶夫的领导下,哈韦尔在战后积极参与前沿聚变研究,包括ZETA(零能量热核装置)计划。1946年,乔治·佩吉特·汤姆森爵士在伦敦帝国理工学院开始核聚变研究,随后转移至奥利博恩领导的奥尔德马斯顿AEI实验室。牛津大学的彼得·托内曼团队也独立开展了相关研究。1951年,考克饶夫安排将牛津团队转至哈韦尔。考克饶夫批准了托内曼团队在哈韦尔建设ZETA装置,同时批准奥利博恩团队建设较小的Sceptre装置。
与此同时,洛斯阿拉莫斯实验室詹姆斯·塔克领导的团队也在研究聚变,考克饶夫与美方达成协议,双方将同步公开研究成果,最终于1958年共同发布。尽管考克饶夫始终乐观地认为突破即将到来,但聚变能始终未能如愿实现,仍是难以企及的目标。
\subsubsection{考克饶夫的“愚行”}
\begin{figure}[ht]
\centering
\includegraphics[width=8cm]{./figures/7be1168135c0f4c4.png}
\caption{温斯凯尔反应堆的两座烟囱,可见用于容纳考克饶夫过滤器的明显凸起} \label{fig_YHkrf_7}
\end{figure}
作为哈韦尔原子能研究机构(AERE)主任,考克饶夫曾坚持要求在温斯凯尔钚生产反应堆的烟囱上加装高性能过滤器,尽管这需要花费大量资金。这是因为一份报告称,在田纳西州橡树岭X-10石墨反应堆附近发现了氧化铀。由于在烟囱设计完成后才决定加装过滤器,因此这些过滤器在烟囱顶部形成了非常显眼的凸起。

反应堆原本设计为在使用过程中保持清洁且不受腐蚀,因此当时认为不会产生任何需要过滤器拦截的颗粒物。此外,橡树岭发现的氧化铀后来被证明来自化工厂,而非反应堆本身。因此,这些过滤器被戏称为“考克饶夫的愚行”。然而,1957年温斯凯尔两座反应堆之一发生堆芯起火事故时,这些过滤器防止了更严重的放射性物质释放。后来的英国国防部科学顾问特伦斯·普赖斯指出:“事故发生后,‘愚行’这个词似乎已经不再合适。”
\subsection{晚年}
1959年1月24日,剑桥大学正式承认丘吉尔学院(Churchill College)为其学院之一。两天后,学院董事会宣布考克饶夫将出任首任院长。虽然该学院也会教授人文学科和社会科学,但70\%的学生将学习与科学和技术相关的课程。他提名了首批院士,并监督了学院的初期建设。学院建设过程中,关于礼拜堂的位置引发了争议。1961年,学院计划将礼拜堂建在学院入口处,这是剑桥传统做法,但这一计划立即引发坚定无神论者弗朗西斯·克里克(Francis Crick)辞去院士职务以示抗议。1961年,首批本科生入学;1964年6月5日,学院(当时尚未完全建成)由爱丁堡公爵菲利普亲王正式揭幕。
\begin{figure}[ht]
\centering
\includegraphics[width=8cm]{./figures/bdf0e57c31f57f59.png}
\caption{2005年的剑桥丘吉尔学院} \label{fig_YHkrf_8}
\end{figure}
考克饶夫曾于1954年至1956年担任英国物理学会会长,以及英国科学促进会会长。他还曾于1961年至1965年担任堪培拉澳大利亚国立大学(ANU)校监,这一职务主要具有象征性,每年需前往澳大利亚主持一次学位授予仪式。他于1944年发表了卢瑟福纪念讲座。此外,他还是欧洲核子研究中心理事会的英国代表,并担任英国科学与工业研究部核物理分委员会主席。

除了与沃尔顿共同获得休斯奖章和1951年诺贝尔物理学奖外,考克饶夫在其职业生涯中还获得了诸多奖项和荣誉。他于1948年1月被授予下级爵士称号。这是当时的常规做法:科学家很少被授予骑士团勋章,但他于1953年5月被授予巴斯爵级司令勋章(KCB)。或许因为这种情况较为罕见,科学家们通常认为获得功绩勋章是更高的荣誉;考克饶夫于1956年12月获授功绩勋章成员。他还于1954年获得皇家奖章,1955年获得法拉第奖章,1947年获得美国自由勋章,1961年获得和平利用原子能奖。1952年,他被法国授予法国荣誉军团骑士勋章;1955年,被葡萄牙授予基督勋章爵级司令;1958年,被西班牙授予阿方索十世智者大十字勋章。
\begin{figure}[ht]
\centering
\includegraphics[width=6cm]{./figures/446dc5944a42434f.png}
\caption{1962年5月,考克饶夫(左)在加拿大萨斯卡通为萨斯喀彻温加速器实验室举行奠基仪式} \label{fig_YHkrf_9}
\end{figure}
1967年9月18日,考克饶夫在剑桥丘吉尔学院的住所因心脏病发作去世。他被安葬在剑桥升天堂教区公墓,与他的儿子蒂莫西同葬一穴。1967年10月17日,在威斯敏斯特大教堂为他举行了追思纪念仪式。

英国有多栋建筑以他的名字命名:剑桥大学新博物馆园区的考克饶夫楼,其中包括一个讲堂和多个硬件实验室;柴郡戴尔斯伯里实验室的考克饶夫研究所;布莱顿大学的考克饶夫楼;索尔福德大学的考克饶夫楼。澳大利亚国立大学物理科学与工程研究学院最古老的建筑也以他命名为考克饶夫楼。

考克饶夫的档案保存在剑桥丘吉尔档案中心,并向公众开放。这些档案包括他的实验记录本、往来信件、照片(其中有数十张记录乔克河建设过程的照片,编号CKFT 26/4)、论文及政治文件。
\subsection{注释}
\begin{enumerate}
\item “约翰·考克饶夫”。数学谱系项目。
\item Allibone, T. E. (1967)。“约翰·考克饶夫爵士,O.M., F.R.S.”,《英国放射学杂志》,40 (479): 872–873。doi:10.1259/0007-1285-40-479-872。PMID 4862179。
\item Allibone, T. E. “考克饶夫,约翰·道格拉斯爵士(1897–1967),物理学家与工程师”。《牛津国家人物传记辞典》(在线版),牛津大学出版社。doi:10.1093/ref:odnb/2473。(需订阅或英国公共图书馆账户访问)
\item Hartcup & Allibone 1984,第2页。
\item Oliphant, M. L. E.; Penney, L. (1968)。“约翰·道格拉斯·考克饶夫(1897–1967)”,《皇家学会院士传记回忆录》,14: 139–188。doi:10.1098/rsbm.1968.0007。S2CID 57116624。
\item Hartcup & Allibone 1984,第4–5页。
\item Hartcup & Allibone 1984,第5–7页。
\item Hartcup & Allibone 1984,第10–15页。
\item “No. 30993”。《伦敦公报》(增刊),1918年11月5日,第13089页。
\item Hartcup & Allibone 1984,第15–19页。
\item Hartcup & Allibone 1984,第20–25页。
\item Hartcup & Allibone 1984,第34页。
\item Cockcroft, John Douglas。“分子流在表面凝结时出现的现象”,剑桥大学。原文存档于2021年2月25日,检索于2016年9月4日。
\item Cockcroft, J. D. (1928年6月1日)。“分子流在表面凝结时出现的现象”,《皇家学会会刊A》,119 (782): 293–312。Bibcode:1928RSPSA.119..293C。doi:10.1098/rspa.1928.0099。ISSN 1364-5021。
\item Hartcup & Allibone 1984,第33页。
\item Hartcup & Allibone 1984,第37–42页。
\item 伽莫夫,乔治(1928年3月)。“关于原子核的量子理论”,《物理杂志》,51(3):204–212。Bibcode:1928ZPhy...51..204G。doi:10.1007/BF01343196。ISSN 0044-3328。S2CID 120684789。
\item Hartcup & Allibone 1984,第43页。
\item 考克饶夫,约翰;沃尔顿,欧内斯特(1932年4月)。“高速质子对锂的裂变”,《自然》,129(649):649。Bibcode:1932Natur.129..649C。doi:10.1038/129649a0。
\item Hartcup & Allibone 1984,第50–53页。
\item “获奖者:休斯奖章”,英国皇家学会。检索日期:2016年9月4日。
\item “1951年诺贝尔物理学奖”,诺贝尔基金会。检索日期:2016年9月4日。
\item 考克饶夫,J.D.; 沃尔顿,E.T.S.(1932年6月1日)。“高速正离子实验(I):获得高速正离子方法的进一步发展”,《皇家学会会刊A》,-136(830):619–630。Bibcode:1932RSPSA.136..619C。doi:10.1098/rspa.1932.0107。ISSN 1364-5021。
\item 考克饶夫,J.D.; 沃尔顿,E.T.S.(1932年7月1日)。“高速正离子实验(II):高速质子对元素的裂变”,《皇家学会会刊A》,137(831):229–242。Bibcode:1932RSPSA.137..229C。doi:10.1098/rspa.1932.0133。ISSN 1364-5021。
\item 考克饶夫,J.D.; 沃尔顿,E.T.S.(1934年5月1日)。“高速正离子实验(III):重氢离子对锂、硼和碳的裂变”,《皇家学会会刊A》,144(853):704–720。Bibcode:1934RSPSA.144..704C。doi:10.1098/rspa.1934.0078。ISSN 1364-5021。
\item 考克饶夫,J.D.; 沃尔顿,E.T.S.(1935年1月1日)。“高速正离子实验(IV):利用高速质子和二重子产生感生放射性”,《皇家学会会刊A》,148(863):225–240。Bibcode:1935RSPSA.148..225C。doi:10.1098/rspa.1935.0015。ISSN 1364-5021。
\item 考克饶夫,J.D.; 刘易斯,W.B.(1936年3月2日)。“高速正离子实验(V):关于硼裂变的进一步实验”,《皇家学会会刊A》,154(881):246–261。Bibcode:1936RSPSA.154..246C。doi:10.1098/rspa.1936.0049。ISSN 1364-5021。
\item 考克饶夫,J.D.; 刘易斯,W.B.(1936年3月2日)。“高速正离子实验(VI):利用氘核裂变碳、氮和氧”,《皇家学会会刊A》,154(881):261–279。Bibcode:1936RSPSA.154..261C。doi:10.1098/rspa.1936.0050。ISSN 1364-5021。
\item “剑桥大学”。剑桥大学数字图书馆。检索日期:2022年10月12日。
\item “诺贝尔奖”。nobelprize.org。检索日期:2022年10月12日。
\item Hartcup & Allibone 1984,第368–371页。
\item Clarke,N.M.博士。“伯明翰的纳菲尔德回旋加速器”,伯明翰大学。原文存档于2014年4月8日,检索日期:2013年5月2日。
\item Hartcup & Allibone 1984,第89–90页。
\item Hartcup & Allibone 1984,第94页。
\item Hartcup & Allibone 1984,第120–124页。
\item Hartcup & Allibone 1984,第96–103页。
\item “雷达”。《新闻周刊》,1997年1月12日。检索日期:2016年9月4日。
\item Hartcup & Allibone 1984,第108–111页。
\item Hartcup & Allibone 1984,第96页。
\item “No. 36544”。《伦敦公报》(增刊),1944年6月2日,第2586页。
\item Hewlett & Anderson 1962,第277–280页。
\item Hewlett & Anderson 1962,第281–284页。
\item Avery 1998,第184–185页。
\item Gowing 1964,第206–207页、209–214页。
\item Goldschmidt,伯特兰。“加拿大的起点——法国科学家的角色”,加拿大核能学会。原文存档于2016年3月11日,检索日期:2016年5月6日。
\item Laurence,George C. “加拿大核能研究早年历史”(PDF)。检索日期:2016年5月19日。
\item Jones 1985,第246–247页。
\item Close 2015,第102–104页。
\item 曼哈顿区1947,第9.23页。
\item Fidecaro,朱塞佩(1996年12月4日)。“布鲁诺·庞特科沃:从罗马到杜布纳(个人回忆)”,比萨大学,检索日期:2016年4月15日。
\item 曼哈顿区1947,第9.9–9.10页。
\item Hartcup & Allibone 1984,第133–134页。
\item Hartcup & Allibone 1984,第136–137页。
\item Hartcup & Allibone 1984,第139–146页。
\item Hartcup & Allibone 1984,第158页。
\item Hartcup & Allibone 1984,第146–147页。
\item Gowing & Arnold 1974a,第379–382页。
\item Gowing & Arnold 1974,第105–108页。
\item Hartcup & Allibone 1984,第222–223页。
\item Gott,理查德(1963年4月)。“英国独立核威慑力量的发展”,《国际事务》,39(2):238–252。doi:10.2307/2611300。ISSN 1468-2346。JSTOR 2611300。
\item Hartcup & Allibone 1984,第201–204页。
\item Leatherdale,邓肯(2014年11月14日)。“温斯凯尔堆:考克饶夫的‘愚行’避免了核灾难”,BBC,检索日期:2016年9月3日。
\item Hartcup & Allibone 1984,第211页。
\item Hartcup & Allibone 1984,第250–258页。
\item “No. 38161”。《伦敦公报》(增刊),1947年12月30日,第2页。
\item Hartcup & Allibone 1984,第193页。
\item “No. 39863”。《伦敦公报》(增刊),1953年5月26日,第2943页。
\item “No. 40960”。《伦敦公报》(增刊),1956年12月28日,第4页。
\item “No. 39462”。《伦敦公报》,1952年2月8日,第789页。
\item Hartcup & Allibone 1984,第284页。
\item “考克饶夫楼”,剑桥大学,检索日期:2016年9月5日。
\item “考克饶夫研究所”,考克饶夫研究所官网,检索日期:2016年9月5日。
\item “展示资料:布莱顿大学考克饶夫楼”,布莱顿大学。原文存档于2016年9月14日,检索日期:2016年9月5日。
\item “新建300万英镑超级实验室揭幕”,索尔福德大学,检索日期:2016年9月5日。
\item “考克饶夫楼”,原文存档于2016年9月15日,检索日期:2016年9月5日。
\item “约翰·考克饶夫爵士文献档案”,丘吉尔档案中心,ArchiveSearch,检索日期:2021年9月30日。
\end{enumerate}
\subsection{参考文献}
\begin{itemize}
\item Avery, Donald(1998)《战争的科学:加拿大科学家与盟军军事技术》,多伦多:多伦多大学出版社。ISBN 978-0-8020-5996-3。OCLC 38885226。
\item Close, Frank(2015)《半衰期:布鲁诺·庞特科沃分裂的生活,物理学家还是间谍》,纽约:Basic Books。ISBN 978-0-465-06998-9。OCLC 897001600。
\item Gowing, Margaret(1964)《英国与原子能 1939–1945》,伦敦:麦克米兰出版社。OCLC 3195209。
\item Gowing, Margaret;Arnold, Lorna(1974)《独立与威慑:英国与原子能,1945–1952,第一卷,政策制定》,伦敦:麦克米兰出版社。ISBN 978-0-333-15781-7。OCLC 611555258。
\item Gowing, Margaret;Arnold, Lorna(1974a)《独立与威慑:英国与原子能,1945–1952,第二卷,政策执行》,伦敦:麦克米兰出版社。ISBN 978-0-333-16695-6。OCLC 59047125。
\item Hartcup, Guy;Allibone, T. E.(1984)《考克饶夫与原子》,布里斯托尔:阿冯:A. Hilger。ISBN 978-0-85274-759-9。OCLC 12666364。
\item Hewlett, Richard G.;Anderson, Oscar E.(1962)《新世界,1939–1946》(PDF),大学公园:宾夕法尼亚州立大学出版社。ISBN 978-0-520-07186-5。OCLC 637004643。检索日期:2013年3月26日。
\item Jones, Vincent(1985)《曼哈顿:陆军与原子弹》,华盛顿特区:美国陆军军事历史中心。OCLC 10913875。原文(PDF)存档于2014年10月7日,检索日期:2013年8月25日。
\item 曼哈顿区(1947)《曼哈顿区历史,第一册,第四卷,第九章——对加拿大堆项目的支持》,华盛顿特区:美国能源部。OCLC 889323140。
\end{itemize}
\subsection{延伸阅读}
\begin{itemize}
\item Cathcart, Brian(2005)《大教堂里的苍蝇:剑桥科学家小组如何赢得劈裂原子竞赛》,伦敦:企鹅出版社。ISBN 978-0-14-027906-1。OCLC 937140229。
\end{itemize}
\subsection{外部链接}
\begin{itemize}
\item 约翰·考克饶夫1963年5月2日口述历史访谈记录,美国物理学会尼尔斯·玻尔图书馆与档案馆
\item 约翰·考克饶夫1967年3月28日口述历史访谈记录,美国物理学会尼尔斯·玻尔图书馆与档案馆
\item Find a Grave 上的约翰·考克饶夫页面
\item 丘吉尔档案中心提供的传记
\item 诺贝尔奖官网上的约翰·考克饶夫页面(可在维基数据上编辑)
\item 另一篇诺贝尔奖传记
\item 1958年访问杜布纳联合核研究所(苏联)(已于2021年5月18日存档于Wayback Machine)
\item “与约翰·考克饶夫相关的档案资料”,英国国家档案馆(可在维基数据上编辑)
\end{itemize}