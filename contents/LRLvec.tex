% 拉普拉斯—龙格—楞次矢量
% keys LRL 矢量|守恒证明
% license Xiao
% type Tutor

\pentry{开普勒问题\nref{nod_CelBd}, 柱坐标系\nref{nod_Cylin}}{nod_dd98}

在开普勒问题% \addTODO{什么是开普勒问题?}
中,我们定义\textbf{拉普拉斯—龙格—楞次矢量(Laplace-Runge-Lenz Vector)} (缩写为 \textbf{LRL 矢量})为
\begin{equation}\label{eq_LRLvec_1}
\bvec A = \bvec p \cross \bvec L - mk \uvec r~.
\end{equation}
其中 $\bvec p$ 为质点动量, $\bvec L$ 为轨道角动量, $\cross$ 表示\enref{矢量叉乘}{Cross}, $k$ 是一个常数(对万有引力\upref{Gravty} $k = GMm$, 对库仑力\upref{ClbFrc} $k = -Qq/(4\pi\epsilon_0)$。 $\uvec r$ 为质点位矢 $\bvec r$ 的单位矢量。 在开普勒问题中, 可以证明 $\bvec A$ 是一个守恒量。

\subsection{守恒证明}
我们下面证明 $\dot{\bvec A} = \bvec 0$。 对\autoref{eq_LRLvec_1} 求时间导数, 考虑到中心力场中质点角动量 $\bvec L$ 守恒, 有
\begin{equation}\label{eq_LRLvec_2}
\dot{\bvec A} = \dot{\bvec p}\cross \bvec L  - mk\dot{\uvec r}~.
\end{equation}
其中由牛顿第二定律\upref{New3} 和万有引力定律/库仑力, 有
\begin{equation}\label{eq_LRLvec_3}
\dot{\bvec p} = \bvec F = - \frac{k}{r^2}\uvec r~,
\end{equation}
又由“极坐标中单位矢量的偏导\upref{DPol1}” 得
\begin{equation}\label{eq_LRLvec_4}
\dot{\uvec r} = \pdv{\uvec r}{\theta} \dv{\theta}{t} = \dot\theta\uvec \theta~.
\end{equation}
最后由\autoref{eq_CenFrc_4}~\upref{CenFrc}, 极坐标系中的角动量等于($\uvec z$ 是垂直于轨道平面的单位矢量, 由右手定则\upref{RHRul}决定, 参考柱坐标系\upref{Cylin})
\begin{equation}\label{eq_LRLvec_5}
\bvec L = mr^2\dot \theta \uvec z~.
\end{equation}
将\autoref{eq_LRLvec_3} 至\autoref{eq_LRLvec_5} 代入\autoref{eq_LRLvec_2} 得
\begin{equation}
\dot{\bvec A} = -\frac{k}{r^2}\uvec r \cross (mr^2\dot\theta\uvec z) - mk\dot\theta\uvec\theta
=-mk\dot\theta (\uvec r\cross \uvec z + \uvec \theta)
= \bvec 0~.
\end{equation}
最后一个等号成立是因为 $\uvec r\cross\uvec z = -\uvec\theta$, 可以类比直角坐标系中的 $\uvec x\cross\uvec z = -\uvec y$。 证毕。

% \addTODO{习题:证明二体问题中
% \bvec A_B = \frac{mu}{m_B} \bvec A
% \bvec A_A = -\frac{mu}{m_A} \bvec A
% \bvec A = \bvec A_B - \bvec A_A
% 其中 \bvec A 是等效天体(质量为 mu)的 LRL 矢量。}

\subsection{如何发现 LRL 矢量}
\pentry{运动积分\nref{nod_415a}}{nod_07b6}
根据分析力学的理论,一个有 $s$ 自由度的系统应当有 $2s-1$ 个运动积分,也就是对时间有 $2s-1$ 个守恒量。而开普勒问题本质是在空间内的三维问题(我们可以降维到二维是因为角动量守恒,使得运动轨迹在同一平面内),有 $s=3$ 的自由度,也就应当有 $2s-1=5$ 个运动积分。容易发现系统能量 $E$、角动量在各方向的分量 $L_x, L_y, L_z$ 共 $4$ 个守恒量,仍然差一个,所以需要考虑构造一个守恒量。

\subsubsection{猜测}
这个方法有一些很巧妙的成分在内。我们首先通过“猜测”考虑 LRL 矢量是由 $\bvec r \cross \bvec L$ 与 $\bvec p \cross \bvec L$ 构成。$\bvec L$ 守恒,考虑坐标 $\bvec r$ 与动量 $\bvec p=m\bvec v$ 与其的叉积
$\bvec r \cross \bvec L, \bvec p \cross \bvec L$,两者对于时间的全微分
$$\bvec P= \dv{\left(\bvec r \cross \bvec L\right)}{t}, \bvec Q =\dv{\left(\bvec p \cross \bvec L\right)}{t} ~,$$
可以计算得到
$$\begin{aligned}
\bvec P &= \bvec p \cross \bvec L/m, \\
\bvec Q &= - \frac{1}{r} \pdv{V(r)}{r} \bvec r \cross \bvec L ~.
\end{aligned}$$
而对于开普勒问题,$V(r) = -k/r$,这样就不难得到
\begin{equation}
\dv{\left(\bvec p \cross \bvec L - \frac{m k \bvec r}{r}\right)}{t}=0 ~.
\end{equation}

\subsection{LRL 矢量的模长与方向}
将 LRL 矢量除以 $mk$ 可以得到一个无量纲矢量 $\bvec e$ 与 LRL 矢量 $\bvec A$ 同向(下面会看到为什么这个矢量叫 $\bvec e$)
$$\bvec e = \frac{\bvec A}{mk} = \frac{m}{k} \left(\bvec v \cross \bvec L \right)- \hat{\bvec{r}} ~.$$
考虑到 $\bvec p = m \bvec v = m \dot{\bvec r}$,将其代入 $\bvec A$ 的表达式就会有
$$\bvec A = m^2 (\dot{\bvec r})^2 \bvec r - m^2(\dot{\bvec r} \cdot \bvec r)\dot{\bvec r} - km \bvec r/r ~.$$
所以 $\bvec A \cdot \bvec L = 0, \bvec A \cdot \bvec r = \bvec L^2 - kmr$。后面一个式子可以化为 $Ar\cos \phi = L^2-kmr$,$\phi$ 是轨道的夹角,于是
$$r(\phi) = L^2/(km + A \cos \phi) ~,$$
也就是 $r(\phi) = (L^2/km)/(1+ (A/km) \cos \phi)$,发现 $A/km$ 就是离心率。这就是 $\bvec e$ 得名的原因。

而另外的,$\bvec A$ 的方向与 $\bvec L$ 垂直。将 $\bvec A$ 置于椭圆的几何中心可以发现其与椭圆的长轴是同方向的。所以另外有一有趣推论如下:
\begin{corollary}{}
LRL 矢量的变化对应轨道方向的变化。对于进动问题来说进动角就是 LRL 矢量变化的角。
\end{corollary}

刚才另外提到了,运动积分应当有 $5$ 个,但是 LRL 矢量有三个分量 $A_x, A_y, A_z$,为什么只对应一个呢?我们发现,有两个约束条件:
\begin{equation}
\left\{
\begin{aligned}
\bvec A \cdot \bvec L &= 0 ~, \\
|\bvec A| &= kme = \sqrt{m^2k^2 + 2m E L^2} ~. 
\end{aligned}
\right.
\end{equation}
所以实际仅提供了一个运动积分。

\subsection{LRL 矢量对应的系统对称性}
\pentry{经典场论基础\nref{nod_classi}}{nod_8991}
\autoref{the_classi_1}~\upref{classi} 诺特定理指出,任意一个系统对称性对应一个守恒律,其逆定理一个守恒律也对应一个系统的对称性也总是成立。特别的,LRL 矢量的对称性来源是平方反比力,不同于其他的对称性来源于几何对称性。

实际上对应的对称性是:
\begin{equation}
\begin{pmatrix}
-y\dot y-z\dot z+2\dot x y -x \dot y +2\dot x z-x\dot z \\
-x \dot x - z \dot z + 2 \dot y z - y \dot z + 2 \dot y x - y \dot x \\
-x \dot x - y \dot y + 2 \dot z x - z \dot x + 2 \dot z y - z \dot y 
\end{pmatrix} ~.
\end{equation}

对于开普勒问题的总能量即哈密顿量 $H = p^2/(2m) + \alpha/r$,不难推导得到以下对称性:
\begin{enumerate}
\item $H < 0$,$\{$角动量 $,$ LRL 矢量 $\}$ 张成的李群 $\sim SO(4)$;
\item $H = 0 $,张成的李群 $\sim$ $\mathbb R^3$ 空间中运动群的李代数;
\item $H > 0$,张成的李群 $\sim SO(1, 3)$。
\end{enumerate}
下面提供一个思路:

角动量的运动积分是 $\bvec J_1 = [\bvec r, \bvec p]$,其中 $[\bvec a, \bvec b]$ 是相空间的交换子,这对应角动量守恒。
而 LRL 矢量的运动积分是 $\bvec{B_1} = [\bvec p/m, \bvec{J_1}] + \alpha \bvec r / r$。考虑物理量间的泊松括号:
\begin{equation}
\{p_i, {J_1}_{j}\} = \begin{pmatrix}
0 & p_3 & -p_2 \\
-p_3 & 0 & p_1 \\
p_2 & -p_1 & 0
\end{pmatrix}~.
\end{equation}
从而角动量的运动积分与 LRL 矢量的运动积分的泊松括号
\begin{equation}
\{{J_1}_i, {B_1}_j\} = \begin{pmatrix}
0 & {B_1}_3 & -{B_1}_2 \\
-{B_1}_3 & 0 & {B_1}_1 \\
{B_1}_2 & -{B_1}_1 & 0 
\end{pmatrix} ~.
\end{equation}
不妨再考虑 $\{{B_1}_i, {B_1}_j\}$,
\begin{equation}
\{{B_1}_i, {B_1}_j\} = \begin{pmatrix}
0 & \lambda {J_1}_3 & -\lambda {J_1}_2 \\
-\lambda {J_1}_3 & 0 & \lambda {J_1}_1 \\
\lambda {J_1}_2 & -\lambda {J_1}_1 & 0 
\end{pmatrix}~,\ \lambda = -2H/m ~.
\end{equation}
就不难得到结论。