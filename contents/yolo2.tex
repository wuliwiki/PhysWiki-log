% yolov5 算法代码解析二
% keys CNN
% license Usr
% type Wiki


\begin{enumerate}
\item detect代码详细分析
\end{enumerate}
首先导入python的包。
\begin{lstlisting}[language=python]
import argparse # 解析命令行参数的库
import os # 与操作系统进行交互的文件库 包含文件路径操作与解析
import sys # sys模块包含了与python解释器和它的环境有关的函数。
from pathlib import Path # Path能够更加方便得对字符串路径进行处理
 
import cv2 # sys模块包含了与python解释器和它的环境有关的函数。
import torch #pytorch 深度学习库
import torch.backends.cudnn as cudnn #让内置的cudnn的 auto-tuner 自动寻找最
#适合当前配置的高效算法,来达到优化运行效率的问题
\end{lstlisting}
因为有第一个库,我们可以直接在命令行里输入相关参数,而不用进py文件里一个个去修改,具体后面来介绍。接着获取当前文件的绝对路径。
\begin{lstlisting}[language=python]
FILE = Path(__file__).resolve() 
 # __file__指的是当前文件(即detect.py),FILE
#最终保存着当前文件的绝对路径,比如D://yolov5/detect.py
ROOT = FILE.parents[0] 
 # YOLOv5 root directory  ROOT保存着当前项目的父目录,#比如 D://yolov5
if str(ROOT) not in sys.path:  # sys.path即当前python环境可以运行的路径,
#假如当前项目不在该路径中,就无法运行其中的模块,所以就需要加载路径
    sys.path.append(str(ROOT)) 
     # add ROOT to PATH  把ROOT添加到运行路径上
ROOT = Path(os.path.relpath(ROOT, Path.cwd()))
# relative ROOT设置为相对路径
\end{lstlisting}
这个resolve()方法是Path类的一个方法,它返回当前路径的绝对路径。如果路径是相对的,resolve()会将其转换为绝对路径。此外,它还会解析路径中的任何符号链接(在支持符号链接的操作系统上。
对于ROOT = FILE.parents[0],这一行
FILE:这是一个变量,代表了一个文件的路径。这个路径可以是绝对路径(从根目录开始的完整路径),也可以是相对路径(相对于当前工作目录的路径)。
.parents:这是pathlib.Path对象的一个属性,它返回一个包含当前路径所有父目录的迭代器。例如,如果FILE是'/home/user/documents/file.txt',那么FILE
.parents将会是一个迭代器,包含'/home/user/documents'
、'/home/user'、'/home'
等目录路径。
[0]:通过对parents迭代器使用索引[0],我们获取了FILE路径的第一个父目录,也就是最直接的父目录。在上面的例子中,就是'/home/user/documents'。
ROOT =:这是将获取到的父目录路径赋值给变量ROOT的操作。之后,ROOT就可以用来引用这个父目录路径了。

\begin{lstlisting}[language=python]
from models.common import DetectMultiBackend
from utils.datasets import IMG_FORMATS,
 VID_FORMATS, LoadImages, LoadStreams
from utils.general 
import (LOGGER, check_file, check_img_size, 
check_imshow, check_requirements, colorstr,
 increment_path, 
  non_max_suppression,
 print_args, scale_coords, 
strip_optimizer, xyxy2xywh)
from utils.plots import Annotator, colors, save_one_box
from utils.torch_utils import select_device, time_sync
\end{lstlisting}
这些都是用户自定义的库,不用太管。主要是降低代码复杂度的。
\begin{lstlisting}[language=python]
'''=======================二、设置main函数==================================='''
def main(opt):
    # 检查环境/打印参数,主要是requrement.txt的包是否安装,#
    #用彩色显示设置的参数
    check_requirements(exclude=('tensorboard', 'thop'))
    # 执行run()函数
    run(**vars(opt))
 
 
# 命令使用
# python detect.py --weights 
#runs/train/exp_yolov5s/weights/best.pt 
#--source  data/images/fishman.jpg # webcam
if __name__ == "__main__":
    opt = parse_opt() # 解析参数
    main(opt) # 执行主函数
\end{lstlisting}
 
命令使用是这样的:
python detect.py --weights runs/train/exp_yolov5s/weights/best.pt --source  data/images/fishman.jpg 

这里调用了check_requirements()和run()函数。把命令行参数opt转化为字典传输给run()函数。
check requirements(exclthop')):检查程序所需的依赖项是否已安装。
run(**vars(opt)): 将 opt 变量的属性和属性值作为关键字参数传递给 run() 函数。

if name =='main’:的作用:
一个python文件通常有两种使用方法,第一是作为脚本直接执行,第二是import 到其他的 python 脚本中被调用(模块重用)执行。因此,ifname ='main’:的作用就是控制这两种情况执行代码的过程,在 if name = 'main':下的代码只有在第一种情况下(即文件作为脚本直接执行)才会被执行,而impon 到其他脚本中是不会被执行的。

opt= parse opt0: 解析命令行传进的参数。该段代码分为三部分,第一部分定义了一些可以传导的参数类型,第二部分对于imgsize部分进行了额外的判断(640*640),第三部分打印所有参数信息,opt变量存储所有的参数信息,并返回。main(opt): 执行命令行参数。该段代码分为两部分,第一部分首先完成对于requirements.txt的检查,检测这些依赖包有没有安装;第二部分,将opt变量参数传入,执行run函数。

接着设置参数。\begin{lstlisting}[language=none]

def parse_opt():
   
    parser = argparse.ArgumentParser()

    parser.add_argument('--weights', nargs='+',
     type=str, default=ROOT / 'yolov5s.pt',
      help='model path(s)')
 
    parser.add_argument('--source', type=str, 
    default=ROOT / 'data/images', help=
    'file/dir/URL/glob,
     0 for webcam')
  
    parser.add_argument('--imgsz', '--img',
     '--img-size', nargs='+', type=int, default=[640],
      help='inference size h,w')
    parser.add_argument('--conf-thres', type=float,
     default=0.5, help='confidence threshold')
    parser.add_argument('--iou-thres', type=float,
     default=0.45, help='NMS IoU threshold')
    parser.add_argument('--max-det', type=int,
     default=1000, help='maximum detections per image')
    parser.add_argument('--device', default='',
     help='cuda device, i.e. 0 or 0,1,2,3 or cpu')
    parser.add_argument('--view-img', 
    action='store_true', help='show results')
    parser.add_argument('--save-txt', 
    action='store_true', help='save results to *.txt')
    parser.add_argument('--save-conf',
     action='store_true', help='save confidences
      in --save-txt labels')
    parser.add_argument('--save-crop',
     action='store_true', help='save cropped
      prediction boxes')
    parser.add_argument('--nosave',
     action='store_true', help='do not save
      images/videos')
    parser.add_argument('--classes',
     nargs='+', type=int, help='filter by class:
      --classes 0, or --classes 0 2 3')
    parser.add_argument('--agnostic-nms',
     action='store_true', help='class-agnostic NMS')
    parser.add_argument('--augment',
     action='store_true', help='augmented
      inference')
    parser.add_argument('--visualize',
     action='store_true', help='visualize features')
    parser.add_argument('--update',
     action='store_true', help='update all models')
    parser.add_argument('--project',
     default=ROOT / 'runs/detect',
      help='save results to project/name')
    parser.add_argument('--name',
     default='exp', help='save results to project/name')
    parser.add_argument('--exist-ok',
     action='store_true', help=
     'existing project/name ok, do not increment')
    parser.add_argument('--line-thickness',
     default=3, type=int, help=
     'bounding box thickness (pixels)')
    parser.add_argument('--hide-labels',
     default=False, action='store_true',
     help='hide labels')
    parser.add_argument('--hide-conf',
     default=False, action='store_true',
      help='hide confidences')
    parser.add_argument('--half',
     action='store_true', help=
     'use FP16 half-precision inference')
    parser.add_argument('--dnn',
     action='store_true', help=
     'use OpenCV DNN for ONNX inference')
 
   
 
    opt = parser.parse_args() # 扩充维度
    opt.imgsz *= 2 if len(opt.imgsz) == 1 else 1  # expand
    print_args(FILE.stem, opt) # 打印所有参数信息
    return opt
\end{lstlisting}
这里代码过长有点变形了,看github上的源文件吧。
这个都是描述的参数信息。

详细如下
weights:模型权重文件的路径,默认为YOLOv5s的权重文件路径。
source:输入图像或视频的路径或URL,或者使用数字0指代摄像头,默认为YOLOv5自带的测试图像文件夹。
data:数据集文件的路径,默认为COCO128数据集的配置文件路径。
imgsz:输入图像的大小,默认为640x640。
iou thres:非极大值抑制的IoU阈值,默认为0.45.。
conf thres:置信度阈值,默认为0.25。
max det:每张图像的最大检测框数,默认为1000.
device:使用的设备类型,默认为空,表示自动选择最合适的设备.
view img:是否在屏幕上显示检测结果,默认为False.