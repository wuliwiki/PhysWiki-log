% yolov5 算法代码解析二
% keys CNN
% license Usr
% type Wiki


\begin{enumerate}
\item detect代码详细分析
\end{enumerate}
首先导入python的包。
\begin{lstlisting}[language=python]
import argparse # 解析命令行参数的库
import os # 与操作系统进行交互的文件库 包含文件路径操作与解析
import sys # sys模块包含了与python解释器和它的环境有关的函数。
from pathlib import Path # Path能够更加方便得对字符串路径进行处理
 
import cv2 # sys模块包含了与python解释器和它的环境有关的函数。
import torch #pytorch 深度学习库
import torch.backends.cudnn as cudnn #让内置的cudnn的 auto-tuner 自动寻找最
#适合当前配置的高效算法,来达到优化运行效率的问题
\end{lstlisting}
因为有第一个库,我们可以直接在命令行里输入相关参数,而不用进py文件里一个个去修改,具体后面来介绍。接着获取当前文件的绝对路径。
\begin{lstlisting}[language=python]
FILE = Path(__file__).resolve() 
 # __file__指的是当前文件(即detect.py),FILE
#最终保存着当前文件的绝对路径,比如D://yolov5/detect.py
ROOT = FILE.parents[0] 
 # YOLOv5 root directory  ROOT保存着当前项目的父目录,#比如 D://yolov5
if str(ROOT) not in sys.path:  # sys.path即当前python环境可以运行的路径,
#假如当前项目不在该路径中,就无法运行其中的模块,所以就需要加载路径
    sys.path.append(str(ROOT)) 
     # add ROOT to PATH  把ROOT添加到运行路径上
ROOT = Path(os.path.relpath(ROOT, Path.cwd()))
# relative ROOT设置为相对路径
\end{lstlisting}
这个resolve()方法是Path类的一个方法,它返回当前路径的绝对路径。如果路径是相对的,resolve()会将其转换为绝对路径。此外,它还会解析路径中的任何符号链接(在支持符号链接的操作系统上。
对于ROOT = FILE.parents[0],这一行
FILE:这是一个变量,代表了一个文件的路径。这个路径可以是绝对路径(从根目录开始的完整路径),也可以是相对路径(相对于当前工作目录的路径)。
.parents:这是pathlib.Path对象的一个属性,它返回一个包含当前路径所有父目录的迭代器。例如,如果FILE是'/home/user/documents/file.txt',那么FILE
.parents将会是一个迭代器,包含'/home/user/documents'
、'/home/user'、'/home'
等目录路径。
[0]:通过对parents迭代器使用索引[0],我们获取了FILE路径的第一个父目录,也就是最直接的父目录。在上面的例子中,就是'/home/user/documents'。
ROOT =:这是将获取到的父目录路径赋值给变量ROOT的操作。之后,ROOT就可以用来引用这个父目录路径了。

\begin{lstlisting}[language=python]
from models.common import DetectMultiBackend
from utils.datasets import IMG_FORMATS,
 VID_FORMATS, LoadImages, LoadStreams
from utils.general 
import (LOGGER, check_file, check_img_size, 
check_imshow, check_requirements, colorstr,
 increment_path, 
  non_max_suppression,
 print_args, scale_coords, 
strip_optimizer, xyxy2xywh)
from utils.plots import Annotator, colors, save_one_box
from utils.torch_utils import select_device, time_sync
\end{lstlisting}
这些都是用户自定义的库,不用太管。主要是降低代码复杂度的。
\begin{lstlisting}[language=python]
'''=======================二、设置main函数==================================='''
def main(opt):
    # 检查环境/打印参数,主要是requrement.txt的包是否安装,#
    #用彩色显示设置的参数
    check_requirements(exclude=('tensorboard', 'thop'))
    # 执行run()函数
    run(**vars(opt))
 
 
# 命令使用
# python detect.py --weights 
#runs/train/exp_yolov5s/weights/best.pt 
#--source  data/images/fishman.jpg # webcam
if __name__ == "__main__":
    opt = parse_opt() # 解析参数
    main(opt) # 执行主函数
\end{lstlisting}

 
命令使用是这样的
# python detect.py --weights runs/train/exp_yolov5s/weights/best.pt --source  data/images/fishman.jpg 
