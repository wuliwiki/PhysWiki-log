% 北京大学 2003 年 考研 量子力学
% license Usr
% type Note


\subsection{每小题6分共30分}

\begin{enumerate}
        \item 写出线性、厄米算符的定义式。
        \item 证明:厄米算符的平均值必为实数。
        \item 证明:厄米算符的本征值必为实数。
        \item 满足什么对易关系的算符称为角动量算符?
        \item 证明:如果量子力学中角动量对应的算符不是线性算符,则就不可能存在态叠加原理。
    \end{enumerate}
 \subsection{每小题20分,共40分} 
    \begin{enumerate}
        \item 一个处在一维无限深方势阱中的粒子,试求出其能量本征值的表达式及相应的本征函数。
        \item 试分析这个粒子可能处在什么态上。
    \end{enumerate}
  \subsection{30分}
    已知$C_s$原子的$I=7/2$,其基态电子自旋磁矩和核磁矩的相互作用算符为$A\hat{\vec I}\cdot\hat{\vec S}$,$A$是常数,求$C$原子基态能级的$hf_s$。
  