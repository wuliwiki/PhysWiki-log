% 北京大学 2001 年 考研 固体物理
% license Usr
% type Note

\textbf{声明}:“该内容来源于网络公开资料,不保证真实性,如有侵权请联系管理员”

\subsection{(16分)}
说明金铜$Cu$的品体结构,布拉伐格子,所属晶系、点群和空间群,每个单胞(Conventionalunit cell)中的钢$Cu$原子数;如果品格常数为$a$ ,求正格子空间 W-S原胞的体积和第一布里湖区的体积。
\subsection{(12 分)}
倒格子矢量为$K_h=h_1b_1+h_2b_2+h_3b_3$·

(1)求布里渊区边界方程。

(2)证明正格子中:族晶面($h_1h_2h_3$)和倒格矢$K_h$正交。

(3)倒格矢$K_h$长度正比于品而族($h_1h_2h_3$)面间距$d_{h_1h_2h_3}$的倒数。
\subsection{(16分)}
一维双原子链上最近邻原子间的办常数等于$\beta$,最近邻原子问的距离为$a$,令两种原子的质量分别为$M$和$m$,设$M>m$,试求色散关系$\omega(k)$。证明当$M=m$时,色散关系$\omega(k)$变成一维单原子链的色散关系。(要求推导过程)
\subsection{(16分)}
由泡利不相容原理,金属中费米面附近的自由电子容易被激发,费米能级以下很低能级上的自由电了很难激发,通常称为费米冻结。用此物理图象、

(1)估算在室温下金属中自由电子的比热。

(2)算$T\to0K$金属中白由电子的泡利自旋顺性磁化率。
\subsection{(12分)}
已知一维品体的电子能带为:$E(k) = \frac{\hbar^2}{ma^2} \left( \frac{5}{6} - \cos ka + \frac{1}{6} \cos 2ka \right)$其中$a$是品格常数,试求能带宽度,带顶空穴的的电荷、有效质量和涨经典运动速度。
\subsection{(18分)}
实验测晶格常数为$a$的金属铜的费米面全部在第一布里湖区中近似为球面,只在<111>方向上伸出8个“颈”,与第一布里渊区的正六边形边界面相截成半径为$\rho $的园。其中一个正六边形边界面是倒格矢$G=\frac{2\pi}{a}(1,1,1)$的垂直平分面,与此倒格欠相应的周期势的傅立叶系数为一$U_G$,用自由电了模型处理在第一布里渊区内费米球的大小,用二重简并的近自由电子近似处理第一布里湖区止六边形边界面上电子的能态,求

(1)费米球的半径;

(2)圆周上任点一点的电子能量和周期势的傅立叶系数$U$的关系:

(3)周期势的傅立叶系数"$U_G$与半径$\rho $的关系。(提示:品体中每个铜$Cu$原子只贡献一个自由电子,无论费采面的形状如何,费米面上任何一点的能是都等于费米能。)
\subsection{(10 分)}
以下三小题中任选一题,多作不加分。

(1)简述半导体能带结构的基木特循,并说明它和绝缘体、金属能带结构的基本区。

(2)简述铁磁性的外斯分子场理论利海森堡模型。

(3)从物理图象上说明在迈斯纳态超导体内的电场强度和磁感应强度为何值。