% Julia 调用 C 语言
% license Xiao
% type Tutor

\begin{issues}
\issueDraft
\end{issues}

\begin{itemize}
\item 参考\href{https://docs.julialang.org/en/v1/manual/calling-c-and-fortran-code/}{这里}。 以及\href{https://discourse.julialang.org/t/how-to-make-julia-call-c-c-coded-function/54780/5}{这里的例子}。
\item 基本没有 overhead (如果不考虑 inline function)
\item 下面是 C 语言的例子, 如果是 C++ 需要把要导出的函数定义在 \verb|extern "C" {}| 中。
\end{itemize}

\begin{lstlisting}[language=cpp]
int myfun () {
 return 2;
}
double myfun2 (float i, float j) {
 return i+j;
}
\end{lstlisting}

编译动态链接库:
\begin{lstlisting}[language=bash]
gcc -o test.o -c test.c
gcc -shared -o test.so test.o -fPIC
\end{lstlisting}

在 Julia 中调用
\begin{lstlisting}[language=julia]
test = joinpath(@__DIR__, "test.so")
a = ccall((:myfun,test), Int32, ())
b = ccall((:myfun2,test), Float64, (Float32,Float32), 2.5, 1.5)
\end{lstlisting}
其中 \verb|@__DIR__| 是当前路径。 注意 \verb|ccall| 是一个关键字, 不是函数。 \verb|:函数名| 的类型是 \verb|Symbol|。 第二个参数是返回类型, 第三个参数是函数的参数类型, 后面是具体的参数。 如果只有一个参数, 用 \verb|(类型,)| 即 1-Tuple。 没有参数时, \verb|()| 表示 0-Tuple。
