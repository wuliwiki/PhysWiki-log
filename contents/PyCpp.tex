% Python 调用 C/C++ 程序
% license Xiao
% type Tutor

\begin{issues}
\issueDraft
\end{issues}

\begin{itemize}
\item 参考\href{https://www.geeksforgeeks.org/how-to-call-c-c-from-python/}{这里}。
\item 大概就是先编译一个动态链接库。 但注意 python 只能导入其中的 C 语言函数, 不能直接导入对象等。
\item 所以如果写 cpp 文件, 必须在 \verb`extern "C" {}` 定义要导出的函数(可以只是简单把 cpp 函数和对象封装一下, 但不能使用相同的函数名否则编译器会报错)。
\begin{lstlisting}[language=cpp, caption=test.cpp]
int myfun_imp() {
	return 123456;
}

extern "C" {
	int myfun() { return myfun_imp(); }
}
\end{lstlisting}
\item \enref{编译}{gppLib} 出动态链接库(\verb`test.so`): \verb`g++ -shared -fPIC -o test.so test.cpp`
\item 在同目录打开 python, 确定当前路径用 \verb`os.getcwd()`
\item \verb`from ctypes import cdll`
\item \verb`lib = cdll.LoadLibrary('./test.so')`
\item 现在调用 \verb`lib.myfun()` 就会返回 \verb`123456` 了!
\end{itemize}

\subsubsection{传递数组}
\begin{itemize}
\item 参考\href{https://stackoverflow.com/questions/5862915/passing-numpy-arrays-to-a-c-function-for-input-and-output}{这里}。 例如定义以下函数
\begin{lstlisting}[language=cpp]
void cfun(const double *indatav, size_t size, double *outdatav) 
{
    size_t i;
    for (i = 0; i < size; ++i)
        outdatav[i] = indatav[i] * 2.0;
}
\end{lstlisting}
\item 修改函数的 protorype 使用 numpy 的指针
\begin{lstlisting}[language=python]
import ctypes
from numpy.ctypeslib import ndpointer
lib = ctypes.cdll.LoadLibrary("./ctest.so")
fun = lib.cfun
fun.restype = None
fun.argtypes = [ndpointer(ctypes.c_double, flags="C_CONTIGUOUS"),
                ctypes.c_size_t,
                ndpointer(ctypes.c_double, flags="C_CONTIGUOUS")]
\end{lstlisting}
\item 然后就可以调用了
\begin{lstlisting}[language=python]
indata = numpy.ones((5,6))
outdata = numpy.empty((5,6))
fun(indata, indata.size, outdata)
\end{lstlisting}
\end{itemize}
