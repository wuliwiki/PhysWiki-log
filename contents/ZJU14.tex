% 浙江大学 2014 年 考研 量子力学
% license Usr
% type Note

\textbf{声明}:“该内容来源于网络公开资料,不保证真实性,如有侵权请联系管理员”

\subsection{第一题: 简答题 (30 分)}

(1) 一维问题的能级的最大简并度最大是多少?

(2) 什么是量子力学中的守恒量,它们有什么性质。

(3) 什么是受激辐射?什么是光电效应?

(4) 试写出非简并微扰论的能级修正公式(到二阶)。

(5) 由正则对易关系 $[\hat{x}, \hat{p}] = i\hbar$ 导出角动量的三个分量

\[L_x = i \hbar \left( y \frac{\partial}{\partial z} - z \frac{\partial}{\partial y} \right)~\]

\[L_y = i \hbar \left( z \frac{\partial}{\partial x} - x \frac{\partial}{\partial z} \right)~\]

\[L_z = i \hbar \left( x \frac{\partial}{\partial y} - y \frac{\partial}{\partial x} \right)~\]

的对易关系。
\subsection{第二题 (20 分)}
原子序数较大的原子的最外层电子感受到的原子核和内层电子的总位势可以表示为
\[V(r) = -\frac{e^2}{r} - \lambda \frac{e^2}{r^2}, \quad \lambda = 1~\]
试求其基态能量。
\subsection{第三题 (25 分)}
设电子以给定的能量 $E = \frac{h^2 k^2}{2m}$ 自左入射,遇到一个方势阱
\[V(x) = \begin{cases} 0, & x < 0, x > a \\\\-V_0, & 0 \leq x \leq a \end{cases}~\]
\begin{itemize}
  \item (a)求反射系数和透射系数;
  \item (b)给出发生共振隧穿的条件;
  \item (c)考虑到电子有自旋(自旋向下或向上),你能否借用上面的结果,设计一个量子调控装置,使反射回来的只有自旋向上的电子而没有自旋向下的电子?
\end{itemize}
\subsection{第四题 (25 分)}
试求屏蔽库仑场 $V(r) = \frac{Q}{r}e^{-r/a}$ 的微分散射截面。

提示:可直接用中心势散射的玻恩近似公式的化简形式
\[\sigma(\theta) = \frac{4m^2}{\hbar^4} \left| \int_0^\infty \frac{r \sin(Kr)}{K} V(r) dr \right|^2, \quad \text{其中 } K = 2k \sin\frac{\theta}{2}.~\]
\subsection{第四题 (25 分)}
许多物理问题可以化成两能级系统,如 $\hat{H} = \hat{H}_0 + \hat{H}' = 
\begin{pmatrix}
A + a & b \\
b & B + a
\end{pmatrix}$,其中 $a$, $b$ 为实数,并且远小于 $A - B$,
\begin{itemize}
  \item (a) 试求两能级的精确值;
  \item (b) 再用微扰公式写出能级(到二级近似),并比较两种结果。
\end{itemize}
\subsection{第四题 (25 分)}
当前冷原子物理研究非常活跃。在实验中,粒子常常是被束缚在谐振子势中,因此其哈密顿量为
\[\hat{H}_0 = \frac{\hat{P}^2}{2m} + \frac{1}{2} m \omega^2 \hat{r}^2~\]
假如粒子间有相互作用 $H' = J \hat{S}_1 \cdot \hat{S}_2$,其中 $\hat{S}_1$, $\hat{S}_2$ 分别代表粒子1和粒子2的自旋,参数 $J > 0$。
\begin{itemize}
  \item (a) 如果把两个自旋 $\frac{1}{2}$ 的全同粒子放在上述势阱中,试写出基态能量和基态波函数;
  \item (b) 如果把两个自旋1的全同粒子放在上述势阱中,试写出基态能量和基态波函数(注意:参数在不同范围内,情况会不同)。
\end{itemize}