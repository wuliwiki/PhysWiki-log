% 序数、超限数
% keys 保序映射|序型|序数|超限数
% license Xiao
% type Tutor

\begin{issues}
\issueTODO
\end{issues}

\pentry{序关系\nref{nod_OrdRel}}{nod_7498}
序型是相互同构的偏序集的公共内在属性,良序集的序型称为序数,当强调良序集是无限集时,序数也称超限数。任意两个序数都是可比较的,这是本节的重点。
\subsection{序数}
\begin{definition}{保序映射}
设 $M,M'$ 是两偏序集,$f$ 是 $M$ 到 $M'$ 的映射。如果 $a\leq b(a,n\in M)$ 推出 $f(a)\leq f(b)$,则映射 $f$ 称为\textbf{保序的}。若 $f$ 是双射,同时 $f(a)\leq f(b)$ 当且仅当 $a\leq b$ 时成立,则 $f$ 称为偏序集 $M$ 与 $M'$ 的\textbf{同构映射},这时也称它们\textbf{相互同构}。
\end{definition}
\begin{definition}{序型、序数}
若两偏序集 $M,M'$ 相互同构,则称它们具有同一\textbf{序型}。良序集(\autoref{def_OrdRel_1})的序型称为\textbf{序数},无限的良序集的序数也称\textbf{超限数}。
\end{definition}
\begin{definition}{全序集的有序和}
设 $M_1,M_2$ 时序型分别为 $\theta_1,\theta_2$ 的两个不相交的全序集,在 $M_1\cup M_2$ 中定义全序关系,其中 $M_1$ 的两个元素的序关系如同在 $M_1$ 中一样, $M_2$ 的元素的序关系如同在 $M_2$ 中一样,并且 $M_1$ 的任一元素前于 $M_2$ 的任一元素。这样的全序集称为 $M_1$ 与 $M_2$ 的\textbf{有序和},记作 $M_1+M_2$,其序型称为序型 $\theta_1$ 与 $\theta_2$ 的有序和,记作 $\theta_1+\theta_2$。
\end{definition}
\begin{theorem}{}
有限个良序集的有序和仍是良序集。
\end{theorem}
\begin{corollary}{}
序数的有序和仍是序数。
\end{corollary}
\begin{definition}{有序积}
设 $M_1,M_2$ 是序型分别为 $\theta_1,\theta_2$ 的全序集。
\end{definition}
\begin{theorem}{}
两个良序集的有序积仍是良序集。
\end{theorem}
\begin{corollary}{}
序数的有序积仍是序数。
\end{corollary}
\subsection{序数的比较}
