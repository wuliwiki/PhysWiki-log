% 旋转的陀螺为什么不会倒(科普)
% keys 陀螺|章动|进动|旋转|经典力学
% license Xiao
% type Art


不旋转的陀螺,尖端朝下放在桌上无法立住,因为几乎不可能让陀螺完美对称且完美竖直,于是在重力矩作用下就会倾倒。一旦陀螺旋转起来,尽管重力矩依然要拉着陀螺倾倒,但陀螺似乎只是在不停改变旋转轴的方向,一直不倾倒。

考虑一个空塑料瓶,称从它的瓶盖到瓶底的轴为其“长轴”。将塑料瓶扔到空中,使得其长轴有翻转的角速度,你会发现塑料瓶的长轴大致在一个平面内不停翻转,瓶盖的指向变化范围也差不多有$360^\circ$。但如果扔的同时让塑料瓶本身还绕着长轴旋转,那么塑料瓶的长轴就会在一个双圆锥面内反转,瓶盖变化范围显著小于$360^\circ$。

上述两个现象有一个共同的名字:进动。简单来说,进动就是指,旋转的刚体在旋转轴方向变化的时候会受到一个回转力矩,从而改变其旋转轴变化的方向。我们通过受力分析就能理解进动的成因。


\subsection{术语准备}


为了方便接下来的讨论,我需要定义一些概念,这样能大大简化描述并降低理解的难度。


首先确定研究对象:一个固定在轻质杆上的均质圆环\footnote{这只是简化模型,请不要纠结杆为什么没有质量以及圆环怎么固定在杆上的。把这个简化模型研究清楚以后,结论可以套在真实的复杂模型上。},如\autoref{fig_PrecPS_1} 所示。圆环的半径为$r$,总质量为$M$;杆的两个端点分别取名为$A$和$B$,它们到圆环的几何中心的距离相等;称圆环的几何中心为$O$点。设$A$、$B$间的距离为$2d$。

\begin{figure}[ht]
\centering
\includegraphics[width=6cm]{./figures/649e4f73fbad7bc3.pdf}
\caption{陀螺模型示意图。} \label{fig_PrecPS_1}
\end{figure}


要描述$AB$轴的方向变化,可以用$A$和$B$的相对速度。比如说,如果陀螺在平移,那么两个点的相对速度为$0$,$AB$轴的方向不会改变;但两个点只要有非零的相对速度,那么$AB$轴的方向就一定会改变。为方便计,我们称$AB$轴的方向改变为陀螺的偏转。


以点$O$为参考点,定义陀螺受到的力矩。具体来说,如果陀螺在某点$P$处受到力$\bvec{F}$作用,那么陀螺受到的力矩就是$\bvec{\tau}=\bvec{r}\times \bvec{F}$,其中$\bvec{r}$是从$O$到$P$的位移向量。$\bvec{\tau}$的方向可以用右手定则确定:伸出手掌,四指指向$\bvec{r}$的方向,调整手掌的方向使得$\bvec{F}$的方向垂直从手背穿入、手心传出,此时竖起大拇指,拇指所指方向即为$\bvec{\tau}$的方向。


当陀螺受到沿着$AB$轴的力矩的时候,并不会因此偏转。如果受到的力矩垂直于$AB$轴,则陀螺会偏转,但$A$相对$B$的\textbf{加速度}却并不是沿着力矩的方向,而是垂直于力矩的方向。举个例子,从你的视角看,如果对点$A$和$B$分别施加一个水平向左和水平向右的等大的力,那么陀螺受到的力矩是水平指向你的,但陀螺因此得到的偏转方向、或者说$A$相对$B$的加速度,是水平向左的。



但无论如何,力矩的效果是给$A$提供一个相对$B$的加速度,而我们描述的正是$A$相对$B$的运动状态,因此接下来讨论力矩时只讨论它提供的相对加速度。称$A$相对$B$的速度为陀螺的\textbf{偏转速度},加速度为\textbf{偏转加速度}。

另外请注意,力矩和力一样都是向量,因此可以分解为垂直$AB$轴和沿着$AB$轴方向的分量,其中只有第一个分量影响陀螺的偏转,第二个分量影响的是圆环自转的加速度。




\subsection{受力分析}



当圆环本身没有自转的时候,如果有偏转速度,则$AB$轴会一直在一个平面内以恒定的角速度转动,此时偏转速度虽然一直在改变,但没有脱离以$d$为半径的圆周。

但如果圆环绕着$AB$轴转起来,此时再有偏转速度,就会出现垂直于偏转速度的偏转加速度。


在某一时刻,建立惯性系的三维直角坐标系以方便讨论。以$O$点此刻所在位置为坐标系原点,$OA$为$z$轴;取圆环上某一点$P$,以$OP$为$x$轴;$y$轴则垂直$OAP$平面,且整个坐标系符合右手定向,具体如\autoref{fig_PrecPS_2} 所示。





\begin{figure}[ht]
\centering
\includegraphics[width=14cm]{./figures/3dd80df4184c6aa8.pdf}
\caption{建立的惯性系三维直角坐标系示意图。为了清楚传达,此处使用了三视图来表达各质点的位置关系。俯视图里$A$、$B$、$O$三点重合;主视图中蓝色的$\otimes$表示“垂直纸面向里”的方向,即$P$点速度的方向。} \label{fig_PrecPS_2}
\end{figure}

令圆环绕着$AB$自转,使得这一瞬间质点$P$的速度方向沿着$y$轴正向,如\autoref{fig_PrecPS_2} 所示,将该速度表示为
\begin{equation}
\bvec{v} = 
\begin{pmatrix}
0\\v\\0
\end{pmatrix}
~. 
\end{equation}

现在考虑陀螺有沿着$y$轴\textbf{正方向}的\textbf{偏转速度},则一小段时间后$P$点的速度会多出来一个$z$向分量,指向$z$轴\textbf{负方向}。这意味着,\textbf{如果}要保持陀螺的偏转速度在$yz$平面内,就\textbf{需要}对$P$点施加一个指向$z$轴\textbf{负方向}的力;反过来,这等同于说,$P$点受到了一个指向$z$轴\textbf{正方向}的力\footnote{类比加速的电梯模型。当你身处一个太空中加速上升的电梯时,为了保持和电梯相对静止,你需要受到一个指向电梯上方的力来和电梯同步加速;反过来,你也可以认为在电梯参考系看来,你受到了一个指向电梯下方的力的作用,需要再受到一个指向电梯上方的力来平衡。},因此才需要额外给$P$点一个力来平衡它。

\begin{exercise}{}
考虑圆环上其它质点,它们也会因为偏转速度的存在而受到一个指向$z$轴正或负方向的力,哪些点受到的力是正向、哪些是负向?有两个点不会因为偏转速度而受力,是哪两个点?
\end{exercise}


$P$点由于偏转速度所受的力,对整个陀螺产生了偏转力矩,这个力矩的作用是产生了一个沿着$x$轴\textbf{负方向}的\textbf{偏转加速度}。在给定质点质量以及陀螺自转角速度时,该偏转加速度仅和偏转速度相关,偏转加速度的大小正比于偏转速度的大小。

考察圆环上所有点不难发现,任何点产生的偏转力矩和偏转加速度都在同一个方向上,因此所有质点提供的偏转加速度的总和就是沿着$x$轴\textbf{负方向}。称所有质点产生的偏转力矩的总和为\textbf{回转力矩}。










\subsection{陀螺进动}


\subsubsection{无外力环境下的陀螺进动}


文章开头提到的“扔塑料水瓶”实验,其实是模拟了无外力环境下的陀螺进动。


“偏转加速度大小正比于偏转速度大小、方向垂直于偏转速度”,这个情况和电磁学中静磁场的洛伦兹力很像。带电粒子仅受静磁场的洛伦兹力时,其加速度大小总是正比于速度大小,方向垂直于速度的方向,结果是带电粒子在平面内做圆周运动。陀螺的偏转速度和偏转加速度同理,只不过$A$、$B$两点是在一个球面而非平面上运动,但即便如此,$A$点的运动轨迹也因此成了绕着某固定轴轴旋转的一个小圆,其半径小于$d$;而$AB$轴则绕着该固定轴,画出一个双圆锥的形状。


这种$AB$轴绕着固定轴画出双圆锥形状而非一个原盘面的现象,就叫做\textbf{进动(precession)}。





\subsubsection{地球上的陀螺不倒之谜}



我们考虑一个很有视觉冲击力的情况:陀螺的$B$点被绳子吊起来,让陀螺高速自转,结果陀螺并不会倒下,而是绕着绳子所在的轴旋转,如\autoref{fig_PrecPS_3} 所示。


\begin{figure}[ht]
\centering
\includegraphics[width=10cm]{./figures/9d408551ac225165.pdf}
\caption{陀螺进动示意图。图中在地球表面,重力方向竖直向下。陀螺的$B$点吊在绳子上,即$B$点位置固定,同时陀螺的圆环部分绕$AB$轴自转,自转方向如图中红色弯曲箭头表示。陀螺有偏转速度,表现为$A$点的速度,如图中橙色直箭头所示。} \label{fig_PrecPS_3}
\end{figure}


陀螺如图中那样偏转时,回转力矩倾向于将陀螺抬起,而重力矩倾向于将陀螺压下,两个力矩平衡,故陀螺不会倒下。








