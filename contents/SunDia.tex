% 日晷的计算
% license Xiao
% type Tutor

\begin{issues}
\issueOther{一根垂直地面的杆如何判断时间?(只能用顶点判断)}
\end{issues}

\subsection{水平日晷}
\pentry{解三棱锥顶角\nref{nod_PrmSol}}{nod_652c}

常见的水平日晷如\autoref{fig_SunDia_2} 和\autoref{fig_SunDia_1} 所示。 把图片打印后, 把直角三角形的底边垂直固定在半圆的 12 点方向, 然后把半圆水平放于地面, 12 点方向指向正北(地轴北)即可。 此时三角形的斜边, 也就是日晷的指针与地轴平行。 太阳绕日晷指针以每小时 15° (一天 360°)的角速度匀速转动。

\begin{figure}[ht]
\centering
\includegraphics[width=8cm]{./figures/cc4c4cd83c583428.png}
\caption{水平日晷(来自 Wikipedia)} \label{fig_SunDia_2}
\end{figure}
该类型日晷的一个画图网页见\href{https://www.blocklayer.com/sundial.aspx}{这里}。
\begin{figure}[ht]
\centering
\includegraphics[width=10cm]{./figures/facac378267a6b72.png}
\caption{北纬 30° 的日晷} \label{fig_SunDia_1}
\end{figure}

日晷圆盘上的刻度与所在纬度有关, 若将其放在北极或南极, 那么直角三角形的斜边就变为一个和地面垂直的线段, 且表盘上的刻度是均匀的。 相反, 若把这种日晷放在赤道上, 那么三角形的斜边将会与 12 点的刻度共线, 此时这种日晷将失效。

刻度的具体计算并不复杂, 公式为
\begin{equation}
\beta = \tan^{-1}(\sin\alpha \tan \theta)~.
\end{equation}
其中 $\beta$ 是表盘上某个刻度到 12 点刻度的夹角; $\theta$ 是太阳关于地轴(指针)旋转的角度——每小时增加 15 度, 正午为 0 度; $\alpha$ 是当地纬度。 可以验证在北极点处即 $\alpha = \pi/2$ 时 $\beta = \theta$。

证明: 令\autoref{eq_PrmSol_3}  中 $\theta_2 = \pi/2$, $\theta_1 = \theta$ 即可。 表盘的中心就是三棱锥顶点。

我们给出水平日晷的 \enref{Matlab 画图}{MatPlt}代码, 结果类似\autoref{fig_SunDia_1}。
\begin{lstlisting}[language=matlab, caption=sunDial.m]
alpha = pi/6; % 纬度
th = linspace(-pi/2, pi/2, 13);
beta = atan(sin(alpha).*tan(th));
th0 = linspace(-pi/2, pi/2, 1000);
figure; plot(cos(th0),sin(th0), 'k'); hold on;
for i = 1:13
    plot([0,cos(beta(i))], [0,sin(beta(i))], 'k');
end
axis equal; view(-90, 90);
axis([-0.1, 1.1, -1.1, 1.1]);
\end{lstlisting}

\subsection{赤道平面日晷}
另一种在任何纬度都适用的方案是让刻度盘始终与指针保持垂直, 而指针始终平行于地轴。 这种日晷的结构相对更复杂, 但表盘上的刻度始终是均匀的, 只需调整表盘与底座之间的夹角就能适用于不同纬度。
\begin{figure}[ht]
\centering
\includegraphics[width=8cm]{./figures/b2b8356c122bb018.png}
\caption{故宫的日晷(来自 Wikipedia)} \label{fig_SunDia_3}
\end{figure}
