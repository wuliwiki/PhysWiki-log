% 轨道角动量(量子力学)
% 轨道角动量|角动量|量子力学|升降算符

\begin{issues}
\issueDraft
\issueMissDepend
\end{issues}

\pentry{量子力学的算符和本征问题\upref{QM1}, 角动量定理、角动量守恒\upref{AMLaw}, 对易厄米算符与共同本征矢\upref{Commut}, 升降算符\upref{RLop}}

\subsection{角动量的量子化}

经典力学中单个粒子的角动量公式是
\begin{equation}\label{QOrbAM_eq1}
\bvec L = \bvec r \cross \bvec p
\end{equation}
其中 $\bvec r$ 是参考点到物体的位矢, $\bvec p$ 是粒子的动量.或者写成直角坐标系中的分量形式(令原点为参考点)
\begin{equation}\label{QOrbAM_eq2}
L_x = y p_z - z p_y \qquad
L_y = z p_x - x p_z \qquad
L_z = x p_y - y p_x
\end{equation}   
这是我们第一次接触量子力学中的三维问题. 和一维问题一样, 量子力学的基本假设告诉我们, 三个方向的角动量算符就是把\autoref{QOrbAM_eq2} 中的 $x, y, z, p_x, p_y, p_z$ 替换为对应的算符.

我们也可以使用矢量算符的概念, 例如把 $\bvec r = x\uvec x + y\uvec y + z\uvec z$ 看作位置矢量算符, 那么它作用在三维波函数 $\psi(\bvec r) = \psi(x, y, z)$ 是一个复矢量场\upref{Vfield}
\begin{equation}
\bvec r \psi(\bvec r) = x \psi(\bvec r) \uvec x + y \psi(\bvec r) \uvec y + z \psi(\uvec r) \uvec z
\end{equation}
同理可以定义动量的矢量算符 $\bvec p = p_x \uvec x + p_y\uvec y + p_z\uvec z$. 那么角动量矢量算符可以用\autoref{QOrbAM_eq1} 定义. 我们还可以定义角动量平方(标量)算符(也可以记为非粗体的 $L^2$)
\begin{equation}
\bvec L^2 = \bvec L \vdot \bvec L = L_x^2 + L_y^2 + L_z^2
\end{equation}
除了 $x, y, z$ 三个方向的角动量分量, 我们可以将任意方向的角动量分量 $\uvec a \vdot \bvec L$ 表示为算符且都与 $\bvec L^2$ 算符对易.
\begin{equation}
L_a = a_x L_x + a_y L_y + a_z L_z \qquad (a_x^2 + a_y^2 + a_z^2 = 1)
\end{equation}

角动量算符 $L^2, L_z$ 通常在球坐标中表示, 详见 “球坐标系中的角动量算符\upref{SphAM}”.

\subsection{对易关系}

对于角动量分量,理想的状况是,如果能解出本征方程
\begin{equation}
\bvec L \psi  = \bvec l\psi \qquad (\text{误})
\end{equation}
我们就能得到矢量本征值 $\bvec l$,然后测量 $\bvec L$ 本征态的结果就一定是 $\bvec l$. 但事实上, $\bvec L$ 或者其他矢量算符几乎从来不单独使用,因为上式无解.为什么? 要解上式,充分必要条件就是要存在 $\psi$,使三个分量同时有解
\begin{equation}
L_x \psi  = l_x \psi \qquad
L_y \psi  = l_y \psi \qquad
L_z \psi  = l_z \psi 
\end{equation}   
不幸的是,$L_x$, $L_y$, $L_z$ 中任意两个都不对易,所以没有共同的本征函数\upref{Commut}. 可以证明三个算符之间的对易关系为
\begin{equation}
[L_x, L_y] = \I\hbar L_z \qquad
[L_y, L_z] = \I\hbar L_x \qquad
[L_z, L_x] = \I\hbar L_y
\end{equation}

事实上,三个分量中我们只能最多知道一个(海森堡不确定原理).% 链接未完成,一定要解释一下有些量为什么不能同时得到),
通常情况下,我们选择解 $L_z$ 的本征方程 $L_z \psi = l_z\psi$. 

比较幸运的是, $L^2$ 和 $L_x, L_y, L_z$ 都对易(或者任意 $L_n$),所以必然存在一套本征函数,同时是 $L_x, L_y, L_z$ 其中一个和 $L^2$ 的本征函数\upref{Commut}. 我们习惯计算 $L^2$ 和 $L_z$ 的共同本征矢.

\subsection{升降算符和本征值}

如果要解 $L^2$ 和 $L_z$ 的共同本征函数, 通常的方法是先把算符的表达式转换到球坐标中再解本征方程\upref{SphNab}.但是我们现在先用一种更简单的方法, 就是在简谐振子问题\upref{QSHOop}中已经见过的升降算符法, 来绕过本征函数直接求出共同波函数的简并情况以及对两个算符的本征值.

我们把共同本征函数记为 $\ket{l,m}$, 令其同时满足两个本征方程
\begin{equation}
L^2\ket{l,m} = l(l+1)\hbar^2 \ket{l,m}
\end{equation}
\begin{equation}
L_z\ket{l,m} = m\hbar \ket{l,m}
\end{equation}
可以证明 $L_z$ 的升降算符分别为
\begin{equation}
L_\pm = L_x \pm \I L_y
\end{equation}
根据升降算符\upref{RLop} 中的一种定义,要证明它们是升降算符,只要证明 $[L_z, L_\pm] \propto L_\pm$ 即可.结论是 % 链接未完成
\begin{equation}
[L_z, L_\pm] =  \pm \hbar L_ \pm
\end{equation}
可知 $L_\pm$ 每作用在 $L_z$ 的本征态上一次, 会把本征值增加 $\pm\hbar$. 另外易证 $[L^2, L_\pm] = 0$, 所以 $L_\pm$ 并不会改变 $L^2$ 的本征值. 总结起来就是
\begin{equation}
L_\pm \ket{l,m} = A_\pm \ket{l,m \pm 1}
\end{equation}
其中 $A_\pm$ 是归一化系数. 类似简谐振子的升降算符, $A_\pm$ 的计算为(详见 “轨道角动量升降算符归一化\upref{QLNorm}”)
\begin{equation}
L_\pm \ket{l, m}  = \hbar \sqrt{l(l + 1) - m(m \pm 1)} \ket{l, m \pm 1} 
\end{equation}

\begin{table}[ht]
\centering
\caption{轨道角动量符号(剩下的按字母表顺序排序)}\label{QOrbAM_tab1}
\begin{tabular}{|c|c|c|c|c|c|c|c|c|}
\hline
符号 & S & P & D & F & G & H & I & K\\
\hline
$L$ & 0 & 1 & 2 & 3 & 4 & 5 & 6 & 7\\
\hline
符号 & L & M & N & O & Q & R & T & U \\
\hline
$L$ & 8 & 9 & 10 & 11 & 12 & 13 & 14 & 15 \\
\hline
\end{tabular}
\end{table}
