% 线性算子的行列式
% license Xiao
% type Tutor

\begin{issues}
\issueTODO
\end{issues}

\pentry{线性映射的张量积\nref{nod_linSW},行列式\nref{nod_Deter},对偶空间\nref{nod_DualSp},向量空间的对称/反对称幂\nref{nod_vecSAS}}{nod_b2cb}
% Giacomo:帮我看看到底是哪篇文章引用了对偶空间。。。

\subsection{体积形式}

% \begin{theorem}{}
% 全体反对称 $k$-形式(参考对称/反对称多线性映射\upref{SASmap})构成的向量空间与 ${\large \wedge}^k V^*$ 自然同构(不依赖基的选取)。
% \end{theorem}

% \addTODO{在合适的文章中添加这个证明}
% \textbf{证明:}全体反对称 $k$-形式构成的向量空间自然同构于 $({\large \wedge}^k V)^*$\autoref{the_vecSAS_1}~\upref{vecSAS},它自然同构于 ${\large \wedge}^k V^*$(TODO:证明)。

% 全体体积形式构成的向量空间 ${\large \wedge}^k V^* \cong \mathbb{F}$\footnote{有时候我们要求体积k式非零。}。

我们把 $n$ 维向量空间 $V$ 上的反对称 $n$-形式称为\textbf{体积形式}。

当$\mathbb{F} = \mathbb{R}$时,这个定义符合我们的直觉,体积指的是定向平行多胞体的有向体积(signed volume)。

\begin{definition}{平行多胞体}
$n$ 维向量空间 $V$ 的一个 $k$ 维\textbf{平行多胞体}(又称\textbf{超平行体})简称\textbf{平行 $k$-胞体},是由 $k$ 个向量 $v_1, \cdots, v_k$ 生成的闭子集
\begin{equation}
P: = \left\{ \sum_i a_i v_i \mid a_i \in [0, 1] \right\}~.
\end{equation}
如果 $\{v_1, \cdots, v_k\}$ 线性无关我们就称 $P$ 是\textbf{非退化的};如果 $\{v_1, \cdots, v_k\}$ 选定了一组顺序我们就称它是\textbf{定向的}\footnote{严格来说这时候我们应该记定向平行多胞体为 $(P, v_1, \cdots, v_k)$。}。
\end{definition}

\begin{example}{}
(定向)平行 $1$-胞体就是一条(有向)线段,$2$-胞体就是平行四边形,$3$-胞体就是平行六面体。
\end{example}

\begin{figure}[ht]
\centering
\includegraphics[width=6cm]{./figures/67dd2ddf5f1abfaa.png}
\caption{平行四边性} \label{fig_APdet_3}
\end{figure}
% 如果上下白边太宽可以切掉,但我不知道怎么操作

\begin{figure}[ht]
\centering
\includegraphics[width=6cm]{./figures/baab9f7390c98860.png}
\caption{平行六面体} \label{fig_APdet_2}
\end{figure}

\begin{example}{有向长度}
$V$ 上的一个反对称 $1$-形式(即线性函数) $l: V \to \mathbb{F}$ 给每个定向平行 $1$-胞体 $(P, v_1)$ 定义了一个有向长度 $l(v_1)$。
\end{example}
更一般的,反对称 $k$-形式 $\omega$ 给每个定向平行 $k$-胞体定义了一个有向 $k$ 维体积:
\begin{equation}
\opn{vol}_k(P, v_1, \cdots, v_k): = \omega(v_1, \cdots, v_k)~.
\end{equation}

\addTODO{画图}
\begin{example}{}
我们以三维空间的正方体 $P: = [0, 1]^3 \subseteq \mathbb{R}^3$ 为例,定向平行六面体 $(P, e_1, e_2, e_3)$ 和 $(P, e_2, e_1, e_3)$ 是关于平面 $x = y$ 对称的,因此我们把它们的体积是为相反的(一正一负)。
\end{example}


\subsection{行列式}

$f$ 的 $k$ 阶张量幂$f^{\otimes k}: V^{\otimes k} \to W^{\otimes k}$ 保留了 $V^{\otimes k}$ 元素的反对称性,
% \begin{equation}
% \begin{aligned}
% f(v_1 \cdots v_k) &= \sum_{\sigma \in S_n} f(v_{\sigma(1)}) \otimes \cdots \otimes f(v_{\sigma(k)}) \\
% &= f(v_1) \cdots f(v_k)~,
% \end{aligned}~
% \end{equation}

\begin{equation}
\begin{aligned}
f(v_1 \wedge \cdots \wedge v_k) &= \sum_{\sigma \in S_n} \opn{sign}(\sigma) f(v_{\sigma(1)}) \otimes \cdots \otimes f(v_{\sigma(k)}) \\
&= f(v_1) \wedge \cdots \wedge f(v_k)~.
\end{aligned}~
\end{equation}

% 因此可以定义 ${\large \wedge}^k f: {\large \wedge}^k V \to {\large \wedge}^k W$ 和 $\opn{Sym}^k f: \opn{Sym}^k V \to \opn{Sym}^k W$

如果 $f$ 是一个线性算子,即 $f: V \to V$,我们发现 $k$ 阶反对称幂子空间 ${\large \wedge}^k V$ 是 $f^{\otimes k}$ 的不变子空间\upref{InvSP},因此 ${\large \wedge}^k f$ 是 ${\large \wedge}^k V$ 上的线性算子。

特别的记 $V$ 的维度为 $n$,我们有 $\dim({\large \wedge}^n V) = 1$,因此
\begin{equation}
{\large \wedge}^k f: {\large \wedge}^n V \to {\large \wedge}^n V~
\end{equation}
是一个一维到一维的映射,此时存在 $\lambda \in \mathbb{F}$ 满足
\begin{equation}
({\large \wedge}^k f)(v) = \lambda v~
\end{equation}
$\lambda$ 就是我们想要的 $f$ 的行列式。

\begin{definition}{行列式}
线性算子 $f: V \to V$ 被定义为 $\lambda$。
\end{definition}

\addTODO{证明1:线性算子的行列式等于它对应的矩阵的行列式}
\addTODO{证明2:线性算子的行列式等于体积形式的变化因子}

