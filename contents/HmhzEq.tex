% 一维齐次亥姆霍兹方程
% keys 亥姆霍兹方程|二阶常系数|线性|齐次|微分方程
% license Xiao
% type Tutor

\begin{issues}
\issueNeedCite
\end{issues}

\pentry{二阶常系数微分方程\nref{nod_Ode2}}{nod_82f9}

一维齐次亥姆霍兹方程可以记为
\begin{equation}
\dv[2]{y}{t} + \omega^2 y = 0~,
\end{equation}
这里 $\omega$ 为实数。

\subsection{通解}
这个方程属于二阶常系数线性齐次方程, 可以假设 $\E^{rt}$ 为方程的解, 代入原方程得特征方程
\begin{equation}
r^2 + \omega^2 = 0~.
\end{equation}
解得 $r = \pm \I \omega$, 即方程在复数域的通解为
\begin{equation}
y = C_1 \E^{\I\omega t} + C_2 \E^{-\I\omega t}~,
\end{equation}
其中 $C_1, C_2$ 是复常数。

如果选择恰当的 $C_1$ 和 $C_2$, 可以使通解变为实数函数。 令
\begin{equation}
C_1 = C_{1R} + \I C_{1I}~, \qquad
C_2 = C_{2R} + \I C_{2I}~.
\end{equation}
把 $y(t)$ 分解为实部和虚部, 令虚部为零, 可得所有可能的实数解
\begin{equation}\label{eq_HmhzEq_1}
\begin{aligned}
y(t) &= [(C_{1R} + C_{2R}) \cos\omega t + (C_{2I} - C_{1I}) \sin\omega t] \\
& + \I [(C_{1R} - C_{2R}) \sin \omega t + (C_{1I} + C_{2I}) \cos \omega t]~.
\end{aligned}
\end{equation}
令虚部为 $0$, 则 $C_{1R} = C_{2R}$, $C_{1I} = -C_{2I}$。
\begin{equation}
y = 2C_{1R}\cos\omega t + 2C_{2I}\sin\omega t~,
\end{equation}
所以最一般的实数通解具有 $A\cos\omega t + B\sin\omega t$ 的形式, 对比系数
\begin{equation}
A = 2C_{1R} = 2C_{2R}~,\quad B = 2C_{2I} = -2C_{1I}~.
\end{equation}
