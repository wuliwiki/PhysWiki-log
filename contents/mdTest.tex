% md_latex_测试
% license Usr
% type Test

不仅是标题,所有地方的 \verb`\label{标签}` 都转换为 \verb`{#标签}`

\begin{equation}
1+1~;
\end{equation}

\subsubsection{md 模式}
\begin{itemize}

\subsubsection{文章信息}文章信息风发
风f
\item 所有 \verb`\f风ootnote{文字}` 变为 \verb`[^数字]`,代码最后用 \verb`[^数字] 文字` (每个占一行或多行ff)。每个数方法fef风风发生生世世生生世世是事实,,feaaaaffffffffff风风风e字唯 一即可,且只能被得到的风发反,点点滴滴方法啊风风风呃风反复复为丰富ef俄方方法定义和引用一次。 latex 转 md 时从 1 开始按顺序生babababa风风成。风丰富啊我发



\item 【先不做】\verb`itemize` 环境转换时直接把 \verb`\begin{itemize}` 和 \verb`\end{itemize}` 去掉,并把中间的 \verb`\item` 替换为 \verb`-` 即可。 中间如果有空行不要管。
\item 【先不做】\verb`enumerate` 环境转换时也类似,但是把 \verb`\item` 替换为数字。 如果能hi we fa b g i a b gei b ge g bai be g bao be g ba j o n g be o n g j o n g bo n g bo n g bo n g b做到的话,如果两个数字之间按下了回车,那就自动插入适当的数字并把后面的连续数字都加一。
\item \verb`lstlisting` 环境(见f f f f \enref{Sample.tex}{Sample})转换后开头为:\verb|```语言 标题|(占一行),结尾 \verb|```|(占一行)。 其中 \verb`标题` 可选。



\item \verb`equation` 环境(如果内部不直接嵌套 \verb`aligned` 环境)变为 \verb`::: ... :::`。
\item \verb`figure` 环境变为 \verb`![标题](文件名){#标签}{width=8cm}`,其中 \verb`文件名` 不要包含路径,但包含拓展名。
\item 其他环境如 \verb`example`,\verb`exercise`,\verb`lemma`, 开头用 \verb`:::环境名 标题 {#标签}`,结尾用 \verb`:::`。 标签可选。
\end{itemize}

