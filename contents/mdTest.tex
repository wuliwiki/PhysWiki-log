% md_latex_测试
% license Usr
% type Test

\subsubsection{md 模式}
\begin{itemize}
\item 仅在前端实现,读取 latex 时转成 md, 保存时转成 latex。 完全不影响后端。
\item 菜单栏中的按钮和高亮算法也要做相应的修改
\item \verb`\subsection{标题}\label{标签}` 变为 \verb`## 标题 {#标签}` 其中标签的定义是可选的。
\item \verb`\subsubsection` 也同理,用 \verb`###`。 注意转换成 latex 时, \verb`标题` 两边的空格不要包含进去, \verb`{#标签}` 里面不能有空格。
\item 不仅是标题,所有地方的 \verb`\label{标签}` 都转换为 \verb`{#标签}`
\item 所有 \verb`\autoref{标签}` 变为 \verb`@标签`,后面加一个空格(无论 \verb`\autoref{标签}` 后面是否已经有空格)。 转换回去的时候后面的空格删掉。 如果 md 的 \verb`标签` 中出现了非字母数字和下划线的字符, 就报错并提示是否忘了加空格。
\item 所有 \verb`\upref{标签}` 变为 \verb`@^标签`,后面加一个空格。
\item 所有 \verb`\aref{文字}{标签}` 变为 \verb`[文字](#标签)`
\item 所有 \verb`\cite{标签}` 变为 \verb`[@标签]`
\item 所有 \verb`\footnote{文字}` 变为 \verb`[^数字]`,代码最后用 \verb`[^数字] 文字` (每个占一行或多行)。 每个数字唯一即可,且只能被定义和引用一次。 latex 转 md 时从 1 开始按顺序生成。
\item 粗体 \verb`\textbf{文字}` 变为 \verb`**文字**`
\item \verb|\verb`代码`|(注意 delimiter 也可能是其他任意单个符号如 \verb`\verb+代码+` 变为 \verb|`代码`|, 如果\verb`代码`中已经有 \verb|`| 符号, 就使用多个作为 delimiter 如 \verb|``代码``|。 如果 \verb`代码` 的第一个或最后一个字符是 \verb|`|, 就用一个空格和 delimiter  隔开, 但转换回 latex 时删掉。
\item 【先不做】\verb`itemize` 环境转换时直接把 \verb`\begin{itemize}` 和 \verb`\end{itemize}` 去掉,并把中间的 \verb`\item` 替换为 \verb`-` 即可。 中间如果有空行不要管。
\item 【先不做】\verb`enumerate` 环境转换时也类似,但是把 \verb`\item` 替换为数字。 如果能做到的话,如果两个数字之间按下了回车,那就自动插入适当的数字并把后面的连续数字都加一。
\item \verb`lstlisting` 环境(见 \enref{Sample.tex}{Sample})转换后开头为:\verb|```语言 标题|(占一行),结尾 \verb|```|(占一行)。 其中 \verb`标题` 可选。
\item \verb`\begin{equation}\begin{aligned} ... \end{aligned}\end{equation}` 环境变为 \verb`:::a ... :::` 注意两个相邻命令之间可能有空格或回车。
\item \verb`equation` 环境(如果内部不直接嵌套 \verb`aligned` 环境)变为 \verb`::: ... :::`。
\item \verb`figure` 环境变为 \verb`![标题](文件名){#标签}{width=8cm}`,其中 \verb`文件名` 不要包含路径,但包含拓展名。
\item 其他环境如 \verb`example`,\verb`exercise`,\verb`lemma`, 开头用 \verb`:::环境名 标题 {#标签}`,结尾用 \verb`:::`。 标签可选。
\end{itemize}


\subsection{标题}\label{abc} 

## 标题 {#标签}