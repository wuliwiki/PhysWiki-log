% 粒子物理标准模型
% keys 标准模型|粒子物理
% license Usr
% type Tutor

\begin{issues}
\issueMissDepend
\issueAbstract
\end{issues}

符号 $h.c.$ 代表前一项的厄密共轭。

对 G-W-S 电弱统一模型先进行总结。假设 $U(1)$ 对称性生成的算符是 $\Upsilon$(对应弱超荷是 $Y$),$T^i$ 是(弱)同位旋空间 $SU(2)$ 对称性的生成的算符,经过 Weinberg 转动后有场 $B_\mu$ 与 $W^i_\mu$,则协变导数化为 
\begin{equation}
D_\mu = \partial_\mu + \mathrm i g_1 \Upsilon B_\mu + \mathrm i g_2 T^i W_\mu^i~.
\end{equation}
从而拉氏量(密度)可以写作 
\begin{equation}
	\begin{aligned}
		\mathcal L_{GWS} =& -\frac{1}{4} B_{\mu\nu} B^{\mu\nu} - \frac{1}{4} W_{\mu\nu}^i W^{i\mu\nu}\\
		&+ \sum_{x = e, \mu, \tau} \mathrm i \left(\bar{L}_x \gamma^\mu D_\mu L_x + \bar{R}_x \gamma^\mu D_\mu R_x\right)\\
		&+ \sum_{x=1}^3 \mathrm i \left(\bar{L}_{x} \gamma^\mu D_\mu L_x + \bar{R}_{ux} \gamma^\mu D_\mu R_{ux} + \bar{R}_{dx} \gamma^\mu D_\mu R_{dx}\right) \\
		&+(D_\mu \phi)^\dagger (D_\mu \phi) - m^2 \phi^\dagger \phi - \lambda (\phi^\dagger \phi)^2 \\
		&+ \sum_{x = e, \mu, \tau} g_1 \left(\bar{L}_x \phi R_x + \bar{R}_x \phi^\dagger L_x\right) \\
		&- \sum_{(u, d) \to (c, s), (t, b)} \left(g_d \bar{L}_u \phi R_d + g_u \bar{L}_u \phi_C R_u + h.c.\right) ~.
	\end{aligned}
\end{equation}

其中各个求和代表替换为对应的粒子。在计算路径积分的时候要注意加入 F-P 鬼项。对 G-W-S 模型的拉氏量略作解释:
前两项分别给出自由规范场 $B_\mu$ 和 $W_\mu^i$, 而后第二行给出了轻子场, 第三行给出了夸克场.
接下来第四行给出的是 Higgs 场, 第五行给出 Higgs 场与轻子的耦合, 第六行给出 Higgs 场与夸克的耦合.

囊括强相互作用,是加入夸克在色空间的 $SU(3)$ 对称性,仅需要推广协变微商为 
\begin{equation}
	D_\mu = \partial_\mu + \mathrm i g_1 \Upsilon B_\mu + \mathrm i g_2 T^1 W_\mu^i + \mathrm i g_3 T^a A^a_\mu ~,
\end{equation}
从而用 $g_3$ 考虑夸克与胶子的耦合,并在拉氏量密度中增加自由胶子场
\begin{equation}
	-\frac14 F_{\mu\nu}^a F^{a\mu\nu} ~.
\end{equation}
此处 $F^a$ 是对应味 $a$ 的自由胶子场。

标准模型基于前提假设:
\begin{itemize}
	\item 初始时轻子无质量,质量通过与 Higgs 场耦合获得,有弱 $SU(2)$ 和 $U(1)$ 对称性;
	\item 初始时夸克与轻子一致,无质量而通过与 Higgs 场耦合得到,有弱 $SU(2)$ 和 $U(1)$ 对称性;
	\item 夸克弱作用的(规范)本征态是质量本征态叠加而成(利用 CKM 矩阵);
	\item 存在真空对称性自发破缺的 Higgs 场;
	\item 轻子和夸克与 Higgs 场间有 Yukawa 耦合。
\end{itemize}