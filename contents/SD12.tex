% 苏州大学 2012 年硕士物理考试试题
% keys 苏州大学|2012年|考研|物理
% license Copy
% type Tutor

\textbf{声明}:“该内容来源于网络公开资料,不保证真实性,如有侵权请联系管理员”


\textbf{科目代码:838}

常用物理常数和公式:

真空介电常数$\varepsilon_0=8.85 *10^{-12} \quad C^2/(N.m^2)$

库伦常数 $k=1/(4\pi \varepsilon_0)=9*10^9 \quad N.m^2/c^2$

真空磁导率 $\mu_0=1.26*10^{-6}\quad H/m$

电子电量 $e=+1.6*10^{-19}\quad C$

电子质量 $m_e=9.11*10^{-32}\quad kg=510 \quad keV/c^2$

真空光速 $c=3.00* 10^8 \quad m/s$

普朗克常数$h=6.63x10^{-34}\quad JS$

玻尔效曼常数$k_B =1.38*10^{-23}J/K$

$Hc=1240 eV.nm$

$1 eV=1.6*10^{-19}J$

$F=k Q_1 Q_2/r^2$

$Q=CV$

$C=\varepsilon A/d$

$F=qE$

$V=kQ/r$

$F=ILB\sin \theta$

$F=qvB$

$d\sin \theta =m\lambda $

$P=h/\lambda$

\begin{enumerate}
\item 如下图所示,水平面上两个固定的点电荷,带电量都是$+e$,两者相距$ 2a$,一个$\alpha $粒子(带电量$+2e$)沿者这两个点电荷的中垂线快速穿过,试计算$\alpha$粒子在什么位置时的受力最大?
\begin{figure}[ht]
\centering
\includegraphics[width=6cm]{./figures/07431eed4f8c2041.png}
\caption{} \label{fig_SD12_1}
\end{figure}
\item 如图所示,三个半径相同的均匀导体圆环两两正交,在各交点处彼此连接,每个圆环的电阻为$R$,求 $A$点到$B$点之间的等效电阻$R_{AB}$。
\begin{figure}[ht]
\centering
\includegraphics[width=6cm]{./figures/d227e5891aec7425.png}
\caption{} \label{fig_SD12_2}
\end{figure}
\item 如下图所示,一个内半径为$ a$外半径为$b$的均匀带电绝缘环状薄片,以角速度 $w $绕着过中心O点垂直于环片平面的轴旋转,环片上总带电量为$Q$,求环片中心O点的磁感应强度$B$的大小和方向。
\begin{figure}[ht]
\centering
\includegraphics[width=6cm]{./figures/20fe004321fff1f4.png}
\caption{} \label{fig_SD12_3}
\end{figure}
\item 用长度为$L$的细金属线一端连着一个绝缘球$P$另一端悬挂在O点,构成一个圆锥摆,$P$作水平匀速圆周运动时金属丝与整直线夹角为$\theta$,角速度为$\omega$,如图所示。在空间分布有水平方向的匀强磁场$B$,则金属丝上$P$点与O点之间的最小电势差为多大?最大电势差为多大?
\begin{figure}[ht]
\centering
\includegraphics[width=6cm]{./figures/14c063306619e8de.png}
\caption{} \label{fig_SD12_4}
\end{figure}
\item 一同轴圆柱形电容器,外导体筒的内半径为$a=2cm$,内导体筒的外半径为$x$,两筒之间充满均匀的各向同性电介质,电介质的击穿场强为$1.0*10^7 V/m$,当$x$多大时该电容器所能承受的电压最大?能承受的最大电压为多大?
\begin{figure}[ht]
\centering
\includegraphics[width=6cm]{./figures/99a15d6960354d48.png}
\caption{} \label{fig_SD12_5}
\end{figure}
\item 杨氏双缝实验中,双缝间距为$0.6mm$,接收屏与双缝同距$1m$。用波长成分为5$50nm$和$650nm$的光源做照明光,试求:\\
(1)两种光波分别形成的条纹间距;\\
(2)两组条纹间距与级数之间的关系;这两组条纹有重合的可能吗?
\item 在棱镜($n1=1.52$)表面上镀一层增透膜,使之适用于$441.6nm $波长的激光,膜的厚度应取何值?
\item 波长为$\lambda$的单色平行光沿着与单缝衍射屏成$\alpha$的方向入射至宽度为$a$的单缝上,求:\\
(1)各级衍射极小的衍射角$\theta$值;\\
(2)接收屏上以中央明纹为界明条纹的分布数目如何?
\item 厚度为 $12um$的方解石晶片,其光轴平行于表面,放置在两正交偏振片之间,晶片光轴与第一片偏振片的偏振化方向夹角为45°,若要使波长$650nm$光通过此系统后呈现极大,晶片厚度至少要磨去多少?
\item 一X射线束包含$ 0.095nm$~$0.130nm$ 波段的波长射线。晶体的晶面间距$d=0.3nm$,与题图中所示晶面族相联系的衍射的X射线束能否产生?
\begin{figure}[ht]
\centering
\includegraphics[width=6cm]{./figures/c8fc4eaf9159047f.png}
\caption{} \label{fig_SD12_6}
\end{figure}
\item 已知一束电子以 $4.0x10^6m/s$的速度运动,问:\\
(1)如果电子从静止开始加速,则需要多高的电压才能达到这个速度?\\
(2)该电子的德布罗意波长多大?
\item 能量为$62 keV$的X射线与物质中的电子发生康普顿散射,已知电子静能为$510keV$,则在与入射线成180度角方向上所散射的X射线波长是多少?
\end{enumerate}