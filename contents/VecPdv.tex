% 偏导数(多元矢量值函数)
% 偏导数|求导法则|求导|混合偏导|高阶偏导

\begin{issues}
\issueDraft
\end{issues}

\pentry{矢量的导数\upref{DerV}, 偏导数\upref{ParDer}}
与标量函数的偏导类似, 对一个多元的矢量函数 $\bvec v(x_1, x_2\ldots x_N)$, 如果把其他自变量都看做常数而对 $x_i$ 求导, 那么就得到矢量函数关于 $x_i$ 的\textbf{偏导数}。
\begin{equation}
\pdv{\bvec v}{x_i} = \lim_{\Delta x_i \to 0} \frac{\bvec v(x_1 \ldots x_i+\Delta x_i\ldots x_N) -  \bvec v(x_1 \ldots x_i\ldots x_N)}{\Delta x_i}
\end{equation}

直角坐标系中对多变量矢量函数求偏导就是对矢量的各个分量分别求偏导(见下文“求导法则”)。

\addTODO{高阶混合偏导。同样与顺序无关。}

与标量函数的偏导\upref{ParDer}类似, 多元矢量函数的高阶导数也要声明各阶导数是对哪个变量进行的。
