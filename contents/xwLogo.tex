% 小时百科图标
% keys 小时百科|图标|logo|正态分布
% license Xiao
% type Tutor

小时百科的图标是由 4 条丝带组成 3D 立体图, 从数学上来看, 每条丝带的形状分别是一个\enref{高斯分布(正态分布)}{GausPD}; 物理上, 四条丝带可以分别看作量子力学中\enref{自由粒子的高斯波包}{GausWP}随时间演化; 艺术上, 可以寓意为海浪。

目前图标分为立体版本(\autoref{fig_xwLogo_2} )和扁平化版本(\autoref{fig_xwLogo_1} )。

\begin{figure}[ht]
\centering
\includegraphics[width=8.3cm]{./figures/de015e34d8c8a49f.png}
\caption{立体图标, 完成于 2014 年 10 月} \label{fig_xwLogo_2}
\end{figure}

\begin{figure}[ht]
\centering
\includegraphics[width=8cm]{./figures/3a5293e99ec2e15b.pdf}
\caption{扁平化图标, 完成于 2020 年 9 月} \label{fig_xwLogo_1}
\end{figure}

\subsection{参数}
图标使用\enref{透视投影}{proj3D}, 每个带的长宽高分别在\autoref{fig_xwLogo_3} 中标出。
\begin{figure}[ht]
\centering
\includegraphics[width=12cm]{./figures/156f0d63ddb5d672.png}
\caption{图标参数} \label{fig_xwLogo_3}
\end{figure}

% \addTODO{量子力学参数, 相机参数, 光源材质参数, Matlab 代码}
