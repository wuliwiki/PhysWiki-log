% 光滑映射(流形)
% keys 切空间|微分几何|流形|tangent space|manifold|浸入|浸没|submersion|immersion|微分|differential|光滑映射|smooth map
% license Xiao
% type Tutor

\pentry{流形\nref{nod_Manif}, 光滑映射\nref{nod_SmthM}}{nod_c6c1}

\subsection{流形上的光滑映射}

我们已经熟悉了欧几里得空间中的光滑映射。由于流形是局部同胚于欧几里得空间的,因此我们可以方便地导出流形上光滑映射的概念。

\begin{definition}{光滑映射}
给定两个流形间的映射 $f:M\to N$,如果对于\textbf{任意}点 $p\in M$,存在一个 $M$ 的图 $(U, \varphi)$ 和一个 $N$ 的图 $(V, \phi)$,使得 $p\in U$,$f(U)\subseteq V$,且 $\phi\circ f\circ\varphi^{-1}$ 是一个欧几里得空间之间的光滑映射,那么称 $f$ 是流形间的\textbf{光滑映射(smooth function)}。
\end{definition}

由于光滑流形定义中“相容”的要求,如果 $f$ 是流形间的光滑映射,那么它在任何图上导出的欧几里得空间之间的映射,都是光滑映射。


\subsection{微分}

给定两个流形间的光滑映射,也就给定了流形上的道路之间的映射。由于道路等同于切向量,我们还可以导出一个\textbf{切空间之间}的映射,这个映射就被称为\textbf{微分}。

\begin{definition}{微分}
对于两个流形 $M$ 和 $N$,给定它们之间的一个光滑映射 $f:M\to N$。对于 $p\in M$ 处出发的一条道路 $v:I\to M$,我们可以导出 $f(p)\in N$ 处出发的道路 $f\circ v:I\to N$。

于是我们可以定义映射 $\dd f_p: T_pM\to T_{f(p)}N$,使得对于任意 $v\in T_pM$,有 $\dd f_p(v)=f\circ v$。称 $\dd f$ 为 $f$ 在点 $p$ 处的\textbf{微分(differential)}\footnote{注意英文术语,不是数学分析中的differentiation。}。
\end{definition}

$\dd f_p$ 是切空间之间的线性映射,因此如果给定了 $p\in M$ 和 $f(p)\in N$ 的局部坐标系(图)以后,也就能顺便导出 $T_pM$ 和 $T_{f(p)}N$ 的坐标系,从而可以把 $\dd f_p$ 表示成一个矩阵——它就是 $f$ 的Jacobi矩阵。这也是为什么我们使用“微分”这个术语来称呼它。

整个流形 $M$ 上所有点处的 $\dd f_p$ 可以简单表示为一个 $\dd f:TM\to TN$。这样,$\dd f_p$ 可以视为 $\dd f|_p$,即 $\dd f$ 在 $p$ 点处的限制。

由于可以用矩阵来表示微分,自然就有了\textbf{秩(rank)}的概念,参见线性代数中“矩阵的秩”的定义。

\addTODO{线性代数中“矩阵的秩”的定义}

\begin{definition}{浸入}
给定流形间的光滑映射 $f:M\to N$。如果在任意点 $p\in M$ 处都有 $\opn{rank}\dd f_p=\opn{dim}T_pM$,则称 $f$ 为\textbf{浸入(immersion)}。
\end{definition}

\begin{definition}{浸没}
给定流形间的光滑映射 $f:M\to N$。如果在任意点 $p\in M$ 处都有 $\opn{rank}\dd f_p=\opn{dim}T_pN$,则称 $f$ 为\textbf{浸没(submersion)}。
\end{definition}

浸入映射的另一个等价定义是说,$\dd f$ 处处是\textbf{单射};而浸没映射也可以定义为,$\dd f$ 处处是\textbf{满射}。

这两个术语的含义,通过以下例子可以直观感受到:

\begin{example}{浸入与浸没}
考虑三维欧几里得空间中的两个子流形 $S^1=\{(x, y, z)\in\mathbb{R}^3|x^2+y^2=1, z=0\}$ 和 $S^2=\{(x, y, z)\in\mathbb{R}^3|x^2+y^2+z^2=1\}$。

定义映射 $f:S^1\to S^2$ 为 $f(x, y, 0)=(x, y, 0)$,那么这是一个\textbf{浸入}。

定义映射 $g:S^2\to S^1$ 为 $f(x, y, z)=(x, y, 0)$,那么这是一个\textbf{浸没}。
\end{example}

















