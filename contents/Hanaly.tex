% 氢原子波函数分析
% 氢原子|波函数|动量谱|电离|光电子

\pentry{库仑波函数\upref{CulmWf}}

\subsection{光电子动量谱}

本文使用原子单位制\upref{AU}。 在计算类氢原子光电离时, 当外场消失后, 每个能量本征态(散射态)的概率就固定不变了。 然而动量的本征态系数还是会变(除非时间无穷大)。 要得到时间无穷大时电子的三维动量分布, 我们可以直接将波函数投影到库仑波函数(渐进平面波)上。 事实上这样的动量谱通常被称为 angular resolved energy spectrum (毕竟是能量的本征态), 为了方便我们还是直接叫做动量谱。
\begin{equation}\label{eq_Hanaly_6}
P(\bvec k) = \abs{f(\bvec k)}^2 = \abs{\braket*{\Psi^{(-)}_{\bvec k}}{\Psi(\bvec r)}}^2
\end{equation}
归一化条件为
\begin{equation}\label{eq_Hanaly_4}
\int P(\bvec k) \dd[3]{k} = \iint P(\bvec k) k^2\dd{\Omega}\dd{k} = 1~.
\end{equation}
氢原子的波函数在球坐标系中表示为
\begin{equation}
\Psi(\bvec r, t) = \frac{1}{r}\sum_{l,m} \psi_{l,m}(r, t) Y_{l,m}(\uvec r)~,
\end{equation}
那么\autoref{eq_Hanaly_6} 中内积的具体计算方法见\autoref{eq_CulmWf_3}~\upref{CulmWf}。

被电离的\textbf{光电子(PE 或 photo-electron)} 的能量谱如何计算呢? 先看 $k$ 绝对值的概率分布 $P(k)$, 其归一化为
\begin{equation}\label{eq_Hanaly_2}
\int_0^\infty P(k) \dd{k} = 1~.
\end{equation}
对比\autoref{eq_Hanaly_4} 和\autoref{eq_Hanaly_2} 得
\begin{equation}\label{eq_Hanaly_7}
P(k) = k^2 \int P(\bvec k) \dd{\Omega}~.
\end{equation}
要得到能谱, 使用 “随机变量的变换\upref{RandCV}” 中的方法,
\begin{equation}\label{eq_Hanaly_1}
P(k)\dd{k} = P(E)\dd{E} = P(E)\dd{k^2/2} = P(E)k\dd{k}~,
\end{equation}
所以有
\begin{equation}
P(E) = \frac{1}{k}P(k)~.
\end{equation}

把\autoref{eq_CulmWf_3}~\upref{CulmWf} 代入\autoref{eq_Hanaly_6} 再代入\autoref{eq_Hanaly_7} 得
\begin{equation}
P(k) = \sum_{l,m} \abs{f_{l,m}(k)}^2 = \frac{2}{\pi} \sum_{l,m} \abs{\int F_l(k, r) \psi_{l,m}(r) \dd{r}}^2~,
\end{equation}
其中 $f_{l,m}(k) = \braket{C_{l,m}(k)}{\Psi(\bvec r)}$ 是总波函数在归一化库仑球面波上的投影。 虽然这个公式看起来只包括了径向动能, 但实际上也有角向的动能, 体现在 $l$ 量子数里面\footnote{想一下库仑函数的微分方程中 $l$ 是如何决定角向动能的? 注意与 $m$ 无关。}。

\subsection{额外任意势能的平均能量}
球坐标中的额外势能如果表示为\autoref{eq_HyTDSE_6}~\upref{HyTDSE}
\begin{equation}
V'(\bvec r) = \sum_{l,m} V'_{l,m}(r) Y_{l,m}(\uvec r)~.
\end{equation}
那么对应的能量为
\begin{equation}\label{eq_Hanaly_3}
\begin{aligned}
E' &= \mel{\Psi}{V'(\bvec r)}{\Psi}\\
&= \sum_{l_1,m_1}\sum_{l_2,m_2}\sum_{l,m} \mel{Y_{l_1,m_1}}{Y_{l,m}}{Y_{l_2,m_2}} \int \psi_{l_1,m_1}^*(r) V'_{l,m}(r) \psi_{l_2,m_2}(r) \dd{r}~.
\end{aligned}
\end{equation}
