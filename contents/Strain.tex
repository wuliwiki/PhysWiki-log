% 位移与应变
% license Xiao
% type Tutor

\begin{issues}
\issueTODO 
应变协调方程
\end{issues}

\footnote{本文参考了冯西桥的《弹性力学》课程与陆明万的《弹性理论基础》}
在应力中我们讨论了如何表示材料中各处的受力,现在我们讨论如何表示材料中各处的变形。此处我们只探讨小变形的情况,即材料变形的程度非常轻微。
\subsection{位移}
\begin{figure}[ht]
\centering
\includegraphics[width=10cm]{./figures/edaabaa9e35c7e81.pdf}
\caption{材料变形后,材料中的每一点的位置发生改变} \label{fig_Strain_1}
\end{figure}
\begin{figure}[ht]
\centering
\includegraphics[width=5cm]{./figures/e20948b95a32f560.pdf}
\caption{上图的叠加} \label{fig_Strain_2}
\end{figure}

材料变形后,材料中每一点都移动了。例如,原先位于$\bvec x_0$的点在变形后运动到了$\bvec a_0$位置,原先位于$\bvec x_1$的点运动到了$\bvec a_1$位置...如果我们在弹性力学中暂且假定“变形前后,原先的一个点不会分裂为两个,也不会有两个点合并为一个”,那么我们就可以在点变形前的位置$\bvec x$与变形后的位置$\bvec a$之间建立一种映射$$\bvec x\rightarrow \bvec a~,$$亦即定义一个函数。$$\bvec a = \bvec a (\bvec x)~.$$

\subsubsection{位移函数}
为了更好地表示$\bvec x$处点位置的变化,我们定义\textbf{位移函数}
\begin{equation}
\bvec u (\bvec x) =\bvec a (\bvec x) - \bvec x~,
\end{equation}
由此,点$\bvec x$的位置变化也可以表述为
\begin{equation}
\bvec x + \bvec u (\bvec x) =\bvec a (\bvec x) ~.
\end{equation}

$\bvec u$是一个矢量函数,也被称为位移场。它的3个分量都是关于点坐标的函数,分别表示点在$x_1, x_2, x_3$\footnote{为了方便表示,以$x_1, x_2, x_3$轴代指$x,y,z$轴}各方向上的位移。
$$\bvec u(\bvec x) = 
\begin{pmatrix}
u_1(\bvec x)\\
u_2(\bvec x)\\
u_3(\bvec x)\\
\end{pmatrix}
=
\begin{pmatrix}
u_1(x_1, x_2, x_3)\\
u_2(x_1, x_2, x_3)\\
u_3(x_1, x_2, x_3)\\
\end{pmatrix}~.
$$
由于变形前后点不合并、分裂的性质,$\bvec u(\bvec x)$是一个“好函数”,单值、连续且多阶可导。

\subsection{应变}
有时我们使用\textbf{应变}$\varepsilon$来描述材料的变形。类似于应力,应变也定义在材料中的每一个微元处,可记为一个$3\times3$的矩阵(二阶张量).应变矩阵也是对称矩阵$\varepsilon=\varepsilon^T$,有$6$个独立变量。
\begin{equation}
\varepsilon = 
\begin{pmatrix}
\varepsilon_{11}&\varepsilon_{12}& \varepsilon_{13}\\
\varepsilon_{21}&\varepsilon_{22}& \varepsilon_{23}\\
\varepsilon_{31}&\varepsilon_{32}& \varepsilon_{33}\\
\end{pmatrix}~.
\end{equation}

应变角标的含义与应力的类似,第一个角标表示面的法方向,第二个角标表示变形的方向。我们可将应变分为两类,正应变(两角标相同)与切应变(两角标不同)。

\subsubsection{正应变}
\begin{figure}[ht]
\centering
\includegraphics[width=8cm]{./figures/47df1007f99de062.png}
\caption{正应变示意图。改绘自冯西桥的《弹性力学》} \label{fig_Strain_3}
\end{figure}
正应变与微元的长度变化有关。如\autoref{fig_Strain_3} 所示,$x_1$方向上,微元的原本长度为$l_0 = \dd x_1$,变形后长度为$ l =\dd x_1+\pdv{u_1}{x_1}\dd x_1$,长度变化$\Delta l = l-l_0 = \pdv{u_1}{x_1}\dd x_1$,那么定义正应变
\begin{equation}
\varepsilon_{11} = \frac{\Delta l}{l_0} =\pdv{u_1}{x_1}~.
\end{equation}

\subsubsection{切应变}
\begin{figure}[ht]
\centering
\includegraphics[width=8cm]{./figures/e28f8319c988db73.png}
\caption{切应变示意图。改绘自冯西桥的《弹性力学》} \label{fig_Strain_4}
\end{figure}
而切应变与微元的角度变化有关。如\autoref{fig_Strain_4} 所示,定义切应变
\begin{equation}
\varepsilon_{12} = \frac{1}{2}(90^\circ - \alpha) = \frac{1}{2}\left(\pdv{u_1}{x_2}+\pdv{u_2}{x_1}\right)~,
\end{equation}
有时也使用工程切应变$\gamma$。别紧张,他们只是相差一个系数
\begin{equation}
\gamma_{12} = 2\varepsilon_{12} = 90^\circ - \alpha~.
\end{equation}

\subsection{应变几何方程}
推广上一节的结论,我们共可以得到$6$个联系应变与位移的独立方程。这$6$个方程被称为\textbf{应变几何方程}。
\begin{equation}
\varepsilon_{ij} = \frac{1}{2}\left(\pdv{u_i}{x_j}+\pdv{u_j}{x_i}\right) \qquad (i,j=1,2,3)~.
\end{equation}

\subsection{应变协调方程}

%由于$\bvec u(\bvec x)$的可多阶导的“好函数”性质,可以证明
\subsection{画廊:经典的应变类型}
图中黑色为原微元体,红色为变形后的微元体。
\begin{figure}[ht]
\centering
\includegraphics[width=10cm]{./figures/948a4db32a42d183.pdf}
\caption{$u_1=1.5$。各应变均为$0$,微元形状不变、只发生平移。} \label{fig_Strain_9}
\end{figure}

\begin{figure}[ht]
\centering
\includegraphics[width=10cm]{./figures/0b685aed347c7f2f.pdf}
\caption{$u_1=0.2x_1$。$\varepsilon_{11}=0.2$(其余项为$0$),微元发生拉伸变形} \label{fig_Strain_5}
\end{figure}

\begin{figure}[ht]
\centering
\includegraphics[width=10cm]{./figures/68fae878c07ce1a2.pdf}
\caption{$u_1=0.2x_2$. $\varepsilon_{12}=0.1$,微元发生简单的剪切变形} \label{fig_Strain_6}
\end{figure}

\begin{figure}[ht]
\centering
\includegraphics[width=10cm]{./figures/d4d794eae704a440.pdf}
\caption{$u_1=0.2x_2, u_2=0.2x_1$.$\varepsilon_{12}=0.2$,微元发生剪切变形} \label{fig_Strain_8}
\end{figure}
\begin{figure}[ht]
\centering
\includegraphics[width=10cm]{./figures/ceba7ed8aa873986.pdf}
\caption{同上,但是是俯视图} \label{fig_Strain_7}
\end{figure}

\subsection{附录:应变模拟器}
用以绘制上述应变示意图的octave/matlab代码
\begin{lstlisting}[language=matlab]
%定义位移函数,你可以设置自己的线性位移函数
u1 = @(x,y,z) -0.2*x + 0.2*y;
u2 = @(x,y,z) 0;
u3 = @(x,y,z) 0;

A(1,:)=[0,0,0];
A(2,:)=[1,0,0];
A(3,:)=[1,1,0];
A(4,:)=[0,1,0];
A(5,:)=[0,1,1];
A(6,:)=[0,0,1];
A(7,:)=[1,0,1];
A(8,:)=[1,1,1];

hold on
axis equal

%绘制顶点
%for i = 1:8
%    scatter3(A(i,1),A(i,2),A(i,3),'k');
%endfor

view(-30,60)
xlabel('Axis 1','fontsize',15)
ylabel('Axis 2','fontsize',15)
zlabel('Axis 3','fontsize',15)

line([A(1,1), A(2,1)],[A(1,2), A(2,2)],[A(1,3), A(2,3)],'color','k');
line([A(1,1), A(4,1)],[A(1,2), A(4,2)],[A(1,3), A(4,3)],'color','k');
line([A(1,1), A(6,1)],[A(1,2), A(6,2)],[A(1,3), A(6,3)],'color','k');

line([A(2,1), A(7,1)],[A(2,2), A(7,2)],[A(2,3), A(7,3)],'color','k');
line([A(2,1), A(3,1)],[A(2,2), A(3,2)],[A(2,3), A(3,3)],'color','k');

line([A(3,1), A(4,1)],[A(3,2), A(4,2)],[A(3,3), A(4,3)],'color','k');
line([A(3,1), A(8,1)],[A(3,2), A(8,2)],[A(3,3), A(8,3)],'color','k');

line([A(4,1), A(5,1)],[A(4,2), A(5,2)],[A(4,3), A(5,3)],'color','k');

line([A(6,1), A(5,1)],[A(6,2), A(5,2)],[A(6,3), A(5,3)],'color','k');
line([A(6,1), A(7,1)],[A(6,2), A(7,2)],[A(6,3), A(7,3)],'color','k');

line([A(8,1), A(5,1)],[A(8,2), A(5,2)],[A(8,3), A(5,3)],'color','k');
line([A(8,1), A(7,1)],[A(8,2), A(7,2)],[A(8,3), A(7,3)],'color','k');

for i = 1:8
    B(i,1) = u1(A(i,1),A(i,2),A(i,3))+A(i,1);
    B(i,2) = u2(A(i,1),A(i,2),A(i,3))+A(i,2);
    B(i,3) = u3(A(i,1),A(i,2),A(i,3))+A(i,3);
endfor

%for i = 1:8
%    scatter3(B(i,1),B(i,2),B(i,3),'r');
%endfor

line([B(1,1), B(2,1)],[B(1,2), B(2,2)],[B(1,3), B(2,3)],'color','r');
line([B(1,1), B(4,1)],[B(1,2), B(4,2)],[B(1,3), B(4,3)],'color','r');
line([B(1,1), B(6,1)],[B(1,2), B(6,2)],[B(1,3), B(6,3)],'color','r');

line([B(2,1), B(7,1)],[B(2,2), B(7,2)],[B(2,3), B(7,3)],'color','r');
line([B(2,1), B(3,1)],[B(2,2), B(3,2)],[B(2,3), B(3,3)],'color','r');

line([B(3,1), B(4,1)],[B(3,2), B(4,2)],[B(3,3), B(4,3)],'color','r');
line([B(3,1), B(8,1)],[B(3,2), B(8,2)],[B(3,3), B(8,3)],'color','r');

line([B(4,1), B(5,1)],[B(4,2), B(5,2)],[B(4,3), B(5,3)],'color','r');

line([B(6,1), B(5,1)],[B(6,2), B(5,2)],[B(6,3), B(5,3)],'color','r');
line([B(6,1), B(7,1)],[B(6,2), B(7,2)],[B(6,3), B(7,3)],'color','r');

line([B(8,1), B(5,1)],[B(8,2), B(5,2)],[B(8,3), B(5,3)],'color','r');
line([B(8,1), B(7,1)],[B(8,2), B(7,2)],[B(8,3), B(7,3)],'color','r');

\end{lstlisting}

\begin{figure}[ht]
\centering
\includegraphics[width=8 cm]{./figures/83d30a92addcf27f.png}
\caption{简单的剪切变形} \label{fig_Strain_10}
\end{figure}

另一个以点阵代替正方体的实现的octave/matlab代码:
\begin{lstlisting}[language=matlab]
clc
clear

function s = SHPERE(x,y,z,r)
  [X,Y,Z] = sphere(50);
  X = r*X+x;
  Y = r*Y+y;
  Z = r*Z+z;
  s = surf(X,Y,Z);
end

u1 = @(x,y,z) 0.2*y;
u2 = @(x,y,z) -0.05*y;
u3 = @(x,y,z) 0;

hold on
axis equal

for x = 0:3
  for y = 0:3
    for z = 0:3
      a = u1(x,y,z)+x;
      b = u2(x,y,z)+y;
      c = u3(x,y,z)+z;
      SPHERE(a,b,c,0.5);
    end
  end
end
\end{lstlisting}
