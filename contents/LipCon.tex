% 李普希茨条件
% keys 李普希茨条件|Lipschitz Condition|一致连续
% license Xiao
% type Wiki

\pentry{度量空间\upref{Metric}}
李普希茨条件(Lipschitz)条件描述的对象是度量空间中的映射,它描述那些像点的距离受到原点距离影响的映射。李普希茨条件的最初形式是由德国数学家李普希茨在其1864年关于周期函数的傅里叶级数收敛性的研究中提出的\cite{Li}。本文介绍的是一般度量空间中的李普希茨条件。李普希茨条件在证明常微分方程的存在及唯一定理中起到作用。
\begin{definition}{李普希茨条件}\label{def_LipCon_1}
设 $A$ 是度量空间 $(M_1,d)$ 到 $(M_2,d')$ 的映射,$L$ 是一个正实数。若 $A$ 满足
\begin{equation}\label{eq_LipCon_1}
d(Ax,Ay)\leq Ld'(x,y),\quad\forall x,y\in M_1~,
\end{equation}
则称 $A$ 为满足具有常数 $L$ 的\textbf{李普希茨条件},记作 $A\in \mathrm{Lip} L$。满足\autoref{eq_LipCon_1} 的最小常数称为\textbf{李普希茨常数}。
\end{definition}
\begin{example}{压缩映射}
若\autoref{def_LipCon_1} 中的 $(M_2,d')=(M_1,d)$,且 $0<L<1$,则此时 $A$ 便是 $M_1$ 中的\textbf{压缩映射}\upref{ComMap}。
\end{example}

\begin{theorem}{满足李普希茨条件的映射必一致连续}\label{the_LipCon_1}
若 $A$ 是度量空间\upref{Metric} $(M_1,d)$ 到 $(M_2,d')$ 的满足具有常数 $L$ 的李普希茨条件的映射,则 $A$ 必一致连续。
\end{theorem}
度量空间中的映射的一致连续性定义如下:
\begin{definition}{一致连续}\label{def_LipCon_2}
设 $f$ 是度量空间 $(M_1,d)$ 到 $(M_2,d')$ 的映射,若对每一正数 $\epsilon$,都有一个 $\delta>0$ 存在,使得只要
\begin{equation}
d(x,y)<\delta~,
\end{equation}
就有
\begin{equation}
d'(Ax,Ay)<\epsilon~.
\end{equation}
则称 $A$ 为\textbf{一致连续}的。

\end{definition}
现在来证明\autoref{the_LipCon_1} 。

\textbf{证明:}只要取 $0<\delta<\frac{\epsilon}{L}$(由实数的稠密性,这是可以做到的),那么
\begin{equation}
d(Ax,Ay)\leq Ld'(x,y)< \epsilon~.
\end{equation}
由\autoref{def_LipCon_2} ,$A$ 一致连续。

\textbf{证毕!}

\begin{theorem}{凸且紧子集上的连续可微映射满足李普希茨条件}\label{the_LipCon_2}
设 $V$ 是 $\mathbb R^n$ 空间任意凸的和紧致的子集,则任一连续可微映射(\autoref{def_GofODE_2}~\upref{GofODE}) $f:V\rightarrow \mathbb R^m$ 在 $V$ 上满足具有常数 $L$ 的李普希茨条件,$L$ 等于 $f$ 在 $V$ 上的\textbf{上确界},即($f_*$ 是可微函数 $f$ 的导数\upref{DoDifM})
\begin{equation}
L=\sup_{x\in V}\abs{f_{*x}}~.
\end{equation}
\end{theorem}
\textbf{证明:}要证 $f$ 满足具有常数 $L$ 的李普希茨条件,就是要证
\begin{equation}\label{eq_LipCon_3}
\abs{f(y)-f(x)}\leq L\abs{y-x},\quad\forall x,y\in\mathbb R^n~.
\end{equation}
设 $z(t)=x+t(y-x),t\in[0,1]$,即 $z(t)$ 是连接点 $x$ 和 $y$ 的线段,$V$ 的凸集性意味着这一线段属于 $V$\footnote{于是才可以利用微积分基本定律}。
由微积分基本定理(牛顿—莱布尼兹公式\upref{NLeib}),
\begin{equation}\label{eq_LipCon_2}
f(y)-f(x)=\int_0^1\dv{}{\tau}f(z(\tau))\dd\tau=\int_0^1f_{*z(\tau)}\dot z(\tau)\dd\tau~.
\end{equation}
由于 $f$ 是在 $V$ 上连续可微,于是 $f_*$ 连续。而 $\mathbb R^n$ 中紧致的子集是闭的, 魏尔斯特拉斯第一定理 $\mathbb R^n$ 的闭集上连续函数必有界,第二定理表明 $\mathbb R^n$ 的闭集上连续函数必能达到其上下确界,于是
 \begin{equation}
 \abs{f_*}(x)\leq L ,\forall x\in V ~.
 \end{equation}
所以成立
\begin{equation}
\abs{\int_0^1f_{*z(\tau)}\dot z(\tau)\dd\tau}\leq\int_0^1L\abs{y-x}\dd\tau=L\abs{y-x}~.
\end{equation}
由\autoref{eq_LipCon_2} ,于是\autoref{eq_LipCon_3} 成立。

\textbf{证毕!}