% 势能(简介)
% license Xiao
% type Tutor

%感觉直接从保守力定义的势能太抽象了?我想写一篇文章从另一个角度介绍一下势能。不知道会不会更合适...

\begin{issues}
\issueDraft
\end{issues}

\footnote{本文受安宇的《大学物理》、周磊教授的《电动力学》以及\cite{GriffE}的启发。部分图片来自Pixabay。} 势能可能是物理中运用最为广泛,却又最令人费解的概念之一(\textsl{哪个物理概念不令人费解呢?})。本文提供了一个大致的思路,主要面向具有一定物理基础的读者。

以下我们假定粒子间的相互作用力都是保守力(例如静电场力,重力等都是保守力。。。)由于我们没有特别基于某一个保守力定义势能,因此这些论述可以很容易地推广至多种情形(相应的,也不会涉及具体的数学计算);且假定系统及其影响范围均只局限于有限区域内。

\subsection{“单一粒子的势能”}
怎么样理解“一个粒子的势能”?我们可以认为粒子的势能是“将该粒子加入系统所需要的能量”。
\begin{figure}[ht]
\centering
\includegraphics[width=8cm]{./figures/718084b3881383f1.png}
\caption{一个装满糖果的罐子} \label{fig_POTENT_4}
\end{figure}
想象一下,我们要将一颗糖果装进一个已经装着不少糖果的罐子中的特定位置。我们当然不能指望这颗新糖果能直接出现那里;我们顶多先将他放到瓶口,然后\textsl{使尽洪荒之力}将他塞进瓶中.

\begin{figure}[ht]
\centering
\includegraphics[width=10cm]{./figures/b8267e2ee1ca394f.pdf}
\caption{将粒子从无穷远处移动至特定位置时,需要克服已有粒子的排斥} \label{fig_POTENT_1}
\end{figure}

同样的,我们往已有系统中加入一个新的粒子时,也要先将他放在无穷远处,然后再移动到特定位置;\textbf{而将粒子从无穷远处移动至特定位置时,需要克服已有粒子的排斥(或者说粒子间的作用力、内力),这就需要额外的外力对粒子做功以克服这些阻碍。}

根据我们对机械能定理(能量守恒)的理解,这部分外力的功不能平白无故的消失了,他一定被转换为了某种能量,此即为粒子的势能。
$$\Delta E_p = W_{ext}=\int^a_\infty \bvec F_{ext} \dd \bvec l~. $$

我们设粒子处于无穷远处时,它的势能为零,那么
$$E_p = \Delta E_p + 0 = \int^a_\infty \bvec F_{ext} \dd \bvec l~.$$

举一个势能的简单例子:相对于地面,为什么一个在山顶的人会具有重力势能?因为他要克服重力才能从地面来到山顶,这就需要额外的能量输入,比如他身体里的化学能,或者驱动缆车的电能...相反地,如果这个人的随身物件不慎从山顶滑落(\textsl{请看管好随身物品!不要伤及花花草草!}),那么物件的重力势能将重新转换为别的形式的能量,例如动能、热能以及物件被破坏时伴随的塑性变形能。

\subsubsection{势能与内力}
从上文我们看出,\textbf{势能源自于粒子与已有粒子间的交互}(见上文粗体字部分),因此由系统内部的物理量表述势能应该更为合适;而上文中我们却使用了抽象的外力做功。怎么用更实在的内力功来表现势能呢?

根据动能定理,粒子动能的变化与内、外力做功和有关;由于始末态粒子动能都为零,因此
$$W_{in} + W_{ext} = \Delta E_k = 0 \Rightarrow W_{in} = -W_{ext}~.$$
又因为
$$W_{in} = \int^a_\infty \bvec F_{in} \dd \bvec l~, $$
所以
$$E_p = - \int^a_\infty \bvec F_{in} \dd \bvec l = \int^\infty_a \bvec F_{in} \dd \bvec l~.$$
%或者说
%$$\Delta E_p = - W_{in}$$
%(注意这与$\Delta E_p = W_{ext}$的区别。)

\subsubsection{势能的性质}
我们一直没有明确粒子从无穷远运动至特定位置的具体路径。事实上,由于内力的保守性质(这是我们在开头假定的),\textbf{保守内力做功的多少与具体路径无关、而只与始末状态有关},因此具体的路径无关紧要。%(如果做功与路径有关,那么我们就不能定义势能了,因此大家说重力势能、电势能、却很少人说“摩擦力势”。)

换句话说,势能是一种状态量,它与新加入粒子的性质与位置、已有粒子的性质与分布等有关,但与历史无关 (加入粒子的具体路径等)。
$$E_p=E_p(\bvec r,...)~.$$

尽管此节的标题是“单一粒子的势能”,但从上文的探讨中,我们已经发现势能实际上源自于粒子与已有粒子的交互。如果只有单一的粒子,势能的概念也就不能存在了。因此,严格地说我们不能说“单一粒子的势能”。

\subsection{系统的势能}
那么我们怎么理解系统的势能?运用上一节的经验,系统的势能可以理解为“组建系统所需要的能量”。

\begin{figure}[ht]
\centering
\includegraphics[width=8cm]{./figures/46de144ab504c911.pdf}
\caption{组建系统时,将粒子逐个移动至恰当的位置} \label{fig_POTENT_2}
\end{figure}
假定组成系统的粒子最初都位于无穷远处、且互相相隔无穷远。现在,为了组建系统,我们要将这些粒子一个一个地移动到恰当的位置。

\begin{itemize}
\item 当我们放置第一个粒子时,由于其他的粒子现在还在无穷远处,此时没有阻力,也不需要外力做功;
\item 而当放置第二个粒子时,则需要克服来自第一个粒子的阻力,\textsl{如同我们在上一节所看到的那样,这就需要外力做功,而这部分能量就使系统的势能上升。。。};
\item 继续放置第三个粒子,此时就需要克服来自第一、第二个粒子的阻力...
\item ...
\end{itemize}
以此类推,直到我们放置完最后一个粒子后,外力做的总功即转换为系统的势能。
同样的,系统的势能也是一种状态量,它取决于系统中粒子的性质与分布,但与系统的历史(系统曾经历的过程)无关。
$$E_p=E_p(S,...)~.$$

比如说,我们可以利用这种思路来计算电荷系统的电势能,详见 "\enref{电势、电势能}{QEng} "的 \autoref{sub_QEng_1}~\upref{QEng} 。

\subsection{场的抽象}
\begin{figure}[ht]
\centering
\includegraphics[width=10cm]{./figures/abc07d14ce7cc2a3.pdf}
\caption{粒子感受到的是场的作用力} \label{fig_POTENT_3}
\end{figure}
现在,我们认为粒子间的交互是通过场实现的(电场、重力场...)。因此,系统的势能实质上是系统(组成系统的所有粒子)所产生的场的能量。简单地改变上文的用词,我们就能理解这一点:

\textsl{现在,我们往已有系统中加入一个新的粒子。这个新粒子最初位于无穷远处,将粒子从无穷远处移动至特定位置时,需要克服已有场的排斥,这就需要额外的外力对粒子做功以克服阻碍。 根据我们对机械能定理(能量守恒)的理解,这部分外力的功不能平白无故的消失了,他一定被转换为了场的能量...}

\begin{figure}[ht]
\centering
\includegraphics[width=8cm]{./figures/43e564ced29dd66e.png}
\caption{来自周磊教授的奇妙比喻:粒子是冲浪的人,而场是奔腾的大海。不仅要看到冲浪者的运动(粒子的机械动能),也要看到大海的涌动(场的能量)。虽然从技术上说,这种图景并非完全正确...} \label{fig_POTENT_5}
\end{figure}
