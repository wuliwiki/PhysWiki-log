% 单纯形与复形
% 单纯形|simplex|复形|complex|几何|单纯剖分|三角剖分

\pentry{连通性\upref{Topo3}}

\subsection{单纯形}

单纯形可以被认为是欧几里得空间中的一种子集,最初的概念来源于对拓扑空间的三角剖分,而单纯形就是剖分出来的每个“三角形”.这里之所以要打引号,是因为单纯形不仅仅指二维的三角形,也包括三维的锥形,以及从零维到任意维的拓展.

\begin{definition}{几何无关点集}
给定欧几里得空间中 $n$ 个点构成的集合 $\{\bvec{r}_i\}|_{i=0}^{n-1}$.任取一个 $\bvec{r}_k$,我们以它为起点构造出若干向量 $\bvec{v}_i=\bvec{r}_i-\bvec{r}_k$.

如果集合 $\{\bvec{v}_i\}_{i\not= k}$ 是一个\textbf{线性无关}向量组,那么我们说 $\{\bvec{r}_i\}|_{i=0}^{n-1}$ 是一个\textbf{几何无关(geometrically independent)}点集.
\end{definition}

注意几何无关点集和线性无关向量组有一点点区别,那就是几何无关点集里多了一个“起点”的位置.这个起点没法简单地排除在外,因为任意一个点都可以做起点,没必要确定谁更特殊.

\begin{definition}{单纯形}
给定欧几里得空间中一个几何无关点集 $A=\{\bvec{r}_i\}$.

记 $[\bvec{r}_0, \bvec{r}_1, \bvec{r}_2, \cdots, \bvec{r}_q]$ 为集合 $\{\bvec{r}=\sum\lambda_i\bvec{r}_i|\lambda_i\geq 1, \sum\lambda_i=1\}$,称这个集合为由 $A$ 张成的 $q$ 维\textbf{单纯形(simplex)},简称 $q$-单形.

\end{definition}

比如说,三维空间里 $3$ 个几何无关点 $\bvec{r}_0, \bvec{r}_1, \bvec{r}_2$ 可以张成一个 $2$-单形,就是以这三个点为顶点的平面三角形.类似地,一个 $3$-单形就是一个三棱锥.一个 $0$-单形就是一个单点集,即 $[\bvec{r}]=\{\bvec{r}\}=\bvec{r}$.


\begin{definition}{标准单纯形}
由欧几里得空间中的原点和各坐标轴上距离原点为 $1$ 的点构成的单纯形,称为\textbf{标准单纯形}.
\end{definition}



\begin{definition}{面}

令 $L$ 为单形 $[\bvec{r}_0, \cdots, \bvec{r}_q]$,则对于 $r\leq q$,$L$ 的一个 $r$ 维面就是 $\{\bvec{r}_0, \cdots, \bvec{r}_q\}$ 的一个 $r$ 阶子集所张成的单形.

\end{definition}


\begin{definition}{规则相处}
对于两个单形 $A$ 和 $B$,如果 $A\cap B$ 既是 $A$ 的面,也是 $B$ 的面,那么称 $A$ 和 $B$ 是\textbf{规则相处}的.
\end{definition}


从上面的表述可以看出,尽管我们一开始引入单纯形概念的时候依赖于欧几里得空间的性质,是一种高度几何化的语言,但是表示时我们其实只关心是哪些顶点在构成一个单纯形.这就使得我们可以将单纯形的概念抽象化,将 $[\bvec{r}_0, \cdots, \bvec{r}_q]$ 视作这 $q+1$ 个元素的一个组合,忽视掉几何特征,从而可能代数地描述这些结构.从这个角度来说,单形的面就是其子集,而脱去了几何概念之后我们完全可以认为任何单形都是规则相处的——这也是几何语言里要强调规则相处的意义.

\subsection{复合形}

\begin{definition}{复合形}
一个\textbf{复合形(complex)},简称\textbf{复形},是一个单纯形的集合,其中\textbf{各单形规则相处},且每个单形的面也都是该复形的一个元素.

一个复形中维度最大的单形的维度,称为该复形的维度.
\end{definition}

\begin{figure}[ht]
\centering
\includegraphics[width=4cm]{./figures/SimCom_1.pdf}
\caption{复形的例子.这个复形是由一个三角形、四条线段和四个点构成的.} \label{SimCom_fig1}
\end{figure}

\begin{figure}[ht]
\centering
\includegraphics[width=5cm]{./figures/SimCom_2.pdf}
\caption{复形的例子.这个复形是由一个三角形、四条线段和五个点构成的} \label{SimCom_fig2}
\end{figure}


\begin{example}{复形的例子}
\begin{enumerate}
\item 三维欧几里得空间中,集合 $\{(x, 0)|x\in[0,1]\}, \{(0, y)|y\in[0, 1]\}, \{(0, 0)\}, \{(1, 0)\}, \{(0, 1)\}$ 所构成的集合,是一个复形.这个复形中一共有五个元素,分别是两条线段和三个点.
\item 如图\autoref{SimCom_fig1} 所示,四个点 $a, b, c, d$ 几何无关,并且构成了复形 $\{[a], [b], [c], [d], [a, b], [a, c], [b, c], [a, d], [a, b, c]\}$.
\item 如图\autoref{SimCom_fig2} 所示,五个点构成了复形 $\{[a], [b], [c], [d], [e], [a, b], [a, c], [b, c], [d, e], [a, b, c]\}$.和上一个例子相比,这个复形是有两个连通分支的.
\item \textbf{闭包复形} 一个单形 $L$ 的全体面的集合,构成一个复形,称为 $L$ 的\textbf{闭包复形},记为 $\opn{Cl} L$.
\item \textbf{边缘复形} 一个单形 $L$ 的全体真面\footnote{即作为真子集的面.}的集合,构成一个复形,称为 $L$ 的\textbf{边缘复形},记为 $\opn{Bd} L$.
\end{enumerate}
\end{example}

\begin{definition}{子复形}
一个复形 $K$ 的子集 $J$ 如果还是一个复形,那么称 $J$ 是 $K$ 的\textbf{子复形}.
\end{definition}

\begin{definition}{骨架}
$q$ 维复形 $K$ 的全体维度\textbf{小于等于}$r$ 的单形之集合,称为 $K$ 的 $r$ 维骨架,记为 $K^r$.
\end{definition}










