% 数论函数
% keys 数论|函数
% license Usr
% type Tutor

\pentry{映射\upref{map}}

\begin{issues}
\issueDraft
\issueMissDepend
\end{issues}

\begin{definition}{数论函数}
以自然数集或正整数集为定义域,以一数集\footnote{指复数域的一个子集。}为值域的函数称为数论函数。
\end{definition}

下面列出常见的数论函数(均定义在自然数集上,其中的 $\lambda$ 均是实数):
\begin{example}{简单数论函数}
\begin{itemize}
\item 单位函数 $I(n)$
\begin{equation}
I(n) =
\begin{cases}
1\quad (n = 1)\\
0\quad (n > 1)~.
\end{cases}
\end{equation}
\item 单值函数 $u(n)$
\begin{equation}
u(n)\equiv1~.
\end{equation}
\item 恒等函数 $e(n)$
\begin{equation}
e(n)=n~.
\end{equation}
\item 幂函数 $n^\lambda$
\item 对数函数\footnote{在数论研究中,对数函数的底默认为自然常数 $e$。} $\log n$
\end{itemize}
\end{example}
\begin{example}{与因数有关的数论函数}
\begin{itemize}
\item 除数函数 $d(n)$
\begin{equation}
d(n)=\sum_{d|n} 1=
\begin{cases}
1&(n=1)\\
\prod\limits_{i=1}^{s}(\alpha_i+1)&(n=p_1^{\alpha_1}p_2^{\alpha_2}\cdots p_s^{\alpha_s})~.
\end{cases}
\end{equation}
\item 除数和函数 $\sigma(n)$
\begin{equation}
\sigma(n)=\sum_{d|n}d=
\begin{cases}
1&(n=1)\\
\prod\limits_{i=1}^{s}\dfrac{p_i^{\alpha_i+1}-1}{p_i-1}&(n=p_1^{\alpha_1}p_2^{\alpha_2}\cdots p_s^{\alpha_s})~.
\end{cases}
\end{equation}
\item 除数幂和函数 $\sigma_\lambda(n)$
\begin{equation}
\sigma_\lambda(n)=\sum_{d|n}d^\lambda~.
\end{equation}
\item 不同素因子个数 $\omega(n)$
\begin{equation}
\omega(n)=\sum_{p|n}1=
\begin{cases}
0\quad (n=1)\\
s\quad(n=p_1^{\alpha_1}p_2^{\alpha_2}\cdots p_s^{\alpha_s})~.
\end{cases}
\end{equation}
\item 全部素因子个数(按重数计)$\Omega(n)$
\begin{equation}
\Omega(n)=\sum_{p^r\|n}r=
\begin{cases}
0&(n=1)\\
\sum\limits_{i=1}^{s}\alpha_i&(n=p_1^{\alpha_1}p_2^{\alpha_2}\cdots p_s^{\alpha_s})~.\\
\end{cases}
\end{equation}
\end{itemize}
\end{example}
\begin{example}{著名的数论函数}
\begin{itemize}
\item 素数计数函数 $\pi(n)$
\begin{equation}
\pi(n)=\sum_{p\leq n} 1~.
\end{equation}
\item $\mathrm{M\ddot{o}bius}$ 函数 $\mu(n)$
\begin{equation}
\mu(n)=
\begin{cases}
1&\ (n=1)\\
0&\ (n=l^2k,\ l>1,\ l,k\in\mathbb{N})\\
(-1)^s&\ (n=p_1p_2p_3\cdots p_s)~.
\end{cases}
\end{equation}
\item $\mathrm{Euler}$ 函数 $\varphi(n)$
\begin{equation}
\varphi(n)=\sum_{1\leq d\leq n,(d,n)=1}1~.
\end{equation}
\item $\mathrm{van\ Mangoldt}$ 函数 $\Lambda(n)$
\begin{equation}
\varLambda(n)=
\begin{cases}
\log p&\ (n=p^s)\\
0&\ (\omega(n)\neq 1)~.
\end{cases}
\end{equation}
\item $\mathrm{Liouville}$ 函数 $\lambda(n)$
\begin{equation}
\lambda(n)=(-1)^{\Omega(n)}~.
\end{equation}
\end{itemize}

\end{example}
