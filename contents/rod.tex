% 轻杆模型

\pentry{速度\upref{VnA}}

\bb{轻杆}在这里的意思是质量可以忽略不计, 所以也没有转动惯量. 假设我们只能在轻杆的两端对其施加两个力, 这两个力会满足什么条件呢? 首先, 轻杆受到的合力必须为 0, 否则他就会马上被加速到无限快. 这意味着这两个力大小相等, 方向相反. 其次, 它受到的和力矩必须也为 0, 否则就会瞬间拥有无限大的角速度. 这意味着两个力必须共线, 即都延杆的方向.

现在, 无论轻杆如何运动, 这两个力对轻杆做功总是为 0.

\subsection{不可伸长条件}
我们来思考 “不可伸长” 这一条件会对杆两端的速度和加速度带来什么约束. 令杆的两端点的位置矢量分别为 $\vec r_1$ 和 $\vec r_2$, 则杆的长度平方为 $(\vec r_2 - \vec r_1)^2$. 所以长度不随时间变化的条件可以表示为
\begin{equation}\label{rod_eq2}
\dv{t} (\vec r_2 - \vec r_1)^2 = \vec 0
\end{equation}
由矢量求导法则
\begin{equation}\label{rod_eq1}
(\vec r_2 - \vec r_1) \vdot (\vec v_2 - \vec v_1) = \vec 0
\end{equation}
其中 $\vec v$ 是端点的速度. 如果我们将 $\vec r_1$ 指向 $\vec r_2$ 的单位矢量记为 $\uvec r$, 则上式变为
\begin{equation}
\vec v_2 \vdot \uvec r = \vec v_1 \vdot \uvec r
\end{equation}
也就是说, 两个端点的速度延杆的分量需要在任意时刻都相等.

若我们想知道对加速度的约束, 只需对\autoref{rod_eq1} 两边再求一次时间导数, 即\autoref{rod_eq2} 的二阶时间导数.
\begin{equation}
\dv[2]{t} (\vec r_2 - \vec r_1)^2 = (\vec v_2 - \vec v_1)^2 + (\vec r_2 - \vec r_1) \vdot (\vec a_2 - \vec a_1) = \vec 0
\end{equation}

为了理解该式, 让我们把两个速度分解成平行杆和垂直杆得两个分量, 令垂直杆的单位矢量分别为 $\uvec x$ 和 $\uvec y$, 则
\begin{equation}
\vec v_i = v_{i, x} \uvec x + v_{i, y} \uvec y + v_{i, r} \uvec r \quad (i = 1, 2)
\end{equation}

(过程未完成)

结论:
\begin{equation}
\vec a_2 \vdot \uvec r - \vec a_1 \vdot \uvec r  = \omega^2 r
\end{equation}
$\omega$ 是杆的瞬时角速度的大小.
