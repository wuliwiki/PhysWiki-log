% Julia 数学函数速览
% keys 数学函数速览
% license Xiao
% type Tutor

本文授权转载自郝林的 《Julia 编程基础》。 原文链接:\href{https://github.com/hyper0x/JuliaBasics/blob/master/book/ch05.md}{第 5 章 数值与运算}。


\subsection{5.7 数学函数速览}

Julia 预定义了非常丰富的数学函数。一些常用的函数如下:

\begin{itemize}
\item \textbf{数值类型转换:} 主要有\verb|T(x)|和\verb|convert(T, x)|。其中,\verb|T|代表目的类型,\verb|x|代表源值。
\item \textbf{数值特殊性判断:} 有\verb|isequal|、\verb|isfinite|、\verb|isinf|和\verb|isnan|。
\item \textbf{舍入:} 有四舍五入的\verb|round(T, x)|、向正无穷舍入的\verb|ceil(T, x)|、向负无穷舍入的\verb|floor(T, x)|,以及总是向\verb|0|舍入的\verb|trunc(T, x)|。
\item \textbf{除法:} 有\verb|cld(x, y)|、\verb|fld(x, y)|和\verb|div(x, y)|,它们分别会将商向正无穷、负无穷和\verb|0|做舍入。其中的\verb|x|代表被除数,\verb|y|代表除数。另外,与之相关的还有取余函数\verb|rem(x, y)|和取模函数\verb|mod(x, y)|,等等。
\item \textbf{公约数与公倍数:} 函数\verb|gcd(x, y...)|用于求取最大正公约数,而函数\verb|lcm(x, y...)|则用于求取最小正公倍数。圆括号中的\verb|...|的意思是,除了\verb|x|和\verb|y|,函数还允许传入更多的数值。但要注意,这里的数值都应该是整数。
\item \textbf{符号获取:} 函数\verb|sign(x)|和\verb|signbit(x)|都用于获取一个数值的符号。但不同的是,前者对于正整数、\verb|0|和负整数会分别返回\verb|1|、\verb|0|和\verb|-1|,而后者会分别返回\verb|false|、\verb|false|和\verb|true|。
\item \textbf{绝对值获取:} 用于获取绝对值的函数是\verb|abs(x)|。一个相关的函数是,用于求平方的\verb|abs2(x)|。
\item \textbf{求根:} 函数\verb|sqrt(x)|用于求取\verb|x|的平方根,而函数\verb|cbrt(x)|则用于求取\verb|x|的立方根。
\item \textbf{求指数:} 函数\verb|exp(x)|会求取\verb|x|的自然指数。另外还有\verb|expm1(x)|,为接近\verb|0|的\verb|x|计算\verb|exp(x)-1|。
\item \textbf{求对数:} \verb|log(x)|会求取\verb|x|的自然对数,\verb|log(b, x)|会求以\verb|b|为底的\verb|x|的对数,而\verb|log2(x)|和\verb|log10(x)|则会分别以\verb|2|和\verb|10|为底求对数。另外还有\verb|log1p(x)|,为接近\verb|0|的\verb|x|计算\verb|log(1+x)|。
\end{itemize}

除了以上函数之外,Julia 的\verb|Base|包中还定义了很多三角函数和双曲函数,比如\verb|sin|、\verb|cos|、\verb|atanh|、\verb|acoth|等等。另外,在 \href{https://github.com/JuliaMath/SpecialFunctions.jl}{SpecialFunctions.jl 包}里还有许多特殊的数学函数。不过这个包就需要我们手动下载了。
