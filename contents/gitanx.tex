% git-annex 笔记

\addTODO{本文暂时搁置了, 最新版本只能在 tree 中用 symlink, 在不支持 symlink 的文件系统如 NTFS 和 exFAT 中, \verb|init| 会出错。 要不还是看看 \verb|bup| 吧!}
\begin{itemize}
\item \href{https://git-annex.branchable.com/}{主页}里面的 archivist 太符合我的需求了。
\item \href{https://stackoverflow.com/questions/39337586/how-do-git-lfs-and-git-annex-differ}{和 git-lfs 的对比} 但是 LFS 的缺点是必须要有服务器, 官方并不支持本地 repo!
\item 官方\href{https://git-annex.branchable.com/walkthrough/}{入门教程}。
\item 标准流程: \verb|cd 我的文件夹; git init; git annex init; 修改文件; git annex add .|, \verb|git commit -a -m added|
\item 注意 commit 了以后, 所有 tree 中的文件都会变成 symlink, 真正的文件文件名变为 hash, 后缀名不变, 被存在 \verb|.git/annex/objects/xx/xx/SHA256E-...| 中。
\item 在其他目录添加一个备份: \verb|cd 其他目录; git clone 我的文件夹; cd 其他目录/我的文件夹; git annex init "备份1", git remote add remote名 我的文件夹; cd 我的文件夹; git remote add remote名1 备份1|
\item \verb|git mv|, \verb|git status| 等命令都可以照常使用。
\item 
\end{itemize}
