% 无限深阶梯势阱
% 阶梯势能|薛定谔方程|散射

\begin{issues}
\issueDraft
\end{issues}

\pentry{定态薛定谔方程\upref{SchEq}}

使用原子单位
\begin{equation}
V(x) =
\begin{cases}
+\infty  & (x \leqslant -l)\\
0  & (-l \leqslant x < 0)\\
V_0  & (0 \leqslant x \leqslant l)\\
+\infty  & (l \leqslant x)
\end{cases}
\end{equation}

阶梯两侧得波数满足
\begin{equation}
k^2 = 2V_0 + k_1^2
\end{equation}

令波函数的一个解为
\begin{equation}
\psi_1(x) =
\begin{cases}
A_1 \exp(\I k x)  & (-l \leqslant x < 0)\\
C_1 \exp(\I k_1 x) + D_1 \exp(-\I k_1 x) & (0 \leqslant x \leqslant l)
\end{cases}
\end{equation}
在原点匹配函数值和导数, 得
\begin{equation}
\begin{cases}
A_1 = C_1 + D_1\\
kA_1 = k_1 C_1 - k_1 D_1
\end{cases}
\end{equation}
解得
\begin{equation}
\leftgroup{
    C_1 = \frac{k_1 + k}{2k_1} A_1\\
    D_1 = \frac{k_1 - k}{2k_1} A_1
}
\end{equation}

令波函数的一个解为
\begin{equation}
\psi_2(x) =
\begin{cases}
B_2 \exp(-\I k x)  & (-l \leqslant x < 0)\\
C_2 \exp(\I k x) + D_2 \exp(-\I k x) & (0 \leqslant x \leqslant l)
\end{cases}
\end{equation}
在原点匹配函数值和导数, 得
\begin{equation}
\begin{cases}
A_1 = C_1 + D_1\\
-kA_1 = k_1 C_1 - k_1 D_1
\end{cases}
\end{equation}
