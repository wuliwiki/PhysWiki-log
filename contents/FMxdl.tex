% 费马小定理(综述)
% license CCBYNCSA3
% type Wiki

本文根据 CC-BY-SA 协议转载翻译自维基百科\href{https://en.wikipedia.org/wiki/Fermat\%27s_little_theorem}{相关文章}。

在数论中,费马小定理指出:如果 $p$ 是一个质数,那么对于任意整数 $a$,数 $a^p - a$ 都是 $p$ 的整数倍。用模运算的记号表示,就是:
$$
a^p \equiv a \pmod{p}~
$$
例如,当 $a = 2$、$p = 7$ 时,有 $2^7 = 128$,而 $128 - 2 = 126 = 7 \times 18$,正好是 7 的倍数。

如果 $a$ 不能被 $p$ 整除,也就是说 $a$ 与 $p$ 互素,那么费马小定理等价于以下陈述:$a^{p-1} - 1$ 是 $p$ 的整数倍,或者用符号表示为:
$$
a^{p-1} \equiv 1 \pmod{p}~
$$
例如,当 $a = 2$、$p = 7$ 时,有 $2^6 = 64$,而 $64 - 1 = 63 = 7 \times 9$,也是 7 的倍数。

费马小定理是费马素性检验的理论基础,也是初等数论中的一个基本定理。该定理得名于皮埃尔·德·费马,他在 1640 年提出了这一结论。之所以称为“小定理”,是为了将其与费马大定理区分开来。
\subsection{历史}
\begin{figure}[ht]
\centering
\includegraphics[width=6cm]{./figures/00da106992b247ea.png}
\caption{皮埃尔·德·费马} \label{fig_FMxdl_1}
\end{figure}
皮埃尔·德·费马最早是在一封写于 1640 年 10 月 18 日的信中向他的朋友兼知己弗雷尼克尔·德·贝西陈述了这一定理。他的表述等价于如下内容:

如果 $p$ 是质数,而 $a$ 是任意一个不能被 $p$ 整除的整数,那么 $a^{p-1} - 1$ 一定能被 $p$ 整除。

费马的原始陈述是:

Tout nombre premier mesure infailliblement une des puissances −1 de quelque progression que ce soit, et l'exposant de la dite puissance est sous-multiple du nombre premier donné −1; et, après qu'on a trouvé la première puissance qui satisfait à la question, toutes celles dont les exposants sont multiples de l'exposant de la première satisfont tout de même à la question.

这段话翻译成中文,并加入一些注释和公式以便理解,可以表述为:

每一个质数 $[p]$ 必定整除某一个“某个等比级数 $[a, a^2, a^3, \dots]$”中的“某个幂减一”的结果(即存在某个 $t$,使得 $a^t - 1$ 可被 $p$ 整除),且这个幂的指数 $t$ 是 $p - 1$ 的一个约数。找到第一个满足条件的幂之后,所有指数是这个指数 $t$ 的倍数的幂,同样也满足这一性质。

也就是说,如果某个 $t$ 满足 $a^t \equiv 1 \pmod{p}$,那么所有 $t$ 的倍数也都满足 $a^{kt} \equiv 1 \pmod{p}$。

费马并没有讨论当 $a$ 是 $p$ 的倍数时的情形,也并没有给出他的证明,只是写道:

Et cette proposition est généralement vraie en toutes progressions et en tous nombres premiers; de quoi je vous envoierois la démonstration, si je n'appréhendois d'être trop long.

翻译为中文是:

“这个命题在所有的数列和所有的质数中都是普遍成立的;我本可以把它的证明寄给你,只是担心写得太长了。”\(^\text{[5]}\)

欧拉在 1736 年发表的论文中首次给出了费马小定理的正式证明。这篇论文题为 《Theorematum Quorundam ad Numeros Primos Spectantium Demonstratio》(英文译名为:“关于素数的一些定理的证明”),发表于圣彼得堡科学院的期刊中\(^\text{[6][7]}\)。但实际上,莱布尼茨早在 1683 年之前的一份未发表手稿中,就几乎给出了相同的证明\(^\text{[3]}\)。

“费马小定理”这一术语最早可能出现在1913 年库尔特·亨泽尔所著的《Zahlentheorie》(数论)一书中\(^\text{[8]}\):

Für jede endliche Gruppe besteht nun ein Fundamentalsatz, welcher der kleine Fermatsche Satz genannt zu werden pflegt, weil ein ganz spezieller Teil desselben zuerst von Fermat bewiesen worden ist.

翻译为中文:

“在每一个有限群中都存在一个基本定理,这通常被称为‘费马小定理’,因为其中的一个非常特殊的部分最早是由费马证明的。”

在英语中较早使用该术语的例子出现在 A.A. 阿尔伯特于 1937 年出版的《现代高等代数》一书中,第 206 页提到:

“所谓的‘小’费马定理”\(^\text{[9]}\)。
\subsubsection{后续发展历史}
一些数学家曾独立提出了一个相关猜想(有时被误称为“中国猜想”):当且仅当 $p$ 是质数时,有 $2^p \equiv 2 \pmod{p}$。实际上,“如果”部分是正确的,它是费马小定理的一个特例;然而,“只有当”这部分是错误的。

例如:
$$
2^{341} \equiv 2 \pmod{341}~
$$
但 $341 = 11 \times 31$ 是合数,它是 以 2 为底的伪素数。详见下文。
\subsection{证明}
费马小定理有多种不同的证明方法。其中一种常见的做法是将它作为欧拉定理的推论来证明。
\subsection{推广}
欧拉定理是费马小定理的推广形式:对于任意模数 $n$ 和任意与 $n$ 互素的整数 $a$,都有
$$
a^{\varphi(n)} \equiv 1 \pmod{n}~
$$
其中 $\varphi(n)$ 表示欧拉函数,即从 1 到 $n$ 中与 $n$ 互素的整数的个数。费马小定理实际上就是欧拉定理的一个特例,因为如果 $n$ 是质数,则 $\varphi(n) = n - 1$。

欧拉定理的一个推论是:对于任意正整数 $n$,如果整数 $a$ 与 $n$ 互素,那么对于任意整数 $x$ 和 $y$,有
$$
x \equiv y \pmod{\varphi(n)} \quad \Rightarrow \quad a^x \equiv a^y \pmod{n}.~
$$
这可以由欧拉定理直接推出:如果$x \equiv y \pmod{\varphi(n)}$,那么存在整数 $k$ 使 $x = y + k\varphi(n)$,于是
$$
a^x = a^{\,y + \varphi(n)k} = a^y \cdot \left(a^{\varphi(n)}\right)^k \equiv a^y \cdot 1^k \equiv a^y \pmod{n}.~
$$
如果 $n$ 是质数,这也是费马小定理的一个推论。这一性质在模运算中应用广泛,因为它允许将指数很大的模幂运算化简为指数小于 $n$ 的情形,从而大大简化计算。

欧拉定理在公钥密码学中(尤其是 RSA 密码系统)被广泛应用于 $n$ 不是质数的情形,通常方式如下\(^\text{[10]}\):如果$y = x^e \pmod{n}$,

当已知 $\varphi(n)$ 时,从 $y, e, n$ 中恢复 $x$ 是容易的【11】。实际上,**扩展欧几里得算法**可以用来计算 $e$ 关于 $\varphi(n)$ 的模逆元,即找到一个整数 $f$,使得
$$
ef \equiv 1 \pmod{\varphi(n)}.~
$$
于是有:
$$
x \equiv x^{ef} \equiv (x^e)^f \equiv y^f \pmod{n}.~
$$
另一方面,如果 $n = pq$ 是两个不同质数的乘积,那么$\varphi(n) = (p - 1)(q - 1)$,在这种情况下,从 $n$ 和 $e$ 计算 $f$ 的难度与计算 $\varphi(n)$ 的难度相同(这一点尚未被严格证明,但目前没有已知算法可以在不知道 $\varphi(n)$ 的情况下直接计算 $f$)。已知 $n$ 的情况下,计算 $\varphi(n)$ 的难度本质上与分解 $n$ 相同,因为$\varphi(n) = (p - 1)(q - 1)$,反过来,质因数 $p$ 和 $q$ 又是方程$x^2 - (n - \varphi(n) + 1)x + n = 0$
的整数解。

因此,RSA 密码系统的基本思想是:如果将消息 $x$ 按照$y = x^e \pmod{n}$,的方式加密(其中 $n$ 和 $e$ 是公开的),那么在现有的知识条件下,如果不能找到 $n$ 的(秘密)质因数 $p$ 和 $q$,就无法解密。

费马小定理还与卡迈克尔函数及卡迈克尔定理有关,并且与群论中的拉格朗日定理存在联系。
\subsection{逆命题}
费马小定理的逆命题在卡迈克尔数的情形下是不成立的。然而,其一个稍弱的变体是莱默定理:

如果存在一个整数 $a$,使得
$$
a^{p-1} \equiv 1 \pmod{p}~
$$
并且对于所有整除 $p - 1$ 的质数 $q$,都有
$$
a^{(p-1)/q} \not\equiv 1 \pmod{p},~
$$
那么 $p$ 是质数。

该定理构成了卢卡斯素性检验(Lucas primality test,一种重要的素数判定方法)以及普拉特素数证明的理论基础。
\subsection{伪素数}
如果 $a$ 与 $p$ 互素,且 $a^{p-1} - 1$ 能被 $p$ 整除,那么 $p$ 不一定是质数。如果 $p$ 不是质数,那么 $p$ 被称为**以 $a$ 为底的(费马)伪素数(Fermat pseudoprime to base $a$)。

第一个以 2 为底的伪素数是在 1820 年由皮埃尔·弗雷德里克·萨吕斯发现的:$341 = 11 \times 31$【12】【13】。  

如果一个数 \( p \) 对于所有与 \( p \) 互素的整数 \( a \) 都是以 \( a \) 为底的费马伪素数,则称其为卡迈克尔数。  

另外,如果一个数 \( p \) 满足以下等式:  
$$
\gcd\left(p, \sum_{a=1}^{p-1} a^{p-1}\right) = 1,~
$$

那么 $p$ 要么是质数,要么是卡迈克尔数。
\subsection{米勒–拉宾素性检验}

Miller–Rabin 素性检验**使用了费马小定理的如下推广形式【14】:

如果 $p$ 是一个奇质数,且$p - 1 = 2^s d$其中 $s > 0$ 且 $d$ 为奇数 $>0$,那么对于所有与 $p$ 互素的整数 $a$,要么$a^d \equiv 1 \pmod{p}$要么存在一个整数 $r$ 使得 $0 \le r < s$ 且$a^{2^r d} \equiv -1 \pmod{p}$.

这个结果可以由费马小定理推出,原因在于:如果 $p$ 是奇质数,那么模 $p$ 的整数构成一个有限域,其中 1(模 $p$)恰好有两个平方根:1 和 −1(模 $p$)。

注意,对于 $a \equiv 1 \pmod{p}$,显然有 $a^d \equiv 1 \pmod{p}$,因为同余关系在乘方运算下是兼容的。同理,对于 $a \equiv -1 \pmod{p}$,由于 $d$ 是奇数,显然有$a^d = a^{2^0 d} \equiv -1 \pmod{p}$

因此,通常选择随机的 $a$ 在区间 $1 < a < p - 1$ 内进行测试。Miller–Rabin 检验**的使用方法如下:给定一个需要检测素性的奇整数 $p$,写成 $p - 1 = 2^s d$(其中 $s > 0$,$d$ 为奇数 $> 0$),然后随机选择一个 $1 < a < p - 1$ 的整数,计算$b = a^d \pmod{p}$如果 $b$ 既不等于 1 也不等于 −1,则反复对 $b$ 进行模 $p$ 平方运算,直到得到 −1 或者已经平方了 $s - 1$ 次。如果 $b \neq 1$,且在平方过程中从未得到 −1,则 $p$ 是合数,且 $a$ 是 $p$ 为合数的见证。否则,$p$ 是一个以 $a$ 为底的强伪素数(strong probable prime to base $a$),即它可能是质数也可能不是。如果 $p$ 是合数,则该检验仍然错误地将其判定为强伪素数的概率最多为 $\frac14$。在这种情况下,$p$ 称为强伪素数,而 $a$ 称为强骗子。因此,经过 $k$ 次非决定性的随机测试后,$p$ 仍为合数的概率最多为 $4^{-k}$,通过增加 $k$ 可以使这一概率任意接近于零。

总结:该检验要么证明一个数是合数,要么以任意可控的小错误概率断言它是质数。该方法实现非常简单,并且计算效率高于所有已知的确定性检验。因此,它通常会在正式的素数证明之前使用。
\subsection{另见}
\begin{itemize}
\item 费马商
\item 弗罗贝尼乌斯自同态
\item $p$-导数
\item 分母为质数的分数:与费马小定理相关性质的数
\item RSA 加密算法
\item 同余表
\item 模乘逆元
\end{itemize}
\subsection{注释}
\begin{enumerate}
\item Long 1972,第 87–88 页。
\item Pettofrezzo & Byrkit 1970,第 110–111 页。
\item Burton 2011,第 514 页。
\item Fermat, Pierre(1894),Tannery, P.;Henry, C.(编),《费马全集·第二卷:书信集》(Oeuvres de Fermat. Tome 2: Correspondance),巴黎:Gauthier-Villars,第 206–212 页(法文)。
\item Mahoney 1994,第 295 页(英文译文见此)。
\item Euler, Leonhard(1736),“Theorematum quorundam ad numeros primos spectantium demonstratio” ,《圣彼得堡科学院院刊》,第 8 卷,第 141–146 页。
\item Ore 1988,第 273 页。
\item Hensel, Kurt(1913),《数论》,德国柏林与莱比锡:G. J. Göschen,第 103 页。
\item Albert 2015,第 206 页。
\item Trappe, Wade;Washington, Lawrence C.(2002),《编码理论与密码学导论》,Prentice-Hall,第 78 页,ISBN 978-0-13-061814-6。
\item 如果 $y$ 与 $n$ 不互素,则欧拉定理不成立,但这种情况极少见,因此通常不予考虑。事实上,如果偶然出现这种情况,就可以轻易分解 $n$,从而破解对应的 RSA 实例。
\item Sloane, N. J. A.(编),“数列 A128311($2^n - 1 - 1$ 除以 $n$ 的余数)”,在线整数数列百科全书,OEIS 基金会。
\item Sarrus, Frédéric(1819–1820),“Démonstration de la fausseté du théorème énoncé á la page 320 du IXe volume de ce recueil” [对本集第九卷第 320 页所述定理之错误性的证明],《纯与应用数学年刊》,第 10 卷,第 184–187 页。
\item Rempe-Gillen, Lasse;Waldecker, Rebecca(2013-12-11),“4.5.1 引理(模质数的单位根)”,《初学者的素数判定》,美国数学学会,ISBN 9780821898833。
\end{enumerate}
\subsection{参考文献}
\begin{itemize}
\item Albert, A. Adrian(2015)[1938],《现代高等代数》,剑桥大学出版社,ISBN 978-1-107-54462-8。
\item Burton, David M.(2011),《数学史/导论》(The History of Mathematics / An Introduction,第 7 版),麦格劳-希尔出版社,ISBN 978-0-07-338315-6。
\item Long, Calvin T.(1972),《初等数论导论》(Elementary Introduction to Number Theory,第 2 版),列克星敦:D. C. Heath and Company,LCCN 77171950。
\item Mahoney, Michael Sean(1994),《皮埃尔·德·费马的数学生涯,1601–1665》,第 2 版),普林斯顿大学出版社,Bibcode:1994mcpd.book…..M,ISBN 978-0-691-03666-3。
\item Ore, Oystein(1988)[1948],《数论及其历史》,多佛出版社,ISBN 978-0-486-65620-5。
\item Pettofrezzo, Anthony J.;Byrkit, Donald R.(1970),《数论基础》,恩格尔伍德克利夫斯:Prentice Hall,LCCN 71081766。
\end{itemize}
\subsection{延伸阅读}
\begin{itemize}
\item Paulo Ribenboim(1995),《素数记录新书》(The New Book of Prime Number Records,第 3 版),纽约:施普林格出版社,ISBN 0-387-94457-5,第 22–25、49 页。
\end{itemize}
\subsection{外部链接}
\begin{itemize}
\item 维基共享资源上的费马小定理相关媒体
\item 《雅诺什·鲍耶与伪素数》(János Bolyai and the pseudoprimes,匈牙利语)
费马小定理——cut-the-knot 网站
\item 欧拉函数与欧拉定理——cut-the-knot 网站
\item 费马小定理与 Sophie 的证明
\item “Fermat's little theorem”,数学百科全书,EMS Press,2001 [1994]
\item Weisstein, Eric W. “Fermat's Little Theorem”,MathWorld。
\item Weisstein, Eric W. “Fermat's Little Theorem Converse”,MathWorld。
\end{itemize}
