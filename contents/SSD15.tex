% 首都师范大学 2015 年物理硕士考试试题
% keys 首都师范大学|2015年|考研|物理
% license Copy
% type Tutor



\textbf{声明}:“该内容来源于网络公开资料,不保证真实性,如有侵权请联系管理员” 


\begin{enumerate}
\item 一人站在山脚下$O$点向山坡上扔石子,石子初速为$v_0$,与水平夹角为$\theta$(斜向上),山坡与水平面成$a$ 角。\\
(1)如不计空气阻力,求石子在山坡上的落地点到$O$点的距离$s$;\\
(2)如果$a$值与$v_0$值一定,$\theta$取何值时$s$最大,并求出最大值$S_{max}$。
\item 质量为$m$的子弹以速度$v_0$水平射入沙土中,设子弹所受阻力与速度成正比,比例系数为$k$,忽略子弹的重力,求:\\
(1)子弹射入沙土后,速度随时间变化的函数关系式;\\
(2)子弹射入沙土的最大深度。
\item 一质量均匀分布的柔软细绳铅直地悬挂着,绳的下端刚好触到水平桌面,如果把绳的上端放开,绳将落在桌面上,试求在绳下落的过程中,任意时刻作用于桌面的压力。
\item 如图1所示,质量为$M$的均匀细棒,长为$L$,可绕过端点$O$的水平光滑轴在竖直面内转动,当棒竖直静止下垂时,有一质量为$m$的小球飞来,垂直击中棒的中点,由于碰撞,小球碰后以初速度为零自由下落,而细棒碰撞后的最大偏角为$\theta$,求小球击中细棒前的速度值
\begin{figure}[ht]
\centering
\includegraphics[width=8cm]{./figures/6f622f0610ec9517.png}
\caption{} \label{fig_SSD15_1}
\end{figure}
\item 半径为$b$的薄金属球壳,带电量$Q$。求:\\
(1)它的电容;\\
(2)距球心$r$处的电场能量密度;\\
(3)电场总能量;\\
(4)计算当电量无穷小的电荷从无穷远处移到金属球壳时所需的功;\\
(5)若球外再同心放置一个半径为$a$的金属球壳,且两金属球壳之间有电势差$V$,问内球面的半径要多大才能使得这一表面附近的电场最小?
\item 如图,一半径为$3a$的无限长带电圆柱,内有半径为$a$孔柱,孔轴与圆柱轴平行,两轴相距为$a$。圆柱中通有电流$I$,该电流沿截面均匀分布,并从纸面流出。求:\\
(1)过两轴平面$P$上各点的磁场强度;\\
(2)孔内各点的磁场强度。
\begin{figure}[ht]
\centering
\includegraphics[width=8cm]{./figures/ed8ea8ce01fdd13c.png}
\caption{} \label{fig_SSD15_2}
\end{figure}
\item 有一半径为$r$的金属圆环,电阻为$R$,置于磁感应强度为$B$的匀强磁场中。初始时刻环面与$B$垂直,后将圆环以匀角速度$\omega$绕通过环心并处于环面内的轴线旋转$\pi/2$。求:\\
(1)在旋转过程中环内通过的电量;\\
(2)环中的电流;\\
(3)外力所作的功。
\end{enumerate}