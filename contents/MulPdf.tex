% 多变量分布函数
% 概率分布|统计|分布函数|概率

% 未完成: 名字改为多维概率分布函数

\pentry{概率分布函数\upref{RandF}, 重积分\upref{IntN}}

\subsection{二维分布函数}
我们以打靶为例来引入二元概率分布函数, 以及在直角坐标系和极坐标系中如何分析各种平均值。 我们在靶上建立 $x$-$y$ 直角坐标系, 类比一维概率分布函数的定义, 二维概率分布可以用一个二元函数表示为 $f(x, y)$, 也叫\textbf{概率密度(probability density)}。 子弹落在一个长方形区域内(令 $x \in [x_1, x_2]$, $y \in [y_1, y_2]$) 的概率可以用二重积分表示
\begin{equation}
P = \int_{y_1}^{y_2} \int_{x_1}^{x_2} f(x, y) \dd{x} \dd{y}~.
\end{equation}
从几何上理解, $f(x, y)$ 可以看作一张三维空间中的曲面, 而这个二重积分则表示曲面和 $x$-$y$ 平面之间的一个体积(\autoref{fig_MulPdf_1})。

\begin{figure}[ht]
\centering
\includegraphics[width=7cm]{./figures/5bea347882213c85.png}
\caption{$f(x, y)$ 曲面下的体积} \label{fig_MulPdf_1}
\end{figure}

类比一维情况, 概率归一化条件为
\begin{equation}
\int_{-\infty}^{+\infty} \int_{-\infty}^{+\infty} f(x, y) \dd{x} \dd{y} = 1~.
\end{equation}
由于概率的量纲是 1, $f(x, y)$ 的国际单位量纲就是 $\Si{m^{-2}}$, 即面积的倒数。

\subsection{平均值}
类比\autoref{eq_RandF_3}~\upref{RandF}, 离散情况下, 令 $P_i$ 是 $(x_i, y_i)$ 出现的概率, 函数 $g(x,y)$ 的平均值为
\begin{equation}
\ev{g} = \sum_{i=1}^N g(x_i, y_i) P_i~.
\end{equation}
拓展到连续的情况, 类比\autoref{eq_RandF_2}~\upref{RandF}, $g(x, y)$ 的平均值可以定义为
\begin{equation}
\ev{g} =  \int_{-\infty}^{+\infty} \int_{-\infty}^{+\infty} g(x, y) f(x, y) \dd{x} \dd{y}~.
\end{equation}
例如位置矢量的 $x$ 和 $y$ 分量平均值分别为
\begin{equation}\label{eq_MulPdf_2}
\ev{x} = \iint x f(x, y) \dd{x}\dd{y}~,
\end{equation}
\begin{equation}
\ev{y} = \iint y f(x, y) \dd{x}\dd{y}~.
\end{equation}
以上两式也可以用位置矢量 $\bvec r$ 表示为(这时 $\bvec g$ 的函数值是矢量)
\begin{equation}\label{eq_MulPdf_6}
\ev{\bvec r} = \iint \bvec r f(x, y) \dd{x}\dd{y}~,
\end{equation}
点 $(x,y)$ 到原点距离的距离 $r$ 的平均值
\begin{equation}
\ev{r} = \iint \sqrt{x^2 + y^2} f(x, y) \dd{x}\dd{y}~.
\end{equation}
注意区分该式和\autoref{eq_MulPdf_6} 式的意义, 平均位置和平均距离是两个不同的概念: 如果 $f(x, y)$ 是关于原点中心对称的函数, 则平均位置为 $(0, 0)$, 而平均距离不为零。

距离平方 $r^2$ 的平均值
\begin{equation}\label{eq_MulPdf_7}
\ev{r^2} = \iint (x^2 + y^2) f(x, y) \dd{x}\dd{y}~.
\end{equation}

\subsection{概率分布的极坐标表示}
\pentry{极坐标系\upref{Polar}}

除了直角坐标系外, 我们也可以使用极坐标\upref{Polar} 表示分布函数, 记为\footnote{在数学上, 该式两边是两个不同的二元函数, 应该使用不同的函数名如 $g(r, \theta) = f(x, y)$, 但在一些物理教材中, 我们有时为了方便不区分函数名, 直接把 $g(r, \theta)$ 写成 $f(r, \theta)$。} $f(r, \theta)$, 那么概率归一化条件变为
\begin{equation}\label{eq_MulPdf_1}
\int_0^\infty \int_{0}^{2\pi} f(r, \theta)\,\, r\dd{\theta}\dd{r} = 1~,
\end{equation}
函数 $g(r, \theta)$ 的平均值变为
\begin{equation}
\ev{g} = \int_0^\infty \int_{0}^{2\pi} g(r, \theta) f(r, \theta)\,\, r\dd{\theta}\dd{r}~.
\end{equation}
例如\autoref{eq_MulPdf_7} 变为
\begin{equation}\label{eq_MulPdf_3}
\ev{r^2} = \int_0^\infty \int_{0}^{2\pi} r^2 f(r, \theta)\,\, r\dd{\theta}\dd{r}
= \int_0^\infty \int_{0}^{2\pi} r^3 f(r, \theta) \dd{\theta}\dd{r}~.
\end{equation}

% 习题未完成

\subsection{圆对称分布}
继续打靶的例子, 我们接下来讨论圆对称的概率分布(在三维情况下叫做球对称), 即 $f(r, \theta)$ 的值只和 $r$ 有关而与 $\theta$ 无关的情况。 为了书写方便可记 $f(r) \equiv f(r, \theta)$。 归一化条件\autoref{eq_MulPdf_1} 化简为(先完成关于 $\theta$ 积分得 $2\pi$)
\begin{equation}
2\pi \int_0^\infty r f(r) \dd{r} = 1~.
\end{equation}

一个容易混淆的概念是, $f(r)$ 并不是变量 $r$ 的概率分布函数。 也就是说 $\int_a^b f(r) \dd{r}$ 并不是子弹落在圆环 $r \in [a, b]$ 内的概率。 若将 $r$ 的概率分布函数记为 $F(r)$, 子弹落在圆环中的概率应该是
\begin{equation}\label{eq_MulPdf_4}
P_{ab} = \int_a^b F(r) \dd{r} = \int_a^b 2\pi r f(r) \dd{r}~.
\end{equation}
由于两个定积分在任何区间都相等, 两个被积函数必须相等, 即
\begin{equation}\label{eq_MulPdf_5}
F(r) = 2\pi r f(r)~.
\end{equation}
这时以上的平均值也可以得到进一步化简, 例如\autoref{eq_MulPdf_3} 变为
\begin{equation}
\ev{r^2} = \int_0^\infty r^2 F(r) \dd{r}
= 2\pi \int_0^\infty r^3 f(r) \dd{r}~.
\end{equation}
