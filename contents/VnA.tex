% 速度、加速度
% 位移|速度|加速度|积分|矢量|求导

\pentry{速度、加速度(一维)\upref{VnA1}, 牛顿第二定律的矢量形式\upref{New2}, 矢量的导数\upref{DerV}, 矢量积分} % 未完成

在大学物理中,“位移”,“速度”和“加速度”都是矢量,既有大小也有方向。如果没有特殊说明,它们一般是指“瞬时速度”和“瞬时加速度”

\begin{figure}[ht]
\centering
\includegraphics[width=8cm]{./figures/VnA_1.pdf}
\caption{位矢、速度与加速度示意图} \label{VnA_fig1}
\end{figure}

\subsection{速度}

质点在运动时,其位矢\upref{Disp} $\bvec r$ 是时间 $t$ 的函数。假定质点在 $t_1$ 时刻的位矢为 $\bvec r(t_1)$,经过时间 $\Delta t$后, 位矢为 $\bvec r(t_1 + \Delta t)$, 所以物体在 $\Delta t$ 时间内的位移\upref{Disp}为
\begin{equation}
\Delta \bvec r = \bvec r (t_1 + \Delta t) - \bvec r (t_1)
\end{equation}
那么即可定义 $t_1$ 时刻质点的速度
\footnote{假设$\bvec r$可导}
\begin{equation}
\bvec v(t_1) = \lim_{\Delta t \to 0} \frac{\Delta \bvec r}{\Delta t} = \lim_{\Delta t \to 0} \frac{\bvec r (t_1 + \Delta t) - \bvec r(t_1)}{\Delta t}
\end{equation}
即速度是位置矢量关于时间的导数\upref{DerV}, 同样是时间的矢量函数。

\begin{theorem}{}
速度方向总是垂直于位矢方向。
\begin{equation}
\bvec r \cdot \bvec v = 0
\end{equation}
\end{theorem}

简单的例子: 匀速圆周运动的速度(求导法)\upref{CMVD}。

\subsection{加速度}

通常情况下,质点运动轨迹上的每一点都会对应一个确定的速度矢量\footnote{注意上面的速度在定义时虽然取了两点,但是取极限以后,速度和位置是一一对应的,也就和时间一一对应,而不是两个位置和时间对应一个速度。}, 类比速度的定义, 加速度的定义为
\begin{equation}\label{VnA_eq4}
\bvec a(t_1) = \lim_{\Delta t \to 0} \frac{\Delta \bvec v}{\Delta t}
= \lim_{\Delta t \to 0} \frac{\bvec v(t_1 + \Delta t) - \bvec v (t_1)}{\Delta t} = \dv{\bvec v}{t}
\end{equation}
结合速度的定义,加速度为
\begin{equation}
\bvec a = \dv{\bvec v}{t} = \dv{t} \qty( \dv{\bvec r}{t} ) = \dv[2]{\bvec r}{t}
\end{equation}
所以,加速度是速度对时间的导数,或者位矢对时间的二阶导数。

加速度可以有垂直于速度的分量,与平行于速度的分量,详见曲线运动的加速度\upref{PCuvMo}。

简单的例子: 匀速圆周运动的速加速度(求导法\upref{Der})。

\subsection{由速度或加速度计算位矢}

如果已知速度关于时间的函数 $\bvec v(t)$, 以及初始时间 $t_0$ 和位置 $\bvec r_0$, 该如何得到位移—时间函数 $\bvec r(t)$ 呢? 类比一维的情况\upref{VnA1}, 我们也可以通过矢量函数的定积分\upref{IntV}(见\autoref{IntV_ex1}) 来求出速度—时间函数进而求出位移—时间函数
\begin{equation}\label{VnA_eq7}
\bvec v(t) = \bvec v_0 + \int_{t_0}^{t} \bvec a(t) \dd{t}
\end{equation}
\begin{equation}\label{VnA_eq8}\ali{
\bvec r(t) &= \bvec r_0 + \int_{t_0}^{t} \bvec v(t) \dd{t}
}\end{equation}

简单的例子: 匀加速运动\upref{ConstA}。
