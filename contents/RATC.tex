% 转动与平动的类比
% license Xiao
% type Tutor

\begin{issues}
\issueDraft
\end{issues}

\subsection{定轴转动}

以下的表格类比了刚体的平动与转动。形式上两者高度相似。
\begin{figure}[ht]
\centering
\includegraphics[width=8cm]{./figures/cf47259b588cc9c6.pdf}
\caption{刚体的平动} \label{fig_RATC_1}
\end{figure}

\begin{figure}[ht]
\centering
\includegraphics[width=6cm]{./figures/3e4c985fb4fd6e30.pdf}
\caption{刚体的定轴转动} \label{fig_RATC_2}
\end{figure}

\begin{table}[ht]
\centering
\caption{运动学量}\label{tab_RATC_1}
\begin{tabular}{|c|c|}
\hline
平动&转动\\
\hline
位置 $r$ & 角度 \upref{RigRot}$\theta$ \\
\hline
速度 $v=\dv{r}{t}$ & 角速度 \upref{RigRot} $\omega = \dv{\theta}{t}$ \\
\hline
加速度 $a = \dv{r}{t}$ & 角加速度 \upref{RigRot} $\alpha = \dv{\omega}{t}$ \\
\hline
\end{tabular}
\end{table}

\begin{table}[ht]
\centering
\caption{动力学量}\label{tab_RATC_2}
\begin{tabular}{|c|c|}
\hline
平动&转动\\
\hline
力 $F$ & 力矩 \upref{Torque}$\tau=\abs{r}\abs{F} \sin \theta$\\
\hline 
功 $W = F \dd r$ & (力矩的)功 \upref{RBKE} $W=\tau \dd \theta$\\
\hline
质量 $m$ & 转动惯量 \upref{RigRot} $I = \int r_\perp^2 \dd m$ \\
\hline
动量 $p=mv$ & 角动量 \upref{AMLaw} $L=I\omega$ \\
\hline
动能 $E_k = \frac{1}{2}mv^2$ & (转动)动能 $E_k = \frac{1}{2} I\omega^2$ \\
\hline
\end{tabular}
\end{table}

\begin{table}[ht]
\centering
\caption{动力学定理}\label{tab_RATC_3}
\begin{tabular}{|c|c|}
\hline
平动&转动\\
\hline 
牛二 $F=ma$& $\tau = I \alpha$ \upref{RigRot}\\
\hline
动量定理 $F=\dv{p}{t}$ & 角动量定理 $\tau=\dv{L}{t}$ \upref{RigRot} \\
\hline
动能定理 $W = \Delta E_k$ & 动能定理 $W = \Delta E_k$ \\
\hline
\end{tabular}
\end{table}

\subsection{定点转动}
\pentry{刚体的瞬时转轴、角速度的矢量相加\upref{InsAx}}
\addTODO{……}
