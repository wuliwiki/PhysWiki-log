% 线性相关 线性无关
% 矢量空间|线性相关|线性无关|线性代数
% 未完成: 已经有相同标题的词条了! 可以把这个放到第二级

\pentry{矢量空间\upref{LSpace}}

如果存在至少一组不全为零系数 $c_i$ 使几个矢量的线性组合等于零, 这些矢量就被称为\textbf{线性相关}的.
\begin{equation}
\sum_i^N c_i \bvec v_i = \bvec 0
\end{equation}
注意对于实数(复数)矢量空间, 式中所有的系数都是实数(复数), 下同.

对于任何一个系数不为零的项 $j$, 矢量 $\bvec v_j$ 都可以表示为其他矢量的线性组合. 只需把上式除以 $c_j$ 即可
\begin{equation}
\bvec v_j = \sum_{i \ne j}\frac{c_i}{c_j} \bvec v_i
\end{equation}
如果不存在这样的系数, 这些矢量就是\textbf{线性无关}的.

\begin{example}{几何矢量的线性无关}\label{LinInd_ex2}
我们来看在三维几何矢量空间中, 线性无关有什么几何意义. 若两个矢量 $\bvec v_1$ 和 $\bvec v_2$ 线性相关, 意味着存在全不为零实数 $c_1, c_2$ 使
\begin{equation}
c_1 \bvec v_1 + c_2 \bvec v_2 = \bvec 0
\end{equation}
所以 $\bvec v_1 = c_2 \bvec v_2 / c_1$. 这个推导使可逆的, 所以两个几何矢量线性相关当且仅当它们共线, 或者说两个几何矢量线性无关当且仅当它们不共线.

再来看三个矢量的情况. 类比两个矢量的情况, 则线性相关意味着
\begin{equation}
\bvec v_3 = \frac{c_1}{c_3} \bvec v_1 +  \frac{c_2}{c_3} \bvec v_2
\end{equation}
由几何矢量加法的(几何)定义, 要么这三个矢量都共线, 要么 $\bvec v_3$ 落在 $\bvec v_1$ 和 $\bvec v_2$ 所在的平面上. 该过程的逆过程也成立, 所以三个几何矢量线性相关当且仅当它们都共线或者共面, 或者说三个几何矢量线性无关当且仅当它们不共面且两两不共线.
\end{example}
