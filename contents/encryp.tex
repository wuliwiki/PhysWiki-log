% 如何给文件加密(含 python 加密脚本)
% license MIT
% type Tutor

\pentry{Python 基础\upref{Py}}

以下给出一些常用的 python 加密函数,可以加密单个文件,加密字符串,加密一个文件夹中所有文件/文件名。 注意该脚本只能在 linux 环境使用(因为需要调用命令行的 \verb`openssl`)。在 Windows 中可以通过 WSL\upref{WSLnt} 中使用该脚本。 事实上也可以用 \verb|pyOpenSSL| 模块来增加可移植性。

加密算法为 \verb`aes-256-cbc`, \verb`-nosalt` 意味着同样的数据和密码生成的加密文件一摸一样(以支持网盘的 “秒传” 功能)。若去掉该选项可能会让加密变得更安全一些。

完整程序下载见 \href{https://github.com/MacroUniverse/MyPythonLibrary/tree/master/encrypt}{GitHub} 或\href{https://pan.baidu.com/s/1y4Asx-oS4ShGlN9kIyRZeg?pwd=3q1d#list/path=\%252F}{网盘分享}

\subsection{使用说明}
\begin{itemize}
\item 目前无需依赖任何第三方 python 库, 但需要在 Linux 命令行安装 \verb`openssl`, 例如 \verb`sudo apt install openssl`。
\item 压缩包解压后, 在 \verb`=== what to run ===` 下面输入想要运行的函数, 例如想加密一个字符串,用 \verb`print(encrypt_str('some string', password))`。 注意不要替换 \verb`password`, 脚本运行后会提示手动输入密码
\item 在命令行中 \verb`cd` 到脚本所在路径, 运行 \verb`./encrypt.py` 即可。
\item 加密后的字符串是 base64 的, 要解密字符串,用 \verb`decrypt_str('m1slBeA8WhZiP8YBgMtw/9zxV2G', password)`
\addTODO{需要另外支持加密和解密 base16384 的字符串}
\end{itemize}

\subsubsection{用户函数(不含内部函数)}
\begin{itemize}
\item 加密文件: \verb`encrypt(要加密的文件, 加密后的文件, password)`
\item 解密文件: \verb`decrypt(要加密的文件, 加密后的文件, password)`
\item 加密字符串:\verb`加密的字符串 = encrypt_str(要加密的字符串, password)`
\item 解密字符串:\verb`解密的字符串 = decrypt_str(要解密的字符串, password)`
\item 加密文件夹:\verb`encrypt_files_in_folder(要加密的文件夹, 输出前缀, password)`。 加密后会生成新的文件夹 \verb`输出前缀-要加密的文件夹`, 原文件夹不会改变。
\item 解密文件夹:\verb`decrypt_files_in_folder(要解密的文件夹, 输出前缀, password)`。 解密后会生成新的文件夹 \verb`输出前缀-要解密的文件夹`, 原文件夹不会改变。
\item 加密文件夹中的文件(夹)名:\verb`encrypt_names_in_folder(要加密的文件夹, password)`。 \verb`要加密的文件夹` 本身的名字不会被加密, 而包含的所有文件(夹)名会被加密后重命名,注意文件内容不会改变。 加密后的文件(夹)名以 \verb`file_extension` 字符串结尾, 默认是 \verb`.eNc`。 若加密后的名字超出长度, 则会创建文件 \verb`要加密的文件夹/enc-long-name-dic.txt` 作为对照表(解密时该表会被自动读取)。为了尽可能使加密名称的长度小于系统限制, 加密后的文件名采用 \verb`base16384_utf8_chinese_sorted.txt` 中的 16384 个汉字编码。
\item 解密文件夹中的文件(夹)名:\verb`decrypt_names_in_folder(要解密的文件夹, password)`。
\item \verb`encrypt_folder(要加密的文件夹, 输出前缀, password)` 相当于先调用 \verb`encrypt_files_in_folder(要加密的文件夹, 输出前缀, password)` 再调用 \verb`encrypt_names_in_folder(输出前缀 + 要加密的文件夹, password)`
\addTODO{需要检查绝对路径名的长度是否超出 256 个字符, 如果超出直接把超出长度的目录或者最后一个目录放入长名对照表}
\item \verb`decrypt_folder(要解密的文件夹, 输出前缀, password)` 与上一条相反。
\end{itemize}

\subsubsection{内部参数和函数}
\begin{itemize}
\item 可以通过修改 \verb`file_extension` 指定加密名称的拓展名。
\item 可以通过修改 \verb`dic_file` 指定用于指定加密名称超出长度的对照表文件名。
\item \verb`base16384_str` 字符串从文件 \verb`base16384_utf8_chinese_sorted.txt` 中读取 16384 个字符。
\item \verb`base64字符串 = utf8_to_base64(字符串)` 用于把字符串转换为 base64 编码。
\item \verb`自定义编码字符串 = base64_to_custom_base(base64字符串, 编码字符串)` 使用 \verb`编码字符串`(如 \verb`base16384_str`)对 \verb`base64字符串` 进行编码。
\item \verb`base64字符串 = custom_base_to_base64(自定义编码字符串, 编码字符串)` 使用 \verb`编码字符串`(如 \verb`base16384_str`)对 \verb`自定义编码字符串` 进行解码。
\end{itemize}
