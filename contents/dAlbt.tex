% 拉格朗日方程的证明、达朗贝尔原理
% keys 拉格朗日方程证明|广义力|非约束力|达朗贝尔原理
% license Xiao
% type Tutor

\pentry{广义力\nref{nod_LagEqQ}}{nod_57a9}

本文使用牛顿第二定律证明拉格朗日方程, 其中会使用到一个关于约束的定理叫做\textbf{达朗贝尔原理(D'Alembert's principle)}。 另一种推导方法见 “\enref{最小作用量、哈密顿原理}{HamPrn}”。

\subsection{由牛顿第二定律证明拉格朗日方程}
下面来证明包含额外广义力的拉格朗日方程(\autoref{eq_LagEqQ_1}~\upref{LagEqQ})
\begin{equation}
\dv{t} \pdv{L}{\dot q_i} = \pdv{L}{q_i} + Q_i^{(e)}~.
\end{equation}
注意以下所有对 $q_i, \dot q_i$ 或 $t$ 的\enref{偏导}{ParDer}都是把 $q, \dot q, t$ 作为变量($q, \dot q$ 分别是 $q_1, q_2\dots q_N$ 和 $\dot q_1, \dot q_2\dots \dot q_N$ 的简写), 即对其中一个变量求导数而把其他变量看做常数。 另外我们假设势能 $V$ 和位矢 $\bvec r_j$ 都不显含 $\dot q$, 即对任何 $\dot q_i$ 的偏导为零。 所以该证明不适用于 $V$ 为广义势能(例如 “\enref{拉格朗日电磁势}{EMLagP}”)的情况。

把系统看成质点组, 每个质点质量为 $m_j$, 位置矢量是广义坐标的函数 $\bvec r_j(q_1,\dots,q_N,t)$, 那么系统动能为
\begin{equation}
T = \frac12 \sum_j m_j \bvec v_j^2 = \frac12 \sum_j m_j \dot{\bvec r}_j \vdot \dot{\bvec r}_j~.
\end{equation}
由\enref{矢量内积的求导法则}{DerV}得($i=1,\dots,N$, 下同)
\begin{equation}\label{eq_dAlbt_5}
\pdv{T}{\dot q_i} = \sum_j m_j \dot{\bvec r}_j \vdot \pdv{\dot{\bvec r}_j}{\dot q_i}~,
\end{equation}
其中质点 $j$ 的速度可以用\enref{全导数}{TotDer} 公式
\begin{equation}
\dot{\bvec r}_j = \sum_k \pdv{\bvec r_j}{q_k} \dot q_k  + \pdv{\bvec r_j}{t}~.
\end{equation}
对 $\dot q_i$ 求偏导,注意位矢与$\dot q_i$ 无关,所以求偏导时 $\pdv*{r_j}{q_k}$ 与 $\pdv*{r_j}{t}$ 可看做常数。
\begin{equation}\label{eq_dAlbt_27}
\pdv{\dot{\bvec r}_j}{\dot q_i} = \pdv{\bvec r_j}{q_i}~.
\end{equation}
代入\autoref{eq_dAlbt_5} 并对时间求导得到拉格朗日方程的左边(注意我们假设 $\pdv*{V}{\dot q_i} = 0$)
\begin{equation}\label{eq_dAlbt_8}
\dv{t} \pdv{L}{\dot q_i} = \dv{t} \pdv{T}{\dot q_i} = \sum_j m_j \ddot{\bvec r}_j \vdot \pdv{\bvec r_j}{q_i}  + \sum_j m_j \dot{\bvec r}_j \vdot \dv{t} \pdv{\bvec r_j}{q_i}~.
\end{equation}
拉格朗日方程(\autoref{eq_LagEqQ_1}~\upref{LagEqQ})的右边为
\begin{equation}
\pdv{L}{q_i} = \pdv{T}{q_i} - \pdv{V}{q_i} + Q_i^{(e)}~.
\end{equation}
其中右边第一项为
\begin{equation}\label{eq_dAlbt_28}
\pdv{T}{q_i} = \sum_j m_j \dot{\bvec r}_j \vdot \pdv{\dot{\bvec r}_j}{q_i} = \sum_j m_j \dot{\bvec r}_j \vdot \pdv{q_i} \dv{\bvec r_j}{t}~,
\end{equation}
第二项是包含在势能中的广义力(\autoref{eq_LagEqQ_3}~\upref{LagEqQ})
\begin{equation}\label{eq_dAlbt_29}
Q_i^{(V)} = - \pdv{V}{q_i} = \sum_j \qty(-\pdv{V}{x_j} \pdv{x_j}{q_i} - \pdv{V}{y_j}\pdv{y_j}{q_i} - \pdv{V}{z_j} \pdv{z_j}{q_i}) = \sum_j \bvec F_j^{(V)} \vdot \pdv{\bvec r_j}{q_i}~.
\end{equation}
其中
\begin{equation}\label{eq_dAlbt_2}
\bvec F_j^{(V)} = - \grad_j V = -\pdv{V}{x_j}\uvec x - \pdv{V}{y_j}\uvec y - \pdv{V}{z_j} \uvec z~,
\end{equation}
注意这属于非约束力。 不包含于势能中的额外广义力的定义为(上标 $e$ 表示 extra)
\begin{equation}
Q_i^{(e)}(q, \dot q, t) = \sum_j \bvec F_j^{(e)} \vdot \pdv{\bvec r_j}{q_i}~,
\end{equation}
也是非约束力。 我们把非约束力称为\textbf{主动力}。 每个质点的总主动力可以分为由势能产生的部分和其他
\begin{equation}
\bvec F_j^{(a)} = \bvec F_j^{(V)} + \bvec F_j^{(e)}~,
\end{equation}
那么总广义力为
\begin{equation}\label{eq_dAlbt_1}
Q_i = Q_i^{(V)} + Q_i^{(e)} = \sum_j \bvec F_j^{(a)} \vdot \pdv{\bvec r_j}{q_i}~.
\end{equation}
所以要证明拉格朗日方程,即证明\autoref{eq_dAlbt_8} 等于\autoref{eq_dAlbt_28} 加\autoref{eq_dAlbt_1}, 首先需要证明
\begin{equation}
\dv{t} \pdv{\bvec r_j}{q_i} = \pdv{q_i} \dv{\bvec r_j}{t}~,
\end{equation}
也就是证明全导数和偏导数运算可对易。使用\enref{全导数}{TotDer} 的定义,以及\enref{混合偏导}{ParDer} 的性质,有
\begin{equation}
\dv{t} \pdv{\bvec r_j}{q_i} = \sum_k \pdv{q_k} \pdv{\bvec r_i}{q_i} \dot q_k  + \pdv{t} \pdv{\bvec r_j}{q_i} = \sum_k \pdv{q_i} \pdv{\bvec r_i}{q_k} \dot q_k + \pdv{q_i} \pdv{\bvec r_j}{t} = \pdv{q_i} \dv{\bvec r_j}{t}~,
\end{equation}
然后我们需要证明
\begin{equation}\label{eq_dAlbt_4}
\sum_j \qty(\bvec F_j^{(a)} - m\ddot{\bvec r}_j) \vdot \pdv{\bvec r_j}{q_i}  = 0~.
\end{equation}
即可证明拉格朗日方程。该式被称为\textbf{达朗贝尔原理}。注意由于这里的 $\bvec F_j^{(a)}$ 为质点 $j$ 所受的非约束力而不是合力,所以求和项的小括号一般不为 0。

\subsection{达朗贝尔原理}
\pentry{虚位移、虚功原理\nref{nod_VirWrk}}{nod_6014}
令第 $j$ 个质点所受合力等于主动力(非约束力)加约束力 $\bvec F_j = \bvec F_j^{(a)} + \bvec F_j^{(c)}$。 由牛顿第二定律 $\bvec F_j - m\ddot{\bvec r}_j = 0$, 所以
\begin{equation}
\sum_j \qty(\bvec F_j^{(a)}+\bvec F_j^{(c)} - m\ddot{\bvec r}_j) \vdot \pdv{\bvec r_j}{q_i} = 0~.
\end{equation}
现在我们只需要求
\begin{equation}\label{eq_dAlbt_3}
Q_i^{(c)} = \sum_j  \bvec F_j^{(c)} \vdot \pdv{\bvec r_j}{q_i}  = 0~,
\end{equation}
即可得到\autoref{eq_dAlbt_4}。 也就是\textbf{所有约束力产生的广义力都为零}。 由\autoref{eq_VirWrk_4}~\upref{VirWrk}, 达朗贝尔原理也可以表述为: \textbf{在约束允许的范围内进行任意虚位移, 所有约束力的\enref{虚功}{VirWrk}之和为零}。 

事实上达朗贝尔原理并不对所有约束都适用, 甚至不对所有稳定约束适用。 所以当列出拉格朗日方程时, 我们就必须要保证系统中所有的约束都符合达朗贝尔原理, 否则就需要用额外的广义力去修正(\autoref{ex_dAlbt_1})才能使拉格朗日方程仍然成立。 常见的约束都满足达朗贝尔原理, 这可以结合 “\enref{欧拉—拉格朗日方程}{Lagrng}” 中的例题思考, 找出例题中的约束力, 并验证他们的确不做虚功。 这里再给出一个双摆的例子:

\begin{example}{双摆的约束力}
在\enref{双摆模型}{Pendu3}中, 若连接两质点的是轻杆, 那么第一个质点 $m_1$ 受到延两条杆方向的力, 分别记为 $\bvec F_1, \bvec F_2$。 第二个质点 $m_2$ 受到延第二条杆的力 $\bvec F_3$。 首先由于 $m_1$ 的摆动方向始终垂直于 $\bvec F_1$, $\bvec F_1$ 从不做功。 其次由于杆都是轻杆, 有 $\bvec F_2 = -\bvec F_3$, 且无论双摆怎么运动, $\bvec F_2, \bvec F_3$ 做功之和都为零(否则第二条杆就会具有大于零的动能, 以无穷大的速度运动)。 可见单个约束力的虚功未必为零, 而是求和后为零。
\end{example}

在符合达朗贝尔原理的稳定约束中, 约束力无论是虚功还是实功都为零。 但在符合达朗贝尔原理的不稳定约束中, 约束力的实功可能不为零(思考\autoref{ex_Lagrng_3}~\upref{Lagrng})。 证明: 所有约束力的真实功率为(\autoref{eq_LagEqQ_5}~\upref{LagEqQ})
\begin{equation}
P^{(c)} = \sum_i Q_i^{(c)} \dot q_i + \sum_j \bvec F_j^{(c)} \vdot \pdv{\bvec r_j}{t} = \sum_j \bvec F_j^{(c)} \vdot \pdv{\bvec r_j}{t}~.
\end{equation}
稳定约束中, 即约束方程中不出现时间, 即 $\bvec r_j$ 不显含时间, 即 $\pdv*{\bvec r_j}{t} = \bvec 0$, 所以约束力实功功率为零。

对于非达朗贝尔约束, \autoref{eq_dAlbt_3} 不为零, 这时只需要把它加到拉格朗日右边进行修正即可:
\begin{equation}
\dv{t}\pdv{L}{\dot q_i} - \pdv{L}{q_i} = Q_i^{(e)} + Q_i^{(c)}~,
\end{equation}
$Q_i^{(e)}$ 是不包含于势能中的非束缚广义力。 证明留作习题。

\begin{example}{非达朗贝尔约束}\label{ex_dAlbt_1}
令水平轨道有两个滑块质量为 $m_1,m_2$, 坐标为 $x_1<x_2$。 第二个滑块与第一个滑块间有一个智能电动伸缩杆。 伸缩杆会进行测量并主动伸缩使两个滑块的坐标始终满足
\begin{equation}
x_2 = 2x_1~,
\end{equation}
且伸缩杆的质量可以忽略不计, 所以杆对两滑块的作用力始终等大反向。 这是一个稳定约束, 但做功却不为零。 设两滑块受非约束力分别为 $F_1,F_2$, 那么杆的力为
\begin{equation}
F_c = \frac{2m_2F_1 - m_1F_2}{m_1+2m_2}~.
\end{equation}

该系统只有一个自由度, 姑且用 $x_1$ 作为广义坐标, 那么 $F_1,F_2$ 产生的广义力为
\begin{equation}
Q^{(e)} = F_1 + 2F_2~,
\end{equation}
而约束力产生的广义力为
\begin{equation}
Q^{(c)} = -F_c + 2F_c = F_c~,
\end{equation}
所以广义力可能对系统做功。 拉格朗日方程为
\begin{equation}
\dv{t}\pdv{T}{\dot x_1} - \pdv{T}{x_1} = Q^{(e)} + Q^{(c)}~,
\end{equation}
其中动能为
\begin{equation}
T = \frac{1}{2}m_1\dot x_1^2 + \frac{1}{2}m_2 \dot x_2^2 = \frac{1}{2}(m_1 + 4m_2)\dot x_1^2~.
\end{equation}
\end{example}
另一个反例是, 若把摩擦力看作约束力, 则达朗贝尔原理同样不适用, 因为摩擦力总是做负功。
