% 跃迁概率(一阶微扰)
% 一阶微扰理论|电磁波|含时微扰|跃迁|能级|长度规范|速度规范

\pentry{电磁波包的频谱\upref{WpEng}, 含时微扰理论\upref{TDPTc}, 速度规范\upref{LVgaug}}

本文使用原子单位制\upref{AU}.一阶微扰理论就是单光子电离, 即两能级之间的能量等于单个光子的能量.

\subsection{长度规范下的微扰跃迁理论}
含时微扰理论(\autoref{TDPTc_eq1}~\upref{TDPTc}) 为
\begin{equation}\label{HionCr_eq1}
c_i(+\infty) = -\I \int_{-\infty}^{+\infty} \mel{i}{H'(t)}{j} \E^{\I\omega_{ij} t} \dd{t}
\end{equation}
$\ket{i},\ket{j}$ 可以是束缚态或者散射态, 需要正交归一化\upref{ScaNrm}. 长度规范中, 使用偶极子近似后电磁波哈密顿为(\autoref{LenGau_eq7}~\upref{LenGau})
\begin{equation}
H'(t) = -q\bvec {\mathcal E}(t) \vdot \bvec r
\end{equation}
$\bvec {\mathcal E}$ 只是 $t$ 的函数, 可以分离
\begin{equation}\label{HionCr_eq9}
\mel{i}{H'(t)}{j} = -q\bvec {\mathcal E}(t) \vdot \mel{i}{\bvec r}{j}
\end{equation}
令电场的傅里叶变换(\autoref{FTExp_eq6}~\upref{FTExp})为
\begin{equation}
\tilde {\bvec {\mathcal E}}(\omega) = \frac{1}{\sqrt{2\pi}} \int_{-\infty}^{\infty} \bvec {\mathcal E}(t) \E^{-\I\omega t} \dd{t}
\end{equation}
则\autoref{HionCr_eq9} 代入\autoref{HionCr_eq1} 得
\begin{equation}\label{HionCr_eq7}
c_i(+\infty) = \I q \mel{i}{\bvec r}{j} \vdot \int_{-\infty}^{+\infty} \bvec {\mathcal E}(t) \E^{\I\omega_{ij} t} \dd{t} = \I \sqrt{2\pi} q \mel{i}{\bvec r}{j} \vdot \tilde {\bvec {\mathcal E}}(-\omega_{ij})
\end{equation}
令 $\tilde {\bvec {\mathcal E}}(\omega) = \tilde {{\mathcal E}}(\omega)\uvec e$, 跃迁概率(密度)为
\begin{equation}\label{HionCr_eq2}
P_{ij} = \abs{c_i(+\infty)}^2 = 2\pi q^2 \abs{\uvec e \vdot \mel{i}{\bvec r}{j}}^2 \abs{\tilde {{\mathcal E}}(\omega_{ij})}^2
\end{equation}
使用能量面密度的频率分布(\autoref{WpEng_eq3}~\upref{WpEng})
\begin{equation}\label{HionCr_eq8}
s(\omega) = 2c\epsilon_0 \abs{\tilde {{\mathcal E}}(\omega)}^2
\end{equation}
得
\begin{equation}\label{HionCr_eq6}
P_{ij} = \frac{\pi q^2}{c\epsilon_0} \abs{\uvec e \vdot\mel{i}{\bvec r}{j}}^2 s(\omega_{ij})
\end{equation}
对于束缚态 $\ket{i}$, $P_{ij}$ 是从 $\ket{j}$ 跃迁到 $\ket{i}$ 的概率; 而对于连续态的 $\ket{i}$ (如原子电离), 若 $\ket{i}$ 对应出射方向的渐进动量 $\bvec k$, 那么 $P_{ij}$ 是 $\bvec k$ 空间的三维概率密度分布函数\upref{PTCont}.

另外注意只有电磁波包中频率为 $\omega_{ij}$ 的平面波分量对跃迁有贡献, 所以我们也可以直接将能量差 $\omega_{ij}$ 替换为 $\omega$.

对氢原子的具体计算见\autoref{HyIon2_eq1}~\upref{HyIon2}.

\subsection{速度规范下的微扰跃迁理论}
\footnote{参考\cite{Merzbacher} 含时微扰相关章节.}速度规范\upref{LVgaug}中,
\begin{equation}
H'(t) = -\frac{q}{m}\bvec A \vdot \bvec p = \frac{\I q}{m}\bvec A \vdot \grad
\end{equation}
$\bvec A$ 只是 $t$ 的函数, 可以分离
\begin{equation}
\mel{i}{H'(t)}{j} = -\frac{q}{m}\bvec A(t) \vdot \mel{i}{\bvec p}{j}
\end{equation}
令矢势的傅里叶变换为
\begin{equation}
\tilde {\bvec A}(\omega) = \frac{1}{\sqrt{2\pi}} \int_{-\infty}^{+\infty} \bvec A(t) \E^{-\I\omega t} \dd{t}
\end{equation}
代入\autoref{HionCr_eq1} 得
\begin{equation}\label{HionCr_eq4}
c_i(t) = \frac{\I q}{m} \mel{i}{\bvec p}{j} \vdot \int_{-\infty}^{+\infty}  \bvec A(t) \E^{\I\omega_{ij} t} \dd{t} = \I\sqrt{2\pi}\frac{q}{m} \mel{i}{\bvec p}{j} \vdot \tilde {\bvec A}(-\omega_{ij})
\end{equation}
令 $\tilde {\bvec A}(\omega) = \tilde {A}(\omega)\uvec e$, 则跃迁概率为
\begin{equation}\label{HionCr_eq3}
P_{ij} = \abs{c_i(t)}^2 = \frac{2\pi q^2}{m^2} \abs{\uvec e \vdot\mel{i}{\bvec p}{j}}^2 \abs{\tilde {A}(\omega_{ij})}^2
\end{equation}
结合波包的频谱公式(\autoref{WpEng_eq5}~\upref{WpEng} 变为原子单位)
\begin{equation}
s(\omega) = \frac{c}{2\pi} \omega^2 \abs{\tilde {A}(\omega_{ij})}^2
\end{equation}
\begin{equation}\label{HionCr_eq5}
P_{ij} = \frac{4\pi^2 q^2}{c m^2 \omega_{ij}^2} \abs{\uvec e \vdot\mel{i}{\bvec p}{j}}^2 s(\omega_{ij})
\end{equation}
对于束缚态 $\ket{i}$, $P_{ij}$ 是从 $\ket{j}$ 跃迁到 $\ket{i}$ 的概率; 而对于连续态的 $\ket{i}$ (如原子电离), 若 $\ket{i}$ 对应出射方向的渐进动量 $\bvec k$, 那么 $P_{ij}$ 是 $\bvec k$ 空间的三维概率密度分布函数\upref{PTCont}.

另外注意只有电磁波包中频率为 $\omega_{ij}$ 的平面波分量对跃迁有贡献, 所以我们也可以直接将能量差 $\omega_{ij}$ 替换为 $\omega$.

\subsubsection{两种规范比较}
注意 $\ket{i}, \ket{j}$ 是没有电磁场时的能量本征态, 波函数与规范无关. 把\autoref{DipEle_eq3}~\upref{DipEle} 和\autoref{WpEng_eq4}~\upref{WpEng} 带入\autoref{HionCr_eq4} 可以证明两种规范等效(\autoref{HionCr_eq7}  等于\autoref{HionCr_eq4}). 但是如果例如 $\ket{i}$ 是平面波, 则不同规范结果不同.
