% 相移
% keys 内反射|外反射|相移
% license Xiao
% type Tutor

\begin{issues}
\issueMissDepend
\end{issues}

\pentry{菲涅尔公式、布儒斯特角、临界角、内反射\upref{Fresnl}}{nod_75fd}

当光线以不同入射角入射时,振幅反射系数 $r$ 与振幅透射系数 $t$  可能出现负数情形,其负号的物理意义为电场 $\bvec{E}$ 的相应分量反向,相当于发生 $\pi$ 弧度的\textbf{相移}。相移在光的干涉与衍射中有重要影响。
下面我们分情况讨论相移:

1. 若 $\bvec{E}$ 平行于入射面,令
$$ r_p =  \dfrac{n_2\cos{\theta_i} - n_1\cos\theta_t}{n_1\cos\theta_t + n_2\cos\theta_i} < 0 ~.$$
化简得
$$ \sin(\theta_i - \theta_t)\cos(\theta_i + \theta_t) > 0~.$$

对于外反射情形,$n_t > n_i \Rightarrow \theta_i > \theta_t$,则
$$\theta_i + \theta_t > \frac{\pi}{2}~.$$
以 $\theta_i + \theta_t = \pi/2$ 为界,由于 $\theta_i \propto \theta_t$, 则当 $\theta_i > \theta_B$ 时, $\theta_i + \theta_t > \pi/2$。 也即,当入射角大于布儒斯特角时,电场的平行分量发生 $\pi$ 弧度的相移。

对于内反射情形,$ n_t < n_i \Rightarrow \theta_i < \theta_t$,则
$$\theta_i + \theta_t < \frac{\pi}~.{2}~.$$
同理可得,当入射角小于布儒斯特角时,电场的平行分量发生 $\pi$ 弧度的相移。

2. 若 $\bvec{E}$ 垂直于入射面,令
$$r_s  = \frac{n_1\cos{\theta_i} - n_2\cos\theta_t}{n_1\cos\theta_i + n_2\cos\theta_t} < 0~,$$
化简得
$$\sin(\theta_t - \theta_i) < 0~.$$
同理可知,对于外反射,电场垂直分量相移恒为 $\pi$; 对于内反射,入射角小于临界角 $\theta_c$ 时,相移恒为 0。

3. 我们并没有讨论 $t_s$ 和 $t_p$ 与入射角关系,是由于无论入射角取何值, $ t$ 恒大于0。

\begin{figure}[ht]
\centering
\includegraphics[width=12cm]{./figures/5898bf6ae0781860.png}
\caption{外反射: $r$、$t$ 作为入射角的函数} \label{fig_PhaSft_1}
\end{figure}

\begin{figure}[ht]
\centering
\includegraphics[width=13cm]{./figures/5d4418705a8da129.png}
\caption{外反射:相移与入射角的关系} \label{fig_PhaSft_4}
\end{figure}

内反射:  $r$、$t$ 作为入射角的函数(同\autoref{fig_PhaSft_1})

\begin{figure}[ht]
\centering
\includegraphics[width=13cm]{./figures/98019453584a36ce.png}
\caption{内反射: 相移与入射角的关系} \label{fig_PhaSft_2}
\end{figure}

综上所述,在考虑一般的问题时,我们可以使用以下口诀来判断电场分量所发生的相移: 
\textbf{外直内平相差 $\pi$}。
意思是,当 $n_t > n_i$ 时,电场的垂直分量有相移 $\pi$; 当 $n_t < n_i$ 时,电场的平行分量有相移 $\pi$。
