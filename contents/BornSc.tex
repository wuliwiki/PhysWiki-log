% 量子散射的波恩近似
% keys 量子散射|波恩近似|积分方程

\pentry{量子散射\upref{ParWav}, Lippmann-Schwinger 方程\upref{LipSch}}

我们还是要解出连续态的不含时波函数,且无穷远处的动量为 $\bvec k_i$ (入射平面波的动量)。
\begin{equation}
-\frac{\hbar^2}{2m} \laplacian\psi  + V\psi  = E\psi~,
\end{equation}
\begin{equation}
(\laplacian + k^2)\psi  = \frac{2m}{\hbar^2} V\psi  \equiv U(\bvec r)\psi~.
\end{equation}
这是非齐次亥姆霍兹方程,其格林函数% 未完成: 格林函数 亥姆霍兹方程的格林函数
为(球面波)
\begin{equation}
G(R) =  -\frac{\E^{\I kR}}{4\pi R}~,
\end{equation}
满足
\begin{equation}
(\laplacian + k^2)G(\bvec r) = \delta(\bvec r)~.
\end{equation}
\textbf{薛定谔方程的积分形式}, 即 Lippmann-Schwinger 方程为
\begin{equation}\label{eq_BornSc_5}
\psi(\bvec r) = \psi_0(\bvec r) + \int G(\abs{\bvec r - \bvec r'})U(\bvec r')\psi (\bvec r') \dd[3]{r'}~.
\end{equation}
$\psi_0$ 是自由粒子波函数,由于无穷远处积分项消失($1/r$), $\psi_0(\bvec r\to\infty)$ 要求具有动量 $\bvec k_i$,唯一的选择是平面波
\begin{equation}
\psi_0(\bvec r) = A\E^{\I\bvec k_i \vdot \bvec r}~.
\end{equation}
由于微分截面定义在无穷远处,我们把格林函数取无穷远处的极限(远场),注意这个极限在定义中,所以并不算是一个近似。这是关于 $\bvec r'$ 的平面波
\begin{equation}
\abs{\bvec r - \bvec r'} \approx r - \uvec r \vdot \bvec r' \approx r~,
\end{equation}
\begin{equation}
G(\bvec r, \bvec r') =  - \frac{\E^{\I k \abs{\bvec r - \bvec r'}}}{4\pi\abs{\bvec r - \bvec r'}} \to  - \frac{\E^{\I kr}}{4\pi r} \E^{-\I \bvec k_f \vdot \bvec r'}~.
\end{equation}
其中 $\bvec k_f = k\uvec r$ 是出射的方向,注意 $\abs{\bvec k_i} = \abs{\bvec k_f}$ 意味着弹性散射。

积分方程求近似解的一般方法是先把一个近似解代入积分内,积分得到一阶修正后的解,再次代入,得到二阶修正后的解,以此类推迭代。波恩近似中,假设势能相对于入射动能较弱,积分项相当于微扰,所以令初始(零阶)波函数为 $\psi_0(\bvec r)$。代入\autoref{eq_BornSc_5} 得一阶修正的波函数,叫做\textbf{第一波恩近似}
\begin{equation}
\psi ^{(1)}(\bvec r) = A\E^{\I\bvec k_i \vdot \bvec r} - A \frac{m}{2\pi\hbar^2} \frac{\E^{\I kr}}{r}\int \E^{\I (\bvec k_i - \bvec k_f) \vdot \bvec r'} V(\bvec r') \dd[3]{r'}~.
\end{equation}
根据定义,散射幅为
\begin{equation}
f(k, \hat r) =  - \frac{m}{2\pi\hbar^2} \int \E^{\I (\bvec k_i - \bvec k_f) \vdot \bvec r'} V(\bvec r') \dd[3]{r'}~,
\end{equation}
这相当于势能函数的空间傅里叶变换。

\subsection{高阶波恩近似}
把\autoref{eq_BornSc_5} 多次代入\autoref{eq_BornSc_5} 的积分中,得到精确解的“积分级数”形式
\begin{equation}\ali{
\psi(\bvec r) &= \psi_0(\bvec r) + \int \dd[3]{r'} G(k,\bvec r,\bvec r')U(\bvec r')\psi_0(\bvec r')  \\
&+ \int \dd[3]{r'} G(k,\bvec r,\bvec r')U(\bvec r') \int \dd[3]{r''} G(k,\bvec r',\bvec r'')U(\bvec r'')\psi_0(\bvec r'') \\
&+ \int \dd[3]{r'} G(k,\bvec r,\bvec r')U(\bvec r')\times\\
&\int \dd[3]{r''}G(k,\bvec r',\bvec r'')U(\bvec r'')\int \dd[3]{r'''}G(k,\bvec r'',\bvec r''')U(\bvec r''') \psi_0(\bvec r''')
  ... 
}~\end{equation}
若只计算是指包含前 $n$ 行,就叫第 $n$ 波恩近似。具体计算时,偶尔会用到二阶,基本不会用到三阶或以上。

非常有趣的是,即使我们不假设零阶波函数是平面波,波函数展开成上式时取前 $n$ 行的结果仍然是相同的。
