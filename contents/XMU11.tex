% 厦门大学 2011 年 考研 量子力学
% license Usr
% type Note

\textbf{声明}:“该内容来源于网络公开资料,不保证真实性,如有侵权请联系管理员”

\subsection{(25分)简述题(每小题5分)}

(1) 波函数有没有物理含义?它的物理含义体现在哪里?物理上对波函数有哪些要求?

(2) 什么是态矢?在什么情况下,可以用态矢来描述量子体系的特征?定念具有哪些性质?

(3) 利用测不准关系,估算一维无限深势阱 $V(x) = \begin{cases} 
0, & |x| < a \\\\
\infty, & |x| \geq a
\end{cases}$ 中质点为$m$的粒子的基态能量。

(4) 在希尔伯特 (Hilbert) 空间中,正交归一的完备基矢组设为{k)},试分别就分立谱和连续谱情形,写出这组基天完备性的数学表示式。

(5) 一来单能量无相互作用的粒子流,经中心势场 $V(r)$ 弹性散射后,在 $r=0$ 处的渐近解为
$$ \psi = e^{ikz} + f(\theta) \frac{e^{ikr}}{r},~$$
试解释$\psi$表达式中第一项和第二项的物理意义,并说明 $f(\theta)$ 的含义。
\subsection{(25分)}
一维谐振子能量本征态记为 $\psi_n(x)$, 假设一个谐振子初始状态为

$$\psi(x,0) = \frac{1}{\sqrt{2}} \left[ \psi_0(x) + \psi_2(x) \right]~$$

试求: (1) 任意时刻  ($t > 0$)  的波函数 $\psi(x,t)$; 

(2) 经过多少时间第一次演化为状态  $$\frac{1}{\sqrt{2}} \left[ \psi_0(x) + i \psi_2(x) \right]~$$.
\subsection{(25分)}
设质量为 $m$ 的一维粒子在如下势场中运动:
\[V(x) =\begin{cases}     \infty, & x \leq 0 \\\\    -\frac{\alpha \hbar^2}{mx}, & x > 0\end{cases}~\]
其中 $\alpha > 0$ 是一个系统参数。

(1) 证明 $\psi_0(x) = 
\begin{cases}
    0, & x \leq 0 \\\\
    N xe^{-\alpha x}, & x > 0
\end{cases}
$ 
是该系统定念薛定谔方程的一个解,并求出相应的本征值:

(2) 求出归一化因子 $N$。参考公式:
\[\int_0^\infty x^n e^{- \alpha x} dx = \frac{n!}{\alpha^{n+1}}~\]

(3) 对于 $\psi_0(x)$ 态,计算平均动能和平均势能。

(4) 对于 $\psi_0(x)$ 态,计算何处几率密度最大。

(5) 判断 $\psi_0(x)$ 态是否为基态?给出判断理由。
\subsection{(25分)}
对于任意一个幺正算符 \( U \)

(1) 若把 \( U \) 表示成 \( U = A + iB \) 的形式,证明 \( A \) 和 \( B \) 必为厄米 (Hermite) 算符,且满足 \( A^2 + B^2 = 1 \), \([A, B] = 0 \)。

(2) 若把 \( U \) 表示成 \( U = e^{iF} \),证明 \( F \) 必为厄米算符。
\subsection{(25分)}
某单价原子,已知价电子的波函数(不考虑自旋)为
$$\psi(r,\theta,\varphi)=R_{32}(r)\frac{1}{\sqrt{2}}[Y_{21}(\theta,\phi)+Y_{20}(\theta,\phi)]~$$
其中$Y_{im}$为轨道角动量算符广$\hat{L}^2$和$\hat{L}_2$的共同本征函数。

求:(1)$\hat{L}^2$和$\hat{L}_2$的可能测量值和相应的概率;

(2)平均值$\overline{L}_x,\overline{L}_y,\overline{L^2_x},\overline{L^2_y}$
\subsection{(25分)}
自旋为$\frac{1}{2}$的三维各向同性振子,受到微扰$H'=\lambda \vec{\sigma}\cdot \vec{r}$的作用($\vec{\sigma}$是泡利(Pauli)自旋算符)。

(1)写出在受到微扰作用之前,体系的哈密顿(Hamiton)量$H_0$、能级.$E^0_n$和能量本征函数$\psi^0_{n_1n_2n_yn_x}(x,y,z,S,)$指出$n$与$n_1,n_2,n_3$之间的关系。

(2)将能量本征态$\ket{n_1,n_2,n_3,m_3}$简记为$\ket{n}$,若体系基态是$\ket{0}=\ket{000\frac{1}{2}}$,证明在微扰矩阵元中$\langle n | H' | 0 \rangle$中,只有$\langle 100-\frac{1}{2} | H' | 000\frac{1}{2} \rangle$,$\langle 010\frac{1}{2} | H' | 000\frac{1}{2} \rangle$和$\langle 001\frac{1}{2} | H' | 000\frac{1}{2} \rangle$不为零,并求出其值.

(3)求基态能级的微扰修正(准确到二级近似)。【提示】一维谐振子的能量本征态,$\psi_n(x)$满足
$$x\psi_n(x)=\frac{1}{\alpha}\left[\sqrt{\frac{n}{2}}\psi_{n-1}(x)+\sqrt{\frac{n+1}{2}}\psi_{n+1}(x)\right]~$$