% 浙江大学 2003 年 考研 量子力学
% license Usr
% type Note

\textbf{声明}:“该内容来源于网络公开资料,不保证真实性,如有侵权请联系管理员”

\subsection{第一题(35 分):}
1. 如果 $\psi_1$ 和 $\psi_2$ 是某一体系含时薛定谔方程的两个解

1) 它们的线性组合 $\psi = a\psi_1 + b\psi_2$,($a, b$ 是常数),是否满足同样的含时薛定谔方程?

2) 若令 $\psi' = \psi_1\psi_2$,你认为 $\psi'$ 是否满足同样的含时薛定谔方程?

2. 质量相同的两个粒子分别在宽度不同的两个一维无限深势阱中,试问势阱中的基态能量低,还是宽势阱中的基态能量低?

3.1) 你是否认识这三个矩阵:
\[\begin{pmatrix}0 & 1 \\\\1 & 0\end{pmatrix}\quad\begin{pmatrix}0 & -i \\\\i & 0\end{pmatrix}\quad\begin{pmatrix}1 & 0 \\\\0 & -1\end{pmatrix}~\]
在量子力学中他们称为什么?

2) 大家知道,$[\hat{x}, \hat{p}] = i\hbar$ 为量子力学中最基本的对易关系(这里 $\hat{x}$ 和 $\hat{p}$ 分别是位置算符和动量算符)

和动量算符,你是否记得角动量 $\hat{L}_x, \hat{L}_y, \hat{L}_z$ 之间的对易关系?请写出来!

3) 请算一下
\[[[\hat{L}_x, \hat{L}_y], \hat{L}_z] + [[\hat{L}_y, \hat{L}_z], \hat{L}_x] + [[\hat{L}_z, \hat{L}_x], \hat{L}_y] = ?~\]
\subsection{第二题(20 分):}
有一个双势阱(与量子前沿问题有关)

\[V(x) =\begin{cases} \infty, & x < 0 \\\\-V_0, & 0 < x < a \\\\0, & a < x < 2a \\\\-V_0, & 2a < x < 3a \\\\\frac{1}{2} V_0, & 3a < x \end{cases}~\]
这里 $V_0 > 0$,试写出各区域内波函数的合理形式以及连接各区域的边界条件(不必具体求解)

\subsection{第三题(25分):}
处在均匀电场中的二维带电谐振子的哈密顿量为
\[\hat{H} = \frac{1}{2m} (p_x^2 + p_y^2) + \frac{1}{2} m\omega^2 (x^2 + y^2) + eEx~\]
(其中电场强度 $E$ 为常数)

(1) 求出其能级。

(2) 电场 $E$ 的大小会产生什么影响?
\subsection{第四题(20 分):}
如果把原子实看作由一个点核和价电子均匀分布在半径为 $a_0$ 的球内所组成,那么其散射势可表示为
\[V(r) =\begin{cases} \frac{ze^2}{r} - \frac{r}{R}, & r < a_0 \\\\0, & r > a_0 \end{cases}~\]
其中 $R = \frac{a_0^2}{ze^2}$,试用玻恩近似求散射截面。