% 氢原子的跃迁偶极子矩阵元和选择定则
% 选择定则|宇称|3j 符号

\pentry{3j 符号\upref{ThreeJ}, 含时微扰理论\upref{TDPT}, 类氢原子的束缚态\upref{HWF}}

氢原子的\textbf{选择定则(selection rule)}是指在哪些情况下\textbf{跃迁偶极子矩阵元(transition dipole matrix element)} $\mel{\psi_{n,l,m}}{\bvec r}{\psi_{n',l',m'}}$ 为零。 该矩阵在含时微扰理论中出现, 如果矩阵元为零, 说明在一阶微扰近似下 $\ket{\psi_{n',l',m'}}$ 不能在变化电场的作用下跃迁到 $\ket{\psi_{n,l,m}}$。 但即使矩阵元为零, 仍有可能通过多阶微扰仍然存在耦合(即矩阵的 $N$ 次方中该矩阵元不为零)。 从物理上来看就是先从初态跃迁到中间态, 再从中间态跃迁到末态。 如果高阶微扰的跃迁也被禁止(即矩阵的任意次方中该矩阵元都为零), 那么就是绝对禁止的。 % 未完成: 不确定是不是这么定义的

\subsection{偶极子矩阵元的计算}
相比于算符对易的方法\footnote{参考\cite{GriffQ}。}, 3j 符号的好处是不仅能得到选择定则,还可以直接算出偶极子矩阵元角向积分的具体值而无需手动积分\footnote{当然, 手动 3j 符号也比较繁琐, 可以借助 Wolfram Alpha 或 Mathematica, Matlab 中我也写了一个程序(同样可以符号计算), 还没放进百科。}。 长度规范\upref{LenGau}中电场哈密顿量为 $\bvec{\mathcal{E}}\vdot \bvec r$ (长度规范\upref{LenGau})。 其中 $\bvec r$ 的三个分量可以用球谐函数表示为(\autoref{eq_SphHar_4}~\upref{SphHar})
\begin{equation}
x = \sqrt{\frac{2\pi}{3}} r (Y_{1,-1} - Y_{1,1})~, \qquad
y = \I \sqrt{\frac{2\pi}{3}} r (Y_{1,-1}+Y_{1,1})~, \qquad
z = \sqrt{\frac{4\pi}{3}} rY_{1,0}~,
\end{equation}
所以
\begin{equation}\label{eq_SelRul_3}
\quad\mel{\psi_{n,l,m}}{\bvec{\mathcal E}\vdot\bvec r}{\psi_{n',l',m'}} = \bvec{\mathcal E}\vdot\mel{\psi_{n,l,m}}{\bvec r}{\psi_{n',l',m'}}~,
\end{equation}

\begin{equation}
\begin{aligned}
&\qquad \mel{\psi_{n,l,m}}{\bvec r}{\psi_{n',l',m'}}
= \sqrt{\frac{4\pi}{3}}\int R_{n,l}(r) r R_{n',l'}(r) r^2 \dd{r}\times\\
&\Big[\frac{\uvec x}{\sqrt 2} \qty(\mel{Y_{l,m}}{Y_{1,-1}}{Y_{l',m'}} - \mel{Y_{l,m}}{Y_{1,1}}{Y_{l',m'}})\\
&+ \frac{\I \uvec y}{\sqrt 2} \qty(\mel{Y_{l,m}}{Y_{1,-1}}{Y_{l',m'}} + \mel{Y_{l,m}}{Y_{1,1}}{Y_{l',m'}})\\
&+ \uvec z\mel{Y_{l,m}}{Y_{1,0}}{Y_{l',m'}}\Big]~.
\end{aligned}
\end{equation}
把三个球谐函数之积的积分用 3j 符号表示(\autoref{eq_SphHar_5}~\upref{SphHar}) 得跃迁偶极子矩阵元为
\begin{equation}
\begin{aligned}
&\quad\mel{\psi_{n,l,m}}{\bvec r}{\psi_{n',l',m'}}\\
&= (-1)^m\sqrt{(2l+1)(2l'+1)} \pmat{l & 1 & l'\\ 0 & 0 & 0} \int_0^\infty R_{n,l}(r) r R_{n',l'}(r) r^2 \dd{r}\times\\
&\Big\{\frac{\uvec x}{\sqrt 2} \qty[\pmat{l & 1 & l'\\ -m & -1 & m'} - \pmat{l & 1 & l'\\ -m & 1 & m'}]\\
&+ \frac{\I \uvec y}{\sqrt 2} \qty[\pmat{l & 1 & l'\\ -m & -1 & m'} + \pmat{l & 1 & l'\\ -m & 1 & m'}]\\
&+  \uvec z \pmat{l & 1 & l'\\ -m & 0 & m'}\Big\}~.
\end{aligned}
\end{equation}

Mathematica\upref{Mma} 代码如下(\verb|HydrogenR| 的定义见这里\upref{HWF})
\begin{lstlisting}[language=mma]
Dipole[Z_, n1_, l1_, m1_, n2_, l2_, m2_] :=\
(-1)^m1 Sqrt[(2 l1 + 1) (2 l2 + 1)]\
  ThreeJSymbol[{l1, 0}, {1, 0}, {l2, 0}] Integrate[\
  HydrogenR[Z, n1, l1, r]\[Conjugate] HydrogenR[Z, n2, l2, r] r^3, {r,\
    0, +∞}] {(ThreeJSymbol[{l1, -m1}, {1, -1}, {l2, m2}] - \
     ThreeJSymbol[{l1, -m1}, {1, 1}, {l2, m2}])/Sqrt[\
   2], (ThreeJSymbol[{l1, -m1}, {1, -1}, {l2, m2}] + \
     ThreeJSymbol[{l1, -m1}, {1, 1}, {l2, m2}])/Sqrt[2], \
  ThreeJSymbol[{l1, -m1}, {1, 0}, {l2, m2}]}
\end{lstlisting}
矩阵元列表见 “氢原子的跃迁偶极子矩阵元列表\upref{HDipM}”。

\subsection{选择定则}
令 $\Delta m = m - m'$, $\Delta l = l - l'$。 使用 3j 的选择定则(\autoref{eq_ThreeJ_1}~\upref{ThreeJ}) 规定第二行的三个 $m$ 相加为零, 所以只有满足
\begin{equation}\label{eq_SelRul_4}
\Delta m =
\begin{cases}
0 & (\text{$z$ 方向}) \\
0, \pm 1 & (\text{$x, y$ 方向})
\end{cases}~
\end{equation}
时跃迁矩阵元可能不为零。

由三角约束(\autoref{eq_AMAdd_2}~\upref{AMAdd}) $\abs{l-l'} \leqslant 1 \leqslant l + l'$ 得 $\Delta l = 0, \pm 1$。 但由\autoref{eq_ThreeJ_8}~\upref{ThreeJ} 得 $l + l' + 1$ 为奇数时 $\mel{Y_{l,m}}{Y_{1,m_1}}{Y_{l',m'}}$ 为零, 所以只有满足
\begin{equation}\label{eq_SelRul_2}
 \Delta l = \pm 1~
\end{equation}
时跃迁矩阵元可能不为零。

\autoref{eq_SelRul_4} \autoref{eq_SelRul_2} 是两条最常见的选择定则, 它们不满足时矩阵元必为零, 但我们并没有证明满足了必不为零。 一般来说 3j 符号还有其他选择定则, 找到所有 3j 符号(或 CG 系数)为零的情况是十分困难的。
\addTODO{需核实 “满足了也由少数情况可能为零”, 需引用相关文献。 至少在\autoref{tab_HDipM_1}~\upref{HDipM} 中符合选择定则的都是非零。}

\subsection{物理意义}
$z$ 方向的电场不会改变电子 $z$ 方向的角动量, 所以 $L_z$ 守恒\upref{QMcons}, $\Delta m$ 为 0。 另外, 虽然我们的计算中并未使用光子的概念(详见量子电动力学), 但 $\Delta l = 0, \pm 1$ 暗示了光子的自旋\upref{Spin}为 $l=1$, 而事实的确如此。
