% 共轭空间与代数共轭空间
% keys 共轭空间|代数共轭空间
% license Xiao
% type Tutor

\pentry{线性连续泛函\nref{nod_LinCon}}{nod_c37f}
\cite{Ke1}共轭空间是由在\enref{拓扑线性空间}{tvs}上\enref{线性连续泛函}{LinCon}的全体构成的线性空间,而代数共轭则是由线性泛函的全体构成的线性空间。

\begin{lemma}{}\label{lem_ConSpa_1}
设 $L$ 是\enref{线性空间}{LSpace} ,则其上的线性泛函的全体在下面的加法和数乘之下构成一个线性空间:

\textbf{加法:} $(f_1+f_2)(x):=f_1(x)+f_2(x)$;

\textbf{数乘:} $(\alpha f)(x):=\alpha f(x)$。
\end{lemma}

\begin{definition}{共轭空间}
设 $E$ 是拓扑线性空间,则其上的全体\enref{线性连续泛函}{LinCon}在\autoref{lem_ConSpa_1} 的加法和数乘下构成的线性空间称为与 $E$ \textbf{共轭}的空间,记为 $E^*$。
\end{definition}

\begin{definition}{代数共轭空间}
在拓扑线性空间上所有(不一定连续)线性泛函的全体在\autoref{lem_ConSpa_1} 的加法和数乘下构成的线性空间称为与 $E$ \textbf{代数共轭}的空间,记为 $E^{\#}$。
\end{definition}




