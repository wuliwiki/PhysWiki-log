% 多元傅里叶变换

\pentry{傅里叶变换(指数)\upref{FTExp}, 多元函数的傅里叶级数\upref{NdFuri}}

$N$ 维空间中, 性质良好的标量函数 $f(\bvec r) = f(x_1, x_2, \dots, x_N)$ 都可以用 $N$ 维平面波 $\E^{\I \bvec k\vdot \bvec r}$ 展开。
\begin{align}
g(\bvec k) &= \frac{1}{\sqrt{2\pi}^N} \int f(\bvec r) \E^{-\I \bvec k \vdot \bvec r} \dd[N]{r}\label{NFTran_eq1}\\
f(\bvec r) &= \frac{1}{\sqrt{2\pi}^N} \int g(\bvec k) \E^{\I \bvec k \vdot \bvec r} \dd[N]{k}
\end{align}
这就是 $N$ 元函数的傅里叶变换和反变换。 这个过程有点类似于多元函数的傅里叶级数\upref{NdFuri}, 事实上 $N$ 维平面波就是许多一维平面波的乘积, 令 $\bvec k = (k_1, k_2, \dots, k_N)$, 有
\begin{equation}
\ket{\bvec k} = \frac{1}{\sqrt{2\pi}^N}\E^{\I \bvec k \vdot \bvec r} = \frac{1}{\sqrt{2\pi}} \E^{\I k_1 x_1} \frac{1}{\sqrt{2\pi}} \E^{\I k_2 x_2} \dots \frac{1}{\sqrt{2\pi}} \E^{\I k_N x_N}
\end{equation}
满足正交归一化条件(见狄拉克 delta 函数\upref{Delta})
\begin{equation}
\braket{\bvec k'}{\bvec k} = \int \frac{1}{\sqrt{2\pi}^N}\E^{-\I \bvec k' \vdot \bvec r} \frac{1}{\sqrt{2\pi}^N}\E^{\I \bvec k \vdot \bvec r} \dd[N]{r} = \delta(\bvec k'- \bvec k)
\end{equation}
\addTODO{其实应该引用多元 delta 函数}

多元傅里叶变换的证明可以类比傅里叶变换(指数)\upref{FTExp}, 不再赘述。

\begin{example}{$N$ 维高斯波包}
要计算 $n$ 维高斯函数
\begin{equation}\label{NFTran_eq2}
f(\bvec r) = \exp(-a\bvec r^2) \qquad (a > 0)
\end{equation}
的傅里叶变换,只需将\autoref{NFTran_eq2} 代入\autoref{NFTran_eq1} 并利用\autoref{Erf_eq2}~\upref{Erf} ,即得
\begin{equation}
g(\bvec k) = \frac{1}{(\sqrt{2a})^N}\E^{-\frac{k^2}{4a}}
\end{equation}
%\addTODO{推导类比\autoref{FTExp_ex1}~\upref{FTExp}}
\end{example}
