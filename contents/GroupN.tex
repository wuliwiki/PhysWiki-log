% 群论笔记

参考文献: Mathematical Methods for Physicists (Arfken et. al.)

\subsection{17.1 Introduction to Group Theory}
\begin{itemize}
\item \textbf{群}的定义: 定义了乘法的集合, (1) 闭合, (2) 结合律, (3) 单位元, (4) 逆元

\item \textbf{阶数}: 离散群的阶数等于元素数, 连续群的阶数等于参数的个数

\item 满足交换律的群就叫\textbf{阿贝尔(abelian)}群

\item \textbf{循环(cyclic)}群: 群内的所有元素都可以用某个元素表示为 $I, a, a^2, \dots$

\item 循环群是阿贝尔群

\item 两个群中的元素如果一一对应(包括乘法运算), 则他们是 \textbf{isomorphic} 的, 如果是多对一, 就是 \textbf{homomorphic} 的

\item 如果一个群的子集也是一个群, 就称为\textbf{子群(subgroup)}
\end{itemize}

\subsection{17.2 Representation of Groups}

\subsection{17.3 Symmetry and Physics}

\subsection{17.4 Distrete Groups}

\subsection{17.5 Distrete Products}
\begin{itemize}
\item \textbf{直积}: 两个群 $G$ 和 $H$ 通过直积运算得到一个新群 $G \otimes H$, 元素为 $(g, h)$, 其中 $g\in G$, $h\in H$. $G \otimes H$ 中的乘法运算为对应元素分别运算 $(g_1, h_1)(g_2, h_2) = (g_1 g_2, h_1 h_2)$, 注意两个元素的乘法可能不一样.
\end{itemize}

\subsection{17.6 Symmetric Group}
\begin{itemize}
\item \textbf{对称群(symmetric group)} $S_n$ 是 $n$ 个不同对象的 permutation 操作组成的群, 所以共有 $n!$ 个元素
\end{itemize}

\subsection{17.7 Continuous Groups}
\begin{itemize}
\item $SO(n)$ (\textbf{special orthogonal})表示 $n$ 维空间旋转群($n(n-1)/2$ 阶), 如果加上 reflection, 就是 $O(n)$.
\item $SU(n)$ (\textbf{special unitary}), 阶数是 $n^2 - 1$
\item \textbf{李群(Lie group)}: 
\end{itemize}
