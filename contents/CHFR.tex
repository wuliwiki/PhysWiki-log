% 詹姆斯·查德威克
% license CCBYSA3
% type Wiki

(本文根据 CC-BY-SA 协议转载自原搜狗科学百科对英文维基百科的翻译)

\textbf{詹姆斯·查德威克爵士},CH, FRS (1891年10月20日 –1974年7月24日),英国物理学家,于1935年被授予诺贝尔物理学奖,以表彰他在1932年对中子的发现。1941年,他撰写了莫德报告,这促使美国政府开始认真对待原子弹研究工作。第二次世界大战期间,他是曼哈顿计划英国团队的负责人。因为他在物理方面的成就而被封爵。

查德威克于1911年毕业于曼彻斯特维多利亚大学,在那里他师从欧内斯特·卢瑟福 (被称为“核物理之父”)。[1] 在曼彻斯特,他继续在卢瑟福的指导下学习,直到1913年获得理学硕士学位。同年,查德威克被授予来自皇家展览委员会的1851年研究奖学金。他选择在柏林的汉斯·盖格(Hans Geiger)领导下研究β辐射。查德威克使用盖革(Geiger)最近开发的盖革计数器(Geiger counter),能够证明β辐射产生了连续光谱,而不是所想象的离散线。 第一次世界大战在欧洲爆发时,他还在德国,接下来的四年在Ruhleben拘留所度过。

战后,查德威克跟随卢瑟福来到了剑桥大学的卡文迪许实验室,在卢瑟福指导下,查德威克于1921年6月在剑桥大学冈维尔与凯斯学院获得了他的哲学博士的学位。十多年来,他一直是卢瑟福在卡文迪许实验室的助理研究主任,当时它是世界上最重要的物理研究中心之一,吸引着像约翰·考克饶夫,诺曼·费瑟,和马克·奥利芬特这样的学生。查德威克通过测量中子质量从而发现了中子。 他预计中子将成为对抗癌症的主要武器。 查德威克于1935年离开卡文迪许实验室成为利物浦大学的物理学教授,在那里他对一个过时的实验室进行了大修,并通过安装回旋加速器使其成为了核物理研究的重要中心。

在第二次世界大战期间,作为“管状合金”项目的一部分,查德威克(Chadwick)进行了研究以制造原子弹,而他在曼彻斯特的实验室和周围地区遭到德国空军轰炸的骚扰。 当《魁北克协定》将他的项目与美国曼哈顿项目合并时,他成为英国使团的一员,并在洛斯阿拉莫斯实验室和华盛顿特区工作。他赢得了项目总监小莱斯利·R·格罗夫斯(Leslie R. Groves,Jr.)几乎完全的信任,这使所有人感到惊讶。由于他的努力,查德威克于1945年1月1日获得了新年荣誉骑士勋章。1945年7月,他观看了Trinity核试验。 此后,他担任联合国原子能委员会的英国科学顾问。 由于对大科学的趋势不满意,查德威克于1948年成为冈维尔和凯斯学院的硕士。他于1959年退休。

\subsection{教育和早期生活}
詹姆斯·查德威克于1891年10月20日出生于柴郡 柏灵顿,[2][3] 是棉纺工人约翰·约瑟夫·查德威克和佣人安妮·玛丽·诺尔斯的第一个孩子。他以祖父的名字命名为詹姆斯。1895年,他的父母搬到了曼彻斯特并把他交给了他的外祖父母照顾。他去了伯灵顿十字小学,并获得了曼彻斯特文法学校, 但是他的家庭不得不拒绝这个奖学金因为他们负担不起仍需支付的小额费用。相反,他参加了在曼彻斯特的中央男子文法学校,并在那里与父母团聚。他现在有两个弟弟,哈利和休伯特;一个妹妹在婴儿期就去世了。16岁时,他参加了两次大学奖学金考试,两次都获得了奖学金。[4]

1908年,查德威克选择进入曼彻斯特维多利亚大学。他本打算学习数学,但误报了物理。像大多数学生一样,他住在家里,每天在街上步行6.4公里,往返于家和大学。在他的第一年结束时,他获得了学习物理的黑金波托姆奖学金。物理系由欧内斯特·卢瑟福领导,他给四年级学生分配研究项目,并指示查德威克设计一种方法来比较两种不同来源的放射性能量。这个方法是,可以用1克(0.035盎司)镭的活性来测量它们,镭的测量单位被称为居里。卢瑟福建议的方法是行不通的——查德威克知道但不敢告诉卢瑟福——所以查德威克坚持下去,最终设计出了所需的方法。这一结果成为查德威克的第一篇论文,与卢瑟福合著,发表于1912年。[5] 他于1911年以一等荣誉毕业。[6]

查德威克设计了一种测量伽马射线的方法,然后开始测量各种气体和液体对伽马射线的吸收。这一次,由此产生的论文仅仅是以他的名义发表的。他于1912被授予了理学硕士学位,并被任命为拜尔研究员。第二年,他被授予1851年展览奖学金,这使他能够在欧洲大陆的一所大学里学习和研究。1913年,他当选去柏林的Physichalisch-Technische-Reichsanstalt医院,研究汉斯·盖革领导下的beta辐射。[7]他使用盖革(Geiger)最近开发的盖革计数器(Geiger counter),该计数器比早期摄影技术提供更高的准确性,他能够证明β辐射不会像以前所认为的那样产生离散的线条,而是在某些区域出现峰值的连续光谱。[8][9][10][11] 在参观盖革的实验室时,阿尔伯特·爱因斯坦告诉查德威克:“我可以解释这两件事中的任何一件,但我不能同时解释它们。”[10] 连续光谱在很多年内仍将是无法解释的现象。[12]

第一次世界大战开始时,查德威克还在德国,他被关押在柏林附近的鲁勒本拘留营,在那里,他被允许在马厩中建立一个实验室,并使用简易材料如放射性牙膏进行科学实验。在查尔斯·德拉蒙德·埃利斯的帮助下,他致力于磷的电离以及一氧化碳和氯的光化学反应。[13][14] 他于1918年11月与德国停战协定生效后被释放,并返回其父母在曼彻斯特的家中,在此期间他为1851年展览专员撰写了他的发现。[15]

卢瑟福给了查德威克一个在曼彻斯特的兼职教师职位,使他能够继续研究。[15] 他查看了铂,银和铜的核电荷,并通过实验发现这与原子序数相同,误差小于1.5%。[16]1919年4月,卢瑟福成为剑桥大学卡文迪许实验室的主任,几个月后查德威克加入他的行列。 查德威克(1920)在1920年被授予麦克斯韦(Clerk-Maxwell)奖学金,并在剑桥的冈维尔(Gonville)和凯斯学院(Caius College)攻读哲学博士(PhD)。 他论文的前半部分是他对原子数的研究。 后半部分,他研究了核内的力。 他的学位于1921年6月获得授予。[17] 11月,他成为冈维尔和凯厄斯学院的研究员。[18]

\subsection{研究员}
\subsubsection{2.1 剑桥}
\begin{figure}[ht]
\centering
\includegraphics[width=8cm]{./figures/299eb9c75bdfce1e.png}
\caption{卡文迪许实验室是一些物理学伟大发现的发源地。它由德文郡公爵于1874年创立(卡文迪许是他的姓),其首任教授是詹姆斯·克拉克·麦克斯韦。[2]} \label{fig_CHFR_1}
\end{figure}
查德威克在克拉克-麦克斯韦的学生任期于1923年届满,他的继任者是俄罗斯物理学家彼得·卡皮萨。科学和工业研究部咨询委员会主席威廉·麦考密克爵士安排查德威克担任卢瑟福的助理研究主任。在这个角色中,查德威克帮助卢瑟福挑选博士生。在接下来的几年里,这些人之中包括约翰·考克饶夫、诺曼·费瑟和马克·奥利芬特,他们都成为查德威克的好朋友。由于许多学生不知道他们想研究什么,卢瑟福和查德威克会提出一些话题。查德威克编辑了实验室制作的所有论文。[19]

1925年,查德威克遇到了艾琳·斯图尔特·布朗,她是一名利物浦股票经纪人的女儿。两人于1925年8月结婚,[19] 卡皮萨是伴郎。这对夫妇有一对双胞胎女儿乔安娜和朱迪思,出生于1927年2月。[20]

在他的研究中,查德威克继续探测原子核。1925年,自旋的概念允许物理学家解释塞曼效应,但它也产生了无法解释的异常。当时人们认为原子核由质子和电子组成,因此,假设质量数为14的氮原子核包含14个质子和7个电子。这给了它正确的质量和电荷,但错误的自旋。[21]

1928年在剑桥举行的关于$\beta$粒子和$\gamma$射线的会议上,查德威克再次会见了盖革。盖革带来了他的盖革计数器的新模型,这个模型被他的博士后学生瓦尔特·米勒改进了。查德威克自战争以来从未使用过闪烁计数器,新的盖革-米勒计数器可能是对剑桥当时使用的依赖人眼观察的闪烁技术的重大改进。它的主要缺点是它检测到了$\alpha$、$\beta$和$\gamma$辐射,以及镭,卡文迪许实验室通常在实验中使用镭,发射了所有这三种辐射,因此不适合查德威克的想法。然而,钋是$\alpha$发射体,莉泽·迈特纳(Lise Meitner)从德国向查德威克(Chadwick)发送了大约2毫米(约0.5微克)的钋。[22][23]

在德国,瓦尔特·博特和他的学生赫伯特·贝克尔用钋轰炸铍α粒子,产生一种不寻常的辐射。查德威克让他的澳大利亚1851年展览学者休·韦伯斯特复制了他们的结果。对查德威克来说,这是他和卢瑟福多年来一直假设的事情的证据:中子,一种理论上不带电荷的核粒子。[22] 然后在1932年1月,费瑟让查德威克注意到了另一个令人惊讶的结果。弗雷德里克·和伊雷娜·约里奥-居里利用钋和铍作为伽马辐射的来源,成功地从石蜡中撞击出质子。卢瑟福和查德威克不同意;质子太重了。但是中子只需要少量的能量就能达到同样的效果。在罗马,埃托雷·马约拉纳得出了同样的结论:裘力奥-居里夫妇发现了中子,但并没有意识到。[24]

查德威克放弃了所有其他职责,在费瑟的帮助下,专注于证明中子的存在,[25] 经常工作到深夜。他设计了一个简单的设备,由一个装有钋源和铍靶的圆柱体组成。所产生的辐射随后可以被导向诸如石蜡的材料;被转移的粒子是质子,它们将进入一个小电离室,在那里它们可以被示波器检测到。[24]
\begin{figure}[ht]
\centering
\includegraphics[width=8cm]{./figures/8b35a67b67be4339.png}
\caption{欧内斯特·卢瑟福爵士的实验室} \label{fig_CHFR_2}
\end{figure}
1932年2月,在用中子进行了大约两周的实验后,[26] 查德威克寄了一封信给自然杂志,标题为“中子的可能存在”。[26] 随后,他将自己的发现进行了细致的总结并发给了皇家学会学报A版,文章名称为“中子的存在”。[27][28] 他的中子的发现是理解原子核的里程碑。罗伯特·巴赫(Robert Bacher)和爱德华·康登(Edward Condon)在阅读查德威克(Chadwick)的论文时认识到,如果中子的自旋为$1/2$,并且氮原子核由七个质子和七个中子组成,那么时空理论中的异常,如氮的自旋,将得到解决。[29][30]

理论物理学家尼尔斯·玻尔和维尔纳·海森堡考虑了中子是否可以是像质子和电子一样的基本核粒子,而不是质子-电子对。[31][32][33][34] 海森堡表明,最好将中子描述为一种新的核粒子,[33][34] 但其确切性质仍不清楚。查德威克在他1933年的《贝克里安讲座》中估计,中子的质量约为1.0067 u。因为质子和电子的总质量为1.0078 u,这意味着中子作为质子-电子复合物的结合能约为2 MeV,这种推论听起来很合理,[35] 虽然很难理解一个具有如此小结合能的粒子是如何稳定的。[34] 估算如此小的质量差需要进行精确的测量,然而,在1933–4年获得了一些矛盾的结果。通过用α粒子轰击硼,弗雷德里克和伊雷娜·约里奥-居里获得了中子质量的很大值,但是欧内斯特·劳伦斯在加州大学分校的团队产生了一个小值。[36] 然后,来自纳粹德国的难民,卡文迪许实验室的一名研究生莫里斯·戈德哈伯(Maurice Goldhaber)向查德威克(Chadwick)建议,氘核可以被208Tl的2.6 MeV伽马射线光解 (当时称为钍C" ):
\begin{figure}[ht]
\centering
\includegraphics[width=14.25cm]{./figures/430c7fe86f569a94.png}
\caption \label{fig_CHFR_3}
\end{figure}
中子质量的精确值可以从这个过程中确定。查德威克和戈德哈伯尝试了这个方法,发现有效。[37][38][39] 他们测量了质子产生的动能为1.05 MeV,而中子的质量在方程中是未知的。查德威克和戈德哈伯根据质子和氘核的质量计算得出,它要么是1.0084原子单位,要么是1.0090原子单位。[40][39] (中子质量的现代公认值是1.00866 u。) 中子的质量太大,不可能是质子——电子对。[40]

查德威克因为发现中子,于1932年被皇家学会授予休斯奖章,1935年被授予诺贝尔物理学奖,1950年被授予科普利奖章,1951年被授予富兰克林奖章。[41] 他对中子的发现使得在实验室通过捕获慢中子和随后的β衰变产生比铀重的元素成为可能。与其他原子核中存在的电场力排斥的带正电的α粒子不同,中子不需要克服任何库仑势垒,因此甚至可以穿透和进入甚至最重的元素(如铀)的原子核。 这激发了恩里科·费米(Enrico Fermi)研究核与慢中子碰撞所引起的核反应,费米为此所做的工作在1938年获得了诺贝尔奖。[41]

沃尔夫冈·泡利在1930年12月4日提出了另一种粒子,以解释查德威克在1914年报告的β辐射的连续光谱。因为不是所有$\beta$辐射的能量都可以被解释,所以能量守恒定律似乎被违反了,但泡利认为,如果涉及另一个未被发现的粒子,这可以得到纠正。[42] 泡利也称这种粒子为中子,但它显然与查德威克的中子不同。费米把它改名为中微子,意大利语是“小中子”。[43] 1934年,费米提出了他的β衰变理论,这解释了从原子核发射的电子是由中子衰变为质子、电子和中微子所产生的。[44][45] 中微子可以解释缺少的能量,但是很难观察到质量很小且没有电荷的粒子。鲁道夫·佩尔斯和汉斯·贝特计算出中微子很容易穿过地球,所以探测到它们的机会很小。[46][47] 弗雷德里克·雷纳斯(Frederick Reines)和克莱德·科万(Clyde Cowan)将于1956年6月14日通过将探测器放置在附近核反应堆的大型反中微子通量中来确认中微子。[48]

\subsection{2.2 利物浦}
\begin{figure}[ht]
\centering
\includegraphics[width=6cm]{./figures/6888d807f458834f.png}
\caption{"红砖"维多利亚大楼,利物浦大学} \label{fig_CHFR_4}
\end{figure}
随着英国大萧条的到来,政府对科学的资助变得更加节俭。 同时,劳伦斯的最新发明回旋加速器有望彻底改变实验核物理,查德威克认为,除非卡文迪许实验室也收购了它,否则它将会落伍。 因此,他在卢瑟福(Rutherford)的领导下感到恼火,后者坚信没有大型昂贵的设备仍然可以实现良好的核物理,并拒绝了回旋加速器的要求。[49]

查德威克本人总体上是大科学的批评者,尤其是劳伦斯,他认为劳伦斯的方法粗心大意,以科学为代价专注于技术。当劳伦斯在1933年的索尔维会议上假定存在一种新的,迄今未知的粒子时,他声称这是无限能量的可能来源,查德威克回答说,结果更可能归因于设备污染。[50] 劳伦斯在伯克利重新检查了他的结果后才发现查德威克是正确的,而卢瑟福和奥利芬特在卡文迪许进行了一次调查,发现氘融合形成3号氦,从而引起了劳伦斯观察到的影响。 这是另一个重大发现,但是奥列芬特-卢瑟福粒子加速器是昂贵的最新设备[51][52][53][54]

1935年3月,查德威克(Chadwick)接受了妻子家乡利物浦大学里昂·琼斯(Lyon Jones)物理学教席的邀请,以接替莱昂内尔·威尔伯福斯(Lionel Wilberforce)。 实验室非常陈旧,无法继续使用直流电,但是查德威克抓住了这个机会,于1935年10月1日就任主席。1935年11月宣布的查德威克获得诺贝尔奖很快提升了该大学的声望。[55] 他的奖牌在2014年以32.9万美元的价格拍卖。[56]

查德威克开始为利物浦购置回旋加速器。他首先花了700英镑整修利物浦陈旧的实验室,这样一些部件就可以在内部制造。[57]他说服了这所大学提供2000英镑,并从皇家学会获得了另外2000英镑的资助。[58] 为了建造回旋加速器,查德威克引进了两位年轻的专家,伯纳德·金赛和哈罗德·沃克,他们曾在加利福尼亚大学与劳伦斯共事。当地一家电缆制造商捐赠了用于线圈的铜导线。回旋加速器的50吨磁体由大都会-维克斯公司在特拉福德公园制造,该公司也制造了真空室。[59] 回旋加速器于1939年7月完工并运行。总费用为5,184英镑,比查德威克大学和皇家学会收取的总费用还多,因此查德威克从他的159,917印度卢比(8,243英镑)诺贝尔奖金中支付了其余的费用。[60]

在利物浦,医学和科学系紧密合作。 查德威克自动成为这两个系的委员会成员,1938年,他被任命为以德比勋爵为首的委员会,负责研究利物浦的癌症治疗安排。 查德威克预计,由37英寸回旋加速器产生的中子和放射性同位素可用于研究生化过程,并有可能成为抗击癌症的武器。[61][62]

\subsection{第二次世界大战}
\subsubsection{3.1 合金管工程和莫德报告}
在德国,奥托·哈恩(Otto Hahn)和弗里茨·斯特拉斯曼(Fritz Strassmann)用中子轰击了铀,并指出,钡是一种较轻的元素,是所生产的产品。迄今为止,该方法仅生产了相同或较重的元素。 1939年1月,迈特纳(Meitner)和她的侄子奥托·弗里施(Otto Frisch)用解释这一结果的论文震惊了物理学界。[63] 他们认为,被中子轰击的铀原子可以分解成两个大致相等的碎片,他们称其为裂变。他们计算得出这将导致大约200 MeV的释放,这意味着能量释放比化学反应要大几个数量级。[64] 弗里希通过实验证实了他们的理论。[65] 哈恩很快注意到,如果在裂变过程中释放中子,则可能发生连锁反应。[66]法国科学家皮埃尔·乔利奥(Pierre Joliot),汉斯·冯·哈尔班(Hans von Halban)和刘·科沃斯基(Lew Kowarski)很快证实,每个裂变确实发射了多个中子。[67] 玻尔在与美国物理学家约翰·惠勒(John Wheeler)合着的论文中指出,裂变更可能发生在铀235同位素中,铀仅占天然铀的0.7\%。[68][69]
\begin{figure}[ht]
\centering
\includegraphics[width=10cm]{./figures/c849fc60a3510804.png}
\caption{英国主要物理学家。从左到右:威廉·彭尼、奥托·弗里希、鲁道夫·佩尔斯和约翰·考克饶夫。他们戴着自由勋章。} \label{fig_CHFR_5}
\end{figure}
查德威克不相信1939年国家会与德国爆发另一场战争的可能性,于是带着家人去瑞典北部一个偏远的湖边度假。因此,第二次世界大战爆发的消息令他震惊。查德威克决心不再在拘留营里为另一场战争耗费时间,他以最快的速度赶到了斯德哥尔摩,但当他和家人到达那里时,他发现斯德哥尔摩和伦敦之间的所有空中交通已经中断。他们乘坐流浪者轮船回到英国。 当他到达利物浦时,查德威克发现曾与回旋加速器一起工作的波兰博士后研究员约瑟夫·罗特布拉特(Joseph Rotblat)如今已处于贫困状态,因为他被波兰切断了资金。查德威克立即聘请罗特布拉特为讲师,尽管他英语掌握地不好。[70]

1939年10月,查德威克收到科学和工业研究部秘书爱德华·阿普尔顿爵士的信,要求他就原子弹的可行性发表意见。 查德威克谨慎地回应。 他没有排除这种可能性,而是仔细地解决了所涉及的许多理论和实践难题。 查德威克决定用Rotblat进一步研究氧化铀的性质。[71]1940年3月,伯明翰大学的奥托·弗里希和鲁道夫·佩尔斯重新研究了一篇被称为《弗里希-佩尔斯备忘录》的论文中涉及的理论问题。他们没有观察铀金属,而是考虑了纯铀-235的球体会发生什么,发现不仅可能发生连锁反应,而且可能只需要很少的时间,1 kg 铀-235就可以释放出数吨炸药的能量。[72]
\begin{figure}[ht]
\centering
\includegraphics[width=10cm]{./figures/788a9a32c6dbc4e9.png}
\caption{被闪电战摧毁的利物浦部分地区} \label{fig_CHFR_6}
\end{figure}
为进一步调查此事,成立了空战科学调查委员会的一个特别小组委员会,称为 MAUD委员会。它由乔治·汤姆森爵士主持,其原始成员包括查德威克,以及马克·奥利芬特、约翰·考克饶夫和菲利普·穆恩。[73] 当其他团队研究铀浓缩技术时,查德威克在利物浦的团队专注于确定铀-235的核反应截面。[74] 到1941年4月,实验证实铀-235的临界质量可能为8 kg 或者更少。[75] 德国空军对利物浦实验室周围的爆炸不断,使他对这类问题的研究变得复杂。 窗户经常被吹破,以至于用纸板代替。。[76]

1941年7月,查德威克被选中撰写MAUD报告的最终草案,该报告于1941年10月由范内瓦尔·布什(Vannevar Bush)提交给总统富兰克林·D·罗斯福(Franklin D. Roosevelt)时,启发了美国政府投入数百万美元用于追求原子弹。[77] 当乔治·佩格姆和哈罗德·尤里访问英国查看该项目时,[78] (现在被称为合金管工程)[79] 查德威克告诉他们:“我希望我能告诉你炸弹不会起作用,但我90\%确信它会起作用。”[78]

在最近一本关于炸弹项目的书中,格雷厄姆·法梅洛写道,“查德威克比任何其他科学家都更努力地把炸弹给了丘吉尔。 ...查德威克几乎被测试到了极限。"[80] 查德威克担心睡不着觉,于是开始服用安眠药,并在剩余的大部分时间里一直服用。查德威克后来说,他意识到“核弹不仅是可能的——而且是不可避免的。这些想法迟早不会是我们特有的。不久,每个人都会想到它们,一些国家会把它们付诸行动”。[81] 赫尔曼·邦迪爵士认为,幸运的是查德威克,而不是卢瑟福,是当时英国物理学的元老,否则查德威克的声望可能会压倒他对“展望”原子弹前景的兴趣。[82]

\subsubsection{3.2 曼哈顿计划}
\begin{figure}[ht]
\centering
\includegraphics[width=10cm]{./figures/17141cc1750d8c33.png}
\caption{麦肯齐·金,富兰克林·罗斯福和温斯顿·丘吉尔在第一魁北克会议,1943年} \label{fig_CHFR_7}
\end{figure}
由于空中轰炸的危险,查德威克夫妇把他们的双胞胎送到了加拿大,作为政府疏散计划的一部分。[83] 查德威克不愿意把合金管工程搬到那里,认为英国是同位素分离厂的更好所在地。[84] 这项工作的巨大消耗在1942年变得更加明显: 即使是一个试点分离工厂也将花费100多万英镑,并且会耗尽英国的资源,更不用说一个全面规模的工厂了,估计其成本在2500万英镑左右。它必须在美国建造。[85] 在英国人开始相信联合项目是必要的同时,美国的曼哈顿计划的进展使得英国的合作显得不那么重要,尽管美国人仍然渴望利用查德威克的才能。[86]

合作问题必须在最高级别上讨论。1943年9月,首相·温斯顿·丘吉尔和罗斯福总统谈判了《魁北克协定》,恢复了英国、美国和加拿大之间的合作。查德威克、奥列芬特、佩尔斯和西蒙被合金管工程的主任华莱士·阿克斯爵士召到美国来从事曼哈顿计划。魁北克协定成立了一个新的联合政策委员会来指导联合项目。美国人不喜欢阿克斯,所以查德威克被任命为联合政策委员会的技术顾问和英国使团团长。[87]

1943年11月,查德威克离开罗特布拉特,开始参观曼哈顿计划的设施,除了生产钚的汉福德区,他不被允许参观。除了格罗夫斯和他的副手之外,他成为唯一一个能接触到美国所有铀炸弹研究和生产设施的人。在观察田纳西州橡树岭(Oak Ridge)的K-25气体扩散设施的工作后,查德威克意识到他在战时英国建造工厂是多么的错误:巨大的建筑永远无法向德国空军隐藏。[88] 1944年初,他和妻子及双胞胎搬到了洛斯阿拉莫斯(新墨西哥州) ,他们现在用加拿大口音说话。[89] 出于安全原因,他被冠以詹姆斯·查菲的绰号。[90]
\begin{figure}[ht]
\centering
\includegraphics[width=10cm]{./figures/190dc4e8fe64a2c2.png}
\caption{查德威克(左)和少将 莱斯利·格罗夫斯小,曼哈顿计划的主管} \label{fig_CHFR_8}
\end{figure}
查德威克承认美国人不需要英国的帮助,但这仍然有助于项目早日圆满结束。他与曼哈顿项目负责人莱斯利·R·格罗夫斯少将紧密合作,竭尽全力支持这项工作。[91] 他还努力让英国科学家尽可能多地参与该项目,以促进查德威克致力于的战后英国核武器项目。格罗夫斯通过查德威克向特定科学家提出的请求往往会立即遭到目前雇用他们的公司、部委或大学的拒绝,但只有通过给予合金管工程压倒一切的优先地位才能得到解决。[92]因此,英国团队对项目的成功至关重要。[93]

尽管他比英国的任何人都更了解这个项目,[94] 查德威克仍然无法进入汉福德区。1946年,洛尔·波塔勋爵(Lord Portal)受到汉福德的访问。“这是查德威克在战时被拒绝进入的唯一一家工厂,现在他问格罗夫斯是否可以陪伴波多尔。 传送门不会看到太多。”[95] 由于他的努力,查德威克于1945年1月1日获得了新年荣誉骑士勋章。[96] 他认为这是对整个合金管工程项目工作的认可。[97]

到1945年初,查德威克大部分时间都在华盛顿哥伦比亚特区,1945年4月,他的家人从洛斯阿拉莫斯搬到了华盛顿州杜邦圆环的一所房子里。[97] 他出席了7月4日的联合政策委员会会议,当时元帅·亨利·梅特兰·威尔逊爵士同意英国对日使用原子弹,[98] 并于7月16日进行了三位一体(核试验) ,第一颗原子弹被引爆。[99] 在其坑内是钋铍调制中子引发剂,这是查德威克十年前用来发现中子的技术的发展。[100] 曼哈顿计划的《纽约时报》记者威廉·劳伦斯(William L. Laurence)写道:“在历史上,从来没有人活着看到自己的发现对人的命运产生如此明显的影响。”[101]

\subsection{晚年}
战争结束后不久,查德威克被任命为原子能咨询委员会(ACAE)成员。他还被任命为联合国原子能委员会的英国科学顾问。他与ACAE成员帕特里克·布莱克特发生冲突,后者不同意查德威克的观点,即英国需要获得自己的核武器;但查德威克的立场最终被采纳了。他于1946年回到英国,发现这个国家仍然饱受战时配给和短缺的困扰。[102]

这时,利物浦大学副校长詹姆斯·芒福德爵士在他的日记中写道,“他从未见过像查德威克这样身体、精神和心理如此疲惫的人”,因为他“充分考虑了道德决策的深度 因为甚至从来没有更多的幸运者被要求凝视……并遭受……由于他的科学工作而产生的几乎无法支持的责任感。”[103]

1948年,查德威克接受了成为冈维尔和凯厄斯学院硕士的邀请。这份工作声望很高,但定义不明确;大师是学院名义上的校长,但权力实际上存在于一个由13名研究员组成的委员会中,其中一人是大师。作为校长,查德威克努力提高学院的学术声誉。他把研究奖学金的数量从31个增加到49个,并试图把人才引进学院。[104]这涉及到有争议的决定,例如1951年雇佣了中国生物化学家曹天钦[105] 和匈牙利出生的经济学家彼得·鲍尔。在所谓的农民起义中,由帕特里克·哈德利(Patrick Hadley)领导的同僚投票将查德威克(Chadwick)的老朋友选为理事会成员,并用鲍尔(Bauer)代替了他。查德威克(Chadwick)的更多朋友在随后的几年中被撤职,他于1958年11月退休。正是在他的主导下,弗朗西斯·克里克冈维尔和凯厄斯学院的博士生詹姆斯·杜威·沃森发现了脱氧核糖核酸的结构。[104]

多年来,查德威克(Chadwick)获得了许多荣誉,包括美国的勋章和德国的梅勒奖。[106] 他在1927年当选为皇家学会院士。[107] 1946年,他成为荷兰皇家艺术与科学学院的外籍成员。[108] 1970年1月1日,他因“为科学服务”而被授予“新年荣誉勋章”,[109] 并赴白金汉宫举行了揭幕仪式。尽管他去利物浦庆祝自己八十岁生日,但他变得更加虚弱,很少离开公寓。 作为一生的无神论者,他没有理由在以后的生活中采用宗教信仰。 他于1974年7月24日在睡眠中死亡。[106]

他的论文保存在剑桥的丘吉尔档案中心,并向公众开放。[110] 利物浦大学的查德威克实验室是以他的名字命名的,[111] 其实验物理学主席詹姆斯·查德威克爵士也是如此,他在1991年以他的名字命名,作为庆祝他诞辰100周年的一部分。[112] 月球上的一个陨石坑也以他的名字命名。[113] 曼彻斯特大学化学工程和分析科学学院的一部分詹姆斯·查德威克大楼是以他的名字命名的。[114] 英国原子能管理局官方历史学家洛纳·阿诺德将他描述为“一名物理学家、科学家兼外交官、一名善良、明智和人道的人”[115]

\subsection{参考文献}
[1]
^"Ernest Rutherford". Figures in Radiation History. Michigan State University. Archived from the original on 29 June 2015. Retrieved 3 June 2014..

[2]
^Falconer 2004..

[3]
^Oliphant 1974..

[4]
^Brown 1997, pp. 3–5..

[5]
^Rutherford & Chadwick 1912..

[6]
^Brown 1997, pp. 6–14..

[7]
^Brown 1997, pp. 16–21..

[8]
^Chadwick 1914..

[9]
^Chadwick & Ellis 1922..

[10]
^Weiner 1969..

[11]
^Jensen 2000, pp. 88–90..

[12]
^Brown 1997, pp. 24–26..

[13]
^"Obituary: Sir James Chadwick". The Times. 25 July 1974. p. 20, column F..

[14]
^"Obituary: Sir Charles Ellis". The Times. 15 January 1980. p. 14, column F..

[15]
^Brown 1997, p. 39..

[16]
^Brown 1997, pp. 43..

[17]
^Brown 1997, pp. 43–50..

[18]
^Brown 1997, p. 58..

[19]
^Brown 1997, pp. 73–76..

[20]
^Brown 1997, p. 85..

[21]
^Brown 1997, pp. 92–93..

[22]
^Brown 1997, pp. 95–97..

[23]
^Sublette 2006..

[24]
^Brown 1997, pp. 103–104..

[25]
^"Oral History interview transcript with Norman Feather, Session I". American Institute of Physics, Niels Bohr Library and Archives. 25 February 1971..

[26]
^"This Month in Physics History: May 1932: Chadwick reports the discovery of the neutron". APS News. 16 (5): 2. 2007..

[27]
^Chadwick 1932b..

[28]
^Chadwick 1933..

[29]
^Whaling 2009, pp. 8–9..

[30]
^Bacher & Condon 1932..

[31]
^Heisenberg 1932a..

[32]
^Heisenberg 1932b..

[33]
^Heisenberg 1933..

[34]
^Bromberg 1971..

[35]
^Brown 1997, pp. 115–116..

[36]
^Heilbron & Seidel 1989, pp. 153–157..

[37]
^Goldhaber 1934..

[38]
^Chadwick & Goldhaber 1934..

[39]
^Chadwick & Goldhaber 1935..

[40]
^Brown 1997, pp. 122–125..

[41]
^"James Chadwick – Biography". The Nobel Foundation. Retrieved 21 April 2013..

[42]
^Brown 1997, pp. 119–120..

[43]
^Close 2012, pp. 15–18..

[44]
^Fermi 1968..

[45]
^Close 2012, pp. 22–25..

[46]
^Close 2012, pp. 26–28..

[47]
^Bethe, H; Peierls, R (7 April 1934). "The Neutrino". Nature. 133 (3362): 532. Bibcode:1934Natur.133..532B. doi:10.1038/133532a0. ISSN 0028-0836..

[48]
^Close 2012, pp. 37–41..

[49]
^Brown 1997, pp. 129–132..

[50]
^Herken 2002, p. 10..

[51]
^Heilbron & Seidel 1989, pp. 165–167..

[52]
^Oliphant & Rutherford 1933..

[53]
^Oliphant, Kinsey & Rutherford 1933..

[54]
^Oliphant, Harteck & Rutherford 1934..

[55]
^Brown 1997, pp. 134–139..

[56]
^Gannon, Megan (4 June 2014). "Sold! Nobel Prize for Neutron Discovery Auctioned for 329,000". Yahoo News. Retrieved 16 September 2014..

[57]
^Brown 1997, p. 142..

[58]
^Brown 1997, pp. 149–151..

[59]
^Holt 1994..

[60]
^Brown 1997, pp. 173–174..

[61]
^King 1997..

[62]
^Brown 1997, p. 150..

[63]
^Brown 1997, p. 170..

[64]
^Meitner & Frisch 1939..

[65]
^Frisch 1939..

[66]
^Hahn & Strassmann 1939..

[67]
^von Halban, Joliot & Kowarski 1939..

[68]
^Gowing 1964, pp. 24–27..

[69]
^Bohr & Wheeler 1939..

[70]
^Brown 1997, pp. 174–178..

[71]
^Gowing 1964, pp. 38–39..

[72]
^Gowing 1964, pp. 39–41..

[73]
^Gowing 1964, p. 45..

[74]
^Gowing 1964, p. 63..

[75]
^Brown 1997, p. 206..

^Brown 1997, p. 204..

[77]
^Bundy 1988, pp. 48–49..

[78]
^Gowing 1964, p. 85..

[79]
^Gowing 1964, p. 109..

[80]
^Farmelo 2013, p. 119..

[81]
^Brown 1997, p. 205..

[82]
^Bondi 1997..

[83]
^Brown 1997, pp. 197–198..

[84]
^Brown 1997, pp. 218–219..

[85]
^Gowing 1964, pp. 141–142..

[86]
^Gowing 1964, p. 152..

[87]
^Gowing 1964, pp. 166–171..

[88]
^Brown 1997, p. 253..

[89]
^Brown 1997, pp. 250–261..

[90]
^Hoddeson et al. 1993, p. 95..

[91]
^Brown 1997, pp. 247–51..

[92]
^Gowing 1964, pp. 241–244..

[93]
^Szasz 1992, p. xvi..

[94]
^Gowing 1964, p. 329..

[95]
^Brown 1997, p. 317..

[96]
^"No. 36866". London Gazette (Supplement). 29 December 1944. 下级勋位爵士.

[97]
^Brown 1997, p. 279..

[98]
^Brown 1997, p. 290..

[99]
^Brown 1997, p. 292..

[100]
^Brown 1997, p. 287..

[101]
^Laurence 1946, p. 26..

[102]
^Brown 1997, pp. 306, 316..

[103]
^Brown 1997, p. 323..

[104]
^Brown 1997, pp. 340–353..

[105]
^Zhang 2010..

[106]
^Brown 1997, pp. 360–363..

[107]
^Massey & Feather 1976, p. 11..

[108]
^"J. Chadwick (1891–1974)". Royal Netherlands Academy of Arts and Sciences. Retrieved 21 July 2015..

[109]
^"No. 44999". London Gazette (Supplement). 30 December 1969. 荣誉的伴侣.

[110]
^"The Papers of Sir James Chadwick". Janus. Retrieved 26 April 2013..

[111]
^"Liverpool Science Places". Scienceplaces.org. Archived from the original on 15 August 2014. Retrieved 6 August 2014..

[112]
^"University Chairs and their Holders Past and Present" (PDF). University of Liverpool. Retrieved 1 August 2014.[永久失效连结].

[113]
^"Planetary Names: Crater, craters: Chadwick on Moon". United States Geological Survey. Archived from the original on 22 November 2017. Retrieved 12 August 2012..

[114]
^"James Chadwick Building - directions". The University of Manchester. Retrieved 18 May 2016..

[115]
^Arnold 1998..