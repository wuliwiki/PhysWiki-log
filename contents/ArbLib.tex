% Arb 任意精度计算库

\pentry{C++ 基础\upref{Cpp0}}

Arb 是一个支持任意精度计算的 C/C++ 库(也提供 python 接口\upref{PyFlnt}), 支持对计算进行严谨的误差估计, 即每个任意精度浮点数 $z$ 都会伴随一个误差半径 $r$, 使得精确结果必定落在复平面上以 $z$ 为圆心半径为 $r$ 的圆盘中. 通过增加浮点数的精度, 就可以用数值方法无限逼近理论值. Arb 还提供了许多特殊函数的计算, 例如 $\Gamma$ 函数\upref{Gamma}, 超几何函数\upref{HypGeo} 等, 以及常用的线性代数功能和离散傅里叶变换\upref{DFT}等. 官方主页 \href{https://arblib.org/}{arblib.org} 包含详细的文档.

\subsection{安装}
以 Ubuntu 为例, 最方便的安装方式就是使用 \verb|apt| 安装. 首先安装 dependency
\begin{lstlisting}[language=bash]
sudo apt install libmpfr-dev libflint-dev
\end{lstlisting}
其中 \verb|MPFR| 和 \verb|flint| 两个包分别用于任意精度浮点数以及数论. 然后安装 Arb
\begin{lstlisting}[language=bash]
sudo apt install libflint-arb-dev libflint-arb2
\end{lstlisting}

但目前这并不是 Arb 的最新版本(例如没有实现库仑函数\upref{CulmF}). 要获得最新版本, 可以直接从 \href{https://github.com/fredrik-johansson/arb/}{GitHub} 下载源码编译即可(默认使用 gcc 编译器). 目前笔者使用的版本是 release 2.19.0 (ubuntu 20.04 和 22.04 均可直接用 apt 安装更高版本).

可以用 \verb|./configure --help| 查看编译选项, 若所有的包都安装在默认目录则不需要编译选项.
\begin{lstlisting}[language=bash]
./configure [编译选项];
make -j4;
sudo make install;
\end{lstlisting}
其中 \verb|-j4| 是使用 4 线程进行编译, 也可以改成其他数字.

要在 windows 上安装, 参考\href{https://github.com/ahrvoje/arb4win}{这个}. 这里面的 dll 是可以直接拿来用的, 无需编译.

\subsection{编译}
在 Ubuntu 中如果你使用 \verb|apt| 安装, 在编译程序是需要加 \verb|-lflint-arb| 选项. 如果你直接从源码编译, 则需要加 \verb|-larb| 选项. 对一些编译器(例如 intel 的 \verb|icpc|), 可能还需要加上 \verb|-lflint, -lmpfr, -lgmp|, 如果 link 阶段遇到问题可以试试加上.

一个 C++ 例程: 该程序用 80 bit 初始精度计算复参数的超几何函数\upref{HypGeo} $_1F_1$ 并于 Mathematica 的结果比较. 如果 Arb 估计的结果精度小于 16 位有效数字则显示警告. 初始精度越高, 结果的有效数字也越高, 具体取决于 $_1F_1$ 的参数. 读者可以尝试在程序内加入一个循环, 若结果有效数字不够, 则把初始精度加倍再次计算, 直到达到精度为止.

\begin{lstlisting}[language=cpp, caption=test.cpp]
#include "arb_hypgeom.h"
#include "acb_hypgeom.h"
#include <complex>
#include <iostream>

using namespace std;

typedef double Doub, Doub_O;
typedef const double Doub_I;
typedef complex<double> Comp, Comp_O;
typedef complex<double> Comp_I;
typedef int Int;
typedef const int Int_I;
typedef int & Int_O, Int_IO;

// 1F1 hypdergeometric function with complex arguments
Comp hypergeom1F1(Comp_I a, Comp_I b, Comp_I z)
{
	slong prec = 80; // set precision bit (log10/log2 = 3.322)
	Comp res;
	arb_t temp1;
	arb_init(temp1);
	acb_t a1, b1, z1, res1;
	acb_init(a1); acb_init(b1); acb_init(z1); acb_init(res1);
	// error range is set to 0
	acb_set_d_d(a1, real(a), imag(a));
	acb_set_d_d(b1, real(b), imag(b));
	acb_set_d_d(z1, real(z), imag(z));
	acb_hypgeom_1f1(res1, a1, b1, z1, 0, prec);
	// acb_printn(res1, 100, 0); printf("\n"); // print result
	int digits = acb_rel_accuracy_bits(res1)/3.321928;
	if (digits < 16) {
		cout << "warning: hypergeom1F1 error too large" << endl;
	}
	acb_get_real(temp1, res1);
	res.real(arf_get_d(arb_midref(temp1), ARF_RND_NEAR));
	acb_get_imag(temp1, res1);
	res.imag(arf_get_d(arb_midref(temp1), ARF_RND_NEAR));
	acb_clear(a1); acb_clear(b1); acb_clear(z1);
    acb_clear(res1); arb_clear(temp1);
	return res;
}

int main()
{
	cout << "hypergeom1F1(1.23+1.23I, 1.23+1.23I, 1.23+1.23I) = " << endl;
	cout << hypergeom1F1(Comp(1.23,1.23),Comp(1.23,1.23),Comp(1.23,1.23))
    cout << endl;
	printf("Mathematica: 1.143503984180676 + 3.224470526790991i\n");
}
\end{lstlisting}
编译:
\begin{lstlisting}[language=makefile]
g++ -o test.x test.cpp -larb -lflint
\end{lstlisting}

确认版本: \verb|arb.h| 头文件中定义了版本宏
\begin{lstlisting}[language=cpp]
#define __ARB_VERSION 2
#define __ARB_VERSION_MINOR 23
\end{lstlisting}

\subsubsection{常用函数和常数}
数学常数(\verb|prec| 是结果的二进制精度, 根上面的 \verb|hypergeom1F1| 不同):
\begin{itemize}
\item \verb|void arb_const_pi(arb_t z, slong prec)| 圆周率
\item \verb|void arb_const_sqrt_pi(arb_t z, slong prec)| 根号圆周率
\item \verb|arb_const_log2|
\item \verb|void arb_const_e(arb_t z, slong prec)|
\end{itemize}


\subsection{内部实现}
\begin{itemize}
\item 要配合 gdb 学习可以自己编译一个 debug 版本的 libarb 库, 见 Automake 笔记\upref{automk}.
\item \verb|ARF_PREC_EXACT| 的定义是 \verb|std::numeric_limits<slong>::max()|
\end{itemize}

\pentry{FLINT 库笔记\upref{Flint}}
\subsubsection{arf\_t 的数据结构}
\begin{itemize}
\item \verb|arf_t| 是任意精度浮点数, \href{https://arblib.org/arf.html}{文档}.
\item \verb|arf_t| 的数据结构, 看懂下面的代码就了解了. 基本就是小数部分 \verb|.d|(例如 0.100101110, 小数点后面用一个 flint 大整数表示), 指数部分 \verb|.exp|(一个 flint 大整数), 和 \verb|.size| 部分(最后一 bit 是符号位, 1 表示复数, \verb|.size >> 1| 是用到的 limb 的个数, 这可能和 alloc 的 limb 的个数是不一样的).
\item 当 \verb|.size == 0| 时, \verb|arf_t| 是某个 \textbf{special} 值, 即: \verb|.exp == ARF_EXP_ZERO| 时, arf 为零; \verb|.exp == ARF_EXP_POS_INF| 时, arf 为正无穷; \verb|.exp == ARF_EXP_NEG_INF| 时, arf 为负无穷; \verb|.exp == ARF_EXP_NAN| 是, arf 为 nan; 如果不是 special, 就叫做 \textbf{normal}.
\item 相关函数: \verb|arf_is_special|, \verb|arf_is_zero|, \verb|arf_is_pos_inf|, \verb|arf_is_neg_inf|, \verb|arf_is_nan|, \verb|arf_is_normal|, \verb|arf_is_finite|, \verb|arf_is_inf|, \verb|arf_is_one|
\end{itemize}

\begin{lstlisting}[language=cpp]
#define ARF_NOPTR_LIMBS 2
#define ARF_NOPTR_D(x) ((x)->d.noptr.d) // 小数部分的 limb 指针(非 alloc)
#define ARF_PTR_D(x) ((x)->d.ptr.d) // 小数部分的 limb 指针(alloc)

typedef struct
{ // mp_limb_t 是 GMP 整数的一个 limb
    mp_limb_t d[ARF_NOPTR_LIMBS];
}
mantissa_noptr_struct; // arf 的小数部分至少有两个 limb

typedef struct
{
    mp_size_t alloc; // GMP limb 的 alloc 的个数, 并不一定全部用到.
    mp_ptr d; // mp_limb_t (GMP 整数的一个 limb)的指针
}
mantissa_ptr_struct;

typedef union
{
    mantissa_noptr_struct noptr;
    mantissa_ptr_struct ptr;
}
mantissa_struct;

typedef struct
{
    fmpz exp; // 指数(fmpz 是 flint 的任意精度整数)
    mp_size_t size; // 最后 1bit 是符号, >>1 是小数部分 limb 的数量
    mantissa_struct d; // 小数部分
	// 当 size <= ARF_NOPTR_LIMBS 时使用 d.noptr, 否则使用 d.ptr
}
arf_struct;

typedef arf_struct arf_t[1];
typedef arf_struct * arf_ptr;
typedef const arf_struct * arf_srcptr;

/* Raw size field (encodes both limb size and sign). */
#define ARF_XSIZE(x) ((x)->size)

// 获取用到的 limb 的个数
#define ARF_SIZE(x) (ARF_XSIZE(x) >> 1)
// 获取符号位, 1 为负数
#define ARF_SGNBIT(x) (ARF_XSIZE(x) & 1)

// x 是 arf_t, 获取其小数部分的 limb 指针和 limb 个数
// 当 limb 个数小于 ARF_NOPTR_LIMBS = 2 时, 不存在额外 alloc 的空间.
#define ARF_GET_MPN_READONLY(xptr, xn, x)   \
    do {                                    \
        xn = ARF_SIZE(x);                   \
        if (xn <= ARF_NOPTR_LIMBS)          \
            xptr = ARF_NOPTR_D(x);          \
        else                                \
            xptr = ARF_PTR_D(x);            \
    } while (0)


// get a quad precision number from arf_t type
// similar to arf_get_d()
inline void arf_get_q(Qdoub_O v, const arf_t x, arf_rnd_t rnd)
{
	arf_t t;
	// mp_limb_t is the type of a limb, with FLINT_BITS bits
	// typedef const mp_limb_t *mp_srcptr;
	mp_srcptr tp; // pointer to least significant limb
	mp_size_t tn; // # of limbs

	arf_init(t);
	arf_set_round(t, x, 113, rnd);
	ARF_GET_MPN_READONLY(tp, tn, t);
	if (tn == 1)
		v = (Qdoub)(tp[0]);
	else if (tn == 2)
	    // FLINT_BITS 是每个 limb 的 bit 数, 即 sizeof(mp_limb_t)*8
		v = (Qdoub)(tp[1]) + (Qdoub)(tp[0]) * ldexpq(1,-FLINT_BITS);
	else if (tn == 3)
		v = (Qdoub)(tp[2]) + (Qdoub)(tp[1]) * ldexpq(1,-FLINT_BITS)
		    + (Qdoub)(tp[0]) * ldexpq(1,-2*FLINT_BITS);
	else if (tn == 4)
		v = (Qdoub)(tp[3]) + (Qdoub)(tp[2]) * ldexpq(1,-FLINT_BITS)
		    + (Qdoub)(tp[1]) * ldexpq(1,-2*FLINT_BITS)
			+ (Qdoub)(tp[0]) * ldexpq(1,-3*FLINT_BITS);
	else
		SLS_ERR("not implemented!");

	v = ldexpq(v, ARF_EXP(t) - FLINT_BITS);

	if (ARF_SGNBIT(t)) // 1 for negative
		v = -v;
	arf_clear(t);
}
\end{lstlisting}

\subsubsection{mag\_t 的数据结构}
\begin{lstlisting}[language=cpp]
typedef struct
{
    fmpz exp; // 指数部分, flint 整数
    mp_limb_t man; // 小数部分, 单个 limb
}
mag_struct; // 用于表示误差半径, 非负浮点数

typedef mag_struct mag_t[1];
typedef mag_struct * mag_ptr;
typedef const mag_struct * mag_srcptr;

MAG_INLINE int
mag_is_inf(const mag_t x)
{
    return (MAG_MAN(x) == 0) && (MAG_EXP(x) != 0);
}
\end{lstlisting}

它表示的值为 \verb|x->man * 2^ (x->exp - MAG_BITS)| 其中 \verb|#define MAG_BITS 30|. % 已验证

\subsubsection{arb\_t 的数据结构}
\href{https://arblib.org/arb.html}{文档}.

\begin{lstlisting}[language=cpp]
typedef struct
{
    arf_struct mid; // 区间中点
    mag_struct rad; // 误差半径
}
arb_struct;

typedef arb_struct arb_t[1];
typedef arb_struct * arb_ptr;
typedef const arb_struct * arb_srcptr;
\end{lstlisting}

\subsubsection{acb\_t 数据结构}
就是两个 \verb|arb| 类型
\begin{lstlisting}[language=cpp]
typedef struct
{
    arb_struct real;
    arb_struct imag;
}
acb_struct;

typedef acb_struct acb_t[1];
typedef acb_struct * acb_ptr;
typedef const acb_struct * acb_srcptr;
\end{lstlisting}
