% 海森堡绘景

\begin{issues}
\issueDraft
\end{issues}

\pentry{薛定谔方程\upref{TDSE}}

\footnote{参考 Wikipedia \href{https://en.wikipedia.org/wiki/Heisenberg_picture}{相关页面}.}薛定谔方程\upref{TDSE}通常使用的是动量表象\upref{moTDSE}和\textbf{薛定谔绘景}, 在海森堡绘景中, 波函数(态矢)不随时间改变, 而测量量的算符随时间改变. 海森堡绘景相当于在薛定谔绘景的基础上做了一个基底变换, 类似于位置和动量表象\upref{moTDSE}的关系.

本文中,角标 $H$ 代表海森堡绘景, 角标 $S$ 代表薛定谔绘景. 例如波函数分别记为 $\psi_H(\bvec r, t)$ 和 $\psi_S(\bvec r)$, 后者不是时间的函数, 它的定义是
\begin{equation}
\psi_S(\bvec r) = \psi_H(\bvec r, 0)
\end{equation}

\addTODO{演化子是什么?链接}

使用演化子(propagator) $U(t)$, 波函数之间的关系为
\begin{equation}
\psi_H(\bvec r, t) = U(t) \psi_H(\bvec r, 0) = U(t) \psi_S(\bvec r)
\end{equation}
薛定谔方程在海森堡绘景中可以记为
\begin{equation}
H(U\psi_H) = \I\hbar \pdv{t} (U\psi_H)
\end{equation}
由于 $\psi_H$ 不含时, 两边抵消, 得演化子满足方程
\begin{equation}
H U(t) = \I\hbar \pdv{t} U(t)
\end{equation}

定义海森堡绘景中的算符为
\begin{equation}
Q_H(t) = U\Her(t) Q_S(t) U(t)
\end{equation}
对时间求导得
\begin{equation}
\dv{t}Q_H = \frac{\I}{\hbar} [H_H, Q_H(t)] + \qty(\pdv{Q_S}{t})_H
\end{equation}
平均值公式仍然和薛定谔绘景相同
\begin{equation}
\mel{\psi_H}{Q_H}{\psi_H} = \mel{\psi_S}{Q_S}{\psi_S}
\end{equation}
证明:
\begin{equation}
\mel{\psi_S}{Q_S}{\psi_S} = \mel{\psi_H}{U\Her(t)Q_SU(t)}{\psi_H} = \mel{\psi_H}{Q_H}{\psi_H}
\end{equation}
证毕.
