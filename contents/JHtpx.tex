% 几何拓扑学(综述)
% license CCBYSA3
% type Wiki

本文根据 CC-BY-SA 协议转载翻译自维基百科\href{https://en.wikipedia.org/wiki/Geometric_topology}{相关文章}

\begin{figure}[ht]
\centering
\includegraphics[width=6cm]{./figures/d945e3a478caa898.png}
\caption{由一组博罗米恩环所界定的赛弗特曲面;这些曲面可以作为几何拓扑中的研究工具。} \label{fig_JHtpx_1}
\end{figure}
在数学中,几何拓扑学研究的是流形及其之间的映射,尤其是一个流形嵌入到另一个流形中的情形。
\subsection{历史}
作为一个独立于代数拓扑的研究领域,几何拓扑学可以追溯到1935 年,当时通过Reidemeister 扭结对透镜空间进行了分类。这项工作首次要求区分那些同伦等价但不同胚的空间,也由此催生了简单同伦理论。“几何拓扑学”这一术语用来描述这一研究方向则是相对较近才出现的用法。\(^\text{[1]}\)
\subsection{低维拓扑与高维拓扑的区别}
流形在高维与低维中的行为存在显著差异。高维拓扑通常指维度 5 及以上的流形,或者从相对角度来看,指余维数3 及以上的嵌入问题。低维拓扑主要研究4 维及以下**的流形,或者余维数不超过 2 的嵌入问题。


四维流形具有特殊性:从某些角度(如拓扑结构)看,四维表现得像高维;而从其他角度(如可微结构)看,四维又表现得像低维。这种双重特性导致了许多四维独有的现象,例如 $\mathbb{R}^4$ 上的奇异可微结构。因此,四维流形的拓扑分类在理论上是可处理的,其核心问题包括:一个拓扑流形是否承认可微结构?如果承认,可能存在多少种不同的可微结构?值得注意的是,光滑四维情形仍然是广义庞加莱猜想的最后一个未解案例,可参见Gluck 扭转。

这种差异主要源于手术理论的适用范围:在五维及更高维度,手术理论适用;事实上,在某些情况下,它在四维的拓扑层面上也适用,但证明过程非常复杂。在四维及以下维度(拓扑意义上,三维及以下),手术理论则无法使用。

因此,研究低维流形时,一种常见的思路是:假设手术理论在低维情形下也成立,它会预测哪些结果?然后通过比较实际情况与这种预测的差异,来理解低维流形中那些偏离高维规律的特殊现象。
\begin{figure}[ht]
\centering
\includegraphics[width=8cm]{./figures/d454ad50d06c800e.png}
\caption{惠特尼技巧需要至少 2+1 个维度,因此手术理论需要 5 个维度才能适用。} \label{fig_JHtpx_2}
\end{figure}
造成5维与低维行为差异的精确原因在于:惠特尼嵌入定理—— 也是手术理论的关键技术 —— 需要至少2+1个维度才能发挥作用。简单来说,惠特尼技巧允许我们“解开”打结的球面(更准确地说,是去除浸入映射中的自交点)它的实现依赖以下步骤:借助一个2维的盘;通过一个增加1个维度的同伦对该盘进行变形。这样,在余维数大于2的情况下,就可以避免产生新的自交点,因此在余维数大于 2 的情形下,嵌入问题可以通过手术理论进行分析。在手术理论中,关键步骤发生在中间维度;当中间维度的余维数大于2(粗略说,超过 2.5 维即可,因此总维度至少5维即可),惠特尼技巧就能起作用。这一结论直接导出了Smale 的 h-同调定理,它适用于5维及更高维度,并构成了手术理论的基础。

在4 维中,惠特尼技巧的一个变形版本可以使用,这就是Casson把手。但由于维度不足,一个惠特尼盘会引入新的扭结,而这些扭结又需要通过新的惠特尼盘来消除,进而产生一个盘的序列(“塔”)。这个无限延伸的“塔”最终收敛到一个拓扑可行但不可微分的映射。因此,在4 维情况下,手术理论在拓扑意义上是成立的,但在可微意义上不成立。
\subsection{几何拓扑中的重要工具}
\subsubsection{基本群}
在所有维度中,流形的基本群都是一个极其重要的不变量,并决定了该流形的许多结构特性:
在1、2 和 3 维中,可能的基本群范围受到严格限制;而在4维及更高维度中,任何有限呈现群都可以作为某个流形的基本群。注意,只需证明 4 维或 5 维流形的情形成立,再将它们与球面相乘即可推广到更高维度。
\subsubsection{可定向性}
如果一个流形能够保持一致的方向选择,那么它就是可定向的;并且,一个连通的可定向流形恰好有两种不同的取向。在这一框架下,可定向性有多种等价表述,可以根据具体应用和泛化程度来选择:一般拓扑流形的可定向性,通常借助同调理论(的方法来描述;而在可微流形中,由于存在更丰富的结构,可以通过微分形式来表达。此外,可定向性的一个重要推广是纤维丛的可定向性:当一个空间族由另一个空间参数化时,需要在每个纤维空间中选择一个方向,并且这种方向选择需要随参数连续变化。
\subsubsection{柄分解}
\begin{figure}[ht]
\centering
\includegraphics[width=6cm]{./figures/a020f34992161e17.png}
\caption{一个附加了三个1-柄(1-handles)的三维球(3-ball)。} \label{fig_JHtpx_4}
\end{figure}
一个 $m$ 维流形 $M$ 的柄分解是一个嵌套序列:
$$
\emptyset = M_{-1} \subset M_0 \subset M_1 \subset M_2 \subset \cdots \subset M_{m-1} \subset M_m = M~
$$
其中,每个 $M_i$ 都是通过在 $M_{i-1}$ 上附加一个 $i$-柄($i$-handle)得到的。柄分解对于流形的作用类似于CW 分解对拓扑空间的作用。柄分解的意义在于提供一种与CW 复形类似但适用于光滑流形的语言。因此,一个$i$-柄就是一个 $i$-胞腔在光滑情形下的类比。柄分解在流形中自然出现,尤其是在莫尔斯理论 中。而柄结构的修改则与塞尔夫理论密切相关。
\subsubsection{局部平坦性}
局部平坦性是指嵌入在高维拓扑流形中的子流形所具备的一种性质。在拓扑流形的范畴下,局部平坦的子流形与光滑流形范畴中的嵌入子流形扮演了类似的角色。

假设一个 $d$ 维流形 $N$ 被嵌入到一个 $n$ 维流形 $M$ 中(其中 $d < n$)。如果 $x \in N$,我们说 $N$ 在 $x$ 处是局部平坦的,当且仅当存在一个邻域 $U \subset M$,使得拓扑对 $(U, U \cap N)$ 同胚于 $(\mathbb{R}^n, \mathbb{R}^d)$,其中 $\mathbb{R}^d$ 是标准地嵌入 $\mathbb{R}^n$ 的子空间。换句话说,存在一个同胚 $U \to \mathbb{R}^n$,使得 $U \cap N$ 的像正好对应于 $\mathbb{R}^d$。
\subsubsection{Schönflies 定}理
广义 Schönflies 定理表明:如果一个 $(n-1)$ 维球面 $S$ 被局部平坦地嵌入到 $n$ 维球面 $S^n$ 中(即该嵌入可以扩展成一个“加厚的球面”),那么这对 $(S^n, S)$ 同胚于 $(S^n, S^{n-1})$,其中 $S^{n-1}$ 是 $n$ 维球体的赤道。布朗和马祖尔因为各自独立证明了该定理而获得了Veblen 奖\(^\text{[2][3]}\)。
\subsection{几何拓扑的分支}
\subsubsection{低维拓扑}
低维拓扑包括以下内容:
\begin{itemize}
\item 曲面(2-流形,2-manifolds)
\item 3-流形(3-manifolds)
\item 4-流形(4-manifolds)
\end{itemize}
它们各自都有独立的理论,但彼此之间也存在一定的联系。

低维拓扑具有强烈的几何特征,这一点在以下事实中有所体现:二维情形:统一化定理,每个曲面都可以赋予一种常曲率度量,从几何角度来看,它必然属于以下三种几何之一:正曲率 / 球面几何,零曲率 / 平直几何,负曲率 / 双曲几何;三维情形:几何化猜想(geometrization conjecture,现已成为定理)每个三维流形都可以分解成若干部分,每一部分都有八种可能几何之一。

此外,二维拓扑可以通过单复变函数的复几何来研究(因为黎曼曲面是复曲线)。根据统一化定理,每一个共形类的度量都等价于唯一的复结构。四维拓扑则可以从二复变函数的复几何(复曲面,complex surfaces)的角度来研究,尽管并非所有四维流形都可以赋予复结构。
\subsubsection{结理论}
结理论研究的是数学中的“结”。虽然它的灵感来自日常生活中的鞋带或绳结,但数学中的结有一个关键区别:两端连接在一起,形成一个闭合曲线,因此无法像普通绳结那样解开。用数学语言来说,一个结就是三维欧几里得空间 $\mathbf{R}^3$ 中的一个圆的嵌入。由于研究的是拓扑性质,这里的“圆”并不是指传统几何意义上的圆,而是指所有与圆同胚的对象。如果两个数学结可以通过 $\mathbf{R}^3$ 中的同胚变形相互转化,就称它们是等价的。这种变形相当于不断调整结的形状,但不允许剪断绳子或让绳子穿过自身。

为了获得更深入的理解,数学家对“结”的概念进行了多种推广:可以在其他三维空间中考虑结;也可以使用不同于圆的对象;在高维情形下,所谓的“高维结”是指$n$ 维球面嵌入到 $m$ 维欧几里得空间中的情形。
\subsubsection{高维几何拓扑}
在高维拓扑中,和手术理论是两大核心工具。

\textbf{特征类}是一种将拓扑空间 $X$ 上的主丛与$X$ 的上同调类联系起来的方法。它刻画了丛的“扭曲程度”,特别是该丛是否存在全局截面。换句话说,特征类是一个全局不变量,用来衡量“局部乘积结构”和“全局乘积结构”之间的偏差。特征类是代数拓扑、微分几何和代数几何中的一个统一性概念。

\textbf{ 手术理论}是一系列用于“可控地”将一个流形转化为另一个流形的技术,由Milnor于 1961 年提出。“手术”指的是切除流形的一部分,并用另一流形的对应部分替换它,并确保切口或边界的匹配。手术理论与柄分解密切相关,但并不完全相同。它是研究和分类三维以上流形的重要工具。

更技术性地说,手术理论的核心思想是:从一个结构明确、性质已知的流形$M$出发,通过手术得到一个具有特定性质的新流形 $M'$,并且可以明确控制这一过程对流形的同调群、同伦群或其他不变量的影响。

1963 年,Kervaire和Milnor对奇异球的分类研究,使手术理论成为高维拓扑研究中的主要工具。
\subsection{参见}
\begin{itemize}
\item 分类:流形的映射
\item 几何拓扑主题列表
\item 管接法
\end{itemize}
\subsection{参考文献}
\begin{enumerate}
\item "What is geometric topology?". math.meta.stackexchange.com*. 2018 年 5 月 30 日检索。
\item Brown, Morton (1960). A proof of the generalized Schoenflies theorem. Bull. Amer. Math. Soc., vol. 66, pp. 74–76. MR 0117695.
\item Mazur, Barry (1959). On embeddings of spheres. Bull. Amer. Math. Soc., 65, 59–65. MR 0117693.
\end{enumerate}
\begin{itemize}
\item R. B. Sher and R. J. Daverman (2002). *Handbook of Geometric Topology. North-Holland. ISBN 0-444-82432-4.
\end{itemize}