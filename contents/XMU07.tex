% 厦门大学 2007 年 考研 量子力学
% license Usr
% type Note

\textbf{声明}:“该内容来源于网络公开资料,不保证真实性,如有侵权请联系管理员”

\subsection{一、(25 分)简述题(每小题5分)}
\begin{enumerate}
        \item 什么是叠加原理?
        \item 什么是基态?写出二维各向同性谐振子基态的能级表达式。
        \item 什么是厄米(Hermite)算符?动量算符是不是厄米算符?
        \item 粒子在中心力场中运动,问:$L_z$ 和 $p_y$ 是守恒量吗?为什么?
        其中 $L_z$ 、$p_y$ 分别为轨道角动量和动量在y方向的分量。
        \item "设力学量算符 $\hat{F}$ 和 $\hat{G}$ 能相互对易,若 $\psi$ 是 $\hat{F}$ 的特征态,则 $\psi$ 也是 $\hat{G}$ 的特征态。" 这句话对吗?试举一例说明。
    \end{enumerate}
\subsection{二、(25 分)}
设质量为m的一维自由粒子初始态为 $\psi(x,t)$ ,证明在足够长时间后
$$\psi(x,t) = \sqrt{\frac{m}{\hbar t}} e^{-i\frac{\pi}{4}} \exp \left[ \frac{imx^2}{2\hbar t} \right] \cdot \varphi \left( \frac{mx}{\hbar t} \right)~$$

$$\text{式中}\varphi(k) = \frac{1}{\sqrt{2\pi}} \int_{-\infty}^{+\infty} \psi(x, 0) e^{-ikx} dx \quad \text{是} \quad \psi(x, 0) \quad \text{的傅里叶}~$$
$$\text{(Fourier) 变换.[ 附:} \quad \lim_{a \to \infty} \sqrt{\frac{\alpha}{\pi}} e^{i \frac{\pi}{4}} \exp(-i \alpha x^2) = \delta(x)]~$$
\subsection{三、(25 分)}
求证在角动量$z$分量$\hat{L_z}$ 的本征态下 $\overline{L_z} = \overline{L_y} = 0.$
$$\overline{L_x L_y} = \frac{1}{2} m \hbar^2 i = - \overline{L_y L_x}.~$$
\subsection{四、(25分)}
对于一维谱振子,取基态试探波数形式为$e^{-\lambda x^2}$,$\lambda$为参数。用变分法求基态能量,并与严格解比较,
\subsection{五、(25分)}
设氢原子的状态是
$$\psi = \begin{pmatrix}\frac{1}{2} R_{21}(r)Y_{11}(\theta, \varphi) \\\\-\frac{\sqrt{3}}{2} R_{21}(r)Y_{10}(\theta, \varphi)\end{pmatrix}~$$

(1) 求轨道角动量 $z$ 分量 $\hat{L}_z$ 和自旋角动量$z$ 分量 $\hat{S}_z$ 的平均值。

(2) 求总磁矩 $\hat{\vec M} = -\frac{e}{2\mu} \hat{L} - \frac{e}{\mu} \hat{\vec S}$ 的 $z$ 分量的平均值 (用波尔磁矩表示)。
\subsection{六、(25分)}
质量为 $m$ 的粒子在一维势场 $V(x)$ 中运动,能级为 $E_n^{(0)}$, $n = 1, 2, 3, \ldots$

如受到微扰 $H' = \frac{\lambda}{m} p$ 作用 ($\lambda$ 为常数,$p$ 是 $p_x$ 的简写),求能级修正。

(准确到二级近似)

[提示] $$\quad \sum_k \left( E_k^{(0)} - E_n^{(0)} \right) \left| x_{kn} \right|^2 = \frac{\hbar^2}{2m}.~$$