% C 和 C++ 性能优化笔记
% license Xiao
% type Note

\begin{issues}
\issueDraft
\end{issues}

\pentry{C++ 基础\nref{nod_Cpp0}}{nod_9a4c}

\subsection{跑分!不要微优化!}
\begin{itemize}
\item 除非你在写很多人用的高性能库, 否则不要微优化。
\item 先做跑分, 找到性能瓶颈。
\item 通常 90\% 的性能提升空间都集中于 10\% 以下的代码。
\end{itemize}


\subsection{内存管理}
\begin{itemize}
\item 使用 workspace 减少内存分配, 参考\upref{Lapack} 的做法。 如果内存占用不是问题, 甚至可以把所有内存分配在程序一开始进行。
\item 内存池的概念可以了解一下。
\end{itemize}

\subsection{g++ 和 clang++ 浮点数优化选项}
\pentry{双精度和四精度浮点数笔记(C++)\nref{nod_FltCpp}}{nod_2dea}
\begin{itemize}
\item \verb`-ffast-math`: enable all of the following
\item \verb`-fno-math-errno`: Disable setting "errno" after calling math functions.
\item \verb`-funsafe-math-optimizations`: Allow optimizations that may violate strict IEEE compliance.
\item \verb`-fno-signed-zeros`: Ignore the signedness of zero, treating both positive and negative zeros as the same.
\item \verb`-fno-trapping-math`: Assume that floating-point exceptions are not enabled and that no traps will occur.
\item \verb`-ffinite-math-only`: Assume that arguments and results of mathematical functions are finite, i.e., no NaNs or infinities.
\end{itemize}
