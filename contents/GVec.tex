% 几何向量
% keys 线性代数|几何向量|单位向量|标量|平行四边形法则|线性组合|线性相关|线性无关|基底|向量空间|坐标
% license Xiao
% type Tutor

\pentry{集合(高中)\nref{nod_HsSet}}{nod_919d}

\textbf{向量(vector)}又被翻译为\textbf{矢量}。这个概念最初来自直观的几何向量。早在民国时期,学者们引入vector的概念时就将它译成了汉语。 由于当时的物理学家和数学家没有太多交集, 物理学家将它翻译为向量, 而数学家翻译为矢量。 90 年代时, 国家名词委员会商议确定一个统一的 vector 译名, 但却无法轻易割舍两个译名中的任何一个, 因为它们都非常信达雅地表明了vector 的含义:向量即有方向的量, 矢量即像箭矢一样的量。 大概是出自物理学家和数学家的互相尊重, 最后确定的方案是双方互换译名, 从此物理学界称矢量, 数学界称向量\footnote{以下加粗部分为力学家朱照宣教授的回忆。\textbf{在20世纪90年代初,国家名词委为此(vector)召开会议,想协调双方,由主任钱三强亲自主持。 我曾戏称这是个 “一字会”。 当时的情况是, 学科有分支, 术语有派生, 犹如家族有后裔。 祖宗互相谦让, 但子孙繁多, 已无法协调。 钱先生在会上没有说倾向于哪方面的话。 矢量、向量的分歧,一直维持到今。 力学这学科, 和数学、物理同样有 “亲”, 力学中 vector 用什么? 当年我在 “一字会” 后还有情绪, 埋怨钱先生作为领导“不表态”。 过了好些年,才懂得这类事,最多只能因势利导, 不能靠行政命令或专家拍板。 事实上, 台湾物理界至今用的还是 “向量”。}}。 \textbf{小时百科中不区分两个译名的使用}。

% Giacomo:要不要改一下最后这句话?

\subsection{几何向量}

我们来回顾高中学的\textbf{几何向量(geometric vector)}, 也叫\textbf{欧几里得向量(Euclidean vector)}。 粗略地说,向量是空间中的一些有长度有方向的箭头。 我们对它的位置不感兴趣, 所有长度和方向相同的几何向量都视为同一几何向量。 小时百科中几何向量用正黑体表示, 如 $\bvec{a}$。 在手写时, 可以在字母上方加箭头表示, 如 $\overrightarrow{a}$。 特殊地, 如果一个几何向量的长度等于 1, 那么它就是一个\textbf{单位向量(unit vector)}, 小时百科中在几何向量上面加上 “\^{}” 符号表示单位向量, 如 $\uvec a$。 为了与向量区分, 我们把单个的实数或复数称为\textbf{标量(scalar)}。

\begin{figure}[ht]
\centering
\includegraphics[width=5cm]{./figures/952d5a60949289d0.pdf}
\caption{几何向量是有长度和方向的量} \label{fig_GVec_4}
\end{figure}
本文中讨论的几何向量是经典物理学中最常见的一类向量, 是一种具有\textbf{长度}和\textbf{方向}的量, 因此可以画成箭头来表示。生活中这样具有长度和方向的量十分常见: 速度有大小有方向,因此可以表示为箭头,箭头的长度代表速度的大小;\enref{加速度}{VnA}也有大小,有方向; 一个物体(看成一点)从一个地方运动到另一个地方,那么从起点到终点可以画一根箭头, 这箭头就是\enref{位移向量}{Disp}。不止在物理学中, 一切领域里具有方向和大小概念的量都被称为几何向量。 

为什么要强调是“几何”向量呢?在数学中,向量的含义要比几何向量更广泛,也就是说,几何向量只是数学家所研究的向量的一种,并不是唯一的一种, 广义来说任何\enref{向量空间}{LSpace}中的元素都叫做向量。\footnote{以后会看到, 本文介绍的向量是一个 “实数域上的二维/三维内积线性空间” 中的元素;这个名字听起来很吓人,但它其实只是三个概念的结合:实数域,内积和二维/三维线性空间。实数域即我们所熟悉的实数,它作为一个域使得我们可以用它来构建线性空间,而内积则是线性空间上向量的“长度和夹角”的推广。这些概念我们都会在对应文章中深入讨论。}

\begin{definition}{几何向量的长度(模长)}
对于任何一个几何向量 $\bvec{a}$,我们把它的长度记做 $|\bvec{a}|$ (绝对值的记号)。对于更一般的向量我们把“长度”称为\textbf{模长},记做 $\|\bvec a\|$ (范数的记号)\footnote{这是因为一般的向量不一定像几何向量一样是一段线段,“长度”没有意义。}。
\end{definition}


\begin{definition}{几何向量间的夹角 - 方向}\label{def_GVec_1}
对于任何两个几何向量 $\bvec{a}, \bvec{b}$,我们把它们的\textbf{夹角}记做 $\angle \bvec{a} \bvec{b}$ (或者 $\angle \bvec{a} O \bvec{b}$,$O$是它们的共同起点),它是一个 $0^{\circ}$ 到 $180^{\circ}$ (角度制),或者 $0 \opn{rad}$ 到 $2 \pi \opn{rad}$ (弧度制)之间的一个值(两个端点可以取到),我们有 $\angle \bvec{a} \bvec{b} = \angle \bvec{b} \bvec{a}$。

如果 $\angle \bvec{a} \bvec{b}$ 为零角,我们就称 $\bvec{a}, \bvec{b}$ \textbf{同向};如果 $\angle \bvec{a} \bvec{b}$ 为平角,我们就称 $\bvec{a}, \bvec{b}$ \textbf{反向};同向和反向统称\textbf{相互平行/共线};如果 $\angle \bvec{a} \bvec{b}$ 为直角,我们就称 $\bvec{a}, \bvec{b}$ \textbf{相互垂直}。

另一方面,对于平面向量,我们也有 $\bvec{a}$ 到 $\bvec{b}$ 的\textbf{有向夹角} (同样记做 $\angle \bvec{a} \bvec{b}$,要根据上下文判断是否有向),我们取逆时针方向为正方向(参考\enref{角与有向角(高中)}{HsAngl})。我们有 $\angle \bvec{a} \bvec{b} = - \angle \bvec{b} \bvec{a} + 2 k \pi$。
\end{definition}

\begin{definition}{零向量}
特别地,长度为零的几何向量称为\textbf{零向量(zero vector)},记做 $\bvec{0}$,零向量没有方向之分,我们规定\textbf{任何方向的零向量都是同一个向量},我们规定\textbf{它与任何几何向量平行}。
\end{definition}
零向量依然是向量, 要注意和数字零 $0$ 进行区分。

\begin{theorem}{}
非零几何向量之间的共线关系是一个等价关系,即
\begin{enumerate}
\item $\bvec{A}$ 和它本身是共线的;
\item 共线关系是相互的(对称的)—— $\bvec{A}$ 和 $\bvec{B}$ 共线等价于 $\bvec{B}$ 和 $\bvec{A}$ 共线;
\item $\bvec{A}$ 和 $\bvec{B}$ 共线,同时 $\bvec{B}$ 和 $\bvec{C}$ 共线,我们有 $\bvec{A}$ 和 $\bvec{C}$ 共线。
\end{enumerate}
\end{theorem}
\addTODO{拓展链接到“等价关系”}


\subsubsection{几何向量与起点无关}

需要注意的是,为了避免引入过多的要素从而导致概念过于复杂,我们目前讨论的几何向量只有两个本质属性,长度和方向。将几何向量进行任意平移后, 它仍然是原来那个几何向量\footnote{在微分几何中,我们讨论一种叫“切向量”的对象,这就是一种和所在位置有关的抽象的向量。},这也是为什么我们说平行关系就是共线关系。

因为,由于几何向量平移并不改变几何向量本身,我们完全可以把所有几何向量的起点都挪到一起,以这个公共起点为\textbf{原点},那么空间中\textbf{向量的终点所在点}就和\textbf{向量本身}一一对应。 当把所有向量的起点都挪到原点后,我们就可以用终点的几何坐标来描述几何向量。

这可能会带来一个问题:例如在经典力学中, 力是一个向量 $\bvec F$(与起点无关), 但力在物体上的作用效果和力的作用点(这通常画成箭头的起点)相关。 这似乎和 “平移不变” 矛盾。实际上,在完整描述一个力对物体的作用时,我们使用两个向量,一个是力的向量(只包含力的大小和方向),一个则是力的作用点的\enref{位置向量}{Disp} $\bvec r$,专门指明作用点的位置。
\begin{figure}[ht]
\centering
\includegraphics[width=8cm]{./figures/41cac5e19d7f5bd4.pdf}
\caption{由于力的效果与作用点有关,因此需要一个额外的位置向量来指明力的作用点} \label{fig_GVec_5}
\end{figure}

\subsection{坐标和维度}
从现在开始,如无特别说明,我们默认几何向量起点在原点。这样,我们就可以把一个向量理解为空间中的一个点,即它的终点。本节仅从直觉上引入坐标的概念,基本上是符合多数人的几何直觉的。要更准确地描述相关概念,请参考“\enref{基底和坐标}{Gvec2}”。

我们知道,实数轴可以看成一个一维的几何空间。如果以数字 $0$ 所在的点为原点,那么我们也可以把这根轴本身看成一个几何向量的集合,每个数字所在的点都是一个几何向量。如\autoref{fig_GVec_1} 所示,点 $P$ 表示一个长度为 $3.14$ 的向量,而点 $Q$ 表示一个长度为 $6$ 的向量。这样一来,每个实数 $x$ 都可以表示一个向量,其长度为 $\abs{x}$;当 $x>0$ 时,对应的向量指向正方向,当 $x<0$ 时指向负方向,而对于 $x=0$,它对应的是零向量。

\begin{figure}[ht]
\centering
\includegraphics[width=11cm]{./figures/85f050446702332b.pdf}
\caption{实数轴上的几何向量。} \label{fig_GVec_1}
\end{figure}

实数轴上的向量只有一个方向可以选择,虽然有正负之分,但都沿着一条线。因此只需要用一个实数就可以唯一地表示一个向量。这样的实数,就是对应向量在实数轴上的一个坐标,该实数的绝对值就是向量的“长度”。


这里所说的向量的长度和日常经验可能有所不同。当我们讨论一个长度为 $1\Si{m}$ 的位移向量时,我们可以有不同的坐标来描述这一个向量,或者说,可以把 $1\Si{m}$ 对应到不同的实数。如果我们用每格长 $1\Si{m}$ 的坐标轴去度量,那么这个向量的“长度”就是 $1$;如果用每格长 $1\Si{mm}$ 的坐标轴去度量,那么其“长度”就是 $1000$。因此我们所说的长度并不具有绝对的意义,而只有相对的意义\footnote{严格来说,这是因为对于同一个向量空间,我们可以赋予不同的范数/\enref{内积}{InerPd}。}:同一个向量的长度在不同坐标表示下可能不一样,但是无论在什么坐标表示下,两条“共线”的向量的长度之比都是一样的。


实数轴上的向量没有方向区分(最多是正负之分),因此不用考虑方向,方便我们专心讨论长度问题。实数轴被称为一个\textbf{一维}的几何向量空间,因为所有向量只需要\textbf{一个数字}就能\textbf{唯一}确定。

是否存在需要更多数字才能确定一个向量的空间呢?当然有,二维平面就是这样空间,其上的向量就需要两个数字来描述。这个时候向量就有方向之分了,如\autoref{fig_GVec_2} 中向量 $P$ 的方向“北偏东 $45^\circ$”和 $Q$ 的方向“正东”就有所区别了。

\begin{figure}[ht]
\centering
\includegraphics[width=5cm]{./figures/91b42d4f6322d1b8.pdf}
\caption{二维平面上的几何向量。} \label{fig_GVec_2}
\end{figure}

因为向量是“具有长度和方向”的量,我们可以用一个数字来表示向量的长度,一个数字来表示向量的方向(从给定轴算起逆时针旋转的角度,即辐角),这就是\enref{极坐标}{Polar}的表示方法。当然,我们更熟悉的是用直角坐标系来表示,如\autoref{fig_GVec_2} 所示,如果向量 $P$ 和 $Q$ 的长度都是 $1$,但 $P$ 的辐角为 $\pi/4$,$Q$ 的辐角为 $0$,那么直角坐标系下就可以把 $P$ 表示为 $(\sqrt{2}/2, \sqrt{2}/2)$,把 $Q$ 表示为 $(1, 0)$。

同理,三维空间中的向量需要用三个数字来确定。

\begin{figure}[ht]
\centering
\includegraphics[width=7cm]{./figures/0eebf3233146a327.pdf}
\caption{三维空间里的几何向量。} \label{fig_GVec_3}
\end{figure}

向量的坐标表示,不一定直接体现了向量的两个要素:“方向”和“长度”,但是一定蕴含了这两个要素。比如说,\autoref{fig_GVec_3} 中向量 $P$ 的长度并没有直接出现在其坐标 $(a_x, a_y, a_z)$ 中,但是可以用坐标计算出来:$\sqrt{a_x^2+a_y^2+a_z^2}$。

我们容易想象出 $1$ 到 $3$ 维的几何向量空间以及它们的两种运算, 但却很难想象更高维的情况。作为本章的主要目的,当你理解了本章的所有甚至只是部分文章后,应当可以熟悉该如何讨论任意维度的空间。讨论的核心在于,$n$ 维空间中的每个向量都需要 $n$ 个数字来确定,我们可以“定义”一个直角坐标系,把 $3$ 维直角坐标系中的性质推广开来。

\subsubsection{拓展:斜坐标系}

当笛卡尔首次提出“坐标”的概念时,他并没有指定必须是“直角坐标”。实际上,他认为任何两根不平行的直线都可以用来刻画二维平面上的坐标,具体方式详见\enref{斜坐标系}{ObSys}。不只是二维平面,任意维度的空间中都可以使用斜坐标系来描述空间中的点。

使用直角坐标系的好处是高度对称,因为当你选定相互垂直的 $x$ 轴和 $y$ 轴以后,$y$ 轴的两边到 $x$ 轴的角度都是 $\pi/2$;另外,几何学中的勾股定理依赖于直角;虽然我们也可以用余弦定理% \addTODO{链接}
来描述任意三角形的边长关系,但比起勾股定理要麻烦得多,因为表达式中多了一项。使用直角坐标系能让讨论简洁很多。

\subsection{参考书推荐}
对于几何向量在高中数学以及几何学中的用途,张景中院士的《绕来绕去的向量法》一书的内容通俗易懂且内容丰富, 单墫的《向量与立体几何(数学奥林匹克命题人讲座)》也是一本实用的小册子。 感兴趣的读者可用这两本读物作参考。 另外,数学界也借用物理中“质心”的概念,发展出了 “质点几何学” 分支, 其本质仍然和向量几何一模一样, 只不过换了个观点看问题,实质上是仿射几何的一种优化版表述; 在上述张景中院士的书中提到了质点几何学, 莫绍揆教授也出版了专门讨论此话题的《质点几何学》,但并不建议读者特意花太多精力学习\footnote{如果你好奇的话,只需要知道向量法中的\textbf{定比分点公式}就是连接质点几何和向量几何的桥梁,就可以把几何问题在这两个观点之间互相转化了。}。
