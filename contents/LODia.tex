% 线性算子对角化的充要条件
% keys 对角化|充要条件
% license Xiao
% type Tutor
\pentry{本征矢量与本征多项式\upref{EigVM}}{nod_ff1e}
\begin{definition}{}
 $n$ 维矢量空间 $V$ 中,若有一基底,使得在该基底下,线性算子 $\mathcal{A}$ 对应的矩阵 $A$ 取对角形式
 \begin{equation}\label{eq_LODia_3}
 A=\begin{pmatrix}
 \lambda_1&0&\cdots&0\\
 0&\lambda_2&\cdots&0\\
 \vdots&\vdots&\cdots&\vdots\\
 0&0&\cdots&\lambda_n
 \end{pmatrix}~,
 \end{equation}
 则称算子 $\mathcal{A}$ 是\textbf{可对角化}的。\autoref{eq_LODia_3} 通常又记为 $A=\mathrm{diag}(\lambda_1,\cdots,\lambda_n)$。
\end{definition}
\begin{theorem}{线性算子可对角化的充要条件}\label{the_LODia_1}
定义在域 $\mathbb{F}$ 上的 $n$ 维矢量空间,其上的线性算子 $\mathcal{A}$ 可对角化的充要条件为:$\mathcal{A}$ 的本征多项式\upref{EigVM} $\mathrm{det}{\mathcal{A}-t \mathcal{E}}$ 的 所有根都在 $\mathbb{F}$ 上,且每个本征值 $\lambda$ 的几何重数等于代数重数\upref{EigVM}。
\end{theorem}
\subsection{证明}
先引入一个引理
\begin{lemma}{}\label{lem_LODia_1}
属于不同本征值的特征矢量必然线性无关,且 $\sum\limits_{i=1}^nV^{\lambda_i}$ 是直和(各 $\lambda_i$ 不相同)。
\end{lemma}
\textbf{证明:}对每个 $V^{\lambda_i}$ 选取一个本征矢量 $\bvec{e_i}$ ,现证明它们线性无关。

当 $m=1$ 时显然。对 $m$ 用数学归纳法,设存在线性关系式
\begin{equation}\label{eq_LODia_1}
\sum_{i=1}^{m}\alpha_i\bvec e_i=\bvec 0~,
\end{equation}
对某一系数 $\alpha_i$ 不为0,不失一般性,设 $\alpha_1\neq0$。两边作用 $\mathcal A$:
\begin{equation}\label{eq_LODia_2}
\sum_{i=1}^m\alpha_i\lambda_i\bvec e_i=\bvec 0~.
\end{equation}
$\lambda_m\times$\autoref{eq_LODia_1} -\autoref{eq_LODia_2} :
\begin{equation}
\sum_{i=1}^{m-1}\alpha_i(\lambda_m-\lambda_i)\bvec e_i=\bvec 0~.
\end{equation}
由归纳假定 $\alpha_i(\lambda_m-\lambda_i)=0,\;(i=1\cdots m-1)$。 这和
\begin{equation}
\alpha_1\ne0,\;\lambda_m\neq\lambda_i,\;i<m\Rightarrow \alpha_1(\lambda
_m-\lambda_1)\neq0~
\end{equation}
矛盾。于是线性无关性得证!

由本征子空间 $V^\lambda$ 的定义(\autoref{def_EigVM_1}~\upref{EigVM}),任意非零矢量 $\bvec e_i\in V^{\lambda _i}$ 都是本征矢量。所以
\begin{equation}
V^{\lambda_i}\cap\sum_{j\neq i}V^{\lambda_j}=\bvec 0~,
\end{equation}
即  $\sum\limits_{i=1}^nV^{\lambda_i}$ 是直和。

\textbf{证毕!}
\subsubsection{\autoref{the_LODia_1} 的证明}
\textbf{充分性}:设$\lambda_i\;(i=1,\cdots, m)$ 是不同本征值。由于本征多项式所有根都在 $\mathbb{F}$ 内(意味着所有本征值代数重数之和=$n$),且代数重数且每一本征值代数重数和几何重数相同,所以
\begin{equation}
\sum_{i=1}^m \mathrm{dim}\;V^{\lambda_i}=n=\mathrm{dim}\; V~,
\end{equation}
于是
\begin{equation}
V=\bigoplus_{i=1}^m V^{\lambda_i}~.
\end{equation}
\begin{issues}
直和词条部分应证明上两式等价并链接
\end{issues}
于是把 $V^{\lambda_i}$ 的基底合并就得到 $V$ 的一个基底。在此基底下 $\mathcal{A}$ 是对角化的\autoref{the_InvSP_1}~\upref{InvSP}。

\textbf{必要性:}设$\lambda_i\;(i=1,\cdots, m)$ 是 $\mathcal A$ 的不同本征值。由于 $\mathcal A$ 是对角化的,所以
\begin{equation}
A=\mathrm{diag}\;(\underbrace{\lambda_1,\cdots,\lambda_1}_{\mathrm{dim}\;V^{\lambda_1}};\cdots;\underbrace{\lambda_m,\cdots,\lambda_m}_{\mathrm{dim}\;V^{\lambda_m}})~.
\end{equation}
因为矢量空间在域 $\mathbb{F}$ 上,所以矩阵 $A$ 的元素 $\lambda_i\in \mathbb{F}$。
于是
\begin{equation}
\mathrm{det}(\mathcal{A}-t\mathcal E)=(\lambda_1-t)^{\mathrm{dim}\;V^{\lambda_1}}\cdots(\lambda_m-t)^{\mathrm{dim}\;V^{\lambda_m}}~.
\end{equation}
上式的解释即要证的结论。

\textbf{证毕!}
