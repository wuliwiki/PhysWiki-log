% 离散型随机变量(高中)
% keys 高中|离散型随机变量
% license Usr
% type Tutor

\pentry{组合(高中)\upref{HsCb}}{nod_fdd9}

\subsection{定义}
试验结果可以用一个变量 $X$ 来表示,并且 $X$ 是随着试验的结果的不同变化的,我们把这样的变量 $X$ 叫做一个\textbf{随机变量(random variable)}。随机变量常用大写字母 $X,Y,\cdots$ 表示。

如果随机变量  $X$ 的所有可能的取值都能一一列举出来,则称 $X$ 为\textbf{离散型随机变量(discrete random variable)}。

\subsection{离散型随机变量的分布列}
要掌握一个离散型随机变量 $X$ 的取值规律,必须知道:
\begin{enumerate}
\item $X$ 所有可能取的值 $x_1$,$x_2$,$\cdots$ ,$x_n$ 。
\item $X$ 取每一个值 $x_i$ 的概率 $p_1$,$p_2$,$\cdots$,$p_n$。
\end{enumerate}
这就是说,需要列出下表:

\begin{table}[ht]
\centering
\caption{分布列}\label{tab_HsDRV_1}
\begin{tabular}{|c|c|c|c|c|c|c|}
\hline
$X$ & $x_1$ & $x_2$ & $\cdots$ & $x_i$ & $\cdots$ & $x_n$ \\
\hline
$P$ & $p_1$ & $p_2$ & $\cdots$ & $p_i$ & $\cdots$ & $p_n$ \\
\hline
\end{tabular}
\end{table}
我们称这个表为离散型随机变量 $X$ 的\textbf{概率分布(probability distribution)},或称为离散型随机变量 $X$ 的\textbf{分布列(distribution series)}。

如果随机变量 $X$ 的分布列为

\begin{table}[ht]
\centering
\caption{二点分布}\label{tab_HsDRV_2}
\begin{tabular}{|c|c|c|}
\hline
$X$ & $1$ & $0$ \\
\hline
$P$ & $p$ & $q$ \\
\hline
\end{tabular}
\end{table}
其中 $0<p<1,q=1-p$ ,则称离散型随机变量 $X$ 服从参数 $p$ 的\textbf{二点分布}。

\subsection{超几何分布}
一般地,设总数为 $N$ 件的两类物品,其中一类有 $M$ 件,从所有物品中任取 $n$ 件 $(n\leqslant N)$,这 $n$ 件中所含这类物品件数 $X$ 是一个离散型随机变量,它取值为 $m$ 时的概率为
\begin{equation}
P(X=m) = \frac{C_M^mC_{N-M}^{n-m}}{C_N^n}(0\leqslant m\leqslant l)~.
\end{equation}
\textsl{注:$l$ 为 $n$ 和 $M$ 中较小的一个。}

我们称离散型随机变量 $X$ 的这种形式的概率分布为\textbf{超几何分布},也称 $X$ 服从参数为 $N,M,n$ 的超几何分布。
