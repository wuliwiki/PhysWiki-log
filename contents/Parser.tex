% 编译器/解释器简介、Bison 笔记
% license Xiao
% type Note

\begin{issues}
\issueDraft
\end{issues}

\begin{itemize}
\item 参考\href{https://www.toptal.com/scala/writing-an-interpreter}{这篇文章},还有\href{https://www.youtube.com/watch?v=Xu4RtLlm42I}{这个}
\item lexer
\item parser
\item interpreter
\item abstract syntax tree (AST)
\item domain-specific languages (DSLs)
\item 一个纯手写 c 编译器\href{https://norasandler.com/2017/11/29/Write-a-Compiler.html}{教程}。
\item baltam 用的是 \href{https://www.gnu.org/software/bison/}{bison} 和 \href{https://www.genivia.com/doc/reflex/html/}{RE/flex}
\item 用 bison 实现一个简单科学计算器(包括变量赋值和基本函数) 的例子 \href{http://web.mit.edu/gnu/doc/html/bison_5.html}{Multi-Function Calculator: mfcalc}。
\item 笔者自己的简陋 Matlab 实现: 见\href{https://github.com/MacroUniverse/bison_test}{这里}。
\item \href{https://en.wikipedia.org/wiki/LR_parser}{LR parser}, \href{https://en.wikipedia.org/wiki/LALR_parser}{LALR parser}, \href{https://en.wikipedia.org/wiki/Recursive_descent_parser}{Recursive Descent parser} (G++ 和 Clang 都用是手写的 Recursive Descent), \href{https://en.wikipedia.org/wiki/GLR_parser}{GLR parser} (例子: \href{http://www.scottmcpeak.com/elkhound/}{Elsa C++ parser} 和 \href{http://www.semanticdesigns.com/Products/FrontEnds/CppFrontEnd.html}{C++ Parser Front End})
\item 上一条参考 C/C++ parser 的一个\href{https://stackoverflow.com/questions/6319086/are-gcc-and-clang-parsers-really-handwritten}{Stack Overflow 问题}。
\item 然而 TeX 的编译器要复杂得多(如果要实现全部功能)见\href{https://groups.google.com/g/comp.text.tex/c/E1736iEOxNI}{这个讨论}。 现成的工具参考\href{https://tex.stackexchange.com/questions/39309/convert-latex-to-html}{这个}。
\end{itemize}

\subsection{Bison 笔记}
\begin{itemize}
\item 安装: \verb|apt install bison|
\item 使用: \verb|bison xxx.y|
\item 编译 \verb|gcc xxx.tab.c -l xxx -o xxx|
\item \href{http://web.mit.edu/gnu/doc/html/bison_5.html}{官方例子}源文件见笔者的 \href{https://github.com/MacroUniverse/bison_test}{GitHub}。
\end{itemize}

\subsubsection{mfcalc 笔记}
\begin{itemize}
\item symbol 就是 VAR, FUNC
\end{itemize}
