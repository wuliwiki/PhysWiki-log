% 匀晶相图
\footnote{本文参考了Callister的Material Science and Engineering An Introduction,刘智恩的《材料科学基础》与Gaskell的Introduction to the Thermodynamics of Materials}

\subsection{匀晶转变}
\pentry{相图(未完成)}
\begin{figure}[ht]
\centering
\includegraphics[width=14cm]{./figures/ISOMOR_1.png}
\caption{匀晶转变} \label{ISOMOR_fig1}
\end{figure}

匀晶转变(\autoref{ISOMOR_fig1} 中b-d段)指“由单一液相直接生成单一固相 l→s”的过程.以下以Ni-Cu合金的平衡冷却为例,介绍匀晶转变的特点.

\begin{itemize}
\item 匀晶转变不是恒成分转变.当α相刚开始生成时,α相中高熔点组分的浓度高于液相;随着温度降低、α相不断生成,该组分的浓度逐渐降低.例如,Ni熔点高于Cu,因此新生成的相α中Ni的浓度更高
\begin{figure}[ht]
\centering
\includegraphics[width=7cm]{./figures/ISOMOR_2.png}
\caption{两相中Ni浓度的变化示意图} \label{ISOMOR_fig2}
\end{figure}

\item 匀晶转变不是恒温转变.相转变时,系统的自由度f=2-2+1=1,因此相转变时,温度可以在一定范围内变化.
\end{itemize}

\subsection{热力学}
\pentry{化学势(未完成),相变平衡条件\upref{PhEquv}}
%好久没摸这块了,希望公式没打错(
在匀晶转变的过程中,在固相与液相中的Cu,Ni的化学势分别相等.假定二者均满足理想条件,由化学势相同,所以得
\begin{align}
&\mu_{Ni,l,mixed}=\mu_{Ni,s,mixed}\\
&\mu_{Cu,l,mixed}=\mu_{Cu,s,mixed}\\
\end{align}
根据化学势的相关结论,对于Ni,有
$$
\mu_{Ni,l}^*+RT \ln x_{Ni,l}=\mu_{Ni,s}^*+RT \ln x_{Ni,s}=\mu_{Ni,l}^*+\Delta G_{Ni, l\rightarrow s} + RT \ln x_{Ni,s}
$$
即
$$
RT \ln x_{Ni,l}=\Delta G_{Ni, l\rightarrow s} + RT \ln x_{Ni,s}
$$
即
\begin{equation}
x_{Ni,l}=x_{Ni,s}e^{\frac{\Delta G_{Ni, l\rightarrow s}}{RT}}
\end{equation}
同理,对于Cu,有
\begin{equation}
x_{Cu,l}=x_{Cu,s}e^{\frac{\Delta G_{Cu, l\rightarrow s}}{RT}}
\end{equation}
再根据
\begin{align}
&x_{Cu,l}+x_{Ni,l}=1\\
&x_{Cu,s}+x_{Ni,s}=1\\
\end{align}
原则上可解方程.
