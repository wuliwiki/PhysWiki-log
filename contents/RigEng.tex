% 刚体定轴转动的力矩做功、动能、动能定理
% license Usr
% type Tutor

\begin{issues}
\issueDraft
\issueOther{可以与\enref{刚体的动能、动能定理}{RBKE}整合}
\issueOther{可以与\enref{刚体定轴转动、转动惯量}{RigRot}整合}
\end{issues}

\pentry{动能 动能定理(单个质点)\nref{nod_KELaw1}, 刚体定轴转动 转动惯量\nref{nod_RigRot}}{nod_10bc}

\subsection{定轴转动的动能、动能定理}
先以定轴转动的平面圆盘为例,计算刚体定轴转动的动能。
\begin{figure}[ht]
\centering
\includegraphics[width=8cm]{./figures/c72a94be06f99341.pdf}
\caption{定轴转动的圆盘} \label{fig_RigEng_1}
\end{figure}
假设该圆盘由若干小块组成,则系统的总动能为各质点的动能之和:
\begin{equation}
E_k=\sum E_{k,i}=\sum \frac{1}{2} \Delta m v_i^2=\frac{1}{2} \sum \Delta m (\omega r_i)^2=\frac{1}{2} \omega^2 \sum \Delta m r_i^2~.
\end{equation}
当每一小块足够小时,可以用积分代替累加。
\begin{equation}
E_k=\frac{1}{2} \omega^2 \int r_i^2 \dd m ~.
\end{equation}
其中 $\int r_i^2 \dd m$ 即被定义为转动惯量I

因此
\begin{equation}
E_k=\frac{1}{2} I \omega^2~.
\end{equation}

\addTODO{直接用质点组计算刚体绕轴转动的动能,对比质点运动的能量。}
\addTODO{为什么定轴转动时\enref{力矩}{Torque}对刚体\enref{做功}{Fwork}等于 $\tau \theta$?功率等于 $\tau\omega$}

\begin{example}{物理摆的角速度}\label{ex_RigEng_1}
\addTODO{在上面两个例题中计算给定夹角 $\theta$ 的角速度, 使用动能定理计算, 说明刚体的势能就是质心的势能。}
\end{example}
