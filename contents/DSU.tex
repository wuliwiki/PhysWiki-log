% 并查集
% keys 并查集|数据结构|C++
% license Xiao
% type Tutor

\footnote{参考 Wikipedia \href{https://en.wikipedia.org/wiki/Disjoint-set_data_structure}{相关页面}。}\textbf{并查集(disjoint-set)}是一个树形、用于维护不相交的集合的数据结构。

对于并查集,主要有如下操作:
\begin{enumerate}
\item $\mathtt{merge}$ 合并两个集合;(“并”)

\item $\mathtt{find}$ 判断两个元素是否属于同一个集合。(“查”)
\end{enumerate}

定义集合的表示方法:“代表元”法,\textbf{即每个集合都会选择一个固定的元素,作为这个集合的代表}。其次需要定义归属关系的表示方法:\textbf{使用一个树形结构存储每个集合,树上的每个结点都是一个元素,树根是集合的代表元素。}我们用 \verb|p[i]| 表示每个结点的父结点,初始化 \verb|p[i]| 都指向自己,树根也指向自己。在合并两个集合时,只需要将其中一个树根的父结点指向另一个树根,即 \verb|p[root_1] = root_2|。在查询每个集合的树根时,朴素的办法就是通过 \verb|p[i]| 存储的值不断递归访问父结点,直至到达树根。为了提高效率,我们引入了\textbf{路径压缩}这种优化方法。

在进行合并和查询的操作中,我们只关心每个集合的根结点,\textbf{所以我们在 $\mathtt{find}$ 操作的时候,把访问过的每个结点都直接指向树根,}这种优化方法被称为\textbf{路径压缩},进行完路径压缩之后,每个结点的父结点都为根结点了。加上路径压缩的并查集每次 $\mathtt{find}$ 的操作均摊时间复杂度为 $O(\log N)$。

还有一种优化方法为\textbf{按秩合并},单独采用“按秩合并”的并查集每次 $\mathtt{find}$ 的操作均摊时间复杂度也为 $O(\log N)$,若同时采用“路径压缩”和“按秩合并”的并查集,每次 $\mathtt{find}$ 的操作均摊时间复杂度为 $O(\alpha(N))$,其中 $\alpha(N)$ 为\textbf{反阿克曼函数},对于 $\forall \ N \leq 2^{2^{10^{19729}}}$,都有 $\alpha(N) < 5$。

并查集的存储:
\verb|int p[N]|

并查集的初始化:
起初所以元素各自构成一个独立的集合,即有 $n$ 棵 $1$ 个点的树。
\begin{lstlisting}[language=cpp]
for (int i = 1; i <= n; i ++ ) p[i] = i;
\end{lstlisting}

并查集的 $\mathtt{find}$ 操作:
\begin{lstlisting}[language=cpp]

// 返回 x 的根结点 + 路径压缩
int find(int x)
{
    // 每个集合中只有根结点的 p[x] 值等于他自己
    if (p[x] != x) p[x] = find(p[x]); // 如果 x 不是根结点,那么就让 x 的父结点直接等于根结点
    return p[x];
}

// 熟练之后可以写成一行:
// return p[x] = (p[x] != x ? find(p[x]) : p[x]);
\end{lstlisting}

\begin{figure}[ht]
\centering
\includegraphics[width=14.25cm]{./figures/442b09a6eb1cc80e.png}
\caption{$\mathtt{find}$ 操作} \label{fig_DSU_1}
\end{figure}

并查集的 $\mathtt{merge}$ 操作:
\begin{lstlisting}[language=cpp]
// 合并 x 和 y 的所在集合,等于让其中一个集合的父结点指向另一个集合的根结点
void merge(int x, int y)
{
    p[find(x)] = find(y);
}
\end{lstlisting}


\begin{figure}[ht]
\centering
\includegraphics[width=14.25cm]{./figures/091fb4195d1097f2.png}
\caption{$\mathtt{merge}$ 操作} \label{fig_DSU_2}
\end{figure}

如果想维护集合的大小呢?可以开一个 \verb|cnt| 数组用于维护集合的大小,最初每个集合的大小初始化为 $1$。

在 $\mathtt{merge}$ 操作的函数多加一句语句:\verb|cnt[b] += cnt[a];|,\textbf{加之前需要判断 $a$ 和 $b$ 是不是不在一个集合内},以免重复加导致答案不对。


以上就是并查集的基本操作了。
