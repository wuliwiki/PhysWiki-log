% 可解群
% keys 可解群|solvable group|导出列|换位子|换位子群|derived series|commutator|commutator group|正规序列|次正规序列|normal series|subnormal series|合成序列|composite series
% license Xiao
% type Tutor


\pentry{换位子群\nref{nod_CmtGrp}}{nod_3aa9}

可解群在Galois理论中起到关键作用,用于判断代数方程的根式可解性,由此而得名。由于我们现在尚未深入Galois理论,就不讨论何谓“可解”,而仅仅从群结构的角度研究这种群的性质。


\subsection{可解群的定义}


\begin{definition}{可解群}\label{def_SlvbGp_1}
给定群$G$,定义$G^{(0)}=G$且对于任意正整数$k$,$G^{(k)}=[G^{(k-1)}, G^{(k-1)}]$。

称
\begin{equation}\label{eq_SlvbGp_1}
G^{(0)}\rhd G^{(1)}\rhd G^{(2)}\rhd \cdots~
\end{equation}
为$G$的\textbf{导出列(derived series)}。

若导出列在有限步内终结于$\{e\}$,或者等价地说,存在非负整数$n$使得$G^{(n)}=\{e\}$,则称$G$\textbf{可解(solvable)}。

\end{definition}


\autoref{def_SlvbGp_1} 的\autoref{eq_SlvbGp_1} 直接使用了正规子群符号$\rhd$,这是由\autoref{the_CmtGrp_2} 保证的。



\autoref{def_SlvbGp_1} 的基础是换位子群的概念,但我们也可以仅用正规子群的概念来定义可解群:



\begin{definition}{(次)正规序列}
给定群$G$,称序列
\begin{equation}
G=G_1\rhd G_2\rhd\cdots G_n=\{e\}~
\end{equation}
为$G$的一个\textbf{次正规序列(subnormal series)}。若该序列还满足$\forall k\in \mathbb{Z}\cap\{1, n\}, G_k\lhd G$,则称之为一个\textbf{正规序列(normal series)}。
\end{definition}



\begin{theorem}{可解群的另一定义}\label{the_SlvbGp_1}
给定群$G$,则$G$是可解群,当且仅当存在$G$的正规序列$G=G_1\rhd G_2\rhd\cdots G_n=\{e\}$,且对于任意正整数$k$,$G_k/G_{k+1}$都是阿贝尔群。
\end{theorem}


\textbf{证明}:

\textbf{必要性}:

设$G$可解,即存在非负整数$n$使得$G^{(n)}=\{e\}$。令$G_k=G^{(k-1)}$,则对于任意正整数$k$,$G_{k+1}\subseteq [G_k, G_k]$,于是由\autoref{the_CmtGrp_1} 可知,$G_k/G_{k+1}$都是\textbf{阿贝尔群}。又由\autoref{the_CmtGrp_2},可知各$G_k\lhd G$,因此
\begin{equation}
G=G_1\rhd G_2\rhd\cdots\rhd G_{n+1}=\{e\}~
\end{equation}
是一个\textbf{正规序列}。

\textbf{充分性}:

设存在$G$的正规序列$G=G_1\rhd G_2\rhd\cdots G_n=\{e\}$,且对于任意正整数$k$,$G_k/G_{k+1}$都是阿贝尔群。

由\autoref{the_CmtGrp_1} 可知,$G_{k+1}\supseteq [G_k, G_k]$对任意正整数$k$成立。考虑到$G_2\supseteq[G_1, G_1]=G^{(1)}$,从而$G_3\supseteq[G_2, G_2]\supseteq [G^{(1)}, G^{(1)}]=G^{(2)}$;以此类推,$G_{k}\supseteq G^{(k-1)}$对任意正整数$k$成立。

因此,$G_n=\{e\}\implies G^{(n-1)}=\{e\}$。

\textbf{证毕}。




\subsection{可解群的结构}



\begin{lemma}{}
可解群的子群和商群都可解。
\end{lemma}



\textbf{证明}:

给定可解群$G$。

设$H$是$G$的子群,则$[H, H]\subseteq [G, G]$,以此类推得$H^{(k)}\subseteq G^{(k)}$对任意非负整数$k$成立。因此$G^{(n)}=\{e\}\implies H^{(n)}=\{e\}$。

设$N\lhd G$。

若$[G, G]\subseteq N$,则任取$g_1, g_2\in G$有$[g_1, g_2]\in N$,于是$[g_1N, g_2N]\subseteq N$,即$[G/N, G/N]=N$。

若$N\subseteq [G, G]$,则类似可得$[G/N, G/N]=G^{(1)}/N$。

以此类推,由$G^{(n)}=\{e\}$即可推出$\qty(G/N)^{(n)}=\{e\}$。


\textbf{证毕}。






\begin{lemma}{}\label{lem_SlvbGp_1}
给定群$G$及其正规子群$N$,若$N$和$G/N$均可解,则$G$可解。
\end{lemma}


\textbf{证明}\footnote{此证明照搬自《代数学基础》。}:

设$G/N$和$N$分别有正规序列
\begin{equation}
    \left\{
    \begin{aligned}
        G/N ={}& Q_0\rhd Q_1\rhd Q_2\rhd\cdots\rhd Q_s=N, \\
        N ={}& N_0\rhd N_1\rhd N_2\rhd\cdots\rhd N_t=\{e\}. 
    \end{aligned}
    \right. ~
\end{equation}

对于$0\leq k\leq s$,令$G_k=\pi^{-1}\qty(Q_k)$;对于$s<k\leq s+t+1$,令$G_k=N_{k-s-1}$。则可以构造序列
\begin{equation}
    G=G_0\rhd G_1\rhd\cdots\rhd G_{s+t+1}=\{e\}. ~
\end{equation}
其中,$G_s$到$G_{s+t+1}$自然构成一个正规序列。而$G_0$到$G_s$也依然满足“全是$G_0$的正规子群,$G_{k-1}/G_{k}$是阿贝尔群”,证明如下:

\begin{equation}
\begin{aligned}
    \forall g\in G, \qty(Q_k\lhd G/N) \iff{}& \qty(\qty(g^{-1}N)\qty(Q_k)\qty(gN)=Q_k)\\
    \iff{}& \qty(g^{-1}N)\qty(\pi^{-1}\qty(Q_k))\qty(gN)=\qty\pi^{-1}\qty(Q_k)\\
    \implies{}& \qty(g^{-1}\pi^{-1}\qty(Q_k)g\subseteq \pi^{-1}\qty(Q_k)). 
\end{aligned}~
\end{equation}
此即证明了$\pi^{-1}\qty(Q_k)\lhd G=G_0$。注意$\pi: G\to G/N$是自然同态。

又根据\autoref{exe_NormSG_2},
\begin{equation}
    \begin{aligned}
        Q_k/Q_{k+1} \cong G_k/G_{k+1}. 
    \end{aligned}~
\end{equation}
故由$Q_k/Q_{k+1}$是阿贝尔群,得证$G_k/G_{k+1}$是阿贝尔群。


\textbf{证毕}。

















