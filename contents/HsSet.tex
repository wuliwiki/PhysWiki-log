% 集合(高中)
% keys 高中|集合
% license Xiao
% type Tutor

\begin{issues}
\issueTODO
\end{issues}

% 之前的内容已经由欄、停敘(993)重写,协议由 Usr 改为 Xiao

学习初中数学时,你很有可能感受到了,每个部分都有它自己的一套体系和语言(比如,几何证明时的“平行”$\mathbin{\!/\mkern-5mu/\!}$就不会出现在解方程的过程中,而解方程时的“未知数”$x$也不会成为几何证明的要素),每个部分原本就这样各自发展着,从未想过彼此之间有什么可以联系起来的可能。但经过一代代数学家们的不断努力,现代数学展现在了世人面前,它最重要的一个特点就是将整座数学大厦建立在了集合论的基础之上,从此各部分不再是孤立的。同时,集合也能够简洁而准确地表达其他数学内容。这使得,集合语言不但成为了所有数学分支的通用语,也使得不论母语是什么的研究者都可以用相同的语言来展现和理解某个数学思想。可以这么说,不能掌握集合,数学之路就寸步难行。

因此,高中数学理所当然地以“集合”作为一切后面学习的开始,同时,高中阶段也并不要求对集合进行过深的探索。下文会首先介绍集合和元素这两个概念、表示方法以及性质,然后再分别介绍集合与元素的关系以及集合之间的关系。

提醒一下:
\begin{itemize}
\item 这篇文章会涵盖不少与集合相关的新概念(当然也包括集合本身),它和你以往的数学经验不太会有直接的关系,但请不要被这些概念的数量吓倒,它只是名字比较新,事实上却与生活经验联系密切,每个完成高中阶段学习的同学都会认为“集合”是最简单的。
\item 请务必理解这部分的内容,哪怕暂时不熟悉这些新概念的名称,也一定要理解他们的意思。两个很有帮助的理解方向是:为什么需要这个概念?它和其他概念之间的有哪些差异?在文中也会尽可能指明它们来帮助理解。如果学有余力,借助这个机会可以感受一下高中数学思考方式的不同,它是会延伸到高等教育中数学乃至其他学科中。由于这些概念很常用,在使用时再逐步熟悉名称就可以,不必因为名称太多记不住而紧张。
\end{itemize}

\subsection{集合与元素}
\addTODO{配图:有一个人前来卖瓜}
\begin{example}{塑料袋与水果}\label{ex_HsSet_1}
我给了你一个塑料袋,求你帮我把它送给另一个人。塑料袋嘛,就是用来装东西的,里面装了三种水果:苹果、香蕉、橙子。至于多少个,我没说你也没看,反正不轻。

你拎着这些东西到了他那里。他随口问了一声:“这个塑料袋里,装没装苹果?”你打开一看,有,就回答他“装了。”你刚想放下,他又问:“这个塑料袋里,装没装西瓜?”你只能又打开一看,没有,就回答他“没装。”他又要张嘴问你,奈何你的手已经不堪重负,于是你把塑料袋放地上了,跟他说:“东西我都放这里了,都在塑料袋里面我也没动,挺多的你自己看吧。”
\end{example}

数学上,这个塑料袋称为集合,而里面的水果就称为元素。就像刚才的问问题的那个人只关注“塑料袋里有没有某种水果”一样,研究集合时,也只关注一个问题,那就是“这个集合里面有没有某个元素”。至于水果好不好吃、塑料袋会不会破、你的手疼不疼,那都不是集合需要关注的问题。

相信你现在已经对这个塑料袋大概有印象了,请记住塑料袋这个例子,每次遇到跟集合相关的问题时,用塑料袋来理解都蛮好用。现在我们回到数学上来,给出一个集合的概念\footnote{这里的集合概念是朴素的,之后因为引发了一些问题,导致数学家们又创立了新的定义,这个问题直到现在还莫衷一是。但就如一开始所说,高中阶段对集合的要求并不这么艰深,毕竟这事现在也没说太准。},不过其实跟刚才说的塑料袋是一样的,它跟塑料袋的区别也不过就是用了书面语罢了。

\begin{definition}{集合}\label{def_HsSet_2}
一定范围内,某些能够确定的(well-defined,也称良定义的)、不同的对象(object)构成的整体(collection)称为\textbf{集合}(set),常用大写拉丁字母指代$A,B,C,\cdots$,称为集合$A$、集合$B$、集合$C$等。

构成集合的每个对象称为该集合的\textbf{元素}\footnote{有的同学会好奇“集合可不可以作为元素”,结论是肯定的,但是在高中阶段不涉及。}(element),常用小写拉丁字母指代$a,b,c,\cdots$,称为元素$a$、元素$b$、元素$c$等。
\end{definition}

其中涉及的一个术语可能对部分读者来说略显陌生——“能够确定”或者说“良定义”。可以参考 \autoref{ex_HsSet_2} 来帮助理解这一概念。

\begin{example}{是否存在一个集合是由“好看的明星”构成的?}\label{ex_HsSet_2}
由于“好看”是一个主观标准,而“明星”是一个不确定的概念,因此,除非给定一个可以判定的“明星”概念,并量化“好看”的标准(即确定一些量来审核颜值),否则根据确定性,在一般的语境下,不存在一个由“好看的明星”构成集合。
\end{example}

介绍完概念,一般会开始分析概念的特性,教材上一般会说“根据定义,集合的元素具有三个特性:确定性、互异性、无序性。”这三个特性有一点太唬人了,我们回到刚才送水果的例子里面来看一看:

\begin{itemize}
\item 如果被问到某种水果是否在里面,你只要打开塑料袋看看,能看到就是在里面,不能看到就是不在里面,不存在某样东西,你翻过塑料袋之后回答对方“我不确定它在不在里面”。
\item 每个种类的水果都是与众不同的。不管放了多少个苹果,打开塑料袋,如果里面有苹果,那就是有苹果。至于有几个,在判断水果的种类这个问题上,水果的数量(忽略0个这种情况)不会影响这个问题的答案。
\item 在塑料袋里,水果的位置、顺序是不确定的、不重要的。你不知道自己会先看到什么,但只要这些水果都保持在这个塑料袋里,就可以保证回答这个问题时都是一样的。当然,你可以为了容易观看和记录,在塑料袋里面给水果排队,但不论是按名称排,还是按颜色排,都不会影响这个问题的答案。
\end{itemize}

核心就是记住一句话“集合关注的只有一点:有没有某个元素”,下面是两道练手题:

\begin{example}{由“1,2,3,3,2,1”构成的集合有几个元素?}
翻译成“塑料袋语言”:一个塑料袋里有两个苹果、两个香蕉、两个橙子,问有几种水果。答:三种。

书面语:由于集合中的元素具有互异性,因此集合中只有“1、2、3”三个元素。
\end{example}

\begin{example}{判断:一个班级的学生按出生日期排列和按身高排列构成不同的集合。}
如果将“班级”看成一个集合,将“学生”看成元素,则由于集合的无序性,不论怎么排序都不影响集合,因此错误。
\end{example}

在实际使用时,“互异性”常成为其它题目的考察背景,在高中阶段较为常见;“确定性”是较为深刻的,影响了集合本质的,因此多出现在数学专业的集合论的探讨中;“无序性”看似与生活中常接触到的“序”(量可以比大小)的概念相冲突,但在数学中,“序”是从“无序”中构建出来的\footnote{如果感兴趣可以参考\autoref{def_CartPr_2} 处有序对的构建思路},高中阶段基本也不涉及。

\subsection{集合的基数}

%这里有一个“元素个数”的概念,在高中教材中一带而过。但由于很重要,这里提及一下,希望你有个印象。

如果一个集合中只有有限多个元素,我们就说它是一个\textbf{有限集},否则就是\textbf{无限集}。对于有限集,我们作如下定义:
\begin{definition}{集合的基数}
有限集$M$中的元素个数,称作集合的\textbf{基数}(cardinality,也称作集合的\textbf{势}),用${\rm card}(M)$表示。规定${\rm card}(\varnothing)=0$。
\end{definition}


这个定义本质上就是发明了两个新的术语。为什么要发明新的术语,而不能直接说“集合的元素个数”呢?这是因为“个数”必须是一个非负整数,比如一个班里有$60$个学生,这里的$60$就是一个非负整数;但对于无限集合,比如说全体正整数构成的集合,它的元素数量可比任何一个非负整数都大。我们朴素的直觉会认为,全体正整数构成的集合,其元素数量是无穷多;但“无穷”并不是一个整数,所谓“集合的元素无穷多”其实是在说“不管$N$是多大的整数,拿走$N$个元素后集合里还剩下元素”,也即“拿不完”——你不能说“不完”是一个整数,对吧?另一方面,无穷集合之间也有元素多少的区分,比如全体实数就比全体整数要多,所以单单一个无穷不足以区分这些集合的元素多少。总之,对于无穷集合,“个数”的概念就不再适用,我们就发明一个新的术语,来拓展“个数”的概念——所谓拓展,就是说当你限制讨论范围是有限集合的时候,集合的基数正是集合的元素个数。

虽然高中并不深入讨论,但集合基数是集合论中非常重要的基础概念,因此我们在此提及。


%可以证明,对于两个有限集合$A,B$:
%\begin{equation}
%{\rm card}(A\cup B)={\rm card}(A)+{\rm card}(B)-{\rm card}(A\cap B)~.
%\end{equation}

\subsection{集合的表示方法}

定义中提到了可以给某个集合或元素命名,但当我们讨论的时候,我总不能没头没脑地说“集合$A$怎么怎么样”,因为你都不知道我说的集合$A$到底是什么样的——是全班同学构成的集合,还是全世界所有人的头发构成的集合呢?数学家们设计了三种表示方法,它们用各自的手段展现一个集合的样子。

\subsubsection{列举法}

列举法的做法就是把集合中所有的元素全部列举出来,写在大括号内,比如用$\{1,2,3\}$表示由数字$1,2,3$三个元素构成的集合。

特别地,由于有些集合的元素比较多,全都写出来的话比较麻烦,因此在不产生歧义的前提下,一般可以将大括号内部过多的元素用“$\cdots$”代替。

\begin{example}{用列举法表示是从$0$到$50$的偶数构成的集合$A$}
\begin{equation}
A=\{0,2, \cdots ,48,50\}~.
\end{equation}
\end{example}

\subsubsection{描述法}

每次都把集合所有的元素都写出来的确是清晰明了,但很麻烦,也无法从本质上体现这个集合的特征。

构建集合时,通常会设计一个判断标准,满足判断标准的元素$x$就会认为在这个集合里。比如说,如果$\mathbb{Z}$是全体整数构成的集合,那么你自然就能确定,$114$这个数字就在集合$\mathbb{Z}$中,这里的判断标准就是“$x$是一个整数”。

数学上,为了简洁考虑,我们按如下格式来描述集合的性质:
\begin{equation}
A = \{x\mid x\text{满足某性质}\}~, 
\end{equation}
相当于用汉语说“$A$是由满足某性质的$x$构成的集合”。





%记作$p(x)$\footnote{这里的用法是第一次出现,它表示的是把$p$作用在$x$上,产生一个结果,类比一下就像把笔作用在纸上产生了笔迹,这里先大概理解这个记法,在\enref{函数(高中)}{functi}部分还会有更深入的介绍。},意为$x$满足性质$p$。

% 描述法就能在表示集合时利用性质,体现特征,例如:
% \begin{equation}
% A=\begin{Bmatrix} x\mid p(x) \end{Bmatrix}~.
% \end{equation}
% “$ \mid $”用于分割元素与性质。理解起来如下表:
% \begin{table}[h]
% \caption{描述法的符号与读法}\label{tab_HsSet1}
% \centering
% \begin{tabular}{|c|c|c|c|c|c|}
% \hline
% 符号 & $\{$&$x$ & \mid & $p(x)$&$\}$ \\
% \hline
% 读法 & 集合&由$x$构成&,$x$是满足& 性质$p(x)$&的所有元素。 \\
% \hline
% \end{tabular}
% \end{table}

\begin{example}{}

从$0$到$50$的偶数构成的集合$A$,可以描述如下:

\begin{equation}
\begin{aligned}
A={}&\{x \mid x\text{是大于等于}0\text{且小于等于}50\text{的偶数}\}\\
={}&\{x \mid x=2k,k\text{是大于等于}0\text{且小于等于}25\text{的整数}\}\\
={}&\{2k\mid\}~.
\end{aligned}
\end{equation}
\end{example}

\subsubsection{维恩图}
上面的两种方法都着眼于集合和元素的关系,在研究集合之间的关系时就不太直观、好用了。这时,一般会采用图示法,也就是用草图来表示集合。

\textbf{维恩图}(Venn diagram)也叫\textbf{文氏图}、\textbf{韦恩图},是用平面上的区域来表示集合的方法。在进行集合间关系的分析时,可以通过画阴影、图案等方式来分析关系,非常有效,具体效果可以在\autoref{sub_HsSet_1} 和\enref{集合的基本运算(高中)}{HsSeOp}中感受。
% \begin{figure}[ht]
% \centering
% \includegraphics[width=10cm]{./figures/e449e54347ae8e24.png}
% \caption{Venn图} \label{fig_SufCnd_1}
% \end{figure}

\begin{figure}[ht]
\centering
\includegraphics[width=10cm]{./figures/ad496396abf0f369.pdf}
\caption{维恩图的示意图。集合$A$用平面上的黄色区域表示,元素则用平面上的点表示。点$x$在区域中,意味着$x$是集合$A$的元素;点$y$不在区域中,意味着$y$不是集合$A$的元素。} \label{fig_HsSet_1}
\end{figure}

另一种表示方法是在数轴上用直线或曲线来表示范围,这种表示只适用于区间的表示方法,具体使用方式参见\autoref{sub_HsSet_2} 的“区间”。

注意:图示法的缺点就是不够严谨,也因为是草图,所以只能作为自己理解的辅助出现在草纸上,或在书本中作为辅助理解的工具,而不能作为理由直接出现在证明、计算过程或试卷上。

\subsection{元素与集合的关系}

经过刚才的接触,相信你已经对元素和集合的关系有了隐约的感觉,下面明确地给出定义。

\begin{definition}{属于与不属于}
若 $a$ 在集合 $A$ 中,称 $a$ \textbf{属于}(belong to)集合 $A$ ,记作:
\begin{equation}
a \in A~.
\end{equation}

若 $a$ 不在集合 $A$ 中,称 $a$ \textbf{不属于}(not belong to)集合 $A$,记作:
\begin{equation}
a\notin A~
\end{equation}
\end{definition}
是的,元素与集合之间就只有这两种关系。在\autoref{fig_HsSet_1} 中就有$x\in A$和$y\not\in A$。


而由于确定性的要求,任意一个元素,要么属于一个集合,要么不属于这个集合,不存在第三种情况,即对所有的元素$a$:
\begin{equation}
a\in A\qquad\text{或者}\qquad a\not\in A~.
\end{equation}

对于\autoref{ex_HsSet_1} 而言对方的问题就是判断“苹果、西瓜和塑料袋的关系”,而你的回答就是“苹果$\in$塑料袋,西瓜$\notin$塑料袋”。

\subsection{特殊的集合}\label{sub_HsSet_2}

%尽管下面应该开始研究集合间的关系了,但让我们先暂停一下打个岔,看一看那些未来会经常打交道的特殊集合。


\subsubsection{数字集合}


以下列举了一些以“数字”为元素的集合,在高中阶段非常常见:

\begin{itemize}
\item 全体\textbf{整数}构成的集合,记为$\mathbb{Z}$;
\item 全体\textbf{正整数}构成的集合,记为$\mathbb{Z}^+$;
\item 全体\textbf{有理数}构成的集合,记为$\mathbb{Q}$;
\item 全体\textbf{实数}构成的集合,记为$\mathbb{R}$;
\item 全体\textbf{复数}构成的集合,记为$\mathbb{C}$。
\end{itemize}










\subsubsection{区间}

什么叫区间呢?你可以形象地理解为在数轴上割出来一段,比如从$a$到$b$的一段(假设$a<b$)。然而,同样是起点为$a$、终点为$b$的区间,$a$和$b$不一定是区间中的元素,因此相同起止点的区间一共有四个。我们用方括号“$[$”和“$]$”、圆括号“$($”和“$)$”来标记起点或者中点在不在区间里,如下所列举:
\begin{itemize}
\item 起点$a$和终点$b$\textbf{都在}区间中时,称区间为\textbf{闭区间},记作$[a,b]$,用集合的写法记为$\{x\in\mathbb{R}\mid a\leq x\leq b\}$;
\item 起点$a$和终点$b$\textbf{都不在}区间中时,称区间为\textbf{开区间},记作$(a,b)$,用集合的写法记为$\{x\in\mathbb{R}\mid a< x<b\}$;
\item 起点$a$\textbf{在}区间中但起点$b$\textbf{不在}时,称区间为\textbf{左闭右开区间},记作$[a,b)$,用集合的写法记为$\{x\in\mathbb{R}\mid a\leq x< b\}$;
\item 起点$a$\textbf{不在}区间中但起点$b$\textbf{在}时,称区间为\textbf{左开右闭区间},记作$(a,b]$,用集合的写法记为$\{x\in\mathbb{R}\mid a< x\leq b\}$
\end{itemize}



%在书写时取得到的点就用,取不到的点就用圆括号“$($”或者“$)$”。
%在数轴上,一般如果将取不到的点特意用空心圆表示,而能取到的就直接涂成一个大黑点。当然,也别太大太用力,把纸涂坏了。


如果我们把数轴画成半透明的灰色,把区间中的点都涂黑,那么就得到一段黑色线段,这个黑色线段就可以用来表示区间。问题是,单个的点没有大小,所以无论起点或者终点在不在区间里,黑色线段的样子都一样。于是为了区分,我们规定,如果某个起点或者终点不在区间中,就要在它的位置画一个空心圆来表示。

比如,$[-1,2)$在数轴上表示为:

\begin{figure}[ht]
\centering
\includegraphics[width=10cm]{./figures/1982272a72d89849.pdf}
\caption{区间$[-1, 2)$的示意图} \label{fig_HsSet_2}
\end{figure}


要注意,数轴是全体实数的集合,所以区间里的元素都是实数。


如果一个区间没有起点或者终点,我们就说它是一个\textbf{无穷区间}:
\begin{itemize}
\item 集合$\{x\in \mathbb{R}\mid x> a\}$记为$(a, +\infty)$;
\item 集合$\{x\in \mathbb{R}\mid x\geq a\}$记为$[a, +\infty)$;
\item 集合$\{x\in \mathbb{R}\mid x< a\}$记为$(-\infty, a)$;
\item 集合$\{x\in \mathbb{R}\mid x\leq a\}$记为$(-\infty, a]$。
\end{itemize}

你应该注意到了,无穷符号$\infty$的旁边永远是圆括号,意味着“无穷”不可能是区间的端点。这是因为无穷并非数字,而是“没有尽头”的意思。无即“没有”,穷即“尽头”,不能说“没有尽头”是一个数字,对吧?

但是在实际交流的过程中,我们还是会在\textbf{语法上}把它当作一个数字来描述。比如说,我们也许会说,当$x$逐渐减小到$0$时,“$1/x$逐渐增大到无穷”,但这实际上是“$1/x$逐渐增大,但没有上限”的简便说法。类似地,我们也会说,对于任何实数$a$,“必有$a<+\infty$”,这也并非真正的数字之间大小比较,而是说“$a$并非实数增大的尽头”,即“$a$并非最大的实数”。

为方便引用,我们把上述关于无穷的结论总结如下:
%\begin{definition}{无穷\footnote{此处以定义给出是方便引用,不做定义理解。}}
%\textbf{无穷}记作$\infty$。关于无穷的具体学习会在本科阶段进行,高中涉及到这个概念时不会超出下面三点范围:
\begin{enumerate}
\item 无穷不是个数字
\item 无穷不在实数集里,因此涉及无穷的区间端点都只能取“开”。
\item $-\infty$比所有实数都小,$+\infty$比所有实数都大。
\end{enumerate}
%\end{definition}

% 全体实数构成的集合用区间也可以表示,记作$(-\infty,+\infty)$。如果想表示某个单独的不等关系时,利用区间就可以表示为:
% \begin{itemize}
% \item 小于$b$的数,记作$(-\infty,b)$,用集合的写法记为$\{x \mid x< b\}$;
% \item 小于等于$b$的数,记作$(-\infty,b]$,用集合的写法记为$\{x \mid x\leq b\}$;
% \item 大于$a$的数,记作$(a,+\infty)$,用集合的写法记为$\{x \mid x> a\}$;
% \item 大于等于$a$的数,记作$[a,+\infty)$,用集合的写法记为$\{x \mid x\geq a\}$;
% \end{itemize}

\subsubsection{空集}

就像有空塑料袋一样,也有一个“空集”的概念。

\begin{definition}{空集}
如果任何元素都不属于某个集合,则这个集合称为\textbf{空集}(empty set),记作 $\varnothing$,即对所有的元素$a$:
\begin{equation}
a\notin\varnothing~.
\end{equation}
\end{definition}

最开始学习的时候,会很容易把它和$0$联想起来,他们的确存在一些联系\footnote{空集的元素数量是0},但建议你学习时就还是把它当成“空塑料袋”就好了,尤其不要把空集当成$\{0\}$,后者是包含一个元素“$0$”的集合。

\begin{definition}{*有限集和无限集}
若集合含有有限多个元素,则称之为\textbf{有限集},否则称之为\textbf{无限集}\footnote{这里给出是因为教材上有提及,但如前面所说高中不涉及这一部分}。
\end{definition}

因为空集元素个数是0,所以空集也是有限集。

\subsubsection{数集}

由于数字在数学领域有特别的地位,也非常常用,于是数学家们把“只有数字构成的集合”简称\footnote{随着学习的深入,你会越来越感受到这帮人真的是一个字都不愿意多说。}为\textbf{数集}(number set),某些特殊的数集采用特定的记号,如下表所示。

\begin{table}[ht]
\centering
\caption{特殊数集及符号}\label{tab_HsSet2}
\begin{tabular}{|c|c|c|c|c|c|}
\hline
集合名称 &自然数集  &正整数集  & 整数集 & 有理数集& 实数集 \\
\hline
集合记号\footnote{在高等数学领域,这些数集的记号为$\mathbb{N,N^+,Z,Q,R}$,既表示他们的地位特殊,同时这些集合的定义都是广泛明确的,使用这个记号会方便交流。} & ${\rm \mathbf{N}}$ & ${\rm \mathbf{N^+}}$ 或 ${\rm \mathbf{N^*}}$ & ${\rm \mathbf{Z}}$ & ${\rm \mathbf{Q}}$ & ${\rm \mathbf{R}}$ \\
\hline
\end{tabular}
\end{table}

\subsection{集合与集合的关系}\label{sub_HsSet_1}
下面我们来研究集合之间的关系,之前我们说过,一个集合完全由它的元素决定,因此,集合间的关系最终也都反映到元素上。
就像两个装的水果一样的塑料袋,我们这时会说他们“一样”一样,我们先来定义相等:
\begin{definition}{集合相等}
对集合A、B,若他们的元素完全相同,则称他们\textbf{相等}(equal),记作:
\begin{equation}
A=B~.
\end{equation}
\end{definition}

所以,这里就有一个判断集合相等的方式:挨个判断这两个集合里的全部元素是否一样。

\addTODO{这个例子有点问题。小袋子套在大袋子里,那么就有小袋子$\in$大袋子。}

假设你有两个塑料袋,一个是小袋子,另一个是大袋子。你先把小袋子装满了各种水果,然后把这个小袋子放进大袋子里。现在,大袋子里面有小袋子和里面的水果。接下来,有两种选择:一种是你不再往大袋子里添加任何东西,另一种是你可以再往大袋子里放一些不同种类的水果。无论你做了哪种选择,因为小袋子里的所有水果种类都已经包含在大袋子里,如果只关注袋子里的水果种类的话,那么我们说小袋子是大袋子的“子集”\footnote{注意要求的前提是不能把塑料袋也当成“一种水果”。但有时的研究考虑的是物品种类,这样塑料袋和各种水果就都算成不同的物品了,这个问题不在当前的讨论范围。如果专业点解释的话,就是在高中阶段,一个集合不会成为另一个集合的元素。}。这个“子集”有点类似于数量之间“小于等于”的关系。

\begin{definition}{子集}\label{def_HsSet_3}
对两个集合$A,B$,若$A$的所有元素都属于集合$B$,即对所有的元素$a$,只要有$a\in A$,就有$a\in B$,则称集合 $A$ 是集合 $B$ 的\textbf{子集}(subset),或者说集合$B$\textbf{包含}集合$A$、集合$A$\textbf{包含于}集合$B$,记作
\begin{equation}
A \subseteq B\qquad\text{或者}\qquad B \supseteq A~.
\end{equation}
否则,若存在$A$的某个元素不属于集合$B$,即$\exists a\in A,a\notin B$,则称集合 $A$ 不是集合$B$的子集,或者说集合$B$\textbf{不包含}集合$A$、集合$A$\textbf{不包含于}集合$B$,记作
\begin{equation}
A \nsubseteq B\qquad\text{或者}\qquad B \nsupseteq A~.
\end{equation}
规定,空集是任何集合的子集,即对任意一个集合$A$:
\begin{equation}
\varnothing \subseteq A~.
\end{equation}
\end{definition}

用维恩图表示就是这样的
\addTODO{维恩图:B包含A,不包含:AB相交、AB不相交、A包含B}

注意区分子集和属于,两者的对象一个是元素与集合的关系,一个是集合间的关系。根据定义,显然任何一个集合都是它本身的子集,即
\begin{equation}
A \subseteq A~.
\end{equation}

因此,对于任何一个集合 $A$ 都有空集和它自身是它的子集。$A=\varnothing$时,他自身就是空集。
同时,根据子集的定义,也可以得到另一个判定集合相等的方法:

\begin{theorem}{根据子集关系判断集合相等}
如果两个集合互为对方的子集,那么他们相等,即:
\begin{equation}\label{eq_HsSet_1}
A\subseteq B,B\subseteq A\implies A=B~.
\end{equation}
\end{theorem}

有时,想要明确表达两个包含关系的集合没有相等,就像只想研究数量之间“小于”的关系时,会用到真子集的概念。

\begin{definition}{真子集}
对于两个集合,$A$ 与 $B$,如果 $A\subseteq B$ ,并且 $A \ne B$,我们就说集合 $A$ 是集合 $B$ 的\textbf{真子集},或者集合 $A$ \textbf{真包含于}集合 $B$、集合 $B$ \textbf{真包含}集合 $A$,记作:\footnote{这里给出的记法是人教版高中课本上的,在高中阶段请只使用这种写法。事实上,还有$A\subset B,B\supset A$和$A\subsetneq B,B\supsetneq A$两种写法用来表示“$A$是$B$的真子集”,且前一种更常用。}
\begin{equation}
A \subsetneqq B\qquad\text{或者}\qquad B \supsetneqq A~.
\end{equation}
\end{definition}

注意,由于维恩图是草图,在使用维恩图时不易区分“真子集”与“子集”的概念,教材中使用图\addTODO{上面包含的图序号}来表示“真子集”。实际使用时,建议只表示“包含”关系,并在运算时时刻注意是否可以取等的条件,并作标记来防止错误。

\begin{exercise}{用列举法列出满足$\{1\}\subseteq A\subsetneqq\{1,2,4\}$的所有集合$A$}
对$\{1\}\subseteq A$,由子集的定义,$1\in A$。对$A\subsetneqq\{1,2,4\}$由真子集的定义,$A\neq \{1,2,4\}$,且$2,4$可以属于$A$。因此:

$A$可以是$\{1\},\{1,2\},\{1,4\}$。
\end{exercise}
\subsubsection{全集}

就像\autoref{ex_HsSet_1} 里的“水果”,在具体研究时,通常会划定研究的范围,这个划定的范围含有要研究的全部元素。

\begin{definition}{全集}\label{def_HsSet_1}
研究集合间的关系时,如果要研究的集合全都是某个集合的子集,也即涉及到的要研究的元素全都在这个集合中,则称这个集合为\textbf{全集}(universal set),常用符号 $U$ 表示。
\end{definition}

\addTODO{全集维恩图}

全集是一个相对的概念,是为了规定研究范围而定下来的。尽管它的名字很容易给人一种“包含世间万事万物的集合”的感觉,但请不要把它和引号里的那个集合等同起来\footnote{那个集合是不存在的,或者说满足这个概念的“事物”不是集合。},他们没有任何关系。


\subsection{总结}

终于,从完全陌生开始,给高中的第一块内容构建了一个夯实的基础。概念很多,要记住的符号也很多,下面列出的是这一篇文章,涉及到的知识点,供你自查:

\begin{itemize}
\item 集合、元素的概念
\item 集合的三种表示方法
\item 特殊的集合:空集、全集、数集、区间的概念
\item 集合与元素的关系:属于、不属于的概念
\item 集合之间的关系:相等、子集、真子集的概念
\item 集合相等的判断方法
\end{itemize}














