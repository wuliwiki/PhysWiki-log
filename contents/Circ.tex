% 电路
% keys 零势能点|平衡|电容|净电荷
% license Xiao
% type Tutor

\begin{issues}
\issueDraft
\end{issues}

\pentry{欧姆定律\nref{nod_Resist}}{nod_fa2b}

\subsection{零势能点}
虽然我们可以规定电路中某点电势为零, 但大多数情况下这么做没有什么意义, 因为在这些电路中我们只讨论两点间的电压(电势差)。 例如电阻器两端的电压与电流和电阻值有关而与零势点的选取无关。
% \addTODO{...}

\subsection{不存在净电荷}
宏观上来说, 我们一般假设电路中的任意一点不能存在净电荷(电容器除外)。 这是因为电路中通常讨论的电压所能产生的净电荷往往可以忽略不计。 这里通过一个例子说明。

\begin{example}{}
一个半径为 $1\Si{cm}$ 的导体小球与无穷远之间的电容约为 $1.1\e{-12} \Si{F}$(见\autoref{ex_Cpctor_1})。 如果一个 $2\Si{V}$ 电源的正负极分别通过长导线连接两个这样的小球, 每个小球上最终只能积累 $\pm 1.1\e{-12} \Si{C}$ 的电荷, 仅相当于 $1\Si{A}$ 的电流流动 $1.1\e{-12}$ 秒。
\end{example}
这就是为什么在电路中, 我们一般认为\textbf{开路中电流为零, 只有回路才能产生电流}。

有了这个假设, 我们还可以得到\enref{基尔霍夫电流定律}{Kirch}。这个定理简单来说就是若一条电流为 $I$ 的导线分叉为两条电流为 $I_1$ 和 $I_2$ 导线, 那么必有 $I_1 + I_2 = I$。 如果该式不成立, 就说明节点处在不断积累净电荷, 从而违反了以上假设。

\subsection{瞬间平衡}
我们通常假设电路可以瞬间达到平衡状态。 例如给电源两端接上电阻后, 我们假设回路中的电流瞬间就能达到最终的稳定值而无需经历由小变大的过程。 又例如给电源两端接上电容后, 我们假设电容瞬间被充满。 在实际中达到平衡所需的时间也是极短的, 这时因为电路中自由电子的密度很高, 每克材料中都约有阿伏伽德罗常数数量级的自由电子, 电路中常见的电流对应的电子运动速度是极慢的, 所以它们加速的时间可以忽略。

\subsection{接地}
在电路中如果我们要规定零势点, 可以用接地符号。 这时我们假设大地的电势为零, 且电流可以流入和流出。

在实际运用中, 接地符号往往并不需要真的接地, 只是一种用于简化电路图的手段, 相当于把所有接地的点都用导线连接起来。 例如
\addTODO{图: 电源负极接地, 正极并联一些电阻电容, 它们的另一端也接地}
