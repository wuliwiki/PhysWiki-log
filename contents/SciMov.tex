% Matlab 数理教学动画制作笔记
% license Usr
% type Note

\subsection{背景和大小}
\begin{itemize}
\item \verb`saveas()` 使用的画幅是 figures 中的 \verb`Position` 尺寸。
\item figure 的 \verb`Position` 属性和 axis 的 \verb`InnerPosition` 属性可以都设为一样的像素,这样就会有 “全屏” 效果。
\item \verb`set(gca, 'XTick', []);` 可以删除坐标点和数字。
\end{itemize}

\subsection{导出透明背景图片}
\begin{itemize}
\item figure 中, \verb`set(gcf, 'Color', 'none');` 以及 \verb`set(gca, 'Color', 'none');` 就会移除背景,显示为黑色。
\item Windows 10/11 中,透明 png 在桌面的 icon 也是透明的。或者在 paint 画图软件中透明的部分是方格。
\item 例如用 \verb`plot()` 在 figure 中画了一个函数图,如何保存为背景透明的图片呢? Matlab 这方面的支持并不怎么好。
\item \verb`saveas(gcf, '...svg')` 和 \verb`print()` 无论是 svg 还是 png 都无法导出透明背景。 目前的方法是导出 svg 然后用一个脚本移除 \verb`<g><rect ... /></g>` 元素,另外也可以移除边框。
\item \verb`exportgraphics(gcf,'transparent.eps', 'ContentType','vector', 'BackgroundColor','none')` 据说可以导出透明矢量图(未测试)。
\item \verb`[I,map,alpha] = imread()` 可以检查 png 是否含有透明部分(\verb`map` 默认输出空矩阵)。
\item \verb`imshow()` 然后 \verb`set(gca, 'AlphaData', alpha)` 可以显示透明效果(透明部分是棋盘格)
\end{itemize}
