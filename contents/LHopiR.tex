% 洛必达法则
% keys 洛必达法则|导数|微分中值定理
% license Xiao
% type Tutor

\pentry{微分中值定理\nref{nod_MeanTh}}{nod_8b2c}

\textbf{洛必达法则(L'Hospital rule)}是一种对形如 $f(x)/g(x)$ 的函数求极限的方法。

\addTODO{链接:邻域}

\begin{theorem}{$\left(\frac{0}{0}\right)$ 型洛必达法则}
设函数 $f(x),g(x)$ 在 $a$ 点的某一去心邻域 $U^\circ(a,\delta)$ 上可导,而且满足:
\begin{enumerate}
\item $\lim\limits_{x\rightarrow a} f(x)=\lim\limits_{x\rightarrow a}g(x)=0$;
\item $g'(x)\neq 0, \forall x\in U^\circ(a,\delta)$;
\item $\displaystyle\lim\limits_{x\rightarrow a} \frac{f'(x)}{g'(x)}=l$ ($l$ 为有限数或 $+\infty$ 或 $-\infty$);
\end{enumerate}
则有
\begin{equation}
\lim\limits_{x\rightarrow a}\frac{f(x)}{g(x)}=\lim\limits_{x\rightarrow a}\frac{f'(x)}{g'(x)}=l~.
\end{equation}
上面的 $a$ 也可以取为 $+\infty$ 或 $-\infty$,定理仍然成立。
\end{theorem}
洛必达法则可以通过柯西中值定理证明。
\addTODO{证明补充}

对于其中的一种特殊情况 $sf=\lim\limits_{x\rightarrow a}f'(x),sg=\lim\limits_{x\rightarrow a}g'(x)$ 存在且 $sg\neq 0$,可以通过函数的一阶近似式来理解:$f(x)=sf\cdot (x-a)+\order{x-a},g(x)=sg\cdot (x-a)+\order{x-a}$,于是 $\lim\limits_{x\rightarrow a}f(x)/g(x)=sf/sg=\lim\limits_{x\rightarrow a}f'(x)/g'(x)$。利用泰勒展开公式作近似或者直接用洛必达法则,是求解分式函数极限问题的常用方法。

\begin{theorem}{$\left(\frac{\infty}{\infty}\right)$ 型洛必达法则}
设函数 $f(x),g(x)$ 在 $a$ 点的某一去心邻域 $U^\circ(a,\delta)$ 上可导,而且满足:
\begin{enumerate}
\item $\lim\limits_{x\rightarrow a} g(x)=\infty$;
\item $g'(x)\neq 0,\forall x\in U^\circ(a,\delta)$;
\item $\lim\limits_{x\rightarrow a} \frac{f'(x)}{g'(x)}=l$($l$ 为有限数或 $\pm\infty,\infty$);
\end{enumerate}
则有
\begin{equation}
\lim\limits_{x\rightarrow a}\frac{f(x)}{g(x)}=\lim\limits_{x\rightarrow a}\frac{f'(x)}{g'(x)}=l~.
\end{equation}
上面的 $a$ 也可以取为 $+\infty$ 或 $-\infty$,定理仍然成立。
\end{theorem}
\begin{exercise}{}
计算 $\lim\limits_{x\rightarrow 0}\frac{\sin x}{x}$。
\end{exercise}
直接对分子分母同时求导就可以求得:
\begin{equation}
    \lim\limits_{x\rightarrow 0}\frac{\sin x}{x}=\lim\limits_{x\rightarrow 0}\frac{\cos x}{1}=1~,
\end{equation}
这告诉我们 $\sin x$ 和 $x$ 在 $x\rightarrow 0$ 时是等价无穷小量(需引用文章)。

\begin{exercise}{}
计算 $\lim\limits_{x\rightarrow 0}\frac{e^x-1-\sin x}{e^{\sin x} - \cos x - x}$。
\end{exercise}
提示:用洛必达法则,对分子和分母同时求两次导,答案为 $0.5$。

这里我们也可以通过对分子分母作二阶近似来计算。利用 $e^x=1+x+x^2/2+\order{x^2},\sin x=x+\order{x^2},\cos x=1-x^2/2+\order{x^2}$ 对原式进行化简:
\begin{equation}
\begin{aligned}
&e^{\sin x}=e^{x+\order{x^2}}=1+x+x^2/2+\order{x^2}~,\\
&\lim\limits_{x\rightarrow 0}\frac{e^x-1-\sin x}{e^{\sin x} - \cos x - x}=\lim\limits_{x\rightarrow 0}\frac{x^2/2+\order{x^2}}{x^2+\order{x^2}}=\frac{1}{2}~.
\end{aligned}
\end{equation}

% Giacomo: 应该移动到《泰勒级数/展开》
在这个例子中,用一次洛必达法则不再能满足我们的要求,于是我们用了第二次洛必达法则,对分子分母再次求导。这对应着将分子分母的函数用关于 $x$ 的二阶近似公式来表示。从这里我们能看出洛必达法则与泰勒展开公式的联系。事实上,带皮亚诺余项的泰勒展开式可以轻易地由洛必达法则得到。

\addTODO{文章:big O记号}