% 弱相互作用
% license CCBYSA3
% type Wiki

(本文根据 CC-BY-SA 协议转载自原搜狗科学百科对英文维基百科的翻译)

\begin{figure}[ht]
\centering
\includegraphics[width=6cm]{./figures/088ab23b36615a09.png}
\caption{放射性β衰变是由于弱相互作用,这将中子转化为质子、电子和反电子中微子。} \label{fig_RXHZY_1}
\end{figure}
在粒子物理学中,\textbf{弱相互作用}(通常也称为\textbf{弱力}或者\textbf{弱核力})是亚原子粒子之间的负责原子放射性衰变的相互作用机制。弱相互作用在核裂变中起着重要作用,描述弱相互作用的行为和效应的理论有时被称为\textbf{量子味动力学(QFD)}。然而,QFD这个术语很少被使用,因为电弱理论(EWT)是对弱相互作用更好的理解。[1]除此之外,QFD还与处理强相互作用的量子色动力学 (QCD)以及处理电磁力的量子电动力学 (QED)有关。

弱力的有效范围限于亚原子距离,并且小于质子的直径。它是自然界四种已知的力相关的基本相互作用之一,与强相互作用、电磁相互作用和万有引力并列。

\subsection{背景}
粒子物理学的标准模型为理解电磁相互作用、弱相互作用和强相互作用提供了一个统一的框架。当两个粒子(通常但不一定是自旋为半整数的费米子)互相交换自旋为整数的传递力的玻色子时,相互作用便发生了。参与这种交换的费米子可以是基本粒子(例如电子或夸克)也可以是复合粒子(例如质子或中子),尽管在最深的层面上,所有弱相互作用最终都发生在基本粒子之间。

在弱相互作用的情况下,费米子可以交换三种不同类型的作用力载体,称为W+,W−,和Z玻色子。每种玻色子的质量都远远大于质子或中子的质量,这与弱相互作用的短程性质相一致。事实上,这种力之所以被称为弱就是因为它在给定距离上的场强通常比强力或电磁力小几个数量级。

构成中子和质子等复合粒子的夸克有六种“味”,分别是上、下、粲、奇、顶、底,它们赋予这些复合粒子各自的性质。弱相互作用的独特之处在于它允许夸克改变味。这些性质的交换由力载体玻色子所介导。例如,在β−衰变过程中,中子内的下夸克转化为上夸克,从而将中子转化为质子,并产生电子和反电子中微子。

弱相互作用是唯一破坏宇称对称性的基本相互作用,类似地,它也是唯一破坏电荷宇称对称性的相互作用。

涉及弱相互作用现象的其他重要例子包括β衰变以及为太阳的热核过程提供动力的氢融合成氦的聚变过程。随着时间的推移,大多数费米子会因弱相互作用而衰变。这种衰变使得放射性碳定年成为可能,因为碳-14会通过弱相互作用衰变为氮-14 。它还可以产生辐射发光,常用于氚照明以及射线电池的相关领域。[2]

在早期宇宙的夸克时期,电弱力分离为电磁力和弱力。

\subsection{历史}
1933年,恩利克·费米提出了第一个弱相互作用理论,被称为费米相互作用。他认为$\beta$衰变可以用四费米子相互作用来解释,涉及到一个没有作用范围的接触力。[3][4]

然而,它更好地被描述为一个具有有限范围的非接触力场,尽管这个作用范围非常短。1968年,谢尔登·格拉肖、阿卜杜勒·萨拉姆和史蒂芬·温伯格将电磁力和弱相互作用统一起来,证明它们是单个力的两个方面,这个力现在被称为电弱力。[5][6]

$W$及$Z$玻色子的存在直到1983年才被直接证实。[7]

\subsection{性质}
\begin{figure}[ht]
\centering
\includegraphics[width=8cm]{./figures/1d5785bb21584598.png}
\caption{该图描述了由带电的弱相互作用所引起的各种衰变路线以及关于它们发生的可能性的一些指示。线的强度由 CKM 参数给出。} \label{fig_RXHZY_2}
\end{figure}

弱相互作用在许多方面是独特的:
\begin{itemize}
\item 这是唯一能够改变夸克的味的相互作用(即将一种夸克变成另一种夸克)。
\item 这是唯一违反\textbf{P对称性}或者宇称对称的相互作用,也是唯一一种违反电荷宇称\textbf{CP}对称的相互作用。
\item 它由具有显著质量的力载体粒子所介导(传播),这是一个不寻常的特征,在标准模型中由希格斯机制所解释。
\end{itemize}
由于它们的大质量(大约$90 GeV/c$2[8]),这些被称为$W$及$Z$玻色子的力载体粒子的寿命很短,只有不到$10^{-24}$秒。[9]弱相互作用的耦合常数(相互作用强度的指标)在$10^{-7}$和$10^{-6}$之间,而强相互作用的耦合常数和电磁耦合常数的数量级分别为$1$和$10^{-2}$[10],因此弱相互作用在强度方面是“弱”的。[11]弱相互作用的有效范围非常小,大约为$10^{-17}$到$10^{-16}$ m。[11][10]在距离大约$10^{-18}$米时,弱相互作用的强度与电磁力的强度相似,但随着距离的增加,弱相互作用开始以指数级减少,在大约$3\times10^{-17}$米的距离上(只放大了一个半数量级)弱相互作用就会减弱10,000倍。[12]

弱相互作用影响了标准模型的所有费米子,以及希格斯玻色子;中微子只参与重力和弱相互作用,而中微子正是弱力这个名字的来源。[11]弱相互作用既不产生束缚态也不涉及结合能,后两者对于重力来说出现在天文尺度上,对电磁力来说出现在原子水平上,对强核力来说则出现在原子核内。[13]

弱相互作用影响了标准模型的所有费米子,以及希格斯玻色子;中微子只参与重力和弱相互作用,而中微子正是弱力这个名字的来源。[11]弱相互作用既不产生束缚态也不涉及结合能,后两者对于重力来说出现在天文尺度上,对电磁力来说出现在原子水平上,对强核力来说则出现在原子核内。[13]

它最显著的效果是来源于它的第一个独特特征:带电的弱相互作用导致的变味。例如,中子比质子(前者的姊妹核子)重,但它不能在不将两个下夸克中的一个变味为上夸克的情况下衰变为质子。强相互作用和电磁相互作用都不允许发生味的改变,所以中子衰变为质子的过程是通过\textbf{弱衰变}完成的;如果没有弱衰变,奇异性和粲性(与同名夸克相关联)等夸克的性质在所有相互作用中都会是守恒的。

由于弱衰变,所有的介子都是不稳定的。[14]在被称为β衰变的过程中,中子中的一个下夸克可以通过发射一个虚W−玻色子而变成一个上夸克,这个W−玻色子接着则会变成一个电子和一个反电子中微子。[15]另一个例子是电子俘获,它是放射性衰变的一个常见变体,在这个过程中,原子中的质子和电子发生相互作用,产生一个中子(即质子中的一个上夸克变成下夸克),并发射一个电子中微子。

由于W玻色子的大质量,依赖于弱相互作用的粒子变换或衰变(例如涉及到味变化的过程)通常比仅依赖于强相互作用或电磁相互作用的变换或衰变发生得要慢得多。例如,一个中性$\pi$介子的衰变涉及的是电磁相互作用,因此寿命只有大约$10^{-16}$秒。与之相比,带电π介子只能通过弱相互作用发生衰变,因此寿命约为$10^{-8}$秒,这比中性$\pi$介子的寿命长了一亿倍。[16]一个特别极端的例子是自由中子的弱力衰变,大约需要15分钟。[15]

\subsubsection{3.1 弱同位旋和弱超荷}
\begin{figure}[ht]
\centering
\includegraphics[width=14.25cm]{./figures/f83f3020026b21ee.png}
\caption\label{fig_RXHZY_4}
\end{figure}

所有粒子都有一个叫做弱同位旋(符号为 $T_3$)的性质,它是一个量子数,并控制粒子在弱相互作用中的行为。弱同位旋在弱相互作用中扮演的角色与电荷在电磁相互作用中扮演的角色相同。所有的左手费米子的弱同位旋值为 $+\frac{1}{2}$ 或 $-\frac{1}{2}$。例如,上夸克的弱同位旋为 $+\frac{1}{2}$,下夸克则为 $-\frac{1}{2}$。夸克从不会通过弱相互作用衰变为拥有相同弱同位旋的夸克:弱同位旋为 $+\frac{1}{2}$ 的夸克只会衰变为弱同位旋为 $-\frac{1}{2}$ 的夸克,反之亦然。
\begin{figure}[ht]
\centering
\includegraphics[width=6cm]{./figures/ba9df2f205a6dab5.png}
\caption{$\pi+$通过弱相互作用发生衰变} \label{fig_RXHZY_5}
\end{figure}
在任何给定的相互作用中,弱同位旋都是守恒的:进入相互作用的粒子的弱同位旋数之和等于从该相互作用中离开的粒子的弱同位旋数之和。例如,一个弱同位旋值为+1的(左手)$\pi^+\text{通常衰变为一个}\nu_\mu\text{(弱同位旋为}+\frac{1}{2}\text{)和一个}\mu^+\text{(作为右手反粒子,弱同位旋为}+\frac{1}{2}$)。

随着电弱理论的发展,另一个被称为“弱超荷”的性质得到了发展。它取决于粒子的电荷和弱同位旋,其定义为:
$$Y_{W}=2(Q-T_{3})~$$
在这个式子中,$Y_{W}$是给定类型的粒子的弱超荷,$Q$是它的电荷(单位为基本电荷),$T_{3}$是它的弱同位旋。有些粒子的弱同位旋为零,但所有自旋为$\frac{1}{2}$的粒子都有非零的弱超荷。弱超荷是电弱规范群中$U(1)$分量的生成元。

\subsection{ 相互作用类型}
弱相互作用有两种类型(称为顶点)。第一种类型称为“带电流相互作用”,因为它是由携带着非零电荷的$W^{+}$玻色子或者$W^{-}$玻色子所介导的,并对$\beta$衰变现象负责。第二种类型被称为“中性流相互作用”,因为它是由中性的 $Z$ 玻色子所介导的。这两种类型的相互作用遵循不同的选择定则。

\subsubsection{4.1 带电流相互作用}
\begin{figure}[ht]
\centering
\includegraphics[width=6cm]{./figures/3b2f9e698de6f000.png}
\caption{中子经由β衰变产生质子、电子和反电子中微子的费曼图,中间涉及到一个大质量的W−玻色子。} \label{fig_RXHZY_6}
\end{figure}

在一种类型的带电流相互作用中,一个带电的轻子(例如电荷为$-1$的电子或 $\mu$ 子)可以吸收一个 $W^+$ 玻色子(电荷为+1的粒子)并转变成一个相应的中微子(电荷为$0$),这个中微子的类型(“味”,即电子型、$\mu$ 子型或 $\tau$ 子型)与参与相互作用的那个轻子的类型相同,例如:
$$\mu^- + W^+ \rightarrow \nu_\mu~$$
类似地,一个下型夸克(符号为d,电荷量为$-\frac{1}{3}$)可以通过发射一个$W^-$玻色子或吸收一个$W^+$玻色子而转换成一个上型夸克(符号为u,电荷量为$+\frac{2}{3}$)。更准确地说,这个下型夸克变成了上型夸克的一个量子叠加态:也就是说,它有可能成为三种上型夸克中的任何一种,相应的概率出现在CKM矩阵表中。相反,一个上型夸克可以发射一个$W^+$玻色子或者吸收一个$W^-$玻色子,从而转化为一个下型夸克,例如:
\begin{equation}
\begin{aligned}
d &\rightarrow u + W^{-} \\\\
d + W^{+} &\rightarrow u \\\\
c &\rightarrow s + W^{+} \\\\
c + W^{-} &\rightarrow s
\end{aligned}~
\end{equation}
W玻色子是不稳定的,它会快速衰变,寿命非常短。例如:
\addTODO{公式}

$W$玻色子也可以衰变成其他产物,对应的概率各不相同。

在所谓的中子$\beta$衰变中(如上图所示),中子的一个下夸克发射出了一个虚的$W^-$玻色子,并转化为一个上夸克,这导致中子被转化成质子。由于过程中涉及的能量(即下夸克和上夸克之间的质量差),$W^-$玻色子只能转化成一个电子和一个反电子中微子。设在旁边,这个过程可以被表示为:
$$d \rightarrow u + e^{-} + \bar{\nu}_e~$$

\subsubsection{4.2 中性流相互作用}
在中性流相互作用中,一个夸克或一个轻子(例如电子或μ子)发射或吸收一个中性的Z玻色子。例如:
\begin{equation}
e^- \rightarrow e^- + Z^0~
\end{equation}
像W玻色子一样,Z玻色子也会迅速发生衰变,[18]例如:
\begin{equation}
Z^0 \rightarrow b + \bar{b}~
\end{equation}
带电流相互作用的选择定则受到手性和弱同位旋的严格限制,与之不同的是,Z玻色子所介导的中性流相互作用涉及标准模型中的任何费米子,包括左手和右手的粒子和反粒子,虽然相互作用的强度有所差异。

\subsection{电弱理论}
粒子物理学的标准模型将电磁相互作用和弱相互作用描述为单个电弱相互作用的两个不同方面。这个理论大约是在1968年由谢尔登·格拉肖、阿卜杜勒·萨拉姆和史蒂芬·温伯格发展起来的,他们因其工作获得了1979年的诺贝尔物理学奖。[20]希格斯机制为三种大质量规范玻色子($W^{+}$,$W^{-}$,$Z^{0}$,即弱相互作用的三种载体)以及无质量的光子($Y$,电磁相互作用的载体)的存在提供了解释。[21]

根据电弱理论,在非常高的能量下,宇宙有四个希格斯场的分量,它们的相互作用由四种无质量的规范玻色子所介导,每种都类似于光子,这形成了一个复标量希格斯场二重态。然而,在低能量下,这种规范对称性自发破裂到电磁学的$U(1)$对称性,因为希格斯场之一获得了真空期望值。这种对称性破缺预计会产生三种无质量玻色子,但相反,它们会被其他三个场所整合并通过希格斯机制获得质量。这三个玻色子整合产生了负责弱相互作用的$W^{+}$,$W^{-}$和$Z^{0}$玻色子。第四个规范玻色子是负责电磁相互作用的光子,它保持无质量状态。[21]

这个理论已经给出了许多预测,包括在Z玻色子和W玻色子被发现之前对它们的质量的预测。2012年7月4日,大型强子对撞机的CMS和ATLAS实验团队独立宣布:他们已经确认了对一个先前未知的质量介于$125-127GeV/c^{2}$之间的玻色子的正式发现。这种粒子的行为迄今为止与希格斯玻色子“是一致的”,但他们谨慎地补充说,在确定新玻色子是某种类型的希格斯玻色子之前,还需要进一步的数据和分析。到2013年3月14日之前,希格斯玻色子被初步确认存在。[22]

如果电弱对称性破缺的能标降低,未破裂的$SU(2)$相互作用最终会受限。$SU(2)$在该能标以上变得受限的替代模型在较低能量下定量地类似于标准模型,但在对称性破缺以上则显著不同。[23]

\subsection{违反对称性}
\begin{figure}[ht]
\centering
\includegraphics[width=10cm]{./figures/c2246516e64858ee.png}
\caption{左手和右手粒子 : p是粒子的动量,S是粒子的自旋。请注意这两个状态之间缺乏反射对称性。} \label{fig_RXHZY_7}
\end{figure}
自然法则长期以来被认为在镜面反射下保持不变。通过镜子观察实验的结果预计与实验装置的镜面反射副本所对应的结果相同。这个所谓的宇称守恒定律已知在万有引力、电磁相互作用和强相互作用的情况下是成立的;它被认为是一项普遍的定律。[24]然而,在20世纪50年代中期杨振宁和李政道提出弱相互作用可能违反这个定律。吴健雄和她的合作者在1957年发现弱相互作用违反了宇称守恒,这使得杨振宁和李政道获得了1957年诺贝尔物理学奖。[25]

虽然弱相互作用曾经由费米理论所描述,但宇称破缺和重正化理论的发现表明我们需要一种新的方法。1957年,罗伯特·马沙克和乔治·苏达山以及稍晚时候的理查德·费曼和默里·盖尔曼提出了一个\textbf{V-A}(矢量减轴矢量或左手)拉格朗日量,用于描述弱相互作用。在这个理论中,弱相互作用只作用于左手粒子(以及右手反粒子)。由于左手粒子的镜面反射是右手粒子,这解释了宇称的最大违反。\textbf{V-A}理论是在Z玻色子被发现之前就被发展起来的,所以它没有包括进入中性流相互作用的右手场。

然而,这个理论允许\textbf{CP}这个复合对称性的存在。\textbf{CP}结合了宇称\textbf{P}(从左到右切换)和荷共轭\textbf{C}(用反粒子交换粒子)。当1964年詹姆斯·克罗宁和瓦尔·菲奇提供了$CP$对称性在$K$介子衰变中也可能破缺的明确证据时,物理学家再次感到惊讶,这为他们赢得了1980年的诺贝尔物理学奖。[26]1973年,小林诚和益川敏英证明$CP$违反在弱相互作用中需要两代以上的粒子,[27]这有效地预测了当时未知的第三代粒子的存在。这一发现为他们赢得了一半的2008年诺贝尔奖物理学奖。[28]

与宇称违背不同,$CP$违背只在有限的情况下发生。尽管它很罕见,但人们普遍认为它是宇宙中物质比反物质多得多的原因,因此形成了安德烈·萨哈罗夫的三个重子生成条件之一。[29]

\subsection{参考文献}
[1]
^Griffiths, David (2009). Introduction to Elementary Particles. pp. 59–60. ISBN 978-3-527-40601-2..

[2]
^"The Nobel Prize in Physics 1979: Press Release". NobelPrize.org. Nobel Media. Retrieved 22 March 2011..

[3]
^Fermi, Enrico (1934). "Versuch einer Theorie der β-Strahlen. I". Zeitschrift für Physik A. 88 (3–4): 161–177. Bibcode:1934ZPhy...88..161F. doi:10.1007/BF01351864..

[4]
^Wilson, Fred L. (December 1968). "Fermi's Theory of Beta Decay". American Journal of Physics. 36 (12): 1150–1160. Bibcode:1968AmJPh..36.1150W. doi:10.1119/1.1974382..

[5]
^"Steven Weinberg, Weak Interactions, and Electromagnetic Interactions"..

[6]
^"1979 Nobel Prize in Physics". Nobel Prize. Archived from the original on 7 July 2014..

[7]
^Cottingham & Greenwood (1986,2001),第8页.

[8]
^W.-M. Yao et al. (Particle Data Group) (2006). "Review of Particle Physics: Quarks" (PDF). Journal of Physics G. 33: 1–1232. arXiv:astro-ph/0601168. Bibcode:2006JPhG...33....1Y. doi:10.1088/0954-3899/33/1/001..

[9]
^Peter Watkins (1986). Story of the W and Z. Cambridge: Cambridge University Press. p. 70. ISBN 978-0-521-31875-4..

[10]
^"Coupling Constants for the Fundamental Forces". HyperPhysics. Georgia State University. Retrieved 2 March 2011..

[11]
^J. Christman (2001). "The Weak Interaction" (PDF). Physnet. Michigan State University. Archived from the original (PDF) on 20 July 2011..

[12]
^"Electroweak". The Particle Adventure. Particle Data Group. Retrieved 3 March 2011..

[13]
^Walter Greiner; Berndt Müller (2009). Gauge Theory of Weak Interactions. Springer. p. 2. ISBN 978-3-540-87842-1..

[14]
^Cottingham & Greenwood (1986,2001),第29页。注意。然而,中性π介子会电磁衰变,当它们的量子数允许时,几个介子大多会强烈衰变。.

[15]
^Cottingham & Greenwood (1986,2001),第28页.

[16]
^Cottingham & Greenwood (1986,2001),第30页.

[17]
^Baez, John C.; Huerta, John (2009). "The Algebra of Grand Unified Theories". Bull. Am. Math. Soc. 0904: 483–552. arXiv:0904.1556. Bibcode:2009arXiv0904.1556B. doi:10.1090/s0273-0979-10-01294-2. Retrieved 15 October 2013..

[18]
^K. Nakamura et al. (Particle Data Group) (2010). "Gauge and Higgs Bosons" (PDF). Journal of Physics G. 37. Bibcode:2010JPhG...37g5021N. doi:10.1088/0954-3899/37/7a/075021..

[19]
^K. Nakamura et al. (Particle Data Group) (2010). "n" (PDF). Journal of Physics G. 37: 7. Bibcode:2010JPhG...37g5021N. doi:10.1088/0954-3899/37/7a/075021..

[20]
^"The Nobel Prize in Physics 1979". NobelPrize.org. Nobel Media. Retrieved 26 February 2011..

[21]
^C. Amsler et al. (Particle Data Group) (2008). "Review of Particle Physics – Higgs Bosons: Theory and Searches" (PDF). Physics Letters B. 667: 1–6. Bibcode:2008PhLB..667....1A. doi:10.1016/j.physletb.2008.07.018..

[22]
^"New results indicate that new particle is a Higgs boson | CERN". Home.web.cern.ch. Retrieved 20 September 2013..

[23]
^Claudson, M.; Farhi, E.; Jaffe, R. L. (1 August 1986). "Strongly coupled standard model". Physical Review D. 34 (3): 873–887. doi:10.1103/PhysRevD.34.873. Retrieved 4 February 2019..

[24]
^Charles W. Carey (2006). "Lee, Tsung-Dao". American scientists. Facts on File Inc. p. 225. ISBN 9781438108070..

[25]
^"The Nobel Prize in Physics 1957". NobelPrize.org. Nobel Media. Retrieved 26 February 2011..

[26]
^"The Nobel Prize in Physics 1980". NobelPrize.org. Nobel Media. Retrieved 26 February 2011..

[27]
^M. Kobayashi; T. Maskawa (1973). "CP-Violation in the Renormalizable Theory of Weak Interaction" (PDF). Progress of Theoretical Physics. 49 (2): 652–657. Bibcode:1973PThPh..49..652K. doi:10.1143/PTP.49.652. hdl:2433/66179..

[28]
^"The Nobel Prize in Physics 1980". NobelPrize.org. Nobel Media. Retrieved 17 March 2011..

[29]
^Paul Langacker (2001) [1989]. "Cp Violation and Cosmology". In Cecilia Jarlskog. CP violation. London, River Edge: World Scientific Publishing Co. p. 552. ISBN 9789971505615..