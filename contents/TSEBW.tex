% 时谐电磁波
% 时谐电磁波|亥姆霍兹方程

\pentry{电场波动方程\upref{EWEq}}
在没有电荷存在的真空中所出现的电磁场称为\textbf{电磁波}。此时,电磁场 $\bvec E,\bvec B$ 波动方程\upref{EWEq}为
\begin{equation}\label{TSEBW_eq2}
\begin{aligned}
\laplacian\bvec E-\frac{1}{c^2}\pdv[2]{\bvec E}{t}=0\\
\laplacian\bvec B-\frac{1}{c^2}\pdv[2]{\bvec B}{t}=0
\end{aligned}
\end{equation}

实际情况中,电磁波的激发源往往以大致确定的频率做正弦振荡,其辐射出的电磁波也以相同频率作正弦振荡,这种以一定频率作正弦振荡的电磁波称为\textbf{时谐电磁波(单色波)}。一般情况下,电磁波不是单色波,此时可通过频谱分析方法分解为不同单色波的叠加。由此可见,时谐电磁波在这里起着最基本的作用,这正如简谐振动在机械振动里所起的作用一样。

设角频率为 $\omega$,则时谐电磁波中电磁场对时间依赖关系为 $\cos\omega t$,可用复数形式表示
\begin{equation}\label{TSEBW_eq1}
\begin{aligned}
\bvec E(\bvec r,t)=\bvec E(\bvec r)\E^{-\I\omega t}\\
\bvec B(\bvec r,t)=\bvec B(\bvec r)\E^{-\I\omega t}
\end{aligned}
\end{equation}
\autoref{TSEBW_eq1} 代入\autoref{TSEBW_eq2} 得
\begin{equation}\label{TSEBW_eq3}
\begin{aligned}
\laplacian\bvec E+k^2\bvec E=0\\
\laplacian\bvec B+k^2\bvec B=0
\end{aligned}
\end{equation}
其中, $k=\omega/c$ 。方程\autoref{TSEBW_eq3} 是电磁场中的亥姆霍兹方程\autoref{RHM_eq2}~\upref{RHM},它是时谐电磁波的基本方程。