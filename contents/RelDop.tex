% 光的多普勒效应
% keys 相对论|光速|多普勒|频率|波长
% license Xiao
% type Tutor
%注意不可以无脑用尺缩效应来推导,因为同时性的相对性。
%基本完成

\pentry{多普勒效应\nref{nod_Dopler},洛伦兹变换\nref{nod_SRLrtz}}{nod_47e2}

约定采用\enref{自然单位制}{NatUni},即 $c=1$。

不论是牛顿力学框架下还是相对论意义下,光的频率都可能在不同参考系之间有所不同。由于观察者和光源之间相对运动,造成观察者所测得的光的频率与光源的频率不一致,这个现象被称为光的多普勒效应。由于光速不变原理,在相对论框架下讨论光的多普勒效应反而更为简单。


\subsection{光的多普勒效应的推导}
由于在任何参考系中,光的速度都一样,因此只需要知道光的波长就可以根据以下公式得到光的频率:
\begin{equation}
f=\frac{1}{\lambda}~.
\end{equation}
其中,$f$ 是光的频率,$\lambda$ 是光的波长,$1$ 是光速。

假设光源所在的参考系为 $K_1$,观察者所在的参考系为 $K_2$,而 $K_2$ 相对 $K_1$ 的运动速度为 $\vec{v}=(v, 0, 0)^T$。设光的频率在 $K_1$ 为 $f$,在 $K_2$ 中为 $f'$,那么光的波长在 $K_1$ 中为 $\lambda=1/f$,在 $K_2$ 中为 $\lambda'=1/f'$。

\subsubsection{一维情况}
设在 $K_1$ 看来,观察者和光的运动方向是一致的,即观测者以 $v$ 的速度背离光源运动。

取 $K_1$ 中光的两个相邻的波峰\footnote{也可以是相邻波谷,或者任何相邻的两个同相位的点。},作为两个以光速运动的点。为简化讨论,不妨设其中一个点的世界线\footnote{见\enref{时空的四维表示}{SR4Rep}或\enref{世界线和固有时}{wdline}。}表示为
\begin{equation}
\pmat{t_1\\t_1\\0\\0}~,
\end{equation}
而另一个点为
\begin{equation}
\pmat{t_2\\t_2+\lambda\\0\\0}~.
\end{equation}

代入洛伦兹变换可得,在 $K_2$ 中两波峰的世界线分别为
\begin{equation}
\pmat{\frac{t_1-vt_1}{\sqrt{1-v^2}}\\\frac{t_1-vt_1}{\sqrt{1-v^2}}\\0\\0}~,
\end{equation}
和
\begin{equation}
\pmat{\frac{t_2-v(t_2+\lambda)}{\sqrt{1-v^2}}\\\frac{(t_2+\lambda)-vt_2}{\sqrt{1-v^2}}\\0\\0}=\pmat{\frac{t_2-vt_2-v\lambda}{\sqrt{1-v^2}}\\\frac{t_2-vt_2+\lambda}{\sqrt{1-v^2}}\\0\\0}~.
\end{equation}

在各参考系中计算波长,就要计算对应参考系中同一时间下两波峰的空间坐标的差。对于 $K_1$,“同一时间”意味着 $t_1=t_2$;对于 $K_2$,“同一时间”意味着 $t_1-vt_1=t_2-v(t_2+\lambda)$,

将“同一时间”条件分别代入两个参考系中两个点的世界线,消去 $t_1$ 计算后发现,两波峰在 $K_1$ 中的距离始终是 $\lambda$,而在 $K_2$ 中的距离始终是 $\frac{(1+v)\lambda}{\sqrt{1-v^2}}$。由于已经设定这束光在 $K_2$ 中的波长是 $\lambda'$,因此有
\begin{equation}
\lambda'=\frac{(1+v)\lambda}{\sqrt{1-v^2}}=\frac{\sqrt{1+v}}{\sqrt{1-v}}\lambda~.
\end{equation}

因此在 $K_2$ 中,光的频率变为

\begin{equation}\label{eq_RelDop_2}
f'=\frac{1}{\lambda'}=\frac{\sqrt{1-v}}{\sqrt{1+v}\lambda}=\frac{\sqrt{1-v}}{\sqrt{1+v}}f~.
\end{equation}

这就是一维情况下光的多普勒效应。对于背离光源运动的观测者来说,光的频率发生了红移($f'<f$)。

\subsubsection{三维情况}
三维情况下,可以预料在 $K_1$ 看来光的传播方向和在 $K_2$ 参考系看来的运动方向不一致的情况。因此情况会较为复杂,我们需要考虑新的方法。

考察\autoref{fig_RelDop_1} 的情况,我们直接在\textbf{$K_2$ 参考系中}进行分析,不妨假设光源 $S$ 正以 $v$ 的速度沿 $x$ 轴正方向运动,并且在 $K_2$ 参考系看来在 $t$ 时刻接收到的电磁波方向与 $x$ 轴成 $\phi$ 角(注意这个角度是在 \textbf{$K_2$ 参考系}测得的)。
\begin{figure}[ht]
\centering
\includegraphics[width=14cm]{./figures/2494c68084ba1ed9.png}
\caption{多普勒效应示意图(参考舒幼生力学\cite{舒幼生}8.3.2节例11的图)} \label{fig_RelDop_1}
\end{figure}

假设在 $0$ 时刻光源位于 $P_0$ 点,它发出的电磁波于 $t$ 时刻被 $B$ 点处的观测者接收到,并且观测者测得的电磁波方向与 $x$ 轴夹角为 $\phi$。那么 $S$ 到 $B$ 的距离为 $r_0=ct=t$(自然单位制下 $c=1$)。而经过 $\dd t$ 的时间后光源移动到了 $P$ 点的位置,在这段时间内光源所经历的固有时(也就是 $K_1$ 参考系中光源所经历的原时)为 $\dd t/\gamma=\sqrt{1-v^2}\dd t$(即钟慢效应),因此在这段时间内光源“振动”的次数为
\begin{equation}\label{eq_RelDop_3}
N'=f\cdot \gamma \dd t = f\sqrt{1-v^2}\dd t~,
\end{equation}
现在考虑光源在 $P$ 点处发出的电磁波在什么时刻被 $B$ 点接受到。如图所示,$P$ 到 $B$ 的距离为
\begin{equation}
r=r_0-v\cos\phi \dd t=t-v\cos\phi\dd t~,
\end{equation}
这也就意味着光源从 $P$ 发出的电磁波经过了 $r/c=r$(自然单位制下 $c=1$) 的时间被 $B$ 点的观测者接收到。经过计算,可以看到 $P_0$ 点发出的电磁波与 $P$ 点发出的电磁波被 $B$ 点观测者接收到的时间差为
\begin{equation}\label{eq_RelDop_4}
\dd t' = \dd t + r - t = \dd t -v\cos\phi\dd t~.
\end{equation}
因此根据\autoref{eq_RelDop_3} 和\autoref{eq_RelDop_4},$K_2$ 参考系中 $B$ 点观测者观测到的光的频率为
\begin{equation}\label{eq_RelDop_1}
f'=\frac{N'}{\dd t'}=\frac{\sqrt{1-v^2}}{1-v\cos\phi} f~.
\end{equation}

这就是三维空间中一般情况下光的多普勒效应。如果 $\phi=0$ 或 $\phi=\pi$,那么情况退化为一维空间的问题。作为验证将 $\phi=\pi$ 代入\autoref{eq_RelDop_1} 后结果和\autoref{eq_RelDop_2} 一致,这种情况对应于光源背离观测者运动(或观测者背离光源运动)的情况。

另一个更具数学技巧和简洁性的推导是利用了波四矢 $k^\mu$ 的变换规则,具体过程见\enref{狭义相对论运动学(无质量粒子)}{SRkmls}。
