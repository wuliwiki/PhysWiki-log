% 加布里埃尔·李普曼(综述)
% license CCBYSA3
% type Wiki

本文根据 CC-BY-SA 协议转载翻译自维基百科\href{https://en.wikipedia.org/wiki/Gabriel_Lippmann}{相关文章}。

乔纳斯·费迪南德·加布里埃尔·李普曼(1845年8月16日-1921年7月12日)是一位出生于卢森堡的法国物理学家和发明家,因其“基于干涉现象的色彩摄影再现方法”而于1908年获得诺贝尔物理学奖。[2]
\subsection{早年生活和教育}
\begin{figure}[ht]
\centering
\includegraphics[width=6cm]{./figures/2135552cea5927ab.png}
\caption{1908年的李普曼} \label{fig_LPM_1}
\end{figure}
加布里埃尔·李普曼于1845年8月16日出生在卢森堡的博讷沃伊(卢森堡语:Bouneweg)。当时,博讷沃伊是霍勒里奇(卢森堡语:Hollerech)市镇的一部分,这个地方常被认为是他的出生地。(如今,博讷沃伊和霍勒里奇都是卢森堡市的区。)他的父亲伊萨耶(Isaïe)是一位出生在梅斯附近恩内里的法国犹太人,经营着位于博讷沃伊的旧修道院内的家族手套制造生意。1848年,李普曼一家搬到了巴黎,在那里李普曼最初由母亲米里亚姆·罗斯(Lévy)辅导,之后进入拿破仑中学(现为亨利四世中学)。据说他是一个注意力不集中但富有思考的学生,对数学特别感兴趣。1868年,他进入巴黎高等师范学院学习,但未通过能够使他进入教师职业的聚集考试,而是选择专攻物理学。1872年,法国政府派他前往海德堡大学,在古斯塔夫·基尔霍夫的鼓励下,他专攻电学,并于1874年获得“优等”博士学位。李普曼于1875年回到巴黎,继续学习,直到1878年成为索邦大学的物理学教授。在索邦大学,他教授声学和光学课程。
\subsection{职业生涯}  
\begin{figure}[ht]
\centering
\includegraphics[width=6cm]{./figures/aac5aa28858d1eac.png}
\caption{李普曼教授在索邦大学物理研究实验室(索邦大学图书馆,NuBIS)} \label{fig_LPM_2}
\end{figure}
李普曼在多年的职业生涯中对多个物理学领域做出了几项重要贡献。
\subsubsection{毛细电计}
\begin{figure}[ht]
\centering
\includegraphics[width=6cm]{./figures/1ca13b737ad1efd4.png}
\caption{李普曼电计(1872年)} \label{fig_LPM_3}
\end{figure}
李普曼的早期发现之一是电现象与毛细现象之间的关系,这使他能够发明一种敏感的毛细电计,后来被称为李普曼电计,并被用于第一台心电图(ECG)机。在1883年1月17日,约翰·G·麦肯德里克在向格拉斯哥哲学学会报告时描述了该设备,内容如下:

李普曼的电计由一根1米长、直径7毫米的普通玻璃管组成,管两端开放,并通过坚固的支架保持竖直位置。下端被拉成一个毛细点,直到毛细管的直径为0.005毫米。管内充满了水银,毛细点浸入稀硫酸中(体积比为1:6的水),并且在含酸的容器底部有少量水银。每根管内的水银都与一根铂金线连接,最后还设置了使毛细点能通过显微镜(放大250倍)观察到的装置。这样的一种仪器非常敏感;李普曼指出,它能够测量出极小的电势差,甚至可以达到1/10,080 Daniell电池的电势差。因此,它是一种非常精密的观察和(通过补偿法可进行标定)测量微小电动势的工具。[10][11]

李普曼的博士论文,提交给索邦大学于1875年7月24日,研究的是电毛细现象。[12]
\subsubsection{压电效应}
1881年,李普曼预言了反向压电效应。[13]
\subsubsection{彩色摄影}
\begin{figure}[ht]
\centering
\includegraphics[width=6cm]{./figures/7830ee920efd30fa.png}
\caption{李普曼在1890年代拍摄的一张彩色照片。它不含任何颜料或染料。} \label{fig_LPM_4}
\end{figure}
最重要的是,李普曼因发明了一种通过摄影重现颜色的方法,该方法基于干涉现象,这使他获得了1908年的诺贝尔物理学奖。

1886年,李普曼开始关注如何将太阳光谱的颜色固定在摄影板上。1891年2月2日,他向科学院宣布:“我成功地将光谱的图像及其颜色固定在摄影板上,且图像保持不变,可以在日光下保持而不受损坏。” 到1892年4月,他能够报告称,他已成功制作出彩色图像,展示了彩色玻璃窗、一组旗帜、一碗橙子上面放着一朵红色罂粟花以及一只五彩斑斓的鹦鹉。他在1894年和1906年分别向科学院提交了两篇论文,介绍了他使用干涉方法进行彩色摄影的理论。
\begin{figure}[ht]
\centering
\includegraphics[width=8cm]{./figures/3793bc8bef519687.png}
\caption{一个驻波。红色点是波节点。} \label{fig_LPM_5}
\end{figure}
光学中的干涉现象是由光的波传播引起的。当给定波长的光被镜子反射回自身时,会产生驻波,就像石头掉入静水中产生的涟漪在水池墙面反射时形成的驻波一样。在普通的非相干光的情况下,驻波仅在靠近反射表面的一个显微薄的空间内是明显的。

Lippmann利用了这一现象,将图像投射到一种特殊的摄影板上,这种板能够记录比可见光波长更小的细节。光通过支撑玻璃板,进入含有亚显微小银盐颗粒的极薄且几乎透明的摄影乳剂。与乳剂紧密接触的液态汞临时镜面反射光线,创造出驻波,其中节点几乎没有影响,而反节点则产生了潜像。经过显影后,得到的是一种层状结构,一种由亚显微银颗粒组成的非常精细的条纹图案,呈现出并行层次的形式,永久记录了驻波。在整个乳剂中,层状结构的间距与所拍摄光线的半波长相对应;λ/(2n),其中λ是空气中光的波长,n是乳剂的折射率。因此,颜色信息是局部存储的。条纹之间的间隔越大,记录的图像颜色波长越长,红色具有最长的波长。

完成后的片子从正面几乎垂直地照射,使用白天光或其他包含可见光谱全波长的白光源。在片子的每个点,波长大约与生成层状结构的光的波长相同的光会强烈地反射回观察者。没有被银颗粒吸收或散射的其他波长的光则会直接穿过乳剂,通常会被显影后的片子背面涂上的黑色抗反射涂层吸收。因此,原始图像形成的光的波长和颜色被重新构建,从而呈现出完整的彩色图像。

实际上,Lippmann工艺并不容易使用。极细颗粒的高分辨率摄影乳剂固有地比普通乳剂不太敏感,因此需要较长的曝光时间。在大光圈镜头和非常明亮的阳光照射下,拍摄非常明亮的物体时,有时可以实现不到一分钟的曝光,但通常需要几分钟的曝光时间。纯光谱色彩再现非常明亮,但现实世界物体反射的波长较宽、界限模糊的宽带则可能出现问题。该过程不会在纸上生产彩色照片,并且通过重新拍摄制作一个好的Lippmann彩色照片的复制品是不可能的,因此每个图像都是独一无二的。为了避免不必要的表面反射,通常会将一个非常浅角度的棱镜粘在完成的片子前面,这使得制作较大尺寸的片子变得不切实际。他早期的照片尺寸为4厘米×4厘米,后来增大为6.5厘米×9厘米。为了获得最佳效果所需的照明和观看安排使得这种方法无法用于日常使用。虽然特殊的片子和带有内置汞储槽的片子支架在大约1900年左右曾短时间商业化销售,但即使是专家用户也发现很难获得一致的好结果,这个过程始终未能从一个科学上优雅的实验室奇趣脱离出来。然而,它确实激发了人们对彩色摄影进一步发展的兴趣。

Lippmann的工艺预示了激光全息摄影,它也基于在摄影介质中记录驻波。Denisyuk反射全息图,通常被称为Lippmann-Bragg全息图,具有类似的层状结构,能够优先反射某些波长。在这种类型的实际多波长彩色全息图中,颜色信息的记录和再现方式与Lippmann工艺完全相同,唯一不同的是,经过记录介质并从物体反射回来的高度相干的激光光线会在一个相对较大的空间体积内产生所需的清晰驻波,从而消除了驻波必须紧贴记录介质的需求。然而,与Lippmann彩色摄影不同的是,激光、物体和记录介质必须在曝光期间稳定到波长的四分之一以内,才能足够有效地记录驻波,或者根本无法记录。
\subsubsection{积分摄影}
在1908年,Lippmann提出了他所称的“积分摄影”方法,其中使用一排紧密排列的小型球面透镜来拍摄场景,记录场景从多个略微不同的水平和垂直位置所看到的图像。当这些图像经过校正并通过类似的透镜阵列观看时,每只眼睛看到的是由所有图像的小部分组成的单一集成图像。眼睛的位置决定了它看到图像的哪些部分。其效果是原始场景的视觉几何形状得到了重建,透镜阵列的边缘似乎成为了一个窗口的边缘,通过这个窗口,场景以真实的大小和三维效果呈现,展示出随着观察者位置的变化而产生的视差和透视变化。[17] 这一使用大量透镜或成像孔径记录被后人称为光场的原理,奠定了光场相机和显微镜技术发展的基础。

当Lippmann于1908年3月展示他“积分摄影”的理论基础时,由于当时缺乏制作具有适当光学特性的透镜屏幕所需的材料,因此无法附上具体的成果。在1920年代,Eugène Estanave使用玻璃Stanhope透镜,Louis Lumière使用赛璐璐材料,进行了一些有前景的实验。[18] Lippmann的积分摄影为三维和动态透镜图像的研究以及彩色透镜工艺奠定了基础。
\subsubsection{时间测量}  
在1895年,Lippmann发展出一种消除时间测量中个人差异的方法,使用了摄影记录,并研究了消除钟摆时钟不规则性的方法,发明了一种比较两个几乎相等周期的钟摆振荡时间的方法。[4]
\subsubsection{共焦镜}  
Lippmann还发明了共焦镜,这是一种天文工具,能够补偿地球的自转,使得可以在没有明显移动的情况下拍摄天空的某个区域。[4]
\subsubsection{布朗随机开关}  
在1900年,他提出了后来被称为布朗随机开关的理论,作为麦克斯韦恶魔的纯机械版,旨在表明气体的动理论与热力学第二定律是不相容的。[19][20]
\subsection{学术 affiliations}
Lippmann自1886年2月8日以来一直是法国科学院的成员,并且在1912年曾担任其主席。[21] 此外,他还是伦敦皇家学会的外籍成员,长时程研究局的成员,[4] 以及卢森堡大公学会的成员。他于1892年成为法国摄影学会成员,并在1896至1899年期间担任该学会的主席。[22] Lippmann还是法国理论与应用光学研究所(Institut d'optique théorique et appliquée)的创始人之一。Lippmann还曾担任法国天文学会(Société Astronomique de France,SAF)主席,任期为1903至1904年。[23]
\subsection{荣誉}  
Lippmann于1881年12月29日被授予法国荣誉军团骑士勋章,1894年4月2日晋升为军官,1900年12月14日晋升为指挥官,并于1919年12月6日晋升为大官员。[24]

在卢森堡市,一所基础科学研究所以Lippmann的名字命名(加布里埃尔·利普曼公共研究中心,Centre de Recherche Public Gabriel Lippmann),并于2015年1月1日与另一重要研究中心合并,形成了新的卢森堡科学与技术研究所(Luxembourg Institute for Science and Technology,LIST)。[25]
\subsection{个人生活}
Lippmann于1888年与小说家维克多·谢尔布利兹(Victor Cherbuliez)的女儿结婚。[4] 1921年7月12日,他在从加拿大返回途中,乘坐法国号邮轮时去世。[26]  
\subsection{参见} 
\begin{itemize}
\item 自立体显示(Autostereoscopy)  
\item 布朗运动棘轮(Brownian ratchet)  
\item 光场相机(Light field camera)  
\item 犹太裔诺贝尔奖得主名单(List of Jewish Nobel laureates) 
\end{itemize} 
\subsection{参考文献}
\begin{enumerate}
\item Gabriel Lippmann 在数学系谱项目(Mathematics Genealogy Project)。
\item “Gabriel Lippmann | French physicist”,2023年8月12日。
\item Hanin Hannouch,《Gabriel Lippmann's colour photography : science, media, museums》,阿姆斯特丹,2022年。ISBN 978-94-6372-855-3。OCLC 1304814408。
\item “Gabriel Lippmann”*,诺贝尔基金会。存档于2016年4月5日,检索于2010年12月4日。
\item Jacques Bintz,《Gabriel Lippmann 1845–1921》,收录于《Gabriel Lippmann: Commémoration par la section des sciences naturelles, physiques et mathématiques de l’Institut grand-ducal de Luxembourg du 150e anniversaire du savant né au Luxembourg, lauréat du prix Nobel en 1908》(卢森堡:卢森堡大公科学院自然、物理与数学科学部门与卢森堡大学科学和医学史研讨会合编,1997年)。检索于2010年12月4日。
\item Josef Maria Eder,《摄影史》(History of Photography),第四版,纽约:多佛出版社,1978年;ISBN 0-486-23586-6,p. 668。(此多佛版为1945年哥伦比亚大学出版社版的再版,最早出版于1932年,德文标题为《摄影史》(Geschichte der Photographie))。
\item 摘自《诺贝尔讲座,物理学1901–1921》,Elsevier出版公司,阿姆斯特丹,1967年。
\item 详见《1908年诺贝尔物理学奖》页面上的长篇传记。
\item Hans I. Bjelkhagen,《Lippmann, Gabriel Jonas (1845–1921) 法国科学家与物理学家》,收录于John Hannavy(编辑)《19世纪摄影百科全书》(Encyclopedia of nineteenth-century photography),第一版,纽约:Routledge,2008年,pp. 132, 320, 647, 808, 862–3, 990–1, 1183, 1433–4。ISBN 978-0-415-97235-2。OCLC 123968757。
\item John G. M'Kendrick,《关于一种对生理学家有用的Lippmann毛细电计简单形式的说明》(Note on a Simple Form of Lippmann's Capillary Electrometer useful to Physiologists)。
\item 德文描述详见《毛细电计》(Kapillārelektromēter),Meyers Konversationslexikon,Verlag des Bibliographischen Instituts,莱比锡与维也纳,1885–1892。检索于2010年12月5日。
\item “关于Gabriel Lippmann”,Gabriel Lippmann研究中心(Centre de Recherche Public – Gabriel Lippmann)。存档于2011年7月22日,检索于2017年9月28日。
\item Lippmann, G. (1881). “电的守恒原理”(Principe de la conservation de l'électricité)*,《化学与物理年鉴》(Annales de chimie et de physique,法文),第24卷,p. 145。
\item Bolas, T. 等,《彩色摄影手册》(A Handbook of Photography in Colours),伦敦:Marion & Co.,1900年,pp. 45–59。(2010年2月11日从archive.org检索)。
\item Wall, E. J.,《实用彩色摄影》(Practical Color Photography),波士顿:美国摄影出版公司,1922年,pp. 185–199。(2010年9月5日从archive.org检索)。
\item Klaus Biedermann,《Lippmann与Gabor在成像上的革命性方法》(Lippmann's and Gabor's Revolutionary Approach to Imaging),诺贝尔奖网站(Nobelprize.org)。检索于2010年12月6日。
\item Lippmann, G. (1908年3月2日)。*“可逆样品:综合摄影”(Épreuves réversibles. Photographies intégrales)*,《法国科学院报告》(Comptes Rendus de l'Académie des Sciences),第146卷,第9期,pp. 446–451。
\item Timby, Kim,《3D与动画柱镜摄影:介于乌托邦与娱乐之间》(3D and Animated Lenticular Photography : Between Utopia and Entertainment),柏林:De Gruyter,2015年,pp. 81–84。ISBN 978-3-11-041306-9。
\item “气体动理论与卡诺原理”(La théorie cinétique des gaz et le principe de Carnot),《数学与物理月刊》(Monatshefte für Mathematik und Physik,法文),第14卷,第1期,A24,1903年12月1日。doi:10.1007/BF01706937。
\item Hoffmann, Peter M (2016年3月1日)。*“分子马达如何从混乱中提取秩序”(How molecular motors extract order from chaos (a key issues review))*,《物理学进展报告》(Reports on Progress in Physics),第79卷,第3期,p. 032601。
\item "Les Membres de l'Académie des sciences depuis sa création (en 1666)"(法文),法国科学院。存档于2008年3月2日,检索于2008年3月1日。
\item Daniel Girardin,《Lippmann的干涉摄影,完美而被遗忘的色彩再现方法》,发表于《DU,文化杂志》第708期:摄影,通向色彩的漫长道路,2000年7-8月,伊丽莎博博物馆(Musée de l'Élysée)。(法文)检索于2010年12月6日。
\item 《法国天文学会通讯》(Bulletin de la Société astronomique de France),1911年,第25卷,pp. 581–586。
\item "LIPPMANN, Jonas Ferdinand Gabriel",Léonore数据库,法国共和国政府。检索于2023年9月24日。
\item 《卢森堡年鉴》2015,出版:Editus,第264页。
\item “Gabriel Lippmann, Scientist, Dies at Sea”,《纽约时报》,1921年7月14日。
\end{enumerate}
\subsection{进一步阅读}
\begin{itemize}
\item Gabriel Lippmann's Colour Photography: Science, Media, Museums,Hanin Hannouch,阿姆斯特丹,2022年。ISBN 978-94-6372-855-3,OCLC 1304814408。
\item J.P. Pier 和 J.A. Massard(编)《Gabriel Lippmann: Commémoration par la section des sciences naturelles, physiques et mathématiques de l’Institut grand-ducal de Luxembourg du 150e anniversaire du savant né au Luxembourg, lauréat du prix Nobel en 1908》, 卢森堡:卢森堡大公国科学院自然科学、物理学和数学部门,1997年。139页。
\item Ernest Lebon,《Savants du jour : biographie, bibliographie analytique des écrits》,包括《Gabriel Lippmann的肖像》, 1909–1913,巴黎:Gauthier-Villars,1911年,第70页。
- Isabelle Bergoend,《Le Dagobert optique》,Thierry Marchaisse出版社,2015年。
\end{itemize}
\subsection{外部链接}
\begin{itemize}
\item [Gabriel Lippmann on Nobelprize.org](https://www.nobelprize.org/prizes/physics/1908/lippmann/facts/) 编辑此条目于Wikidata,包括1908年12月14日的诺贝尔讲座《Colour Photography》。
\item [Gabriel Lippmann在犹太百科全书](https://www.jewishencyclopedia.com)。
\item [Centre de Recherche Public – Gabriel Lippmann](https://www.crpgl.lu)。
\end{itemize}