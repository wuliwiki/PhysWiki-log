% 留出法
% 模型评估 留出法 hold-out

\pentry{数据\upref{datast},模型评估\upref{MoEva}}

\textbf{留出法}(Hold-out)是评估一个机器学习模型性能的常用方法之一。对于一个整理好的数据集,随机选择一部分样本作为训练数据即训练集,用于训练模型,剩下的部分用于测试模型,作为测试集。

在划分训练集和测试集的时候,须要遵循的原则之一是保持数据分布的一致性。比如,一个二分类任务,其数据样本的标签值为+或-。那么,在使用留出法划分训练集测试集时,要保证训练集中标签为+(或-)的样本比例与测试集中标签+(或-)的样本比例相同。通常可以采用分层采样的方法来实现这一原则。

第二个要遵循的原则是多次反复划分,然后取多次测试的平均性能。由于每次随机划分所得到的训练集和测试集中的样本往往不相同,因此在不同的训练样本和测试样本下,所得出的模型性能显然会有一定的差异。为了尽可能消除这种由于随机划分数据集所产生的偏差,可以采用本条原则。

第三个原则是测试集不可过大或者过小。如果测试集过大,则训练集会过小,由此训练出来的模型可能无法学习到整个原始数据集的规律。反之,如果测试集过小,训练集过大,模型可能会比较容易学习到原始数据中的规律,但由于测试集过小,测试出的性能难以代表模型的真实性能。在机器学习实践中,训练集与测试集的比例通常设置为3:1或4:1。


\begin{figure}[ht]
\centering
\includegraphics[width=14cm]{./figures/fb2f020a34d5bc5b.png}
\caption{留出法示意图} \label{fig_holdou_1}
\end{figure}