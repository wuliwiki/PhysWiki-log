% FTP/SFTP 笔记

\subsection{FTP}
\begin{itemize}
\item FTP 软件可以用来在 linux 上面上传下载文件
\item 其中一个软件叫做 vsftpd (very secure file transfer portal)
\item 安装方法 \verb`sudo apt-get  install vsftpd`
\item 设置文件 的储存路径是 /etc/vsftpd.conf, 确保其中的命令 local_enable = YES 和 write_enable = YES  没有被注释掉.
(if read only, use "sudo vi vsftpd.conf")
\item 为使 ftp server 的 config 文件生效, 必须 restart server. 用 sudo service vsftpd restart 或 stop 或 start
\end{itemize}

\subsection{Filezilla}
\begin{itemize}
\item 比起 Filezilla 和 WinSCP, 还是强烈推荐 MobaXterm, 直接集成 ssh 和 sftp
\item Filezilla 是其他电脑上的传文件程序.
\item 登录与 SSH 相同. host 框输入 ip 地址, 用户名和密码登录, port 22, 即可. 以后登录可以用 Quickconnect 按钮右边的箭头选择账号直接登录.
\item 可以同时打开多个窗口.
\item connect in the same way as ssh.
\end{itemize}

\subsection{WinSCP}
\begin{itemize}
\item 比起 Filezilla 和 WinSCP, 还是强烈推荐 MobaXterm, 直接集成 ssh 和 sftp
\item 在 windows 中, 感觉 WinSCP 比 Filezilla 要人性化要强大
\item select explorer mode when installing.
\item save connection, automatic login, create desktop shortcut, etc.
\item 后台传输: View > Preferences > Transfer > Background > Transfer on background by default
\item 如果无法显示中文, 可以试试在 WinScp 的登录界面点 Advanced > server environment > UTF-8 encoding for filenames 关闭. 如果还不行, 可以试试重新上传文件.
\item 可以先登录一个服务器然后再像命令行一样 ssh 到另一台服务器(叫做 tunneling), 在 SCP 中新建链接, 窗口中填写最终服务器的登录信息(如 ut15a24p), 然后选 advanced -> tunnel, 填写中间服务器的登录信息即可.
\item 更新文件夹(如 littleshi.cn/online)可以用工具栏中的同步选项(选 make remote up to date), 这样如果本地文件的修改时间比服务器上的文件新, 就会自动上传. 另外如果普通上传方法经常卡住, 也可以用同步试一试.
\end{itemize}

\subsection{Windows Network Drive}
\begin{itemize}
\item 想把 sftp 目录映射到 windows 作为网络磁盘? 目前还做不到. 然而 ftp 可以. 对于 Mac 共享, 可以使用 SMB (详见 Mac 笔记).
\end{itemize}
