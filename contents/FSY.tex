% 放射源
% license CCBYSA3
% type Wiki

(本文根据 CC-BY-SA 协议转载自原搜狗科学百科对英文维基百科的翻译)

\begin{figure}[ht]
\centering
\includegraphics[width=6cm]{./figures/78ec3184bf8b861b.png}
\caption{一种新的密封铯-137辐射源,它处于最终状态。} \label{fig_FSY_3}
\end{figure}

放射源是指发出电离辐射的已知数量放射性核素,典型的几种辐射类型有$Y$射线、$\alpha$粒子、$\beta$粒子和中子辐射。

放射源可用于辐照(这里辐射对目标材料起到了重要的电离作用),可以作为用于辐射测量过程和辐射防护仪器校准的辐射计量源,也用于工业过程测量,如造纸和钢铁工业中的厚度测量。放射源可以密封在容器中(高穿透性辐射)或者沉积在表面上(弱穿透性辐射),或者将它们置于流体中。

作为一种辐射源,它们被应用于放射治疗医学和工业中,例如工业射线成像、食品辐射、消毒、灭虫和聚氯乙烯的辐射交联。

放射性核素是根据它们发射的辐射类型和特征、发射强度和衰变半衰期来分类的。常见的放射性核素包括钴-60,[1] 铱-192,[2] 和锶-90。[3] 在国际单位制(SI)中,放射源活度的测量单位是贝克勒尔,但是在部分地区(比如美国)仍然使用历史上的单位:居里。尽管美国国家标准技术研究所(NIST)强烈建议使用国际单位制。[4] 出于健康卫生的目的,在欧盟国际单位制是强制使用的。

\subsection{密封源}
许多放射源是密封的,这意味着它们要么永久地完全包裹在胶囊中,要么牢固地结合在表面上。胶囊通常由不锈钢、钛、铂或其他惰性金属制成。[5] 密封源的使用消除了由于操作不当造成的放射性物质扩散到环境中的几乎所有风险,[5] 但是容器并不是用于衰减辐射,因此需要进一步的屏蔽来辐射防护。[6] 在不需要把放射源用化学或者物理手段包含在液体或者气体的应用场合,密封源几乎都可以适用。

\subsubsection{1.1 密封源的分类[7]}
国际原子能机构(IAEA)根据密封源相对于最低危险源的活度对其进行分类(其中危险源可能对人类造成重大伤害)。使用的比率是A/D,其中A是放射源活度,D是最小危险活度。
\begin{table}[ht]
\centering
\caption\label{FSY}
\begin{tabular}{|c|c}
\hline
\textbf{种类} & \textbf{A/D}\\
\hline
1 & ≥1000\\
\hline
2 & 10–1000\\
\hline
3 & 1–10\\
\hline
4 & 0.01–1\\
\hline
5 & <0.01\\
\hline
\end{tabular}
\end{table}
请注意,放射性输出足够低而不会对人类造成伤害的源(如烟雾探测器中使用的源)不在此列。
\begin{figure}[ht]
\centering
\includegraphics[width=6cm]{./figures/822a0fee92bde094.png}
\caption{2007年用于国际原子能机构(IAEA)第1,2和3类的ISO放射性危险标志,被定义为能够造成死亡或严重伤害的危险源。[5]} \label{fig_FSY_1}
\end{figure}

\subsection{校准源}
\begin{figure}[ht]
\centering
\includegraphics[width=6cm]{./figures/2af299a2daae5101.png}
\caption{使用平板源校准的手持大面积α闪烁探针。} \label{fig_FSY_2}
\end{figure}
校准源主要用于辐射测量仪器的校准,辐射测量仪器用于过程监控或辐射防护。

胶囊源用于$\beta$、$Y$和$X$光仪器校准,其中辐射从一个点有效发射。高放射性源通常用在校准池中:一个有厚墙的房间,用于保护操作者和提供光源曝光的远程操作。

板源常用于放射性污染仪器的校准。其表面固定有已知量的放射性物质,例如$\alpha$和/或$\beta$发射器,以允许校准用于污染调查和人员监测的大面积辐射探测器。这种测量通常是检测器接收的单位时间的计数,例如每分钟计数或每秒计数。

与胶囊源不同,板源发射材料必须在表面上,以防止容器衰减或由于材料本身造成的自屏蔽。这对于容易被小质量阻止的$\alpha$粒子尤其重要。布拉格曲线显示了自由空气中的衰减效应。

\subsection{未密封源}
未密封源是指不在永久密封容器中的放射源,广泛用于医疗目的。[8] 当放射源需要溶解在液体中以注射到患者体内或被患者摄入时,便使用它们。未密封源在工业中也以类似的方式作为放射性示踪剂用于泄漏检测。

\subsection{处理}
尽管程度较轻,但过期放射源的处置仍然对其他核废料的处置提出了类似的挑战。用过的低放射性源有时会非常不活跃,因此适合通过正常的废物处理方法(通常是填埋)进行处理。其他处置方法与高放射性废物类似,根据废物的活性使用不同深度的钻孔。[9]

一个臭名昭著的忽视处置高放射性污染源的事件是戈亚尼亚事故,该事故造成了数人死亡。

\subsection{参考文献}
[1]
^"C-188 Cobalt-60 Source". Nordion Inc. Retrieved 22 March 2016..

[2]
^"Iridium-192". Isoflex. Retrieved 22 March 2016..

[3]
^"Radioactive sources: isotopes and availability". Retrieved 22 March 2016..

[4]
^"NIST Guide to the SI, Chapter 5 (paragraph 5.2)". NIST. Retrieved 22 March 2016..

[5]
^"Implementation of the Control of High-activity Sealed Radioactive Sources and Orphan Sources (HASS) directive for nuclear licensed sites". Retrieved 22 March 2016..

[6]
^"Disused Sealed Source Management". International Atomic Energy Agency. Retrieved 22 March 2016..

[7]
^Radiation protection and safety of radiation sources : International basic safety standards (PDF). Vienna: International Atomic Energy Agency. 2014. ISBN 978-92-0-135310-8. ISSN 1020-525X..

[8]
^"Radiation Protection Glossary". Retrieved 22 March 2016..

[9]
^Disposal Options for Disused Radioactive Sources (PDF). International Atomic Energy Agency. 2005. ISBN 92-0-100305-6. ISSN 0074-1914..