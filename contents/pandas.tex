% Pandas 笔记(Python)
% license Xiao
% type Note

\begin{issues}
\issueDraft
\end{issues}

\begin{itemize}
\item \verb|conda install pandas|
\item 官网的\href{https://pandas.pydata.org/docs/getting_started/index.html}{入门教程}。
\item W3schools 的\href{https://www.w3schools.com/python/pandas/default.asp}{教程}。
\end{itemize}
安装 pandas 库:\verb|pip install pandas|。
加载 pandas 库:
\begin{lstlisting}[language=python]
import pandas as pd
\end{lstlisting}
\subsection{DataFrame}
参考了官网的\href{https://pandas.pydata.org/docs/getting_started/index.html}{入门教程}。
\begin{itemize}
\item 
利用 pandas 可以将一个具有一定格式的字典 dict 转化为 DataFrame 格式,从而可以对 DataFrame 进行许多便利的查询、修改和统计操作。例如 \verb|dict = {"Name":["A","B","C"], "Age":[1,2,3]}|,可以用 \verb|pd.DataFrame(dict))| 建立 DataFrame。
\item 查询某一列 \verb|df["Age"| 会返回那一列的数据以及数据类型。
\item 统计 DataFrame 的信息用\verb|df.describe()|,会返回 count,mean,std,min,25\%,50\%,75\%,max 等数据统计结果。
\item 保存到 \verb|csv| 文件可以用 \verb|df.to_csv('filename.csv')| 命令。
\end{itemize}
