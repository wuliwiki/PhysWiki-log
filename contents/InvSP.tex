% 不变子空间
% keys 不变子空间
% license Xiao
% type Tutor

\pentry{线性算子\nref{nod_LiOper}}{nod_f877}

注:如无特殊声明,以下向量空间都是域$\mathbb{F}$上的向量空间。

\begin{definition}{不变子空间}
相对于向量空间 $V$ 上的\enref{线性算子}{LiOper} $\mathcal{A}: V \rightarrow V$,\enref{子空间}{SubSpc} $U \subseteq V$ 被称为\textbf{不变的},如果 $\mathcal{A} U \subset U$。此时$\mathcal{A}|_U: U \to U$ 是一个 $U$ 上的线性算子。
\end{definition}

\begin{example}{}
算子 $\mathcal{A}$ 的\textbf{核} $\mathrm{Ker}\;\mathcal A$ 和\textbf{像} $\mathrm{Im}\;\mathcal{A}$
\begin{equation}\label{eq_InvSP_3}
\begin{aligned}
\mathrm{Ker}\;\mathcal{A}&=\{ v\in V|\mathcal{A} v= 0\}\\
\mathrm{Im}\;\mathcal{A}&=\{ w\in V| w=\mathcal{A} v,\forall v\in V\}~,
\end{aligned}
\end{equation}
都是 $\mathcal A$ 的不变子空间。
\end{example}

\begin{theorem}{}
有限维度向量空间 $V$ 的子空间 $U$ 是算子 $\mathcal{A}$ 的不变子空间,当且仅当存在基底 $\mathcal{B}$, $\mathcal{B}_U: = \mathcal{B} \cap U$ 是 $U$ 的基底,使得算子 $\mathcal{A}$ 的矩阵 $A$ 在基底 $\mathcal{B}$ 下可以写成上三角分块矩阵
\begin{equation}
A = \begin{pmatrix}
A_U & B\\
0 & C
\end{pmatrix}~.
\end{equation}
其中,$A_U$ 是 $\mathcal{A}|_{U}: U \to U$ 的在 $\mathcal{B}_U$ 下的矩阵形式。
\end{theorem}

\begin{theorem}{}
对于有限维度向量空间 $V$,如果子空间 $U$ 是可逆算子 $\mathcal{A}$ 的不变子空间,那么 $\mathcal{A} U = U$。
\end{theorem}
\begin{exercise}{}
证明它。提示:可以先找一组基。
\end{exercise}

\begin{example}{}
上述定理在无限维度时不成立:对于空间
$$
V: = \{ (\dots, a_{-1}, a_0, a_1, \dots) \mid a_i \in \mathbb{F}\}~,
$$
即向双向无限延伸的序列的;$\mathbb{A}$是右移算符(它的逆运算是左移动算符,因此是可逆的),那么我们可以取
$$
U: = \{ (\dots, 0, a_0, a_1, \dots) \in V \}~,
$$
即 $a_i = 0, \forall i < 0$,可以发现
$$
\mathcal{A} U = \{ (\dots, 0, a_0, a_1, \dots) \in U \mid a_0 = 0 \}~
$$
是 $U$ 的真子集。
\end{example}

\begin{theorem}{}\label{the_InvSP_1}
向量空间 $V$ 可以写成算子 $\mathcal{A}$ 的两个不变子空间的直和,当且仅当存在基底 $(e_1, \dots, e_n)$ 使得算子 $\mathcal{A}$ 可以表示成分块对角矩阵
\begin{equation}
A=\begin{pmatrix}
A_1 & 0\\
0 & A_2
\end{pmatrix}~.
\end{equation}

\end{theorem}

\textbf{证明:}1. ($\Rightarrow$),记两个不变子空间为 $U$ 和 $V$,取 $U$ 的基底 $\{e_1 ,\cdots, e_m\}$,$W$ 的基底 $\{e_{m+1}, \cdots, e_n\}$,则由\autoref{the_DirSum_1}~\upref{DirSum},$\{e_{1}, \cdots, e_n\}$ 是 $V$ 的基底。

由于 $\mathcal{A}u\in U, \mathcal{A} w\in W,\forall  u\in U, w\in W$,则
\begin{equation}\label{eq_InvSP_2}
\begin{aligned}
\mathcal{A} e_j&=\sum_{i=1}^m a_{ij} e_i\quad (j=1,\cdots ,m)~,\\
\mathcal{A} e_j&=\sum_{i=m+1}^n a_{ij} e_i\quad (j=m+1,\cdots ,n)~.
\end{aligned}
\end{equation}
记 $\mathcal{A}|_U$ 的矩阵为 $A_1$,$\mathcal{A}|_W$ 的矩阵为 $A_2$,我们得到
\begin{equation}
A=(a_{ij})=\begin{pmatrix}
A_1 & 0\\
0 & A_2
\end{pmatrix}~.
\end{equation}

2. ($\Leftarrow$),记 $m: = \dim(A_1)$,$U: = \langle e_1, \cdots, e_m\rangle$,$W: = \langle e_{m + 1}, \cdots, e_n\rangle$,可得 $U, W$ 是 $\mathcal{A}$ 的不变子空间,$V = U \oplus W$。

\textbf{证毕!}

\begin{definition}{算子的直和}
若向量空间 $V$ 是算子 $\mathcal{A}$ 的不变子空间 $U,W$ 的直和 $V=U\oplus W$ ,则称算子 $\mathcal{A}$ 是其限制在 $U,W$ 上的算子 $\mathcal{A}_U,\mathcal{A}_W$ 的直和,并记作
\begin{equation}
\mathcal{A}=\mathcal{A}_U \oplus \mathcal{A}_W~.
\end{equation}
(有时亦记做 $\dot{+}$)。

\end{definition}
