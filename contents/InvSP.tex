% 不变子空间
% 不变子空间



\pentry{线性算子代数\upref{LiOper}}
\begin{definition}{不变子空间}
子空间\upref{SubSpc} $U\in V$ 相对于线性算子\upref{LiOper} $\mathcal{A}:V\rightarrow V$ 是\textbf{不变的},如果 $\mathcal{A}U\subset U$.
\end{definition}
\begin{example}{}
算子 $\mathcal{A}$ 的\textbf{核} $\mathrm{Ker}\;\mathcal A$ 和\textbf{像} $\mathrm{Im}\;\mathcal{A}$
\begin{equation}\label{InvSP_eq3}
\begin{aligned}
\mathrm{Ker}\;\mathcal{A}&=\{ v\in V|\mathcal{A} v= 0\}\\
\mathrm{Im}\;\mathcal{A}&=\{ w\in V| w=\mathcal{A} v,\forall v\in V\}
\end{aligned}
\end{equation}
都是 $\mathcal A$ 的不变子空间.
\end{example}
\begin{theorem}{}\label{InvSP_the1}
$n$ 维矢量空间 $V$ 是算子 $\mathcal{A}$ 的 $m$ 维不变子空间 $U$ 和 $(n-m)$ 维不变子空间 $W$ 的直和,当且仅当算子 $\mathcal{A}$ 的矩阵 $A$ 在某基底下具有分块对角形式
\begin{equation}\label{InvSP_eq1}
A=\begin{pmatrix}
A_U&0\\
0&A_W
\end{pmatrix}
\end{equation}
其中,$A_U,A_W$ 分别是 $m$ 阶方阵和 $(n-m)$ 阶方阵.即
\begin{equation}
V=U\oplus W,\mathcal{A}U\subset U,\mathcal{A}W\subset W\Leftrightarrow A=\begin{pmatrix}
A_U&0\\
0&A_W
\end{pmatrix}
\end{equation}

\end{theorem}
\textbf{证明:}1.$
V=U\oplus W,\mathcal{A}U\subset U,\mathcal{A}W\subset W\Rightarrow A=\begin{pmatrix}
A_U&0\\
0&A_W
\end{pmatrix}
$

设 $U$ 的基底为 $(\hat e_1,\cdots,\hat e_m)$ ,$W$ 的基底为 $(\hat e_{m+1},\cdots,\hat e_n)$,则由\autoref{DirSum_the1}~\upref{DirSum},$(\hat e_{1},\cdots,\hat e_n)$ 是 $V$ 的基底.

由于 $\mathcal{A}u\in U, \mathcal{A} w\in W,\forall  u\in U, w\in W$,则
\begin{equation}\label{InvSP_eq2}
\begin{aligned}
\mathcal{A}\hat e_j&=\sum_{i=1}^m a_{ij}\hat e_i,\quad j=1,\cdots ,m\\
\mathcal{A}\hat e_j&=\sum_{i=m+1}^n a_{ij}\hat e_i,\quad j=m+1,\cdots ,n
\end{aligned}
\end{equation}
由线性算子与矩阵的对应关系\autoref{LiOper_eq2}~\upref{LiOper},知算子 $\mathcal{A}$ 的对应矩阵 $A$ 即为 
\begin{equation}
A=(a_{ij})=\begin{pmatrix}
A_U&0\\
0&A_W
\end{pmatrix}
\end{equation}

2.$
A=\begin{pmatrix}
A_U&0\\
0&A_W
\end{pmatrix}\Rightarrow V=U\oplus W,\mathcal{A}U\subset U,\mathcal{A}W\subset W
$

任选 $V$ 的基底 $(\hat e_1,\cdots,\hat e_n)$, 由算子和矩阵对应关系\autoref{LiOper_eq2}~\upref{LiOper},即得 $A$ 对应的算子 $\mathcal{A}$ 具有关系式\autoref{InvSP_eq2} .而这意味着由基底 $(\hat e_1,\cdots,\hat e_m)$ 和 $(\hat e_{m+1},\cdots,\hat e_n)$ 张成的空间 $U=\langle\hat e_1,\cdots,\hat e_m\rangle$ 和 $W=\langle\hat e_{m+1},\cdots,\hat e_n\rangle$ 是算子 $\mathcal{A}$ 的不变子空间,而由基底 $(\hat e_1,\cdots,\hat e_n)$ 的线性无关性可知,$V=U\oplus W$.

\textbf{证毕!}

\autoref{InvSP_eq1} 中,可以把 $A_U,A_W$ 看成是算子 $\mathcal{A}$ 分别限制在 $U$ 和 $W$ 上的算子 $\mathcal{A}_U$ 和 $\mathcal{A}_W$ 对应的矩阵.
\begin{definition}{算子的直和}
若矢量空间 $V$ 是算子 $\mathcal{A}$ 的不变子空间 $U,W$ 的直和 $V=U\oplus W$ ,则称算子 $\mathcal{A}$ 是其限制在 $U,W$ 上的算子 $\mathcal{A}_U,\mathcal{A}_W$ 的直和,并记作
\begin{equation}
\mathcal{A}=\mathcal{A}_U\dot{+}\mathcal{A}_W
\end{equation}
此时称算子 $\mathcal{A}$ 对应的矩阵 $A$ 是 $\mathcal{A}_U,\mathcal{A}_W$ 对应矩阵 $A_U$ 和 $A_W$ 的直和,并记作
\begin{equation}
A=\begin{pmatrix}
A_U&0\\
0&A_W
\end{pmatrix}=A_U\dot{+}A_W
\end{equation}

\end{definition}