% Python 服务器环境配置

\pentry{Python 简介\upref{Python}}

\begin{itemize}
\item \verb|sudo apt install python3.9|
\item \verb|sudo apt install python3-pip|
\item \verb|pip3 install numpy|
\item 老系统上面(例如 16.04), 如果 apt 不能直接安装新版本, 要自己编译安装. 从官网下载源码: \verb|wget https://www.python.org/ftp/python/3.6.3/Python-3.6.3.tgz| 解压 \verb|tar -xvf Python-3.6.3.tgz|, \verb|cd Python-3.6.3|, \verb|apt-get install build-essential zlib1g-dev|, \verb|./configure|, \verb|make|(可以加 \verb|-j4| 选项), \verb|make install| 就可以了. 检查版本: \verb|python3.6 -V|, 但安装 \verb|3.10| 的时候会出现编译错误.
\item \verb|pip| 是 python 的默认包管理器, 最广泛使用. 另外也可以用 conda, 但不能免费商用.
\item \verb|Python Package Index (PyPI)| 是 \verb|pip| 中安装包的主要来源.
\item 注意 \verb|pip3 install 包名称| 中的名称未必是 \verb|using 包名称| 的名称! 如果找不到前者, 会提示 \verb|Could not find a version that satisfies the requirement|
\item \verb|python| 命令的路径一般是 \verb|/usr/bin/python|, 它是一个 soft link, 链接到具体版本. 注意如果直接修改这个连接可能会发生一些错误(\verb|lsb_release| 等)
\end{itemize}
