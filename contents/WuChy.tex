% Cauchy-Schwarz 不等式
% keys 不等式
% license Usr
% type Tutor

\begin{issues}
\issueOther{整合到 柯西—施瓦茨不等式\upref{CSNeq}}
\end{issues}

Cauchy-Schwarz不等式是一个在众多背景下都有应用的不等式,\ 例如线性代数,\ 数学分析,\ 概率论,\ 向量代数以及其他许多领域,\ 这就意味着它有很多等价形式或者推广形式.\ 它被认为是数学中最重要的不等式之一.\ 此不等式最初于1821年被Cauchy提出,\ 其积分形式在1859被Buniakowsky提出,\ 而积分形式的现代证明则由Schwarz于1888年给出.\ 

在给出Cauchy-Schwarz不等式之前,\ 先给出几个常用不等式作为引理并且给出相关证明.\ 
\begin{lemma}{三角不等式}
$\forall \boldsymbol{x},\boldsymbol{y}\in\mathbb{R}^n$,满足
\begin{equation}
\lvert\lvert\boldsymbol{x}\rvert-\lvert\boldsymbol{y}\rvert\rvert\leqslant\lvert\boldsymbol{x}\pm\boldsymbol{y}\rvert\leqslant\lvert\boldsymbol{x}\rvert+\lvert\boldsymbol{y}\rvert~
\end{equation}
\end{lemma}
\begin{lemma}{均值不等式}
$\forall \boldsymbol{x}\in\mathbb{R}^n$,定义以下和式
\begin{align}
H_n&=\frac{1}{\frac{1}{n}\sum\limits_{i=1}^n\frac{1}{x_i}}=\frac{n}{\frac{1}{x_1}+\frac{1}{x_2}+\cdots+\frac{1}{x_n}}\\
G_n&=\sqrt[n]{\prod_{i}^n x_i}=\sqrt[n]{x_1x_2\cdots x_n}\\
A_n&=\frac{1}{n}\sum_{i=1}^n x_i=\frac{x_1+x_2+\cdots+x_n}{n}\\
Q_n&=\sqrt{\frac{1}{n}\sum_{i=1}^n x_i^2}=\sqrt{\frac{x_1^2+x_2^2+\cdots+x_n^2}{n}}~
\end{align}
\end{lemma}
这些和式满足
\begin{equation}
H_n\leqslant G_n\leqslant A_n \leqslant Q_n~
\end{equation}
\begin{lemma}{Bernoulli不等式}
$\forall x > -1,n \in \mathbb{N}^{\ast}$满足
\begin{equation}
(1+x)^n \geq 1+nx~
\end{equation}
\end{lemma}
\begin{lemma}{Young不等式}
设$p,q>1,\frac{1}{p}+\frac{1}{q}=1$,$\forall a,b\geqslant 0$,\ 满足 
\begin{equation}
a b \leqslant \frac{1}{p}a^{p}+\frac{1}{q}b^{q}~
\end{equation}
\end{lemma}
\begin{theorem}{Cauchy-Schwarz不等式}
$\forall \boldsymbol{x},\boldsymbol{y}\in\mathbb{R}^n$,满足
\begin{equation}
(\boldsymbol{x}^{\top}\boldsymbol{y})^2\leqslant \Vert \boldsymbol{x}\Vert_2^2\Vert \boldsymbol{y}\Vert_2^2~
\end{equation}
序列形式表示为
\begin{equation}
\left(\sum_{i=1}^{n}x_iy_i\right)^2\leqslant \left(\sum_{i=1}^nx_i^2\right)\left(\sum_{i=1}^ny_i^2\right)~
\end{equation}
\end{theorem}