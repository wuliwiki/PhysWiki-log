% 光与物质粒子的统一(相对论点粒子的作用量)
% keys 相对论|非相对论|光|作用量
% license Usr
% type Tutor
\pentry{作用量原理}{nod_28f0}
\footnote{A.Zee,Einstein Gravity in a Nutshell}本节将从作用量的视角介绍牛顿力学到相对论的自然过渡,最终给出适用于物质粒子和光的作用量。阅读本节需带着“民主”的思想:时间和空间、粒子和光应当被平等对待。
\subsection{Newton力学到狭义相对论}\label{sub_RAct_1}
\subsubsection{缘起}
首先看Newton力学中物质粒子的Euler-Lagrange作用量:
\begin{equation}\label{eq_RAct_5}
S=\int\dd t\qty[\frac{1}{2}m\qty(\dv{\vec x}{t})^2-V(x)]~.
\end{equation}
这一作用量是相当“笨重”的。在这里,让我们只考虑自由粒子,把势 $V(x)$ 给丢掉。那么上式可写为
\begin{equation}\label{eq_RAct_2}
S=\int\dd t\frac{1}{2}m\qty(\dv{\vec x}{t})^2=m\int\dd t\frac{(\dd{\vec x})^2}{2\dd t}~.
\end{equation}
上式对待 $\dd{\vec x}$ 和 $\dd t$ 的方式是不平等的,$\dd{\vec x}$ 平方的出现来自于旋转不变性,但是为什么 $\dd t$ 只值得1次幂?多么奇怪的组合呀,$\frac{(\dd{\vec x})^2}{\dd t}$,一个东西的平方除以另一个东西。这冒犯了我们自由主义的思想,因此,我们需要进行改变。

\subsubsection{平方根}
让我们回想一下初次学习平方根概念的情景。我们知道,25的平方根是5,36的平方根是6,等等。但是,对一个不是整数的平方的数,它的平方根是什么?比如24。使用古老的试错方法:计算 $4.9^2,4.8^2,\ldots.$ 很快就能很好的估计这个平方根了。在学会了用字母代表数值的想法后,我们就可以得到我们想要的公式:
\begin{equation}\label{eq_RAct_1}
\sqrt{a^2-\epsilon^2}\approx a-\frac{\epsilon^2}{2a}+\cdots.~
\end{equation}
验证一下右边,$\qty(a-\frac{\epsilon^2}{2a})^2=a^2-\epsilon^2+\frac{\epsilon^4}{4a^2}$,即:\autoref{eq_RAct_1} 的误差是高阶的。

\subsubsection{写出相对论的作用量}
比较\autoref{eq_RAct_2} 和\autoref{eq_RAct_1} ,我们改写\autoref{eq_RAct_1} 为
$
\frac{\epsilon^2}{2a}\approx-\sqrt{a^2-\epsilon^2}+ a+\cdots
$。因此
\begin{equation}
\frac{(\dd{\vec x})^2}{2\dd t}=c\frac{(\dd{\vec x})^2}{2c\dd t}\approx-c\sqrt{(c\dd t)^2-\dd{\vec x}^2}+c^2\dd t~.
\end{equation}
上面第一个等式中添加 $c$ 是为了保持量纲一致,可知 $c$ 具有速度的量纲。注意到唯一具有本征意义的速度是光速,因此 $c$ 代表着光速。因此,我们将自由点粒子的作用量改写为
\begin{equation}
S=-mc\int \sqrt{(c\dd t)^2-\dd{\vec x}^2}+\int mc^2\dd t.~
\end{equation}
项 $\int mc^2\dd t =m c^2(t_{\mathrm{final}}-t_{\mathrm{intial}})$ 对待 $\dd t,\dd{\vec x}$ 的方式不一样,但幸运的是,它们的变分消失,因此作用量原理允许我们丢掉这一项。令 $c=1$,现在我们得到如下作用量
\begin{equation}\label{eq_RAct_4}
S=-m\int \sqrt{(\dd t)^2-\dd{\vec x}^2}.~
\end{equation}
这一作用量对待时间和空间的方式相当的“民主”。然而,积分往往写为 $\int\dd t$ 的形式,这很简单:
\begin{equation}\label{eq_RAct_3}
S=-m\int\dd t \sqrt{1-\qty(\dv{\vec x}{t})^2}.~
\end{equation}
这便得到了相对论自由粒子的作用量。

\subsubsection{著名的公式}
为了回到Newton力学,将\autoref{eq_RAct_3} 展开
\begin{equation}\label{eq_RAct_6}
S=-m\int\dd t \qty{\frac{m}{2}\qty(\dv{\vec x}{t})^2-m+\cdots}.~
\end{equation}
我们不知道如何结合势能,像\autoref{eq_RAct_5} 一样,仅仅在\autoref{eq_RAct_4} 中加入 $-\int\dd t V(x)$,则会在此得到因子 $\dd t$。但若我们有\autoref{eq_RAct_5} 一样的势能,那么在非相对论极限\autoref{eq_RAct_6} 下,$m$ 应该归属于 $V(x)$ 中,即 \autoref{eq_RAct_6} 中的项 $m$ 是一类特殊的势能项。这表明,即使是一个静止的粒子也有能量。若我们恢复 $c$,则 
\begin{equation}
S=-mc\int \sqrt{(c\dd t)^2-\dd{\vec x}^2}=\int\dd t \qty{\frac{m}{2}\qty(\dv{\vec x}{t})^2-mc^2+\cdots}.~
\end{equation}
因此,对静止的粒子,其具有能量
\begin{equation}
E=mc^2.~
\end{equation}

\subsection{更对称的形式}
定义对角元为 $(-1,1,1,1)$ 的对角矩阵 $\eta_{\mu\nu}$,我们可将相对论自由粒子的作用量写为
\begin{equation}\label{eq_RAct_7}
S=-m\int \sqrt{-\eta_{\mu\nu}\dd x^\mu\dd x^\nu}.~
\end{equation}
注意到 $\eta_{\mu\nu}\dd x^\mu\dd x^\nu$ 是两邻近点的Minkowskian距离的平方,因此粒子的作用量(忽略整体常数因子)是它在时空中所穿越的距离。事实上,我们所能做出的坐标不变的量只能是代表粒子轨迹的世界线的“长度”或本征时间,即 $\int\dd \tau=\int \sqrt{-\eta_{\mu\nu}\dd x^\mu\dd x^\nu}$。我们称比例因子是粒子的质量。因此,质量可定义为几何(世界线的长度)和物理(作用量)之间的转化因子。

需要注意的是,\textbf{作用量\autoref{eq_RAct_7} 中的记号 $x^\mu$ 代表的是粒子的时空坐标,而不是时空本身}。这可以通过考虑多个粒子(用指标 $a$ 标记)的作用量看到: $S=-\sum_a m_a\int \sqrt{-\eta_{\mu\nu}\dd x_a^\mu\dd x_a^\nu}$。为了避免记号的混乱,更好的办法是用 $X^\mu$ 代表粒子的时空坐标,作用量则写为
\begin{equation}\label{eq_RAct_8}
S=-m\int \sqrt{-\eta_{\mu\nu}\dd X^\mu\dd X^\nu}.~
\end{equation}


\subsection{无质量粒子的作用量}
由于作用量\autoref{eq_RAct_8} 对无质量粒子恒为0,因此对无质量粒子而言这一作用量是无效的。为了公平的对待粒子和光,我们只需要将\autoref{eq_RAct_8} 改写为下面的形式
\begin{equation}\label{eq_RAct_9}
\tilde S=-\frac{1}{2}\int\dd\zeta\qty(\sigma(\zeta)\qty(\dv{X}{\zeta})^2+\frac{m^2}{\sigma(\zeta)})~.
\end{equation}
其中,$\qty(\dv{X}{\zeta})^2=-\eta_{\mu\nu}\dv{X^\mu}{\zeta}\dv{X^\nu}{\zeta}$。这一作用量在 $m=0$ 时是有效的。因此我们,只需要验证\autoref{eq_RAct_8} 和\autoref{eq_RAct_9} 等价即可。注意\autoref{eq_RAct_9} 的动力学变量不光是 $X^\mu$,也包含 $\sigma(\zeta)$。而 $\dv{\sigma}{\zeta}$ 不在作用量中出现,因此由Euler-Lagrange方程,得 $\sigma$ 满足方程
\begin{equation}\label{eq_RAct_10}
\frac{m^2}{\sigma(\zeta)^2}=\qty(\dv{X}{\zeta})^2.~
\end{equation}
这是代数方程,不是微分方程。因此,$\sigma$ 并不具有动力学而仅仅与 $X^\mu(\zeta)$ 相结合。
利用\autoref{eq_RAct_10} 消去 $\tilde S$ 中的 $\sigma(\zeta)$,便回到了 $S$。因此,在产生粒子的运动方程的意义上,两作用量是等价的。

现在,令 $m=0$,就得到了无质量粒子适用的作用量
\begin{equation}
 S_{\mathrm{massless}}=\frac{1}{2}\int\dd\zeta\qty(\sigma(\zeta)\eta_{\mu\nu}\dv{X^\mu}{\zeta}\dv{X^\nu}{\zeta}).~
\end{equation}
对 $\sigma$ 变分,可以获得 $\eta_{\mu\nu}\dv{X^\mu}{\zeta}\dv{X^\nu}{\zeta}=0$。换句话说 $(\dd{\vec X})^2=(\dd X^0)^2$,即无质量粒子以光速运动。


