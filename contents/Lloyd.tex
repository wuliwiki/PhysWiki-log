% 劳埃德镜实验
% keys 劳埃德镜|干涉|光密介质
% license Xiao
% type Tutor

\pentry{杨氏双缝干涉实验\upref{Young}, 半波损失\upref{WvLost}}{nod_6162}

劳埃德(H.Lloyd)于1834年提出了一种更简单的观察干涉的装置。如\autoref{fig_Lloyd_1} 所示,$MN $ 为一块平玻璃板,用作反射镜,$S_1$ 是一狭缝光源,从光源发出的光波,一部分掠射(即入射角接近 $90^\circ$)到玻璃平板上,经玻璃表面反射到达屏上;另一部分直接射到屏上。这两部分光也是相干光,它们同样是用分波阵面得到的。反射光可看成是由虚光源 $S_2 $ 发出的。$S_1$ 和 $S_2 $ 构成一对相干光源,对干涉条纹的分析与杨氏实验也相同。中画有阴影的区域表示相干光在空间叠加的区域。这时在屏上可以观察到明暗相间的干涉条纹。
\begin{figure}[ht]
\centering
\includegraphics[width=8cm]{./figures/0bcf914eb449414f.pdf}
\caption{劳埃德镜实验} \label{fig_Lloyd_1}
\end{figure}
应该指出,在劳埃德镜实验中,如果把屏幕移近到和镜面边缘 $N $ 相接触,即\autoref{fig_Lloyd_1} 中 $E' $ 的位置,这时从 $S_1$ 和 $S_2$ 发出的光到达接触处的路程相等,应该出现明纹,但实验结果却是暗纹,其他的条纹也有相应变化。这一实验事实说明了由镜面反射出来的光和直接射到屏上的光在 $N $ 处的相位相反,即相位差为 $\pi$. 由于直射光的相位不会变化,所以只能认为光从空气射向玻璃平板发生反射时,反射光的相位跃变了 $\pi$。

进一步的实验表明: 光从光疏介质射到光密介质界面反射时,在掠射(入射角 $i=90^\circ$ 或正入射,即 $i = 0$)的情况下,反射光的相位较之入射光的相位有 $\pi$ 的突变,这一变化导致了反射光的波程在反射过程中附加了半个波长,故常称为\textbf{半波损失}。今后在讨论光波叠加时,若有半波损失,在计算波程差时必须计及,否则会得出与实际情况不符的结果。
