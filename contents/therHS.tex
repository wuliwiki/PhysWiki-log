% 热学初步(高中)

\begin{issues}
\issueTODO
\end{issues}
% 分子动理论|气体等x定律|固体液体|热力学定律
% 缩减一部分,把第二章的前两小节合并,第二章整体作为一个新的小节
% 或者直接拆分成分子动力学和热力学初步算了,麻烦
% 第二章和第三章作为热力学初步内容

%\pentry{相互作用\upref{HSPM02}}% 分子动力学

\subsection{温度和温标}
\subsubsection{状态参量与平衡态}
以研究容器中气体的热学性质为例,我们的研究对象是一个由大量分子所组成的系统,称之为热力学系统,简称\textbf{系统}。在我们的经验常识中,一小罐热的气体会在室温下逐渐变凉,最终和室温温度保持一致,这个过程中,这一小罐热的气体即是系统,而系统之外与之产生相互作用的其他物体统称为\textbf{外界}。外界影响系统,导致系统的某些物理量发生变化。在这个例子中,热空气温度逐渐下降,最终和外界空气保持一致,热空气(系统)的状态也随之改变。在热学中,为了确定系统的状态,需要用一些物理量来进行描述,这些物理量叫做系统的状态参量,比如体积$V$是描述系统空间范围的几何参量,压强$p$是描述系统之间或内部力的作用的力学参量,温度$T$是确定系统冷热程度的热学参量……

但是,往往在任意时刻确定系统的状态是困难的,只有在系统处于\textbf{平衡态},也即系统经过了足够长的时间的演化,内部的各个部分的状态参量达到稳定状态不再改变时,我们才可以比较准确的描述系统的状态。

\subsubsection{温度和热平衡}
在上述的例子中,可以发现温度逐渐下降,说明系统的热学性质发生了改变,并在足够长的时间之后罐中空气温度和外界保持一致。在这个过程中,系统和外界相互接触,并发生了热传导,最终各自的状态参量不再发生改变,达到了平衡状态,这种平衡叫做\textbf{热平衡}。如果两个系统之间处于热平衡,那它们必然可以被某一相等的状态参量所描述,这里的状态参量则是\textbf{温度}。换言之,温度是决定一个系统是否处于热平衡的物理量。实验表明,如果两个系统分别与第三个系统达到热平衡,则两个系统之间一定也是处于热平衡的,因为它们都具有相同的温度,这个结果被称为\textbf{热平衡定律}。

\subsubsection{温度计}
为了测量温度,人们发明了温度计。首先,温度计需要有一种测温物质,这种测温物质的某些物理性质会随着温度的改变而发生改变,比如说水银的热胀冷缩可以制成水银温度计,气体的压强随温度变化可以制成气体温度计,电阻随温度变化可以制成电阻温度计,由不同金属随温度升高膨胀程度不同制成的双金属温度计……需要注意的是,每一种温度计都有其工作范围,在工作范围之外原来的变化关系未必成立,不能够继续测量温度。为了使用方便,人们尽可能的采用具有线性变化的物质来制作温度计。另一方面,人们需要定义该特性和温度之间的对应关系,每一种不同的定义方式对应于不同的\textbf{温标},同时,还需要定义温度的零点和分度方法。例如,我们常见的摄氏度定义一个标准大气压下冰水混合物的温度为$0^\circ \mathrm{C}$,水沸腾的温度为$100^\circ \mathrm{C}$,并将其间的刻度平均分成$100$份,每份为$1^\circ \mathrm{C}$。

但是当我们进行对热力学的学习时,我们将采用热力学温标,对应的就是\textbf{热力学温度},单位是\textbf{开尔文},简称\textbf{开},符号是$\mathrm{K}$。摄氏温度$t$和热力学温度$T$之间的换算关系由国际计量大会所确定:$$T=t+273.15\mathrm{K}$$可以发现,摄氏温度和热力学温度的温度差都是相同的,也即$1^\circ \mathrm{C}=1\mathrm{K}$。

\subsection{气体的等温变化}
为了探索气体的热学性质,首先先考察一种特殊的情况。在温度不变的条件下,一定质量的气体,它的压强和体积将满足什么样的变化关系呢?这种变化被称之为气体的\textbf{等温变化}。英国科学家玻意耳和法国科学家马洛特各自通过实验发现,一定质量的气体,在温度不变的情况下压强$p$与体积$V$成反比,即$$p\propto\dfrac{1}{V}$$稍作整理可以得到$$pV=C$$其中$C$是常数,该定律也称\textbf{玻意耳定律}。但是实际应用中我们并不是特别关系常数$C$的具体取值,而更关系在初末状态时系统状态参量的关系,因此上式可以继续整理得到$$p_1V_1=p_2V_2$$其中$p_1$、$V_1$和$p_2$、$V_2$可以分别看成初末状态下系统的压强和体积。
\subsection{气体的等压变化}
同理,我们也将研究气体在压强不变的情况下,气体体积和温度之间的变化关系,这种变化叫气体的\textbf{等压变化}。实验表明,在压强不变时,一定质量的某种气体的体积随温度$T$线性变化,当温度采用热力学温度$T$时,所绘制的$V-T$图像中,等压线是一条过原点的直线。

法国科学家盖-吕萨克首先发现了这一线性关系,并将其表述为:一定质量的某种气体,在压强不变的情况下,其体积$V$与热力学温度$T$成正比,即$$V=CT$$其中$C$是常量,称之为\textbf{盖-吕萨克定律}。同上所述,我们更关心体系初末状态之间的关系,即有$$\dfrac{V_1}{T_1}=\dfrac{V_2}{T_2}$$,其中$V_1$、$T_1$和$V_2$、$T_2$可以分别看成初末状态下系统的体积和热力学温度。
\subsection{气体的等容变化}
\subsection{固体和液体简介}
\subsection{功、热、内能}
\subsection{热力学第一定律}
\subsection{能量守恒定律}
\subsection{热力学第二定律}



%% 画图时间

%% 错别字纠正
