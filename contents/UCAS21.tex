% 中国科学院大学 2021 年考研 811 量子力学
% keys 量子力学|考研
% license Copy
% type Tutor

\textbf{声明}:“该内容来源于网络公开资料,不保证真实性,如有侵权请联系管理员”

一、
	(30分) 能量为$  E $的粒子, 从负无穷远处沿着 $x$ 轴入射, 势场为
	\[ 
	V(x)=\left\{\begin{aligned}&0, &  x<0 \\ &V_{0}, & 0<x<a \\ &\infty, &  \quad x>a\end{aligned}\right. ~.
	 \] 
	其中 $E>V_{0}>0$, 求全空间的波函数 $\psi(x)$ 。



二、(30分)已知 $H$ 的两个本征态为 $\left|u_{1}\right\rangle,\left|u_{2}\right\rangle$ 分别对应于能昰 $E_{1}, E_{2}$。 力学量 $A$ 的本征态为	
$\left|\phi_{1}\right\rangle=\frac{\left|u_{1}\right\rangle+\left|u_{2}\right|}{\sqrt{2}}$ 和 $\left|\phi_{2}\right\rangle=\frac{\left|u_{1}\right\rangle-\left|u_{2}\right\rangle}{\sqrt{2}}$, 分别对应本征值 $a_{1}, a_{2}, t=0$ 时, 粒子处于 $\left|\phi_{1}\right\rangle$ 态。求:
\begin{enumerate}
	\item
	$t$ 时刻的态 $|\psi(t)\rangle$;
	\item
	 $t$ 时刻力学量 $A$ 的期望值 $\langle A\rangle(t)$。
	
	
	
\end{enumerate}

三、
对于自旋 $\frac{1}{2}$ 的体系, 已知哈密顿量 $H=S_{x}+S_{y}$。
\begin{enumerate}
	\item
	求 $H$ 的本征值和本征态;
	\item
	求在 (1) 问的低能级态 $\left|\psi_{-}\right\rangle$下测到 $S_{z}=+\frac{\hbar}{2}$ 的概率;
	\item 
	设 $t=0$ 时体系初态为 $\left|\psi_{-}\right\rangle $。$ t>0$ 时加入 $z$ 方向磁场 $\vec{B}=(0,0, B)$, 求 $t$ 时刻粒子处于初态的概 率。
	
\end{enumerate}



四、
(30分)
在宽度为$ 2a $的一维无限深势井中存在两个粒子,忽略相互作用。
\begin{enumerate}
	\item
	求能级 $E_{n_{1}, n_{2}}$;
	\item
	若两粒子为自旋 $\frac{1}{2}$ 的全同粒子,求基态与第一激发态的能级简并度以及相应的波函数;
	\item
	若两粒子为自旋 $\frac{1}{2}$ 的全同粒子,求基态与第一激发态的能级简并度以及相应的波函数。
	
	
	
\end{enumerate}

五、
氢原子处于基态,受到脉冲电场
$$
\mathscr{E}(t)=\mathscr{E}_{0} \delta(t) ~
$$
作用, $\mathscr{E}_{0}$ 为常数。已知波尔半径为 $a_{0}$, 矩阵元 $z_{n 1}=\iiint \psi_{n}^{*} z \psi_{1} \mathrm{dV}$。
\begin{enumerate}
	\item
利用选择定则判断粒子会跃迁到哪些激发态上;
	
	
	\item
	计算越签到各个激发态的概率;
	
	
	\item
	计算$ t $时刻粒子处于基态的概率。
	
	
	
	
	
	
\end{enumerate}
