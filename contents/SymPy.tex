% Python 符号计算简介
% keys 符号计算|python

\subsection{什么是符号计算?}
符号计算又称计算机代数,通俗地说就是用计算机推导数学公式,如对表达式进行因式分解、化简、微分、积分、解代数方程、求解常微分方程等。在SciPy 数值微分与积分\upref{SciPy}部分我们已经介绍了如何利用\verb|python|实现相关数值计算,这里面我们将进一步介绍符号计算在\verb|python|中的实现。那么数值计算与符号计算有什么区别与联系呢?个人认为:首先在数值计算过程中,所有出现的\verb|变量|或者\verb|参数|在使用之前必须给定具体取值,并且计算结果大多数是近似的;相反的是,在符号计算过程中,变量可以预先不给定取值,计算结果是准确的,解析的。不太准确的表述为:符号计算就是对表达式进行的操作;数值计算是对数据进行的操作。

\subsection{SymPy 库}
在\verb|python|中,专门进行符号计算的库是 \href{https://www.sympy.org/en/index.html}{sympy} (symbol python的简写)。利用这个库可以进行符号表达式的加减乘除等四则运算、符号化简、求导、积分、极限、解方程(组)、解微分方程(组)等等。下面我们将进行逐一介绍。

\subsubsection{符号变量的定义}
我们首先导入\verb|sympy|库:
\begin{lstlisting}[language=python]
import sympy as sm
\end{lstlisting}
为了代码简洁,后面默认在代码之前均引入了这个库。例如

定义单个变量(类型为 \verb|sympy.core.symbol.Symbol|)
\begin{lstlisting}[language=python]
x0 = sm.symbols('x0')
\end{lstlisting}
同时定义多个变量
\begin{lstlisting}[language=python]
x0 = sm.symbols('x0')
x1 = sm.symbols('x1')
\end{lstlisting}
这样定义理论上没有问题,但是显得繁琐。上述代码可以改写为
\begin{lstlisting}[language=python]
x0, x1 = sm.symbols('x0, x1')
\end{lstlisting}
对于多个变量的定义方法类似。需要说明的是\verb|sm.symbols('x0, x1')|中的\verb|,|用空格隔开也是可以的。即\verb|sm.symbols('x0 x1')|。

当定义多个连续变量时,也可以这样
\begin{lstlisting}[language=python]
x, y, z = sm.symbols('x:z')
x4, x5, x6, x7 = sm.symbols('x4:8')
\end{lstlisting}
注意不包括\verb|x8|.

\subsubsection{符号常量与函数}
\verb|sympy|库中预置了许多常量:圆周率$\pi$,自然常数$e$,无穷大$\infty$,虚数单位$\I$等等。它们分别为
\begin{lstlisting}[language=python]
sm.pi # 类型为 sympy.core.numbers.Pi
sm.E  # 类型为 sympy.core.numbers.Exp1
sm.I  # 类型为 sympy.core.numbers.ImaginaryUnit
sm.oo # 类型为 sympy.core.numbers.Infinity 
\end{lstlisting}
里面也有许多常用函数,比如 $\sin(x)$, $\arcsin(x)$, $\sinh(x)$, $\exp(x)$, $\log(x)$, $\sqrt{x}$。 对应 \verb|python| 代码为:
\begin{lstlisting}[language=python]
sm.sin(), sm.asin(), sm.sinh(), sm.exp(), sm.log(), sm.sqrt()
# 返回的类型由返回的表达式决定, 例如
# type(sm.sqrt(2)) = sympy.core.power.Pow
# type(sm.sqrt(4)) = sympy.core.numbers.Integer
# type(sm.sqrt(8)) = sympy.core.mul.Mul
# type(sm.sin(sm.pi)) = sympy.core.numbers.Zero
# type(sm.sin(sm.pi/2)) = sympy.core.numbers.One
# type(sm.sin(sm.pi/4)) = sympy.core.mul.Mul
\end{lstlisting}
上面只是列举了部分函数与常量,\verb|sympy|库只能还有非常多,感兴趣的读者可以参考\href{https://docs.sympy.org/latest/index.html}{官方文档}
另外需要注意的是,\verb|sympy|中的符号不能与\verb|numpy, scipy|中的混用,否则会报错,例如
\begin{lstlisting}[language=python]
>>> import numpy as np
>>> x = sm.symbols('x')
>>> np.exp(x) # 尝试用 numpy 来计算 exp(x),注意此处 x 是符号变量.
\end{lstlisting}
此处返回错误
\begin{lstlisting}[language=none]
Traceback (most recent call last):
File "<stdin>", line 1, in <module>
AttributeError: 'Symbol' object has no attribute 'exp'
\end{lstlisting}
正确的用法为
\begin{lstlisting}[language=python]
>>> sm.exp(x) # Use SymPy's version instead.
exp(x)
\end{lstlisting}
我们在看这个例子
\begin{lstlisting}[language=python]
>>> x = sm.symbols('x')
>>> (2/3) * sm.sin(x) # 2/3 返回一个浮点数而不是有理数.
0.666666666666667*sin(x)
>>> sm.Rational(2, 3) * sm.sin(x) # 使用有理数 2/3
\end{lstlisting}
为了巩固上面符号计算的一些细节我们来做一个练习。
\begin{example}{}
用\verb|sympy|库的函数写一个函数返回如下表达式:
\begin{align}
\frac{2}{5}\E^{x^2-y}\cosh(x+y)+\frac{1}{2}\log(xy+1)~.
\end{align}


对应代码如下
\begin{lstlisting}[language=python]
def cal():
   x, y  = sm.symbols('x,y')
   express = sm.Rational(2,5)*sm.exp(x**2-y)* \
   sm.cosh(x+y)+Rational(1,2)*sm.log(x*y+1)
   return express
\end{lstlisting}
\end{example}

\subsubsection{符号加与符号乘运算}
利用\verb|sympy|计算$\sum_{i=1}^4{x+\I y}$与$\prod_{i=1}^5{x+\I y}$
\begin{lstlisting}[language=python]
>>> x, y, i = sm.symbols('x y i') # 定义三个符号变量
>>> sm.summation(x + i*y, (i, 1, 4)) # 求和
4*x + 10*y
>>> sm.product(x + i*y, (i, 0, 5)) # 求积
x*(x + y)*(x + 2*y)*(x + 3*y)*(x + 4*y)*(x + 5*y)
\end{lstlisting}

\subsubsection{符号表达式化简}
\begin{table}[ht]
\centering
\caption{常用的化简函数}\label{tab_SymPy_1}
\begin{tabular}{c|c}
Function &Description \\
\hline
\verb|sm.cancel()| &分子分母化简 \\
\hline
\verb|sm.expand()| &展开表达式\\
\hline
\verb|sm.factor()| &因式分解表达式\\
\hline
\verb|sm.radsimp()| &分母有理化\\
\hline
\verb|sm.simplify()| &化简表达式\\
\hline
\verb|sm.trigsimp()| &仅仅化简表达式的三角函数部分\\
\hline
\end{tabular}
\end{table}

例如,化简表达式
\begin{lstlisting}[language=python]
>>> x = sm.symbols('x')
>>> expr = (x**2 + 2*x + 1) / ((x+1)*((sm.sin(x)/sm.cos(x))**2 + 1))
>>> print(expr)
(x**2 + 2*x + 1)/((x + 1)*(sin(x)**2/cos(x)**2 + 1))
>>> sm.simplify(expr) # 化简表达式
(x + 1)*cos(x)**2
\end{lstlisting}

展开表达式
\begin{lstlisting}[language=python]
>>> expr = sm.product(x + i*y, (i, 0, 3))
>>> print(expr)
x*(x + y)*(x + 2*y)*(x + 3*y)
>>> expr_long = sm.expand(expr) # 展开表达式
>>> print(expr_long)
x**4 + 6*x**3*y + 11*x**2*y**2 + 6*x*y**3
\end{lstlisting}

约分表达式
\begin{lstlisting}[language=python]
>>> expr_long = expr_long / (x + 3*y)
>>> expr_short = sm.cancel(expr_long) # Cancel out the denominator.
x**3 + 3*x**2*y + 2*x*y**2
\end{lstlisting}

因式分解
\begin{lstlisting}[language=python]
>>> sm.factor(expr_short) 
x*(x + y)*(x + 2*y)

\end{lstlisting}

三角函数化简
\begin{lstlisting}[language=python]
>>> sm.trigsimp(2*sm.sin(x)*sm.cos(x))
sin(2*x)
\end{lstlisting}
需要说明的是以上各种表达式化简不会改变原有表达式。

\subsubsection{替换与计算表达式}
\verb|sympy|库中提供了表达式替换函数\verb|subs|与计算函数\verb|evalf|, 它的用法为
\begin{lstlisting}[language=python]
# 定义两个符号变量
>>> x,y = sm.symbols('x y')
>>> expr = sm.expand((x + y)**3)
>>> print(expr)
x**3 + 3*x**2*y + 3*x*y**2 + y**3
# 用2*x替换原来表达式中的y
>>> expr.subs(y, 2*x)
27*x**3
# 同时替换多个变量
>>> new_expr = expr.subs({x:sm.pi, y:1})
>>> print(new_expr)
1 + 3*pi + 3*pi**2 + pi**3
# 将符号计算转为数值计算
>>> new_expr.evalf() 
71.0398678443373
>>> expr.evalf(subs={x:1, y:2})
27.0000000000000
\end{lstlisting}
另外,\verb|sympy|库还提供了函数\verb|lambdify|函数,此函数相当于匿名函数,它的用法是
\begin{lstlisting}[language=python]
 sm.lambdify(变量或变量列表, 符号表达式)
\end{lstlisting}
例如
\begin{lstlisting}[language=python]
# 定义单个符号变量
>>> f = sm.lambdify(x, sm.sin(x)**2)
>>> print(f(0), f(np.pi/2), f(np.pi), sep=' ')
0.0 1.0 1.4997597826618576e-32
# 定义多个符号变量
>>> f = sm.lambdify((x,y), sm.sin(x)**2 + sm.cos(y)**2)
>>> print(f(0,1), f(1,0), f(np.pi, np.pi), sep=' ')
0.2919265817264289 1.708073418273571 1.0
\end{lstlisting}
仔细的观察我们发现,虽然我们定义的是符号变量与符号表达式,但是该函数返回值是浮点型数据。默认情况下,\verb|lambdify|函数调用了\verb|math|模块,也就是说\verb|sm.sin()|被转化为\verb|math.sin()|.

\subsubsection{符号方程(组)求解}
\verb|sympy|库中提供了\verb|solve|函数进行方程的求解,它的用法为
\begin{lstlisting}[language=python]
sm.solve(表达式,求解变量)
\end{lstlisting}
它的返回值是列表形式。例如
\begin{lstlisting}[language=python]
# 定义两个符号变量x y
>>> x,y = sm.symbols('x y')
# 求解方程 x^2 - 2x + 1 = 0,在第一个参数中不需要写都等于0
>>> sm.solve(x**2 - 2*x + 1, x)
[1] 
#求解x^2 - 1 = 0
>>> sm.solve(x**2 - 1, x)
[-1, 1]
>>> sm.solve(x/(y-x) + (x-y)/y, x)
[y*(-sqrt(5) + 3)/2, y*(sqrt(5) + 3)/2]
#求解方程组
>>> sm.solve([x + y - 1,x - y -3],[x,y])
 {x: 2, y: -1}
\end{lstlisting}

利用\verb|solve_linear_system|求解线性方程组,例如
\begin{align}
x+y+z=5 ~.\\
2 x+4 y+3 z=2~. \\
5 x+10 y+2 z=4~.
\end{align}

对应的代码为:
\begin{lstlisting}[language=python]
>>> x, y, z = sm.symbols('x y z')
# 首先写出系数增广矩阵
>>> M = sm.Matrix([ [1, 1, 1, 5],
[2, 4, 3, 2],
[5, 10, 2, 4] ])
# 调用函数
>>> sm.solve_linear_system(M, x, y, z)
{x: 98/11, y: -45/11, z: 2/11}
# 它的返回值为字典类型。
\end{lstlisting}

\subsubsection{符号求导}
\verb|sympy|提供了两种方法计算符号表达式的导数,分别为\verb|sm.Derivative()|与\verb|sm.diff()|,用法为:
\begin{lstlisting}[language=python]
>>> x, y = sm.symbols('x y')
>>> f = sm.sin(y) * sm.cos(x)**2
# 对符号变量x求导
>>> df = sm.Derivative(f, x)
>>> print(df)
Derivative(sin(y)*cos(x)**2, x)
# Derivative与doit结合使用才能实际求导,
# Derivative只返回一个形式导数
>>> df.doit() 
-2*sin(x)*sin(y)*cos(x)
# diff方法直接求导
>>> sm.diff(f, x)
-2*sin(x)*sin(y)*cos(x)
# 分别对x,y,x进行多次求导
>>> sm.diff(f, x, y, x)
2*(sin(x)**2 - cos(x)**2)*cos(y) # N
\end{lstlisting}

\subsubsection{符号积分}
\verb|sympy|提供了两种方法计算符号表达式的积分,分别为\verb|Integral()|与\verb|sm.integrate()|,这两个方法与求导的两种方法一一对应。用法为:
\begin{lstlisting}[language=python]
# 计算不定积分
>>> sm.integrate(sm.sec(x), x)
-log(sin(x) - 1)/2 + log(sin(x) + 1)/2
# 计算定积分
>>> sm.integrate(sm.cos(x)**2, (x,0,sm.pi/2))
pi/4
#计算多重积分
>>> sm.integrate(y**2 * x**2, (x,0,2), (y,-1,1))
16/9
\end{lstlisting}
