% 电流
% 电流|电荷|截面

\begin{issues}
\issueDraft
\end{issues}

\pentry{导数\upref{Der}}

我们所熟知的\textbf{电流},是导线中载流子(那些能够被电场驱动而运动的电荷)的定向运动.在一般的导体中,载流子常常是自由电子;而一些半导体材料中载流子也可以是空穴(可以看作是携带一个单位正电荷的粒子).那么电流的方向与电子定向运动的方向相反,与空穴运动的方向相同.一个简化版的示意图如\autoref{I_fig1} 所示.
\begin{figure}[ht]
\centering
\includegraphics[width=10cm]{./figures/I_1.pdf}
\caption{电流的经典模型.本图以正电荷为载流子.} \label{I_fig1}
\end{figure}

若在导线中取一个任意横截面,并定义一个正方向,通过该截面的电流定义为
\begin{equation}\label{I_eq1}
I = \dv{q}{t}
\end{equation}
其中 $\dd{q}$ 为 $\dd{t}$ 时间内由正方向通过截面的净电荷量\footnote{具体来说,正电荷向正方向流动和负电荷向负方向流动,电流都取正号,反之取负号.}.

考虑一根电荷(只考虑那些载流子)线密度为 $\lambda$ 的导线中电荷都以速度 $v$ 沿正方向运动,时间 $\dd{t}$ 内通过的长度为 $v\dd{t}$,通过的电荷为 $\lambda v\dd{t}$,所以电流为
\begin{equation}
I = \lambda v
\end{equation}
或者可以再考虑导线的横截面积 $S$,设载流子的电荷为 $q$,数密度为 $n$,那么有 $\lambda = q S n$.因此电流还可以表示成
\begin{equation}
I = q Snv
\end{equation}
\subsection{电流的微观模型}