% 线性方程组与增广矩阵
% keys 线性映射|齐次方程|解集|零空间
% license Xiao
% type Tutor

\begin{issues}
\issueDraft
\end{issues}

\pentry{线性方程组\nref{nod_LinEqu}}{nod_e021}

\subsection{增广矩阵}

下面我们从线性代数(矩阵)的角度理解线性方程组。

考虑 $m$ 个 $n$ 元一次方程构成的线性方程组:
\begin{equation}
\leftgroup{
a_{1 1}x_1 + a_{1 2}x_2 + \dots + a_{1 n}x_n &= y_1, \\
a_{2 1}x_1 + a_{2 2}x_2 + \dots + a_{2 n}x_n &= y_2, \\
\vdots \\
a_{m 1}x_1 + a_{m 2}x_2 + \dots + a_{m n}x_n &= y_m~.}
\end{equation}

我们可以把他改写成矩阵与列向量的等式:
\begin{equation}
\mat{A} \bvec{x} = \bvec{b} ~,
\end{equation}

其中
\begin{equation}
\mat{A} = \bmat{
a_{1 1} & a_{1 2} & \dots & a_{1 n} \\
a_{2 1} & a_{2 2} & \dots & a_{2 n} \\
\vdots & \vdots & \ddots & \vdots \\
a_{m 1} & a_{m 2} & \dots & a_{m n}
}, \bvec{b} = \bmat{
    y_1 \\
    y_2 \\
    \vdots \\
    y_m
}~.
\end{equation}

更近一步的,我们会使用\textbf{增广矩阵} $\bmat{\bvec{A} \mid \bvec{b}}$,即
\begin{equation}
\left[{\begin{array}{cccc|c}
a_{1 1} & a_{1 2} & \dots & a_{1 n} & y_1 \\
a_{2 1} & a_{2 2} & \dots & a_{2 n} & y_2 \\
\vdots & \vdots & \ddots & \vdots & \vdots \\
a_{m 1} & a_{m 2} & \dots & a_{m n} & y_m 
\end{array}}\right]~
\end{equation}
来表示一个线性方程组。

\subsection{超定、恰定、不定方程组}

我们称 $m$ 个 $n$ 元一次方程构成的线性方程组为
\begin{enumerate}
\item \textbf{超定的}(overdetermined),如果 $m > n$;
\item \textbf{恰定的},如果 $m = n$;
\item \textbf{不定的/欠定的}(underdetermined),如果 $m < n$;
\end{enumerate}



