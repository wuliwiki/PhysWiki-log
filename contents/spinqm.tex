% 自旋与旋转
% license Usr
% type Tutor
\pentry{自旋角动量\upref{Spin}}



\subsubsection{自旋对态矢作用}
 通过对角动量理论的学习,我们已经知道,在向量空间中,$\mathrm e^{-\mathrm i\hat J_i\phi}$可以描述经典向量绕生成元$\hat J_i$对应的轴旋转$\phi$。但对于可以选择任意表象的态矢而言,这种绕固定轴的转动还可以影响别的观测结果,即期望值。

以三维空间中自旋$1/2$的粒子为例,其自旋期望值为$(\overline{\hat S_x},\overline{\hat S_y},\overline{\hat S_z})$。设该粒子的初始态矢为$\ket{a}$,态矢绕$z$轴“转动”后变为$\mathrm e^{-\mathrm i \hat S_z\phi}\ket{a}$。则期望值变化为:

\begin{equation}
\bra{a}\hat S_i\ket{a}\rightarrow \bra{a}\mathrm e^{\mathrm i \hat S_z\phi}\hat S_i\mathrm e^{-\mathrm i \hat S_z\phi}\ket{a}~.
\end{equation}
在$\hat S_z$表象下计算$\mathrm e^{\mathrm i \hat S_z\phi}\hat S_x\mathrm e^{-\mathrm i \hat S_z\phi}$得:

\begin{equation}
\begin{aligned}
\mathrm e^{\mathrm i \hat S_z\phi}\hat S_x\mathrm e^{-\mathrm i \hat S_z\phi}&=\mathrm e^{\mathrm i \hat S_z\phi}\left(\frac{1}{2}(\ket{-}\bra{+}+\ket{+}\bra{-})\right)\mathrm e^{-\mathrm i \hat S_z\phi}\\
 &=\frac{1}{2}\left(\mathrm e^{-\mathrm i \phi}\ket{-}\bra{+}+\ket{+}\bra{-}\mathrm e^{\mathrm i \phi}\right)\\
 &=\frac{1}{2}\left[\opn{cos}\phi(\ket{-}\bra{+}+\ket{+}\bra{-})+\mathrm i\opn{sin}\phi(\ket{+}\bra{-}-\ket{-}\bra{+})\right]\\
 &=\opn{cos}\phi \hat S_x-\opn{sin}\phi \hat S_y~.
\end{aligned}
\end{equation}
因此,$\hat S_x$的期望值变化为:
\begin{equation}
\overline{\hat S_x}\rightarrow  \overline{\hat S_x}\opn{cos}\phi-\overline{\hat S_y}\opn{sin}\phi~.
\end{equation}
同理可以计算出其他分量的期望值变化:
\begin{equation}
\overline{\hat S_y}\rightarrow \overline{\hat S_y}\opn{cos}\phi+\overline{\hat S_x}\opn{sin}\phi~,
\end{equation}
\begin{equation}
\overline{\hat S_z}\rightarrow \overline{\hat S_z}~.
\end{equation}
因此,自旋期望值可看作经典矢量,态矢绕自旋$z$分量“旋转”相当于该矢量绕自旋$z$分量“旋转”:
\begin{equation}
\begin{pmatrix}
 \opn{cos}\phi &-\opn{sin}\phi  &0 \\
  \opn{sin}\phi & \opn{cos}\phi  & 0\\
  0& 0 &1
\end{pmatrix}
\begin{pmatrix}
 \overline{\hat S_x}\\
  \overline{\hat S_y}\\
 \overline{\hat S_z}
\end{pmatrix}
=
\begin{pmatrix}
  \overline{\hat S'_x}\\
  \overline{\hat S'_y}\\
 \overline{\hat S'_z}
\end{pmatrix}~.
\end{equation}
可以利用贝克-豪斯多夫(Baker-Hausdorff)公式计算$\mathrm e^{\mathrm i \hat S_z\phi}\hat S_x\mathrm e^{-\mathrm i \hat S_z\phi}$。
计算过程表明自旋期望值的变化适用于任意角动量期望值的变化(即也适用于轨道角动量算子期望值)。

也就是说,态矢绕任意角动量分量$\hat J_i$的指向旋转$\phi$,相当于该角动量期望值矢量$\overline{\hat J_i}$在三维空间的旋转。

