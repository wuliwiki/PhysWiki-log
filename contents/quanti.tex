% 标量场的量子化
% keys 标量场|量子化
% license Xiao
% type Tutor
\pentry{经典场论基础\nref{nod_classi},标量场\nref{nod_qed2}}{nod_6dd3}
这一节的目的是给大家介绍如何对最简单的克莱因-戈登场进行量子化。

量子化的步骤是:把 $\phi$ 场和 $\pi$ 场升级为算符,然后在上面加入合适的对易关系。在场论里,因为 $\phi$ 可以类比于坐标,而 $\pi$ 可以类比于动量,那么场的正则对易关系为
\begin{equation}
\begin{aligned}
& [\phi(\mathbf x),\pi(\mathbf y)] = i \delta^{(3)}(\mathbf x- \mathbf y) \\
& [\phi(\mathbf x),\phi(\mathbf y)]  = [\pi(\mathbf x),\pi(\mathbf y)] = 0~,
\end{aligned}
\end{equation}
一般来说在动量空间里面研究问题比较方便。那么我们把 $\phi$ 场换到动量空间中。那么克莱因-戈登方程的形式为
\begin{equation}
\bigg[\frac{\partial^2}{\partial t^2}+(|\mathbf p|^2+m^2)\bigg] \phi(\mathbf p, t) = 0~,
\end{equation}
这也就是一个能量为 $\omega_{\mathbf p}$ 的简谐振子的运动方程。$\omega_{\mathbf p}$ 的表达式如下
\begin{equation}
\omega_{\mathbf p} = \sqrt{|\mathbf p|^2+m^2}~,
\end{equation}
现在我们来找克莱因-戈登场的谱。用的是跟量子力学里面学到的方法类似的方法,首先我们要对场 $\phi$ 和场 $\phi$ 进行量子化
\begin{equation}\label{eq_quanti_3}
\begin{aligned}
& \phi(\mathbf x) = \int \frac{d^3p}{(2\pi)^3} \frac{1}{\sqrt{2\omega_{\mathbf p}}}\bigg( a_{\mathbf p} e^{i \mathbf p \cdot \mathbf x} + a_{\mathbf p}^\dagger e^{-i\mathbf p \cdot \mathbf x} \bigg)~, \\
& \pi(\mathbf x) = \int \frac{d^3p}{(2\pi)^3} (-i) \sqrt{\frac{\omega_{\mathbf p}}{2}} \bigg( a_{\mathbf p} e^{i \mathbf p \cdot \mathbf x} - a^{\dagger}_{\mathbf p} e^{-i \mathbf p \cdot \mathbf x} \bigg)~.
\end{aligned}
\end{equation}
可以证明,正则对易关系可以化简为如下的形式
\begin{equation}\label{eq_quanti_2}
[a_{\mathbf p},a_{\mathbf p'}^\dagger] = (2\pi)^3 \delta^{(3)} (\mathbf p - \mathbf p')~.
\end{equation}
前面我们已经推导过哈密顿量的表达式,现在把 $\phi$ 场和 $\pi$ 场的表达式代入,就可以得出用产生湮灭算符来表示的哈密顿量的表达式。
\begin{equation}\label{eq_quanti_1}
H = \int \frac{d^3p}{(2\pi)^3} \omega_{\mathbf p} \bigg(  a^\dagger_{\mathbf p} a_{\mathbf p} + \frac{1}{2} [a_{\mathbf p},a^\dagger_{\mathbf p}] \bigg)~.
\end{equation}
我们可以看出,这里的第二项是正比于 $\delta(0)$ 的。这一项是真空能项。这一项实验上是测不到的。因为实验只能测到跟基态的差值。因此以后我们都会忽略这一项。

由\autoref{eq_quanti_1} 以及对易关系,可以推导出
\begin{equation}
[H,a_{\mathbf p}^\dagger] = \omega_{\mathbf p} a^\dagger_{\mathbf p}~, \quad [H,a_{\mathbf p}] = -\omega_{\mathbf p} a_{\mathbf p}~.
\end{equation}

$a_{\bvec p},a_{\bvec p}^\dagger$ 被称为产生、湮灭算符。$a_p^\dagger a_p$ 的本征值为非负整数,其物理意义是动量为 $\bvec p$ 的粒子的数目。因此,场的动量算符可以用产生湮灭算符表达为:
\begin{equation}
\bvec P=-\int \frac{\dd{}^3 \bvec x}{(2\pi)^3}\dot\phi\nabla\phi=\int \frac{\dd{}^3 \bvec p}{(2\pi)^3 }\bvec k a_{\bvec p}^\dagger a_{\bvec p}~.
\end{equation} 

我们可以从\autoref{eq_quanti_1} 看出,自由标量场的哈密顿量在动量空间就是一系列无相互作用的谐振子的和,因此量子化的自由标量场的希尔伯特空间可以由所有谐振子的基态,再加上所有可能的产生算符构造出来。这种特殊的粒子数表象中的希尔伯特空间(Hilbert Space)被称为 Fock 空间。自由标量场的基态是唯一的,我们将它记作 $|0\rangle$,也被称为真空态。任意湮灭算符作用于基态仍得到基态。
\begin{equation}
a_{\bvec p} |0\rangle = |0\rangle,\forall \bvec p~.
\end{equation}
选取归一化条件 $\langle 0|0\rangle=1$。在真空态之上的激发被解释为粒子,即可以利用产生算符 $a^\dagger$ 构造单粒子态
\begin{equation}
|\bvec p\rangle = \sqrt{2\omega_{\bvec p}}a^\dagger_{\bvec p} |0\rangle~.
\end{equation}
可以继续将产生算符 $a^\dagger$ 作用于这个单粒子态,这相当于有多个粒子占据这一量子态,即 Klein-Gordon 场描述的是玻色子系统。$\langle \bvec k|\bvec p\rangle = 2\omega_{\bvec k}(2\pi)^3\delta(\bvec k-\bvec p)$。容易证明这是洛伦兹不变量。与这些单粒子态对应的完备关系可表达为
\begin{equation}
\mathbb{1}_1 = \int \frac{\dd{}^3 \bvec k}{(2\pi)^{3}}\frac{1}{2\omega_{\bvec k}}|\bvec k\rangle \langle \bvec k|~.
\end{equation}
$\mathbb{1}_1$ 表示单粒子态子空间中的单位算符。$\dd{}^3 \bvec k/((2\pi)^3 2\omega_{\bvec k})$ 为洛伦兹不变的体积元,这可以从下式看出
\begin{equation}
\int \frac{\dd{}^3\bvec k}{(2\pi)^3}\frac{1}{2\omega_{\bvec k}}=\int \frac{\dd{}^4 k}{(2\pi)^3} \delta(k^2-m^2)~.
\end{equation}
产生算符作用于基态得到的一切量子态构成 Fock 空间的一组基底,即这些多粒子态在整个 Hilbert 空间上是完备的。

下面我们来探究量子态以及算符在洛伦兹变换下的结果。在 Fock 空间上,坐标的洛伦兹变换体现为量子化的算符的幺正变换。对于一个洛伦兹变换 $\Lambda$,它将体系的坐标 $x$ 变为 $x'=L x$,动量 $\bvec k$ 变为 $\bvec k'$,记为 $k'=L \bvec k$。设 $\Lambda$ 对应的 Hilbert 空间的算符为 $U(\Lambda)$,那么我们记
\begin{equation}
U(\Lambda)|\bvec k\rangle=|L \bvec k\rangle~.
\end{equation}
$U(\Lambda)$ 为幺正变换,所以单粒子态之间的内积 $\langle \bvec k|\bvec k'\rangle$ 在洛伦兹变换下保持不变。那么对于产生湮灭算符,我们要求
\begin{equation}
\begin{aligned}
&U(\Lambda)a^\dagger_{\bvec k}U^{-1}(\Lambda)=\sqrt{\frac{\omega_{L \bvec k}}{\omega_{\bvec k}}} a^\dagger_{L \bvec k}~,\\
&U(\Lambda)a_{\bvec k}U^{-1}(\Lambda)=\sqrt{\frac{\omega_{L \bvec k}}{\omega_{\bvec k}}} a_{L \bvec k}~.
\end{aligned}
\end{equation}
类似地,可以得到场算符 $\phi(x)$ 的表达式
\begin{equation}\label{eq_quanti_4}
\begin{aligned}
U(\Lambda)\phi(x)U^{-1}(\Lambda)=\phi(Lx)~.
\end{aligned}
\end{equation}

