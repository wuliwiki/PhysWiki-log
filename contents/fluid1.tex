% 流体运动的描述方法
% keys 欧拉法|拉格朗日法
% license Usr
% type Tutor

\begin{issues}
\issueDraft
\end{issues}

\pentry{速度 加速度\nref{nod_VnA}}{nod_d89c}
流体力学中,流体运动的描述方法包括拉格朗日法与欧拉法,两种方法在物理意义和数学表达方面各有特点,下面我们具体解释。

\subsection{流体与固体的区别}
我们从流体与固体的区别引入两种描述方法,固体无论处于静止还是运动都可以通过有限的静变形承受剪切力,其形状不易变化,在运动学中,可以只从几何角度来描述物体的位置随时间的变化。

流体在静止时无法通过有限的静变形承受剪切力,在运动状态下虽能产生剪切力,剪切力却不能维持流体内部各点位置的有序,反而使其产生连续不断的变形,流体内部各质点位置变化很大,单从几何角度描述其运动将复杂而困难,需要专门的处理方法。

固体力学中,只着眼于需要描述的物体,物体之外的都叫做环境,某一物体在环境中运动,与环境之间发生力的作用从而改变运动状态,这种运动描述方法称为拉格朗日法。

流体也适用拉格朗日法,但流体不断变形,从几何角度需要进行复杂的坐标变换,因此,直接描述一小部分流体在整个流场空间(环境)的运动不大方便。我们考虑是不是可以只研究一个特定不变的空间呢,着眼于流体经过这个空间时发生的变化以及与这个空间的相互作用,是否也能全面描述流体的运动呢,实践证明这也是可行的,这种方法称为欧拉法。

\subsection{拉格朗日法}
我们先尝试使用拉格朗日法描述流体的运动,考虑一个无穷小的不规则流体微团A,为研究其运动,我们跟随此微团在时间和空间中运动,从而描述微团的空间位置随时间变化的轨迹。可以想象我们是坐在船上,船跟随河水向前运动。

运动过程中,微团形状不断变化,但仍将其看作一个整体,独立的变量为时间 $t$ 与变动的空间坐标 $s=(x,y,z)$,拉格朗日法这样描述流体微团的运动:

在 $t$ 时刻,微团A的空间位置为 $\bvec r=r(t,s)$,某一瞬时的位置与时间和空间坐标都有关。

在 $t$ 时刻,微团A的速度为 $\bvec V=\dv*{\bvec r}{x}$,
某一瞬时的速度是空间坐标对时间的导数。

在 $t$ 时刻,微团A的加速度为 $\bvec a=\dv*{\bvec V}{x}=\dv*[2]{\bvec r}{x}$。

因为微团在空间不断运动,拉格朗日法的坐标本身也在空间内不断变化,由于一般的流体微团内部都包含大量的质点,且由于微团的变形,不同质点的运动情况复杂多样,拉格朗日法不再方便。

\subsection{欧拉法}
欧拉法描述流体运动的方式则是将注意力集中到流体随时间流过的固定的空间位置A上,研究一个固定的发生流体运动的空间,空间形状可以任意而空间坐标是固定的。可以想象我们站在桥上,观察桥下河水的流动。

欧拉法的独立变量是时间 $t$ 和所研究空间的固定坐标 $s=(x,y,z)$,欧拉法这样描述通过固定空间的流体的运动:

在 $t$ 时刻,A处流体质点的速度为 $\bvec V=u(t.x,y,z)\uvec i+v(t,x,y,z)\uvec j+w(t,x,y,z)\uvec k$,速度是某一时刻位于空间某一点的流体质点的速度,其与时间和空间坐标都有关。

至于 $t$ 时刻的加速度,需要使用物质导数(Substantial  Derivative)的概念,参见\enref{物质导数}{fluid2},实际上,此时的A处流体质点的加速度为 $\bvec a=\pdv*{\bvec V}{t}+\left ( \bvec V\cdot \Nabla \right )\bvec V$,这个式子由两部分组成,前一部分只与时间相关,后一部分与空间相关,使用这种表达,欧拉法更容易描述流体运动的具体情况。
