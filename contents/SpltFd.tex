% 分裂域
% keys splitting field|正规扩张|regular extension
% license Xiao
% type Tutor


\pentry{域的扩张\upref{FldExp}}

本节我们要介绍一个在代数中非常基础且重要的概念:分裂域。简单来说,分裂域就是在一个域中添加某个多项式的全体根所得到的扩域。从分裂域出发,我们可以讨论代数扩域的自同构问题。

关于分裂域的进一步讨论,请参阅\textbf{正规扩张}\upref{NomEx}词条。


\subsection{分裂域的存在性}


\begin{definition}{分裂域}
给定域$\mathbb{F}$及其上一个多项式$f(x)$。若存在扩域$\mathbb{K}/\mathbb{F}$,使得$f(x)$在$\mathbb{K}$上可以分解为$f(x)=\prod_{i=1}^n (x-a_i)$,且$\mathbb{K}=\mathbb{F}(a_1, a_2, \cdots, a_n)$,则称$\mathbb{K}$是$f(x)\in \mathbb{F}[x]$上的\textbf{分裂域(splitting field)}。
\end{definition}

定义看起来有些绕口,先说$f$在$\mathbb{K}$中可以分解,也就是说每一个根都存在,再说$\mathbb{K}$可以看成用这些根对$\mathbb{F}$进行扩域的结果。这么定义是因为我们要先确定元素$a_i$都存在,而为此就需要先确定$\mathbb{K}$存在。但是定义中只说了“若$\mathbb{K}$存在”,这个假设到底成立与否呢?答案是肯定的。

\begin{theorem}{分裂域的存在性}
给定域$\mathbb{F}$及其上一个多项式$f(x)$,则$f(x)\in \mathbb{F}[x]$上的分裂域存在。
\end{theorem}

\textbf{证明}:

当$\opn{deg}f=1$时,定理自然成立,此时$f\in\mathbb{F}[x]$的分裂域就是其本身。

首先在环$\mathbb{F}[x]$上对元素$f(x)$进行因式分解\footnote{也就是画出它的一棵\textbf{真因子树}\upref{FctTre}。},得到其不可约因子。任选其中一个不可约因子$h(x)$,如果$\opn{deg} h = 1$,则跳过本段接下来的步骤。构造商环$\mathbb{F}(x)/\langle h(x) \rangle =\mathbb{F}[a_1]=\mathbb{F}(a_1)$
\footnote{由于$h(x)$是$\mathbb{F}[x]$中的不可约元素,且$\mathbb{F}[x]$是主理想整环,因此易证$\langle h(x) \rangle$是$\mathbb{F}[x]$上的极大理想,从而$\mathbb{F}(x)/\langle h(x) \rangle$是域。}
,记为$\mathbb{F}_1$。

由\textbf{多项式环}\upref{RPlynm}的\autoref{the_RPlynm_1}~\upref{RPlynm},$(x-a_1)|h(x)$,因此在$\mathbb{F}_1$上可以分解出$h_1(x)=h(x)/(x-a_1)$。如果$\opn{deg}h_1 = 1$,则跳过本段接下来的步骤。对$h_1(x)$进行相同的操作:构造商环$\mathbb{F}(x)/\langle h_1(x) \rangle =\mathbb{F}[a_2] = \mathbb{F}(a_2)=\mathbb{F}_2$。

以此类推,直到$h(x)$在$\mathbb{F}_{k_1}$上分解为一阶多项式之积。

接下来,取$f$在$\mathbb{F}_{k_1}$上的不可约因子$g(x)$,如果$\opn{deg} g = 1$,则跳过本段接下来的步骤。执行相同的扩域操作,直到得到$\mathbb{F}_{k_1+k_2}$,使得$g$在$\mathbb{F}_{k_1+k_2}$上分解为一阶多项式之积。

以此类推,最终可以得到$\mathbb{F}_k$,使得$f$在$\mathbb{F}_k$上可以分解为一阶多项式之积。则$\mathbb{F}_k$就是$f\in\mathbb{F}[x]$的分裂域。

\textbf{证毕}。

该证明过程的大体思路,就是看$f$的根是否在已知的域中。根$a$的最小多项式$h(x)$必是$f(x)$的一个不可约因子。如果$a$在已知的域中,那么$f$就可以因式分解出一阶多项式因子$(x-a)$;否则,就添加$a$进行一次单扩域,这次扩域至少能把$a$纳入,但也有可能把其它根一起纳入。因此我们容易得到以下推论:

\begin{corollary}{}
设$\mathbb{K}$是$f(x)\in \mathbb{F}[x]$上的分裂域,则$[\mathbb{K}:\mathbb{F}]\leq \opn{deg}f$。
\end{corollary}

\begin{corollary}{}
设$\mathbb{K}$是$f(x)\in \mathbb{F}[x]$上的分裂域,$\mathbb{M}$是$\mathbb{K}$和$\mathbb{F}$之间的\textbf{中间域},则$\mathbb{K}$也是$f(x)\in \mathbb{E}[x]$上的分裂域。
\end{corollary}

为了加深理解,我们讨论一个分裂域的例子。添加元素的过程中会遇到的主要情况在这里都出现了。

\begin{example}{分裂域的一个例子}


在有理数域$\mathbb{Q}$上有多项式$f(x)=(x^2-2)^2(x^2-3)(x^2-6)(x^2+1)$,其在$\mathbb{Q}$上有五个阶数大于$1$的不可约因子:$(x^2-2), (x^2-2), (x^2-3), (x^2-6), (x^2+1)$。

考虑因子$(x^2-2)$,得到扩域$\mathbb{Q}(\sqrt{2})$。在$\mathbb{Q}(\sqrt{2})$上,$f$有分解:
\begin{equation}
f(x)=(x+\sqrt{2})^2(x-\sqrt{2})^2(x^2-3)(x^2-6)(x^2+1)~,
\end{equation}

取其阶数大于$1$的不可约因子$x^2-3$,得到扩域$\mathbb{Q}(\sqrt{2}, \sqrt{3})$。

在$\mathbb{Q}(\sqrt{2}, \sqrt{3})$上,$f$有分解:

\begin{equation}
\begin{aligned}
f(x)=&(x+\sqrt{2})^2(x-\sqrt{2})^2(x+\sqrt{3})(x-\sqrt{3})\times\\
&(x+\sqrt{2}\sqrt{3})(x-\sqrt{2}\sqrt{3})(x^2+1)~,
\end{aligned}
\end{equation}

取其阶数大于$1$的不可约因子$x^2+1$,得到扩域$\mathbb{Q}(\sqrt{2}, \sqrt{3}, \I)$。

你可以验证,在最后这个扩域下,$f$可分解为一阶多项式之积。因此
\begin{equation}
\begin{aligned}
&\mathbb{Q}(\sqrt{2}, \sqrt{3}, \I)=\\
&\{a+A\I+(b+B\I)\sqrt{2}+(c+C\I)\sqrt{3}\\
&+(d+D\I)\sqrt{6} | a, A, b, B, c, C, d, D\in\mathbb{Q}\}~,
\end{aligned}
\end{equation}
就是$f\in\mathbb{Q}[x]$的分裂域。

\end{example}

\begin{corollary}{}
设$\mathbb{K}$是$f(x)\in \mathbb{F}[x]$上的分裂域,则$[\mathbb{K}:\mathbb{F}]\leq \opn{deg}f$。
\end{corollary}

\begin{corollary}{}
设$\mathbb{K}$是$f(x)\in \mathbb{F}[x]$上的分裂域,$\mathbb{M}$是$\mathbb{K}$和$\mathbb{F}$之间的\textbf{中间域},则$\mathbb{K}$也是$f(x)\in \mathbb{E}[x]$上的分裂域。

\end{corollary}


注意,给定一个域$\mathbb{F}$和其上一个\textbf{不可约}多项式$f$,则$f\in\mathbb{F}[x]$的分裂域\textbf{不一定是}$\mathbb{F}[x]/\langle f(x) \rangle $,因为添加$f$的一个根进行单扩张,不一定囊括了$f$的所有根。




\begin{example}{单扩张不等于分裂域的例子}

在$\mathbb{Q}$上添加$x^3-2$的一个根$\sqrt[3]{2}$得到$\mathbb{Q}(\sqrt[3]{2})$,但这个单扩域里并没有$x^3-2$的剩下两个根$\omega\sqrt[3]{2}$和$\omega^2\sqrt[3]{2}$,其中$\omega=-1/2+\I\sqrt{3}/2$是$3$次单位根。

\end{example}










\subsection{分裂域的唯一性}

从\textbf{开拓}(\autoref{def_FldExp_6}~\upref{FldExp})的角度来说,如果存在域同构$\sigma:\mathbb{F}_1\to\mathbb{F}_2$,将其开拓为环同构$\sigma:\mathbb{F}_1[x]\to\mathbb{F}_2[x]$,任取$f\in\mathbb{F}_1[x]$,设$f\in\mathbb{F}_1[x]$的分裂域为$\mathbb{K}_1$,$\sigma(f)\in\mathbb{F}_2[x]$的分裂域为$\mathbb{K}_2$,则$\sigma$可以开拓为$\mathbb{K}_1\to\mathbb{K}_2$的同构。

上述开拓的角度或许有些绕,但考虑到“同构的域就是同一个域”,我们完全可以大大简化上述表达:

\begin{theorem}{}
给定域$\mathbb{F}$和其上一个多项式$f$以后,所构造出来的分裂域是唯一的,或者说构造出来的两个分裂域都是同构的。
\end{theorem}





%证明非常简单,只需要利用多项式环和分式域的唯一性即可。多项式环的唯一性依赖于\autoref{the_RPlynm_2}~\upref{RPlynm},分式域的唯一性由\autoref{the_FrcFld_1}~\upref{FrcFld}得到。





\begin{theorem}{}\label{the_SpltFd_3}
设$\mathbb{K}$是多项式$f\in\mathbb{F}[x]$的分裂域,$\mathbb{E}$是$\mathbb{K}$的扩域。

则对于$\mathbb{E}$的任意保$\mathbb{F}$\textbf{自同态}$\sigma$,有$\sigma(\mathbb{K})=\mathbb{K}$。
\end{theorem}

\textbf{证明}:

设$f=(x-a_1)(x-a_2)\cdots(x-a_n)$,其中各$a_i\in\mathbb{E}$,则$\mathbb{K}=\mathbb{F}(a_1, a_2, \cdots, a_n)$.

由于是同态,$\sigma$必将$f$的根映射为另一根,也就是对$f$的根的置换。因此
\begin{equation}
\begin{aligned}
\sigma(\mathbb{K})&=\mathbb{F}(\sigma(a_1), \sigma(a_2), \cdots, \sigma(a_n))\\
&=\mathbb{F}(a_1, a_2, \cdots, a_n)\\
&=\mathbb{K}~,
\end{aligned}
\end{equation}

\textbf{证毕}。




\subsection{正规扩张}

分裂域的性质,其实对应的是一种非常重要的域扩张,它与代数方程的根式解问题息息相关。

\begin{definition}{正规扩张}
设$\mathbb{K}/\mathbb{F}$是一个\textbf{代数}扩域。如果对于$\mathbb{F}$上的任意不可约多项式$f$,要么$f$在$\mathbb{K}$中无根,要么就所有根都在$\mathbb{K}$中,则称$\mathbb{K}/\mathbb{F}$是一个\textbf{正规扩张}。
\end{definition}

实际上,\textbf{有限情况下}正规扩张和分裂域是等价的概念,尽管它们表述差异很大。或者换句话说,分裂域的一个重要性质,就是正规性。


\begin{theorem}{有限扩张时,正规扩张等价于分裂域}\label{the_SpltFd_2}
设$\mathbb{K}/\mathbb{F}$是一个\textbf{有限}扩域,那么有:

$\mathbb{K}/\mathbb{F}$为正规扩张$\iff$ $\mathbb{K}$是某个多项式$f\in\mathbb{F}[x]$的分裂域。
\end{theorem}

\textbf{证明}:

$\Leftarrow$:

设$\mathbb{K}$是多项式$f\in\mathbb{F}[x]$的分裂域。取不可约的$h(x)\in\mathbb{F}[x]$且$\exists a\in\mathbb{K}$使得$h(a)=0$。我们要证明$h$的根都在$\mathbb{K}$中。


设$h\in\mathbb{K}[x]$的分裂域为$\mathbb{E}$,$\sigma:\mathbb{E}\to\mathbb{E}$是域自同构。则据\autoref{the_SpltFd_3} ,$\sigma(\mathbb{K})=\mathbb{K}$。于是,$\sigma(a)\in\mathbb{K}$。

由$\sigma$的任意性(即任意一个$\mathbb{E}$自同构,也即任意一个$h$的根的置换),知$h$的根都在$\mathbb{K}$中。



$\Rightarrow$:

由于$\mathbb{K}/\mathbb{F}$为有限扩张,故存在$a_1, a_2, \cdots, a_n\in \mathbb{K}$,使得$\mathbb{K}=\mathbb{F}(a_1, a_2, \cdots, a_n)$。

设$a_i$在$\mathbb{F}$上的最小多项式为$f_i(x)$,令$f(x)=\prod_{i=1}^n f_i(x)$。

由于$\mathbb{K}/\mathbb{F}$为正规扩张,而各$f_i$在$\mathbb{K}$上至少有一个根,故$f(x)$可以在$\mathbb{K}$上写为一次多项式的乘积:
\begin{equation}
f(x) = \prod_{i=1}^k (x-b_i)~,
\end{equation}
且各$a_i\in\{b_j\}$,各$b_i\in\mathbb{K}=\mathbb{F}(a_1, \cdots, a_n)$。

于是$f\in\mathbb{F}$的分裂域为$\mathbb{F}(b_1, \cdots, b_k)=\mathbb{F}(a_1, \cdots, a_n)=\mathbb{K}$。

\textbf{证毕}。


% \begin{corollary}{}\label{cor_SpltFd_2}
% 设$\opn{ch}\mathbb{F}=p\neq 0$,如果$f$是$\mathbb{F}$上的不可约多项式,$a$是$f$的任意一个根,那么$f\in\mathbb{F}[x]$的分裂域就是$\mathbb{F}(a)$。
% \end{corollary}

% \textbf{证明}:

% 由正规扩张的定义以及“有限正规扩张就是分裂域”,显然$f$有一个根$a$在$\mathbb{K}$中,那么$f$的所有根就必须在$\mathbb{K}$中。

% \textbf{证毕}。



\begin{example}{正规扩张的反例}
$\mathbb{Q}(2^{1/3})$不是$\mathbb{Q}$的正规扩张。因为存在多项式$f(x)=x^3-2$,它在$\mathbb{Q}$上不可约,有一个根$2^{1/3}$在$\mathbb{Q}(2^{1/3})$上,但另外两个根都是复数,不在其中。

显然,另外两个根的模都是$2^{1/3}$,与正实轴的夹角分别为$\pm 2\pi/3$。
\end{example}

\begin{exercise}{}
求$x^3-2\in\mathbb{Q}[x]$的分裂域。
\end{exercise}



\begin{corollary}{}
设有域扩张$\mathbb{E}/\mathbb{F}$。则任意$f\in\mathbb{F}[x]$的分裂域$\mathbb{K}$,在$\mathbb{E}$中最多只有一个。
\end{corollary}









\subsection{分裂域的自同构数目}

由\autoref{the_FldExp_4}~\upref{FldExp}第2条,可知,域自同构一定把每个多项式的根映射到其它根上,并且对于任意两个根$\alpha, \beta$,总存在域自同构$\sigma$使得$\sigma \alpha=\beta$。

因此,如果域$\mathbb{F}$上有一个多项式$f(x)g(x)$,则$fg$关于$\mathbb{F}$的分裂域,是先求$f\in\mathbb{F}[x]$的分裂域$\mathbb{F}_1$后再求$g\in\mathbb{F}_1$的分裂域。


\begin{theorem}{}\label{the_SpltFd_1}
给定域$\mathbb{F}$和其上一个多项式$f$,设$f\in\mathbb{F}[x]$的分裂域是$\mathbb{K}$,$\mathbb{K}$到自身的保$\mathbb{F}$自同构数量为$N$,那么$N\leq[\mathbb{K}:\mathbb{F}]$。

当且仅当$f$的每一个不可约因子$h$的不同根数目恰为$\opn{deg}h$时\footnote{即无重根时。},等号成立。
\end{theorem}

\textbf{证明}:

我们主要用数学归纳法和对单扩张情况的讨论来证明。

不妨设$f$的各不可约因子互不相同。设$\sigma:\mathbb{K}\to\mathbb{K}$是保$\mathbb{F}$自同构,$h$是$f$在$\mathbb{F}$上的一个不可约因子,其在$\overline{\mathbb{F}}$上的全体根为$\{\alpha_i\}_{i=1}^m$,$\opn{deg}h=n\geq m$。记$h\in\mathbb{F}[x]$的分裂域为$\mathbb{F}_1$。

考虑单扩张$\mathbb{F}(\alpha_1)$的保$\mathbb{F}$自同构。由于$\alpha_1$可以在这种自同构下映射到$\mathbb{F}(\alpha_1)$\textbf{中的}任意$\alpha_i$上,并且确定了$\alpha_1$的映射就确定了整个自同构映射,故这种自同构的数量等于$\mathbb{F}(\alpha_1)$中所包含的$h$的根的数量。

因此,如果$\mathbb{F}(\alpha_1)$包含所有$\alpha_i$,则自同构数量等于所有根的数量,定理成立(包括等号的充要条件):由\autoref{the_FldExp_1}~\upref{FldExp}和“多项式不同根的数目小于等于其次数”即可。

如果$\mathbb{F}(\alpha_1)$不包含所有$\alpha_i$,那就要用上归纳法了。

当$\opn{deg}f=1$时,定理显然成立。下设定理对于任意$\opn{deg}f<n$的情况成立。

先考虑$h$无重根的情况,设$\mathbb{F}_1$是$h\in\mathbb{F}[x]$的分裂域。

此时,$h$的根的数目$m=\opn{deg}f=n$。由\autoref{the_FldExp_1}~\upref{FldExp},$[\mathbb{F}(\alpha_1):
\mathbb{F}]=n$。只要确定了$\alpha_1$的映射便确定了$\mathbb{F}(\alpha_1)$中其它$\alpha_i$的映射。

1. 给定$\mathbb{F}_1$作为$\mathbb{F}$上线性空间的一组基$\{\nu_i\}\subseteq\{\alpha_j^k\}$,并\textbf{任挑一个}$\alpha_i$来构造保$\mathbb{F}$单同态$\sigma: \mathbb{F}(\alpha_1)\to\mathbb{F}_1$,其中$\sigma(\alpha_1)=\alpha_i$。则根据\autoref{the_FldExp_5}~\upref{FldExp}的证明过程,可知总能\textbf{唯一地}把$\sigma$开拓为与$\{\nu_i\}$关联的保$\mathbb{F}$自同构$\sigma: \mathbb{F}_1\to\mathbb{F}_1$。于是,我们得到了$n$个保$\mathbb{F}$自同构。

2. 但这些自同构不是全部,因为只是针对一组基构造出来的。$\mathbb{F}_1$上所有的保$\mathbb{F}$自同构,应该是上段构造的自同构和全体保$\mathbb{F}(\alpha_1)$复合的结果\footnote{这是因为$\mathbb{F}_1$的保$\mathbb{F}$自同构只有两种类型,保$\mathbb{F}(\alpha_1)$和不保$\mathbb{F}(\alpha_1)$的,而$\sigma(\mathbb{F}(\alpha_1))$只取决于$\sigma(\alpha_1)$,所以总可以用一个不保$\mathbb{F}(\alpha_1)$的自同构把$\sigma$复合成保$\mathbb{F}(\alpha_1)$的。}。

3. 据归纳假设,$\mathbb{F}_1$上的保$\mathbb{F}(\alpha_1)$自同构的数量,等于扩张次数$[\mathbb{F}_1:\mathbb{F}(\alpha_1)]$。而$\mathbb{F}(\alpha_1)$的保$\mathbb{F}$自同构又有$n$个,也等于扩张次数$[\mathbb{F}(\alpha_1):\mathbb{F}]$。所以由\autoref{the_FldExp_3}~\upref{FldExp},$[\mathbb{F}_1:\mathbb{F}]=[\mathbb{F}_1:\mathbb{F}(\alpha_1)][\mathbb{F}(\alpha_1):\mathbb{F}]$,进而知定理的等号情况成立。

对于$h$有重根的情况,第1. 步和第2. 步中至少有一步不能取等号,进而定理的非等号情况成立。

对于整个$f$的情况,则在讨论完$\mathbb{F}_1$后,取$f$在$\mathbb{F}_1$上的不可约因子继续讨论,得到其分裂域$\mathbb{F}_2$,再取$f$在$\mathbb{F}_2$上的不可约因子继续讨论,直到将$f$完全分裂。由于不同的多项式的根之间不可能互相映射到,因此计算同构数量的时候可以简单相乘,而不必像1. 和2. 那样讨论。


% 记$h$的根为$\alpha_1, \alpha_2, \cdots, \alpha_r$,其分裂域为$\mathbb{F}_1=\mathbb{F}(\alpha_1,\cdots,\alpha_r)$,且$\opn{deg}h=m$,则向量组$\{1, \alpha_1, \alpha_1^2, \cdots, \alpha_1^{m-1}, \alpha_2, \cdots, \alpha_r^{m-1}\}$张成$\mathbb{F}$上的线性空间$\mathbb{F}_1$\footnote{参见\autoref{the_FldExp_1}~\upref{FldExp}的证明。}。





% \addTODO{没写完证明。可能需要韦达定理}
% 即,确定了 不可约 多项式的其中一个根映射到哪个根以后,其它根的映射是不是也确定了?注意一定要是不可约,这样所有根的最小多项式就都是这个不可约多项式。
% update:写完了,确实用了韦达定理

% 显然,如果$h\in\mathbb{F}[x]$不可约,那它的根都以它为最小多项式。由于最小多项式决定了元素的所有代数性质,故同一个最小多项式的各根在代数意义上不可区分,故将$h$的根映射为另一根依然是域同构。

% 也就是说,每个$\sigma$,都对应$f$各不可约因式的根的一个置换。但不是所有的置换都对应一个$\sigma$。

% 首先,恒等映射显然可以充当$\sigma$。所以下面我们就讨论不是恒等映射的情况。

% 考虑\autoref{eq_VietaF_1}~\upref{VietaF}的第一行。如果已知$\sigma(x_1)=x_2\not = x_1$,那么应有
% \begin{equation}\label{eq_SpltFd_1}
% \begin{aligned}
% \sigma(x_2) &= \sigma\qty(-\qty(\frac{a_{n-1}}{a_n}+x_1+x_3+x_4+\cdots+x_n))\\
% &= -\qty(\frac{a_{n-1}}{a_n}+\sigma(x_1)+\sigma(x_3)+\sigma(x_4)+\cdots+\sigma(x_n))\\
% &= -\qty(\frac{a_{n-1}}{a_n}+x_2+\sigma(x_3)+\sigma(x_4)+\cdots+\sigma(x_n))\\
% \end{aligned}
% \end{equation}

% 注意\autoref{eq_SpltFd_1} 最后一行出现了$x_2$项,意味着$\sigma(x_2)\not = x_2$。以此类推,非恒等映射的$\sigma$总是把一个根映射为另一个根。因此,任取两个不同的保$\mathbb{F}$自同构$\sigma_1$、$\sigma_2$,不可能出现$\sigma_1(x_1)=\sigma_2(x_1)$的情况——把$x_1$换成任意其它根也一样。

% 这意味着只要确定了$\sigma(x_1)$,也就确定了所有$\sigma(x_i)$。加上$\sigma$是恒等映射的情况,我们发现,每个$\sigma$对应一个$\sigma(x_1)=x_k$。也就是说,仅考虑$h(x)$决定的保$\mathbb{F}$自同构,应该和$h$的根的数目一样。







% 把$f$的所有不可约因式都考虑上,则总的保$\mathbb{F}$同构数目应该是各因式的根的数目之积。

% 又由\autoref{the_FldExp_3}~\upref{FldExp},即可得证明。


\textbf{证毕}。





\begin{example}{}

给定有理数域$\mathbb{Q}$及其上的多项式$f(x)=x^2-2$。则$f\in\mathbb{Q}[x]$的分裂域为$\mathbb{Q}(\sqrt{2})=\{a+b\sqrt{2}|a, b\in\mathbb{Q}\}$。

$\mathbb{Q}(\sqrt{2})$一共有两个保$\mathbb{Q}$自同构:第一个就是恒等映射,第二个$\sigma$则定义如下:
\begin{equation}
\sigma(a+b\sqrt{2})=a-b\sqrt{2}~.
\end{equation}
也就是说,$\sigma$把根$\pm\sqrt{2}$映射到根$\mp\sqrt{2}$。


而$[\mathbb{Q}(\sqrt{2}):\mathbb{Q}]=2$,因此这是一个\autoref{the_SpltFd_1} 取等号的例子。

\end{example}


\begin{example}{}
给定实数域$\mathbb{R}$及其上的多项式$f(x)=x^2+x+1$,则$f\in\mathbb{R}[x]$的分裂域为$\mathbb{R}(\omega)=\{a+b\omega+c\omega^2 \mid a, b, c\in\mathbb{R}\}$,其中$\omega$是$1$的三次单位根$\frac{1}{2}(-1+\I\sqrt{3})=\E^{\frac{2\pi}{3}\I}$。

$\mathbb{R}(\omega)$一共有两个保$\mathbb{R}$自同构:第一个是恒等映射;第二个则是将$\omega$映射到$\omega^2$、$\omega^2$映射到$\omega$的映射。

$[\mathbb{R}(\omega):\mathbb{R}]=2$。这可以从\autoref{the_FldExp_1}~\upref{FldExp}得到,也可以验证$\omega^2=-1-\omega$得到。
\end{example}



\begin{exercise}{}
给定实数域$\mathbb{R}$及其上的多项式$f(x)=x^4+x^3+x^2+x+1$,则$f\in\mathbb{R}[x]$的分裂域为$\mathbb{R}(\gamma)=\{a+b\gamma+c\gamma^2+d\gamma^3+e\gamma^4 \mid a, b, c, d, e\in\mathbb{R}\}$,其中$\gamma$是$1$的五次单位根$\E^{\frac{2\pi}{5}\I}$。

$\mathbb{R}(\gamma)$一共有四个保$\mathbb{R}$自同构。找出它们。

提示:考虑$\gamma$被映射到某根$\gamma'$,那么$\gamma^2$就被映射到$\gamma'^2$。所以每个自同构唯一对应一个根$\gamma'$。
\end{exercise}



\begin{exercise}{}\label{exe_SpltFd_1}
求$x^3-2\in\sqrt{Q}[x]$的分裂域及其所有保$\mathbb{Q}$自同构。
\end{exercise}



\begin{exercise}{}
求$x^p-1\in\mathbb{Q}[x]$的分裂域及其所有保$\mathbb{Q}$自同构。这里$p$为素数。注意判断$x^p-1$是否为不可约多项式。
\end{exercise}


\begin{exercise}{}
求$x^6-1\in\mathbb{Q}[x]$的分裂域及其所有保$\mathbb{Q}$自同构。
\end{exercise}














