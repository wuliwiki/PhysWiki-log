% 有限深不对称方势阱

\begin{issues}
\issueTODO
\issueOther{需要讨论散射态, 即 $E > V_1$ 的两种情况}
\end{issues}

\pentry{有限深方势阱\upref{FSW}}
\begin{figure}[ht]
\centering
\includegraphics[width=8cm]{./figures/afeddbbc3be88731.pdf}
\caption{一维不对称势阱} \label{fig_AMW_1}
\end{figure}
本节我们来解下面的一维不对称势阱的离散谱(\autoref{fig_AMW_1})。
\begin{equation}
V(x)=\left\{\begin{aligned}
&V_1\quad(x<0)\\
&0\quad (0<x<L)\\
&V_2\quad(x>L)
\end{aligned}\right.
,\qquad (V_1 < V_2)
\end{equation}
对于离散谱,能量 $E$ 需小于无穷远处势能,故 $E<V_1$. 

在 $x<0$ 区域内的其薛定谔方程为
\begin{equation}
\frac{\hbar^2}{2m}\dv[2]{\psi(x)}{x}+(E-V_1)\psi(x)=0~.
\end{equation}
满足边界条件 $\psi(-\infty)\to 0$ 的波函数为
\begin{equation}
\psi(x)=C_1 e^{\kappa_1 x},\quad \kappa_1=\frac{1}{\hbar}\sqrt{2m(V_1-E)}~.
\end{equation}

在 $x>L$ 区域内的其薛定谔方程为
\begin{equation}
\frac{\hbar^2}{2m}\dv[2]{\psi(x)}{x}+(E-V_2)\psi(x)=0~.
\end{equation}
满足边界条件 $\psi(+\infty)\to 0$ 的波函数为
\begin{equation}
\psi=C_2 e^{-\kappa_2 x},\quad \kappa_2=\frac{1}{\hbar}\sqrt{2m(V_2-E)}~.
\end{equation}

在阱内 $0 < x < L$ ,薛定谔方程为
\begin{equation}
\frac{\hbar^2}{2m}\dv[2]{\psi(x)}{x}+E\psi(x)=0~.
\end{equation}
$\psi$ 可取以下形式:
\begin{equation}
\psi=C\sin(kx+\delta),\quad k=\frac{\sqrt{2mE}}{\hbar}~.
\end{equation}

根据势阱边上 $\psi'/\psi$ 的连续性条件,得
\begin{equation}
k\cot\delta=\kappa_1=\sqrt{\frac{2m}{\hbar^2}V_1-k^2},\quad k\cot(Lk+\delta)=-\kappa_2=-\sqrt{\frac{2m}{\hbar^2}V_2-k^2}~,
\end{equation}
或
\begin{equation}
\sin\delta=\frac{k\hbar}{\sqrt{2mV_1}},\quad\sin(kL+\delta)=-\frac{k\hbar}{\sqrt{2mV_2}}~.
\end{equation}
消去 $\delta$ 后,得下列超越方程:
\begin{equation}\label{eq_AMW_1}
kL=n\pi-\arcsin\frac{k\hbar}{\sqrt{2mV_1}}-\arcsin\frac{k\hbar}{\sqrt{2mV_2}}~.
\end{equation}
其中,$k=1,2,3,\cdots$,反正弦函数取值介于 $0$ 到 $\pi/2$ 之间。上式之根确定了能级 $E=k^2\hbar^2/2m$。对每一个 $n$ 一般来讲只有一个根;$n$ 值按能级的递增次序编号。其左右两边的图像大致如\autoref{fig_AMW_2} .
%\addTODO{画出\autoref{eq_AMW_1} 两边的函数图}
\begin{figure}[ht]
\centering
\includegraphics[width=9cm]{./figures/47803e6796abaa03.png}
\caption{\autoref{eq_FSW_3} 左右两式对应的函数图像(上层浅红色代表右式,下层浅橘色代表左式),其中 $\frac{\hbar}{\sqrt{2mV_1}}=0.5,\frac{\hbar}{\sqrt{2mV_2}}=0.1,n=1$} \label{fig_AMW_2}
\end{figure}
由于反正弦函数宗量不能超过1,$k$ 值显然只能介于 $0$ 到 $\sqrt{2mV_1}/\hbar$ 之间。\autoref{eq_AMW_1} 左边随 $k$ 单调增加,右边却随 $k$ 单调减小。因此\autoref{eq_AMW_1} 有根的必要条件为 $k=\sqrt{2mV_1}/\hbar$ 时,\autoref{eq_AMW_1} 右边小于左边。特别是,由 $n=1$ 所得的下列不等式
\begin{equation}\label{eq_AMW_2}
L\frac{\sqrt{2mV_1}}{\hbar}\geq\frac{\pi}{2}-\arcsin\sqrt{\frac{V_1}{V_2}}~.
\end{equation}
给出了阱中至少存在一个能级的条件。由此可见,给定了不相等的 $V_1$ 和 $V_2$ 后,总可以找到一个很窄的阱宽 $L$,使得该阱中不能存在离散能级,\autoref{fig_AMW_2} 中可看出在 $0<L<0.5$ 处\autoref{eq_FSW_3} 无解。对 $V_1=V_2$ 而言,\autoref{eq_AMW_2} 总能得到满足。