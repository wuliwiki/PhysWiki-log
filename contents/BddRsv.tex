% 有界算子的预解式
% license Xiao
% type Tutor

\pentry{有界算子的谱\nref{nod_BddSpe}}{nod_246c}

\begin{definition}{预解式}
设 $X$ 是复巴拿赫空间, $T:X\to X$ 是有界线性算子。 定义 $T$ 的\textbf{预解式(resolvent)} 为复变量 $z\in\mathbb{C}$ 的算子值函数
$$
R(z;T):=(z-T)^{-1}~,
$$
如果 $z$ 使得上式有意义(即 $z$ 属于 $T$ 的预解集)。
\end{definition}

然而我们首先要确认这个式子确实对某些 $z$ 有意义。 引入下列定义:

\begin{definition}{冯诺依曼级数}
设 $X$ 是巴拿赫空间, $A:X\to X$ 是有界算子。 算子 $A$ 的几何级数
$$
\text{Id}+A+A^2+\dots~
$$
称为\textbf{冯诺依曼级数(von Neumann series)}。
\end{definition}
不难看出在 $\|A\|<1$ 时, 它按照算子范数收敛到 $(\text{Id}-A)^{-1}$, 这与通常的数值几何级数很类似。 

借助冯诺依曼级数, 立刻看出当 $|z|>\|T\|$ 时成立
$$
(z-T)^{-1}
=\frac{1}{z}(1-z^{-1}T)^{-1}
=\frac{1}{z}\sum_{n=0}\frac{T^n}{z^n}~,
$$
所以可见 $R(z;T)$ 对于充分大的 $z$ 确实是全纯函数。

据此便可以证明如下基本命题:

\begin{theorem}{}
\begin{enumerate}
\item 如果 $T:X\to X$ 是有界线性算子, 那么预解集 $\rho(T)$ 是开集, 而谱集 $\sigma(T)$ 是紧集。
\item 如果 $T:X\to X$ 是有界线性算子, 那么它至少有一个谱点, 也就是说 $\sigma(T)$ 是非空紧集。
\end{enumerate}
\end{theorem}
\textbf{证明} 
对于 1., 如果 $\lambda_0\in\rho(T)$, 那么 $A:=\lambda_0-T$ 是有界的可逆算子。 于是可作冯诺依曼级数
$$
A^{-1}(\text{Id}+zA^{-1}+z^2A^{-2}+\dots+z^nA^{-n}+\dots)~.
$$
如果 $|z|<\|A^{-1}\|^{-1}$, 那么这个级数收敛到 $A^{-1}(\text{Id}-zA^{-1})^{-1}=(\lambda_0-z-T)^{-1}$。 这说明预解集的点的某个邻域还包含在预解集内, 从而预解集是开集, 而谱集是闭集。 另一方面, 我们已经知道, 只要 $|z|>\|T\|$, $(z-T)^{-1}$ 便是全纯函数, 于是谱集包含在圆 $|z|\leq\|T\|$ 内。 所以谱集是紧集。

对于 2., 如果 $(z-T)^{-1}$ 对于所有复数 $z$ 都存在, 那么根据上面的级数展开, 可以看出它是全复平面上的算子值全纯函数, 而且由于 $|z|>2\|T\|$ 时有
$$
(z-T)^{-1}=\frac{1}{z}\sum_{n=0}\frac{T^n}{z^n}~,
$$
而右边级数的模显然小于2, 所以 $f(z)$ 是有界的全纯函数。 按照刘维尔定理, 它只能是常值函数, 这当然不可能。 \textbf{Q.E.D.}

从证明中可见 $R(z;T)$ 是定义在 $\rho(T)$ 上的算子值全纯函数。
