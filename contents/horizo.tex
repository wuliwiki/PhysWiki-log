% 视界
% license Usr
% type Tutor

\begin{issues}
\issueTODO
部分词条需要引用该词条。
\end{issues}

\subsection{可观测宇宙}
宇宙伊始至今,光在这漫漫长途中走过的共动距离$r_0$符合方程:
\begin{equation}
\int_{0}^{r_{0}} \frac{d r}{\sqrt{1-k (r/R)^{2}}}=\int_{0}^{t_{0}} \frac{c d t}{a(t)}~.
\end{equation}
在把尺度因子归一化后,物理距离便等同于共动距离。这是我们原则上能探测的宇宙大小,是实际宇宙的一部分,故称为\textbf{可观测宇宙(observarable universe)}。因为光速最快,所以大于该距离的粒子并不能影响到此时的我们,因此这个距离又被叫作\textbf{粒子视界(particle horizon)}。不过实际上,光子在退耦前与其他带电粒子作用频繁,整个宇宙都是不透光的,我们能测得的宇宙大小要偏小一些。

举个例子,假设我们生活在$k=\Lambda=0$,物质主宰的宇宙,代入$a=(t/t_0)^{2/3}$,可算得:

\begin{equation}\int_0^{r_0}dr=ct_0^{2/3}\int_0^{t_0}\frac{dt}{t^{2/3}}\quad\Longrightarrow\quad r_0=3ct_0 ~,\end{equation}
可见宇宙的膨胀效应使得光走过的距离大于光速乘以宇宙年龄。

经过合理拓展,在$t$时刻,可观测宇宙(粒子视界)的大小r符合方程
\begin{equation}
\int_{0}^{r} \frac{d r}{\sqrt{1-k (r/R)^{2}}}=a(t)\int_{0}^{t} \frac{c \dd t'}{a'(t')}~.
\end{equation}

\subsection{宇宙事件视界}
与粒子视界不同,\textbf{宇宙事件视界(cosmological event horizon)}指的是从某个时刻开始,光所能走过的最大共动距离。有的早期宇宙模型预测宇宙将在未来某个时间点开始塌缩。设该时刻表示为$t_{BC}$(Big Crunch),在这种情况下,光走过的共动路程为
\begin{equation}
d_H=ca_{BC}\int^{t_{BC}}_{t}\frac{\dd t'}{a(t')}~.
\end{equation}