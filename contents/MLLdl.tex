% 莫雷拉定理(综述)
% license CCBYSA3
% type Wiki

本文根据 CC-BY-SA 协议转载翻译自维基百科\href{https://en.wikipedia.org/wiki/Morera\%27s_theorem}{相关文章}。

在复分析中,莫雷拉定理(Morera's theorem,以贾钦托·莫雷拉 Giacinto Morera 命名)给出了一个判断函数是否为全纯函数的判据。设 $f$ 是定义在复平面开集 $D$ 上的连续复值函数,如果它对 $D$ 中的每一条分段 $C^1$ 的闭曲线 $\gamma$ 都满足
$$
\oint_{\gamma} f(z)\,dz = 0,~
$$
那么 $f$ 必定在 $D$ 上是全纯函数。这个条件等价于 $f$ 在 $D$ 上存在一个原函数(反导函数)。不过该定理的逆命题一般并不成立:一个全纯函数并不一定在其定义域上都有原函数,除非附加额外条件。如果定义域 $D$ 是单连通的,那么逆命题成立,这正是柯西积分定理的内容,即全纯函数沿闭合曲线的积分为零。

一个典型的反例是函数 $f(z) = 1/z$,它在 $\mathbf{C} - \{0\}$ 上是全纯的。在 $\mathbf{C} - \{0\}$ 中任意一个单连通邻域 $U$ 上,$1/z$ 都有一个原函数,可以写为$L(z) = \ln(r) + i\theta, \quad z = re^{i\theta}$.
由于 $\theta$ 的取值可以相差任意整数倍的 $2\pi$,所以在区域 $U$ 中,只要能连续地选取 $\theta$ 的分支,就可以在该 $U$ 上定义出 $1/z$ 的一个原函数。(无法在包含原点的闭合曲线上连续地定义 $\theta$,正是 $1/z$ 在整个定义域 $\mathbf{C} - \{0\}$ 上没有原函数的根本原因。)此外,由于常数的导数为 0,原函数中加上任意常数仍然是 $1/z$ 的一个原函数。

从某种意义上说,$1/z$ 这个反例具有普遍性:对于任何在其定义域上没有原函数的解析函数,其根本原因都可以追溯到 $1/z$ 在 $\mathbf{C} - \{0\}$ 上没有原函数这一事实。
\subsection{证明}
\begin{figure}[ht]
\centering
\includegraphics[width=6cm]{./figures/3611ee82bcb8dce2.png}
\caption{如果沿任意闭合曲线 $C$ 的积分都为零,那么函数 $f$ 在区域 $D$ 上是全纯的。} \label{fig_MLLdl_1}
\end{figure}
\begin{figure}[ht]
\centering
\includegraphics[width=6cm]{./figures/d72f0fbe5191af2c.png}
\caption{从点 $a$ 到点 $b$ 的两条路径上的积分是相等的,因为它们的差值等于沿一条闭合回路的积分。} \label{fig_MLLdl_2}
\end{figure}
该定理有一个相对初等的证明方法,即通过显式构造一个 $f$ 的原函数来完成。

不失一般性,假设区域 $D$ 是连通的。选定 $D$ 中的一个点 $z_0$,对任意 $z \in D$,令 $\gamma: [0,1] \to D$ 是一条分段 $C^1$ 的曲线,满足$\gamma(0) = z_0, \quad \gamma(1) = z$.定义函数
$$
F(z) = \int_{\gamma} f(\zeta)\,d\zeta~
$$
为了说明这个定义是良好的,假设有另一条分段 $C^1$ 曲线 $\tau: [0,1] \to D$,同样满足$\tau(0) = z_0, \quad \tau(1) = z$.考虑曲线 $\gamma \tau^{-1}$(即先沿着 $\gamma$ 然后反向沿着 $\tau$ 的组合曲线),它是一条闭合的分段 $C^1$ 曲线。于是有
$$
\int_{\gamma} f(\zeta)\,d\zeta + \int_{\tau^{-1}} f(\zeta)\,d\zeta
= \oint_{\gamma \tau^{-1}} f(\zeta)\,d\zeta
= 0~
$$
由此可得
$$
\int_{\gamma} f(\zeta)\,d\zeta = \int_{\tau} f(\zeta)\,d\zeta~
$$
证明了 $F(z)$ 的定义与所选路径无关。利用 $f$ 的连续性估计差商,可以得出$
F'(z) = f(z)$.如果选择不同的 $z_0$,函数 $F$ 只会相差一个常数,即在新旧 $z_0$ 之间沿任意分段光滑曲线积分 $f$ 的结果,但这不会改变导数。

由于 $f$ 是全纯函数 $F$ 的导数,因此 $f$ 也是全纯的。全纯函数的导数仍然全纯,可以通过以下事实来证明:全纯函数是解析的,即可以表示为收敛幂级数,而幂级数可以逐项求导。这就完成了证明。
\subsection{应用}
莫雷拉定理是复分析中的标准工具,几乎所有涉及以非代数方式构造全纯函数的论证中都会用到它。
\subsubsection{一致收敛}
例如,假设 $f_1, f_2, \dots$ 是一列全纯函数,它们在某个开圆盘上一致收敛于一个连续函数 $f$。根据柯西定理,对于圆盘内的任意闭合曲线 $C$,都有
$$
\oint_{C} f_n(z)\,dz = 0~
$$
对每个 $n$ 都成立。由于一致收敛,可以得到
$$
\oint_{C} f(z)\,dz
= \oint_{C} \lim_{n \to \infty} f_n(z)\,dz
= \lim_{n \to \infty} \oint_{C} f_n(z)\,dz
= 0~
$$
对任意闭合曲线 $C$ 都成立。由莫雷拉定理可知,$f$ 必然是全纯的。这个结论还可以用来证明,对于任意开集 $\Omega \subseteq \mathbf{C}$,所有有界解析函数 $u: \Omega \to \mathbf{C}$ 组成的集合 $A(\Omega)$,在取上确界范数的意义下是一个巴拿赫空间。
\subsubsection{无穷和与积分}
莫雷拉定理还可以结合Fubini 定理和Weierstrass M-检验,用于证明由无穷和或积分定义的函数是解析的,例如黎曼$\zeta$函数
$$
\zeta(s) = \sum_{n=1}^{\infty} \frac{1}{n^s}~
$$
或Gamma 函数
$$
\Gamma(\alpha) = \int_0^{\infty} x^{\alpha - 1} e^{-x} \, dx~
$$
具体来说,可以通过写成
$$
\oint_C \Gamma(\alpha) \, d\alpha
= \oint_C \int_0^{\infty} x^{\alpha - 1} e^{-x} \, dx \, d\alpha~
$$
来处理合适闭曲线 $C$ 上的积分,然后利用Fubini 定理交换积分次序,得到
$$
\int_0^{\infty} \oint_C x^{\alpha - 1} e^{-x} \, d\alpha \, dx
= \int_0^{\infty} e^{-x} \oint_C x^{\alpha - 1} \, d\alpha \, dx~
$$
由于映射 $\alpha \mapsto x^{\alpha - 1}$ 是解析的,有
$$
\oint_C x^{\alpha - 1} \, d\alpha = 0~
$$
因此上式的双重积分为 0。

类似地,在处理$\zeta$函数的情形下,利用M-检验可以证明在闭曲线上的积分与无穷求和可以互换,从而证明$\zeta$函数在适当区域内的解析性。
\subsection{假设的弱化}
莫雷拉定理中的假设条件可以大幅放宽。特别是,只要区域 $D$ 中的任意一个闭合(实心)三角形 $T$ 都满足
$$
\oint_{\partial T} f(z)\,dz = 0~
$$
那么这个条件就足以刻画全纯性:即 $f$ 在 $D$ 上全纯,当且仅当上述条件成立。这一结果还带来了前述“全纯函数一致收敛极限仍然全纯”结论的推广:如果 $f_1, f_2, \dots$ 是定义在开集 $\Omega \subseteq \mathbb{C}$ 上的一列全纯函数,并且在 $\Omega$ 的每一个紧子集上一致收敛于某个函数 $f$,那么 $f$ 也是全纯的。
\subsection{参见}
\begin{itemize}
\item 柯西–黎曼方程
\item 路径积分方法
\item 留数(复分析)
\item 米塔格–莱夫勒定理
\end{itemize}
\subsection{参考文献}
\begin{itemize}
\item Ahlfors, Lars (1979), Complex Analysis, International Series in Pure and Applied Mathematics, McGraw-Hill, ISBN 978-0-07-000657-7, Zbl 0395.30001.
\item Conway, John B. (1973), Functions of One Complex Variable I, Graduate Texts in Mathematics, vol. 11, Springer Verlag, ISBN 978-3-540-90328-4, Zbl 0277.30001.
\item Greene, Robert E.; Krantz, Steven G. (2006), Function Theory of One Complex Variable, Graduate Studies in Mathematics, vol. 40, American Mathematical Society, ISBN 0-8218-3962-4.
\item Morera, Giacinto (1886), "Un teorema fondamentale nella teorica delle funzioni di una variabile complessa", Rendiconti del Reale Instituto Lombardo di Scienze e Lettere (意大利语), 19 (2): 304–307, JFM 18.0338.02.
\item Rudin, Walter (1987) [1966], Real and Complex Analysis (3rd ed.), McGraw-Hill, pp. xiv+416, ISBN 978-0-07-054234-1, Zbl 0925.00005.
\end{itemize}
\subsection{外部链接}
\begin{itemize}
\item “Morera theorem”,Encyclopedia of Mathematics, EMS Press, 2001 [1994]
\item Weisstein, Eric W. “Morera's Theorem”, MathWorld.
\end{itemize}