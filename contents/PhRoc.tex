% 光子火箭

\begin{issues}
\issueDraft
\end{issues}

一个光子的能量
\begin{equation}
\hbar\omega = mc^2
\end{equation}
一个光子的动量
\begin{equation}
p = mc = \frac{\hbar\omega}{c}
\end{equation}
功率和推力的关系
\begin{equation}
F = p \dv{N}{t} = \frac{\hbar\omega}{c}\dv{N}{t} = \frac{P}{c}
\end{equation}
所以 $1N$ 的推力需要 $3\times 10^8\Si{W}$ 的功率! 这与光子的能量无关.

如果使用光子火箭, 用反物质湮灭产生的光子推动, 那效率就还可以.
\begin{equation}
F = \frac{P}{c} = c\dv{m}{t}
\end{equation}
产生 100 吨的推力只需要每秒消耗 3.3 克反物质, 而功率却是惊人的 $3\times 10^{14}\Si{W}$, 可以每秒钟汽化约 $10^5$ 立方米的水(从常温常压到水蒸气).

\subsection{一般情况}
若使用经典力学, 用于推进的介质喷出速度越大, 同样的推力功率就越大.

动能
\begin{equation}
P = \frac{v^2}{2}\dv{m}{t}
\end{equation}
推力
\begin{equation}
F = v\dv{m}{t} = \frac{2}{v} P
\end{equation}

所以反物质火箭哪怕使用一点点介质, 都会大大增加推力.
