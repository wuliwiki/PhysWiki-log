% 磁场中闭合电流的合力
% keys 安培力|磁场|回路|曲线积分
% license Xiao
% type Tutor

% 未完成: 感觉这个推导太复杂了, 如果用匀强磁场 
\pentry{安培力\upref{FAmp}, 斯托克斯定理\upref{Stokes}, 静磁场的高斯定律\upref{MagGau}, 矢量算符运算法则\upref{VopEq}}

假设空间中有任意磁场 $\bvec B(\bvec r)$, 无限细的闭合电流回路 $L$ 中有电流 $I$, 则其受到的安培力可以用线积分表示为
\begin{equation}
\bvec F = -\oint I \bvec B(\bvec r)\cross\dd{\bvec r} ~.
\end{equation}
电流方向若和积分方向相同,则取正,否则去负。

若磁场是匀强磁场, 则立即得到
\begin{equation}
\bvec F = I\qty(\oint \dd{\bvec r}) \cross \bvec B = \bvec 0~.
\end{equation}
\addTODO{引用 $\oint \dd{\bvec r}=\bvec 0$ 的出处}

\subsection{证明}

下面为了书写方便, 使用 $\bvec x_i$($i=1,2,3$) 代替 $\bvec x, \bvec y, \bvec z$, 用 $F_i$($i=1,2,3$)代替 $F_x, F_y, F_z$, 以此类推。

若磁场是任意的, 那么
\begin{equation}
\ali{
\bvec F &= \oint_L I \dd{\bvec r} \cross \bvec B\\
&= \sum_i \uvec x_i I\oint_L \dd{\bvec r} \cross \bvec B  \vdot \uvec x_i\\
&= \sum_i \uvec x_i I\oint_L (\bvec B \cross \uvec x_i) \vdot \dd{\bvec r}\\
&= \sum_i \uvec x_i I \int_\Sigma  \curl (\bvec B \cross \uvec x_i) \vdot \dd{\bvec s}~.
}\end{equation}
求和中 $i=1,2,3$, 下同。 其中用到了斯托克斯定理(\autoref{eq_Stokes_1}~\upref{Stokes}), $\Sigma $ 是以闭合曲线 $L$ 为边界的曲面。上式中(\autoref{eq_VopEq_6}~\upref{VopEq})
\begin{equation}
\begin{aligned}
&\quad\curl (\bvec B \cross \uvec x_i)\\
&= \bvec B (\div \uvec x) + (\uvec x_i\vdot\bvec\nabla )\bvec B - \uvec x_i (\div \bvec B) - (\bvec B \vdot\bvec\nabla)\uvec x_i\\
&= (\uvec x\vdot\bvec\nabla)\bvec B\\
&= \pdv{\bvec B}{x_i}~,
\end{aligned}
\end{equation} 
这里用到了 $\uvec x$ 的任意微分为 0 以及 $\div \bvec B = 0$ 的性质。 对称地, 将上式中的 $\uvec x$ 替换成 $\uvec y$ 和 $\uvec z$ , 等式也成立。 所以
\begin{equation}
\bvec F = \sum_i \uvec x_i I\int_\Sigma  \pdv{\bvec B}{x_i} \vdot \dd{\bvec s}~.
\end{equation} 
写成分量的形式, 就是
\begin{equation}
F_i = I\int_\Sigma  \pdv{\bvec B}{x_i} \vdot \dd{\bvec s}~.
\end{equation}










