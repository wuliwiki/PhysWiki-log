% Python 的变量
% keys Python|变量类型|整数|浮点|字符
% license Xiao
% type Tutor

\begin{issues}
\issueDraft
\end{issues}

\subsection{Python 中的变量}
%\pentry{Python 简介\upref{Python}}{nod_58b0}
Python 中的\textbf{变量(variable)} 和数学中的不同, Python 的变量可以理解为一个储存数值的容器, 我们可以用等号把一个数值储存在一个变量中。 例如要计算一个长方体的体积, 我们既可以直接把三个数字相乘, 也可以先把这三个数字赋值给三个\textbf{变量}然后相乘
\begin{lstlisting}[language=python]
a = 1
b = 2
c = 3
volumn = a*b*c
\end{lstlisting}

要强调的是, 这里的等号并不是数学上的等于, 而是\textbf{赋值}, 即把等号右边得到的数值储存在左边的变量中。 在第 4 行执行时, 计算机会先计算等号右边表达式的结果 6, 然后将 6 储存在变量 \verb|volumn| 中。 \verb|volumn| 这个变量中并不会包含 \verb|a*b*c| 这个信息, 只储存 6 这个数值。 所以改变 \verb|a, b, c| 后 \verb|volumn| 的值并不会自动改变。

如果要让长方形的某个边长增加 1, 我们可以执行
\begin{lstlisting}[language=python]
a = a + 1
\end{lstlisting}
如果将等号理解为数学上的等于, 这个式子显然是错的。 但正确的理解是, 先把 \verb|a| 当前的值 1 加上 1 得到 2, 然后把 2 \textbf{赋值}给 \verb|a|。 由于我们没有给 \verb|volumn| 重新赋值, 它仍然是 6, 要更新 \verb|volumn|, 只需要重新执行
\begin{lstlisting}[language=python]
volumn = a*b*c
\end{lstlisting}
并用 \verb|print(volumn)| 显示新的值。

等效地, 我们也可以用\textbf{自加运算} \verb|+=|, 将 \verb|a = a + 1| 记为
\begin{lstlisting}[language=python]
a += 1
\end{lstlisting}
注意 \verb|+=| 是一个整体的算符, 中间不能有空格。 类似的运算还有 \verb|-=|, \verb|*=|, \verb|/=| 等。

\subsection{Python 基本变量类型}

\verb|bool|(布尔型), \verb|int|(整型)(长度不限), \verb|float|(浮点型)(双精度浮点), \verb|complex| (复数, 如 \verb|2+3j|), \verb|str|(字符串)。 注意 python 本身没有定义单精度浮点数, 但在 \verb|numpy| 库中有 \verb|float32| 类型。 我们可以用 \verb|type()| 函数查看某个变量的类型。 例如执行
\begin{lstlisting}[language=python]
n = 123; x = 3.14; print(type(n)); print(type(x))
\end{lstlisting}
结果为
\begin{lstlisting}[language=none]
<class 'int'>
<class 'float'>;
\end{lstlisting}
判断变量是否为某个类型
\begin{lstlisting}[language=python]
type(i) == int # true
\end{lstlisting}

用 \verb|is| 判断两个变量是否是同一个对象, 例如 \verb|a = [1,2,3]; b = [1,2,3];|, 那么 \verb|a == b| 返回 \verb|True| 而 \verb|a is b| 返回 \verb|False|。 此时改变 \verb|a| 的元素, \verb|b| 不会改变。 但若令 \verb|b = a;|, 那么 \verb|a, b| 会同时改变, 此时 \verb|a is b| 和 \verb|a == b| 都返回 \verb|True|。

\subsection{整数}
与一些编译语言不同, Python 的整数类型(integer)没有长度限制(除超出了内存大小)。 例如
\begin{lstlisting}[language=python]
print(12345678901234567890123456789 + 1)
\end{lstlisting}
的结果为 \verb|12345678901234567890123456790|。

默认情况下整数用十进制表示, 如果需要输入 \textbf{2 进制(binary)}, 可以在前面加 \verb|0b| 或 \verb|0B|, 例如 \verb|0b1001| 表示 10 进制的 \verb|9|。 类似地, \verb|0o| 或 \verb|0O| 开头表示 \textbf{8 进制(octal)}; \verb|0x| 或 \verb|0X| 开头的表示 \textbf{16 进制(hexadecimal)}, 16 进制中的 10 到 15 分别用大写或小写字母 \verb|a| 到 \verb|f| 表示。 例如 \verb|0xff| 表示十进制的 \verb|255|。 不同进制的整数同样 \verb|int|, 同样没有长度限制。

\subsection{类型转换}
转换格式为 \verb|类型(变量)|。 例如 \verb|int('123')| 会把字符串 \verb|'123'| 变为整数 \verb|123|。
\begin{lstlisting}[language=python]
"'我们在这里完整列举一下'"
a = "小时百科"  # str 字符串
b = 23 # int 整型
c = 5.2 # float 浮点型
d = True # bool 布尔型
int(c) # 返回整型 5 (向零取整)
str(b) # 返回字符串 "23"
float(d) # 返回 1.0
bool(a) # 返回 True (只有空字符串返回 False)
bool(b) # 返回 True (只有 0 返回 False)
bool(c) # 返回 True (只有 0.0 返回 False)
\end{lstlisting}

\subsection{字符串}
raw string: \verb|r'foo\nbar'| 其中 \verb|\n| 会被当成两个字符。

\subsection{变量和对象}
和 C 语言类似的语言不同, Python 中的变量和对象(object)是分开考虑的。 对象可以理解为内存上的一小段某类型的数据, 而变量是指向对象的一个名称。 例如两个变量可以指向同一个对象, 要判断变量 \verb|a| 和 \verb|b| 是否指向同一个对象, 用 \verb|a is b|, 返回一个 \verb|bool|。

\subsection{变量的范围}
\begin{itemize}
\item \verb|dir()| 列出 in scope 变量, 返回类型 \verb|list|
\item \verb|globals()| 列出全局变量, 返回类型 \verb|dict|
\item \verb|locals()| 列出本地变量, 返回类型 \verb|dict|
\end{itemize}
