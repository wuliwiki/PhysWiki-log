% 最优化
% keys 优化|最优化
% license Usr
% type Tutor

\begin{issues}
\issueDraft
\end{issues}

%占位产生链接,后面会写。

\textbf{优化(Optimization,或称最优化)},主要研究模型与算法。优化问题有三个最重要的因素:目标函数、优化变量、优化约束。注意这里的最优化只是名义上的,指的是当前给定约束下和目标下的最优,而非无条件的最好。

\begin{definition}{优化问题}

\textbf{优化问题}一般描述如下:
\begin{equation}
p^*=\min _x f(x),\text{ s.t. } x\in \chi ~.
\end{equation}
其中:
\begin{itemize}
\item $x=(x_1,x_2,\cdots,x_n)^\text{T}\in\mathbb{R}^n$ 是\textbf{决策变量}
\item $f(x):\mathbb{R}^n\rightarrow\mathbb{R}$为\textbf{目标函数}
\item $\chi\subseteq\mathbb{R}^n$ 是\textbf{约束}或\textbf{可行域}
\item $p^*$是\textbf{最优值}
\end{itemize}
\end{definition}

若 $\chi=\mathbb{R}^n$ ,则称优化问题为无约束问题,否则称为有约束问题。有约束问题的可行域一般记作$\chi=\{x\in \mathbb{R}^n|c_i(x)\le 0,i=1,\dots,m;c_i(x)=0,i=m+1,\dots,m+l\}$ 。

由于最小目标并不一定存在,但下确界一定存在,故一般将$\min f(x)$修改为$\inf f(x)$。

\subsection{分类}

根据约束$c$和目标函数$f$的性质分类:
\begin{itemize}
\item若$f,c$均为线性称为线性规划(LP)
\item若$f,c$中有非线性即称为非线性规划
\item若$f$为二次函数,$c$为线性称为二次规划(QP)
\end{itemize}

目标建模方法:范数、正则化、最大似然、松弛等

\subsection{求解方法}

一般无约束优化问题的求解方法通常可以分为:

\begin{itemize}
\item 线搜索法:线搜索法采用$x_{k+1}=x_k+\alpha_k d_k$的方式进行迭代,其中$\alpha_k$是步长,$d_k$是方向。步长的选取方法差不多,根据方向的选取方法不同,分为:梯度下降法和牛顿法。
\item 信赖域法:信赖域法采用二阶微分在邻域内用二阶函数代替。
\end{itemize}

最小二乘问题因为非常常见,因而研究了与其相应的特殊的求解方法。可以将最小二乘问题分为小残量问题和大残量问题。小残量问题一般采用高斯-牛顿(GN)法或LM方法,其中:高斯-牛顿法是线搜索方法,LM法是信赖域法。大残量问题用拟牛顿法来实现。