% 傅里叶变换与矢量空间

\pentry{傅里叶变换\upref{FTExp}}

\subsection{傅里叶变换与矢量空间}
类比傅里叶级数, 我们仍然可以将傅里叶变换看作是矢量空间中两组正交归一基底之间的变换, 我们分别把他们叫做 $x$ 基底和 $k$ 基底. 每个实数 $x_0$ 对应一个基底 $\uvec x_0$, 所有的 $\uvec x_0$ 构成 $x$ 基底. 每个实数 $k_0$ 对应一个基底 $\uvec k_0$, 所有的 $\uvec k_0$ 构成 $k$ 基底. 就像有限维空间中用 $(0, \dots , 1, \dots , 0)$ 表示一组基底的第 $i$ 个关于这组基底的系数, 可以用 $\delta (x - x_0)$ 表示 $\uvec x_0$ 关于所有 $x$ 基底的系数, 用 $\delta (k - k_0)$ 表示 $\uvec k_0$ 关于所有 $k$ 基底的系数.

现在, 函数 $f(x)$ 可以看作是 $x$ 基底的坐标, 而 $g(k)$ 可以看作是 $k$ 基底的坐标. 例如
\begin{equation}
\ket{f} = \int f(x_0) \uvec x_0 \dd{x_0}
\end{equation}
可以验证坐标为
\begin{equation}
\int f(x_0) \delta (x - x_0) \dd{x_0} = f(x)
\end{equation}

若使用 $x$ 基底, 就说在 $x$ 表象下, 若用 $k$ 基底就说在 $k$ 表象下. 傅里叶变换可以看成一个无穷维的酋矩阵. 两组基底之间的变换关系为: $x_0$ 基底在 $k$ 表象下的坐标为 $\E^{-\I k x_0}/\sqrt{2\pi }$, $k_0$ 基底在 $x$ 表象下的坐标为 $\E^{\I k_0 x}/\sqrt{2\pi }$.

$x$ 表象下, $x$ 基底的正交归一化可以记为
\begin{equation}
\uvec x_0 \vdot \uvec x_0' = \int \delta (x - x_0) \delta (x - x_0') \dd{x} = \delta (x_0 - x_0')
\end{equation}
$k$ 表象下, $x$ 基底的正交归一化可以记为
\begin{equation}
\uvec x_0 \vdot \uvec x_0' = \int \frac{\E^{\I k x_0}}{\sqrt{2\pi }} \frac{\E^{-\I k x_0'}}{\sqrt{2\pi }} \dd{k} = \delta (x_0 - x_0')
\end{equation}
注意这只是一种形式上的写法, 不要过度理解.

于是, 我们可以 “证明” $f(x)$ 经过傅里叶变换和反变换后, 仍然可以得到 $f(x)$
\begin{equation}
\begin{aligned}
\int g(k) \frac{\E^{\I k x}}{\sqrt{2\pi }} \dd{k} &= \int \qty(\int f(x') \frac{\E^{-\I k x'}}{\sqrt{2\pi }} \dd{x'}) \frac{\E^{\I k x}}{\sqrt{2\pi }} \dd{k}\\
&= \int f(x') \qty(\int \frac{\E^{-\I k x'}}{\sqrt{2\pi }}\frac{\E^{\I k x}}{\sqrt{2\pi }}\dd{k}) \dd{x'}\\
&= \int f(x') \delta (x - x') \dd{x'}\\
&= f(x)
\end{aligned}
\end{equation}
