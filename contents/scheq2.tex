% 自旋 1/2 粒子的非相对论波函数
% 自旋|薛定谔方程|泡利方程

\pentry{薛定谔方程(单粒子多维)\upref{QMndim},自旋角动量\upref{Spin},自旋角动量矩阵\upref{spinMt}}
\subsection{历史简介}
在量子力学发展的早期,薛定谔首先提出了 Klein Gordon 方程 $\partial^\mu \partial_\mu \phi+(m^2c^2/\hbar^2)\phi=0$,企图描绘遵从相对论变换的电子波动方程,但却遭遇失败.之后薛定谔退而求其次,转而求它的非相对论近似下的方程,得到了著名的薛定谔方程:
\begin{equation}
i\hbar \frac{\partial }{\partial t}\psi = \hat H\psi=\qty[\frac{ 1}{2m}\qty(-i\hbar\nabla)^2+V(x)]\psi
\end{equation}
换言之,粒子波动方程的能量由 $\hat H=(\hat{\bvec p}^2/2m+V)$ 给出,而 $\hat{\bvec p}=-i\nabla/\hbar$ 为动量算符.

虽然非相对论性的薛定谔方程能很好描绘电子的波粒二象性,但却没有给出电子的\textbf{内禀性质},也就是说,电子是个自旋为 $1/2$ 的粒子,则波函数一定是多分量的,而非单分量的.当空间发生旋转的时候,波函数的分量会随着参考系的旋转而发生变换.直到 Pauli (泡利)给出经典电磁场中的自旋 $1/2$ 电子的 Pauli 方程,人们才终于得到了描绘电子的携带自旋信息的非相对论性方程.Pauli 方程为
\begin{equation}
i\hbar\pdv{t} \psi=\qty[\frac{1}{2m}(\hat{\bvec p}-e\bvec A)^2+e\phi-\bvec \mu\cdot \bvec B]\psi = 0
\end{equation}
其中 $e=-q_e$ 为电子的电荷.若用上述方程描述自旋 $1/2$ 的其他粒子,则需要将 $e$ 用相应的电荷代入.其中电子磁矩 $\bvec \mu$ 为
\begin{equation}
\bvec \mu = \frac{e\hbar}{m}\frac{\bvec \sigma}{2}=\frac{e}{m}\bvec S=-g_{\rm spin}\mu_B \bvec S
\end{equation}
$\sigma$ 是 Pauli 矩阵(\autoref{Spin_eq7}~\upref{Spin}),$\bvec S$ 是自旋角动量算符.$\mu_B=|e\hbar/2m|$ 是 Bohr 磁子,$g_{\rm spin}=2$ 被称为自旋朗德(Lande)g 因子.

第一个提出自旋 1/2 粒子的相对论性方程的是 Dirac(狄拉克).Dirac 注意到 Klein Gordon 场中的负能量和负概率等一系列问题的核心,是因为方程中时间偏导是二次的.而一旦开根号变成一次偏导之后,$-i\pdv{(ct)} \phi=\sqrt{-\nabla^2+m^2c^2/\hbar^2}\phi$ 中时间偏导算符和空间偏导算符处于不对等的地位,就不具有洛伦兹协变的形式.为了解决问题, Dirac 创造性地提出了一种明为 Dirac 代数的结构,其中的元素的运算法则不同于复数域的运算\footnote{由于 Dirac 代数具有矩阵形式的表示,可以将其中的每一元素理解为 $4\times 4$ 的矩阵.虽然可以有多种矩阵表示法,但不同的表示法之间可以通过相似联系.},并假设 $-\hbar^2\nabla^2+m^2c^2$ 可以开方为 $\bvec \alpha \cdot \hat{\bvec p}+\beta mc$,其中 $\bvec \alpha$ 的三个坐标分量和 $\beta$ 都是 Dirac 代数的元素.因此需要满足一定的关系:$\alpha_i^2=1,\beta^2=1,\alpha_i\beta+\beta \alpha_i=0,\alpha_i\alpha_j+\alpha_j\alpha_i=2\delta_{ij}$.然而这些关系不可能在复数域内实现,因此 $\bvec \alpha,\beta$ 必然是 Dirac 代数中的元素,可以证明它们可以写成 $4\times 4$ 的矩阵形式,于是 $\psi$ 波函数也将是 $4$ 分量的.

Dirac 方程完美地解释了自旋 $1/2$ 粒子的行为,并且预言了正电子的存在,并给出了朗德 g 因子 $=2$ 的理论计算解释.这也意味着,Pauli 方程实际上是 Dirac 方程的一个非相对论的近似解(可以参考电磁场中的狄拉克方程\upref{DiracE}).

\subsection{自旋 $1/2$ 粒子的非相对论波函数}
这一节我们将沿用 Dirac 的思路,但不从相对论性方程出发,而是直接推导非相对论性的方程.也就是说,我们寻找另一种类似的方式来表达薛定谔方程.
假设薛定谔算符 $\frac{2}{mc^2} \hat H-\sum_i\frac{P_i}{mc}\frac{\hat P_i}{mc}$ 可以开方:
\begin{equation}
\qty(a\frac{\hat H}{mc^2}+b+\sum_ic_i \frac{\hat P_i}{mc})^2=\frac{2}{mc^2} \hat H-\sum_i\frac{\hat P_i}{mc}\frac{\hat P_i}{mc}
\end{equation}
因此自由粒子的薛定谔方程可以改写为
\begin{equation}\label{scheq2_eq2}
\qty(a\frac{\hat H}{mc^2}+b+\sum_ic_i \frac{\hat P_i}{mc})\psi=0
\end{equation}

其中 $a,b,c_i$ 的运算法则可能不同于复数域的法则,而是属于某种特殊的代数.对平方项展开,得到它们所需满足的等式关系:
\begin{equation}\label{scheq2_eq1}
\begin{aligned}
&a^2=0,ab+ba=2,ac_i+c_ia=0\\
&b^2=0,bc_i+c_ib=0\\
&c_ic_j+c_jc_i=2\delta_{ij}
\end{aligned}
\end{equation}
$a,b,c_i$ 所在的代数具有 $4\times 4$ 的矩阵表示.我们不加证明地写出一种矩阵表示方法:
\begin{equation}
\begin{aligned}
&a=-i\frac{1}{2}\pmat{I&-I\\ I&-I},b=i\pmat{I&I\\-I&-I}\\
&c_i=i\pmat{0&\sigma_i\\\sigma_i&0}
\end{aligned}
\end{equation}
不难验证它们满足 \autoref{scheq2_eq1} 的关系式.最后我们将这些表达式代入\autoref{scheq2_eq2} 得到
\begin{equation}
-i\frac{1}{2}\pmat{I&-I\\ I&-I} \frac{\hat H}{mc^2}+i\pmat{I&I\\-I&-I}+i\sum_i\pmat{0&\sigma_i\\\sigma_i&0} \frac{P_i}{mc}\pmat{\phi\\\chi}=0
\end{equation}
其中设波函数的前两个分量为 $\phi$,后两个分量为 $\chi$,即
\begin{equation}\label{scheq2_eq4}
\psi(x)=\pmat{\phi(x)\\\chi(x)}
\end{equation}
上述方程可以改写为两个关于二分量函数 $\phi,\chi$ 的方程:
\begin{equation}
\begin{aligned}
&-i\frac{1}{2}\frac{\hat H}{mc^2}(\phi-\chi)+i(\phi+\chi)+i\frac{\bvec \sigma\cdot \hat{\bvec P}}{mc}\chi=0\\
&-i\frac{1}{2}\frac{\hat H}{mc^2}(\phi-\chi)-i(\phi+\chi)+i\frac{\bvec \sigma\cdot \hat{\bvec P}}{mc}\phi=0
\end{aligned}
\end{equation}
经过一系列的化简,最终可以得到
\begin{equation}\label{scheq2_eq3}
\begin{aligned}
&\qty[\hat H-\frac{(\bvec \sigma\cdot \hat{\bvec P})^2}{2m}]\phi=0\\
&
\qty[\hat H-\frac{(\bvec \sigma\cdot \hat{\bvec P})^2}{2m}]\chi=0
\end{aligned}
\end{equation}
因此根据上面的方程可以知道,$\phi,\chi$ 两分量都满足薛定谔方程,同时它们分别是二分量的,体现了电子波函数的内禀自旋性质.

如果不加外部的电场和磁场,那么 $\hat H$ 和 $\hat P$ 就是自由粒子的能量和动量,且两两对易.根据 Pauli 矩阵的性质可以证明 $(\bvec \sigma\cdot \bvec a)(\bvec \sigma\cdot \bvec b)=\bvec a\cdot \bvec b+i\bvec \sigma\cdot (\bvec a\times \bvec b)$,所以当 $\hat P_i$ 和 $\hat P_j$ 对易时,$(\bvec \sigma\cdot \hat{\bvec P})^2=\hat{\bvec P}^2$,于是上式转化为薛定谔方程的形式.

然而如果加了外加电磁场,$\hat H$ 和 $\hat{\bvec P}$ 应作以下替换(电磁场中的薛定谔方程及规范变换\upref{QMEM})
\begin{equation}
\hat H\rightarrow \hat H-e\phi, \hat{\bvec P}\rightarrow \hat{\bvec P}-e\hat{\bvec A}
\end{equation}
此时 $\hat{\bvec P}-e\hat{\bvec A}$ 的各个分量间不再对易,因此所得到的方程将会有一个 $\bvec \sigma\cdot \bvec B$ 项出现.最终所得的方程被称为 \textbf{Pauli 方程}.具体的细节可以参考泡利方程\upref{Pauli}.