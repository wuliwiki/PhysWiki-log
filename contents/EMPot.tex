% 电磁场标势和矢势
\subsection{结论}
用表示和矢势表示电磁场
\begin{equation}\label{EMPot_eq1}
\vec E = -\grad \Phi - \pdv{\vec A}{t}
\end{equation}
\begin{equation}\label{EMPot_eq2}
\vec B = \curl \vec A
\end{equation}

\subsection{推导}
首先定义 $\vec A$, 则由法拉第电磁感应定律(\autoref{MWeq_eq2}\upref{MWeq})
\begin{equation}
\curl \qty(\vec E + \pdv{\vec A}{t}) = \curl \vec E + \pdv{\vec B}{t} = \vec 0
\end{equation}
这说括号中的矢量可以表示为一个标量函数的梯度,即标势 $\Phi$, 负号是为了在静电场的情况下使得标势等于电势.

\subsection{标势和矢势的麦克斯韦方程组}

将\autoref{EMPot_eq1} 和\autoref{EMPot_eq2} 代入麦克斯韦方程组可以得到两条与麦克斯韦方程组等效的方程
\begin{equation}\label{EMPot_eq4}
\laplacian \Phi + \pdv{t} (\div \vec A) = -\frac{\rho}{\epsilon_0}
\end{equation}
\begin{equation}\label{EMPot_eq5}
\qty(\laplacian \vec A - \mu_0\epsilon_0 \pdv[2]{\vec A}{t}) - \grad \qty(\div \vec A + \mu_0\epsilon_0 \pdv{V}{t}) = -\mu_0\vec J
\end{equation}

