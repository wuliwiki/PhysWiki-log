% VirtualBox 笔记
% license Xiao
% type Note

\begin{issues}
\issueDraft
\end{issues}

虚拟机的好处是可以同时使用不同系统, 可以创建一个独立环境不影响电脑的安全, 可以做任意实验并在很短时间内滚回到以前的某个状态, 可以模拟不同计算机之间的网络通信。

\subsection{安装和基本设置}

\begin{itemize}
\item 用 virtualbox 运行 linux, 首先下载 virtualbox 和 ubuntu 安装包.
\item 安装任何操作系统需要先下载安装系统的 iso 文件(windows 可以用 media creator), Ubuntu 直接从官网下载
\item 如果选择系统的时候只出现 32bit 就检查三件事情 1. 搜索 turn windows features on and off, 检查 Hyper-V 是否开启,需要关闭。 2. 在 EUFI 里面检查 Intel Vitualization Technology 和 Intel VT-d Feature 有没有打开,需要打开。 参考\href{http://www.fixedbyvonnie.com/2014/11/virtualbox-showing-32-bit-guest-versions-64-bit-host-os/}{这里}。
\item 如果启动时提示 BIOS 中禁止了所有 cpu 的虚拟化(VT-x), 就重启然后在 BIOS 中开启相关设置。
\item cpu, 内存等是可以在每次启动虚拟机之前设置的
\item 硬盘选 dynamic 的话,无论虚拟硬盘有多大,实际上占的空间只是实际使用的空间。
\item 如果安装完重启的时候卡住, 按 \verb|ctrl + C| 即可.
\item 至于分辨率只有 1024 的问题, 在命令行安装如下软件, 就可以在 view 菜单中调整分辨率了.
\begin{lstlisting}[language=bash]
sudo apt-get install virtualbox-guest-utils
virtualbox-guest-x11 virtualbox-guest-dkms
\end{lstlisting}
\item 这些包好像在 ubuntu 18.04 上安装不了, 试图从 Virtualbox 的 Device 菜单中选择 Insert Guest Addition CD image, 看是不是同样的功能。 安装完需要重启。 果然, 屏幕尺寸可以调了, 剪切板可以复制了。
\item snapshot 是一个非常重要好用的功能, 在虚拟机列表中右边的菜单中
\item 备份虚拟机, 只需要在虚拟机列表中通过右键菜单找到所在文件夹, 复制即可. 需要恢复, 则在菜单中 \verb|Machne->Add| 即可.
\item 共享文件夹: 打开虚拟机, 菜单中选 device -> shared folders, 创建即可, Folder Path 是 host 中的 path, 可以下拉选择 Folder Name 就是 path 中最后那个文件夹的名字。 shared folder 在 windows guest 中以网络磁盘的形式存在。 Mount point 可以指定一个盘符, 如果想要自动分配盘符, 可以选择 automount。 选择 Make permanent, 重启以后也不会消失。
\item 要开启共享剪贴板, 在菜单栏的 Devices 设置. Drag and Drop 的功能传文件最方便. 从虚拟机里面
\item 如果要断网, 在 setting 里面的 Network 将 Cable Connected 取消勾选即可。
\item Ubuntu guest 中如果挂载 host 的文件夹, 好像只有管理员才能访问。
\end{itemize}

\subsubsection{清理硬盘}
\begin{itemize}
\item 在命令行中用 \verb|VBoxManage modifymedium --compact /path/to/your.vdi| 命令可以清理硬盘。
\end{itemize}

\subsection{搭建局域网}
\begin{itemize}
\item VirtualBox 虚拟机搭建局域网见: \autoref{sub_LinWeb_1}~\upref{LinWeb}
\end{itemize}
