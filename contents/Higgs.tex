% 希格斯粒子
% license CCBYSA3
% type Wiki

(本文根据 CC-BY-SA 协议转载自原搜狗科学百科对英文维基百科的翻译)

  希格斯玻色子是粒子物理标准模型中的基本粒子,由粒子物理理论中的希格斯场量子激发产生。它是以物理学家彼得·希格斯(Peter Higgs)的名字命名的,Higgs在1964年与其他五位科学家一起提出了这种粒子存在的机制。它的存在于2012年由欧洲核子中心(CERN)的ATLAS实验组和CMS实验组基于大型强子对撞机(LHC)上的对撞实验的联合确认。

2013年12月10日,彼得·希格斯(Peter Higgs)和弗朗索瓦·恩格勒(François Englert)两位物理学家,因为他们的理论预测被授予诺贝尔物理学奖。虽然希格斯的名字已经与这个理论(希格斯机制)联系在一起,但在1960年至1972年间,还有其他的一些研究人员在这个问题的不同方面作出了自己独立的贡献。

主流媒体经常将希格斯玻色子称为“\textbf{上帝粒子}”,源于1993年一本关于这个话题的书[1] ,但是许多物理学家,包括希格斯本人,都认为这个绰号是有些夸大的。

\subsection{介绍}

\subsubsection{1.1标准模型}

标准模型是目前物理学家广泛接受的用来解释基本粒子之间力的性质的理论框架,可以用于理解除了引力(一个独立的理论,广义相对论,用于引力理论)之外的已知宇宙中的几乎所有事物。在这个模型中,自然界的基本力来自于我们宇宙的规范不变性和对称性。力是由被称为规范玻色子的粒子传递的[2][3]。

在标准模型中,希格斯粒子是自旋为零的玻色子,没有电荷,也没有色荷。它也非常不稳定,几乎立即衰变为其它粒子。希格斯场是标量场,有两个中性的和两个带电的分量,它们形成了弱同位旋SU(2)对称性的复二重态。希格斯场的势是“墨西哥帽形”的。这导致场在它的基态任何地方都有非零值(包括其他的“空"空间),结果,在非常高的能量之下,电弱相互作用的弱同位旋对称性的破缺。(技术上非零期望值将拉格朗日量中的汤川耦合项转换为质量项。)当这种情况发生时,希格斯场的三个分量被SU(2)和U(1)规范玻色子(“希格斯机制”)吸收,成为现在有质量的传递弱力W及Z玻色子的纵向分量。其余的电中性分量要么表现为希格斯粒子,要么可以单独与费米子耦合(通过汤川耦合),促使这些粒子也获得质量[4]。

\subsubsection{1.2规范玻色子质量问题}

场论在理解电磁场和强力方面取得了巨大成功,但是到了1960年左右,所有试图建立一个将电磁相互作用和弱相互作用结合在一起的 规范不变 的弱力理论都失败了,规范场论的声誉从此受到了巨大的影响。问题在于规范场论中的对称性要求电磁力的规范玻色子(光子)和弱力的规范玻色子(W和Z玻色子)都应该具有零质量。虽然光子确实没有质量,但实验表明弱玻色子有质量[5]。 这意味着要么规范不变性是不正确的理论,要么还有其它未知的理论赋予这些粒子质量,但是所有试图提出能够解决这个问题的理论的尝试似乎都遇到新的理论上的问题。

到20世纪50年代末,物理学家还没有解决这些问题,这些问题是发展成熟的粒子物理学理论遇到的重大障碍。

\textbf{对称性破缺}

到20世纪60年代初,物理学家已经意识到,在某些条件下,至少在某些物理领域,给定的对称定律可能并不总是被遵循 。这就是所谓的对称破缺,并在20世纪50年代末被南部阳一郎(Yoichiro Nambu)证实。对称破缺会导致意想不到的结果。1962年,超导物理学家菲利普·安德森(Philip Anderson)写了一篇研究粒子物理学中对称性破缺的论文,提出对称性破缺可能是解决粒子物理学中规范不变性问题所缺少的一部分。如果电弱对称不知何故被打破了,这也许可以解释为什么电磁玻色子没有质量,而弱玻色子有质量,并解决了问题。不久之后,在1963年,理论上证明这是可能的,至少对于一些有限的(非相对论性的)情况。

\subsubsection{1.3希格斯机制}

继1962年和1963年的论文之后,1964年三组研究人员在《物理评论快报》(PRL)上分别独立发表了对称破缺的论文,这些论文具有相似的结论,适用于所有情况,而不仅仅是一些有限的情况。他们表明,如果宇宙中存在一种不寻常的场,使得电弱对称性的条件将被“打破”,从而使一些基本粒子将获得质量。产生这种机制所需要的场(当时纯粹是假设性的)被称为希格斯场 (以研究人员之一彼得·希格斯(Peter Higgs)的名字命名)以及它导致对称性破缺的机制,称为希格斯机制。这个场的一个关键特征是不像其他已知的场真空期望值在Φ为0处,而是在Φ不为0处能量更少。因此,希格斯场各处具有非零值(或 真空期望值) 。这是第一个能够证明在规范不变理论中,弱规范玻色子如何在对称性的情况下仍然具有质量的假设。

虽然这些想法最初并没有得到广泛的支持或关注,但到了1972年,它们已经发展成为一个全面的理论,并被证明能够给出“合理的”结果,准确地描述当时已知的粒子,并以异常准确的精度预测了随后几年发现的其他几个粒子。 在20世纪70年代,这些理论迅速形成了粒子物理学的标准模型。虽然目前还没有任何直接证据证明希格斯场存在,但即使没有该场存在的证据,其预测的准确性也让科学家相信该理论可能是真的。到了1980年代,希格斯场是否存在,以及整个标准模型是否正确的问题,已经被认为是粒子物理学中最重要的未解问题之一。

\textbf{希格斯场}

根据标准模型,这个所需的场(即 希格斯场)存在于全空间中,并且破坏了电弱相互作用的某些对称性 。通过希格斯机制,这个场使得弱规范玻色子在低于一个极高温度值之下具有质量。当弱玻色子获得质量时,它们只能存在很短的距离 。此外,后来人们意识到,同样的场也会以不同的方式解释为什么物质的其他基本部分(包括电子和夸克)具有质量。

几十年来,科学家们无法确定希格斯场是否存在,因为还没有探测它所需的技术。如果希格斯场确实存在,那么它将不同于任何其他已知的基本领域,但也有可能这些关键思想,甚至整个标准模型,在某种程度上是不正确的 。只有发现希格斯玻色子和希格斯场的存在才能解决了这个问题。

与电磁场等其他已知场不同,希格斯场是标量场,在真空中具有非零常数值。希格斯场的存在成为粒子物理学标准模型中最后一个未经证实的部分,几十年来被认为是“粒子物理学的中心问题”。

这个场的存在,现在被实验研究证实,解释了为什么一些基本粒子有质量,尽管它们的相互作用所遵循的对称性意味着它们应该是无质量的。它还解决了其他几个长期存在的难题,比如为什么弱力范围极小。

尽管希格斯场在任何地方都不是零,其影响无处不在,但证明它的存在绝非易事。原则上,它可以通过检测它的激发态来证明它的存在(希格斯玻色子),但是这些是非常难以生产和检测的。这个重要的基本问题经历了40年的探索,以及迄今为止世界上最昂贵和最复杂的实验设施之一——欧洲核子中心的大型强子对撞机的建造[6] ,就是试图产生希格斯玻色子和其他粒子用于观察和研究。2012年7月4日,CERN宣布发现了一个质量在125~ 127 GeV/c2 之间的新粒子。物理学家怀疑这是希格斯玻色子[7] 。从那以后,这个粒子被证明符合标准模型预测希格斯粒子的许多方面表现、相互作用和衰变,并且具有希格斯玻色子的两个基本属性偶宇称和零自旋。 这也意味着它是自然界中发现的第一个基本标量粒子 。截至2018年,深入研究表明,该粒子的行为仍然符合标准模型对希格斯玻色子的预测。需要进行更多的研究来更精确地验证所发现的粒子是否具有预测的所有性质,或者如一些理论所描述的,是否存在多个希格斯玻色子[8]。

\textbf{希格斯玻色子}

理论假设的希格斯机制做出了数个准确的预测 然而,为了证实它的存在,寻找与”希格斯玻色子”相匹配的粒子,人们进行了广泛的研究 。由于产生,希格斯玻色子所需要的能量,以及即使能量足够,希格斯玻色子的产生也非常罕见,因此很难探测到希格斯玻色子。所以,几十年后,希格斯玻色子存在的第一个证据才被发现。能够寻找希格斯玻色子的粒子对撞机、探测器和计算机花费了30多年时间(约1980年至2010年)才研制出来。

到2013年3月,希格斯玻色子的存在得到了确认,因此,某种类型的希格斯场存在于全空间的概念得到了强有力的支持 。目前正在利用LHC收集的更多数据,进一步研究希格斯场的本质和性质。

\textbf{解释}

各种各样的类比被用来描述希格斯场和玻色子,包括与众所周知的对称破缺效应的类比,例如彩虹和棱镜,电场,波纹,和宏观物体在介质中移动的阻力(例如人们在人群中移动或一些物体在糖浆或糖蜜中移动)。然而,基于简单的运动阻力的类比是不准确的,因为希格斯场不是通过抵抗运动来工作的。

\subsection{意义}

\subsubsection{2.1粒子物理学}

\textbf{标准模型的验证}

希格斯玻色子通过质量产生机制验证了标准模型。随着对其属性进行更精确的测量,一些相关的理论可以得以继续发展或被排除正确的可能性。随着测量场的行为和相互作用的实验手段的发展,这个基本场可能会得到更好的理解。如果希格斯场没有被发现,标准模型将需要被修改或被其他理论取代。

与此相关的是,物理学家普遍认为有超越标准模型的“新”物理的存在,标准模型在某些时候会需要新的内容或者失效。希格斯玻色子的发现,以及在LHC上其他对撞实验,为物理学家分析标准模型失效的部分提供了一个敏感的工具,并且可以提供相当多的证据指导研究人员进入未来的理论发展。

\textbf{弱电相互作用的对称破缺}

在极高的温度之下,电弱对称破缺导致电弱相互作用部分表现为由有质量规范玻色子携带的短程弱力。这种对称破缺是原子和其他结构形成所必需的,也是恒星(如太阳)核反应所必需的。希格斯场在这种对称破缺中起到主要作用。

\textbf{粒子质量的获得}

希格斯场在夸克和带电轻子(通过汤川耦合)获得质量和W和Z规范玻色子(通过希格斯机制)获得质量方面起着关键作用。

值得注意的是,希格斯场不是凭空“创造”质量的(这将违反能量守恒定律),希格斯场也不对所有粒子的质量负责。例如,重子(复合粒子,如质子和中子)的质量的大约99%是由于量子色动力学结合能获得的,这是夸克的动能和在重子内部传递强相互作用的无质量胶子的能量之和[9] 。在基于希格斯玻色子的理论中,“质量”的性质是当基本粒子与希格斯场相互作用(“耦合”)时转移到基本粒子的势能的一种表现,该粒子以能量的形式具有该质量[10]。

\textbf{标量场与标准模型的扩展}

希格斯场是唯一被发现的标量场(自旋为0);标准模型中其它所有的场都是自旋½的费米子或自旋为1玻色子。当希格斯玻色子被发现时,欧洲核子中心总干事罗尔夫-迪特尔·霍耶尔(Rolf-Dieter Heuer)认为,证明标量场的存在几乎和希格斯粒子在确定其他粒子质量中的作用一样重要。它表明,其他理论从暴胀到暗能量假设的标量场,也可能存在[11][12]。

\subsubsection{2.2宇宙学}

\textbf{暴胀场}

目前已经对希格斯场和暴胀之间的可能联系进行了大量的科学研究,希格斯场作为一个假设的场被认为是宇宙在1/2秒膨胀的原因(称为“暴胀时期”)。一些理论认为,基本标量场可能是造成这种现象的原因;希格斯场就是这样一个场,一些论文分析,希格斯场是否也可能是大爆炸期间宇宙指数暴胀的原因。这样的理论是高度试验性的,并且面临着与幺正性相关的重大问题,但如果与其他特征如大的非最小耦合、Brans-Dicke标量或其他“新”物理等相结合,也许是可行的,目前对它们的处理也表明了希格斯暴胀模型在理论上仍然是可行的。

\begin{figure}[ht]
\centering
\includegraphics[width=6cm]{./figures/8587493c2fe34a0d.png}
\caption{图中显示了希格斯玻色子和顶夸克质量,这可能表明我们的宇宙是稳定的,还是一个长期存在的“气泡”。截至2012年, 椭圆是仍然允许两种可能性2σ置信度基于Tevatron和大型强子对撞机的数据。[7]} \label{fig_Higgs_1}
\end{figure}

\textbf{宇宙的本质及其可能的命运}

在标准模型中,可能存在着所谓的“真空”作为宇宙一种并不显现的状态,寿命很长,不完全稳定。在这种情况下,我们所知的宇宙可以通过坍缩成更稳定的真空状态而被破坏摧毁[13][14][15][16][17] 。有时被误报为希格斯玻色子“终结”了宇宙。如果可以更精准的知道希格斯玻色子和顶夸克的质量,并且标准模型提供了在普朗克尺度极限能量下对粒子物理学的精确描述,那么就有可能计算真空是稳定的还是仅仅是寿命长的[18][19][20]。一个质量在 125~127GeV范围内的希格斯玻色子似乎已经在稳定的边缘,但要获得一个明确的答案还需要对顶夸克的质量进行更精确的测量[21]。 新物理可以改变这一局面[22]。

如果对希格斯玻色子的测量表明我们的宇宙处于这种假真空中,那么这就意味着 – 很可能在几十亿年后  – 如果一个真正的真空碰巧成核,宇宙的力、粒子和结构可能会像我们所知的那样不复存在(并被不同的取代) 。这也表明希格斯粒子是自耦合参数 λ 和它的 βλ 函数在普朗克尺度可能非常接近于零,这意味着包括引力理论和基于希格斯玻色子的暴胀理论都是“有趣的”[21][23][24] 。未来的正负电子对撞机将能够提供这种计算所需的顶夸克的精确测量[21]。

\textbf{真空能量与宇宙常数}

进一步大胆的推测,希格斯场也可以认为是真空的能量,在大爆炸的第一个瞬间的极端能量下,导致宇宙成为一种无差别的、极高能量的无特征对称性。在这种推测中,大统一理论的单一统一场被假定为(或模拟为)希格斯场,在目前已知的宇宙力和场出现发生相变时,是通过希格斯场或类似场的连续对称破缺[25]。

希格斯场和目前观测到的宇宙真空能量密度之间的关系(如果有的话)也在科学研究中。正如所观察到的,目前的真空能量密度非常接近零,但是在希格斯场、超对称和其他目前的理论中预期的能量密度通常要大许多数量级。目前还不清楚这些应该如何调和。宇宙常数这个问题仍然是物理学中另一个未解的主要问题。

\subsubsection{2.3实际应用和技术影响}

到目前为止,还没有发现希格斯粒子的直接应用的技术优势。然而,基本理论发现的一个常见模式是一旦发现被进一步探索,随后的实际应用,可能成为对社会重要的新技术的基础[26][27][28]。

粒子物理学的挑战推动了具有广泛重要性的技术重大的进步。例如,万维网最初是一个改善欧洲核子中心通信系统的项目。欧洲核子中心要求处理大型强子对撞机产生的大量数据,这也促使了对云计算与分布式系统领域的贡献。

\subsection{历史}

\subsubsection{3.1理论化}

\begin{figure}[ht]
\centering
\includegraphics[width=6cm]{./figures/1730dcb6a7adef89.png}
\caption\label{fig_Higgs_2}
\end{figure}

\begin{figure}[ht]
\centering
\includegraphics[width=6cm]{./figures/89c718c2344b2a98.png}
\caption{2010年,六位1964年PRL文章的作者获得了 J. J. Sakurai 奖;从左到右分别是 :Kibble, Guralnik, Hagen, Englert, Brout; 右: Higgs.} \label{fig_Higgs_3}
\end{figure}

粒子物理学家研究由基本粒子构成的物质,基本粒子的相互作用由交换规范玻色子充当力的传递者。在20世纪60年代初,已经发现或提出了许多这样的粒子,以及建立了它们相互联系的理论,其中一些已经被重新表述为场论,研究的对象不是粒子和力,而是量子场及其对称性[29] 。然而,试图将四种已知基本力中的两种电磁力和弱核力统一的量子场模型仍然没有成功。

\begin{figure}[ht]
\centering
\includegraphics[width=6cm]{./figures/495ef5e3b4f2141f.png}
\caption{诺贝尔奖获得者 Peter Higgs 在斯德哥尔摩,2013年12月} \label{fig_Higgs_4}
\end{figure}

一个已知的问题是规范不变方法,包括如杨-米尔斯理论(1954)的非阿贝尔模型,曾经很有希望发展成为统一理论,似乎也将已知的有质量粒子预测是无质量的[30] 。Goldstone定理,与某些理论中的连续对称性有关,似乎也排除了许多显而易见的解决方案[31] ,因为它表明,零质量粒子也是存在,只是“看不见”[32] 。根据古拉尔尼克(Guralnik)的说法,物理学家“不知道”如何克服这些问[32]。

粒子物理学家和数学家彼得·沃伊特(Peter Woit )总结了当时的研究状况:

杨和米尔斯在研究非阿贝尔规范理论时遇到了一个大问题:在微扰理论中,它有无质量的粒子,dan 它们与我们看到的任何东西都不对应。解决这个问题的方法现在已经很好地理解了,在量子色动力学(QCD)中实现了禁闭现象,强相互作用消除了长程的无质量胶子态。到60年代早期,人们已经开始了解另一种无质量粒子的来源:连续对称性的自发对称性破缺。菲利普·安德森(Philip Anderson)在1962年夏天发现,当同时存在规范对称性和自发对称性破断时,Nambu–Goldstone 无质量模式可以与无质量规范场模式结合,产生一个有物理质量的矢量场。这就是超导的产生,安德森是(现在也是)超导领域的主要专家之一 [30]。

希格斯机制是矢量玻色子获得静止质量却 没有 明确地破缺规范不变性作为自发对称破缺的副产物的过程[33]。 最初,自发对称破缺背后的数学理论是由南部阳一郎(Yoichiro Nambu)于1960年在粒子物理学框架下构思并发表的[34] ,1962年,菲利普·安德森(Philip Anderson)最早提出这种机制可以为“质量问题”提供可能的解决方案[35] 。安德森在他1963年关于杨-米尔斯理论的论文中总结道,“考虑超导类似...这两种玻色子似乎能够互相抵消...留下有限质量玻色子)[36] ,1964年3月,亚伯拉罕·克莱因(Abraham Klein)和本杰明·李(Abraham Klein)展示了Goldstone定理至少在一些非相对论的情况下可以避免这个问题,并推测在相对论情况下也是可能适用的[37]。

这些方法很快发展成为一个完整的相对论模型,由三组物理学家独立地几乎同时完成:1964年8月由弗朗索瓦·恩格勒(François Englert)和罗伯特·布绕特(Robert Brout)完成[38] ;彼得·希格斯(Peter Higgs)在1964年10月完成[39] ;杰拉德·古拉尼(Gerald Guralnik)、卡尔·哈根(Carl Hagen)和汤姆·基博尔(Tom Kibble GHK)于1964年11月完成[40]。 希格斯也写了一篇简短但重要的文章 ,1964年9月发表的对吉尔伯特反对意见的答复[41] ,这表明如果在辐射规范内计算,Goldstone定理和吉尔伯特的反对意见将变得不相容。 (希格斯后来将吉尔伯特的反对意见描述为促使他撰写自己的论文的动力[42],)Guralnik在1965年进一步考虑了模型的性质[43] ,希格斯在1966年提出[44] ,在1967年被Kibble[45] ,并于1967年由GHK进一步提出[46] 。最初的三篇1964年的论文证明,当规范场论与自发对称性破缺的附加场结合时,规范玻色子始终可以获得有限的质量[33][47] 。1967年,史蒂芬·温伯格(Steven Weinberg) [48] 和阿卜杜勒·萨拉姆( Abdus Salam)[49] 独立展示了希格斯机制如何打破谢尔顿·格拉肖(Sheldon Glashow)的弱相互作用统一模型的电弱对称性[50] ,(本身是施温格工作的延伸),形成了粒子物理学的标准模型。温伯格是第一个观察到这也会为费米子提供质量项的人[51]。

起初,这些关于规范对称性自发破缺的开创性论文很大程度上被忽略了,因为人们普遍认为所讨论的(非阿贝尔规范)理论是一个死胡同,特别是它们不能被重整化。1971-1972年,马丁努斯·维尔特曼(Martinus Veltman)和杰拉德·特·胡夫特(Gerard 't Hooft)在两篇论文中证明了杨-米尔斯的重整化是可能的,两篇论文涵盖了无质量,然后是有质量的场[51] 。他们的贡献,以及重整化团队中其他人的工作 – 包括俄罗斯物理学家路德维希·法捷耶夫(Ludvig Faddeev)、安德烈·斯拉沃诺夫(Andrei Slavnov)、Efim Fradkin和伊戈尔·秋京(Igor Tyutin)的“实质性”理论工作[52] – 最终“影响深远”[53], 但即使最终理论的所有关键要素都已发表,也几乎没有获得更广泛的兴趣。例如,科尔曼在一项研究中发现,在1971年之前,“基本上没有人注意”温伯格的论文[54] 。大卫·波利策()在2004年的诺贝尔演讲中对此进行了讨[53]。 – 现在粒子物理学中引用最多的  – 即使在1970年,根据Politzer,Glashow的弱相互作用教学也没有提到温伯格、萨拉姆或Glashow自己的工作[53]。 在实践中,波利策说,几乎每个人都从物理学家本杰明·李那里学到了这个理论,他把维尔特曼和特·胡夫特的工作与其他人的见解结合起来,并推广了完整的理论[53] 。就这样,从1971年开始,兴趣和接受“爆炸式”增长[53] 这些想法很快被主流所接受[51][53]。

由此产生的电弱理论和标准模型准确地预测了弱中性流、三个玻色子、顶夸克和粲夸克,并且非常精确地预测了其中一些粒子的质量和其他性质。 许多参与者最终获得了诺贝尔奖或其他著名奖项。1974年的论文和综合评论 《现代物理评论》 评论说“虽然没有人怀疑(数学上)这些论点的正确性,但也没有人完全相信,大自然有足够的聪明才智来利用它们”[55] ,他补充说,到目前为止,该理论已经产生了与实验相符的准确答案,但是还不知道该理论是否从根本上正确[56] 。到了1986年和1990年代,理解和证明标准模型的希格斯场成为了“当今粒子物理学的中心问题”[57][57]。

\textbf{PRL文章的简述和影响}

《物理评论快报》50周年庆典上,写于1964年的三篇论文都被认为是里程碑论文。[47] 他们的六位作者也因此获得了2010年J. J. Sakurai理论粒子物理学奖。[57] (同年也出现了一场争论,因为在诺贝尔奖的要求,只有三位科学家可以获奖,但其中六位被认为是论文的作者。[58]三篇PRL论文中的两篇(由希格斯玻和GHK撰写)包含了最终被称为希格斯场的假想场方程和它的假想量子希格斯玻色子。[39][40] 希格斯随后在1966年的论文展示了玻色子的衰变机制;只有有质量的玻色子才能衰变,衰变可以证明这一机制。

在希格斯的论文中,玻色子是有质量的,希格斯在结束语中写道,这个理论的“一个基本特征”是“标量和矢量玻色子的不完全多重态的预测”。[39] (弗兰克·克洛斯(Frank Close) 评论说,20世纪60年代的规范理论学家关注的是无质量矢量玻色子和有质量标量玻色子的存在并不重要;只有希格斯粒子直接解决了这个问题。)在GHK的论文中,玻色子是无质量的,并与有质量的态退耦。[40] 在2009年和2011年的评论中,古拉尼克指出,在GHK模型中,玻色子只有在最低阶近似下是无质量的,但它不受任何约束,并在更高阶获得质量,并补充说,GHK论文是唯一一个显示模型中没有无质量的Goldstone玻色子,并给出了希格斯机制的完整分析。[32] 这三者得出了相似的结论,尽管它们的方法非常不同:希格斯的论文基本上使用了经典技术,恩格尔特和布劳特在微扰理论中涉及到围绕假设的对称破缺真空态计算真空极化,GHK使用算符形式和守恒定律深入探索Goldstone定理的使用方式。 该理论的一些版本预测了不止一种希格斯场和玻色子,并且在希格斯玻色子被发现之前,人们一直在考虑其他的“希格斯场”模型。

\subsubsection{3.2实验性搜索}

为了产生希格斯玻色子,两束粒子被加速到非常高的能量,在粒子探测器内进行碰撞实验。偶尔,尽管很少,希格斯玻色子会作为碰撞副产物的一部分快速产生。由于希格斯玻色子衰变非常快,粒子探测器无法直接探测到它。但是,探测器记录所有的衰变产物 (衰变信号)并根据数据重建衰变过程。如果观察到的衰变产物符合希格斯玻色子的可能的衰变过程(称为 衰变道)这表明希格斯玻色子可能已产生出来。实际上,许多过程可能会产生类似的衰变特征。幸运的是,标准模型精确地预测了其中每一个以及每一个已知过程发生的可能性。因此,如果探测器检测到更多的衰变特征一致地匹配希格斯玻色子,而不是希格斯玻色子不存在的预期,那么这将是希格斯玻色子存在的有力证据。

因为希格斯玻色子在粒子碰撞中的产生可能非常罕见(在LHC是100亿分之一), 和许多其他可能的碰撞事件可以有类似的衰变特征,需要分析数百万亿次碰撞的数据,并且必须“显示相同的图像”,才能得出希格斯玻色子的存在的结论。为了得出一个新的粒子已经被发现的结论,粒子物理学家需要对两个独立的粒子探测器进行统计分析,每个探测器都表明所观察到的衰变特征仅仅是由背景随机事件引起的概率小于百万分之一 – 即观察到的事例数量与没有新粒子时的预期数量相差超过5个标准偏差(σ)。更多的碰撞数据可以更好地确认所观察到的新粒子的物理性质,并允许物理学家确定它是否真的是标准模型所描述的希格斯玻色子或其他一些假设的新粒子。

为了找到希格斯玻色子,需要强大的粒子加速器,因为在低能实验中可能看不到希格斯玻色子。对撞机需要有高亮度,以确保能看到足够多的碰撞,从而得出结论。最后,需要先进的计算设备来处理碰撞产生的大量数据(截至2012年,每年25pb)。[59] 在2012年7月4日的公告中,欧洲核子中心建造了一个名为大型强子对撞机的新对撞机,计划最终碰撞能量为14 TeV – 超过以前对撞机的七倍 – 世界上最大的计算网格LHC计算网格(截至2012年)分析了LHC质子-质子碰撞数据超过 $3\times10^{14}$ 次。

\textbf{2012年7月4日前搜索}

20世纪90年代,欧洲核子中心在大型正负子对撞机进行了第一次大规模的希格斯玻色子研究。在2000年服务结束时,LEP没有发现希格斯粒子存在的确凿证据。 这意味着如果希格斯玻色子要存在,它的质量必须比 $114.4 GeV/c^{2}$要高。[62]

美国的费米国家加速器实验室的1995年发现顶级夸克的对撞机的兆电子伏特加速器继续进行研究,并 已经为此升级。虽然不能保证兆电子伏特加速器能找到希格斯粒子,因为大型强子对撞机(LHC)仍在建造中,它是唯一一个运行的超级对撞机,而计划中的超导超大型加速器在1993年被取消,从未完工。兆电子伏特加速器只能对希格斯质量的进一步范围进行排除,并于2011年9月30日关闭,因为它再也跟不上LHC的进程。数据的最终分析划定了希格斯玻色子质量介于 $147 GeV/c^{2}$ 和 $180 GeV/c^{2}$之间。此外,还有少量(但不是很明显)的事例可能表明希格斯玻色子的质量介于 $115 GeV/c^{2}$ 和 $140 GeV/c^{2}$之间。[63]

位于瑞士欧洲核子中心的大型强子对撞机,是专门设计来确认或排除希格斯玻色子的存在的。日内瓦附近地面下27公里的隧道,最初是LEP的地方,它被设计成用来进行两个质子束碰撞,最初的能量是 3.5 TeV 每束(总计7 TeV),几乎是兆电子伏特加速器的3.6倍,未来可升级到 2 × 7 TeV (共14 TeV)。理论表明,如果希格斯玻色子存在,这些能级的碰撞应该能够找到它存在的痕迹。作为有史以来最复杂的科学仪器之一,它的运行准备被推迟了14个月,原因是首次测试后9天发生了磁体失超事件,导致50多块超导磁体受损,真空系统受到污染。[64][65][66]

LHC的数据收集工作终于在2010年3月开始。[67] 到2011年12月,LHC的两个主要粒子探测器,ATLAS和CMS,已经将希格斯粒子存在的质量范围缩小到大约116-130 GeV (ATLAS)和115-127 GeV (CMS)。[68][69] 此外,还出现了一些有希望的过度事例,这些事例已经“蒸发”,并被证明只是随机波动。然而,从2011年5月左右开始,[70] 两个实验都在他们的结果中看到,缓慢出现少量但一致的额外的γ和4-轻子衰变特征,以及其他几个粒子衰变,所有这些都暗示着在质量125GeV周围有一个新粒子 。[70] 到2011年11月左右,125 GeV的可能性变得“太大而不能忽视”(尽管还远未定论),ATLAS和CMS的团队领导都私下怀疑他们可能发现了希格斯粒子。[70] 2011年11月28日,在两位团队领导和欧洲核子中心总干事的内部会议上,最新的分析第一次在他们的团队之外进行了讨论,这表明ATLAS和CMS可能会在125GeV处达成一个共识,并开始初步准备这一成功的发现。[70] 虽然这一信息当时还不为公众所知,但希格斯玻色子的值可能在115-130 GeV左右,并且在124-126 GeV区域的ATLAS和CMS中多次观察到小而一致的事例过量(被描述为大约2-3 sigma的“诱人暗示”)是公众“非常感兴趣”的知识。[71] 因此,人们普遍预计,大约在2011年底,LHC将提供足够的数据,在2012年底之前排除或确认希格斯玻色子的发现,届时它们的2012年碰撞数据(略高于8 TeV碰撞能量)已经处理过了。[71][72]

\textbf{欧洲核子中心发现候选玻色子}
\begin{figure}[ht]
\centering
\includegraphics[width=6cm]{./figures/cb69982655c4be0e.png}
\caption \label{fig_Higgs_5}
\end{figure}
\begin{figure}[ht]
\centering
\includegraphics[width=6cm]{./figures/88211d13e2b58387.png}
\caption{费曼图显示了由LHC上ATLAS和CMS观测到的与低质量希格斯玻色子候选粒子(约125 GeV)相关的最干净的道。在这个质量下,这种主要产生机制包括从每个质子出来的两个胶子聚变成一个顶夸克圈,顶夸克圈与希格斯场紧密耦合,产生希格斯玻色子 左: 双光子道: 玻色子随后通过与W玻色子圈或顶夸克圈的虚相互作用衰变为2个伽马射线光子。 右: 4-轻子“黄金道”:玻色子发射两个Z玻色子,每个玻色子再衰变到2个轻子(电子, μ介子)。 在两个实验中,对实验数据的分析都超过了 5σ 置信度 [73][74][75]} \label{fig_Higgs_6}
\end{figure}

2012年6月22日,欧洲核子中心宣布了即将举行一个研讨会,[76][77] 不久之后(根据对社交媒体上散布流言的分析,大约从2012年7月1日开始[78])媒体开始散布流言,称这将包括一项重大声明,但尚不清楚这是一个更强的信号还是一个正式的发现。[79][80] 当有报道称提出这种粒子的彼得·希格斯(Peter Higgs)将参加研讨会时,猜测升级为“狂热”的论调,[81][82] “五位顶尖物理学家”被邀请 – 一般认为它象征着1964年活着的五位作者 – 希格斯、恩格尔特、古尔奈尼克、哈根出席,Kibble确认了收到了邀请(布劳特于2011年去世)。[83][84]

2012年7月4日,欧洲核子中心的两个实验宣布他们独立地做出了相同的发现:[85] CMS 发现一个质量为125.3±0.6 GeV/c2的未知玻色子, [86][87] ATLAS发现一个质量为126± 0.6 GeV/c2的未知玻色子。[88][89] 对两个衰变道的联合分析,两个实验在相同的区域独立地达到了5σ的置信度 – 这意味着仅仅偶然获得至少同样强的结果的概率不到300万分之一。当考虑额外的道时,CMS置信度降低到4.9σ。[87]

这两个团队从2011年末到2012年初一直在互相“独立”工作,[70] 这意味着他们没有互相讨论他们的结果,提供了额外的确定性,即任何共同的发现都是对粒子的真实验证。[59] 这一水平的证据,由两个独立的团队和实验独立确认,符合宣布确认发现所需的正式证据水平。

2012年7月31日,ATLAS协作提出了关于"观察新粒子"的额外数据分析,包括来自第三道的数据,这将置信度提高到5.9 sigma(仅通过随机背景效应获得至少同样强有力证据的概率为5.88亿分之一)和质量 126.0 ± 0(stat) ± 0.4 (sys) $GeV/c^{2}$,[89] CMS提高到了5σ和质量 125.3 ± 0.4 (stat) ± 0.5 (sys) $GeV/c^{2}$。[86]

\textbf{新粒子被测试为可能的希格斯玻色子}

在2012年发现之后,质量125 GeV/c2 的粒子是否是希格斯玻色子还没有得到确认。一方面,观测结果与标准模型希格斯玻色子保持一致,粒子衰变进入至少一些预测的道。此外,在实验不确定度范围内,观察到的道的生产率和分支比与标准模型的预测大致匹配。然而,目前实验的不确定度仍为其他解释留下空间,这意味着宣布发现希格斯玻色子还为时过早。 为了收集更多的数据,LHC提议2012年关闭和2013-14年的升级被推迟到2013年的第7周。[90]

2012年11月,在京都的一次会议上,研究人员表示,自7月以来收集的证据与基本标准模型的一致性超过了其它模型,几个相互作用的一系列结果与该理论的预测相匹配。[91] 物理学家马特·斯特拉斯勒(Matt Strassler)强调了“相当多”的证据表明新粒子不是赝标量负宇称粒子(这与希格斯玻色子所需的发现相一致)、“蒸发”或对先前非标准模型发现的缺乏进一步的重要性、预期的标准模型与W及Z玻色子的相互作用、对超对称或反超对称没有“显著的新意义”,以及迄今为止总体上与预期的标准模型希格斯玻色子的结果没有显著的偏差。[92] 然而,标准模型的一些扩展也显示出非常相似的结果;[93] 因此评论者指出,基于在发现之后很长时间仍可以被理解为其他粒子,要想完全了解已经发现的粒子,可能需要数年时间,甚至数十年。[91][92]

这些发现意味着截至2013年1月,科学家们非常确定他们已经发现了一个质量约为125GeV/c2的未知粒子,并且没有被实验的错误或偶然结果所误导。从最初的观察,他们也确信新粒子是某种玻色子。自2012年7月以来,该粒子的行为和性质似乎也非常接近希格斯玻色子的预期行为。即使如此,它仍然可能是希格斯玻色子或其他未知的玻色子,因为未来的实验可能显示出与希格斯玻色子不匹配的行为,所以截至2012年12月,欧洲核子中心仍然只声明新粒子与希格斯玻色子“一致”,[94][94] 科学家还没有肯定地说是希格斯玻色子。[94] 尽管如此,在2012年末,媒体广泛地报道(错误地)宣布一个希格斯玻色子在这一年得到了确认。

2013年1月,欧洲核子中心总干事罗尔夫-迪特尔·霍耶尔(Rolf-Dieter Heuer)表示,根据迄今为止的数据分析,可能在“接近”2013年年中会给出答案,[95] 布鲁克黑文国家实验室物理学副主席在2013年2月表示,对撞机在2015年重启后,“确定的”答案可能需要“再几年”。[96] 2013年3月初,欧洲核子中心研究主任塞尔吉奥·贝尔托卢奇(Sergio Bertolucci)指出,确定自旋为0是今后确定粒子是否至少是某种希格斯玻色子最主要的要求。[97]

\textbf{确认存在和当前状态}

2013年3月14日,欧洲核子中心确认:

CMS和ATLAS对这种粒子的自旋宇称进行了比较,它们都倾向于无自旋和偶宇称(两个符合标准模型的希格斯玻色子基本属性)。再加上新粒子与其他粒子的相互作用,强烈表明它是希格斯玻色子。”  [98]
这也使该粒子成为自然界中第一个被发现的基本标量粒子。[98]

用于验证发现的粒子是希格斯玻色子的实验示例:[92][98]

\begin{table}[h]
\centering
\caption\label{tab_Higgs_1}
\begin{tabular}{|c|p{5cm}|p{5cm}|}
\hline
\textbf{要求} & \textbf{检测或解释途径} & \textbf{研究现状(截至2017年7月)} \\ \hline
自旋为0 & 根据衰变模式。最初观察到的衰变到两个光子(yy)自旋为1已经被排除了,自旋为0和2还未排除。 & 自旋为0已被确认,自旋为2假设被排除,置信度超过99.9\verb|%|。 \\ 
\hline
偶宇称 & 研究衰变产物的角度。如果自旋为0被证实,负宇称也不大可能。 & 偶宇称暂时得到确认。[98][99][100] 自旋为0负宇称假设被排除,置信水平超过99.9\verb|%|。[99][99] \\ 
\hline
衰变道(粒子衰变的结果如预测) & 标准模型预测了125 GeV的希格斯玻色子的衰变情况。这些预期都被观测到了吗?产生率是否与预测的一致?其中最为重要的衰变道,bb, γ γ, τ τ, WW 和 ZZ 都是可能观测到的。 & bb, γ γ, τ τ, WW 和 ZZ 的衰变道都已被观测到。所有观测到的信号强度都符合标准模型的预期。 [102][103] \\ 
\hline
与质量的耦合。(与标准模型粒子的相互作用强度与粒子质量成正比) & 物理学家Adam Falkowski指出希格斯玻色子的基本性质是它是一个自旋为0(标量)的粒子,它也与W和Z玻色子的质量耦合;仅证明自旋为0是不够的。 & 与质量的耦合已被有力得正式(“在95\verb|%|置信水平上$c_{v}$在标准模型$c_{v}=1$的15\verb|%|以内”)。 \\ 
\hline
更高能的实验结果也承受住了考验 & 在2015年LHC重启之后, 能量提高到了 13 TeV, 寻找更多的希格斯玻色子(如其它理论中的预期那样)和针对其他理论模型中的粒子的实验仍在继续。更高能量的实验的结果必须与希格斯理论相一致。 & 对截至到2017年7月的对撞数据结果分析,没有发现不符合标准模型的偏差出现,甚至比低能时,更好地符合预期。.[103]\\ \hline
\end{tabular}
\end{table}

\textbf{自2013年以来的进一步发现}

2017年7月,欧洲核子中心(CERN)确认所有的实验结果仍然与标准模型的预期一致,并直接地将发现的粒子称为“希格斯玻色子”。[103] 截至2019年,大型强子对撞机继续产生的数据结果证实了2013年对希格斯场和粒子的认识。[103][104]

自2015年重启以来,LHC的实验工作包括更详细地探测希格斯场和玻色子,并确认那些不太常见的预测是否正确。特别是,自2015年以来的实验为预测其直接衰变为费米子,如一对底夸克(3.6 σ)提供了强有力的证据,被称为理解其短寿命和其他罕见衰变道的“重要里程碑” – 并也确认衰变到τ轻子对(5.9 σ)的实验。欧洲核子中心将认为“对于建立希格斯玻色子与轻子的耦合至关重要,是测量希格斯玻色子与第三代费米子耦合的重要一步。第三代费米子是电子和夸克的大质量版本,其在自然界中的作用还是一个深奥的谜”。[103] 截至2018年3月19日 ATLAS和CMS的13TeV能量测量的希格斯粒子的质量分别为 124.98±0.28 GeV 和 125.26±0.21 GeV 。

2018年7月,ATLAS和CMS实验报告观测到希格斯玻色子衰变为一对底夸克,约占其衰变总数的60%。[105][106][107]

\subsection{理论基础}

\subsubsection{4.1对于希格斯粒子的理论需求}

\begin{figure}[ht]
\centering
\includegraphics[width=6cm]{./figures/9d6f5c461fcab92a.png}
\caption{“对称破缺图解”:在高能量水平(左)球稳定在中心,结果是对称的。在较低的能量水平(右图),整体的“规则”保持对称,但“墨西哥帽”的势开始发挥作用:“局部”对称不可避免地被打破,因为最终球必须随机地以某种方式滚动。} \label{fig_Higgs_9}
\end{figure}

规范不变性是标准模型等现代粒子理论的重要性质,部分原因是它在电磁学和强相互作用(量子色动力学)等基础物理领域取得了成功。然而,在发展弱核力规范理论或可能的电弱统一相互作用时确实有很大困难。带有质量项的费米子会违反规范对称性,因此不能保持规范不变。(这可以从狄拉克拉格朗日量中的左手和右手的费米子项看出;我们发现没有一个自旋1/2的粒子能够按照质量要求翻转螺旋度,所以它们必须是无质量的。)W及Z玻色子被观测到是有质量的,但是玻色子质量项包含那些明显取决于所选择的规范的项 ,所以这些质量也不能是规范不变的。因此,似乎没有一个标准模型费米子或玻色子能够从以质量作为内在属性作为“开始”,除非放弃规范不变性。如果要保持规范不变性,那么这些粒子必须通过其他机制或相互作用来获得它们的质量。此外,无论赋予这些粒子质量的是什么,都不要“破坏”规范不变性,因为规范不变性是其他理论的基础,并且 不必要求或预测那些在自然界中似乎不存在的无质量粒子或长程作用力(似乎是Goldstone定理的必然结果)。

所有这些重叠问题的解决方案都隐藏在Goldstone定理在数学上一个先前未被注意到的边界情况, 在某些情况下 可能 理论上没有 破坏规范不变性并且也 没有 任何新的无质量粒子或力,但是具有“可感”(可重整化)的数学结果,对称性也被破坏了。这就是众所周知的希格斯机制。

\begin{figure}[ht]
\centering
\includegraphics[width=6cm]{./figures/50a64f18948b68d8.png}
\caption{标准模型描述的某些粒子之间相互作用的概览。} \label{fig_Higgs_10}
\end{figure}

标准模型假设了一个对应这个效应的场,称为希格斯场(符号: $\phi$ ),其在基态具有非零振幅的独特性质;即非零真空期望值。它之所以能产生这种效果,是因为它不寻常的“墨西哥帽”形状的势,其最低点不在它的“中心”。简单地说,与所有其他已知的场不同,这个场的一个关键特征是不像其他已知的场真空期望值在$\phi$为0处,而是在$\phi$不为0处能量更少。 。在某个极高的能量水平下,这种非零真空期望的存在会自发地打破电弱规范对称性,从而产生希格斯机制,并触发与场相互作用的粒子获得质量。这种效应的发生是因为希格斯场的标量场分量作为自由度被有质量的玻色子“吸收”,并通过汤川耦合与费米子耦合,从而产生预期的质量项。当对称性在这些条件下被打破时,Goldstone玻色子就会出现与希格斯场相互作用(以及其他能够与希格斯场相互作用的粒子)的现象而不是成为新的无质量粒子。这两种基础理论的棘手问题是“中和”了彼此,其他的结果是基本粒子根据它们与希格斯场的相互作用强度获得了相应的质量。这是已知的最简单的,能够给规范玻色子质量,同时保持与规范理论兼容的过程。[108] 这个理论的量子是一个标量玻色子,被称为希格斯玻色子。[109]

\subsubsection{4.2希格斯场的性质}

在标准模型中,希格斯场是一个标量快子场 – 标量 意味着它在洛伦兹变换下不变换,而 快子 意味着场(而不是粒子)具有虚质量,并且在某些结构中必须经历对称性破缺。它由四个分量组成:两个中性分量和两个带电分量场。两个带电分量和一个中性场都是Goldstone玻色子,为有质量的W+、W−和Z玻色子纵向第三偏振分量。另外一个中性分量的量子对应于(理论上实现为)有质量的希格斯玻色子,[110] 这个分量也可以通过汤川耦合与费米子相互作用,得到质量。

数学上,希格斯场有虚质量,因此是一个快子场。[111] 虽然快子(移动速度比光快的粒子)是一个纯粹的假设概念,但虚质量场已经在现代物理学中发挥了重要作用。[112][113] 在这种理论中,在任何情况下,任何激发都不会传播得比光快—快子质量的存在与否对信号的最大速度没有任何影响(没有违反因果关系)。[114] 虚质量不是超光速粒子,而是制造了不稳定性:任何一个或多个场激发态为快子都必须自发衰变,并且最终的产物不包含物理上的快子。这个过程被称为快子凝聚,现在被认为是希格斯机制在自然界中如何产生的解释,因此也是电弱对称破缺的原因。

虽然虚质量的概念可能看起来很麻烦,但量子化的只是场,而不是质量本身。因此,在类空分离点的场算符仍然是对易的(或反对易的),并且信息和粒子的传播速度仍然不比光快。[115] 快子凝聚驱动的物理系统已经达到了定域极限 – 可能天真地被认为会产生物理快子 – 到了一个不存在物理快子的另一个稳定状态。一旦像希格斯场这样的快子场达到势的最低点,它的量子不再是快子,而是像希格斯玻色子这样的普通粒子。[116]

\subsubsection{4.3 希格斯玻色子的性质}

由于希格斯场是标量,希格斯玻色子没有自旋。希格斯玻色子也是它自己的反粒子,CP宇称为偶,没有电荷和色荷。[117]

标准模型不能预测希格斯玻色子的质量。[118] 如果质量在115和180 $GeV/c^2$ 之间(与实验观察 125 $GeV/c^2$ 一致),那么标准模型可以在能量尺度上一直到普朗克尺度($10^{19}$ GeV)都是有效的。[119] 许多理论学家基于标准模型不令人满意的性质,期望有在TeV尺度上超越标准模型之外的新物理学出现。[120] 希格斯玻色子(或其他电弱对称破缺机制)允许的最高质量尺度是1.4 TeV。除此之外,标准模型在没有这种机制的情况下会变得不一致,因为在某些散射过程中会违反幺正性。[121]

虽然实验上很困难,但间接估计希格斯玻色子的质量也是可能的。在标准模型中,希格斯玻色子有许多间接效应;最值得注意的是,希格斯圈可以对W及Z玻色子质量的进行微小修正。电弱参数的精确测量,例如费米常数和W/Z玻色子的质量,可以用来计算希格斯玻色子的质量范围。截至2011年7月,精确的电弱测量告诉我们,希格斯玻色子的质量在95\verb|%|的置信水平上可能小于约 161 $GeV/c^2$ (这个上限将增加到 185 $GeV/c^2$ 如果的下限 114.4 $GeV/c^2$ 从LEP-2实验上得到的结果是允许的[122])中。这些间接约束依赖于标准模型是正确的这个假设。如果希格斯玻色子伴有标准模型预测之外的其他粒子,那么在这些质量之上仍然有可能发现希格斯玻色子。[123]

\subsubsection{4.4 产生}

\begin{figure}[ht]
\centering
\includegraphics[width=6cm]{./figures/a77282992b716c36.png}
\caption{希格斯玻色子产生的费曼图} \label{fig_Higgs_11}
\end{figure}

如果希格斯粒子理论是有效的,那么希格斯粒子可以像其他被研究的粒子一样,在粒子对撞机中产生。这包括将大量的粒子加速到极高的能量和非常接近光速,然后让它们一起粉碎。LHC上使用的是质子和铅离子(铅原子的裸核)。在这些碰撞产生的极端能量中,偶尔会产生想要的迷一样的粒子,就可以进一步检测和研究;与理论预期的任何缺失或差异也可以用来改进理论。相关的粒子理论(在这种情况下是标准模型)需要特定的碰撞和不同种类的探测器。标准模型预测希格斯玻色子可以以多种方式形成,[124][125] 虽然在任何碰撞中产生希格斯玻色子的概率总是很小 – 例如,在大型强子对撞机中,每100亿次碰撞只有1个希格斯玻色子产生。 在预期中希格斯玻色子最常见的产生过程是:

\begin{itemize}
\item 胶子聚变。如果碰撞的粒子是强子,如质子或反质子 – LHC和兆电子伏特加速器的情况就是如此 – 那么最有可能的是,将强子束缚在一起的两个胶子发生碰撞。产生希格斯粒子最简单的方法是两个胶子结合形成一个虚夸克圈。由于粒子与希格斯玻色子的耦合与其质量成正比,这一过程更有可能发生在重粒子上。实际上,考虑虚顶夸克和底夸克(最重的夸克)的贡献就足够了。这个过程在LHC和兆电子伏特加速器上是主要贡献,其可能性是其他过程的十倍。[124]
\item Higgs strahlung。如果基本费米子与反费米子碰撞 – 例如夸克和反夸克或电子和正电子 – 它们可以形成一个虚W或Z玻色子,如果它有足够的能量,就可以发射出一个希格斯玻色子。这一过程是在LEP上是主要的生产方式,其中电子和正电子碰撞形成虚Z玻色子,这是兆电子伏特加速器上希格斯玻色子生产的第二大贡献。在LHC上,这一过程仅是第三大过程,因为LHC是将质子与质子碰撞,使得夸克-反夸克碰撞的可能性低于兆电子伏特加速器。Higgs Strahlung也被称为协同产生。[124][125]
\item 弱玻色子融合。另一种可能性是,碰撞的两个(反)费米子交换一个虚W或Z玻色子,发射一个希格斯玻色子。碰撞的费米子不需要是相同的类型。例如,上夸克可以用反下夸克交换Z玻色子。这个过程对于LHC和LEP的希格斯粒子的产生是第二重要的。[125]
\item 顶部融合。通常认为的最终过程是最不可能的(小两个数量级)。这个过程涉及两个碰撞的胶子,每个胶子衰变为一个重夸克-反夸克对。每对正反夸克各出一个夸克或反夸克结合形成希格斯粒子。[124]
\end{itemize}

\subsubsection{4.5 衰变}

\begin{figure}[ht]
\centering
\includegraphics[width=6cm]{./figures/33dce57ee5f4df6c.png}
\caption{标准模型预测的希格斯粒子的衰变宽度取决于它的质量值。} \label{fig_Higgs_12}
\end{figure}

量子力学预测,如果一个粒子有可能衰变为一组更轻的粒子,那么它最终会这样做。[126] 希格斯玻色子也是如此。发生这种情况的可能性取决于多种因素,包括:质量差异、相互作用的强度等。除了希格斯玻色子本身的质量之外,大多数这些因素都是由标准模型确定的。对于质量为 125 $GeV/c^2$ 的粒子标准模型预测平均寿命约为 $1.6\times10^{-22}$ 秒。

\begin{figure}[ht]
\centering
\includegraphics[width=6cm]{./figures/897e77b466e9374b.png}
\caption{标准模型预测的希格斯粒子的不同衰变模式的分支比取决于它的质量值。} \label{fig_Higgs_13}
\end{figure}

希格斯粒子衰变的一种方式是分裂成费米子-反费米子对。一般来说,与轻费米子相比,希格斯粒子更容易衰变为重费米子,因为费米子的质量与其与希格斯粒子相互作用的强度成正比。 根据这一逻辑,最常见的衰变应该是正反顶夸克对。然而,只有希格斯粒子的质量超过~346 $GeV/c^2$顶夸克质量的两倍,这个过程才是可能的。希格斯质量为 125 GeV/c2 标准模型预测最常见的衰变发生在正反底夸克对,发生率为57.7\verb|%|。 在这个质量下第二常见的费米子衰变是正反τ对,只发生了大约6.3\verb|%|的时间。

另一种可能性是希格斯粒子分裂成一对有质量的规范玻色子。希格斯粒子最有可能衰变为一对W玻色子(图中的浅蓝色线),对于质量为 125 $GeV/c^2$的希格斯玻色子发生率为21.5\verb|%|。 W玻色子随后可以衰变为夸克和反夸克,或者衰变为带电轻子和中微子。W玻色子衰变为夸克很难与背景区分开来,衰变为轻子也不能完全重建(因为中微子在粒子碰撞实验中是不可能检测到的)。一个更清晰的信号是通过衰变为一对Z玻色子 ,如果每个玻色子随后衰变为一对易于探测的带电轻子(电子或μ子)给出的(对于质量为 125 $GeV/c^2$的希格斯玻色子发生率为2.6\verb|%|)给出的。

衰变为无质量规范玻色子(即胶子或光子)也是可能的,但需要虚重夸克(顶部或底部)或有质量的规范玻色子的圈作为中间态。 最常见的这种过程是通过虚重夸克圈衰变为一对胶子。这个过程与上面提到的胶子融合过程相反,对于质量为 125 $GeV/c^2$的希格斯玻色子发生率接近8.6\verb|%| 。 更罕见的是通过一对由W玻色子或重夸克的圈衰变到光子,每千次衰变只发生两次这种情况发生。 然而,这一过程与希格斯玻色子的实验研究非常相关,因为光子的能量和动量可以非常精确地测量,从而给出衰变粒子质量的精确重建。

\subsubsection{4.6 其他模型}

如上所述的最小标准模型是希格斯机制已知的最简单的模型,只有一个希格斯场。然而,一个扩展的希格斯场与其他的希格斯粒子二重态或三重态也是可能的,许多标准模型的扩展都具有这一特征。理论上支持的非最小希格斯场是双希格斯双偶模型(2HDM),它预测了五个标量粒子的存在:两个CP宇称为偶数中性希格斯玻色子 $h^0$ 和 $H^0$,CP宇称为奇数中性希格斯玻色子 $A^0$ 和两个带电的希格斯粒子 $H^{\pm}$ 。超对称(SUSY)也预测希格斯玻色子质量和规范玻色子质量之间的关系。 可以包含一个 125 $GeV/c^2$ 的中性希格斯玻色子。

区分这些不同模型的关键方法包括研究粒子的相互作用(“耦合”)和精确的衰变过程(“分支比”),这可以在粒子碰撞中进行测量和实验测试。在I型2HDM模型中,一个希格斯双偶可以与上下夸克耦合,而第二个双偶不行。这个模型有两个有趣的极限,最轻的希格斯粒子只和费米子耦合(“规范恐惧”)或只和规范玻色子耦合(“费米子恐惧”),但不是两者都耦合。在II型2HDM模型中,一个希格斯双偶只与上夸克耦合,另一个只与下夸克耦合。[127] 经过深入研究的最小超对称标准模型(MSSM)包括一个II型2HDM希格斯场,因此它可以被I型2HDM希格斯场的证据推翻。

在其他模型中,希格斯标量是复合粒子。例如,在艺彩理论(technicolor)中,希格斯场的角色是由被艺彩夸克(techniquarks)的强束缚费米子对扮演的。其他模型的特征是顶夸克对(见顶夸克凝聚)。在其他模型中,根本没有希格斯场,电弱对称性被额外维度打破。[128][129]

\subsubsection{4.7 进一步的理论问题和等级问题}

\begin{figure}[ht]
\centering
\includegraphics[width=6cm]{./figures/abd25c38d4547789.png}
\caption{希格斯质量的一阶修正的一个一圈费曼图。在标准模型中,这些修正的影响很可能是巨大的,从而导致所谓的等级问题。} \label{fig_Higgs_14}
\end{figure}

标准模型将希格斯玻色子的质量作为待测量的参数,而不是待计算的值。这在理论上是不令人满意的,特别是因为量子修正(与虚粒子的相互作用相关)显然会导致希格斯粒子的质量远大于观测到的质量,但同时标准模型要求质量在100至1000 GeV的数量级,以确保幺正性(在这种情况下,为幺正性纵向矢量玻色子散射)。[130] 调和这些点似乎需要解释为什么有一个几乎完美的抵消导致大约125 GeV的可见质量,但是不清楚如何做到这一点。因为弱相互作用力大约是引力强度的 $10^{32}$ 倍,希格斯玻色子的质量比普朗克质量或大统一能量小得多,这似乎是因为这些观测结果有一些潜在的联系或原因,而这些联系或原因是未知的,也不是标准模型所描述的,或者是一些无法解释和极其精确的参数微调 – 然而,目前这些解释都没有得到证实。这就是所谓的等级问题。[131] 更广泛地说,等级问题相当于担心未来的基本粒子和相互作用理论不应该有过多的微调或过于微妙的抵消,而应该允许像希格斯玻色子这样的粒子其质量是可计算的。这个问题在某些方面是自旋为0的粒子(如希格斯玻色子)所独有的,这可能会引起与不影响自旋粒子的量子修正相关的问题。[130] 已经提出了许多解决方案,包括超对称、共形解决方案和通过额外维度的解决方案,例如膜宇宙模型。

也有量子平庸性的问题,这表明大约不可能建立一个包含基本标量粒子的自洽的量子场论。[132] 然而,如果量子平庸性被避免,平庸性约束可以在希格斯玻色子质量上设定界限。

\subsection{公开讨论}

\subsubsection{5.1 命名}

\textbf{物理学家使用的名字}

与粒子和场联系最紧密的名字是希格斯玻色子[133] 和希格斯场。一段时间以来,这个粒子是由它的PRL作者的名字组合而成的(有时包括安德森),例如布劳特-恩格尔特-希格斯粒子、安德森-希格斯粒子或恩格尔特-布劳特-希格斯-古尔奈克-哈根-基布尔机制。 这些仍然时常被使用。[133][133] 部分是由承认问题和潜在的共享诺贝尔奖推动的,[133][134] 直到2013年,最合适的名字仍然偶尔成为争论的话题。[133] 希格斯自己更喜欢用所有参与粒子预测人的首字母缩略词来称呼它,或称为“标量玻色子”,或“所谓的希格斯粒子”。[134]

关于希格斯玻色子的名字是如何被专门使用的,已经写了相当多的文章。提供了两种主要解释。 首先,希格斯在他的论文中采取了一个独特的、更清晰的或更明确的步骤来正式预测和检验粒子。 在PRL论文的作者中,只有希格斯的论文 明确地 作有质量玻色子的存在的预测,并计算出它的一些性质;[133][135] 因此,他是“第一个假设有质量粒子存在的人” 《自然》。[133] 物理学家兼作家弗兰克·克洛斯和物理学家兼博客作者彼得·沃伊特都评论说,GHK的论文也是在希格斯和布劳-恩格尔特提交给《物理评论快报》之后完成的。[133][136] 仅仅希格斯粒子就引起了人们对预测有质量的标量 玻色子的关注 ,而其他人所有的注意力都集中在有质量的 矢量 玻色子;[133][136] 这样,希格斯的贡献也为实验者提供了测试理论所需的关键“具体目标”。[137] 然而,在希格斯看来,布劳特和恩格勒没有明确提到玻色子,因为玻色子的存在在他们的工作中是显而易见的,[138] 而根据Guralnik的说法,GHK的论文是对整个对称破缺机制的完整分析,其数学严谨性在其他两篇论文中是没有的,并且在一些解中可能存在有质量粒子。[138] 根据科学历史学家大卫·凯瑟的说法,希格斯的论文还对挑战及其解决方案提供了“特别尖锐”的评述。[134]

另一种解释是,这个名字在20世纪70年代流行是因为它被用作一种方便的速记,或者是因为引用错误。 许多作者(包括希格斯自己[138])把“希格斯”这个名字归功于物理学家本杰明·李 。李在理论的早期阶段是一个重要的民粹主义者,从1972年起,他习惯性地将“希格斯”这个名字作为其组成部分的“方便的简写”[138][133][138][139][140] 至少有一次是早在1966年。[141] 尽管李在脚注中澄清说“‘希格斯’是希格斯、基布尔、古尔尼克、哈根、布劳特、恩格勒特”的缩写,[138] 他对这个术语的使用(也可能是史蒂芬·温伯格在1967年开创性论文中首次错误引用希格斯的论文[133][142][141])意味着在1975-1976年间,其他人也开始专门使用希格斯玻色子这个名字作为速记。

\textbf{绰号}

希格斯玻色子在科学界之外的大众媒体中经常被称为“上帝粒子”。[143][144][145][146][147] 这个昵称来自1993年关于希格斯玻色子和粒子物理学的书, The God Particle: If the Universe Is the Answer, What Is the Question? 作者是诺贝尔物理学奖获得者和费米国家加速器实验室主任利昂·莱德曼。[148] 这本书是莱德曼在美国政府放弃支持超导超大型加速器这个部分建成的泰坦尼克号的背景下写的,[148] [149][150] 大型强子对撞机的竞争对手,计划的碰撞能量为 2 × 20 TeV 这是莱德曼自超导超大型加速器项目1983年成立以来一直倡导的[148][151][152] 并于1993年项目关闭。这本书在一定程度上是为了提高人们对这样一个项目的重要性和必要性的认识,以应对其可能的资金损失。[153] 该领域的主要研究人员莱德曼写道,他想给他的书命名 该死的粒子:如果宇宙是答案,问题是什么? 莱德曼的编辑认为这个标题太有争议,并说服他将标题改为 上帝粒子:如果宇宙是答案,问题是什么?[154]

虽然媒体使用这一术语可能有助于得到更广泛的认识和兴趣,[155] 许多科学家认为这个名字不合适[138][156][156] 因为这是耸人听闻的夸张和误导读者;[157] 粒子也与上帝无关,在基础物理学中留下了许多问题,并且不能解释宇宙的最终起源。据报道,无神论者希格斯对此感到不快,并在2008年的一次采访中表示,他觉得这“令人尴尬”,因为这是“一种滥用”...我认为这可能会冒犯一些人”。[157][158][159] 这个昵称也被主流媒体讽刺了。[160] 科学作家Ian Sample在他2010年的《搜索》一书中指出,物理学家给这个昵称取的是“普遍令人反感”,这可能是物理学史上“最糟糕的嘲弄”,但出版商拒绝了所有提到“希格斯粒子”的书名,因为它们缺乏想象力,而且太不为人知。[161]

莱德曼首先回顾了人类对知识的长期探索,并解释说,他开玩笑地将希格斯场对大爆炸基本对称性的影响与导致和塑造我们当前宇宙的结构、粒子、力和相互作用的明显混乱与圣经中巴别塔的故事进行了类比,在《巴别塔》中,早期创世纪的原始单一语言被分割成许多不同的语言和文化。[162]

如今 ... 我们有了标准模型,它把所有的现实简化到十几个粒子和四个力 ... 这是一个不可思议的简华...非常准确的。 但它也是不完整的,事实上,不自恰... 这个玻色子对今天的物理状态是如此重要,对我们最终理解物质的结构是如此关键,但又是如此难以捉摸,因此我给它起了个绰号:上帝粒子。为什么是上帝粒子?两个原因。第一,出版商不让我们称它为该死的粒子,尽管考虑到它邪恶的本质和它造成的损失,这可能是一个更合适的标题。第二,这和另一本书有某种联系,一本更早的书 ...

— 利昂·莱德曼( Leon M. Lederman) 和 Dick Teresi、上帝粒子:如果宇宙是答案,那么问题是什么 p. 22

莱德曼问希格斯玻色子的加入是否只是为了迷惑和诱惑那些寻求宇宙知识的人,物理学家是否会像故事中叙述的那样被它迷惑,或者最终克服挑战并理解“上帝创造的宇宙是多么美丽”。[163]

\textbf{其他建议}

在2009年英国《卫报》开展的更名比赛上,他们的科学记者选择“香槟酒瓶玻色子”作为最佳意见:所以这不是一个令人尴尬的浮夸的名字,它是令人难忘的,[它]也有一些物理联系。”[164]“希格斯粒子”这个名字也在在物理研究所在线出版物physicsworld.com的一篇评论文章中被提了出来 。[165]

\subsubsection{5.2 教育解释和类比}


关于希格斯粒子的类比和解释,以及该场如何产生质量,[166][167]已经有相当多的公开讨论,包括一些对其本身的解释的报道,以及1993年由时任英国科学大臣威廉·沃尔德格雷夫爵士举办的最佳流行解释竞赛[168] 和世界各地报纸上的文章。

光通过色散棱镜的照片:彩虹效应的产生是因为棱镜的材料对对光子色散的影响程度不同。
一位LHC物理学家和一位 High School Teachers at CERN 教育家建议光的色散 – 彩虹和色散棱镜的原理 – 是希格斯场对称性破缺和质量导致效应的一个有用类比。[169]

\begin{figure}[ht]
\centering
\includegraphics[width=6cm]{./figures/c6147f673ef20786.png}
\caption{光通过色散棱镜的照片:彩虹效应的产生是因为棱镜的材料对对光子色散的影响程度不同。} \label{fig_Higgs_15}
\end{figure}

\begin{table}[ht]
\centering
\caption\label{tab_Higgs}
\begin{tabular}{|c|c}
\hline
 光学中的对称性破缺&在真空中。各种颜色的光(各种波长的光子)都是同样的速度 , 这是一个对称的情形。在一些介质如玻璃、水以及空气中,对称性就被破坏了。这就导致了不同波长的光速度不同 \\
\hline
粒子物理中的对称性破缺&单纯在规范理论中,规范玻色子和其他基本粒子都是物质量的,这也是一个对称的情形。当希格斯场出现的时候,对称性就被破坏了,所以不同类型的粒子获得了不同的质量\\
\hline
\end{tabular}
\end{table}

马特·斯特拉斯勒用电场作为类比:[170]

一些粒子与希格斯场相互作用,而另一些则没有。那些感觉到希格斯场的粒子就好像它们有质量一样。类似的情况也发生在电场中,带电物体被拉来拉去,中性物体可以不受影响地通过。所以你可以把希格斯玻色子的发现看作是在希格斯场(创造希格斯玻色子)中制造波的一种尝试,以证明它确实存在。

《卫报》给出的类似解释:[171]

希格斯玻色子在本质上是一个场的波纹,它在宇宙诞生时就出现了,并一直延续到今天……然而粒子是至关重要的:它是确凿的证据,证明该理论是正确的。

物理学家大卫·米勒著名地将希格斯场对粒子的影响描述为类似于一屋子政党工作人员均匀分布在一个房间里:人群被名人吸引并减缓他们的速度,但不会减缓其他人的速度。 他还提请注意固体物理学中众所周知的效应,在这种效应中,电子的有效质量在晶格存在的情况下可以比通常大得多。

基于阻力效应的类比,包括“糖浆”或“糖蜜”的类比也是众所周知的,但可能会有些误导,因为它们可能被理解为(不正确地)说希格斯场只是抵抗一些粒子的运动,而不是其他粒子的运动 – 简单的电阻效应也可能与牛顿第三定律相冲突。[172]

\subsubsection{5.3 表彰和奖励}
 
在2013年末之前,人们就那些人可以获得更大的荣誉进行了大量讨论,如果希格斯玻色子被证明,就很有可能获得诺贝尔奖,以及有资格获得的人的范围非常广泛。这些人包括使希格斯机制理论成为可能的一系列理论家,1964年PRL论文的理论家(包括希格斯本人),从这些理论推导出有效的电弱理论和标准模型本身的理论家,以及欧洲核子中心(CERN)和其他机构使希格斯场和玻色子在现实中的存在成为可能的实验学家。诺贝尔奖最多只能有3个人分享一个奖项,有一些获奖候选人已经凭借其他工作获得了诺贝尔奖,有一些已经去世(这个奖项只授予在世的人)。与希格斯场、玻色子或机制相关奖项包括:

\begin{itemize}
\item 诺贝尔物理学奖(1979年) – 格拉肖、萨拉姆和温伯格, 对基本粒子之间统一的弱相互作用和电磁相互作用理论的贡献[173]
\item 诺贝尔物理学奖(1999年) – 胡夫特和维尔特曼, 为了阐明物理中电弱相互作用的量子结构[174]
\item J. J. Sakurai 理论粒子物理奖(2010) – 哈根、恩格勒特、古尔尼克、希格斯、布劳特和基布尔, 为了阐明四维相对论规范场论中自发对称破缺的性质和产生矢量玻色子质量的机制[175] (对于上述1964年的论文)
\item 沃尔夫奖(2004) – 恩格勒特、布劳特和希格斯
\item 基础物理学突破奖(2012) – 法比奥拉·吉亚诺提和彼得·詹尼,当时分别是ATLAS和CMS合作的发言人,“对他们来说,他们在欧洲核子中心的大型强子对撞机的ATLAS和CMS合作发现新的类希格斯粒子的科学努力中的领导作用。”[175]
\item 诺贝尔物理学奖(2013年) – 彼得·希格斯(Peter Higgs)和弗朗索瓦·恩格勒(François Englert), 欧洲核子中心的大型强子对撞机的ATLAS和CMS实验从理论上发现了一种机制,这种机制有助于我们理解亚原子粒子的质量起源,并且最近通过发现预测的基本粒子得到了证实[176] 恩格尔特的同事罗伯特·布绕特于2011年去世,诺贝尔奖通常不会在死后授予。[177]
\end{itemize}

此外,《物理评论快报》的50年回顾(2008年)认可了1964年的PRL对称破缺论文和温伯格1967年的论文 轻子模型 (截至2012年,粒子物理学中引用最多的论文)为“里程碑的快报”。[178]

继2012年7月报道观察到希格斯粒子之后,印度几家媒体报道称,认为印度物理学家萨特延德拉·纳特·玻色( Satyendra Nath Bose)的工作被忽视了,因为他在20世纪20年代的工作命名了这类粒子“玻色子”[178][179] (尽管物理学家认为玻色与这一发现的联系并不牢固)。[180]

\subsection{技术方面和数学公式}

在标准模型中,希格斯场是一个四分量标量场,形成弱同位旋SU(2)对称的复二重态:
\begin{table}[h]
\centering
\caption\label{tab_Higgs_2}
\begin{tabular}{|c|c|c|}
\hline
$\Phi = \frac{1}{\sqrt{2}} \left( \begin{array}{c}\phi^1 + i\phi^2 \\\phi^0 + i\phi^3\end{array} \right)$ & & () \\
\hline
\end{tabular}
\end{table}

而在弱超荷U(1)对称性下,场带电荷。

注意: 本文使用常规标度,其中电荷 , Q,弱同位旋, T3, 弱超荷, YW, 与Q = T3 + YW.与其他百科中的文章不同的标度是 $Q = T_3 + \frac{1}{2} Y_W$ .[182] [183] [184]

\begin{figure}[ht]
\centering
\includegraphics[width=6cm]{./figures/81d343b725c4a4d0.png}
\caption{希格斯场的势,以 和 为变量的函数绘制。是一个墨西哥草帽形状的或者像地上放着的香槟瓶。} \label{fig_Higgs_18}
\end{figure}

\begin{table}[h]
\centering
\caption\label{tab_Higgs_3}
\begin{tabular}{|c|c|c|}
\hline
$\mathcal{L}_H = \left| \left( \partial_\mu - ig W_\mu^a \tau^a - \frac{1}{2} ig' B_\mu \right) \phi \right|^2 + \mu^2 \phi^\dagger \phi - \lambda (\phi^\dagger \phi)^2$ & & () \\
\hline
\end{tabular}
\end{table}

其中 $W_\mu^a$ 和 $B_\mu$ 是 SU(2) 和 U(1) 对称的规范玻色子,$g$ 和 $g'$ 是它们各自的耦合常数,$\tau^a = \frac{1}{2} \sigma^a$(其中 $\sigma^a$ 是泡利矩阵)是 SU(2) 对称的一个完全生成元,并且 $\lambda > 0$ 和 $\mu^2 > 0$,所以基态破坏了 SU(2) 对称性。通过 SU(2) 规范变换,希格斯场的基态(势的底部)随着彼此相关的不同基态而退化。总是可以选择一个规范,使得在基态下 $\phi^1 = \phi^2 = \phi^3 = 0$ 的期望值 $\langle \phi \rangle_0$ 在基态(真空期望值或 VEV)是 $\langle \phi \rangle = \frac{1}{\sqrt{2}} v$,其中 $v = \frac{1}{\sqrt{\lambda}}|\mu|$。该参数的测量值为 $\sim 246 \, \text{GeV}/c^2$。它有质量单位,并且是标准模型中唯一不是无量纲的自由参数。式中的二次项 $W_\mu$ 和 $B_\mu$ 的出现,这给了 $W$ 和 $Z$ 玻色子质量:

\begin{table}[h]
\centering
\caption\label{tab_Higgs_4}
\begin{tabular}{|c|c|c|}
\hline
$m_W = \frac{1}{2} v \left| g \right|,$ & & () \\
\hline
\end{tabular}
\end{table}

\begin{table}[h]
\centering
\caption\label{tab_Higgs_5}
\begin{tabular}{|c|c|c|}
\hline
$m_Z = \frac{1}{2} v \sqrt{g^2 + g'^2},$ & & () \\
\hline
\end{tabular}
\end{table}

它们的比例决定了温伯格角,
\begin{equation}
\cos \theta_W = \frac{m_W}{m_Z} = \frac{\left| g \right|}{\\sqrt{g^2 + g'^2}}~
\end{equation}
留下一个无质量的 U(1) 光子 $\gamma$。希格斯玻色子本身的质量为

\begin{table}[h]
\centering
\caption\label{tab_Higgs_6}
\begin{tabular}{|c|c|c|}
\hline
$m_H = \sqrt{2\mu^2} = \sqrt{2\lambda v^2}.$ & & () \\
\hline
\end{tabular}
\end{table}

夸克和轻子通过汤川耦合项与希格斯场相互作用:

\begin{align}
\mathcal{L}_Y = - \lambda_u^{ij} \frac{\phi^0 - i \phi^3}{\sqrt{2}} \overline{u}_L^i u_R^j + \lambda_u^{ij} \frac{\phi^1 - i \phi^2}{\sqrt{2}} \overline{d}_L^i u_R^j\\
- \lambda_d^{ij} \frac{\phi^0 + i \phi^3}{\sqrt{2}} \overline{d}_L^i d_R^j + \lambda_d^{ij} \frac{\phi^1 + i \phi^2}{\sqrt{2}} \overline{u}_L^i d_R^j\\
- \lambda_e^{ij} \frac{\phi^0 + i \phi^3}{\sqrt{2}} \overline{e}_L^i e_R^j - \lambda_e^{ij} \frac{\phi^1 + i \phi^2}{\sqrt{2}} \overline{\nu}_L^i e_R^j + \text{h.c.}~,
\end{align}

其中 $(d, u, e, \nu)^i$ 是左手和右手的夸克和轻子 第一代,$\lambda_u^{ij}$ 是汤川耦合的矩阵,其中 $h.c.$ 表示前面所有项的厄米共轭。在对称破缺状态下,只有包含 $\phi^0$ 留下,产生费米子的质量项。将夸克和轻子场旋转到汤川耦合矩阵是对角矩阵的情况下,可以得到

\begin{table}[h]
\centering
\caption\label{tab_Higgs_7}
\begin{tabular}{|c|c|c|}
\hline
$\mathcal{L}_m = -m_u \bar{u}_L^i u_R^i - m_d \bar{d}_L^i d_R^i - m_e \bar{e}_L^i e_R^i + h.c$., & & () \\
\hline
\end{tabular}
\end{table}

其中 $m_u, m_d, m_e = \frac{1}{\sqrt{2}} \lambda_u v$ 是费米子的质量,$\lambda_u$ 表示汤川矩阵的特征值。

\subsection{笔记}

\begin{enumerate}
\item Note that such events also occur due to other processes. Detection involves a statistically significant excess of such events at specific energies.
\item In the Standard Model, the total decay width of a Higgs boson with a mass of 125 GeV/c2 is predicted to be 4.07×10−3 GeV.[185] The mean lifetime is given by $\tau = \frac{\hbar}{\Gamma}$.
\item It is quite common for a law of physics to hold true only if certain assumptions held true or only under certain conditions. For example, Newton's laws of motion apply only at speeds where relativistic effects are negligible; and laws related to conductivity, gases, and classical physics (as opposed to quantum mechanics) may apply only within certain ranges of size, temperature, pressure, or other conditions.
\item The success of the Higgs-based electroweak theory and Standard Model is illustrated by their predictions of the mass of two particles later detected: the W boson (predicted mass: 80.390 ± 0.018 GeV, experimental measurement: 80.387 ± 0.019 GeV), and the Z boson (predicted mass: 91.1874 ± 0.0021, experimental measurement: 91.1876 ± 0.0021 GeV). The existence of the Z boson was another prediction. Other accurate predictions included the weak neutral current, the gluon, and the top and charm quarks, all later proven to exist as the theory said.
\item Electroweak symmetry is broken by the Higgs field in its lowest energy state, called its ground state. At high energy levels this does not happen, and the gauge bosons of the weak force would therefore be expected to be massless.
\item The range of a force is inversely proportional to the mass of the particles transmitting it.[185] In the Standard Model, forces are carried by virtual particles. The movement and interactions of these particles with each other are limited by the energy–time uncertainty principle. As a result, the more massive a single virtual particle is, the greater its energy, and therefore the shorter the distance it can travel. A particle's mass therefore, determines the maximum distance at which it can interact with other particles and on any force it mediates. By the same token, the reverse is also true: massless and near-massless particles can carry long distance forces. (See also: Compton wavelength and static forces and virtual-particle exchange) Since experiments have shown that the weak force acts over only a very short range, this implies that massive gauge bosons must exist, and indeed, their masses have since been confirmed by measurement.
\item By the 1960s, many had already started to see gauge theories as failing to explain particle physics, because theorists had been unable to solve the mass problem or even explain how gauge theory could provide a solution. So the idea that the Standard Model – which relied on a Higgs field, not yet proved to exist – could be fundamentally incorrect, was not unreasonable. Against this, once the model was developed around 1972, no better theory existed, and its predictions and solutions were so accurate, that it became the preferred theory anyway. It then became crucial to science, to know whether or not it was correct.
\item For example, Huffington Post/Reuters[186] and others[187]
\item The bubble's effects would be expected to propagate across the universe at the speed of light from wherever it occurred. However space is vast – with even the nearest galaxy being over 2 million light years from us, and others being many billions of light years distant, so the effect of such an event would be unlikely to arise here for billions of years after first occurring.[188][188]
\item If the Standard Model is valid, then the particles and forces we observe in our universe exist as they do, because of underlying quantum fields. Quantum fields can have states of differing stability, including 'stable', 'unstable' and 'metastable' states (the latter remain stable unless sufficiently perturbed). If a more stable vacuum state were able to arise, then existing particles and forces would no longer arise as they presently do. Different particles or forces would arise from (and be shaped by) whatever new quantum states arose. The world we know depends upon these particles and forces, so if this happened, everything around us, from subatomic particles to galaxies, and all fundamental forces, would be reconstituted into new fundamental particles and forces and structures. The universe would potentially lose all of its present structures and become inhabited by new ones (depending upon the exact states involved) based upon the same quantum fields.
\item Goldstone's theorem only applies to gauges having manifest Lorentz covariance, a condition that took time to become questioned. But the process of quantisation requires a gauge to be fixed and at this point it becomes possible to choose a gauge such as the 'radiation' gauge which is not invariant over time, so that these problems can be avoided. According to , p. 8:
"the "radiation gauge" condition ∇⋅A( x) = 0 is clearly noncovariant, which means that if we wish to maintain transversality of the photon in all Lorentz frames, the photon field A μ( x) cannot transform like a four-vector. This is no catastrophe, since the photon field is not an observable, and one can readily show that the S-matrix elements, which are observable have covariant structures .... in gauge theories one might arrange things so that one had a symmetry breakdown because of the noninvariance of the vacuum; but, because the Goldstone et al. proof breaks down, the zero mass Goldstone mesons need not appear." [Emphasis in original]
contains an accessible and comprehensive background and review of this area, see external links
\item A field with the "Mexican hat" potential $V(\phi) = \mu^2 \phi^2 + \lambda \phi^4$ and $\mu^2 < 0$ has a minimum not at zero but at some non-zero value $\phi_0$. By expressing the action in terms of the field  $\tilde{\phi} = \phi - \phi_0$  (where $\phi_0$ is a constant independent of position), we find the Yukawa term has a component $g \phi_0 \bar{\psi} \psi$ . Since both g and $\phi_0$are constants, this looks exactly like the mass term for a fermion of mass $g\phi_0$ . The field  $\tilde{\phi}$ is then the Higgs field.
\item The example is based on the production rate at the LHC operating at 7 TeV. The total cross-section for producing a Higgs boson at the LHC is about 10 picobarn,[189] while the total cross-section for a proton–proton collision is 110 millibarn.[189]
\item Just before LEP's shut down, some events that hinted at a Higgs were observed, but it was not judged significant enough to extend its run and delay construction of the LHC.
\item Announced in articles in Time,[190] Forbes,[191] Slate,[192] NPR,[193] and others.[194]
\item In the Standard Model, the mass term arising from the Dirac Lagrangian for any fermion $\psi$ is $-m \bar{\psi} \psi$. This is \textit{not} invariant under the electroweak symmetry, as can be seen by writing $\psi$ in terms of left and right handed components:

$-m \bar{\psi} \psi = -m (\bar{\psi}_L \psi_R + \bar{\psi}_R \psi_L)$\\

i.e., contributions from  $\bar{\psi}_L \psi_L$ and $\bar{\psi}_R \psi_R$ terms do not appear. We see that the mass-generating interaction is achieved by constant flipping of particle chirality. Since the spin-half particles have no right/left helicity pair with the same SU(2) and SU(3) representation and the same weak hypercharge, then assuming these gauge charges are conserved in the vacuum, none of the spin-half particles could ever swap helicity. Therefore, in the absence of some other cause, all fermions must be massless.

\item Other names have included: the "Anderson–Higgs" mechanism,[195] "Higgs–Kibble" mechanism (by Abdus Salam)[196] and "ABEGHHK'tH" mechanism [for Anderson, Brout, Englert, Guralnik, Hagen, Higgs, Kibble and 't Hooft] (by Peter Higgs).[196]
\item Examples of early papers using the term "Higgs boson" include 'A phenomenological profile of the Higgs boson' (Ellis, Gaillard and Nanopoulos, 1976), 'Weak interaction theory and neutral currents' (Bjorken, 1977), and 'Mass of the Higgs boson' (Wienberg, received 1975)
\item In Miller's analogy, the Higgs field is compared to political party workers spread evenly throughout a room. There will be some people (in Miller's example an anonymous person) who pass through the crowd with ease, paralleling the interaction between the field and particles that do not interact with it, such as massless photons. There will be other people (in Miller's example the British prime minister) who would find their progress being continually slowed by the swarm of admirers crowding around, paralleling the interaction for particles that do interact with the field and by doing so, acquire a finite mass.[196][196]
\end{enumerate}

\subsection{参考文献}

\begin{enumerate}
    \item Hill, Christopher T.; Lederman, Leon M. (2013). Beyond the God Particle. Prometheus Books. ISBN 978-1-61614-801-0.
    \item Griffiths 2008, 第49–52页.
    \item Tipler \\& Llewellyn 2003, 第603–604页.
    \item Demystifying the Higgs Boson with Leonard Susskind, Leonard Susskind presents an explanation of what the Higgs mechanism is, and what it means to "give mass to particles." He also explains what's at stake for the future of physics and cosmology. 2012-07-30.
    \item Griffiths 2008, 第372–373页.
    \item Strassler, M. (8 October 2011). "The Known Particles – If The Higgs Field Were Zero". ProfMattStrassler.com. Retrieved 13 November 2012. The Higgs field: so important it merited an entire experimental facility, the Large Hadron Collider, dedicated to understanding it.
    \item Siegfried, T. (20 July 2012). "Higgs Hysteria". Science News. Retrieved 2012-12-09. In terms usually reserved for athletic achievements, news reports described the finding as a monumental milestone in the history of science.
    \item Heilprin, J. (14 March 2013). "Higgs Boson Discovery Confirmed After Physicists Review Large Hadron Collider Data at CERN". The Huffington Post. Archived from the original on 17 March 2013. Retrieved 14 March 2013.
    \item Onyisi, P. (23 October 2012). "Higgs boson FAQ". University of Texas ATLAS group. Retrieved 2013-01-08.
    \item Jammer, Max (2000). Concepts of Mass in Contemporary Physics and Philosophy. Princeton, NJ: Princeton University Press. pp. 162–163., who provides many references in support of this statement.
    \item Dvorsky, George (2013). "Is there a link between the Higgs boson and dark energy?". io9. Retrieved 1 March 2018..
    \item "What Universe Is This, Anyway?". NPR.org (in 英语). 2014. Retrieved 1 March 2018..
    \item M.S. Turner; F. Wilczek (1982). "Is our vacuum metastable?". Nature. 298 (5875): 633–634. Bibcode:1982Natur.298..633T. doi:10.1038/298633a0..
    \item S. Coleman; F. De Luccia (1980). "Gravitational effects on and of vacuum decay". Physical Review. D21 (12): 3305–3315. Bibcode:1980PhRvD..21.3305C. doi:10.1103/PhysRevD.21.3305..
    \item M. Stone (1976). "Lifetime and decay of excited vacuum states". Phys. Rev. D. 14 (12): 3568–3573. Bibcode:1976PhRvD..14.3568S. doi:10.1103/PhysRevD.14.3568..
    \item P.H. Frampton (1976). "Vacuum Instability and Higgs Scalar Mass". Physical Review Letters. 37 (21): 1378–1380. Bibcode:1976PhRvL..37.1378F. doi:10.1103/PhysRevLett.37.1378..
    \item P.H. Frampton (1977). "Consequences of Vacuum Instability in Quantum Field Theory". Phys. Rev. D15 (10): 2922–2928. Bibcode:1977PhRvD..15.2922F. doi:10.1103/PhysRevD.15.2922..
    \item Ellis, J.; Espinosa, J.R.; Giudice, G.F.; Hoecker, A.; Riotto, A. (2009). "The Probable Fate of the Standard Model". Physics Letters B. 679 (4): 369–375. arXiv:0906.0954. Bibcode:2009PhLB..679..369E. doi:10.1016/j.physletb.2009.07.054..
    \item Masina, Isabella (2013-02-12). "Higgs boson and top quark masses as tests of electroweak vacuum stability". Phys. Rev. D. 87 (5): 53001. arXiv:1209.0393. Bibcode:2013PhRvD..87e3001M. doi:10.1103/PhysRevD.87.053001..
    \item Buttazzo, Dario; Degrassi, Giuseppe; Giardino, Pier Paolo; Giudice, Gian F.; Sala, Filippo; Salvio, Alberto; Strumia, Alessandro (2013). "Investigating the near-criticality of the Higgs boson". JHEP. 2013 (12): 089. arXiv:1307.3536. Bibcode:2013JHEP...12..089B. doi:10.1007/JHEP12(2013)089..
    \item Alekhin, S.; Djouadi, A.; Moch, S. (2012-08-13). "The top quark and Higgs boson masses and the stability of the electroweak vacuum". Physics Letters B. 716 (1): 214–219. arXiv:1207.0980. Bibcode:2012PhLB..716..214A. doi:10.1016/j.physletb.2012.08.024..
    \item Salvio, Alberto (2015-04-09). "A Simple Motivated Completion of the Standard Model below the Planck Scale: Axions and Right-Handed Neutrinos". Physics Letters B. 743: 428–434. arXiv:1501.03781. Bibcode:2015PhLB..743..428S. doi:10.1016/j.physletb.2015.03.015..
    \item Bezrukov, F.; Shaposhnikov, M. (2008-01-24). "The Standard Model Higgs boson as the inflaton". Physics Letters B. 659 (3): 703–706. arXiv:0710.3755. Bibcode:2008PhLB..659..703B. doi:10.1016/j.physletb.2007.11.072..
    \item Salvio, Alberto (2013-08-09). "Higgs Inflation at NNLO after the Boson Discovery". Physics Letters B. 727 (1–3): 234–239. arXiv:1308.2244. Bibcode:2013PhLB..727..234S. doi:10.1016/j.physletb.2013.10.042..
    \item Cole, K.C. (2000-12-14). "One Thing Is Perfectly Clear: Nothingness Is Perfect". Los Angeles Times. Retrieved 17 January 2013. [T]he Higgs' influence (or the influence of something like it) could reach much further. For example, something like the Higgs—if not exactly the Higgs itself—may be behind many other unexplained "broken symmetries" in the universe as well ... In fact, something very much like the Higgs may have been behind the collapse of the symmetry that led to the Big Bang, which created the universe. When the forces first began to separate from their primordial sameness—taking on the distinct characters they have today—they released energy in the same way as water releases energy when it turns to ice. Except in this case, the freezing packed enough energy to blow up the universe. ... However it happened, the moral is clear: Only when the perfection shatters can everything else be born..
    \item Higgs Matters – Kathy Sykes, 30 November 2012.
    \item Why the public should care about the Higgs Boson – Jodi Lieberman, American Physical Society (APS).
    \item Matt Strassler's blog – Why the Higgs particle matters 2 July 2012.
    \item Sean Carroll (13 November 2012). The Particle at the End of the Universe: How the Hunt for the Higgs Boson Leads Us to the Edge of a New World. Penguin Group US. ISBN 978-1-101-60970-5..
    \item Woit, Peter (13 November 2010). "The Anderson–Higgs Mechanism". Dr. Peter Woit (Senior Lecturer in Mathematics Columbia University and Ph.D. particle physics). Retrieved 12 November 2012..
    \item Goldstone, J; Salam, Abdus; Weinberg, Steven (1962). "Broken Symmetries". Physical Review. 127 (3): 965–970. Bibcode:1962PhRv..127..965G. doi:10.1103/PhysRev.127.965..
    \item Guralnik, G. S. (2011). "The Beginnings of Spontaneous Symmetry Breaking in Particle Physics". arXiv:1110.2253 [physics.hist-ph]..
    \item Kibble, T. W. B. (2009). "History of Englert–Brout–Higgs–Guralnik–Hagen–Kibble Mechanism (history)". Scholarpedia. 4 (1): 8741. Bibcode:2009SchpJ...4.8741K. doi:10.4249/scholarpedia.8741. Retrieved 2012-11-23..
    \item "The Nobel Prize in Physics 2008". Nobelprize.org. Archived from the original on 2009-01-13..
    \item List of Anderson 1958–1959 papers referencing 'symmetry', at APS Journals[失效连结].
    \item Anderson, P. (1963). "Plasmons, gauge invariance and mass". Physical Review. 130 (1): 439–442. Bibcode:1963PhRv..130..439A. doi:10.1103/PhysRev.130.439..
    \item Klein, A.; Lee, B. (1964). "Does Spontaneous Breakdown of Symmetry Imply Zero-Mass Particles?". Physical Review Letters. 12 (10): 266–268. Bibcode:1964PhRvL..12..266K. doi:10.1103/PhysRevLett.12.266..
    \item Englert, François; Brout, Robert (1964). "Broken Symmetry and the Mass of Gauge Vector Mesons". Physical Review Letters. 13 (9): 321–23. Bibcode:1964PhRvL..13..321E. doi:10.1103/PhysRevLett.13.321..
    \item Higgs, Peter (1964). "Broken Symmetries and the Masses of Gauge Bosons". Physical Review Letters. 13 (16): 508–509. Bibcode:1964PhRvL..13..508H. doi:10.1103/PhysRevLett.13.508..
    \item Guralnik, Gerald; Hagen, C. R.; Kibble, T. W. B. (1964). "Global Conservation Laws and Massless Particles". Physical Review Letters. 13 (20): 585–587. Bibcode:1964PhRvL..13..585G. doi:10.1103/PhysRevLett.13.585..
    \item Higgs, Peter (1964). "Broken symmetries, massless particles and gauge fields". Physics Letters. 12 (2): 132–133. Bibcode:1964PhL....12..132H. doi:10.1016/0031-9163(64)91136-9..
    \item Higgs, Peter (24 November 2010). "My Life as a Boson" (PDF). Talk given by Peter Higgs at Kings College, London, Nov 24 2010. Archived from the original (PDF) on 4 November 2013. Retrieved 17 January 2013. Gilbert ... wrote a response to [Klein and Lee's paper] saying 'No, you cannot do that in a relativistic theory. You cannot have a preferred unit time-like vector like that.' This is where I came in, because the next month was when I responded to Gilbert’s paper by saying 'Yes, you can have such a thing' but only in a gauge theory with a gauge field coupled to the current..
    \item G.S. Guralnik (2011). "Gauge invariance and the Goldstone theorem – 1965 Feldafing talk". Modern Physics Letters A. 26 (19): 1381–1392. arXiv:1107.4592. Bibcode:2011MPLA...26.1381G. doi:10.1142/S0217732311036188..
    \item Higgs, Peter (1966). "Spontaneous Symmetry Breakdown without Massless Bosons". Physical Review. 145 (4): 1156–1163. Bibcode:1966PhRv..145.1156H. doi:10.1103/PhysRev.145.1156..
    \item Kibble, Tom (1967). "Symmetry Breaking in Non-Abelian Gauge Theories". Physical Review. 155 (5): 1554–1561. Bibcode:1967PhRv..155.1554K. doi:10.1103/PhysRev.155.1554..
    \item "Guralnik, G S; Hagen, C R and Kibble, T W B (1967). Broken Symmetries and the Goldstone Theorem. Advances in Physics, vol. 2" (PDF)..
    \item "Physical Review Letters – 50th Anniversary Milestone Papers". Physical Review Letters..
    \item S. Weinberg (1967). "A Model of Leptons". Physical Review Letters. 19 (21): 1264–1266. Bibcode:1967PhRvL..19.1264W. doi:10.1103/PhysRevLett.19.1264..
    \item A. Salam (1968). N. Svartholm, ed. Elementary Particle Physics: Relativistic Groups and Analyticity. Eighth Nobel Symposium. Stockholm: Almquvist and Wiksell. p. 367..
    \item S.L. Glashow (1961). "Partial-symmetries of weak interactions". Nuclear Physics. 22 (4): 579–588. Bibcode:1961NucPh..22..579G. doi:10.1016/0029-5582(61)90469-2..
    \item Ellis, John; Gaillard, Mary K.; Nanopoulos, Dimitri V. (2012). "A Historical Profile of the Higgs Boson". arXiv:1201.6045 [hep-ph]..
    \item Martin Veltman (8 December 1999). "From Weak Interactions to Gravitation" (PDF). The Nobel Prize. p. 391. Archived from the original (PDF) on 2018-07-25. Retrieved 2013-10-09..
    \item Politzer, David (8 December 2004). "The Dilemma of Attribution". The Nobel Prize. Retrieved 22 January 2013. Sidney Coleman published in Science magazine in 1979 a citation search he did documenting that essentially no one paid any attention to Weinberg’s Nobel Prize winning paper until the work of ’t Hooft (as explicated by Ben Lee). In 1971 interest in Weinberg’s paper exploded. I had a parallel personal experience: I took a one-year course on weak interactions from Shelly Glashow in 1970, and he never even mentioned the Weinberg–Salam model or his own contributions..
    \item Coleman, Sidney (1979-12-14). "The 1979 Nobel Prize in Physics". Science. 206 (4424): 1290–1292. Bibcode:1979Sci...206.1290C. doi:10.1126/science.206.4424.1290. PMID 17799637..
    \item Bernstein 1974,第9页.
    \item Bernstein 1974,第9, 36(footnote), 43–44, 47页.
    \item José Luis Lucio; Arnulfo Zepeda (1987). Proceedings of the II Mexican School of Particles and Fields, Cuernavaca-Morelos, 1986. World Scientific. p. 29. ISBN 978-9971504342..
    \item Merali, Zeeya (4 August 2010). "Physicists get political over Higgs". Nature Magazine. Retrieved 28 December 2011..
    \item "Hunt for Higgs boson hits key decision point". MSNBC. 2012-12-06. Retrieved 2013-01-19..
    \item "Welcome to the Worldwide LHC Computing Grid". WLCG – Worldwide LHC Computing Grid. CERN. Retrieved 14 November 2012. [A] global collaboration of more than 170 computing centres in 36 countries … to store, distribute and analyse the ~25 Petabytes (25 million Gigabytes) of data annually generated by the Large Hadron Collider.
    \item "The Worldwide LHC Computing Grid". The Worldwide LHC Computing Grid. CERN. November 2017. It now links thousands of computers and storage systems in over 170 centres across 41 countries. … The WLCG is the world's largest computing grid.
    \item W.-M. Yao; et al. (2006). "Review of Particle Physics" (PDF). Journal of Physics G. 33 (1): 1–1232. arXiv:astro-ph/0601168. Bibcode:2006JPhG...33....1Y. doi:10.1088/0954-3899/33/1/001..
    \item The CDF Collaboration; The D0 Collaboration; The Tevatron New Physics, Higgs Working Group (2012). "Updated Combination of CDF and D0 Searches for Standard Model Higgs Boson Production with up to 10.0 fb−1 of Data". arXiv:1207.0449 [hep-ex]..
    \item "Interim Summary Report on the Analysis of the 19 September 2008 Incident at the LHC" (PDF). CERN. 15 October 2008. EDMS 973073. Retrieved 28 September 2009..
    \item "CERN releases analysis of LHC incident". Media and Press relations (Press release). CERN. 16 October 2008. Retrieved 12 November 2016..
    \item "LHC to restart in 2009". Media and Press relations (Press release). CERN. 5 December 2008. Retrieved 12 November 2016..
    \item "LHC progress report". CERN Bulletin (18). 3 May 2010. Retrieved 7 December 2011..
    \item "ATLAS experiment presents latest Higgs search status". ATLAS homepage. CERN. 13 December 2011. Retrieved 13 December 2011..
    \item Taylor, Lucas (13 December 2011). "CMS search for the Standard Model Higgs Boson in LHC data from 2010 and 2011". CMS public website. CERN. Retrieved 13 December 2011..
    \item Overbye, D. (5 March 2013). "Chasing The Higgs Boson". The New York Times. Retrieved 2013-03-05..
    \item "ATLAS and CMS experiments present Higgs search status" (Press release). CERN Press Office. 13 December 2011. Retrieved 14 September 2012. the statistical significance is not large enough to say anything conclusive. As of today what we see is consistent either with a background fluctuation or with the presence of the boson. Refined analyses and additional data delivered in 2012 by this magnificent machine will definitely give an answer.
    \item "Welcome". WLCG – Worldwide LHC Computing Grid. CERN. Archived from the original on 10 November 2012. Retrieved 29 October 2012..
    \item CMS collaboration (2015). "Precise determination of the mass of the Higgs boson and tests of compatibility of its couplings with the standard model predictions using proton collisions at 7 and 8 TeV". The European Physical Journal C. 75 (5): 212. arXiv:1412.8662. Bibcode:2015EPJC...75..212K. doi:10.1140/epjc/s10052-015-3351-7. PMC 4433454. PMID 25999783..
    \item ATLAS collaboration (2015). "Measurements of Higgs boson production and couplings in the four-lepton channel in pp collisions at center-of-mass energies of 7 and 8 TeV with the ATLAS detector". Physical Review D. 91 (1): 012006. arXiv:1408.5191. Bibcode:2015PhRvD..91a2006A. doi:10.1103/PhysRevD.91.012006..
    \item ATLAS collaboration (2014). "Measurement of Higgs boson production in the diphoton decay channel in pp collisions at center-of-mass energies of 7 and 8 TeV with the ATLAS detector". Physical Review D. 90 (11): 112015. arXiv:1408.7084. Bibcode:2014PhRvD..90k2015A. doi:10.1103/PhysRevD.90.112015..
    \item "Press Conference: Update on the search for the Higgs boson at CERN on 4 July 2012". Indico.cern.ch. 22 June 2012. Retrieved 4 July 2012..
    \item "CERN to give update on Higgs search as curtain raiser to ICHEP conference". Media and Press relations (Press release). CERN. 22 June 2012. Retrieved 12 November 2016..
    \item "Scientists analyse global Twitter gossip around Higgs boson discovery". Phys.org. 23 January 2013. Retrieved 6 February 2013. For the first time scientists have been able to analyse the dynamics of social media on a global scale before, during and after the announcement of a major scientific discovery. De Domenico, M.; Lima, A.; Mougel, P.; Musolesi, M. (2013). "The Anatomy of a Scientific Gossip". Scientific Reports. 3 (2013): 2980. arXiv:1301.2952. Bibcode:2013NatSR...3E2980D. doi:10.1038/srep02980. PMC 3798885. PMID 24135961..
    \item "Higgs boson particle results could be a quantum leap". Times LIVE. 28 June 2012. Retrieved 4 July 2012..
    \item CERN prepares to deliver Higgs particle findings, Australian Broadcasting Corporation. Retrieved 4 July 2012..
    \item "God Particle Finally Discovered? Higgs Boson News At Cern Will Even Feature Scientist It's Named After". Huffingtonpost.co.uk. Retrieved 2013-01-19..
    \item Our Bureau (4 July 2012). "Higgs on way, theories thicken – Wait for news on God particle". The Telegraph – India. Retrieved 2013-01-19..
    \item Thornhill, Ted (2013-07-03). "God Particle Finally Discovered? Higgs Boson News At Cern Will Even Feature Scientist It's Named After". Huffington Post. Retrieved 23 July 2013..
    \item Cooper, Rob (1 July 2012). "God particle is 'found': Scientists at Cern expected to announce on Wednesday Higgs boson particle has been discovered". Daily Mail. London. Retrieved 23 July 2013. Five leading theoretical physicists have been invited to the event on Wednesday – sparking speculation that the particle has been discovered. … [Tom Kibble] told the Sunday Times: 'My guess is that is must be a pretty positive result for them to be asking us out there.'.
    \item Adrian Cho (13 July 2012). "Higgs Boson Makes Its Debut After Decades-Long Search". Science. 337 (6091): 141–143. doi:10.1126/science.337.6091.141. PMID 22798574..
    \item CMS collaboration (2012). "Observation of a new boson at a mass of 125 GeV with the CMS experiment at the LHC". Physics Letters B. 716 (1): 30–61. arXiv:1207.7235. Bibcode:2012PhLB..716...30C. doi:10.1016/j.physletb.2012.08.021..
    \item Taylor, Lucas (4 July 2012). "Observation of a New Particle with a Mass of 125 GeV". CMS Public Website. CERN. Retrieved 4 July 2012..
    \item "Latest Results from ATLAS Higgs Search". ATLAS News. CERN. 4 July 2012. Retrieved 4 July 2012..
    \item ATLAS collaboration (2012). "Observation of a New Particle in the Search for the Standard Model Higgs Boson with the ATLAS Detector at the LHC". Physics Letters B. 716 (1): 1–29. arXiv:1207.7214. Bibcode:2012PhLB..716....1A. doi:10.1016/j.physletb.2012.08.020..
    \item Gillies, James (23 July 2012). "LHC 2012 proton run extended by seven weeks". CERN Bulletin (30). Retrieved 29 August 2012..
    \item "Higgs boson behaving as expected". 3 News NZ. 15 November 2012. Archived from the original on 1 May 2014. Retrieved 15 November 2012..
    \item Strassler, Matt (2012-11-14). "Higgs Results at Kyoto". Of Particular Significance: Conversations About Science with Theoretical Physicist Matt Strassler. Prof. Matt Strassler's personal particle physics website. Retrieved 10 January 2013. ATLAS and CMS only just co-discovered this particle in July ... We will not know after today whether it is a Higgs at all, whether it is a Standard Model Higgs or not, or whether any particular speculative idea...is now excluded. [...] Knowledge about nature does not come easy. We discovered the top quark in 1995, and we are still learning about its properties today... we will still be learning important things about the Higgs during the coming few decades. We’ve no choice but to be patient..
    \item Sample, Ian (14 November 2012). "Higgs particle looks like a bog Standard Model boson, say scientists". The Guardian. London. Retrieved 15 November 2012..
    \item Biever, C. (6 July 2012). "It's a boson! But we need to know if it's the Higgs". New Scientist. Retrieved 2013-01-09. 'As a layman, I would say, I think we have it,' said Rolf-Dieter Heuer, director general of CERN at Wednesday's seminar announcing the results of the search for the Higgs boson. But when pressed by journalists afterwards on what exactly 'it' was, things got more complicated. 'We have discovered a boson – now we have to find out what boson it is' Q: 'If we don't know the new particle is a Higgs, what do we know about it?' We know it is some kind of boson, says Vivek Sharma of CMS [...] Q: 'are the CERN scientists just being too cautious? What would be enough evidence to call it a Higgs boson?' As there could be many different kinds of Higgs bosons, there's no straight answer. [emphasis in original].
    \item John Heilprin (2013-01-27). "CERN chief: Higgs boson quest could wrap up by midyear". NBCNews.com. AP. Retrieved 20 February 2013. Rolf Heuer, director of [CERN], said he is confident that "towards the middle of the year, we will be there." – Interview by AP, at the World Economic Forum, 26 Jan 2013..
    \item Boyle, Alan (2013-02-16). "Will our universe end in a 'big slurp'? Higgs-like particle suggests it might". NBCNews.com. Retrieved 20 February 2013. 'it's going to take another few years' after the collider is restarted to confirm definitively that the newfound particle is the Higgs boson..
    \item Gillies, James (2013-03-06). "A question of spin for the new boson". CERN. Retrieved 7 March 2013..
    \item O'Luanaigh, C. (14 March 2013). "New results indicate that new particle is a Higgs boson". CERN. Retrieved 2013-10-09..
    \item CMS Collaboration (2017). "Constraints on anomalous Higgs boson couplings using production and decay information in the four-lepton final state". Physics Letters B. 775 (2017): 1–24. arXiv:1707.00541. Bibcode:2017PhLB..775....1S. doi:10.1016/j.physletb.2017.10.021..
    \item ATLAS Collaboration (7 October 2013). "Evidence for the spin-0 nature of the Higgs boson using ATLAS data". Phys. Lett. B. 726 (1–3): 120–144. Bibcode:2013PhLB..726..120A. doi:10.1016/j.physletb.2013.08.026. Retrieved 15 September 2014..
    \item Chatrchyan, S.; et al. (CMS collaboration) (2013). "Higgs-like Particle in a Mirror". Physical Review Letters. 110 (8): 081803. arXiv:1212.6639. Bibcode:2013PhRvL.110h1803C. doi:10.1103/PhysRevLett.110.081803. PMID 23473131..
    \item ATLAS; CMS Collaborations (2016). "Measurements of the Higgs boson production and decay rates and constraints on its couplings from a combined ATLAS and CMS analysis of the LHC pp collision data at √s = 7 and 8 TeV". Journal of High Energy Physics. 2016 (8): 45. arXiv:1606.02266. Bibcode:2016JHEP...08..045A. doi:10.1007/JHEP08(2016)045..
    \item "LHC experiments delve deeper into precision". Media and Press relations (Press release). CERN. 11 July 2017. Retrieved 2017-07-23..
    \item "All together now: adding more pieces to the Higgs boson puzzle". ATLAS collaboration. 18 Mar 2019. Retrieved 24 Apr 2019..
    \item "Long-sought decay of Higgs boson observed". Media and Press relations (Press release). CERN. 28 Aug 2018. Retrieved 2018-08-30..
    \item Atlas Collaboration (August 28, 2018). "ATLAS observes elusive Higgs boson decay to a pair of bottom quarks". Atlas (Press release). CERN. Retrieved August 28, 2018..
    \item CMS Collaboration (August 2018). "Observation of Higgs boson decay to bottom quarks". CMS. Retrieved 2018-08-30. CMS Collaboration (24 Aug 2018). "Observation of Higgs boson decay to bottom quarks". CERN Document Server. CERN. Retrieved 2018-08-30. CMS Collaboration (24 Aug 2018). "Observation of Higgs boson decay to bottom quarks". Physical Review Letters. 121 (12): 121801. arXiv:1808.08242. doi:10.1103/PhysRevLett.121.121801..
    \item Peskin & Schroeder 1995,第717–719, 787–791页.
    \item Peskin & Schroeder 1995,第715–716页.
    \item Gunion, John (2000). The Higgs Hunter's Guide (illustrated, reprint ed.). Westview Press. pp. 1–3. ISBN 978-0-7382-0305-8..
    \item Lisa Randall, Warped Passages: Unraveling the Mysteries of the Universe's Hidden Dimensions, p. 286: "People initially thought of tachyons as particles travelling faster than the speed of light ... But we now know that a tachyon indicates an instability in a theory that contains it. Regrettably for science fiction fans, tachyons are not real physical particles that appear in nature.".
    \item Sen, Ashoke (May 2002). "Rolling Tachyon". J. High Energy Phys. 2002 (204): 48. arXiv:hep-th/0203211. Bibcode:2002JHEP...04..048S. doi:10.1088/1126-6708/2002/04/048..
    \item Kutasov, David; Marino, Marcos & Moore, Gregory W. (2000). "Some exact results on tachyon condensation in string field theory". JHEP. 0010 (10): 045. arXiv:hep-th/0009148. Bibcode:2000JHEP...10..045K. doi:10.1088/1126-6708/2000/10/045..
    \item Aharonov, Y.; Komar, A.; Susskind, L. (1969). "Superluminal Behavior, Causality, and Instability". Phys. Rev. 182 (5): 1400–1403. Bibcode:1969PhRv..182.1400A. doi:10.1103/PhysRev.182.1400..
    \item Feinberg, Gerald (1967). "Possibility of Faster-Than-Light Particles". Physical Review. 159 (5): 1089–1105. Bibcode:1967PhRv..159.1089F. doi:10.1103/PhysRev.159.1089..
    \item Peskin & Schroeder 1995.
    \item Flatow, Ira (6 July 2012). "At Long Last, The Higgs Particle... Maybe". NPR. Retrieved 10 July 2012..
    \item "Explanatory Figures for the Higgs Boson Exclusion Plots". ATLAS News. CERN. Retrieved 6 July 2012..
    \item Carena, M.; Grojean, C.; Kado, M.; Sharma, V. (2013). "Status of Higgs boson physics" (PDF). p. 192..
    \item Lykken, Joseph D. (2009). "Beyond the Standard Model". Proceedings of the 2009 European School of High-Energy Physics, Bautzen, Germany, 14 – 27 June 2009. arXiv:1005.1676. Bibcode:2010arXiv1005.1676L..
    \item Plehn, Tilman (2012). Lectures on LHC Physics. Lecture Notes in Physics. 844. Springer. Sec. 1.2.2. arXiv:0910.4182. Bibcode:2012LNP...844.....P. doi:10.1007/978-3-642-24040-9. ISBN 978-3-642-24039-3..
    \item "LEP Electroweak Working Group"..
    \item Peskin, Michael E.; Wells, James D. (2001). "How Can a Heavy Higgs Boson be Consistent with the Precision Electroweak Measurements?". Physical Review D. 64 (9): 093003. arXiv:hep-ph/0101342. Bibcode:2001PhRvD..64i3003P. doi:10.1103/PhysRevD.64.093003..
    \item Baglio, Julien; Djouadi, Abdelhak (2010). "Predictions for Higgs production at the Tevatron and the associated uncertainties". Journal of High Energy Physics. 1010 (10): 063. arXiv:1003.4266. Bibcode:2010JHEP...10..064B. doi:10.1007/JHEP10(2010)064..
    \item Teixeira-Dias (LEP Higgs working group), P. (2008). "Higgs boson searches at LEP". Journal of Physics: Conference Series. 110 (4): 042030. arXiv:0804.4146. Bibcode:2008JPhCS.110d2030T. doi:10.1088/1742-6596/110/4/042030..
    \item Asquith, Lily (22 June 2012). "Why does the Higgs decay?". Life and Physics. London: The Guardian. Retrieved 14 August 2012..
    \item Branco, G. C.; Ferreira, P.M.; Lavoura, L.; Rebelo, M.N.; Sher, Marc; Silva, João P. (July 2012). "Theory and phenomenology of two-Higgs-doublet models". Physics Reports. 516 (1): 1–102. arXiv:1106.0034. Bibcode:2012PhR...516....1B. doi:10.1016/j.physrep.2012.02.002..
    \item Csaki, C.; Grojean, C.; Pilo, L.; Terning, J. (2004). "Towards a realistic model of Higgsless electroweak symmetry breaking". Physical Review Letters. 92 (10): 101802. arXiv:hep-ph/0308038. Bibcode:2004PhRvL..92j1802C. doi:10.1103/PhysRevLett.92.101802. PMID 15089195..
    \item Csaki, C.; Grojean, C.; Pilo, L.; Terning, J.; Terning, John (2004). "Gauge theories on an interval: Unitarity without a Higgs". Physical Review D. 69 (5): 055006. arXiv:hep-ph/0305237. Bibcode:2004PhRvD..69e5006C. doi:10.1103/PhysRevD.69.055006..
    \item "The Hierarchy Problem: why the Higgs has a snowball's chance in hell". Quantum Diaries. 2012-07-01. Retrieved 19 March 2013..
    \item "The Hierarchy Problem | Of Particular Significance". Profmattstrassler.com. Retrieved 2013-10-09..
    \item D. J. E. Callaway (1988). "Triviality Pursuit: Can Elementary Scalar Particles Exist?". Physics Reports. 167 (5): 241–320. Bibcode:1988PhR...167..241C. doi:10.1016/0370-1573(88)90008-7..
    \item Kibble, T. W. B. (2009). "Englert–Brout–Higgs–Guralnik–Hagen–Kibble Mechanism". Scholarpedia. 4 (1): 6441. Bibcode:2009SchpJ...4.6441K. doi:10.4249/scholarpedia.6441. Retrieved 2012-11-23..
    \item Becker, Kate (2012-03-29). "A Higgs by Any Other Name". "NOVA" (PBS) physics. Retrieved 21 January 2013..
    \item "Frequently Asked Questions: The Higgs!". CERN Bulletin (28). Retrieved 18 July 2012..
    \item Woit's physics blog "Not Even Wrong": Anderson on Anderson-Higgs 2013-04-13.
    \item Sample, Ian (2012-07-04). "Higgs boson's many great minds cause a Nobel prize headache". The Guardian. London. Retrieved 23 July 2013..
    \item Sample, Ian (29 May 2009). "Anything but the God particle". The Guardian. Retrieved 2009-06-24..
    \item "Rochester's Hagen Sakurai Prize Announcement" (Press release). University of Rochester. 2010. Archived from the original on 16 April 2008..
    \item C.R. Hagen Sakurai Prize Talk (YouTube). 2010..
    \item Cho, A (2012-09-14). "Particle physics. Why the 'Higgs'?" (PDF). Science. 337 (6100): 1287. doi:10.1126/science.337.6100.1287. PMID 22984044. Archived from the original (PDF) on 2013-07-04. Retrieved 12 February 2013. Lee ... apparently used the term 'Higgs Boson' as early as 1966... but what may have made the term stick is a seminal paper Steven Weinberg...published in 1967...Weinberg acknowledged the mix-up in an essay in the New York Review of Books in May 2012. (See also original article in New York Review of Books[178] and Frank Close's 2011 book The Infinity Puzzle[84]:372 (Book extract) which identified the error).
    \item Weinberg, Steven (2012-05-10). "The Crisis of Big Science". The New York Review of Books. footnote 1. Retrieved 12 February 2013..
    \item Leon Lederman; Dick Teresi (2006). The God Particle: If the Universe Is the Answer, What Is the Question?. Houghton Mifflin Harcourt. ISBN 978-0-547-52462-7..
    \item Kelly Dickerson (September 8, 2014). "Stephen Hawking Says 'God Particle' Could Wipe Out the Universe". livescience.com..
    \item Jim Baggott (2012). Higgs: The invention and discovery of the 'God Particle'. Oxford University Press. ISBN 978-0-19-165003-1..
    \item Scientific American, ed. (2012). The Higgs Boson: Searching for the God Particle. Macmillan. ISBN 978-1-4668-2413-3..
    \item Ted Jaeckel (2007). The God Particle: The Discovery and Modeling of the Ultimate Prime Particle. Universal-Publishers. ISBN 978-1-58112-959-5..
    \item Leon M. Lederman; Dick Teresi (1993). The God Particle: If the Universe is the Answer, What is the Question. Houghton Mifflin Company..
    \item "A Supercompetition For Illinois". Chicago Tribune. 1986-10-31. Retrieved 16 January 2013. The SSC, proposed by the U.S. Department of Energy in 1983, is a mind-bending project ... this gigantic laboratory ... this titanic project.
    \item Diaz, Jesus (2012-12-15). "This Is [The] World's Largest Super Collider That Never Was". Gizmodo. Retrieved 16 January 2013. ...this titanic complex....
    \item Abbott, Charles (June 1987). "Illinois Issues journal, June 1987". p. 18. Lederman, who considers himself an unofficial propagandist for the super collider, said the SSC could reverse the physics brain drain in which bright young physicists have left America to work in Europe and elsewhere..
    \item Kevles, Dan. "Good-bye to the SSC: On the Life and Death of the Superconducting Super Collider" (PDF). California Institute of Technology: "Engineering & Science". 58 no. 2 (Winter 1995): 16–25. Retrieved 16 January 2013. Lederman, one of the principal spokesmen for the SSC, was an accomplished high-energy experimentalist who had made Nobel Prize-winning contributions to the development of the Standard Model during the 1960s (although the prize itself did not come until 1988). He was a fixture at congressional hearings on the collider, an unbridled advocate of its merits..
    \item Calder, Nigel (2005). Magic Universe:A Grand Tour of Modern Science. pp. 369–370. ISBN 978-0-19-162235-9. The possibility that the next big machine would create the Higgs became a carrot to dangle in front of funding agencies and politicians. A prominent American physicist, Leon lederman [sic], advertised the Higgs as The God Particle in the title of a book published in 1993 ...Lederman was involved in a campaign to persuade the US government to continue funding the Superconducting Super Collider... the ink was not dry on Lederman's book before the US Congress decided to write off the billions of dollars already spent.
    \item Lederman, Leon (1993). The God Particle If the Universe Is the Answer, What Is the Question? (PDF). Dell Publishing. Chapter 2, p. 2. ISBN 978-0-385-31211-0. Retrieved 30 July 2015..
    \item Alister McGrath (15 December 2011). "Higgs boson: the particle of faith". The Daily Telegraph. Archived from the original on 2011-12-15. Retrieved 15 December 2011..
    \item Evans, R. (14 December 2011). "The Higgs boson: Why scientists hate that you call it the 'God particle'". National Post. Retrieved 2013-11-03..
    \item Chivers, Tom (2011-12-13). "How the 'God particle' got its name". The Telegraph. London. Retrieved 2012-12-03..
    \item Key scientist sure "God particle" will be found soon Reuters news story. 7 April 2008..
    \item "Interview: the man behind the 'God particle'", New Scientist 13 September 2008, pp. 44–5 (original interview in the Guardian: Father of the 'God Particle', June 30, 2008).
    \item Borowitz, Andy (July 13, 2012). "5 questions for the Higgs boson". The New Yorker..
    \item Sample, Ian (2010). Massive: The Hunt for the God Particle. pp. 148–149 and 278–279. ISBN 978-1-905264-95-7..
    \item Cole, K. (2000-12-14). "One Thing Is Perfectly Clear: Nothingness Is Perfect". Los Angeles Times. p. 'Science File'. Retrieved 17 January 2013. Consider the early universe–a state of pure, perfect nothingness; a formless fog of undifferentiated stuff ... 'perfect symmetry' ... What shattered this primordial perfection? One likely culprit is the so-called Higgs field ... Physicist Leon Lederman compares the way the Higgs operates to the biblical story of Babel [whose citizens] all spoke the same language ... Like God, says Lederman, the Higgs differentiated the perfect sameness, confusing everyone (physicists included) ... [Nobel Prizewinner Richard] Feynman wondered why the universe we live in was so obviously askew ... Perhaps, he speculated, total perfection would have been unacceptable to God. And so, just as God shattered the perfection of Babel, 'God made the laws only nearly symmetrical'.
    \item Lederman, p. 22 et seq: "Something we cannot yet detect and which, one might say, has been put there to test and confuse us ... The issue is whether physicists will be confounded by this puzzle or whether, in contrast to the unhappy Babylonians, we will continue to build the tower and, as Einstein put it, 'know the mind of God'." "And the Lord said, Behold the people are un-confounding my confounding. And the Lord sighed and said, Go to, let us go down, and there give them the God Particle so that they may see how beautiful is the universe I have made"..
    \item Sample, Ian (12 June 2009). "Higgs competition: Crack open the bubbly, the God particle is dead". The Guardian. London. Retrieved 4 May 2010..
    \item Gordon, Fraser (5 July 2012). "Introducing the higgson". physicsworld.com. Retrieved 25 August 2012..
    \item Wolchover, Natalie (2012-07-03). "Higgs Boson Explained: How 'God Particle' Gives Things Mass". Huffington Post. Retrieved 21 January 2013..
    \item Oliver, Laura (2012-07-04). "Higgs boson: how would you explain it to a seven-year-old?". The Guardian. London. Retrieved 21 January 2013..
    \item Zimmer, Ben (2012-07-15). "Higgs boson metaphors as clear as molasses". The Boston Globe. Retrieved 21 January 2013..
    \item "The Higgs particle: an analogy for Physics classroom (section)". www.lhc-closer.es (a collaboration website of LHCb physicist Xabier Vidal and High School Teachers at CERN educator Ramon Manzano). Retrieved 2013-01-09..
    \item Flam, Faye (2012-07-12). "Finally – A Higgs Boson Story Anyone Can Understand". The Philadelphia Inquirer (philly.com). Retrieved 21 January 2013..
    \item Sample, Ian (2011-04-28). "How will we know when the Higgs particle has been detected?". The Guardian. London. Retrieved 21 January 2013..
    \item Goldberg, David (2010-11-17). "What's the Matter with the Higgs Boson?". io9. Archived from the original on 2013-01-21. Retrieved 21 January 2013.CS1 maint: Unfit url (link).
    \item The Nobel Prize in Physics 1979 – official Nobel Prize website..
    \item The Nobel Prize in Physics 1999 – official Nobel Prize website..
    \item American Physical Society – "J. J. Sakurai Prize for Theoretical Particle Physics"..
    \item 2013 Physics – official Nobel Prize website..
    \item Overbye, D. (8 October 2013). "For Nobel, They Can Thank the 'God Particle'". The New York Times. Retrieved 2013-11-03..
    \item Letters from the Past – A PRL Retrospective (50 year celebration, 2008).
    \item Bal, Hartosh Singh (19 September 2012). "The Bose in the Boson". New York Times. Retrieved 21 September 2012..
    \item Alikhan, Anvar (16 July 2012). "The Spark In A Crowded Field". Outlook India. Retrieved 10 July 2012..
    \item Peskin & Schroeder 1995,Chapter 20.
    \item Nakano, T; Nishijima, N (1953). "Charge independence for V-particles". Progress of Theoretical Physics. 10 (5): 581. Bibcode:1953PThPh..10..581N. doi:10.1143/PTP.10.581..
    \item Nishijima, K (1955). "Charge independence theory of V-particles". Progress of Theoretical Physics. 13 (3): 285–304. Bibcode:1955PThPh..13..285N. doi:10.1143/PTP.13.285..
    \item Gell-Mann, M (1956). "The interpretation of the new particles as displaced charged multiplets". Il Nuovo Cimento. 4 (S2): 848–866. Bibcode:1956NCim....4S.848G. doi:10.1007/BF02748000..
    \item "Higgs bosons: theory and searches" (PDF). PDGLive. Particle Data Group. 12 July 2012. Retrieved 15 August 2012..
    \item Irene Klotz (2013-02-18). David Adams; Todd Eastham, eds. "Universe Has Finite Lifespan, Higgs Boson Calculations Suggest". Huffington Post. Reuters. Retrieved 21 February 2013. Earth will likely be long gone before any Higgs boson particles set off an apocalyptic assault on the universe.
    \item Hoffman, Mark (2013-02-19). "Higgs Boson Will Destroy The Universe Eventually". Science World Report. Retrieved 21 February 2013..
    \item Boyle, Alan (2013-02-19). "Will our universe end in a 'big slurp'? Higgs-like particle suggests it might". NBC News' Cosmic log. Retrieved 21 February 2013. [T]he bad news is that its mass suggests the universe will end in a fast-spreading bubble of doom. The good news? It'll probably be tens of billions of years. The article quotes Fermilab's Joseph Lykken: "[T]he parameters for our universe, including the Higgs [and top quark's masses] suggest that we're just at the edge of stability, in a "metastable" state. Physicists have been contemplating such a possibility for more than 30 years. Back in 1982, physicists Michael Turner and Frank Wilczek wrote in Nature that "without warning, a bubble of true vacuum could nucleate somewhere in the universe and move outwards...".
    \item Baglio, Julien; Djouadi, Abdelhak (2011). "Higgs production at the lHC". Journal of High Energy Physics. 1103 (3): 055. arXiv:1012.0530. Bibcode:2011JHEP...03..055B. doi:10.1007/JHEP03(2011)055..
    \item "Person Of The Year 2012". Time. 19 December 2012..
    \item "Higgs Boson Discovery Has Been Confirmed". Forbes. Retrieved 2013-10-09..
    \item Slate Video Staff (2012-09-11). "Higgs Boson Confirmed; CERN Discovery Passes Test". Slate.com. Retrieved 2013-10-09..
    \item "The Year Of The Higgs, And Other Tiny Advances In Science". NPR. 2013-01-01. Retrieved 2013-10-09..
    \item "Confirmed: the Higgs boson does exist". The Sydney Morning Herald. 4 July 2012..
    \item Liu, G. Z.; Cheng, G. (2002). "Extension of the Anderson-Higgs mechanism". Physical Review B. 65 (13): 132513. arXiv:cond-mat/0106070. Bibcode:2002PhRvB..65m2513L. CiteSeerX 10.1.1.242.3601. doi:10.1103/PhysRevB.65.132513..
    \item Close, Frank (2011). The Infinity Puzzle: Quantum Field Theory and the Hunt for an Orderly Universe. Oxford: Oxford University Press. ISBN 978-0-19-959350-7..
\end{enumerate}
