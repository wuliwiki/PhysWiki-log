% 让·勒朗·达朗贝尔(综述)
% license CCBYSA3
% type Wiki

本文根据 CC-BY-SA 协议转载翻译自维基百科\href{https://en.wikipedia.org/wiki/Jean_le_Rond_d\%27Alembert}{相关文章}。

“‘达朗贝尔’重定向至此。其他用法,参见达朗贝尔 (消歧义)。  
不要与德朗布尔混淆。”
\begin{figure}[ht]
\centering
\includegraphics[width=6cm]{./figures/9e2e02aa6a250ffa.png}
\caption{达朗贝尔的粉彩肖像,由莫里斯·昆汀·德·拉图尔创作,1753年。} \label{fig_BR_1}
\end{figure}
让-巴蒂斯特·勒朗·达朗贝尔[a](/ˌdæləmˈbɛər/ DAL-əm-BAIR;[1] 法语:[ʒɑ̃ batist lə ʁɔ̃ dalɑ̃bɛʁ];1717年11月16日-1783年10月29日)是法国数学家、力学家、物理学家、哲学家和音乐理论家。在1759年之前,他与丹尼斯·狄德罗共同担任《百科全书》的编辑。[2] 用于求解波动方程的达朗贝尔公式以他的名字命名。[3][4][5] 波动方程有时也被称为达朗贝尔方程,代数学基本定理在法语中以达朗贝尔命名。
\subsection{早年生活}
达朗贝尔出生于巴黎,是作家克劳丁·盖林·德·坦森和骑士路易-卡缪·德图什(当时任炮兵军官)的私生子。在他出生时,德图什正在国外。出生几天后,他的母亲将他遗弃在巴黎圣让勒朗教堂的台阶上。根据习俗,他以该教堂的守护圣人命名。达朗贝尔被送到一个孤儿院,但他的父亲找到他并将他安置在一位玻璃工的妻子——鲁索夫人家中,达朗贝尔在这里生活了将近50年。[6] 鲁索夫人对他鼓励甚少。当他向她讲述自己的一些发现或所写的东西时,她通常会回复道:

“你永远只会成为一个哲学家——那是什么?不过是一个一生折腾自己、死后才被人谈论的傻瓜罢了。”[7]

德图什暗中资助了让·勒朗的教育,但并不希望他的亲子关系被正式承认。
\subsection{学习与成年生活}

达朗贝尔最初就读于一所私立学校。骑士德图什在1726年去世时,给达朗贝尔留下了每年1200里弗的年金。在德图什家族的影响下,12岁的达朗贝尔进入了詹森派的四国学院(也称为“马扎兰学院”)。在这里,他学习了哲学、法律和艺术,并于1735年获得了艺术学士学位。

在后来的生活中,达朗贝尔轻视他在詹森派那里学到的笛卡尔主义原则,包括“物理推动、先天观念和漩涡论”。詹森派引导达朗贝尔走向教会职业,试图阻止他从事诗歌和数学等方面的兴趣。然而,神学对达朗贝尔来说是“相当空洞的食粮”。他进入了法学院学习了两年,并于1738年被提名为律师。

他对医学和数学也有兴趣。让最初登记的名字是让-巴蒂斯特·达朗贝格(Jean-Baptiste Daremberg),后来他可能出于音韵原因改名为达朗贝尔(d’Alembert)。[8]

后来,为了表彰达朗贝尔的成就,普鲁士的腓特烈大帝曾建议用“达朗贝尔”来命名一颗疑似存在(但实际上不存在)的金星卫星,然而达朗贝尔拒绝了这一荣誉。[9]
\subsection{职业生涯}
\begin{figure}[ht]
\centering
\includegraphics[width=6cm]{./figures/e93464b8ea9b0e1e.png}
\caption{关于流体阻力的新实验} \label{fig_BR_2}
\end{figure}
\textbf{1739年7月,他首次在数学领域做出贡献},在给科学院的一封信中指出了他在《分析学证明》(1708年由夏尔-勒内·雷诺出版)中发现的错误。当时,《分析学证明》是一本标准的数学入门书,达朗贝尔本人曾用它来学习数学的基础。达朗贝尔也精通拉丁文,在他晚年期间致力于塔西佗作品的翻译,得到了包括狄德罗在内的广泛赞誉。
\textbf{1740年},他提交了他的第二篇科学论文,来自流体力学领域的《固体折射的论文》,得到了克莱罗的认可。在这篇论文中,达朗贝尔从理论上解释了折射现象。\\
\textbf{1741年},经过几次失败的尝试后,达朗贝尔被选入科学院。随后他于1746年被选为柏林科学院院士,[10] 并在1748年被选为皇家学会会员。[11]\\
\textbf{1743年},他出版了最著名的著作《动力学论》,在其中发展了他自己的运动定律。[12]\\
\textbf{在1740年代末期《百科全书》组织编纂时},达朗贝尔被聘为与狄德罗共同担任编辑(负责数学和科学部分),一直工作到1757年因一系列危机导致出版暂时中断为止。他为《百科全书》撰写了超过一千篇文章,包括著名的《前言》。达朗贝尔在“怀疑我们认为所见之物在我们之外是否确实存在”时,“放弃了唯物主义的基础”。[13] 通过这种方式,达朗贝尔与唯心主义者贝克莱的观点一致,并预示了康德的先验唯心主义。[需要引用]\\
\textbf{1752年},他描述了如今被称为“达朗贝尔悖论”的现象,即在无粘、不可压缩流体中浸入的物体所受的阻力为零。\\
\textbf{1754年},达朗贝尔被选为法国科学院的成员,并于1772年4月9日成为该院的永久秘书。[14]\\
\textbf{1757年},达朗贝尔在《百科全书》第七卷中的一篇文章中指出,日内瓦的神职人员已从加尔文主义转向纯粹的苏西尼主义,依据是伏尔泰提供的信息。日内瓦的牧师们对此极为愤慨,成立了一个委员会来回应这些指控。在雅各布·韦尔内、让-雅克·卢梭等人的压力下,达朗贝尔最终解释说,他认为不接受罗马教会的人都可以被称为苏西尼主义者,这就是他想表达的全部意思。在回应批评后,他不再参与《百科全书》的编辑工作。[15]\\
\textbf{1781年},他被选为美国艺术与科学学院的外国荣誉会员。[16]
\begin{figure}[ht]
\centering
\includegraphics[width=6cm]{./figures/cf11a63bf625c9d2.png}
\caption{1758年版《动力学论》封面} \label{fig_BR_3}
\end{figure}
\subsection{音乐理论}
达朗贝尔首次接触音乐理论是在1749年,当时他被要求审查让-菲利普·拉莫提交给科学院的一份《备忘录》。这篇文章是他与狄德罗合作撰写的,后来成为拉莫1750年《和声原理论证》的基础。达朗贝尔在评论中高度评价了拉莫的演绎逻辑,将其视为理想的科学模型。他在拉莫的音乐理论中看到了对自己科学思想的支持:一种具有强演绎综合结构的完整系统方法。

两年后的1752年,达朗贝尔在《根据拉莫先生的原理的理论与实践音乐要素》一书中试图对拉莫的作品进行全面评述。[17] 他强调了拉莫的主要观点,即音乐是一门数学科学,能够从一个单一原理推导出所有音乐实践的要素和规则,以及其中所使用的明确的笛卡尔方法。达朗贝尔帮助推广了这位作曲家的作品,同时宣传了自己的理论。[17] 他声称自己“澄清、发展和简化”了拉莫的原理,并认为单一的“共鸣体”(corps sonore)的概念不足以涵盖音乐的全部。[18] 达朗贝尔主张需要三个原则来生成大调、调式和八度音的识别。然而,由于他并非音乐家,他误解了拉莫思想中的一些细微之处,改变和删除了那些无法完美契合他对音乐理解的概念。

虽然最初感激,拉莫最终转而反对达朗贝尔,并表达了他对让-雅克·卢梭在《百科全书》中关于音乐的文章日益不满的声音。[19] 这引发了两人之间的一系列激烈交锋,并导致达朗贝尔与拉莫的友谊破裂。达朗贝尔在1762年版《音乐要素》中撰写了一篇长篇前言,试图总结这场争论,并作为最终的反驳。

达朗贝尔还在狄德罗的《百科全书》著名的《前言》中讨论了音乐状态的各个方面。达朗贝尔声称,相较于其他艺术,音乐“同时诉诸于想象和感官”,但由于“从事音乐的人缺乏足够的创造力和机智”,音乐在表现和模仿现实方面并不如其他艺术丰富。[20] 他希望音乐表达能够处理所有的身体感官,而不仅仅是激情。达朗贝尔认为现代(巴洛克)音乐在他的时代已经达到了完美,因为没有古希腊的经典范例可供研究和模仿。他声称,“时间摧毁了古人可能留给我们的所有此类作品的范例。”[21] 他赞扬拉莫是“那位坚毅、勇敢且多产的天才”,弥补了让-巴蒂斯特·吕利在法国音乐艺术中留下的空白。[22]
\subsection{个人生活}
达朗贝尔曾活跃于多个巴黎沙龙,尤其是玛丽·特雷兹·罗德·乔弗兰、德芳侯爵夫人和朱莉·德·莱斯皮纳斯的沙龙。他深深迷恋上朱莉·德·莱斯皮纳斯,最终搬去与她同住。
\begin{figure}[ht]
\centering
\includegraphics[width=8cm]{./figures/47eb60ab580628c1.png}
\caption{让·勒朗·达朗贝尔的肖像,1777年,作者为凯瑟琳·卢苏里耶。} \label{fig_BR_4}
\end{figure}
\subsection{去世}
他多年来健康不佳,去世是由于膀胱疾病的缘故。作为一位公认的无神论者,[23][24][25] 达朗贝尔被安葬在一个没有标记的公共墓地。
\subsection{遗产}

在法国,代数学基本定理被称为达朗贝尔/高斯定理,因为高斯纠正了达朗贝尔证明中的一个错误。

他还创造了比值判别法,用于判断级数是否收敛。

达朗贝尔算符首次出现在他对振动弦的分析中,并在现代理论物理中具有重要作用。

尽管他在数学和物理学方面取得了重大进展,达朗贝尔也因在《正反面》一文中错误地论证硬币每次出现反面后正面概率会增加而闻名。在赌博中,这种在赢得越多时减少赌注、输得越多时增加赌注的策略被称为“达朗贝尔系统”,一种类型的马丁格尔策略。

在南澳大利亚,法国探险家尼古拉·博丹在前往新荷兰的探险中,将斯宾塞湾西南部的一个小岛命名为“达朗贝尔岛”(Ile d'Alembert)。该岛更为人熟知的英文名称是“利普森岛”,目前是一个保护区和海鸟栖息地。
\subsection{虚构描绘}
狄德罗在《达朗贝尔的梦》中描绘了达朗贝尔,这部作品写于两人疏远之后。书中描写了生病卧床的达朗贝尔在梦中进行关于唯物主义哲学的辩论。

1996年,安德鲁·克鲁米的小说《达朗贝尔原理》以物理学中的达朗贝尔原理为题。小说的第一部分描述了达朗贝尔的生活以及他对朱莉·德·莱斯皮纳斯的迷恋。
\subsection{作品列表}
\begin{itemize}
\item 达朗贝尔, 让·勒朗 (1743). 《动力学论》 (第2版). Gabay (1990年重印).
\item 达朗贝尔, 让·勒朗 (1747a). “关于振动时张紧弦形成的曲线的研究”. 《柏林皇家科学院与美术学院学报》. 第3卷. 第214–219页.
\item 达朗贝尔, 让·勒朗 (1747b). “关于振动时张紧弦形成的曲线的进一步研究”. 《柏林皇家科学院与美术学院学报》. 第3卷. 第220–249页.
\item 达朗贝尔, 让·勒朗 (1750). “对关于振动时张紧弦形成的曲线的论文的补充”. 《柏林皇家科学院与美术学院学报》. 第6卷. 第355–360页.
\item 《关于世界体系中若干重要问题的研究》 (法语). 第1卷. 巴黎: 米歇尔·安托万·大卫出版社. 1754.
\item 《关于世界体系中若干重要问题的研究》 (法语). 第2卷. 巴黎: 米歇尔·安托万·大卫出版社. 1754.
\item 《关于世界体系中若干重要问题的研究》 (法语). 第3卷. 巴黎: 米歇尔·安托万·大卫出版社. 1756.
\item 达朗贝尔, 让·勒朗 (1995). 《百科全书前言》. 翻译:施瓦布, 理查德 N.; 雷克斯, 沃尔特 E. 芝加哥大学出版社.
\item 《动力学论》 (法语). 巴黎: 让-巴蒂斯特·科尼亚尔出版社 (第3版). 1743.
\item 《积分计算备忘录》 (1739), 初版.
\item 《流体的平衡与运动论》 (1744).
\item 《关于风的普遍原因的反思》 (1746).
\item 《关于振动弦的研究》 (1747).
\item 《关于岁差和地轴在牛顿体系中的变化的研究》. 巴黎: 让·巴蒂斯特·科尼亚尔出版社. 1749.
\item 《音乐理论与实践要素》. 里昂: Jombert, Charles Antoine; Bruyset, Jean-Marie (第1版). 1759.
\item 《流体阻力新理论试验》[永久失效链接] (1752).
\item 《哲学要素试论》 (1759).
\item 《关于流体阻力的新实验》 (法语). 巴黎: 让·弗朗索瓦·路易斯·查尔顿出版社. 1777.
\item 《在法兰西学院公众会议上宣读的颂词》 (1779).
\item 《数学小论文》[永久失效链接] (共8卷 1761-1780).
\item 《全集》, 法国国家科学研究中心出版社, 2002. ISBN 2-271-06013-3.
\item 《百科全书或科学、艺术与工艺的合理词典》, 弗拉马里翁出版社, 1993. ISBN 2-08-070426-5.
\item 《关于流体阻力的新实验》, 合著:达朗贝尔 ... 和博苏神父 ... 巴黎: Claude-Antoine Jombert出版社, 国王的工程与炮兵图书商. 1777.
\item 《文学、哲学与历史杂集》. 伦敦: 为C. Henderson出版;T. Becket 和 P. A. De Hondt代理销售, Strand街. 1764.
\item [Opere] (in French). Vol. 1. Paris: A. Belin. 1821.
\item [Opere] (in French). Vol. 2. Paris: A. Belin. 1821.
\item [Opere] (in French). Vol. 3. Paris: A. Belin. 1821.
\item [Opere] (in French). Vol. 4. Paris: A. Belin. 1822.
\item [Opere] (in French). Vol. 5. Paris: A. Belin. 1822.
Oeuvres et correspondances inedites (in French). Paris: Librairie Académique Didier. 1887.
\end{itemize}
\subsection{另见}
\begin{itemize}
\item 自由主义理论家列表
\item 以让·达朗贝尔命名的事物列表
\end{itemize}
\subsection{注释}
\begin{itemize}
\item 在英语中也可写作 D'Alembert
\end{itemize}
\subsection{参考文献}
\begin{enumerate}
\item  "Alembert, d'". 《兰登豪斯韦氏未删节词典》.
\item  "Jean Le Rond d'Alembert | French mathematician and philosopher". 《大英百科全书》. 检索于2021年6月26日.
\item  D'Alembert 1747a.
\item  D'Alembert 1747b.
\item  D'Alembert 1750.
\item  Hall 1906, 第5页.
\item  《实用知识国家百科全书》, 第一卷, 伦敦, Charles Knight, 1847年, 第417页.
\item  "Jean Le Rond d'Alembert | French mathematician and philosopher". 《大英百科全书》. 检索于2023年4月10日.
\item  Ley, Willy. 1952. "金星的卫星"一文, 《银河科幻小说》1952年7月刊. MDP出版, 《银河科幻小说》数字系列, 2016年. 来源于Google Books.
\item  Hankins 1990, 第26页.
\item  "图书馆与档案目录". 皇家学会. 2020年4月10日存档. 检索于2010年12月3日.
\item  D'Alembert 1743.
\item  Friedrich Albert Lange, 《唯物主义历史及其当代重要性批判》, "康德与唯物主义".
\item  \textbf{让·勒朗,号达朗贝尔} (1717-1783),永久秘书 www.academie-francaise.fr/immortels 2012年5月31日存档于Wayback Machine.
\item  Smith Richardson 1858, 第8-9页.
\item  "会员名册,1780-2010:章节A" (PDF). 美国艺术与科学学院. 检索于2011年4月14日.
\item  Christensen 1989, 第415页.
\item  Bernard 1980.
\item  《新格罗夫音乐与音乐家词典》, 第2版, 条目 "Alembert, Jean le Rond d'".
\item  D'Alembert 1995, 第38页.
\item  D'Alembert 1995, 第69页.
\item  D'Alembert 1995, 第100页.
\item  Israel 2011, 第115页:“达朗贝尔虽私下是无神论者和唯物主义者,但在巴黎呈现出‘哲学’的体面公众形象,同时始终与伏尔泰保持一致。”
\item  Force & Popkin 1990, 第167页:“不同于法国和英国的自然神论者,以及像狄德罗、达朗贝尔和霍尔巴赫这样的科学无神论者,英国科学家如大卫·哈特利和约瑟夫·普里斯特利将他们的科学理论视为其圣经观点的证据。”
\item  Horowitz 1999, 第52-53页:“在积极的理论上,伏尔泰的泛心论自然神论与狄德罗的生理唯物主义、达朗贝尔的不可知论实证主义和赫尔维修的社会唯物主义之间存在显著差异。”
\end{enumerate}
\subsection{来源}
\begin{itemize}
\item Bernard, Jonathan W. (1980). “原则与要素:拉莫与达朗贝尔的争论”。《音乐理论杂志》,第24卷(1):37–62页。JSTOR 843738。
\item Briggs, J. Morton (1970). “让·勒朗·达朗贝尔”。《科学传记词典》。第一卷。纽约:查尔斯·斯克里布纳之子公司,第110–117页。ISBN 0-684-10114-9。
\item Christensen, Thomas (1989). “作为科学宣传的音乐理论:达朗贝尔《音乐要素》案例”。《思想史杂志》,第50卷(3):409–27页。doi:10.2307/2709569。JSTOR 2709569。
\item Crépel, Pierre (2005). “《动力学论》”。收录于 Grattan-Guinness, I.(编辑)。《西方数学经典著作》,Elsevier出版社,第159–67页。ISBN 9780444508713。
\item Elsberry, Kristie Beverly (1984). 《根据拉莫先生的原则的理论与实践音乐要素》:新译本并与拉莫理论著作的比较(博士论文)。佛罗里达州立大学。
\item Force, James E.; Popkin, Richard Henry (1990). 《艾萨克·牛顿神学的背景、性质及影响》。斯普林格出版社。ISBN 9780792305835。
\item Grimsley, Ronald (1963). 《达朗贝尔传》。牛津大学出版社。
\item Hall, Evelyn Beatrice (1906). 《伏尔泰之友》。Smith, Elder & Co.出版社。
\item Hankins, Thomas L. (1990). 《达朗贝尔:科学与启蒙》。纽约:戈登与布里奇出版社。ISBN 978-2-88124-399-8。
\item Horowitz, Irving Louis (1999). 《巨兽:政治社会学历史与理论的主流》。Transaction出版社。ISBN 9781412817929。
\item Israel, Jonathan (2011). 《民主启蒙:哲学、革命与人权 1750–1790》。牛津大学出版社。ISBN 978-0-19-954820-0。
\item Smith Richardson, Nathaniel (1858). “伏尔泰与日内瓦”。《教会评论》,第10卷:1–14页。
\end{itemize}
