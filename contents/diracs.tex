% 狄拉克方程的自由粒子解
% keys 狄拉克方程|解|自由粒子
% license Xiao
% type Tutor

现在我们来讨论狄拉克方程的平面波解。解的形式如下
\begin{equation}
\psi(x) = u(p)e^{-ip\cdot x}~, \quad p^2 = m^2~.
\end{equation}
我们现在主要考虑正频率的解,也就是 $p^0>0$ 的解。我们把上式代入狄拉克方程中,得
\begin{equation}\label{eq_diracs_1}
(\gamma^\mu p_\mu - m) u(p) = 0~.
\end{equation}
我们可以先在静止系中分析这个方程,其中 $p=p_0=(m,\boldsymbol 0 )$. 任意动量 $p$ 的解可以通过boost $\Lambda_{\frac{1}{2}}$ 来得到。在静止系中,\autoref{eq_diracs_1} 式变为
\begin{equation}
(m\gamma^0-m)u(p_0) = m\begin{pmatrix}
-1 & 1 \\
 1 & -1
\end{pmatrix}u(p_0) = 0 ~.
\end{equation}
解为
\begin{equation}
u(p_0) = \sqrt{m} \begin{pmatrix}
\xi \\ \xi
\end{pmatrix}~.
\end{equation}
其中 $\xi$ 是任意的两分量的旋量。我们取归一化条件 $\xi^\dagger \xi =1$. $\xi$ 在旋转生成元的作用下按照普通的两分量旋量进行变换。$\xi = \begin{pmatrix}
1 \\0
\end{pmatrix}$ 时,粒子在3-方向具有朝上的自旋。

使用狄拉克方程过后,我们可以选择 $u(p)$ 中四个分量的其中两个分量。因为自旋为 $1/2$ 的粒子只有两个物理态-自旋向上和向下。

现在我们来对静止系的 $u(p)$ 来进行一个boost。我们考虑沿着3-轴方向的boost。首先我们要知道boost对4-动量的矢量有何作用。考虑无穷小的boost
\begin{equation}
\begin{pmatrix}
E \\ p^3
\end{pmatrix} = \bigg[ 1 + \eta \begin{pmatrix}
0 & 1 \\
1 & 0 
\end{pmatrix} \bigg] \begin{pmatrix}
m \\ 0
\end{pmatrix}~.
\end{equation}
其中 $\eta$ 是无穷小的参数。对于有限的 $\eta$,我们可以写
\begin{equation}
\begin{aligned}\nonumber
\begin{pmatrix}
E \\ p^3
\end{pmatrix} & = \exp \bigg[ \eta \begin{pmatrix}
0 & 1 \\
1& 0
\end{pmatrix} \bigg]\begin{pmatrix}
m \\ 0
\end{pmatrix} \\
& = \bigg[ \cosh \eta \begin{pmatrix}
1 & 0 \\ 0 & 1
\end{pmatrix} + \sinh \eta \begin{pmatrix}
0 & 1 \\
1 & 0
\end{pmatrix}\bigg]\begin{pmatrix}
m \\ 0
\end{pmatrix} = \begin{pmatrix}
m \cosh \eta \\ m \sinh \eta
\end{pmatrix}~.
\end{aligned}
\end{equation}
$\eta$ 参数被称为\textbf{快度}。\textbf{在连续的boost变换下,快度是相加的}。
\begin{exercise}{对 $u(p_0)$ 进行boost变换得到 $u(p)$}
\begin{equation}
\begin{aligned}\nonumber
u(p) & = \exp \bigg[ - \frac{1}{2} \eta \begin{pmatrix}
\sigma^3 & 0 \\ 
0 & - \sigma^3
\end{pmatrix} \bigg] \sqrt{m} \begin{pmatrix}
\xi \\ \xi
\end{pmatrix} \\ \nonumber
& = \bigg[ \cosh\big(\frac{1}{2}\eta\big)\begin{pmatrix}
1 & 0 \\
0 & 1
\end{pmatrix} - \sinh (\frac{1}{2}\eta) \begin{pmatrix}
\sigma^3 & 0 \\ \nonumber
0 & - \sigma^3
\end{pmatrix} \bigg]\sqrt{m} \begin{pmatrix}
\xi \\ \xi
\end{pmatrix} \\ \nonumber
& = \begin{pmatrix}
e^{\eta/2}\bigg(\frac{1-\sigma^3}{2}\bigg)+e^{-\eta/2}\bigg(\frac{1+\sigma^3}{2}\bigg) & 0\\
0 & e^{\eta/2}\bigg(\frac{1+\sigma^3}{2}\bigg)+e^{-\eta/2}\bigg(\frac{1-\sigma^3}{2}\bigg)
\end{pmatrix}\sqrt{m}\begin{pmatrix}
\xi \\ 
\xi
\end{pmatrix} \\\nonumber
& = \begin{pmatrix}
[\sqrt{E+p^3}\bigg(\frac{1-\sigma^3}{2}\bigg)+\sqrt{E-p^3}\bigg(\frac{1+\sigma^3}{2}\bigg)]\xi \\
[\sqrt{E+p^3}\bigg(\frac{1+\sigma^3}{2}\bigg)+\sqrt{E-p^3}\bigg(\frac{1-\sigma^3}{2}\bigg)]\xi
\end{pmatrix}~.
\end{aligned}
\end{equation}
最后一行可以化简为
\begin{equation}
u(p) = \begin{pmatrix}
\sqrt{p\cdot\sigma}\xi \\
\sqrt{p\cdot\bar\sigma}\xi
\end{pmatrix}~.
\end{equation}
\end{exercise}
这里开根号我们理解为取每个特征值的正根。下面这个式子非常有用
\begin{equation}
(p\cdot\sigma)(p\cdot \bar\sigma) = p^2 = m^2~.
\end{equation}
一般来说,我们可以考虑具体的旋量 $\xi$。一种比较方便的取法是取 $\sigma^3$ 的本征态。如果 $\xi=\begin{pmatrix}
1 \\ 0
\end{pmatrix}$, 我们可以得到
\begin{equation}\label{eq_diracs_2}
u(p) = \begin{pmatrix}
\sqrt{E-p^3}\begin{pmatrix}
1 \\ 0 
\end{pmatrix} \\
\sqrt{E+p^3}\begin{pmatrix}
1 \\ 0
\end{pmatrix}
\end{pmatrix} \rightarrow \sqrt{2E}\begin{pmatrix}
0  \\ 
\begin{pmatrix}
1 \\ 0
\end{pmatrix}
\end{pmatrix}~.
\end{equation}
对于 $\xi = \begin{pmatrix}
0 \\ 1
\end{pmatrix}$, 
我们有
\begin{equation}\label{eq_diracs_3}
u(p) = \begin{pmatrix}
\sqrt{E+p^3}\begin{pmatrix}
0 \\ 1 
\end{pmatrix} \\
\sqrt{E-p^3}\begin{pmatrix}
0 \\ 1
\end{pmatrix}
\end{pmatrix} \rightarrow \sqrt{2E}\begin{pmatrix}
\begin{pmatrix}
0 \\ 1
\end{pmatrix}  \\ 
0
\end{pmatrix}~.
\end{equation}
在 $\eta\rightarrow \infty$ 极限下,这个态跟一个无质量粒子的两分量的旋量是简并的。

\autoref{eq_diracs_2} 和\autoref{eq_diracs_3} 是helicity算符的本征态
\begin{equation}
h\equiv\hat p \cdot \mathbf S = \frac{1}{2} \hat p_i \begin{pmatrix}
\sigma^i & 0 \\
 0 & \sigma^i
\end{pmatrix}~.
\end{equation}
一个 $h=+1/2$ 的粒子被称作右手的粒子。一个 $h=-1/2$ 的粒子被称作左手的粒子。\textbf{有质量粒子的手性是依赖于参考系的,因为我们总可以把这个粒子boost到另外一个动量在相反方向但是自旋不变的参考系。对于一个无质量的粒子,我们是没有办法找到这样一个boost的。}

Weyl方程的解是具有特定helicity的态。它们分别对应于左手和右手的粒子。\textbf{无质量粒子手性的洛仑兹不变性在Weyl方程中得到了明显的体现。因为 $\psi_L$ 和 $\psi_R$ 对应于洛仑兹群的不同的表示里面。}

\begin{exercise}{写出洛仑兹不变的 $u(p)$ 的归一化条件}
首先我们要知道,$u^\dagger u$ 不是洛仑兹不变的,因为\begin{equation}
u^\dagger u = \begin{pmatrix}
\xi^\dagger \sqrt{p\cdot\sigma} & \xi^\dagger \sqrt{p\cdot\bar\sigma}
\end{pmatrix}\cdot \begin{pmatrix}
\sqrt{p\cdot\sigma} \, \xi \\ 
\sqrt{p\cdot \bar\sigma} \, \xi
\end{pmatrix} = 2 E_{\mathbf p} \xi^\dagger \xi~.
\end{equation}
定义
\begin{equation}
\bar u(p) = u^\dagger (p) \gamma^0 ~.
\end{equation}
我们可以得到
\begin{equation}
\bar u u = 2 m \xi^\dagger \xi ~.
\end{equation}
\end{exercise}
接下来我们来总结一下。狄拉克方程的解可以写成平面波的线性叠加。正频的波有如下形式
\begin{equation}\label{eq_diracs_4}
\psi(x) = u(p) e^{-ip\cdot x}~, \quad p^2 = m^2 ~, \quad p^0>0 ~.
\end{equation}
$u(p)$ 有两个线性独立的解
\begin{equation}
u^s(p) = \begin{pmatrix}
\sqrt{p\cdot \,\, \sigma} \xi^s \\
\sqrt{p\cdot \,\, \bar \sigma} \xi^s
\end{pmatrix}~, \quad s = 1,2~.
\end{equation}
归一化条件为
\begin{equation}
\bar u^r (p) u^s (p) = 2 m \delta^{rs} ~, \quad u^{r\dagger} (p) u^s(p) = 2 E_{\mathbf p} \delta^{rs}~.
\end{equation}
负频的解为
\begin{equation}
\psi(x) = v(p) e^{+i p \cdot x} ~, \quad p^2 = m^2 ~, \quad p^0>0~.
\end{equation}
$v(p)$ 的两个线性独立的解为
\begin{equation}
v^s(p)=\begin{pmatrix}
\sqrt{p\cdot\sigma} \, \eta^s\\
-\sqrt{p\cdot\bar\sigma} \, \eta^s 
\end{pmatrix}~,\quad s = 1,2~.
\end{equation}
归一化条件为
\begin{equation}
\bar v^r(p) v^s(p) = -2 m \delta^{rs}~, \quad {\rm or} \quad v^{r\dagger}(p) v^s(p) = + 2 E_{\mathbf p} \delta^{rs}~.
\end{equation}
$u$ 和 $v$




