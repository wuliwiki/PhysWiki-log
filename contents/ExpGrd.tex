% 指数格点
% keys 指数格点|FEDVR|TDSE|数值解|薛定谔方程
% license Xiao
% type Tutor

\pentry{等比数列\nref{nod_HsGmPg}}{nod_d031}

在氢原子的数值计算中, 我们有时候在某个半径内使用指数格点, 即第 $n+1$ 个 FE 的长度是第 $n$ 个 FE 长度的 $\alpha$ 倍。 如果第一个 FE 的长度为 $d$, 那么有

\begin{equation}\label{eq_ExpGrd_1}
x_i = d \frac{\alpha^i - 1}{\alpha - 1}
\qquad (i = 0, 1, \dots)~.
\end{equation}

如果我们需要把格点变密或者变疏, 一种幼稚的想法是把 $d$ 变大而 $\alpha$ 保持不变。 但这么做以后会发现, 在不同的位置, FE 并没有等比例地变化。

想象我们画出 $x_i$ 的连续函数图。 严格的做法是把该图在 $x$ 方向上缩放 $\beta$ 即可($\beta > 1$ 就是越疏, $\beta < 1$ 就是越密)。 缩放后, \autoref{eq_ExpGrd_1} 变为
\begin{equation}
x_i = d \frac{\alpha^{\beta \cdot i} - 1}{\alpha - 1}
= d_1 \frac{\alpha_1^i - 1}{\alpha_1 - 1}
\qquad (i = 0, 1, \dots)~.
\end{equation}
其中
\begin{equation}
\alpha_1 = \alpha^\beta~,
\qquad
d_1 = d \frac{{\alpha^\beta - 1}}{\alpha - 1}~.
\end{equation}
