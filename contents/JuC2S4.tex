% 第 2 章小结
% 第 2 章|小结

本文授权转载自郝林的 《Julia 编程基础》. 原文链接:\href{https://github.com/hyper0x/JuliaBasics/blob/master/book/ch02.md}{第2章:编程环境}.

\subsubsection{2.4 小结}

在本章,我们一直在讨论 Julia 程序的编写环境.首先,我们详述了 Julia 语言的 REPL 环境的用法,包括在其中编写程序的简单规则、它的主要模式及其使用和切换方法、常用的快捷键,以及其中的代码补全功能.REPL 环境将会是我们编写 Julia 程序时最常用的.

之后,我们介绍了 Julia 项目环境的概念,并在此基础上讲解了仓库目录、程序包存储目录、环境配置文件,以及项目的创建和程序包的引入.这些内容都是我们在开发 Julia 项目时必须掌握的.

从下一章开始,我们就要正式地讲解 Julia 语言的语法了,包括各种程序定义的写法以及各式流程控制语句的用法.我相信,在掌握了前面所述的预备知识之后,你可以很轻松地开始下一阶段的学习.同时,你在之前留下的一些疑虑和问题,也将会在后面被陆续地解开.
