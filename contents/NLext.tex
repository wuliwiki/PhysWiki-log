% 牛顿—莱布尼兹公式(矢量分析)
% keys 牛顿|莱布尼兹公式|高维|散度
% license Xiao
% type Tutor

\pentry{散度 散度定理\nref{nod_Divgnc}}{nod_83ff}

作为牛顿—莱布尼兹公式\upref{NLeib}的一个高维拓展, 有
\begin{equation}\label{eq_NLext_3}
\int \grad f \dd{V} = \oint f \dd{\bvec s}~.
\end{equation}
该公式类似于散度定理\upref{Divgnc}, 但被积函数变为标量而不是矢量。 对于一维情况, 该式就是牛顿—莱布尼兹公式。

事实上梯度定理(\autoref{eq_Grad_15}~\upref{Grad})也可以看作是另一种拓展高维拓展。 

\subsubsection{证明}
我们可以对每个分量依次证明。 两边乘以第 $i$ 个分量的单位矢量 $\uvec x_i$ 得
\begin{equation}
\int \pdv{x_i}f \dd{V} = \oint (f\uvec x_i) \vdot \dd{\bvec s}~.
\end{equation}
由散度定理得
\begin{equation}
\oint (f\uvec x_i) \dd{\bvec s} = \int \div (f\uvec x_i) \dd{V} = \int \pdv{x_i} f \dd{V}~,
\end{equation}
证毕。
