% 命题及其表示法
% 命题|真命题|假命题|原子命题|复合命题

\begin{issues}
\issueTODO
\end{issues}

\subsection{命题}
能表达判断的语言是陈述句, 它称作\textbf{命题}。 一个命题,总是具有一个 “值”, 称为\textbf{真值}。 真值只有\textbf{真(True)}和\textbf{假(False)} 两种。 只有具有确定真值的陈述句才是命题, 一切没有判断内容的句子, 无所谓是非的句子, 如感叹句、疑问句、祈使句等都不能作为命题。

命题有两种类型:
\begin{enumerate}
\item \textbf{原子命题}: 不能分解为更简单的陈述语句
\item \textbf{复合命题}: 由联结词、标点符号和原子命题复合构成的命题
\end{enumerate} 
在此文中,命题用$p$,$q$,$r$来表示

\subsection{命题联结词}
命题联结词可以借助现有的命题来形成新的命题,如果用搭积木来比喻的话,我们已知的命题就像一块块积木,而命题连结词就是这些积木的摆放方式,我们可以把不同的积木(命题)摆在一起(连结起来),得到我们想要的状况(命题),常用的命题联结词有否定,合取,析取,蕴含和等价。

我们常用真值表来表示这些命题联结词作用的效果,所谓真值表就是将原命题和加了联结词后的命题的真值用表列出来。
\begin{definition}{否定词 $\neg$} \end{definition} 
给定命题$p$,则称$\neg p$为命题$p$的否定,其真值表为:

%\begin{definition}[合取] $\wedge$ \end{definition} 
%\begin{definition}[析取] $\vee$ \end{definition} 
%\begin{definition}[条件] $\rightarrow$ \end{definition} 
%\begin{definition}[双条件] $\leftrightarrow$ \end{definition} 