% CentOS 笔记
% license Xiao
% type Note

\begin{issues}
\issueDraft
\end{issues}

\begin{itemize}
\item 目标: 用 Beocat 的 CentOS 7.9.2009 完整 SLISC。
\item CentOS 的库一般很老
\item 支持 Linux Standard Base (LSB) 的 Linux 发行版中的二进制文件一般是兼容的。 CentOS 和 Ubuntu 都支持 LSB, 所以它们也一般是二进制兼容的。 但注意一些动态链接库可能会不同, 所以并不是光拷贝可执行文件本身就行。
\end{itemize}

\subsection{yum 包管理}
\begin{itemize}
\item \verb`yum list installed` 检查已安装的包
\item \verb`yum [-y] install 安装包`。
\item \verb`yun [-y] remove 安装包`
\item 要检查某个包的文件, 用 \verb`yum install yum-utils`, 然后 \verb`repoquery -l 包名`
\item ubuntu 中的 \verb`*-dev` 包一般叫做 \verb`*-devel`
\item \verb`rpm -qf 文件` 可以查询文件属于哪个包
\end{itemize}
