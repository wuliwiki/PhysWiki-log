%三角函数(复数)

\pentry{指数函数(复数)\upref{CExp}}
\subsection{定义}
复数域的正弦函数为
\begin{equation}\label{CTrig_eq1}
\sin z = \frac{\E^{\I z} - \E^{ - \I z}}{2\I}
\end{equation}
复数域的余弦函数为
\begin{equation}\label{CTrig_eq2}
\cos z = \frac{\E^{\I z} + \E^{ - \I z}}{2}
\end{equation}
为什么三角函数要这么定义?因为只有这么定义,才能既“兼容%未完成:引用
”实数范围内的三角函数,同时满足解析%未完成:引用
的要求.

\subsection{与实数函数的“兼容性”}
将\autoref{CTrig_eq1} 中的复数 $z$ 取实数 $x$, 得
\begin{equation}
\sin z = \frac{\E^{\I x} - \E^{ - \I x}}{2\I}
\end{equation} 
根据复数域的指数函数\upref{CExp}(欧拉公式)
,%未完成:引用
 \begin{equation}
{\E^{\I x}} = \cos x + \I\sin x
\end{equation} 
\begin{equation}
{\E^{ - \I x}} = \cos x - \I \sin x
\end{equation} 
代入得
\begin{equation}
\sin z = \frac{\left( {\cos x + \I\sin x} \right) - \left( {\cos x - \I\sin x} \right)}{2\I} = \sin x
\end{equation}  
同理
\begin{equation}
\cos z = \frac{\left( {\cos x + \I\sin x} \right) + \left( {\cos x - \I\sin x} \right)}{2} = \cos x
\end{equation}   
证毕.

\subsection{两角和公式}
利用欧拉公式,容易证明,复数范围内的正余弦函数同样满足两角和公式
\begin{equation}\label{CTrig_eq3}
\sin \left( {{z_1} + {z_2}} \right) = \sin {z_1}\cos {z_2} + \cos {z_1}\sin {z_2}
\end{equation}
\begin{equation}\label{CTrig_eq4}
\cos \left( {{z_1} + {z_2}} \right) = \cos {z_1}\cos {z_2} - \sin {z_1}\sin {z_2}
\end{equation}
\subsection{实部和虚部}
利用两角和公式,令 ${z_1}$ 等于实数 $x$,  ${z_2}$ 等于虚数 $\I y$, 则有
 \begin{equation}
\sin z = \sin \left( {x + \I y} \right) = \sin x\cos \I y + \cos x\sin \I y
\end{equation} 
\begin{equation}
\cos z = \cos \left( {x + \I y} \right) = \cos x\cos \I y - \sin x\sin \I y
\end{equation} 
其中
\begin{equation}
\cos \I y = \frac{\E^{ - y} + \E^y}{2} = \cosh y
\end{equation} 
\begin{equation}
\sin \I y = \frac{\E^{ - y} - \E^y}{2\I} = \I\frac{\E^y - \E^{ - y}}{2} = \I\sinh y
\end{equation} 
代入得
\begin{equation}
\sin z = \sin \left( {x + \I y} \right) = \sin x\cosh y + \I\cos x\sinh y
\end{equation} 
\begin{equation}
\cos z = \cos \left( {x + \I y} \right) = \cos x\cosh y - \I\sin x\sinh y
\end{equation}  
这样,就把正余弦的实部和虚部分开来了(当然也可以根据定义直接得到两式)
\begin{equation}
{\mathop{\rm Re}\nolimits} \left( {\sin z} \right) = \sin x\cosh y,{\mathop{\rm Im}\nolimits} \left( {\sin z} \right) = \cos x\sinh y
\end{equation} 
\begin{equation}
{\mathop{\rm Re}\nolimits} \left( {\cos z} \right) = \cos x\cosh y,{\mathop{\rm Im}\nolimits} \left( {\cos z} \right) =  - \sin x\sinh y
\end{equation}
\subsection{解析性}
由于 ${\E^z}$ 是解析函数,而解析函数的线性组合也是解析函数,所以正余弦函数都是解析函数%未完成:词条
.但也可以根据柯西—黎曼公式%未完成:词条
直接证明.






















