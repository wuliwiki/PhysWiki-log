% OpenMP 笔记

\begin{issues}
\issueDraft
\end{issues}

\pentry{C++ 基础\upref{Cpp0}}

\begin{itemize}
\item 在 Ubuntu 中运行单线程程序, top 命令会显示某个 CPU 的用量是 100. 然而在 Windows 中, 貌似这个程序会在所有 CPU 中来回切换, 使表面上看起来像是在并行(然而实际上并没有). 当在 windows 中用 OpenMP parallel for 的时候, 可以看到所有的 CPU 用量都是 100\%.
\item 在 Visual Studio 中使用 OpenMP, 只需打开 Project Property > C/C++ > Language > Open MP Support 选 Yes 即可。
\item 如果不只是用 pragma 而是用到了 omp 的函数, 需要头文件 \verb`<omp.h>`
\item 如何判断 OpenMP 生效了? 可以用一个 \verb|#pragma omp parallel for|, 如果不是按顺序执行的, 就是生效了。

\item 要指定 parallel for 的线程数上限, 用 \verb`omp_set_num_threads()` (用于 override \verb`OMP_NUM_THREADS` 环境变量)。 注意这里是 “上限”, 当 runtime 系统决定用更少的线程效果会更好时, 就会这么做。 如果不希望是 “上限” 而是严格规定, 用 \verb`omp_set_dynamic(0)` 或者设置环境变量 \verb`OMP_DYNAMIC` 为 \verb`false`。
 或者局部地
\item 还有一种指定线程的方法是用
\begin{lstlisting}[language=cpp]
#pragma omp parallel num_threads(n)
{
#pragma omp for
for () ...
}
\end{lstlisting}
\item 获取当前的总线程数用 \verb`omp_get_num_threads()`, 这个必须在 parallel 里面执行才不等于 1(例如 parallel for 的内部)
\item 获取当前线程编号用 \verb`omp_get_thread_num()`

\item \verb|#pragma omp parallel| 让每个线程执行接下来的一个命令 (或者 {} 中所有的命令). 如果 {} 中声明了变量, 则每个线程中都有一个同名变量。
\item [重要] 在 parallel for 之前声明的变量在 parallel for 内也是唯一的, 在 parallel for 内声明的变量每个线程都有一个。 在 parallel for 之前声明变量是许多常见的 bug 的来源。 尤其是 parallel for 内部的 for 循环的循环变量。 最好的办法是把 parallel for 内每个被赋值的变量都审核一遍。
\item 与 CUDA 不同, parallel for 内部不能有同步, 因为一个线程有可能执行不止一个循环。 如果要同步就用多个 parallel for, 中间就会自动同步。
\begin{lstlisting}[language=cpp]
#pragma omp parallel
{
#pragma omp for
...
// note implicit barrier after for construct
#pragma omp for
...
}
\end{lstlisting}
\item 若要让每个线程执行指定的代码, 用 sections
\begin{lstlisting}[language=cpp]
#pragma omp parallel sections
{
function1();
#pragma omp section
{function2(); function3();}
#pragma omp section
function4();
}
\end{lstlisting}
在这个代码中, function1-function4 分别只执行一次, 其中被 \verb|#pragma omp section| 隔开的三个部分允许并行, 但 function2() 和 function3() 不允许并行。 千万注意两个 section 中不能对同名变量赋值, 也就是说同名变量并不会每个线程都复制一份。

现在有一个问题就是如何并行嵌套循环
\begin{lstlisting}[language=cpp]
for (i = 0; i < Nx; ++i)
for (j = 0; j < Ny; ++j)
{}
\end{lstlisting}
两个 for 之间没有代码。 现在使用的方法是
\begin{lstlisting}[language=cpp]
for (ind = 0; ind < Nx*Ny; ++ind)
{
i = ind/Ny; j = ind%Ny;
...
}
\end{lstlisting}
\item atomic 运算与 CUDA 的概念一样, 在某个命令前面加上 \verb|#pragma omp atomic| 即可
\item \verb|omp_set_nested(N)| 允许 N 重 parallel for 嵌套, 默认不允许嵌套($N=0$), 只对最外层 parallel for 生效。 用 \verb|gdb| 可以追踪线程的数量。 例如以下代码的子函数内的 parallel for 就默认会失效, 除非把注释去掉。
\begin{lstlisting}[language=cpp]
void fun(int i)
{
#pragma omp parallel for
	for (int j = 0; j < 5; ++j)
		cout << "i = " + to_string(i) + ", j = " + to_string(j) + "\n";
}

int main()
{
	// omp_set_nested(1);
#pragma omp parallel for
	for (int i = 0; i < 5; ++i)
		fun(i);
}
\end{lstlisting}
\end{itemize}
