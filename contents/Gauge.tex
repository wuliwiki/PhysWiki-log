% 电磁场的规范变换
% keys 库仑规范|洛伦兹规范|标量势能|矢量势能|麦克斯韦方程组
% license Xiao
% type Tutor

\begin{issues}
\issueDraft 
\end{issues}

\pentry{电磁场标势和矢势\nref{nod_EMPot}}{nod_4d5d}

\footnote{参考 Wikipedia \href{https://en.wikipedia.org/wiki/Gauge_fixing}{相关页面}。}虽然标势和矢势可以唯一确定电磁场,但是同一个电磁场却可能对应不同的标势和矢势。 这是因为在麦克斯韦方程中,有物理意义的是$\bvec E,\bvec B$。我们只需保证:尽管改变四维电磁矢势$(\varphi,\bvec A)$,也对应相同的$\bvec E,\bvec B$即可。

由于$\curl  \bvec A=\bvec B$,利用函数的梯度是一个无旋场,我们可以令$\bvec A'=A+\grad\lambda$,则
\begin{equation}
\bvec B' = \curl \bvec A' = \curl (\bvec A + \grad \lambda) = \curl \bvec A = \bvec B~.
\end{equation}
 如果 $\lambda$ 随时间变化,\autoref{eq_EMPot_1}~\upref{EMPot} 中的电场会改变。 因此我们需要同时修正 $\varphi$, 才能确保变换后电场也不改变。 可以发现只需要令 $\varphi' = \varphi - \pdv*{\lambda}{t}$ 即可
\begin{equation}
\bvec E' = -\grad \varphi' - \pdv{\bvec A'}{t} = -\grad \qty(\varphi - \pdv{\lambda}{t}) - \pdv{t} (\bvec A + \grad \lambda) = \bvec E~.
\end{equation}


这种“保持电磁场不变时,对势 $\pmat{\varphi, \bvec{A}}$ 进行的变换”被称为\textbf{规范变换(gauge transformation)}:
\begin{equation}\label{eq_Gauge_3}
\leftgroup{
&\bvec A' = \bvec A + \grad \lambda\\
&\varphi' = \varphi - \pdv{\lambda}{t}~.
}\end{equation}
任何产生相同电磁场的两组标势矢势都可以通过规范变换联系起来。

除此以外,考虑到麦克斯韦方程只约束了$\bvec A$的散度,因此我们可以限定$\div \bvec A$的结果来减少冗余的自由度,还可同时简化麦克斯韦方程。
基于此,常见的两种规范是\enref{库仑规范}{Cgauge}和\enref{洛伦兹规范}{LoGaug}。下面列出常见的规范条件:
\begin{enumerate}
\item 库伦规范:$\div \bvec A=0$。
\item 洛伦兹规范:$\partial_{\mu}A^{\mu}=0$。
\item 辐射(radiation)规范:$\div \bvec A=0,A^0=0$。($\bvec A$只有两个自由度,对应电磁场只有两个独立自由度,电磁波只有两个独立偏振方向。)
\item 瞬时(temporal)规范:$A^0=0$。
\item 轴(axial)规范:$A^3=0$。
\end{enumerate}
这些规范条件总可以通过$\lambda$的具体选定来实现,读者可自证这一点。