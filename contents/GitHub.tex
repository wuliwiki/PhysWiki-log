% GitHub Desktop 的简单使用

\subsection{Git 与 GitHub}
要了解 GitHub 首先要了解 Git. Git 是一个文件版本控制程序, 通常用于管理程序代码, 但也可以用来管理任何文件. 从某种意义上来讲, Git 相当于一个强大的文本备份软件, 它可以保存一个文件夹内所有文件(这个文件夹叫做 \textbf{repository}, 简称 \textbf{repo})的许多不同时间的 \textbf{snapshot}(快照, 即所有文件某时刻的内容),而且可以对比出每个快 snapshot 相对上一个的变化. 由于我们只能理解文本文件的变化, 所以 Git 不擅长管理二进制文件(如 exe,pdf,docx 等). Git 还可以给文件夹创造不同的 \textbf{branch (分支)}, 例如一个程序写到一定的阶段后, 我们希望将其向两个不同的方向发展为两个不同的程序,就可以创建一个新的分支,分别记录这两个程序的发展.

Git 本身是一个基于控制行的程序, 最初是为了管理 Linux 操作系统的源代码被发明的, 虽然 Git 现在已经有了各种各样的图形界面程序, 但控制行版本的功能仍然是最齐全的. 所以一般的 Git 教程都是控制行版本的教程. 这里不介绍控制行, 而是介绍一个简单但广泛使用的界面程序 GitHub Desktop, 注意该程序只能在 Windows 或 Mac 系统中安装.

Git 的一个重要功能就是可以让许多人合作完成一个工程, 这个功能通常需要在一台服务器中安装 Git Server, 每个合作者加入时把服务器中的 repo \textbf{clone}(克隆)到自己的电脑上.

Git 中的每一个 snapshot 都是通过手动\textbf{commit(提交)}操作完成. 我们可以随时查看当前文件夹中的文件与上次 commit 相比是否发生了改动, 以及改动了什么内容, 也可以选择撤销哪些改动.
 
%未完成
