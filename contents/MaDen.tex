% 度量空间的稠密性
% keys 稠密|可分空间
% license Xiao
% type Tutor

\begin{issues}
补充例子
\end{issues}

\pentry{度量空间的闭包}{nod_8856}
\cite{Ke1}在序集中根据偏序关系有稠密性的概念。比如有理数集是实数集的稠密子集的是指任意满足 $a<b$ 的两个实数 $a,b$,恒有有理数 $c$ 存在,使得 $a\leq c\leq b$。更一般的,偏序集 $B$ 是偏序集 $A$ 的稠密子集是指,任意 $A$ 中满足 $a<b$ 的两元素 $a,b$,恒有 $c\in B$,使得 $a\leq c\leq b$(见稠密性与完备性\upref{OrdCom})。也就是说:$B$ 是 $A$ 的稠密子集,相当于 $A$ 中的两元素“之间”都有 $B$ 的两元素将它们分隔。

同样,在度量空间中,根据“距离关系”也有稠密性的概念。类比序集中的稠密性概念,我们可以猜测:度量空间 $B$ 在 $A$ 中是稠密子集,是指在 $A$ 的任意两不同的元素之间恒有 $B$ 的元素将它们分隔。即,任意 $a\neq b$ 的两元素 $a,b\in A$,恒有 $c\in B$ 存在,使得 $d(a,c)+d(c,b)=d(a,b)$。若将 $a,b$ “分半”,取它们的“中间点”,则中间点和 $a,b$ 之间又有 $B$ 的元素将它们分隔。如此继续分下去,就会发现,任意 $A$ 的元素的任一邻域都必然存在 $B$ 的元素。即 $A$ 的每一点都是 $B$ 接触点。这就是说 $A\subset [B]$(闭包的概念)。

\begin{definition}{稠密子集}\label{def_MaDen_2}
设 $A,B$ 是度量空间 $X$ 的两个子集。若 $A\subset [B]$,则称 $B$ 在 $A$ 中\textbf{稠密}。若 $[A]=X$,则称 $A$ 在 $X$ 中\textbf{处处稠密}。
\end{definition}

由上面定义,若 $A$ 在 $B$ 中不稠密,则 $B$ 中包含有另一与 $A$ 无任一公共点的球 $B'$。
\begin{definition}{无处稠密}\label{def_MaDen_1}
若度量空间 $X$ 的子集 $A$ 在任一球中不稠密,则称 $A$ \textbf{无处稠密}(或处处不稠密)。
\end{definition}

\begin{definition}{可分空间}
设 $X$ 是度量空间。若 $A\subset X$ 处处稠密且可数,则称 $X$ 是\textbf{可分空间}。
\end{definition}


