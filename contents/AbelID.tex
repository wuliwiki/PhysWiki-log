% 阿贝尔微分方程恒等式
% keys 朗斯基行列式|Wronski行列式|Wronskian|线性微分方程|Abel's Identity|Abel's Formula|阿贝尔公式
% license Xiao
% type Tutor

本文翻译并节选自WikiPedia的\href{https://en.wikipedia.org/wiki/Abel\%27s_identity}{Abel's Identity}。


\subsection{综述}

在数学中,\textbf{阿贝尔恒等式(Abel's identity)}(亦称\textbf{阿贝尔公式(Abel's formula)}\footnote{Rainville, Earl David; Bedient, Phillip Edward (1969). Elementary Differential Equations. Collier-Macmillan International Editions.} 或者 \textbf{阿贝尔微分方程恒等式(Abel's differential equation identity)}),是一个等式,用于表示一个二阶齐次线性常微分方程的两个解的朗斯基行列式,只需要用到原方程的系数。这一关系也可以推广到$n$阶的线性微分方程。该恒等式命名自挪威数学家Niels Henrik Abel。

由于阿贝尔恒等式把微分方程不同的线性独立解联系起来了,因此它也可以用来从一个特解得到另一个特解。它给出了解之间很有用的恒等关系,同时也在参数变易法(variation of parameters)等其它技巧中功不可没。在Bessel方程等无法给出简单解析解的方程中尤其有用,因为这些情况下朗斯基行列式非常难算。

用\href{https://en.wikipedia.org/wiki/Liouville\%27s_formula}{Liouville公式}能将阿贝尔恒等式推广到齐次线性微分方程的一阶系统上。



\subsection{公式描述}

考虑二阶齐次线性微分方程
\begin{equation}\label{eq_AbelID_1}
y'' + p(x)y' +q(x)y = 0~.
\end{equation}
其中$p, q$是实数轴上一区间$I$上的连续函数,函数值为实数或者复数。

阿贝尔恒等式是说,对于\autoref{eq_AbelID_1} 的任意两个解$y_1, y_2$以及任意$x_0\in I$,其朗斯基行列式$W[y_1, y_2]$满足以下等式:
\begin{equation}\label{eq_AbelID_4}
W[y_1, y_2](x) = W[y_1, y_2](x_0)\cdot \exp\qty(-\int _{x_0}^x p(z)\dd z), \quad x\in I~.
\end{equation}
(译者注:或者使用$W'=-pW$,见本小节的“定理证明”部分。)


\subsubsection{批注}

\begin{itemize}
\item 特别地,朗斯基行列式$W[y_1, y_2]$在$I$上要么恒等于$0$,要么恒大于$0$,要么恒小于$0$。对于恒大于或小于$0$的情况,$y_1$和$y_2$就是线性无关的。
\item 没必要假设$y_1$和$y_2$的二阶导数连续。
\item 阿贝尔这个定理在$p(x)=0$的时候尤其好用,因为此时$W$是常数。
\end{itemize}


\subsubsection{定理证明}

考虑朗斯基行列式
\begin{equation}
W[y_1, y_2] = y_1y'_2 - y'_1y_2~.
\end{equation}
两边求导得(省略$x$以简化表达)
\begin{equation}\label{eq_AbelID_3}
W' = y_1y_2'' - y''_1y_2~,
\end{equation}
而从原本的微分方程能得到
\begin{equation}\label{eq_AbelID_2}
y''_i = -(py'_i+qy_i)~.
\end{equation}
把\autoref{eq_AbelID_2} 代入\autoref{eq_AbelID_3} 得到
\begin{equation}\label{eq_AbelID_6}
\ali{
    W' &= -y_1(py'_2 + qy_2) + y_2(py'_1 + qy_1)\\
    &= p(-y_1y'_2 + y'_1y_2)\\
    &= -pW~.
}
\end{equation}
这是一个一阶常微分方程,其解正是阿贝尔恒等式\autoref{eq_AbelID_4}。

由于$p$在$I$上连续,因此它在$I$的任何有界闭子区间上有界且可积,从而使得下列函数有良好定义:
\begin{equation}\label{eq_AbelID_5}
V(x) = W(x) \exp\qty(\int_{x_0}^x p(\xi)\dd \xi),\quad x\in I~.
\end{equation}
对\autoref{eq_AbelID_5} 两边求导,再代入\autoref{eq_AbelID_6} 可得
\begin{equation}
\ali{
    V(x) &= ( W'(x) + W(x) p(x) )\exp\qty(\int_{x_0}^x p(\xi)\dd \xi)\\
    &= 0~.
}
\end{equation}
从而$V(x)$是常数,即
\begin{equation}
W(x)\exp\qty(\int_{x_0}^x p(\xi)\dd \xi)~
\end{equation}
是\textbf{常数}。由于$V(x_0)=W(x_0)$,此常数即为$W(x_0)$。


\subsection{推广}

考虑区间$I$上的$n$阶齐次线性微分方程($n$为正整数)
\begin{equation}\label{eq_AbelID_7}
y^{(n)} + p_{n-1}(x)y^{(n-1)} + \cdots + p_1(x)y' + p_0(x)y = 0~,
\end{equation}
其中$p_{n-1}$在$I$上连续。

阿贝尔恒等式的推广形式是说,\autoref{eq_AbelID_7} 的$n$个解$y_1, y_2, \cdots, y_n$的朗斯基行列式$W[y_1, y_2, \cdots, y_n]$,满足下列关系:
\begin{equation}
\ali{
W[y_1, y_2, \cdots, y_n](x) &= W[y_1, y_2, \cdots, y_n](x_0)\exp\qty(-\int_{x_0}^x p_{n-1}(z)\dd z) \\
\forall x, x_0&\in I
}~.
\end{equation}




\subsubsection{直接证明}

先写出$W[y_1, y_2, \cdots, y_n]$的定义:
\begin{equation}
W[y_1, y_2, \cdots, y_n] = 
\begin{vmatrix}
y_1&y_2&\cdots&y_n\\
y'_1&y'_2&\cdots&y'_n\\
y''_1&y''_2&\cdots&y''_n\\
\vdots&\vdots&\ddots&\vdots\\
y_1^{(n-1)}&y_2^{(n-1)}&\cdots&y_n^{(n-1)}\\
\end{vmatrix}~.
\end{equation}
求导得
\begin{equation}\label{eq_AbelID_8}
\ali{
    W'[y_1, y_2, \cdots, y_n] ={}& \\
&\vmat{
y'_1&y'_2&\cdots&y'_n\\
y'_1&y'_2&\cdots&y'_n\\
y''_1&y''_2&\cdots&y''_n\\
\vdots&\vdots&\ddots&\vdots\\
y_1^{(n-1)}&y_2^{(n-1)}&\cdots&y_n^{(n-1)}\\
}+\\
&\vmat{
y_1&y_2&\cdots&y_n\\
y''_1&y''_2&\cdots&y''_n\\
y''_1&y''_2&\cdots&y''_n\\
\vdots&\vdots&\ddots&\vdots\\
y_1^{(n-1)}&y_2^{(n-1)}&\cdots&y_n^{(n-1)}\\
}+\\
&\qquad\vdots\\
&\vmat{
y_1&y_2&\cdots&y_n\\
y'_1&y'_2&\cdots&y'_n\\
y''_1&y''_2&\cdots&y''_n\\
\vdots&\vdots&\ddots&\vdots\\
y_1^{(n)}&y_2^{(n)}&\cdots&y_n^{(n)}\\
}\\
}~.
\end{equation}
容易发现,右端除了最后一项,其它所有项都因为求导而有两行完全相同,于是它们都为零。因此
\begin{equation}\label{eq_AbelID_9}
W'[y_1, y_2, \cdots, y_n] = 
\vmat{
y_1&y_2&\cdots&y_n\\
y'_1&y'_2&\cdots&y'_n\\
y''_1&y''_2&\cdots&y''_n\\
\vdots&\vdots&\ddots&\vdots\\
y_1^{(n)}&y_2^{(n)}&\cdots&y_n^{(n)}\\
}~
\end{equation}
缺失了$y_i^{(n-1)}$项。

注意到各$y_i$都是\autoref{eq_AbelID_7} 的解,可知
\begin{equation}
y_i^{(n)}  + p_{n-2}\,y_i^{(n-2)} + \cdots + p_1\,y'_i + p_0\,y_i = -p_{n-1}y_i^{(n-1)}~,
\end{equation}
对于全体$i\in\{1, 2, 3, \cdots, n\}$都成立。

把\autoref{eq_AbelID_9}  右端行列式的第一行乘以$p_0$后加到最后一行、第二行乘以$p_1$后加到最后一行、第三行……第$n-1$行乘以$p_n$后加到最后一行,行列式的值不变,于是得到
\begin{equation}\label{eq_AbelID_10}
\begin{aligned}
W'[y_1, y_2, \cdots, y_n] &= 
\begin{vmatrix}
y_1&y_2&\cdots&y_n\\
y'_1&y'_2&\cdots&y'_n\\
y''_1&y''_2&\cdots&y''_n\\
\vdots&\vdots&\ddots&\vdots\\
-p_{n-1}y_1^{(n-1)}&-p_{n-1}y_2^{(n-1)}&\cdots&-p_{n-1}y_n^{(n-1)}\\
\end{vmatrix}\\
&=-p_{n-1}
\begin{vmatrix}
y_1&y_2&\cdots&y_n\\
y'_1&y'_2&\cdots&y'_n\\
y''_1&y''_2&\cdots&y''_n\\
\vdots&\vdots&\ddots&\vdots\\
y_1^{(n-1)}&y_2^{(n-1)}&\cdots&y_n^{(n-1)}\\
\end{vmatrix}\\
&= -p_{n-1}W[y_1, y_2, \cdots, y_n]~.
\end{aligned}
\end{equation}
这是关于$W$的微分方程,解方程即得证。



%\subsubsection{利用Liouville公式证明}












