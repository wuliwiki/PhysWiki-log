% 量子不可克隆定理
% keys 量子不可克隆定理

\begin{issues}
\issueDraft
\end{issues}

所谓克隆在物理上可以看作是信息的复制。当我们谈克隆一个原子的时候,并不是说凭空制造出一个新的原子来,而是把原来的原子的状态(动量,各种量子数等)复制到目标原子上。

在经典信息学中,对任意输入的信息进行复制似乎并不是一个困难的事情,我们也希望将它推广到量子信息学中,也就是能够对任意输入的。但是在量子物理中,却有着一个重要的被称作不可克隆定理的结论:
\begin{theorem}{}
不能构造出一个能够确定性地复制任意量子态而且不改变原始量子态的克隆机器。也就是说不存在一个作用在$\mathcal{H}\otimes\mathcal{H}$上的幺正演化$U$,满足$\forall\ket{\psi}$
\begin{equation}
U\ket{\psi}\ket{0}=e^{i\alpha(\psi)}\ket{\psi}\ket{\psi}~.
\end{equation}
\end{theorem}

\textbf{证明1:}
对于任意两个量子态$\ket{\psi_1},\ket{\psi_2}$,有\begin{equation}
 U\ket{\psi_1}\ket{0}=e^{i\alpha(\psi_1)}\ket{\psi_1}\ket{\psi_1},\quad U\ket{\psi_2}\ket{0}=e^{i\alpha(\psi_2)}\ket{\psi_2}\ket{\psi_2}~,
\end{equation}
即
\begin{equation}
\begin{aligned}
\braket{\psi_1}{\psi_2} &=(\bra{\psi_1}\bra{0}U^\dagger)(U\ket{\psi_2}\ket{0})\\
&=(\bra{\psi_1}\bra{\psi_1}e^{-i\alpha(\psi_1)})(e^{i\alpha(\psi_2)}\ket{\psi_2}\ket{\psi_2})\\
&=e^{i(\alpha(\psi_2)-\alpha(\psi_1))}\braket{\psi_1}{\psi_2}^2~,
\end{aligned}
\end{equation}
因此有$|\braket{\psi_1}{\psi_2}|=|\braket{\psi_1}{\psi_2}|^2$,只有当$|\braket{\psi_1}{\psi_2}|=0$或1时才成立。前者代表两个态是正交的,后者代表两个态是同一个态。这和任取两个态的假设相矛盾。

\textbf{证明2:}



