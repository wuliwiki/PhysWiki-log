% 大尺度结构形成
% license Usr
% type Tutor

今天,宇宙在比目前视界更小的尺度上是非常不均匀的。例如,如星系调查结果所示,它是一个可观宇宙中所有星系的三维地图。仅凭肉眼就可以看出各种结构:团块、丝状、墙、空洞...,它们出现在不同的尺度上。其他天文观测,如Lyman-α森林、弱透镜测量、星系团计数等,也证实了宇宙是一个团块状的宇宙。定量地,从所有这些测量中可以提取出物质功率谱P(k)。它通过以下方式定义:
\begin{equation}
\langle \delta_k \delta_{k'} \rangle = (2\pi)^3 P(k) \delta^3(\mathbf k - \mathbf k')~.
\end{equation}
其中,$\delta_k$是密度对比$\delta(\mathbf r)$的傅里叶变换,而$\langle\rangle$表示对k方向的平均。狄拉克δ函数不要与表示扰动的δ混淆,意味着具有不同k ≠ k'的模式在统计上是独立的。这一特性可以从膨胀中理解,膨胀预测了平均统计特性,这些特性编码在P(k)中。由于$P(k)$是密度自相关函数的傅里叶变换,它在傅里叶空间方便地表达了物理空间中物质分布的不均匀性。P(k)的大(小)值意味着存在许多(少)具有特征尺寸的结构。测量结果因此表明,宇宙在所有尺度上都有一定的功率。为了使宇宙的团块状更加明显,有时将$P(k)$(单位为长度的三次方)重新缩放为无量纲的方差。
\begin{equation}
\Delta^2(k) = k^3 P(k) /2\pi^2~.
\end{equation}
小的$\Delta^2$对应于小的密度对比,而$\Delta^2\sim 1$表示比平均密度更大密度的区域。$P(k)$数据的快速操作表明,在大$k$区域确实上升到大值。换句话说,数据显示宇宙在小尺度(大$k$)上表现出大的不均匀性。在标准宇宙学模型中,原始的不均匀性是由具有小振幅的膨胀产生的。这从CMB的近乎完美平滑中得到证实,CMB基本上提供了一张在光子最后一次散射时宇宙光子内容的照片。这提出了一个问题:这些微小的原始不均匀性如何从如此小的振幅增长到我们今天观察到的大对比度?答案是密度扰动的增长主要受到暗物质的驱动。为了定量理解这一过程,我们需要概述早期宇宙中这些扰动的演化。

让我们考虑一个充满一般物质流体的宇宙,我们暂时不指定其确切性质。通过将完整的广义相对论计算近似为其牛顿极限,可以理解这种介质中密度扰动的演化主要特征。这是一个有效的描述,适用于在视界尺度上非相对论性物质的长度尺度。非相对论性流体完全由其密度 \( \rho(r, t) \),速度场 \( v(r, t) \) 以及压力的状态方程 \( \varphi(\rho) \) 所表征。它的引力相互作用由牛顿势 \( \Phi \) 描述。这些量由以下演化(“欧拉”)方程描述:
\begin{equation}\label{eq_LSSFor_1} \begin{cases} 
\frac{\partial \rho}{\partial t} + \nabla \cdot (\rho v) = 0, & \text{连续性} \\
\frac{\partial v}{\partial t} + (v \cdot \nabla)v = -\frac{\nabla \varphi}{\rho} - \nabla \Phi, & \text{牛顿定律} \\
\nabla^2 \Phi = 4\pi G \rho, & \text{泊松方程}
\end{cases} ~.
\end{equation} 
这些方程构成了一组非线性微分方程。对于准均匀宇宙,通过对上述量展开为零阶均匀背景的一阶扰动,可以获得有用的解析信息:
\begin{equation} 
\rho = \rho_0(t) + \rho_1(x, t), \quad \varphi = \varphi_0 + \varphi_1, \quad v = v_0 + v_1, \quad \Phi = \Phi_0 + \Phi_1~. 
\end{equation}
我们首先考虑静态宇宙(无膨胀)的情况。\autoref{eq_LSSFor_1} 简化为一组描述扰动的耦合方程:
\begin{equation}\label{eq_LSSFor_4}
\begin{cases} 
\frac{\partial \rho_1}{\partial t} + \rho_0 \nabla \cdot v_1 = 0, & \text{(静态宇宙)} \\
\frac{\partial v_1}{\partial t} + \frac{v^2_s}{\rho_0} \nabla \rho_1 + \nabla \Phi_1 = 0, \\
\nabla^2 \Phi_1 = 4\pi G \rho_1,
\end{cases} ~.
\end{equation}
其中我们定义了量 \( v^2_s = \frac{\partial \varphi}{\partial \rho} = \frac{\varphi_1}{\rho_1} \),这被解释为流体中的速度(如我们片刻后将看到的)。通过对第一个方程进行时间导数操作,并使用第二和第三个方程进行适当的代换,可以得到密度扰动 \( \rho_1 \) 的演化方程,即Jeans方程:
\begin{equation}\label{eq_LSSFor_2}
\frac{\partial^2 \rho_1}{\partial t^2} - v^2_s \nabla^2 \rho_1 = 4\pi G \rho_0 \rho_1~.
\end{equation}
忽略重力(即,设$G=0$),解是密度波(即声音),以速度$v_s$传播。包括重力,完整的方程表达了压力项(左侧)与坍缩项(右侧)之间的竞争。Jeans长度 \( \lambda_J = \sqrt{\frac{v^2_s}{4\pi G \rho_0}} \) 区分了哪个项占主导:大尺度 \( \lambda > \lambda_J \) 的扰动随时间增长并增长,而小尺度 \( \lambda < \lambda_J \) 的扰动则得到压力的支持。随时间增长的坍缩模式是指数级的,\( \rho_1 \propto e^{\sqrt{4\pi G \rho_0} t} \),通过解微分方程\autoref{eq_LSSFor_2} 可以很容易地验证这一点,当$v_s \rightarrow 0$时。这是Jeans不稳定性,当应用于正常物质时,解释了气体云如何坍缩形成紧凑的天体,例如恒星。Jeans尺度的直观含义是,当云层足够大,其流体静压 \( \tau_{\text{pressure}} \sim \lambda_J / v_s \) 太慢而无法阻止引力吸引时,它就会坍缩。引力吸引在时间尺度 \( \tau_{\text{gravity}} \sim (G\rho_0)^{-1/2} \) 上聚集它。人们也可以定义Jeans质量为 \( M_J = \frac{4\pi \rho_0 \lambda^3_J}{3} \),即包围在半径 \( \lambda_J \) 的球体内的质量。质量 \( M > M_J \) 的扰动是“Jeans不稳定”的并且会坍缩。

我们接下来考虑一个膨胀的宇宙的情况。求解描述平滑背景的零阶量可以得到哈勃膨胀:
\begin{equation}\label{eq_LSSFor_3}\rho_0(t) = \rho_0(t_0) a^3, \quad v_0 = Hr, \quad \nabla \Phi_0 = \frac{4\pi G \rho_0}{3} r~. \end{equation}
到目前为止,宇宙的常规尺度因子 \( a(t) \) 的时间依赖性尚未指定。求解\autoref{eq_LSSFor_1} 将确定物质主导宇宙的膨胀 \( a(t) \)。一般来说辐射和宇宙常数也有助于膨胀,而 \( a(t) \) 由完全相对论的弗里德曼方程给出。\autoref{eq_LSSFor_3} 中的第一个关系只是非相对论性物质在体积膨胀 \( a^3 \) 时的标准稀释。第二个关系是相对于均匀背景的速度场的哈勃定律,其中相对于它,扰动 \( v_1 \) 可以被视为特殊速度。第三个是应用高斯定理到以原点为中心、半径为 \( r \) 的球体的泊松方程的解,忽略外面的无限物质量;这种逻辑上的跳跃,被称为“Jeans诡计”,在完整的相对论处理中由于存在视界而得到正当化。

将\autoref{eq_LSSFor_1} 展开到 \( \delta \) 的一阶,经过繁琐但直接的推导,可以得到线性方程:
\begin{equation}\label{eq_LSSFor_5}\begin{cases} 
\frac{\partial \rho_1}{\partial t} + 3H\rho_1 + H(r \cdot \nabla)\rho_1 + \rho_0\nabla \cdot v_1 = 0, \\
\frac{\partial v_1}{\partial t} + Hv_1 + H(r \cdot \nabla)v_1 + \frac{v_s^2}{\rho_0}\nabla\rho_1 + \nabla\Phi_1 = 0, \\
\nabla^2\Phi_1 = 4\pi G\rho_1,
\end{cases} \text{(膨胀宇宙)} ~.
\end{equation}
与\autoref{eq_LSSFor_4} 相比,\autoref{eq_LSSFor_5} 中的额外项由 $H$ 的出现来识别。接下来,我们定义相对密度(或密度对比) \( \delta(r) \) 并将其展开为共动傅里叶模式:
\[ \delta(r) \equiv \frac{\rho_1(r)}{\rho_0} = \int \frac{d^3k}{(2\pi)^3} \delta_k(t) e^{-ik \cdot x}, \quad \text{其中} x \equiv r a(t)~. \]
通过因子 \( a(t) \) 重新缩放空间向量意味着波数 \( 1/k \) 遵循宇宙的平均演化。这样做是方便的,因为这样,具有不同 \( k \) 的模式 \( \delta_k \) 的方程结果会是解耦的。类似地,我们定义 \( v_k \) 和 \( \Phi_k \),分别是速度扰动 \( v_1 \) 和引力势 \( \Phi_1 \) 的傅里叶变换。在傅里叶空间中,方程(1.11)变为:
\begin{equation}\label{eq_LSSFor_6} \begin{cases} 
\frac{\partial \delta_k}{\partial t} - ik \cdot a v_k = 0, \\
\frac{\partial (a v_k)}{\partial t} - ik v_s^2 \delta_k - ik \Phi_k = 0, \\
\Phi_k = -\frac{4\pi G \rho_0}{k^2 a^2} \delta_k.
\end{cases} ~.
\end{equation}
第二个方程(1.13)可以进一步简化。我们可以将速度扰动分解为 \( v_1 = v_{1\perp} + v_{1\parallel} \),其中 \( \nabla \cdot v_{1\perp} = 0 \)(散度自由或无旋部分)和 \( \nabla \times v_{1\parallel} = 0 \)(旋度自由或无旋部分)。在傅里叶空间中,\( k \cdot v_{k_\perp} = 0 \),\( k \times v_{k_\parallel} = 0 \)。\autoref{eq_LSSFor_6} 中的第二个方程中的 \( v_{k_\perp} \) 仅等于 \( \frac{\partial (a v_{k_\perp})}{\partial t} = 0 \),给出 \( v_{k_\perp} \propto \frac{1}{a} \)。因此,无旋部分 \( v_{k_\perp} \) 随着宇宙的膨胀而衰减,只有无旋部分 \( v_{k_\parallel} \) 存活。由于 \( v_{k_\parallel} \) 与 \( k \) 平行,我们可以在\autoref{eq_LSSFor_6} 中的所有地方替换 \( v_k \rightarrow v_k \hat{k} \)(其中 \( \hat{k} \) 是沿 \( k \) 的单位向量)和 \( k \rightarrow |k| \equiv k \)。此时,我们可以将三个\autoref{eq_LSSFor_6} 合并为一个二阶方程。通过从第一个方程中取 \( v_k \),将其插入第二个方程,并使用第三个方程中的 \( \Phi_k \),我们得到:
\begin{equation}\label{eq_LSSFor_7} \frac{\partial^2 \delta_k}{\partial t^2} + 2H \frac{\partial \delta_k}{\partial t} + \left( \frac{v_s^2 k^2}{a^2} - 4\pi G\rho_0 \right) \delta_k = 0~. 
\end{equation}
这是描述密度扰动的Jeans方程,类似于方程(1.9),但在傅里叶空间中,并且对于具有哈勃参数H的膨胀宇宙。与静态情况类似,压力项(与 \( v_s^2 \) 成正比)在小尺度模式上占优势,即 \( k > k_J \equiv a \sqrt{\frac{4\pi G\rho_0}{v_s^2}} \)。临界Jeans波数 \( k_J \) 现在依赖于 \( a \),表明不同的尺度在不同的时间开始聚集。此外,宇宙的膨胀具有类似摩擦的效果((1.14)中与H成正比的额外项),从而减慢了物质的聚集。接下来,我们需要指定我们所处理的是哪种物质流体,以及在宇宙的哪个演化时期它是相关的。

\subsection{非聚集的重子物质}

在考虑暗物质之前,让我们首先关注与光子耦合的重子物质。这意味着质子(以及原子核)和电子,在宇宙学中统称为“重子物质”,它们通过电磁作用紧密地与光子耦合。它们形成了一个具有相对论性声速 \( v_s \approx c/\sqrt{3} \) 的物质流体,其中 \( c = 1 \) 是光速。有了这个来自相对论物理的额外输入,我们可以继续使用非相对论Jeans方程\autoref{eq_LSSFor_7} 来处理密度扰动。大的声速 \( v_s \approx O(1) \) 意味着光子提供的巨大压力在\autoref{eq_LSSFor_7} 中占主导地位,超过引力项。因此,人们期望这样的流体不会聚集。这通过\autoref{eq_LSSFor_7} 的显式解得到了证实。在一个物质主导的宇宙中,即在最有利于聚集的情况下,因为辐射不会聚集,我们有 \( \rho_0 = \frac{3H^2}{8\pi G} \),\( a(t) = \left(\frac{3H_0 t}{2}\right)^{\frac{2}{3}} \) 和 \( H = \frac{\dot{a}}{a} = \frac{2}{3t} \)。然后\autoref{eq_LSSFor_7} 变为:
\[ \frac{\partial^2 \delta_k}{\partial t^2} + \frac{4}{3t} \frac{\partial \delta_k}{\partial t} + \frac{v_s^2 k^2}{(\frac{3}{2}H_0)^{4/3}t^{4/3}} = 0 \,\,\,\,\text{解出来是} \,\,\,\,\delta_k = \frac{c_1 \cos (c_0 k t^{1/3}+c_2\sin(c_0 k t^{1/3}))}{k t^{1/3}}~. \]
这里我们没有指定系数 \( c_0 \),\( c_1 \) 和 \( c_2 \),它们取决于 \( v_s \) 和 \( H_0 \)。主要结果是这个解是振荡的(压力波,在CMB功率谱中观察到的峰值)并且随时间衰减。与光子紧密耦合的重子流体不会在 \( k > k_J(a) \sim \frac{aH}{v_s} \) 的尺度上聚集,后者等于视界,因为 \( v_s \approx O(1) \)。换句话说,没有暗物质的正常物质只能在晚期聚集,也就是在它与光子解耦之后。

\subsection{暗物质的团块}
冷暗物质的不同之处在于它不与光子相互作用,并且被集体描述为具有非相对论性声速的简单流体,\( v_s = 0 \)。这在宇宙学中产生了巨大的差异。在方程(1.14)中仅保留引力项,并再次考虑宇宙的物质主导阶段,我们得到:
\[ \frac{\partial^2 \delta_k}{\partial t^2} + \frac{4}{3t} \frac{\partial \delta_k}{\partial t} - \frac{2}{3t^2} \delta_k = 0 \,\,\,\, \text{解出来是} \,\,\,\,\delta_k = c_{grow} t^{2/3}+c_{decay} t^{-1}~. \]
这个解包含一个衰减模式,它逐渐消失,而且最重要的是,它包含一个随时间的特定正幂次增长的模式。这就是宇宙从原始准均匀状态演化到今天多尺度结构状态的方式。顺便提一下,注意,在静态情况下发现的指数增长在膨胀宇宙中变成了幂律增长:膨胀像刹车一样减慢了聚集。

为了完整性,我们也可以在辐射主导(RD)时期考虑暗物质流体,由 \( H = \frac{1}{2t} \) 描述。在这种情况下,可以证明\autoref{eq_LSSFor_7} 既失去了 \( v_s^2 k^2 \delta_k/a^2 \) 项(因为暗物质流体的 \( v_s = 0 \)),也失去了 \( 4\pi G \rho_0 \delta_k \) 项(因为相关的 \( \delta_k \) 将是辐射的,然而辐射不会聚集,因此不会在引力势中产生扰动)。因此,\autoref{eq_LSSFor_7} 简化为:
\[ \frac{d^2 \delta_k}{dt^2} + \frac{1}{t} \frac{d \delta_k}{dt} = 0 \,\,\,\,\text{解出来是} \,\,\,\,\delta_k \propto \ln t+\text{常数}  ~. \]
因此,在辐射主导期间,暗物质扰动仅以对数方式增长。类似的计算表明,当宇宙由真空能量或曲率主导时,暗物质不会有效聚集。

总结来说,结构形成的标准历史如下。宇宙在 \( a_{\text{eq}} \sim 1/3400 \) 时变得物质主导。此时,原始暗物质的微小不均匀性 \( \delta_k \sim 10^{-5} \) 开始在长度尺度 \( 1/k \) 上增长,这些尺度是当前视界的$3400$倍以下。更大的长度尺度在它们变得小于膨胀视界时开始聚集。在物质主导期间,当尺度因子 \( a_k \) 大约等于 \( H_0^2 / k^2 \) 时,模式 \( k \) 重新进入视界。到目前为止,我们有 \( \delta_k \sim 10^{-5}(a_0/a_k) \sim 10^{-5}(k/H_0)^2 \),这在尺度 \( H_0/k \lesssim 10^{-5/2} \) 上比当前视界小三阶。

与此相反,正常物质不能聚集,因为它通过电磁作用与光子紧密耦合,并且正在形成压力波。在宇宙演化的后期,大约在 \( a_{\text{rec}} \sim 1/1100 \) 时,温度变得足够低,光子浴足够稀薄,电子和正电子可以结合形成中性氢。此时,正常物质与辐射解耦,并开始落入暗物质已经开始形成的引力势阱中。从这个意义上说,暗物质是构建宇宙中观察到的“宇宙脚手架”的关键成分。

观测到的物质功率谱 \( P(k) \),或者等价地,方差 \( \Delta^2(k) \),证实了这一历史。如果宇宙只包含重子物质,扰动将具有更小的功率,并且在 \( k \) 约等于$0.1 h Mpc^{-1}$ 附近将展示更明显的振荡(见图1.5的底部右侧),这与观测结果明显不符。一个包含正确比例的重子物质和暗物质的宇宙,即按照方程(1.1)给出的比例,与数据一致。功率谱在 \( k \) 大于大约 \( 3400 \Omega_{\text{rad}} H_0 \) 时改变斜率,对应于在物质/辐射平等时进入视界的尺度,因此在此期间在物质主导下增长。在平滑函数之上的小波纹,称为“重子声学振荡”(BAO),是由于重子物质的次要群体产生的压力波的印记。

作为一个旁注,上述形式主义的一个重要教训是,如果暗物质由具有相当速度 \( v \) 的物体组成,那么它就不会用 \( v_s = 0 \) 来描述——相反,它会扩散,抑制小尺度的不均匀性。观测数据因此限制了暗物质的温度,或者等价地,暗物质的平均动能,要远小于暗物质的质量。对于热化粒子来说,这意味着 \( M > \sim \text{keV} \)。









 

