% 非厄米物理导论
% keys 非厄米
% license Usr
% type Tutor

如果要了解非厄米物理,建议阅读参考文献\cite{Ashida:2020dkc}。

量子物理学的一大假设是,给定系统的哈密顿量是厄米的。厄米性保证了孤立系统的概率守恒,并且保证了一个量子态的能量期望值是实数。但是,在自然界中,由于我们总是会出现一些粒子,能量,或者信息流出给定的希尔伯特空间的情况,所以非厄米物理的研究变得十分重要。因为凡是实际的系统,都会出现耗散现象,这时候就需要处理哈密顿量是非厄米的情况。

关于开放系统的研究,最早可以追溯到伽莫夫等人的工作。他们考虑了活动的原子核的衰变现象,并且给出了这个衰变所对应的非厄米哈密顿量。从这里可以推出,由于衰变导致的流流出原子核的过程。后来,在原子分子物理中,根据这个思路,研究者又开发出了新的理论框架,被称为Feshbach投影,或者是Cohen-Tannoudgi投影方法。

另一个关于开放系统的理论研究框架主要是在量子光学领域,在研究控制量子退相干的过程中被开发出来的。量子物理中一个非常有意思的现象经常出现在一个量子系统中有很多粒子的情况下。在原子分子光物理中,我们已经可以研究这种多体问题的开放系统了。对非厄米系统的研究也促进了非幺正系统的研究。这些研究之前被认为仅仅是理论框架,但是现在已经发现能够具有实际的用途了。

非厄米系统应用得广泛的另外一个领域是不守恒的经典系统。比如说非传统的光学,电子电路力学,声学,流体力学,等等。这个应用的基础是薛定谔的波函数理论。波函数把单粒子的量子力学和经典的波函数画上了等号。这个等价性让我们能够把满足周期性边界条件的系统的固体物理中能带的一些计算应用到经典系统之中。在凝聚态物理中,能带拓扑也可以被延伸到非厄米系统中了。这个比传统的厄米的凝聚态系统的能带理论更能产生更丰富的现象。

\begin{table}[ht]
\centering
\caption{非厄米系统的理论框架和非厄米性的起源}\label{tab_nonHer1}
\begin{tabular}{|c|c|c|}
\hline
系统 & 非厄米性的起源 & 理论方法 \\
\hline
光子学 & * & * \\
\hline
* & * & * \\
\hline
* & * & * \\
\hline
* & * & * \\
\hline
* & * & * \\
\hline
* & * & * \\
\hline
* & * & * \\
\hline
* & * & * \\
\hline
* & * & * \\
\hline
\end{tabular}
\end{table}

