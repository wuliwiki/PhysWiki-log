% 电感的串联和并联
% license Xiao
% type Tutor

\begin{issues}
\issueDraft
\end{issues}

\pentry{电感\nref{nod_Induct}}{nod_5a1c}
设电路里有两个自感系数分别为$L_1,L_2$的电感元件,把它们串联或者并联后,可以看成一个整体,并推算出等效的总电感$L$和$L_1,L_2$的关系。


串联:
\begin{equation}
L = L_1 + L_2~.
\end{equation}
\textbf{proof.}

对于电感元件有:
\begin{equation}\label{eq_IndCmb_1}
U=L\frac{\mathrm d I}{\mathrm d t}~,
\end{equation}
由于串联电路里电流恒定,设$\frac{\mathrm d I}{\mathrm d t}=x$,则有:
\begin{equation}
U=Lx=U_1+U_2=L_1x+L_2x~.
\end{equation}
消去$x$得证。
并联:
\begin{equation}
\frac{1}{L} = \frac{1}{L_1} + \frac{1}{L_2}
\quad \text{或} \quad
L = \frac{L_1L_2}{L_1 + L_2}~.
\end{equation}
\textbf{proof.}

由\autoref{eq_IndCmb_1} 得:$I=\frac{1}{L}\int U\mathrm dt$。因为并联电路的支路总电压相等,设$\int U\mathrm dt=y$,则有
\begin{equation}
I=\frac{1}{L}y=I_1+I_2=\frac{1}{L_1}y+\frac{1}{L_2}y~,
\end{equation}
消去$y$即得证。


