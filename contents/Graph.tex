% 图
% 图
\pentry{二元关系\upref{Relat}}
\begin{issues}
\issueOther{该领域(图论)的相关概念尚不完善}
\issueOther{缺少参考文献,}
\issueTODO
\end{issues}
\subsection{1.基础定义}
\begin{definition}{图;顶点;边}
图$G(V,E)$是集合V上的一种二元关系$E$。

集合$V$的元素称为图的顶点,若两个顶点之间有这种确定的二元关系,则称有一条边连这两个点。

一个图的顶点的数目称为这个图的阶,记 作$|G|$,图的边的数目称为它的度,记作$||G||$。
\end{definition}
\begin{definition}{关联;相邻}
\begin{itemize}
\item 若有一条边连一个图的某两个顶点,则称这两个顶点相邻,并称这两个顶点为这条边的端点。
\item 若某一顶点是某一条边的端点,则称这个顶点和这条边关联。
\item 若两条边和同一顶点关联,则称这两条边相邻.
\end{itemize}
\end{definition}
\subsection{2.特殊图元素}
\begin{definition}{特殊点;特殊边}
\begin{itemize}
\item 两个端点是同一个顶点的边称为环。
\item 若某条边的两个端点不是同一个顶点,且只有一条边连这两个顶点,则称这条边为杆。
\item 以某两顶点为端点的边可能不止一条,这时称连这两个顶点的边为重边。
\end{itemize}
\end{definition}
\begin{definition}{特殊图}
\begin{itemize}
\item 只有一个顶点而没有边的图称为平凡图,没有边的图称为孤立图。
\item 既可以有环,也可以有重边的图称为准图.
\\没有环而可能有重边的图称为带重图.
\\没有重边而可能有环的图称为带环图.
\\既没有重边也没有环的图称为简单图,每两个顶点都相邻的简单图称为完全图。n阶完全图记作$K^{n}$
\item 若一个图的阶是有限的,则称这个图为有限图,否则称这个图为无限图。
\item 若一个n阶图的点用 1 , 2 , … , n 来代表,则称它为标定图
\\若在图的每一条边上赋以一个实数或者对于每个节点赋以一个实数,则称它为赋权图。
\end{itemize}
\end{definition}
\begin{theorem}{n阶完全图$K^{n}$的度}
\begin{equation}
||K^{n}||=\frac{n(n-1)}{2}~.
\end{equation}
证明:使用第一数学归纳法:
\\当n=1时,完全图为孤立图,故||K||=0,下设n时成立,考虑n+1的情形:
\\由于$K^{n+1}$可以由$K^{n}$添加1个顶点及n条边得到知:
\\ $||K^{n+1}||=||K^{n}||+n=\frac{n(n-1)}{2}+n=\frac{n(n+1)}{2}$
\\得证.
\end{theorem}
\subsection{3.点的性质}
\begin{definition}{顶点的次(度)}
设点$v \in V$,称图$G$中以顶点$v$作为端点的边数为点$v$的度,记作$d(v)$
\\注:顶点上的环计数时计两次
\end{definition}
\begin{theorem}{简单图中顶点度与边数的关系}
在简单图$G$中,有:
$\sum_{v \in V}{d(v)}=2||G||$\footnote{有时称为握手引理(Handshaking Lemma)}
\\证明:由简单图中的每一条边有且仅有两个端点,且两个相邻的顶点间仅有一条边立得
\end{theorem}
\begin{corollary}{简单图中奇度点个数}
在简单图$G$中,有:$2\mid \sum\limits_{v\in V\atop 2\nmid d(v)}1$
\end{corollary}

\begin{definition}{与度有关的特殊图元素}

\end{definition}
注:以上内容参考了《数学辞海》卷二