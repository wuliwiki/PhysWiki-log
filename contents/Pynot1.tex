% Python 词法与分析
% Python|注释

\begin{itemize}
\item 注释可用于解释 Python 代码。
\item 确保对模块, 函数, 方法和行内注释使用正确的风格。
\item 注释可用于提高代码的可读性。
\end{itemize}

\begin{itemize}
\item \autoref{Pynot1_sub1} 单行注释
\item \autoref{Pynot1_sub2} 多行注释
\end{itemize}

\subsection{单行注释}\label{Pynot1_sub1}
Python 中单行注释以 \verb|#| 开头,例如:
\begin{lstlisting}[language=python]
# 这是一个注释,注意标准格式要在 \verb|#| 后面加一个空格再加上注释
print("你好,世界!")  # 这是一个注释,注意在代码尾部的注释标准格式要在#的前面在再加两个空格
\end{lstlisting}

\subsection{多行注释}\label{Pynot1_sub2}
Python 实际上没有多行注释的语法。
要添加多行注释,您可以为每行插入一个 \verb|#|
\begin{lstlisting}[language=python]
#这是一个注释
#这是一个注释
#这是一个注释
print("你好,世界!")
\end{lstlisting}

或者,以不完全符合预期的方式,您可以使用多行字符串。由于 Python 将忽略未分配给变量的字符串文字,因此您可以在代码中添加多行字符串,并在其中添加注释
\subsubsection{单引号(''')}
\begin{lstlisting}[language=python]
'''
这是多行注释,用三个单引号
这是多行注释,用三个单引号 
这是多行注释,用三个单引号
'''
print("Hello, World!")
\end{lstlisting}

\subsubsection{双引号(""")}
\begin{lstlisting}[language=python]
"""
这是多行注释,用三个双引号
这是多行注释,用三个双引号 
这是多行注释,用三个双引号
"""
print("Hello, World!")
\end{lstlisting}
