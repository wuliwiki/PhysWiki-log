% 量子力学中的基本算符
% keys 动量算符|角动量算符|哈密顿算符|哈密顿量|能量算符|生成元
% license Xiao
% type Tutor

\pentry{经典力学,量子力学的基本原理(量子力学)\nref{nod_QMPrcp}}{nod_ecd8}

本文中,$\hbar=c=1$。

在\textbf{量子力学的基本原理(量子力学)}的\autoref{ex_QMPrcp_1}~\upref{QMPrcp} 和\autoref{ex_QMPrcp_2} 中,我们不加证明地给出了动量、能量(哈密顿)、角动量算符在给定表象下的形式,相当于进行了定义。本文将讨论如何从经典力学中导出这几个算符的定义。

\subsection{无穷小算符}



考虑无穷小算符总是有益的,因为就是微分线性近似,而线性的东西很简单。

如果要求一个算符在自变量(比如时间)趋于$0$的时候趋于恒等算符,即\textbf{连续性},那么对于\textbf{无穷小}自变量$\varepsilon$后,这个算符总可以写为
\begin{equation}\label{eq_OprQM_1}
U(\varepsilon) = 1-\I G\varepsilon~,
\end{equation}
我们称$G$是$U$的生成元。

\subsubsection{厄米与幺正}

当$G$是一个厄米算符时,有
\begin{equation}
U^{\dagger}(\varepsilon)U(\varepsilon) = (1-\I G\varepsilon)(1+\I G\varepsilon) = 1+o(\varepsilon)~,
\end{equation}
其中$o(\varepsilon)$表示比$\varepsilon$更高阶的无穷小。

因此,$G$是厄米算符$\implies$ $U(\varepsilon)$是\textbf{幺正}算符。

量子态归一化要求可观测量是幺正的,因此可观测量的生成元应该是厄米算符。


\subsubsection{由无穷小算符得出任意算符}

如果要求算符满足$U(t_1)U(t_2)=U(t_1+t_2)$,那么可以从无穷小算符得出任意情况的算符:
\begin{equation}
\ali{
    U(t) &= \lim_{n\to\infty}  U(\frac{t}{n})^n\\
    &= \lim_{n\to\infty} \qty(1-\I G\frac{t}{n})^n\\
    &= \exp\qty(-\I G t)~.
}
\end{equation}



\subsection{位置算符}

位置算符的本征矢都是位置精确给定的态,本征值即对应的位置,因此位置算符是$x_0$或$\bvec{x}_0$,即空间坐标。

\textbf{位置表象}下,位置算符的表示为
\begin{equation}
\leftgroup{
    \bra{x_0}x_0 &= x_0\qquad\text{一维情况}\\
    \bra{\bvec{x}_0}\bvec{x}_0&=\bvec{x}_0\qquad\text{三维情况}
    }~.
\end{equation}

% 处于$x_0$或$\pmat{x_0, y_0, z_0}$的位置本征矢,在\textbf{位置表象下}的波函数为$\delta(x-x_0)$或$\delta(x-x_0)\delta(y-y_0)\delta(z-z_0)$。

\addTODO{动量表象下的位置算符由傅里叶变换给出。}





\subsection{动量算符}

\pentry{平移算符\nref{nod_tranOp}}{nod_5dcf}

由经典力学,动量是平移生成元,\autoref{eq_OprQM_1} 中的$G$应是动量算符。

显然,\textbf{动量表象}下的动量算符就是$p$或$\bvec{p}$,这和位置算符的情况一致。

又由\autoref{eq_tranOp_1}~\upref{tranOp},可知\textbf{位置表象}下一维无穷小平移算符为
\begin{equation}
\bra{x}P(\dd x) = \exp(-\dd x \cdot \partial_x) = 1-\dd x\partial_x~.
\end{equation}

代回\autoref{eq_OprQM_1} ,注意$P(\dd x)$相当于$U(\varepsilon)$,即可得到一维动量算符:
\begin{equation}
p  = -\I\partial_x~.
\end{equation}

三维情况类似可得
\begin{equation}
\uvec{p} = -\I\nabla~.
\end{equation}




\subsection{哈密顿算符}

哈密顿算符$H$即能量算符,较为特殊,取决于我们希望将量子力学应用在什么背景下。

对于经典量子力学,我们考虑的背景是经典力学,因此用经典的能量关系
\begin{equation}
E=\frac{p^2}{2m}+V~.
\end{equation}
来定义哈密顿算子,此时有
\begin{equation}
\bra{x}H = \leftgroup{
    &\frac{-\partial_x^2}{2m}+V\qquad\text{一维情况}\\
    &\frac{-\nabla^2}{2m}+V\qquad\text{三维情况}
}~.
\end{equation}

如果使用狭义相对论的能量关系
\begin{equation}
E^2 = p^2+m^2~,
\end{equation}
那么可以得到哈密顿算子的平方
\begin{equation}\label{eq_OprQM_2}
\bra{x}H^2 = \leftgroup{
    -\partial_x^2&+m^2\qquad\text{一维情况}\\
    -\nabla^2&+m^2\qquad\text{三维情况}
}~.
\end{equation}
由此可以把薛定谔方程调整为\enref{Klein-Gordon 方程}{KGeq}。

Klein-Gordon方程是平方形式,并不协变,且有负能量和负概率密度的问题\footnote{一说负概率密度可以解释为粒子的荷,如电荷。},因此狄拉克(Dirac)想办法把\autoref{eq_OprQM_2} 取了平方根,得到\enref{狄拉克方程}{qed4}。



\subsection{角动量算符}\label{sub_OprQM_1}

角动量算符和经典力学有所不同。经典力学中的角动量对应的是量子力学中的\textbf{空间角动量}算符,是空间转动的生成元。但量子力学中还有\textbf{自旋角动量}的概念,是自旋态空间中的转动的生成元。自旋为$1/2$的系统使用一个二维态空间,自旋为$1$的系统使用一个三维态空间,诸如此类。

经典力学中的空间,可以视为量子力学中位置态的态空间,因此我们可以把自旋态空间和位置态空间合并(取直积)为一个“总空间”,以全面描述量子态的角动量。

给定一个空间转动$R$,规定它与一个总空间中的转动算符相联系\footnote{这里$\mathfrak{D}$是沿用了\cite{Sakurai}的习惯,字母$D$代表德语的“转动”(Drehung)。}:
\begin{equation}
R\to \mathfrak{D}(R)~,
\end{equation}
并规定这是一个\enref{群同态}{Group2}。


\subsubsection{空间角动量算符}

空间角动量算符完全依照经典力学的方式来定义,可参见\enref{轨道角动量(量子力学)}{QOrbAM}。


\subsubsection{无穷小转动与总角动量算符}


在总空间中,一个无穷小转动为
\begin{equation}
R(\dd \phi, k) = 1-\I J_k\dd \phi~.
\end{equation}
其中$k$是所绕的轴,$\dd \phi$是无穷小的转动角度。

更一般地说,空间中绕着单位矢量$\uvec{n}$、角度为无穷小量$\dd \phi$的转动被映射为态空间中的算符
\begin{equation}
\mathfrak{D}(\uvec{n}, \dd \phi) = 1-\I(\bvec{J}\cdot\uvec{n})\dd \phi~.
\end{equation}

按照“角动量是转动生成元”的思想,$J_k=\bvec{J}\cdot \uvec{k}$应该是关于$k$轴的角动量算符。







\subsection{时间演化算符}

\pentry{时间演化算符(量子力学)\nref{nod_TOprt}}{nod_5f14}

时间演化算符的生成元是哈密顿算符$H$,因此时间演化算符为
\begin{equation}
\mathcal{U}(t) = \exp\qty(-\I H t)~.
\end{equation}

时间演化算符的详细讨论请参见\enref{时间演化算符(量子力学)}{TOprt},其应用可参见\enref{路径积分(量子力学)}{PIntQM}、\enref{传播子(量子力学)}{PpgtQM}等。


















