% 一维 delta 势能散射
% keys 薛定谔方程|量子力学|散射|势能
% license Xiao
% type Tutor

\begin{issues}
\issueDraft
\end{issues}

\pentry{一维散射(量子)\nref{nod_Sca1D}}{nod_487d}

假设原点处有一 $\delta$ 势能 $V(x) = a\delta(x)$。 当 $a < 0$ 时它是一个势阱, 存在离散的束缚态, 当 $a > 0$ 时它是势垒, 值存在连续的散射态。

原点的边界条件为
\begin{equation}
\Delta \psi' = 2a \psi~,
\end{equation}

非束缚态的本征函数具有二重简并。 如果取对称和反对称作为基底, 分别为($k > 0$)
\begin{equation}
f_\pm(k, x) = \frac{1}{\sqrt{\pi}}
\begin{cases}
\sin(kx + \phi) & (x > 0)\\
\pm f_\pm(k, -x) & (x < 0)
\end{cases}
\qquad  \phi = \arctan(k/a)~,
\end{equation}
\begin{equation}
f_1(k, x) = \frac{1}{\sqrt{\pi}}\sin(kx)~.
\end{equation}
满足归一化条件
\begin{equation}\label{eq_Dsc1D_4}
\braket{f_i(k')}{f_j(k)} = \delta_{i,j} \delta(k - k')~,
\end{equation}
我们把这样的解叫做\textbf{驻波(standing wave)}。

现在我们试图变换每个二维简并空间中的基底, 令
\begin{equation}\ali{
&h_1(k, x) =  C_{11} f_1(k, x) + C_{12} f_0(k, x) ~,\\
&h_2(k, x) =  C_{21} f_1(k, x) + C_{22} f_0(k, x)~,
}\end{equation}
若要满足同样的归一化条件
\begin{equation}\label{eq_Dsc1D_6}
\braket{h_{i'}(k')}{h_i(k)} = \delta_{i,i'}\delta(k - k')~.
\end{equation}
其中
\begin{equation}
\braket{h_{i'}(k')}{h_i(k)} = \sum_{j'} C_{i',j'}^* \bra{f_{j'}(k')} \sum_j C_{i, j} \ket{f_j(k)} = \sum_j C_{i', j}^* C_{i, j} \delta(k - k')~,
\end{equation}
系数需要满足
\begin{equation}
 \sum_j C_{i', j}^* C_{i, j} = \delta_{i, i'}~,
\end{equation}
即变换矩阵 $\mat C$ 必须是一个酉矩阵。

我们先来寻找代表从左边入射并向左边反射且向右边出射的基底 $h_1^{(+)}$, 令边界条件为
\begin{equation}
h_1^{(+)}(x) \propto \exp(\I k x) \quad (x > 0)~.
\end{equation}
解得系数分别为\footnote{注意求解的时候列出的是齐次方程组, 所以该式中的系数同乘一个任意相位因子仍然是解, 这里选择相位因子使得\autoref{eq_Dsc1D_11} 中的入射波具有最简洁的形式。}
\begin{equation}
C_1 = \frac{-\I\E^{\I \phi}}{\sqrt{2}}~, \qquad
C_2 =  \frac{\I}{\sqrt{2}}~.
\end{equation}

代入得(仍然要求 $k > 0$)
\begin{equation}\label{eq_Dsc1D_11}
h_1^{(+)}(k, x) =  \frac{1}{\sqrt{2\pi}}
\begin{cases}
\E^{\I kx} - \cos\phi \E^{\I\phi} \E^{-\I k x} & (x < 0) \\
-\I\sin\phi\E^{\I\phi} \E^{\I kx}  & (x > 0)
\end{cases}~.
\end{equation}

要得到 $h_2$, 我们只需要补全矩阵 $\mat C$ 的第二行, 使其与第一行正交, 得到\footnote{矩阵的第二行同样也可以乘以任意相位因子, 我们选择该因子是的\autoref{eq_Dsc1D_13} 的入射波最简洁。}
\begin{equation}
\mat C = \frac{1}{\sqrt{2}}\pmat{
-\I\E^{\I \phi} & \I \\
-\I\E^{\I \phi} & -\I
}~.\end{equation}

代入得($k > 0$)
\begin{equation}\label{eq_Dsc1D_13}
h_2^{(+)}(k, x) =  \frac{1}{\sqrt{2\pi}}
\begin{cases}
-\I\sin\phi\E^{\I\phi}\E^{-\I kx} &(x < 0) \\
\E^{-\I kx} - \cos\phi \E^{\I \phi} \E^{\I kx} &(x > 0)
\end{cases}~.
\end{equation}
可以直接用归一化积分验证正交归一化条件\autoref{eq_Dsc1D_6} 和原点处的边界条件\autoref{eq_Dsc1D_4} (如果边界条件是线性的, 符合边界条件的函数线性叠加仍然符合边界条件)。

可以发现, $h_1(k, x)$ 和 $h_2(k, x)$ 分别是两个\textbf{入射波(incoming wave)}, 即从所有方向入射, 一个方向出射。

我们只需把 $h_1(k, x)$ 和 $h_2(k, x)$ 取复共轭, 即可得到两个方向的\textbf{出射波(outgoing wave)}, 即从一个方向入射, 各个方向出射。 相对于驻波, 可以称入射波和出射波为\textbf{行波(travelling wave)}。

从入射波到出射波的变换矩阵为
\begin{equation}
\mat T = \pmat{
\E^{\I\phi}\cos\phi & -\I \sin\phi \\
-\I \sin\phi & \E^{-\I\phi}\cos\phi
}~.\end{equation}

在三维问题中, 我们同样可以得到驻波, 入射波和出射波三组不同的本征波函数。

\subsection{波包的散射}
无论是驻波还是行波都不具有物理意义, 因为一个粒子不可能有无限细的能量带宽和无限大的位置分布。 我们现在来考虑如果初始时刻我们有一个波包, 这个波包会如何移动。

我们可以选取一组基底将波包进行分解, 写出各个基底随时间变化的形式, 再进行线性组合即可得到波包随时间的变化。

本文中的例子有一个特殊的地方在于 $\delta$ 函数在原点产生的边界条件要求波函数空间中的所有波函数都必须满足(因为所有的基底都必须满足)。 所以如果初始波包在原点处不为零, 那么初始波包也需要满足原点的边界条件。 若是其他有限函数如方势垒和方势阱, 则没有这个限制。 但一般来说这个限制并不重要, 因为我们大部分情况下考虑波包从无穷远处入射, 初始时刻原点的波函数为零。


