% 氦原子数值解 TDSE

使用角向基底为两个电子总轨道角动量和 $z$ 分量的本征态\upref{AMAdd}
\begin{equation}
\mathcal{Y}_{l_1,l_2}^{L,M} = \sum_{m_1, m_2} C_{l_1 m_1 l_2 m_2}^{L,M} Y_{l_1 m_1}(\uvec r_1) Y_{l_2 m_2} (\uvec r_2)
\end{equation}
系数为 C-G 系数, $m_1 + m_2 \ne M$ 的 C-G 系数全为零.

总波函数所在的空间可以看做角向空间和径向空间的张量积空间. 现在有了角向空间的基底, 总波函数就可以在该基底上展开
\begin{equation}
\ket{\Psi} = \sum_\lambda \ket{R_\lambda}\ket{\mathcal{Y_\lambda}}
\end{equation}
其中 $\lambda$ 是将所有的 $(l_1,l_2,L,M)$ 组合排序后的序号. $\ket{R_\lambda}$ 是二维径向波函数
\begin{equation}
\ket{R_\lambda} = \frac{1}{r_1 r_2} \psi_\lambda(r_1, r_2)
\end{equation}

由张量积空间\upref{DirPro}中的结论, 我们可以求哈密顿算符 $H$ 关于角向基底的“矩阵元” $H_{\lambda, \lambda'}$, 每个矩阵元是径向空间中的一个算符. 列出薛定谔方程的“矩阵形式”, 就得到了一组 couple 的径向波函数的薛定谔方程
\begin{equation}
\sum_\lambda' H_{\lambda, \lambda'} \ket{R_{\lambda'}} = \I \pdv{t} \ket{R_{\lambda'}}
\end{equation}

总哈密顿可以表示为
\begin{equation}
H = H_1 + H_2 + V_{12} + V_{int}
\end{equation}
其中 $H_i \ \ (i = 1, 2)$ 是单个电子的哈密顿算符, 对应对角矩阵.
\begin{equation}
H_i = K_{ri} + \frac{L_i^2}{2m r_i^2} - \frac{2}{r_i}
\end{equation}
其中第二项是对角矩阵是因为 $\mathcal Y$ 基底是 $L_i^2$ 的本征函数.

存在 coupling 的项自然是两电子之间的库仑势能 $V_{12}$
\begin{equation}
\ali{
V_{12} = \frac{1}{\abs{\vec r_2 - \vec r_1}} &= 4\pi \sum_{l,m} \frac{1}{2l+1} \frac{r_<^l}{r_>^{l+1}} Y_{l,m}^*(\uvec r_1) Y_{l,m}(\uvec r_2)\\
&= 4\pi \sum_l \frac{(-1)^l}{\sqrt{2l + 1}} \frac{r_<^l}{r_>^{l+1}} \mathcal{Y}_{l,l}^{0,0} (\uvec r_1, \uvec r_2)
}\end{equation}
要计算矩阵元 $\mel{\mathcal Y_\lambda}{V_{12}}{\mathcal Y_{\lambda'}}$, 就要对六个球谐函数的乘积的线性组合做两次角向积分. 使用\autoref{SphCup_eq1}\upref{SphCup} 可以将上式表示为六个 CG 系数相乘的线性组合(二重求和).
