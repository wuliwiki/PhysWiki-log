% 原子结构(高中)
% keys 原子结构|原子核式结构|原子核|衰变|裂变|聚变|粒子|波尔模型|氢原子光谱|结合能|放射性
% license Xiao
% type Tutor
\begin{issues}
\issueTODO
\end{issues}

% 或者应该把粒子波动性和量子力学初步放在这一章里,主要就是量子力学的内容
% 然后原子结构(核式结构和波尔模型)独立出来,先说整体的原子结构,再说后续的原子核放射性,和书本下一章做结合
% 开篇就是原子的结构 然后核式结构 然后电子的运动规律(光谱) 然后原子核的聚变裂变反应
\subsection{原子结构}
\subsubsection{原子的核式结构模型}
我们知道,物质由原子(或者分子)组成,那么原子是否是不可分的最小单元,是什么组成了原子,组成原子的部分又是如何有机的组合在一起的呢?
为了逐一回答这些问题,一系列的实验逐步向我们揭开原子内部的面纱,科学家们通过对这些实验进行合理解释向我们描述出一个有趣的原子结构。

首先,汤姆孙在阴极射线实验
思考题:为什么要稀薄气体
\subsubsection{波尔原子模型的提出}% 这一节讲一点点量子力学那里


\subsection{原子核}
\subsubsection{原子核的组成}
\subsubsection{认知基本粒子}
\subsubsection{原子衰变}
\subsubsection{核力和结合能}
\subsubsection{核裂变与核聚变}