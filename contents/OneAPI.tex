% Intel OneAPI(含 MKL)安装笔记(Linux 和 WSL2)
% license Xiao
% type Note

\pentry{Lapack 笔记\upref{Lapack}}{nod_b5e9}

如果要旧的 icc/icpc\footnote{注意 icc/icpc (Intel C++ Compiler Classic) 将在 2023 年被淘汰, 被 oneAPI DPC++/C++ Compiler 代替, 命令是 \verb|icx/icpx|, 编译选项有略微的差异。} 以及 mkl, 可以\textbf{先用 oneAPI BASE 只安装 mkl (intel gdb, python 可能也有用), 然后再安装 oneAPI HPC 中的编译器(MPI 可能也有用)}。
装错了有没关系, 可以随时添加或者删除组件。 安装的 GUI 界面可以通过 X11 传输, 所以远程 ssh 也有 GUI 安装。 如果不能用 GUI 会自动切换到 TUI。

也可以通过命令行选项进行非交互的安装, 详见\href{https://www.intel.com/content/www/us/en/develop/documentation/installation-guide-for-intel-oneapi-toolkits-linux/top/installation/install-with-command-line.html#install-with-command-line}{这里}。 例如
\begin{lstlisting}[language=bash]
# 安装 MKL
sudo sh ./l_BaseKit_p_*.sh -a --silent --action install \
    --eula accept --components intel.oneapi.lin.mkl.devel
# 安装 icpc 和 MPI
sudo sh ./l_HPCKit_p_*.sh -a --silent --action install \
    --eula accept --components \
    intel.oneapi.lin.dpcpp-cpp-compiler-pro:intel.oneapi.lin.mpi.devel
\end{lstlisting}

安装完以后需要在 \verb|~/.bashrc| 中加上 \verb|source /opt/intel/oneapi/setvars.sh|。

注意再分配内存时,对齐可以让 SIMD 优化得更好。 MKL 提供自己版本的 \verb|malloc|,也可以用 STL 的对齐机制(\verb|std::align|)。

\subsection{oneAPI BASE}
\begin{itemize}
\item 注意所有 oneAPI 都只提供最\href{https://www.intel.com/content/www/us/en/developer/tools/oneapi/base-toolkit-download.html?operatingsystem=linux&distributions=webdownload&options=offline}{新版本的下载}, 所以尽量把离线安装包保存到本地
\item 安装和卸载都用 \verb`sudo sh ./l_BaseKit_p_2022.2.0.262_offline.sh` 在 WSL2 中会弹出 GUI。 光 MKL 需要 7.4GB。
\item 注意 WSL1 用 .sh 安装包安装会出错。 WSL2 成功。
\item 注意 \verb|oneAPI_BASE| 不包括 icpc 和 icc 编译器! 他们并不包含在 DPC++ 里面。 \href{https://stackoverflow.com/questions/66527842/can-not-find-the-icc-compiler-after-having-installed-intel-oneapi-invoking-from}{详见这里}。
\item MKL 和 icpc 安装完需要 8.6G。
\item 目前 20220807 最新版仅支持 ubuntu 的 20.04 以及一些更早版本
\end{itemize}

\subsection{oneMKL}
\begin{itemize}
\item 这个文件仅仅安装 MKL, 不包含编译器等
\item 安装和卸载都用 \verb`sudo sh ./l_onemkl_p_2022.1.0.223_offline.sh`
\end{itemize}

\subsection{oneAPI HPC}
\begin{itemize}
\item \href{https://www.intel.com/content/www/us/en/developer/tools/oneapi/hpc-toolkit-download.html?operatingsystem=linux&distributions=offline}{oneAPI HCP}, 包括 icpc/ifort 编译器和 MPI 等, 但不包括 MKL!
\item 光装一个 C++ 编译器需要 6G!
\item 安装和卸载都用 \verb`sudo sh ./l_HPCKit_p_2022.2.0.191_offline.sh`
\end{itemize}
