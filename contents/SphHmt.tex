% 球面的同伦群
% keys 球面|代数拓扑|同伦|伦型|群|homology|homology group|topology|group
% license Xiao
% type Tutor

%未完成;表头没画出来,因此表格表达不清。
\pentry{高阶同伦群\nref{nod_HomT4}}{nod_2cd8}

以各球面 $S^n$ 为拓扑空间,则它们的各阶同伦群 $\pi_i$ 列举如下\footnote{文献来源:Allen Hatcher, Algebraic Topology, 2002 by Cambridge University Press, pp.339(Section 4.1)。}:(第一行数字表示同伦群的阶数 $i$,第一列数字表示球面的维度 $n$)

\begin{table}[ht]
\caption{$S^n$ 球面的同伦群 $\pi_i$($0$ 表示只有一个元素的平凡群)}\label{tab_SphHmt_1}
\begin{tabular}{|c|c|c|c|c|c|c|c|c|c|c|c|c|}
\hline
 & 1 & 2 & 3 & 4 & 5 & 6 & 7 & 8 & 9 & 10 & 11 & 12 \\
\hline
1 & $\mathbb{Z}$ & 0 & 0 & 0 & 0 & 0 & 0 & 0 & 0 & 0 & 0 & 0 \\
\hline
2 & 0 & $\mathbb{Z}$ & $\mathbb{Z}$ & $\mathbb{Z}_2$ & $\mathbb{Z}_2$ & $\mathbb{Z}_{12}$ & $\mathbb{Z}_2$ & $\mathbb{Z}_2$ & $\mathbb{Z}_3$ & $\mathbb{Z}_{15}$ & $\mathbb{Z}_2$ & $\mathbb{Z}_2\times\mathbb{Z}_2$\\
\hline
3 & 0 & 0 & $\mathbb{Z}$ & $\mathbb{Z}_2$ & $\mathbb{Z}_2$ & $\mathbb{Z}_{12}$ & $\mathbb{Z}_2$ & $\mathbb{Z}_2$ & $\mathbb{Z}_3$ & $\mathbb{Z}_{15}$ & $\mathbb{Z}_2$ & $\mathbb{Z}_2\times\mathbb{Z}_2$ \\
\hline
4 & 0 & 0 & 0 & $\mathbb{Z}$ & $\mathbb{Z}_2$ & $\mathbb{Z}_2$ & $\mathbb{Z}\times\mathbb{Z}_{12}$ & $\mathbb{Z}_2\times\mathbb{Z}_2$ & $\mathbb{Z}_2\times\mathbb{Z}_2$ & $\mathbb{Z}_{24}\times\mathbb{Z}_3$ & $\mathbb{Z}_{15}$ & $\mathbb{Z}_2$ \\
\hline
5 & 0 & 0 & 0 & 0 & $\mathbb{Z}$ & $\mathbb{Z}_2$ & $\mathbb{Z}_2$ & $\mathbb{Z}_{24}$ & $\mathbb{Z}_2$ & $\mathbb{Z}_2$ & $\mathbb{Z}_2$ & $\mathbb{Z}_{30}$ \\
\hline
6 & 0 & 0 & 0 & 0 & 0 & $\mathbb{Z}$ & $\mathbb{Z}_2$ & $\mathbb{Z}_2$ & $\mathbb{Z}_{24}$ & 0 & $\mathbb{Z}$ & $\mathbb{Z}_2$ \\
\hline
7 & 0 & 0 & 0 & 0 & 0 & 0 & $\mathbb{Z}$ & $\mathbb{Z}_2$ & $\mathbb{Z}_2$ & $\mathbb{Z}_{24}$ & 0 & 0 \\
\hline
8 & 0 & 0 & 0 & 0 & 0 & 0 & 0 & $\mathbb{Z}$ & $\mathbb{Z}_2$ & $\mathbb{Z}_2$ & $\mathbb{Z}_{24}$ & 0 \\
\hline
\end{tabular}
\end{table}

该表格有两个很明显的规律:第一,当同伦群的阶数 $i$ 小于球面的维度 $n$ 的时候,同伦群一定是平凡群;第二,当同伦群的阶数等于球面的维度的时候,同伦群一定是正整数群 $\mathbb{Z}$。
