% 暗物质引言
% keys 暗物质
% license Usr
% type Tutor

在过去的世纪里,我们理解了日常物质的结构。量子力学和相对论教会我们,我们观察到的“物质”的本质常常与我们的直觉不同。这是否意味着物质的科学探索已经结束了呢?还没有。宇宙中还有更多的物质有待理解,这就是被称为暗物质的东西。我们对物质的探索从地球转向了太空。暗物质的存在已经通过从星系到宇宙学尺度的观测得到了证实。然而,这些观测只探测到了暗物质的引力耦合-总质量和其空间分布。为了真正理解暗物质是什么,我们还需要观察它与普通物质的其他可能的相互作用。因此,暗物质领域吸引了大量的关注:目前大约10\%的宇宙学、天体物理学和粒子物理学的论文提到了“暗物质”。

来自暗物质的证据来自天文尺度,从小于千帕秒(小星系的典型大小)到整个可观测宇宙的大小。人们通常关注三个主要的证据:螺旋星系的观测、星系团的观测以及宇宙微波背景(CMB)和宇宙大尺度结构在宇宙学尺度上的观测。这些观测可以一致地解释为具有以下基本属性的暗物质。

所有这些证据都涉及到暗物质的引力效应。目前还没有基于暗物质其他效应(例如,不是由于引力相互作用)的证据。一个自然的替代方案是考虑修改引力,以解释通常归因于暗物质的现象。我们将讨论限制在标准假设上;即暗物质是由未发现的物质粒子组成的,无论是新的基本粒子还是新的宏观物体。

当前宇宙平均暗物质密度通常以参数组合$\Omega_{DM}h^2$表示,其中$\Omega_i = \rho_i/\rho_{cr}$是相对于临界密度$\rho_{cr} = 3H2/8\pi G$的密度,$G$是牛顿常数,现在的哈勃参数写为$H_0 = h \times 100 km/s \cdot Mpc$,其中$h = 0.674 \pm 0.005$。当前最好的确定值是:

\begin{equation}
\Omega_{DM}h^2 = 0.1200 \pm 0.0012~.
\end{equation}

这可以转换为$\Omega_{DM} = 0.264 \pm 0.003$,即暗物质占当前宇宙总物质-能量内容的约26.4\%。由于正常重子物质的密度测量值为$\Omega_b h^2 = 0.02237 \pm 0.00015$,或4.9\%,暗物质占总物质内容的约84\%。此外,当前宇宙中的平均暗物质密度约为$\rho_{DM} \simeq 1.26 keV/ cm^3$。

天体物理和宇宙学观测对暗物质一般属性的认定:数据可以认定暗物质是一种冷的、非相互作用的、稳定的物质,具有绝热非均匀性。

物质:强调暗物质在宇宙演化中表现为物质,即其密度随着体积的倒数而减少。在技术术语中,其状态方程参数是$w = 0$。这与暗能量形成对比,据我们所知,暗能量的密度在宇宙膨胀时不会显著稀释。
冷:意味着暗物质在物质-辐射平等的关键时期表现为非相对论性流体,结构形成开始时,以及随后的星系形成期间。暗物质粒子移动得“慢”,因此它们相互吸引并聚集。如果它们移动得“快”,聚集就不会有效,结构就不会形成和增长。聚集过程确实会使暗物质流体重新加热,因为其组成粒子在坍缩成束缚结构后获得了动能。然而,这还不足以使暗物质相对论化,并且使暗物质的一般冷性失效,除非在极端环境中,例如在紧凑物体周围。

非相互作用(或等价地,无碰撞)意味着暗物质与自身或与普通物质(除了暗物质肯定拥有的引力相互作用之外)的相互作用足够小,可以忽略不计。这区分了暗物质和普通物质。与普通物质有显著的相互作用不同,特别是电磁相互作用,暗物质没有。这就是暗物质之所以被称为“暗”的原因——缺乏与光的相互作用使暗物质变暗。

稳定意味着暗物质自宇宙早期阶段以来就存在,并且直到现在(宇宙当前的年龄约为$t_{Uni} \simeq 13.8 Gyr = 4.35 \times 10^{17} s$)都没有消失。根据暗物质的性质,这将转化为对暗物质粒子(如果由不稳定粒子组成)的衰变寿命或黑洞(如果受到霍金辐射影响)的蒸发率等的定量限制。

绝热意味着在宇宙尺度上,宇宙流体的组成在任何地方都是相同的。暗物质具有与其他组分相同的原始密度非均匀性:暗物质在普通物质和光子密度更高的地方更密集。这发生在所有非均匀性由单一机制产生的情况下。这种机制被认为是宇宙膨胀期间单一膨胀场的量子涨落。所有组分似乎都有一个高斯和准尺度不变的原始涨落谱。