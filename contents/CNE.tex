% 线性泛函的保范延拓
% keys 保范|延拓|线性泛函
% license Usr
% type Tutor

\pentry{线性泛函的延拓\nref{nod_ExLina},赋范空间\nref{nod_NormV}}{nod_b522}

由一般线性泛函延拓的\enref{Hahn-Banach定理}{ExLina},可以得到赋范空间中线性泛函延拓定理的表述,进而很容易得到,赋范空间中,定义在子空间上的线性泛函可以保范的延拓到整个空间上去。详情见下文。

\subsection{保范延拓}

\begin{theorem}{}\label{the_CNE_1}
设 $L$ 是实赋范空间 $E$ 的子空间,$f_0$ 是 $L$ 上的有界线性泛函,则存在 $f_0$ 在 $E$ 上的保范\enref{延拓}{ExLina} $f$,即 $\norm{f}_{E}=\norm{f_0}_{L},f(x)=f_0(x),x\in L$。
\end{theorem}

\textbf{证明:}令 $k=\norm{f_0}_{L}$,则 $k\norm{x}$ 是\enref{齐次凸泛函}{ConFul}。由于 $\abs{f_0(x)}\leq k\norm{x}$,根据 \enref{Hahn-Banach延拓定理}{ExLina},存在 $f_0$ 的在 $E$ 上延拓 $f$,使得 $f(x)\leq k\norm{x}$。进而
\begin{equation}
\norm{f}_E=\sup_{x\neq0}\frac{\abs{f(x)}}{\norm{x}}\leq\sup_{x\neq0}\frac{k\norm{x}}{\norm{x}}=k=\norm{f_0}_L.~
\end{equation}
 
又 $\norm{f}_E\geq \norm{f_0}_L$(因为它们在 $L$ 上一致,由范数定义得到该式),所以$\norm{f}_{E}=\norm{f_0}_{L}$ 。

\textbf{证毕!}


\begin{corollary}{}\label{cor_CNE_1}
设 $x_0$ 是赋范空间 $X$ 中的非零元素,则在 $X$ 上存在线性连续泛函 $f$,使得
\begin{equation}
\norm{f}=1,f(x_0)=\norm{x_0}.~
\end{equation}

\end{corollary}

\textbf{证明:}定义在由  $x_0$ 生成的一维子空间上的线性泛函 $f_0(\alpha x_0)=\alpha\norm{x_0}$,由\autoref{the_CNE_1} ,它可保范延拓到整个 $X$ 上的泛函 $f$,且 
\begin{equation}
\norm{f}=\norm{f_0}=1,\quad f(x_0)=f_0(x_0)=\norm{x_0}.~
\end{equation}


\textbf{证毕!}

