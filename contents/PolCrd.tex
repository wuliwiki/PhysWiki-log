% 极坐标中的曲线方程
% keys 极坐标|曲线方程|切线|求导
% license Xiao
% type Tutor

\begin{issues}
\issueDraft
\end{issues}

\pentry{极坐标\nref{nod_Polar}}{nod_d53d}

极坐标系中, 我们可以用一元函数
\begin{equation}\label{eq_PolCrd_1}
r = f(\theta)~
\end{equation}
表示一条曲线。 简单的例子如\enref{阿基米德螺线}{ArcSpl}或\enref{圆锥曲线}{Cone}。 以下我们讨论如何从函数中计算曲线的一些特征。

\subsection{计算某点切线的方向}
\pentry{导数\nref{nod_Der}}{nod_55f5}

% 图未完成

在 $x$-$y$ 平面直角坐标系中, 我们可以通过求导($\dv*{y}{x}$)计算曲线某点切线, 那么\autoref{eq_PolCrd_1} 表示的极坐标中如何求某点切线的方向呢? 可以证明, 切线与 $\uvec\theta$ 方向的夹角为
\begin{equation}
\tan\alpha = \frac{1}{r} \dv{r}{\theta} = \frac{f'(\theta)}{f(\theta)}~,
\end{equation}
或者说与 $\uvec r$ 方向的夹角为 $\pi/2 - \alpha$。

\begin{figure}[ht]
\centering
\includegraphics[width=8cm]{./figures/b714fe71ebcd2028.png}
\caption{极坐标切线示例} \label{fig_PolCrd_1}
\end{figure}

证明:
对极坐标中的位置矢量$\bvec r = r \uvec r$两边取全微分,得到:
\begin{equation}
\dd{\bvec r} = \dd r \uvec r + r\dd{\uvec r}~.
\end{equation}
由于:
\begin{equation}\label{eq_PolCrd_2}
\dd{\uvec r} = \pdv {\uvec r}{r} \dd r + \pdv {\uvec r}{\theta} \dd\theta~.
\end{equation}

其中,$\pdv {\uvec r}{r}$ 和 $\pdv {\uvec r}{\theta}$ 可由\enref{极坐标系中单位矢量的偏导}{DPol1}处证得,结果带入\autoref{eq_PolCrd_2} ,得到:
\begin{equation}\label{eq_PolCrd_6}
\dd{\bvec r} = \dd r \uvec r+ r \dd \theta\uvec{\theta}~.
\end{equation}
上式中,$\dd{\bvec r}$的方向,即是切线的方向。切线与 $\uvec\theta$ 方向的夹角为:
\begin{equation}
\tan \alpha = \pdv{\bvec r}{\uvec r}\left/ \pdv{\bvec r}{\uvec\theta}\right.
=\frac{1}{r} \dv{r}{\theta} = \frac{f'(\theta)}{f(\theta)}~.
\end{equation}


\subsection{曲线长度}
\pentry{定积分\nref{nod_DefInt}}{nod_f59b}

若用 $\theta \in [\theta_1, \theta_2]$, 来表示曲线的一段, 那么其长度为
\begin{equation}\label{eq_PolCrd_8}
l = \int_{\theta_1}^{\theta_2} \sqrt{f(\theta)^2 + f'(\theta)^2} \dd{\theta}~.
\end{equation}

证明:继\autoref{eq_PolCrd_6} ,

\begin{equation}
\begin{split}
\dd{l} = |\dd{\bvec{r}}| &= \sqrt{(\dd{r})^2 + (r \dd{\theta})^2} \\
&= \sqrt{\left(\dv{r}{\theta}\right)^2 + r^2} \dd{\theta} \\
&= \sqrt{f(\theta)^2 + f'(\theta)^2} \dd{\theta}~.
\end{split}
\end{equation}

对区间 $[\theta_1,\theta_2]$ 积分后,即可求得该段曲线的长度,即\autoref{eq_PolCrd_8}。

\subsection{直线的极坐标方程}

设极坐标为$(\rho,\theta)$,过点$(a,0)$,且与极轴夹角为$\alpha$的直线极坐标方程为:
\begin{equation}\label{eq_PolCrd_3}
\rho\sin(\alpha-\theta)=a\sin\alpha~.
\end{equation}
通过调整变形,可以讨论不同情况下的特殊形式:

1. 直线过极点

此时$a=0$,代入\autoref{eq_PolCrd_3} 可知,此时表达式为:
\begin{equation}
\theta=\alpha~.
\end{equation}

2.与极轴垂直,极点到直线距离为$|a|$。

此时$\displaystyle\alpha={\pi\over2}$,代入\autoref{eq_PolCrd_3} 可知,此时表达式为:
\begin{equation}
\rho\cos\theta=a~.
\end{equation}
若$a>0$则直线在极点右侧,否则直线在极点左侧。

3.与极轴平行,极点到直线距离为$|a|$。

此时由于直线不再过点$(a,0)$,于是不能直接代入得到表达式。根据此时直线表达式为$y=a$,而极坐标中$y=\rho\sin\theta$,可知,此时表达式为:
\begin{equation}
\rho\sin\theta=a~.
\end{equation}
若$a>0$则直线在极点上侧,否则直线在极点下侧。




