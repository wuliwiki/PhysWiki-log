% 极坐标中的曲线方程
% 极坐标|曲线方程|切线|求导

\begin{issues}
\issueDraft
\end{issues}

\pentry{极坐标\upref{Polar}}

极坐标系中, 我们可以用一元函数
\begin{equation}\label{eq_PolCrd_1}
r = f(\theta)
\end{equation}
表示一条曲线。 简单的例子如阿基米德螺线\upref{ArcSpl} 或圆锥曲线\upref{Cone}。 以下我们讨论如何从函数中计算曲线的一些特征。

\subsection{计算某点切线的方向}
\pentry{导数\upref{Der}}

% 图未完成

在 $x$-$y$ 平面直角坐标系中, 我们可以通过求导($\dv*{y}{x}$)计算曲线某点切线, 那么\autoref{eq_PolCrd_1} 表示的极坐标中如何求某点切线的方向呢? 可以证明, 切线与 $\uvec\theta$ 方向的夹角为
\begin{equation}
\alpha = \frac{1}{r} \dv{r}{\theta} = \frac{f'(\theta)}{f(\theta)}
\end{equation}
或者说与 $\uvec r$ 方向的夹角为 $\pi/2 - \alpha$。

\addTODO{推导}

\subsection{曲线长度}
\pentry{定积分\upref{DefInt}}

若用 $\theta \in [\theta_1, \theta_2]$, 来表示曲线的一段, 那么其长度为
\begin{equation}
l = \int_{\theta_1}^{\theta_2} \sqrt{f(\theta)^2 + f'(\theta)^2} \dd{\theta}
\end{equation}

\addTODO{推导}
