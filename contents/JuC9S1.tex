% Julia 类型
% keys 数组 类型
% license Xiao
% type Tutor

本文授权转载自郝林的 《Julia 编程基础》。 原文链接:\href{https://github.com/hyper0x/JuliaBasics/blob/master/book/ch09.md}{第 9 章 容器:数组(上)}。


\subsubsection{9.1 类型}

代表数组的具体类型名为\verb`Array`,它是\verb`AbstractArray`的直接子类型。Julia针对\verb`AbstractArray`类型定义了大量且丰富的操作。因而,\verb`Array`类型也就很自然地成为了这些操作的有效目标。

我们已经知道,\verb`Array`是一个参数化类型,它的全名是\verb`Array{T,N}`。其中的类型参数\verb`T`用于确定数组的元素类型,而类型参数\verb`N`则用于确定数组的维数。这里的\verb`N`的取值通常是一个正整数(也可以是\verb`0`,表示零维数组)。并且,在64位的计算机系统中,它的值不能超出\verb`Int64`类型所能表示的数值范围;在32位的计算机系统中,它的值不能超出\verb`Int32`类型所能表示的数值范围。下面是一些示例:

\begin{lstlisting}[language=julia]
julia> Array{Float64,3}
Array{Float64,3}

julia> Array{Int64,N} where N
Array{Int64,N} where N

julia> Array{Int64}
Array{Int64,N} where N

julia> Array{T,typemax(Int32)} where T
Array{T,2147483647} where T

julia> 
\end{lstlisting}

在一般情况下,我们直接使用的数组的维数都不会太多,大多在三维及以下。尽管在一些程序中可能会用到拥有更多维度的数组,但其维数肯定也比\verb`Int32`类型所能表示的最大值要小得多。所以,这里的类型参数\verb`N`的取值范围对于我们来说相当于没有限制。

正因为一维数组和二维数组都太常用了,所以Julia为它们的类型提供了别名。别名\verb`Vector{T}`代表类型\verb`Array{T,1}`,也就是一维数组的类型。而别名\verb`Matrix{T}`则代表了\verb`Array{T,2}`,即二维数组的类型。其中的vector(向量)和matrix(矩阵)都是线性代数中最核心的概念。从形状上来讲,向量就是由一个个值组成的纵队,而矩阵则是由一个个长度相同的纵队组成的方阵。

顺便说一下,我们在本书中不会专门去讨论相关的数学知识。但是,我们有时候(尤其是讲数组的时候)却不得不提到一点,因为有些对象及其操作基于的正是那些数学概念。不过别担心,我会尽量用精炼、朴实的语言去描述它们。

我们再来说数组类型的其他特点。与元组类型不同,数组类型的字面量永远也无法体现出元素的顺序。这主要是因为数组类型中只有一个可以代表元素类型的参数。想想看,如果一个元组类型的所有参数值全都相同,那么它同样无法体现出元素的顺序。

另外,数组类型也无法体现出其元素的数量。因此,对于一个一维数组,我们可以随意地增减其中的元素值,而不用担心不符合其类型的约束。如:

\begin{lstlisting}[language=julia]
julia> isa([1], Array{Int64,1})
true

julia> isa([1,2,3], Array{Int64,1})
true

julia> isa([1,2,3,4,5], Array{Int64,1})
true

julia> 
\end{lstlisting}

然而,对于多维数组来说,其各个维度上的元素数量却不是随意的。更确切地说,在一个多维数组中,处在同一个维度上的所有低维数组(即维数更低的数组)都应该具有相同的尺寸。这就好比一个方阵,其中的所有纵队的长度都需要相同。又好比一个六面体,它的每一个面都应该是平面,既不能有任何的凹陷,也不能有任何的凸出。只有符合这种规则的数组才能被称为多维数组,其类型如\verb`Array{Int64,2}`。否则,那个数组就只能算是多个数组的嵌套而已,其类型如\verb`Array{Array{Int64,1},1}`。

除了类型字面量上的一些特点,数组类型还具有非转化的特性。因此,\verb`[]`和\verb`[1]`虽然同为一维数组,但是它们的类型之间却不存在继承关系。这是由于空数组\verb`[]`的类型是\verb`Array{Any,1}`,它不是\verb`Array{Int64,1}`的超类型。验证的代码如下:

\begin{lstlisting}[language=julia]
julia> typeof([1]) <: typeof([])
false

julia> Array{Int64,1} <: Array{Any,1}
false

julia> 
\end{lstlisting}

到这里,我们知道了\verb`Array{T,N}`类型中各个类型参数的取值范围,还知道了最常用的一维数组和二维数组的类型别名。另外,我们还了解到,数组类型的字面量上只会体现它的元素类型和维数,而不会体现元素的顺序以及各个维度上的元素数量。尽管如此,多维数组在各个维度上的元素数量仍需满足既定的规则,否则就不能被称为多维数组,而只能算是多个数组的嵌套。这可能看起来比较抽象,不过没有关系,我们在后面会把数组的值与它们的类型放在一起进行解读。

最后,再次强调,数组类型具有非转化的特性。
