% 双指针法运用

\begin{issues}
\issueDraft
\end{issues}

\pentry{双指针算法\upref{TP}}

\begin{example}{快慢指针——链表中点}
给定一个头结点为 \textsl{head} 的非空单链表,返回链表的中间结点。

如果有两个中间结点,则返回第二个中间结点。

\textsl{ex.1}

输入:[1,2,3,4,5] 

输出:此列表中的结点 3 (序列化形式:[3,4,5])
返回的结点值为 3 。

\textsl{ex.2}

输入:[1,2,3,4,5,6]

输出:此列表中的结点 4 (序列化形式:[4,5,6])
\end{example}
