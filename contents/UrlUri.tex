% URL、URI 笔记
% license Usr
% type Note


\begin{issues}
\issueDraft
\end{issues}

\begin{itemize}
\item Schema: 如 http, https, ftp
\item Host: 如 www.example.com 或 ip 地址
\item Query: URL 后面可以跟一个 \verb`?` 加一串字符。 后面是 query string。 例如可以用于更新版本。
\item 端口号(可选)
\item 路径: \verb`www.example.com/images/photo.jpg` 中,\verb`/images/photo.jpg` 是路径
\item Query(可选):
\item Fragment(可选): 不会传递给服务器,仅用于浏览器页面定位
\end{itemize}

\subsubsection{转义}
\begin{itemize}
\item 使用百分号 + 两位 hex 表示一个字节的 ascii 码
\item \verb`%20` (\verb`空格`), \verb`%2F`(\verb`/`),\verb`%3F`(\verb`?`),\verb`%3D`(\verb`=`),\verb`%23`(\verb`#`), \verb`%2B` (\verb`+`), \verb`%27`(\verb`'`),\verb`%22`(\verb`"`) \verb`%3A`(\verb`:`), \verb`%26`(\verb`&`), \verb`%2C`(\verb`,`),\verb`%21`(\verb`!`),\verb`%3C`(\verb`<`), \verb`%3E`(\verb`>`)
\item 如果有 unicode 字符,那么先转换成 UTF-8 编码,然后每个字节用 \verb`%..` 编码即可。
\end{itemize}
