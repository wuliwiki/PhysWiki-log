% Python 异常处理
% keys try|except|异常|python
% license Xiao
% type Tutor

\begin{issues}
\issueDraft
\end{issues}

\begin{lstlisting}[language=python]
try:
    print(var)
except Exception as exc: # as exc 可以省略, 如果不需要使用
    print(exc) # 打印错误信息
else:
    print('done') # 运行正常时执行
finally:
    print('----') # 总会执行
\end{lstlisting}

\verb`Exception` 代表所有异常, 常见的异常类型有: \verb`SyntaxError` 语法错误, \verb`NameError` 名字错误(如变量名未定义), \verb`TypeError` 类型错误, \verb`IOError` 文件读写错误, 等。

如果需要检测多个错误类型, 可以把 \verb`Exception` 替换成多个类型的元组如 \verb`(SyntaxError,NameError)`。

\verb`else` 和 \verb`finally`语句可以省略。

如何产生异常呢? 每个异常类型可以把字符串转换为改异常类型的一个变量, 然后用关键词 \verb`raise` 即可抛出异常。
\begin{lstlisting}[language=python]
e = TypeError('这是一个错误')
raise e
\end{lstlisting}
通常把 \verb`raise` 放在一个判断从句中, 否则程序运行到此处必然会出现异常。

destructor 中不可以抛出错误
