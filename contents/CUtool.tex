% Linux 安装 CUDA Toolkit 笔记
% license Xiao
% type Note

2024-1-15 更新: Ubuntu 22.04 无法安装\upref{NvDrUb} Nvidia 驱动, 因为编译 Linux 内核使用的 gcc 11.3.0 不可能正常安装! CUDA Toolkit \href{https://developer.nvidia.com/cuda-downloads?target_os=Linux&target_arch=x86_64&Distribution=Ubuntu}{只有 20.04 和 22.04 两个选项}, 所以系统只能用 20.04!

如果你装机放了一张 Nvidia 显卡, 那么 Ubuntu 在安装时会自动提供一个 Nouveau 开源驱动。

笔者使用 Ubuntu 20.04 和 GTX 1080 Ti 显卡进行测试。

\subsection{安装 CUDA 和驱动}
注意: 不要改变系统默认的 \verb`gcc` symlink 的版本! 因为需要 \verb`gcc` 编译 kernel module, 必须使用默认版本。 如果已经改变了, 就卸载 \verb`gcc` 和 \verb`g++`, 再重装一次就行。

直接安装 cuda, 是包括显卡驱动的! (远程的话就用 Xwindow 打开 firefox)
\href{https://docs.nvidia.com/cuda/cuda-installation-guide-linux/index.html#ubuntu-installation}{网址}。 直接看 3.6 Ubuntu, 按照步骤来。
\begin{itemize}
\item 点击 pre-installation actions, 下载适合的 deb 安装包即可。 \href{https://developer.nvidia.com/cuda-downloads}{网址}
\item \verb`sudo dpkg -i cuda-repo-<distro>_<version>_<architecture>.deb`   最后一项是安装包的文件名
\item \verb`sudo apt-key add /var/cuda-repo-<version>/7fa2af80.pub`
\item \verb`sudo apt-get update`
\item \verb`sudo apt-get install cuda`
\item 按照 post-installation action 的说明来
\end{itemize}

post-installation action
\begin{itemize}
\item (put this command to \verb`~/.bashrc`)
\item \verb`export PATH=/usr/local/cuda-9.2/bin${PATH:+:${PATH}}`
\item (注意! cuda-9.2 以后可能会变, 具体数字要 cd 到 local 文件夹中查看!
如果不执行这个, nvcc 命令就找不到。 执行后, \verb`nvcc` 和 \verb`nvprof` 应该都能正常工作了。
剩下的步骤暂时还没有进行。
\end{itemize}

\subsection{只安装驱动}
\begin{itemize}
\item 首先更新驱动, 先删除以前驱动 \verb`sudo apt-get purge nvidia*`
\item 然后安装驱动 \verb`sudo add-apt-repository ppa:graphics-drivers`, \verb`sudo apt-get update`, \verb`sudo apt-get install nvidia-390` (或者新的版本)
\item 最后那三个数字是驱动的版本号, 上这个网页可以找到某个显卡的最新驱动版本号。 \href{http://www.nvidia.com/Download/index.aspx?lang=en-us}{网址}
\item 重新启动服务器!
\item 驱动安装完成后, \verb`nvidia-smi` 命令会显示显卡的信息。
\end{itemize}
