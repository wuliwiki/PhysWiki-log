% 牛顿运动定律(高中)
% 牛顿|牛顿运动定律|惯性|加速度|超重|失重

\pentry{机械运动基础\upref{HSPM01}, 相互作用\upref{HSPM02}}

\subsection{牛顿第一定律}

\subsubsection{内容}
一切物体总保持匀速直线运动状态或静止状态,除非作用在它上面的力迫使它改变这种状态。

\subsubsection{惯性}
物体保持原来匀速直线运动状态或静止状态的性质叫做惯性,是物体本身的固有属性。质量是物体惯性大小的唯一量度,物体的质量越大,其运动状态越难改变,惯性越大;质量越小,其运动状态越容易改变,惯性越小。

\subsubsection{对牛顿第一定律的理解}
牛顿第一定律揭示了一切物体都具有保持原来匀速直线运动状态或静止状态的性质,即一切物体都具有惯性,所以牛顿第一定律又叫\textbf{惯性定律}。

牛顿第一定律定性地揭示了运动和力的关系,说明力不是维持物体运动状态的原因,而是改变物体运动状态的原因。

牛顿第一定律是牛顿在总结前人观念的基础上得出的,是在理想实验的基础上加以科学推理和抽象得到的。

牛顿第一定律无法由实验直接验证,它所描述的是一种不受外力的理想状态。

当物体所受合力为 $0$ 的时候,其效果跟不受外力时一致,但不能把 “合力为 $0$” 说成 “不受外力”。

\subsection{牛顿第二定律}
\subsubsection{内容}
物体加速度的大小跟它受到的作用力成正比,跟它的质量成反比,加速度的方向跟作用力的方向相同。

\subsubsection{公式}
\begin{equation}\label{eq_HSPM03_1}
\bvec F=m\bvec a~.
\end{equation}
若物体延直线运动, 则可以用标量代替矢量: $F=m a$。

\subsubsection{力的单位}
在国际单位制中,力的单位是牛顿($\mathrm N$),它是根据牛顿第二定律定义的,使 $1\mathrm{kg}$ 的物体产生 $1\mathrm{m/s^2}$ 加速度的力为 $1\mathrm N$,即 $1\mathrm N=1\mathrm{kg \cdot m/s^2}$。

\subsubsection{特性}
\textbf{因果性}:力是产生加速度的原因。若不存在力,则没有加速度。

\textbf{矢量性}:加速度和力都是矢量,它们的方向始终相同,加速度的方向由力的方向唯一决定。

\textbf{瞬时性}:加速度和力同时产生、变化和消失,为瞬时对应关系。

\textbf{同体性}:应用牛顿第二定律时,公式中的三个物理量都是针对同一个研究对象。

\textbf{独立性}:物体同时受到几个力作用时,每一个力都会产生一个加速度,互不干扰,这些加速度的矢量和为物体的实际加速度,即合外力产生的加速度。

\subsection{牛顿第三定律}
\subsubsection{作用力和反作用力}
两个物体之间的作用是相互的。一个物体对另一个物体施力时,另一个物体也同时对它施加了力。这一对物体之间相互作用的力,叫做\textbf{作用力和反作用力}。

\subsubsection{内容}
两个物体之间的作用力和反作用力总是大小相等,方向相反,作用在同一条直线上。

\subsubsection{表达式}
\begin{equation}
\bvec F=-\bvec {F'}~.
\end{equation}

\subsubsection{特性}

\textbf{普遍性}:“总是”强调两个物体之间的作用力,无论在什么条件下,都是等大、反向、共线的,与两物体的质量、性质、运动状态、运动状态是否改变以及参考系的选取等因素均无关。

\textbf{同时性}:作用力和反作用力总是同时产生、变化和消失,无先后、主次之分。

\textbf{等值性}:作用力和反作用力总是相等的,在平衡状态和非平衡状态下的相互作用力都是等大的。

\textbf{共线性}:作用力和反作用力永远在一条直线上,且方向相反。

\textbf{相互性}:施力物体和受力物体相互作用,施力物体同时也是受力物体。

\textbf{同质性}:作用力和反作用力一定是性质相同的一对力。

\subsection{超重和失重}

\subsubsection{超重}

定义:物体对支持物的压力(或对悬挂物的拉力)大于物体所受重力的现象

产生条件:物体具有向上的加速度

根据牛顿第二定律(选定竖直向上为正方向):$F-G=ma~.$

\subsubsection{失重}

定义:物体对支持物的压力(或对悬挂物的拉力)小于物体所受重力的现象

产生条件:物体具有向下的加速度

根据牛顿第二定律(选定竖直向下为正方向):$G-F=ma~.$

\subsubsection{完全失重}

定义:物体对支持物的压力(或对悬挂物的拉力)等于零的现象

产生条件:物体具有向下的加速度,且加速度大小等于重力加速度

\subsubsection{对超重和失重的理解}

超重、失重与物体的速度无关,只取决于物体的加速度方向。

不论超重还是失重或完全失重,物体的重力都不变,只是物体对支持物的压力(或对悬挂物的拉力)改变。

在完全失重的状态下,一切由重力产生的物理现象都会完全消失。如单摆停摆、天平失效、浸没于液体中的物体不再受浮力、水银气压计失效等,但测力的仪器弹簧测力计是可以使用的,因为弹簧测力计是根据胡克定律制成的测力工具,并非只能测量重力。

尽管物体的加速度不是竖直方向,只要其加速度有竖直分量,物理就会处于超重或失重状态。

尽管整体没有竖直方向的加速度,但只要物体的一部分具有竖直方向上的加速度分量,整体也会出现超重或失重。
