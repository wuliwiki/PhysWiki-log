% 上海海事大学 2014 年数据结构
% 上海海事大学 2014 年数据结构

\subsection{一.判断题(本题10分,每小题1分)}

1、若某顺序表采用顺序存储结构,每个元素占$10$个存储单元,首地址为$200$,则下标为$11$(第$12$个)的元素的存储起始地址为$320$。

2、若对线性表进行的主要操作不是插入和删除,则该线性表宜采用顺序存储结构。

3、对一个空栈按$a,b,c,d,e,f,g$顺序依次读入,经过多次入栈和出栈的操作后,能得到按$f,e,g,d,a,c,b$顺序的出栈序列。

4、假定在顺序表中每个位置插入的概率相同,向一个有$64$个元素的顺序表中插入一个新元素并保持原来顺序不变,平均要移动$33$个元素。

5、含有$3$个结点(元素值均不相同)的二叉排序树共有$30$种。

6、$n$个顶点的连通图至少有$n-1$条边。

7、在无向图$G$的邻接矩阵$A$中,若$A[i][j]$等于$1$,则$A[j][i]$等于$0$。

8、采用顺序检索法在一个有$123$个元素的有序顺序表中查找,若每个元素的查找概率相等,则成功检索的平均查找长度$ASL$为$61$。

9、在散列存储中,装载因子$a$的值越大,发生冲突的可能性就越大。

10、快速排序是一种稳定的排序方法。

\subsection{二。填空题(本题30分,每空2分)}

1.分析下列程序段,其时间复杂度分别为:( (1) )、( (2) ).
\begin{lstlisting}[language=cpp]
i=m=0;
while (m<n) {
    i++; s+=i;
}

m=0;
for(i=1; i<=n; i++)
    for(j=2*i; j<=n; j++)
        m++;
\end{lstlisting}

2.广义表$A=(a, (a, b), (i,j),k),d,e)$的长度是( (3) ),深度是( (4) ),取表头和表尾函数分别为head()和tail(),则head(tail(head(tail(A))))=( (5) ), 而从表中取出原子项$j$的运算为( (6) )。

3.有一个二维数组A.0.[2..9], 每个数组元素占用8个存储单元,并且A[2][5]的存储地址为2080,若按行序为主序方式存储,数组元素 A[4][6]的存储地址是\_ (7)

4.一棵完全二叉树有$600$个结点,则它的深度是( (8) )。

5.已知一个图采用邻接矩阵表示,计算第$i$个结点的入度的方法是( (9) )。

6.一个图的边集为{<A, B>,<A,C>,<A,E>,<B,C>, <B,D>, <C,D>, <E,B>,<E, D>},从顶点A出发进行深度优先搜索遍历访问顶点顺序为( (10) ), 从顶点A出发进行广度优先搜索遍历访问顶点顺序为( (11) ), 对该图进行拓扑排序得到的顶点序列为( (12) ).

7.对$12$个元素的序列进行直接插入排序时,最少的比较次数为( (13) )。

8.( (14) )排序方法采用的是二分法思想,在( (15) )情况下最不利于发挥其长处。

\subsection{三,选择题(本题20分,每空2分)}

1.在数据结构中,从逻辑上可以把数据结构分成( )。 \\
A.动态结构和静态结构 $\qquad$ B.紧凑结构和非紧凑结构 \\
C.线性结构和非线性结构 $\qquad$ D.内部结构和外部结构

2.线性表若采用链式存储结构时,内存中可用存储单元的地址( )。 \\
A.必须是连续的 $\qquad$ B.部分地址必须是连续的 \\
C.一定是不连续的 $\qquad$ D.连续不连续都可以

3.线性表的顺序存储结构是一种(  )的存储结构,而线性表的链式存储结构是一种随机存取的存储结构。 \\
A.随机存取 $\qquad$ B.顺序存取 \\
C.索引存取 $\qquad$ D.散列存取

4.在一个单链表中,已知$q$所指结点是$p$所指结点的前驱结点,若在$q$和$p$之间插入$s$结点,则执行( )。 \\
A. s->next=p->next; p->next=s; $\qquad$ B. p->next=s->next; s->next=p; \\
C. q->next=s;s->next=p; $\qquad$ D. p->next=s; s->next=q;

5.在双链表中的$p$所指结点之后插入$s$所指结点的操作是()。 \\
A. p->next=s; s->prior-p; p->next->prior=s; s->next p->next; \\
B. p->next=s; p->next->prior=s; s->prior-p; s->next=p->next; \\
C. s->prior=p; s->next=p->next; p->next=s; p->next->prior=s; \\
D. s->prior-p; s->next=p->next; p->next->prior-s; p->next=s;

6.串是一种特殊的线性表,其特殊性体现在( )。 \\
A.可以顺序存储 $\qquad$ B.数据元素是一个字符 \\
C.可以链接存储 $\qquad$ D.数据元素可以是多个字符

7.以下( )是稀疏矩阵一般的压缩存储方法。 \\
A.二维数组和三维数组 $\qquad$ B.三元组和散列 \\
C.三元组和十字链表 $\qquad$ D.散列和十字链表

8.采用邻接表存储的图的深度优先遍历算法类似于二叉树的( )。 \\
A.先序遍历 $\qquad$ B.中序遍历 $\qquad$ C.后序遍历 $\qquad$ D.按层遍历

9.以下不需进行关键字的比较的排序方法是(  )。 \\
A.快速排序 $\qquad$ B.归并排序 $\qquad$ C.基数排序 $\qquad$ D.起泡排序

10.有一个长度为$12$的有序表,按二分查找法对该表进行查找,在表内各元素等概率情况下查找成功所需的平均比较次数为()。 \\
A.35/12 $\qquad$ B.37/12 $\qquad$ C.39/12 $\qquad$ D.43/12

\subsection{四,应用题(本题60分,每小题10分)}

1.已知一棵二叉树的先序序列和中序序列分别为, \\
先序序列: ABDEGHCFK \\
中序序列: DBGEHAFKC \\
请画出该树的结构并写出其后序序列。

2.右图是一个无向图,试分别用下列两种方法求它的最小生成树,并给出依次产生的边,连接顶点i和j边用<i, j>的形式表示。
\begin{figure}[ht]
\centering
\includegraphics[width=5cm]{./figures/SMDS14_1.png}
\caption{第四2题} \label{SMDS14_fig1}
\end{figure}
1)用普里姆算法从顶点A开始; \\
2)用克鲁斯卡尔算法。

3.某通信系统中共包含八个字符: A、B、C、D、E、F、G、H,它们出现频率分别为: 0.13. 0.19、0.07、 0.14、0.16、0.22、 0.03、 0.06,试为它们构造一棵Huffman树(哈夫曼树),并设计这些字符的哈夫曼编码。

4.设散列表为HT[0..14],即表的长度为15。散列函数为: H(key)= key\%13,采用线性探测再散列法解决冲突,若插入的关键码序列为{163, 151, 166, 143, 124, 138, 161, 158,130, 67, 232, 89, 213}。 \\
1)试画出插入这13个关键码后的散列表。 \\
2)计算在等概率情况下查找成功的平均查找长度ASL。

5.对于正整数序列{59, 96, 48,39, 86,75,38, 18, 66,92, 22},建立一棵平衡二叉树,然后插入结点32,分别画出该平衡二叉树及插入结点32后的平衡二叉树。

6.已知待排序记录的关键字序列为{ 501, 120, 539,983, 185,852, 276, 632,478,157, 528,616, 208}, 需要按关键字值递增的次序进行排序,请回答下列问题。 \\
1)写出以第一个元素为基准的快速排序进行第一趟扫描后的结果; \\
2)堆排序初始建堆后结果。

\subsection{五.编程题(本题30分,每小题10分)}

1.对给定的$m$,编写一个函数求满足$1*2+2*3+3+...+(n-1)*n<=m$的最大的$n$。

2.试设计一个算法,通过遍历一趟单链表来调整结点顺序,将带头结点的单链表中所有$data$域值小于零的结点都放在不小于零的结点之后。表头指针为$first$,存储结构为:
\begin{lstlisting}[language=cpp]
typedef struct Node {
    int data;
    struct Node *next;
}Node,*List;
\end{lstlisting}

3.试编写函数输出二叉树中每一个结点的层数(设根结点的层数为$1$)。 二叉树用二叉链表存储,定义为:
\begin{lstlisting}[language=cpp]
typedef struct BiNode {
    char data; 
    struct BiNode *lchild, *rchild;
}BiNode,* BiTree;
\end{lstlisting}