% 哈尔滨工业大学 2013 年 考研 量子力学
% license Usr
% type Note

\subsection{(共 60 分,每小题10 分)}
\begin{enumerate}
\item 简单描述玻尔子模型理论的基本假设。由于验证了玻尔理论而获得诺贝尔物理学奖的著名物理学家是谁?实验证明了什么?

\item 系统的哈密顿算符为 
$$\hat{H} = \frac{p^2}{2\mu} + V(x),~$$
计算
$$\left[e^{ikx}[\hat{H}e^{-ikx}]\right],~$$
式中 $a$ 为常数。

\item 线谐振子在 $t = 0$ 时处于
$$ \psi(x, 0) = \frac{1}{2}\varphi_0(x) + \frac{\sqrt{3}}{2}\varphi_1(x) + \frac{1}{\sqrt{2}}\varphi_3(x)~ $$
态中,其中 $\varphi_n(x)$ 为第 $n$ 能量本征态对应的本征函数。\\
(1).求在 $\psi(x, 0)$ 态上能量的可测值、取值概率与平均值;\\\\
(2).写出 $t > 0$ 时刻的波函数及各能量取值概率与平均值。\\\\
\item 平移算符 $\hat{D}(a)$ 的定义为 $\hat{D}(a)\varphi(x) = \varphi(x + a)$,证明 $\hat{D}(a)$ 可以用动量算符表示。

\item 证明:若一个 $2 \times 2$ 矩阵与 $2 \times 2$ 泡利矩阵均匀,则此 $2 \times 2$ 矩阵必可写为单位矩阵与一个常数之积。

\item 求厄米(Hermite)算符 $\hat{F} = \alpha\hat{p} + \beta\hat{x}$ 的本征值为$f$本征态。
\end{enumerate}
\subsection{(15 分)}
失量算符
\[
\hat{j} = \frac{1}{2} \left( \delta(\vec{r} - \vec{r}') \frac{\hat{\vec{p}}}{m} + \frac{\hat{\vec{p}}}{m} \delta(\vec{r} - \vec{r}') \right)~
\]
式中 $\hat{p}$ 为动量算符,证明算符 $\hat{\vec{j}}$ 为厄米(Hermite)算符,并求出它在态 $\psi(\vec{r})$ 下的平均值的表达式。

\subsection{(15 分)}
设 $|m\rangle$ 和 $|n\rangle$ 为 $\hat{L}_z$ 的两个本征态,本征值分别为 $m\hbar$ 和 $n\hbar$,求矩阵元 $\langle m| \hat{L}_x |n \rangle$ 和 $\langle m| \hat{L}_y |n \rangle$ 的关系,并给出矩阵元不为零的条件。
\subsection{(20)}
质量为 $\mu$ 的粒子在一维势场中运动,若其哈密顿量的一个本征函数为
\[
\varphi(x) = a x \exp \left( -\frac{1}{2}b^2 x^2 \right)~
\]
其中 $a, b$ 为实常数。求粒子所处的势场。
\subsection{(20分)}
设系统的哈密顿量可以写为
$$H = \begin{pmatrix} E^0_1 + a & b \\\\ b & E^0_2 + a \end{pmatrix}~$$
其中$a$和$b$为实常量,且远小于$E^0_1$ 。利用微扰论求能量的二级近似,并与精确结果作比较。
\subsection{(20分)}
两个自旋均为$\frac{1}{2}$的粒子组成一个复合系统,$A$粒子处于$A$的自旋$z$分量$S_z=\frac{1}{2}\hbar$的本征态,$B$粒子处于$B$的自旋$x$分量$S_x=\frac{1}{2}\hbar$的本征态。求发现系统总自旋为零的概率。