% 序关系
% 序|偏序|全序|良序|关系

\begin{issues}
\issueAbstract
\end{issues}

\pentry{二元关系\upref{Relat}}

\subsection{偏序集}

\begin{definition}{偏序集、半序集}
如果一个集合 $A$ 上定义了一个序关系 $\prec$,那么称这个集合是带有偏序 $\prec$ 的\textbf{偏序集(partially ordered set)}或\textbf{半序集}.
\end{definition}

带有偏序 $\prec$ 的偏序集 $A$ 通常记作 $(A,\prec)$.在不致混淆的情况下,可以简称为偏序集 $A$.偏序集上的序关系称为\textbf{偏序关系(partially ordered relation)}.

序关系实际上是给集合中的元素排序,因此 $a \prec b$ 又被称为\textbf{$a$ 在 $b$ 前},反之 $b \prec a$ 称为\textbf{$a$ 在 $b$ 后}.这种排序方式并不唯一,而且任意两个元素间不一定可以排出前后.

\begin{example}{}
各种数集(包括 $\mathbb{N},\mathbb{Z},\mathbb{Q},\mathbb{R},\mathbb{C}$)按通常的序关系构成偏序集.

其中复数集 $\mathbb{C}$ 上的序关系比较特殊,复数 $a_1+b_1\I \leq a_2+b_2\I$ 当且仅当 $a_1\leq a_2$ 且 $b_1 \leq b_2$.复数的序关系通常不使用.
\end{example}

这个例子说明序关系实际上是数与数大小关系的抽象.

\begin{example}{}
设集合$A = \{a,b,c\}$,关系$\prec\!= \{(a,a),(a,b),(a,c),(b,b),(c,c)\}$,可以验证 $\prec$ 是 $A$ 上的偏序关系.
%我们可以用下列的图示来表示 $A$ 及其上的偏序关系 $\prec$
\end{example}

\begin{example}{字典序}\label{OrdRel_ex1}
已知 $(A,\prec),(B,\leq)$ 是两个偏序集,那么笛卡尔积 $A\times B$ 按某个序关系 $\leqslant$ 构成偏序集.这个序关系 $\leqslant$ 满足:
\begin{enumerate}
\item $\forall(a_1,b_1),(a_2,b_2) \in A\times B, a_1\prec a_2 \Rightarrow (a_1,b_1)\leqslant(a_2,b_2)$;
\item $\forall(a_1,b_1),(a_2,b_2) \in A\times B, a_1=a_2, b_1\leq b_2 \Rightarrow (a_1,b_1) \leqslant (a_2, b_2)$.
\end{enumerate}

由于字典通常按照这样的顺序编排\footnote{比如,字典中单词in在a后,在it前.},因此这种序关系称为字典序.
\end{example}
\begin{example}{幂集}
一个集合 $A$ 的幂集\footnote{即集合 $A$ 所有子集的集合.}$2^X$ 与子集关系一起构成偏序集.
\end{example}

\begin{example}{}
小时百科的知识树(目录树)可以看做一个偏序集,词条A在词条B前意味着词条A是词条B的预备知识,或预备知识的预备知识,或……
\end{example}

\subsection{全序集}

\begin{definition}{全序集}
偏序集 $(A,\prec)$ 称为\textbf{全序集(totally ordered set)}或\textbf{有序集(ordered set)},当且仅当对任意 $a,b \in A$,$a \prec b$ 或 $b \prec a$.
\end{definition}
全序集上的序关系称为\textbf{全序关系(totally ordered relation)}.
\begin{example}{}
数集 $\mathbb{N},\mathbb{Z},\mathbb{Q},\mathbb{R}$ 按通常的序关系构成全序集.
\end{example}
\begin{example}{}
将\autoref{OrdRel_ex1} 中的偏序集 $A,B$ 换成全序集,则按同样的序关系,$A\times B$ 也构成全序集.
\end{example}

一个显然但重要的事实是,任何偏序集的非空子集按原来的序仍是偏序集,任何全序集的非空子集按原来的序仍是全序集.

\begin{definition}{极小元、极大元}
已知偏序集 $(A,\prec)$:
\begin{enumerate}
\item 如果存在某个 $a \in A$,使得对任意 $x \in A, x\prec a$ 都有 $x = a$,那么称 $a$ 为偏序集 $A$ 的\textbf{极小元(minimal element)}.
\item 如果存在某个 $b \in A$,使得对任意 $x \in A, b\prec x$ 都有 $x = b$,那么称 $b$ 为偏序集 $A$ 的\textbf{极大元(maximal element)}.
\end{enumerate}
\end{definition}

\begin{definition}{最小元、最大元}
已知偏序集 $(A,\prec)$ 的某个非空子集 $B$:
\begin{enumerate}
\item 如果存在某个 $a \in B$,使得 $\forall x \in B, a \prec x$.那么 $a$ 被称为 $B$ 的\textbf{最小元(least element)};
\item 如果存在某个 $b \in B$,使得 $\forall x \in B, x \prec b$.那么 $b$ 被称为 $B$ 的\textbf{最大元(greatest element)}.
\end{enumerate}
\end{definition}

极大元、极小元与最大元、最小元极易混淆.偏序集的极大元、极小元不一定唯一,但最大元、最小元只要存在必然唯一.在全序集中两者统一.\footnote{读者自证.}

\begin{example}{}
集合 $A=\{0,1,2\}$,在上面定义偏序关系 $\prec$ 为 $0\prec 0$, $1\prec 1$, $1\prec 2$, $2\prec 2$.那么子集 $\{0,1\}$ 的极大元为0和1,不存在最大元.
\end{example}

\begin{exercise}{}
证明:如果偏序集的某个子集存在最大元,那么它的极大元必然存在且唯一,并且两者相等.
\end{exercise}

\begin{exercise}{}
证明:对于全序集的任意非空子集,如果存在极大元,则最大元必然存在,且两者相等.
\end{exercise}

\subsection{良序集}

\begin{definition}{良序集}
全序集 $(A,\prec)$ 称为\textbf{良序集(well-ordered set)},当且仅当它的任意非空子集都有极小元.
\end{definition}

良序集上的序关系称为\textbf{良序关系(well-ordered relation)}.

任何良序集的非空子集仍是良序集.