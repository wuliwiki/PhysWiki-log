% 萨特延德拉·纳特·玻色(综述)
% license CCBYSA3
% type Sum

本文根据 CC-BY-SA 协议转载翻译自维基百科\href{https://en.wikipedia.org/wiki/Satyendra_Nath_Bose}{相关文章}。

萨特延德拉·纳特·玻色(Satyendra Nath Bose,FRS,印度国会议员)\(^\text{[1]}\)(/ˈboʊs/;\(^\text{[4]}[a]\)1894年1月1日-1974年2月4日)是印度理论物理学家和数学家。他最著名的成就是在20世纪20年代初对量子力学的研究,奠定了玻色–爱因斯坦统计的基础,并发展出玻色–爱因斯坦凝聚态理论。他是英国皇家学会院士,并于1954年获得印度政府颁发的印度第二高平民荣誉——帕德玛·毗布尚奖。\(^\text{[5][6][7]}\)

遵循玻色统计的粒子被称为玻色子,这个名称是由保罗·狄拉克以玻色的名字命名的。\(^\text{[8][9]}\)

玻色是一位百科全书式的学者,兴趣广泛,涵盖物理、数学、化学、生物、矿物学、哲学、艺术、文学与音乐等多个领域。印度独立后,他参与了许多科研和技术发展委员会的工作。\(^\text{[10]}\)
\subsection{早年生活}
玻色出生于加尔各答(今加尔各答市,Kolkata),是孟加拉卡雅斯特家庭中七个孩子中的长子。\(^\text{[11]}\)他是家中唯一的儿子,之后还有六个妹妹。他的祖籍位于孟加拉省纳迪亚县的巴拉·贾古利亚村。

他五岁开始上学,就在家附近。后来家里搬到果阿巴甘地区,他进入“新印度学校”就读。在学业的最后一年,他转入“印度教学校”。1909年,他通过入学考试(即中学毕业考试,matriculation),并在成绩排名中位列第五。随后,他进入加尔各答的总督学院学习中级理科课程,他的老师包括贾加迪什·钱德拉·玻色、萨拉达·普拉萨纳·达斯以及普拉富拉·钱德拉·雷。玻色于1913年在总督学院获得混合数学专业的理学学士学位,并名列第一。之后,他进入由阿肖托什·穆克吉爵士新成立的理学院学习,并于1915年再次在混合数学专业的理学硕士(MSc)考试中获得第一名。他在硕士考试中的成绩在加尔各答大学的历史上创下了新的纪录,至今仍无人打破。\(^\text{[12]}\)

完成硕士学位后,玻色于1916年进入加尔各答大学理学院担任研究学者,开始研究相对论理论。那是科学进展史上一个令人振奋的时代。量子理论刚刚开始显现其轮廓,许多重要成果接连不断地涌现出来。\(^\text{[12]}\)

他的父亲苏伦德拉纳特·玻色在东印度铁路公司的工程部门工作。1914年,年仅20岁的萨特延德拉·纳特·玻色与乌沙巴蒂·高希结婚,\(^\text{[3]}[13]\)后者是加尔各答一位知名医生的11岁女儿。\(^\text{[14]}\)他们共育有九个孩子,其中两人在幼年早逝。玻色于1974年去世时,留下了妻子、两个儿子和五个女儿。\(^\text{[12]}\)

作为一位语言通,玻色精通多种语言,如孟加拉语、英语、法语、德语和梵语,并熟悉丁尼生勋爵、拉宾德拉纳特·泰戈尔和迦梨陀娑的诗作。在欧洲,他以对希伯来文学和宗教的了解给东道主雅克琳·扎多克-卡恩留下了深刻印象。\(^\text{[15]}\)他还会演奏伊斯拉吉(一种类似小提琴的印度乐器)。\(^\text{[16]}\) 此外,他积极参与创办夜校,这些学校后来被称为“工人学院”。\(^\text{[7][17]}\)
\subsection{研究生涯}
玻色曾就读于加尔各答的印度教学校,随后进入同样位于加尔各答的总督学院,并在两所学校的学习中都取得了最高分,而他的同班同学、后来的天体物理学家梅格纳德·萨哈则名列第二。\(^\text{[7]}\)他师从贾加迪什·钱德拉·玻色、普拉富拉·钱德拉·雷和纳曼·夏尔玛等教师,这些老师激励他立志高远。从1916年到1921年,玻色在加尔各答大学附属的拉贾巴扎理学院的物理系担任讲师。1919年,他与萨哈合作,基于德语和法语对爱因斯坦关于狭义与广义相对论原始论文的翻译,编写了第一本英文教材。

1921年,萨特延德拉·纳特·玻色加入新成立的达卡大学(位于今孟加拉国)物理系,担任副教授。\(^\text{[18]}\)玻色创建了多个全新的系别和实验室,用于教授硕士和荣誉学士课程,并讲授热力学以及詹姆斯·克拉克·麦克斯韦的电磁理论。\(^\text{[19]}\)

自1918年起,玻色与印度天体物理学家梅格纳德·萨哈一起在理论物理与纯数学领域发表了多篇论文。1924年,玻色在达卡大学物理系任职期间,撰写了一篇论文,用一种新颖的方法对相同粒子的状态进行计数,在未引用任何经典物理前提的情况下推导出了普朗克量子辐射定律。这篇论文在奠定量子统计学这一重要分支中起到了开创性的作用。\(^\text{[20]}\)

虽然这篇论文最初未被期刊接收,玻色便直接将其寄给了身在德国的阿尔伯特·爱因斯坦。爱因斯坦意识到该论文的重要性,亲自将其翻译成德语,并代玻色将论文提交给《物理学杂志》发表。

由于这项成就获得认可,玻色得以前往欧洲,在多家X射线和晶体学实验室工作两年,并与路易·德布罗意、玛丽·居里及爱因斯坦本人共事。\(^\text{[7][21][23]}\)
\subsubsection{玻色–爱因斯坦统计}
在达卡大学一次关于辐射理论与紫外灾难的讲座中,\(^\text{[24]}\)玻色试图向学生展示,当时的理论存在不足,因为它预测的结果与实验数据不符。

在解释这一差异的过程中,玻色首次提出:麦克斯韦–玻尔兹曼分布并不适用于微观粒子,因为在这种尺度下,由于海森堡不确定性原理引发的涨落将变得显著。因此,他强调应考虑在相空间中寻找粒子的概率,其中每个状态具有体积 $h^3$,并摒弃了粒子的确定位置与动量的观念。

玻色将这次讲座整理成一篇短文,题为《普朗克定律与光量子假说》,并附上一封信,将其寄给了阿尔伯特·爱因斯坦。\(^\text{[25]}\)

尊敬的先生:

冒昧将随附的文章寄上,恳请您审阅并赐教。我非常希望能知道您对它的看法。您将看到,我尝试在不依赖经典电动力学的前提下,推导出普朗克定律中的系数 $8\pi \nu^2/c^3$,仅假设相空间中最基本的元区域具有体积 $h^3$。

由于我德语水平有限,无法将此文翻译成德文。如果您认为这篇文章值得发表,倘若您能帮忙安排在《物理学杂志》上发表,我将不胜感激。虽然我与您素未谋面,但我在提出此请求时并不感到迟疑,因为我们都视您为师,虽然只能通过您的著作而受益。

不知您是否还记得,曾有人从加尔各答请求允许将您的相对论论文翻译成英文,您当时慷慨应允,那本书后来已出版。我正是翻译您有关广义相对论论文的那位人。

爱因斯坦认同玻色的观点,将玻色的论文《普朗克定律与光量子假说》翻译成德文,并于1924年以玻色的名义发表在《物理学杂志》上。\(^\text{[26]}\)

玻色的解释之所以能够得出准确的结果,是因为光子之间彼此不可区分,因此不能将两个具有相同能量的光子视为两个彼此可识别的不同光子。类比而言,如果在一个平行宇宙中,硬币像光子或其他玻色子那样行为,那么投出两个正面的概率将是三分之一(因为“反-正”和“正-反”应视为同一种情况)。

玻色的这一解释如今被称为玻色–爱因斯坦统计。玻色得出的这个结果奠定了量子统计的基础,尤其开创性地提出了“粒子不可区分性”这一全新的哲学概念,这一点得到了爱因斯坦和狄拉克的认可。\(^\text{[26]}\)

当爱因斯坦与玻色面对面会面时,他问玻色是否意识到自己发明了一种全新的统计方法,而玻色坦率地回答说他并不了解玻尔兹曼统计,因此没有意识到自己的计算方式不同。他对任何向他提出此问题的人都同样坦诚。
