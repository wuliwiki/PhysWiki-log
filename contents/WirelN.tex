% 无线通讯、有线通讯笔记
% license Usr
% type Note


\begin{issues}
\issueDraft
\end{issues}

\textbf{调幅(AM)}:调幅通过改变无线电波的幅度(即信号的强度或高度),来携带基带信号(例如音频信号)。在调幅中,载波的频率保持不变,而是其幅度根据基带信号的变化而变化。

\textbf{调频(FM)}:调频通过改变载波的频率来传输基带信号,而载波的幅度保持不变。基带信号的不同值会导致载波频率的不同程度变化,而这种频率的变化被接收器解码以重建原始信号。

天线的震荡电路只能接受一个频率,那调频信号怎么接收呢? 调频信号的频率波动非常小,电路中对某个频率滤波后仍然存在一定的频率带宽,并不是严格的单频。

\textbf{相位调制(PM)}:在模拟相位调制中,信号的相位变化与原始信号的幅度直接相关。模拟PM不常用于传输数字数据,而是用于传输模拟信号。

\textbf{相位偏移键控(PSK)}:PSK是一种数字调制技术,它通过改变载波的相位来表示数字数据。例如,在二进制相位偏移键控(BPSK),载波的两个相位分别代表0和1。更复杂的形式如四相位偏移键控(QPSK)使用四个不同的相位,每个相位可以表示两位数字信息。

\textbf{正交幅度调制(QAM)}:QAM是一种同时使用幅度和相位变化来传输数据的方法,因此可以视为是调幅和相位调制的结合。通过改变信号的幅度和相位,QAM能够在给定的带宽内传输更多的数据。实际上,QAM信号在某种意义上也涉及到频率的变化,因为相位的改变可以导致频率的瞬时变化,尽管这种频率的变化并不是QAM的直接目标。

Wifi、LTE、5G、有线电视都是使用 QAM 技术的。

\textbf{有线电视}:有线电视(Cable TV)系统早期主要传输模拟信号,但现在多数系统已经迁移到或同时支持数字信号传输。 有线电视使用同轴电缆,也就是说真正传输信号的只有中间一根导线。

\textbf{以太网(Ethernet)} 电缆,特别是典型的双绞线(如Cat5e、Cat6、Cat6a等),包含8根铜导线,这些导线被分为4对,每对线被紧密地绞合在一起。这些导线的设计使得以太网电缆能够支持高速数据通信,同时减少信号干扰和信号衰减。不同的以太网标准和设备配置决定了这些导线在实际应用中的具体用途。

对于 10BASE-T 和 100BASE-TX 标准,只有两对线(4根导线)被用于数据传输,另外两对线在标准应用中未被使用,但可以用于其他目的,如供电(通过 Power over Ethernet,PoE)。
对于 1000BASE-T(Gigabit Ethernet)及更高速率的标准,所有四对线都被用于数据传输,支持全双工通信,即数据可以同时在两个方向上传输。

\textbf{蓝牙}:蓝牙 4.2 以下主要使用\textbf{高斯频移键控(GFSK)}和/或\textbf{差分相位偏移键控(DPSK)}
