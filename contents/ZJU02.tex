% 浙江大学 2002 年 考研 量子力学
% license Usr
% type Note

\textbf{声明}:“该内容来源于网络公开资料,不保证真实性,如有侵权请联系管理员”

\subsection{第一题:从下面四题中任选三题(15分)}

(1)试说明光电效应实验中的“红限”现象,为何光电效应实验中有所谓截止频率的概念?

(2)如何从黑体辐射实验的Planck公式中推出 Stefan 公式?(只要求给出思路)。根据该
公式,能否做出什么测温仪器?

(3)你认为Bohr的量子理论有哪些成功之处?有哪些不成功的地方?试举一例说明

(4)你能从固体与分子的比热问题中得出哪些量子力学的概念?
\subsection{第二题(20 分):}
设氢原子处于状态:
\[\Psi(r,  0, \varphi) = \frac{1}{4} R_{21}(r) Y_{11}(0, \varphi) - \frac{\sqrt{7}}{4} R_{21}(r) Y_{10}(0, \varphi) + \frac{1}{\sqrt{2}} R_{31}(r) Y_{1-1}(0, \varphi)~\]

(1) 测量该原子的能量,测得的可能值为多少?相应的几率为多少?

(2) 测量该原子的角动量平方 $\hat{L}_z^2$ ,测得的可能值为多少?相应的几率为多少?

(3) 测得的角动量分量 $L_z$ 的可能值和相应几率为多少?
\subsection{第三题:(20分)}
一质量为 $m$ 的粒子处于势场 $V(x)$ 中运动,若

(1) 
\[V(x) = \begin{cases} \infty, & |x| > a \\\\0, & |x| \leq a \end{cases}~\]
则该粒子的本征能量为多少?

(2) 
\[V(x) = a \delta(x), \quad a < 0 \text{ 为已知常数, 则该粒子的本征能量为多少?特征长度为多少?}~\]

(3) 
\[V(x) = \begin{cases} V_0\delta(x), & |x| < a \\\\\infty, & |x| \geq a \end{cases}, \quad V_0 > 0~\]
是一个给定的常数,则该粒子满足的方程为何?

(4) 
能量为 $E$ 的平行粒子束,以入射角 $\theta$ 的射向平面 $x = 0$, 在区域 $x < 0$, $V = 0$, 在区域 $x > 0$, $V = -V_0$. 试从量子力学的角度,分析粒子的反射及折射规律。(用 $\theta$ 及$n = \left( 1 + \frac{V_0}{E} \right)^{\frac{1}{2}}$
表示反射几率 $R$ 及折射几率 $D$ 。)
\subsection{第四题:(15分)}
(1) 如何证明一个算符为厄米算符?算符 $\hat{A} = i \hbar x \frac{d}{dx}$ 是否为厄米算符?

(2) 若 $[\hat{x}, \hat{p_x}] = i\hbar$,计算并易于 $[\hat{x}^3, \hat{p_x}^3]$。

(3) 证明厄米算符对应不同本征值的本征函数相互正交。

(4) 为什么物理量要用厄米算符来表示?

下面三组试题(五题、六题与七题、八题),任选一组解答。
\subsection{第五题:(15分)}
在一维谐振子问题中,若谐振子的质量为 \(m\) 相互作用势用 

\[
V(x) = \frac{1}{2} m(\omega_1^2 x^2 + \omega_2 x + e)~
\]

来表示,其中 \(\omega, \omega_2, e\) 为常数。若 \(\langle \hat{x} \rangle_{t-0} = 0\),\(\langle \hat{p} \rangle_{t-0} = 0\),问其位移 \(x\) 的平均值与时间的关系为何?
\subsection{第六题:(15分)}
如果有一个二能级系统 \(\lvert 1 \rangle\), \(\lvert 2 \rangle\),其相应的能量分别为 \(E_1\), \(E_2\),哈密顿算符的有关矩阵元为

\[
\langle 1 \lvert \hat{H} \rvert 1 \rangle = E_1 + b, \quad \langle 2 \lvert \hat{H} \rvert 2 \rangle = E_2 + b, \quad \langle 1 \lvert \hat{H} \rvert 2 \rangle = \langle 2 \lvert \hat{H} \rvert 1 \rangle = a~
\]

其中 \(E_1, E_2, a, b\) 为已知常数,满足一切近似条件。问:

1. 若以 \(\lvert 1 \rangle, \lvert 2 \rangle\) 为零级近似波函数,至一级近似,本征能量为何?

2. 至二级近似,本征能量为何?
\subsection{第七题:(15 分)}
若有一质量为 $m$ 的低能粒子被一强势场散射,若散射时的有效质量为 $\mu$,势场形式为
$$
V(r) =
\begin{cases}
-V_0, & r < a \\
0, & r \geq a
\end{cases}~
$$
$V_0 > 0$,$a$ 为已知常数。问:

1. 使用玻恩近似化还是用分波法比较合适?

2. 试问相移 $\delta_l$ 的正弦与散射势能及散射波函数的关系为何?

3. 求出零级近似下的微分散射截面。

4. 若不知道势场 $V(r)$ 的具体形式,能否利用散射实验来确定 $V(r)$?
\subsection{第八题: (15分)}
试证固体物理学中常用的 Thomas-Reiche-Kuhn 求和规则:

\[\sum_n (E_n - E_a) \left| \langle n | \hat{x} | a \rangle \right|^2 = \frac{\hbar^2}{2m}~\]

其中,\(|n\rangle, |a\rangle\) 为系统的两个任意的能态,\(E_n, E_a\) 为任意两个能级的能量,\(m\) 为粒子的质量。
