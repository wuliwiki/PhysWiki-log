% 2019 年考研数学试题(数学一)
% keys 考研|数学
% license Xiao
% type Tutor

\subsection{选择题}
1.当$x \to 0$时,若$x-\tan x$与$x^k$是同阶无穷小,则$k=$()\\
(A)1 $ \quad$ (B)2 $\quad$ (C)3  $\quad$(D)4 $\quad$

2.设函数$f(x)=\leftgroup{x \abs{x},x \leq 0\\ x \ln x,x>0}$,则$x=0$是$f(x)$的()\\
(A)可导点,极值点 $ \qquad$ (B)不可导点,极值点 \\
(C)可导点,非极值点  $\quad$(D)不可导点,非极值点 

3.设${u_n}$是单调增加的有界数列,则下列级数中收敛的是()\\
(A)$\displaystyle \sum_{n=1}^\infty \frac{u_n}{n} \quad$ 
(B)$\displaystyle \sum_{n=1}^\infty (-1)^n \frac{1}{u_n} \quad$
(C)$\displaystyle \sum_{n=1}^\infty (1- \frac{u_n}{u_{n+1}}) \quad$ 
(D)$\displaystyle \sum_{n=1}^\infty (u_{n+1}^2-u_n ^2)$

4.设函数$Q(x,y)=\frac{x}{y^2}$,如果对上半平面内的任意有向光滑封闭曲线$C$都有$\oint_c P(x,y)\dd{x}+Q(x,y)\dd{y}=0$,那么函数$P(x,y)$可取为()\\
(A)$y-\frac{x^2}{y^3} \quad$ (B)$\frac{1}{y}-\frac{x^2}{y^3}\quad$(C)$\frac{1}{x}-\frac{1}{y}\quad$ (D)$x-\frac{1}{y}\quad$

5.如图所示,有3张平面两两相交,交线相互平行,它们的方程$$a_{il}x + a_{i2}y + a_{i3}z = d_i \quad (i=1,2,3)~$$组成的线性方程组的系数矩阵和增广矩阵分别记为A,