% 二体碰撞问题
\pentry{二体系统\upref{TwoBD}}

注意以下讨论的碰撞不必要求在一瞬间发生, 可以拓展到有限距离的作用力甚至无穷远但不断衰减的作用力。 例如考虑两个带电荷的质点的碰撞。 在无穷远处时, 二者之间的作用可忽略, 此时的速度可定义为初速度。 当发生相互作用后, 把两质点互相远离到相距无穷远时的速度定义为末速度。

\subsection{一维情况}
高中物理中, 若两质点的运动限制在同一直线上且碰撞为完全弹性碰撞, 我们可以联立能量守恒和动量守恒两条式子来解出碰撞后的速度。 但这里介绍另一种更简单的方法, 即利用质心系求解。 令两质点质量分别为 $m_1$ 和 $m_2$, 初速度分别为 $v_{10}$ 和 $v_{20}$, 需要求末速度 $v_1$ 和 $v_2$。 根据定义, 质心的位置为 $x_c = (m_1 x_1 + m_2 x_2)/(m_1 + m_2)$, 等式两边对时间 $t$ 求导, 得质心的速度为
\begin{equation}\label{TwoCld_eq1}
v_c = (m_1 v_1 + m_2 v_2)/(m_1 + m_2)
\end{equation}
现在我们在质心系中考虑该问题。 在三维情况下, 质心系中的二体系统只有三个自由度(二体系统\upref{TwoBD}\autoref{TwoBD_eq3}), 不难类推在一维情况下二体系统只有一个自由度。 所以维度多少, 在质心系中考虑二体碰撞问题将会简单得多。 

由速度叠加原理, 初始时两质点在质心系中的速度分别为
\begin{equation}
v_{c10} = v_{10} - v_c \qquad v_{c20} = v_{20} - v_c
\end{equation}
先考虑质心系中的完全弹性碰撞, 由于两质点的速度大小始终成正比 (质心系\upref{CM}\autoref{CM_eq8}), 为了使能量守恒, 碰撞只能有一种结果, 即两质点的速度方向都取反方向而速度大小保持不变。 现在我们重新回到原来的参考系中, 两质点的末速度分别为
\begin{equation}
v_1 = v_c + (-v_{c10}) = 2v_c - v_{10}  \qquad v_2 = v_c + (-v_{c20}) = 2v_c - v_{20}
\end{equation}
代入\autoref{TwoCld_eq1} 即可得到最后结果。

若问题为非完全弹性碰撞, 可设质心系中碰撞后与碰撞前的能量比值为 $\alpha^2 < 1$, 即速度的比值为 $\alpha$。 碰撞后两质点的质心系速度分别变为 $-\alpha v_{c10}$ 和 $-\alpha v_{c20}$, 变换到原参考系中速度为
\begin{equation}\begin{aligned}
v_1 = v_c + (-\alpha v_{c10}) = (1 + \alpha) v_c - \alpha v_{10}\\
v_2 = v_c + (-\alpha v_{c20}) = (1 + \alpha) v_c - \alpha v_{20}
\end{aligned}\end{equation}


\subsection{二维和三维情况}
由于在多维情况下, 碰撞损失的能量可能与碰撞的角度有关, 这里仅讨论最常见的完全弹性碰撞。








