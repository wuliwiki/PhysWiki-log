% 范德蒙恒等式
% 排列组合|选择的展开定理

\pentry{二项式定理\upref{BiNor}}

\footnote{参考 Wikipedia \href{https://en.wikipedia.org/wiki/Vandermonde's_identity}{相关页面}}\textbf{范德蒙恒等式(Vandermonde's identity)}是指
\begin{equation}\label{ChExpn_eq1}
C_{a + b}^n = \sum_i C_a^i C_b^{n-i} = \sum\limits_i C_a^{n-i}C_b^i
\end{equation}
其中求和是对所有使得表达式有意义的非负整数 $i$ 进行的, 即
\begin{equation}\label{ChExpn_eq2}
\max\qty{0,\, n-b} \leqslant i \leqslant \min\qty{a,\, n}
\end{equation}
当 $a, b \geqslant n$ 时化简为
\begin{equation}
0 \leqslant i \leqslant n
\end{equation}


\subsection{证明1}

假设有编了号的 $a+b$ 个小球。 不分顺序抓取 $n$ 个, 求总共有几种情况(用 $N$ 表示)。

方法1:根据定义, 有 $N = C_{a+b}^n$ 种情况。

方法2:先把球分成 $A$,  $B$ 两组, 分别有 $a$ 个和 $b$ 个。 如果在 $A$ 组中抽取 $i$ 个球(有 $C_a^i$ 种情况), 在 $B$ 组中只能抽取  $n - i$ 个(有 $C_b^{n-i}$ 种情况), 所以一个 $i$ 对应 $C_a^i C_b^{n-i}$ 种情况。 所有可能的 $i$ 一共有 $N = \sum_i C_a^i C_b^{n-i}$ 种情况。

由于这个问题只有一个答案, 所以有 $C_{a+b}^n = \sum_i C_a^i C_b^{n-i}$。 

但 $i$ 的范围具体从多少取到多少, 由 $a$,  $b$ 是否大于 $n$ 来决定。 当 $a,b$ 都大于 $n$ 时, $i$ 可以从 0 取到 $n$,  如果其中至少有一个小于 $n$,  那么 $i$ 的取值不能使 $C$ 的上标大于下标。

证毕。

\subsection{证明2}
我们也可以通过二项式定理来证明
\begin{equation}
\begin{aligned}
(1 + x)^{a+b} &= \sum_{n=0}^{a+b} C_{a+b}^n x^n\\
&=(1+x)^a (1+x)^b\\
&=\sum_{i=0}^a C_a^i x^i  \sum_{j=0}^b C_b^j x^j\\
&=\sum_{n=0}^{a+b} \sum_{i} C_a^i C_b^{n-i} x^n
\end{aligned}
\end{equation}
最后两部可以看作对一个表格求和, 第 $i$ 行第 $j$ 列的值为 $C_a^i C_b^j x^{i+j}$, 求和的方式有两种, 一种是每行每列求和, 另一种是延斜线求和, 每条斜线 $n = i+j$ 的值相同。 注意每条斜线中 $i$ 的范围未必相同, 满足\autoref{ChExpn_eq2}。

由于多项式中每项的系数必须相同, 就有\autoref{ChExpn_eq1}, 证毕。
