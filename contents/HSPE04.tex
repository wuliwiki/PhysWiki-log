% 电路(高中)
% keys 电路|电源|电动势|内阻|电压表|电流表
% license Xiao
% type Tutor

\begin{issues}
\issueDraft
\issueTODO
\end{issues}

\pentry{恒定电流\nref{nod_HSPE03}}{nod_b0a2}

\subsection{常见元器件}

\textbf{电阻器}:有定值和变值两种类型。顾名思义,定值电阻器的电阻是固定的,而变值电阻器接入电路的电阻是可被改变的,如滑动变阻器、电阻箱等。

\textbf{电容器}:具有“通交流隔直流”的作用,也可以作为储能元件使用。

\textbf{电感器}:具有“通直流阻交流”的作用,也可以作为储能元件使用。

\textbf{二极管}:具有单向导电性,属于非线性元件。

\textbf{导线}:连接电源和其他元器件,起输送电能的作用,一般计算时忽略导线的电阻。

\textbf{开关}:控制电路的通断,连接时应断开,闭合后相当于导线。

利用导线,按实际需求将电源和其他元器件连接起来组成的闭合回路,叫做\textbf{电路}。

\subsection{电源}

通过非静电力做功将其他形式能转为电能的装置叫做\textbf{电源},常见的电源有干电池、蓄电池、锂电池、发电机等。

前面提及到,形成电流的条件包括:导体两端存在电势差和导体中存在自由电荷。若有带正电的$A$球和带负电的$B$球,用一根导线将它们相连,那么两球间存在电势差$U_{AB}$,$B$球中的电子会流向$A$球(产生电流),电中和后,$A$、$B$两球间没有电势差,不再形成电流。

为了在电路中形成持续的电流,就需要电源提供非静电力“搬运”自由电荷,保持导体两端的电势差。

\subsubsection{电动势}

电源用非静电力所做的功$W$与被移动的电荷量$q$之比,叫做电动势,用$E$表示:

\begin{equation}
E=\frac{W}{q}~.
\end{equation}

电动势的单位与电势、电势差的单位相同,都是伏特($\mathrm{V}$)。

电动势是描述电源用非静电力做功本领大小的物理量,电源的电动势越大,将其他形式能转化为电能的本领越强。

电路断开时,电源的电动势和电源两极的电压值在数值上相等。

\subsubsection{内阻}

电源的内部也是由导体组成,因此也存在电阻,这部分电阻叫做电源的\textbf{内阻},常用$r$来表示。

在闭合电路中,若外电路的电阻为$R$,则有

\begin{equation}
E=U_R+U_r=IR+Ir~.
\end{equation}

对上式变形可得闭合电路的欧姆定律:

\begin{equation}
I=\frac{E}{R+r}~.
\end{equation}

\subsection{电压表和电流表}

\addTODO{待补充图示}

\textbf{电压表}和\textbf{电流表}是电路实验中常用的两种仪表,它们可由量程较小的灵敏电流计(表头)改装而成。

\subsubsection{灵敏电流计}

灵敏电流计在电路中可看作是一个电阻,特殊之处在于流过它的电流可以在表盘上读出,对灵敏电流计进行改装时,需要先留意以下三个参数:

满偏电流$I_g$:指针指在表盘最大刻度时通过灵敏电流计的电流。

满偏电压$U_g$:通过灵敏电流计的电流达到满偏电流时,加在灵敏电流计两端的电压。

表头内阻$R_g$:灵敏电流计内部线圈的电阻,即灵敏电流计的内阻。

上述三个参数满足欧姆定律,即

\begin{equation}
U_g=I_g R_g~.
\end{equation}

\subsubsection{改装电压表}

如果直接用灵敏电流计作为测量电压的工具,那么它能测得的最大电压值就是$U_g$,$U_g$值较小,不适合实际应用。为了扩大量程,就需要串联一个电阻进行分压,使得改装后的电压表量程达到更大的电压值,这个串联的电阻叫做分压电阻。

设改装后的电压表量程变为$U$,分压电阻为$R$,则有

\begin{equation}
U=I_g(R_g+R)=\frac{R_g+R}{R_g}U_g~.
\end{equation}

可算得需要的分压电阻为

\begin{equation}
R=\frac{U-U_g}{U_g}R_g~.
\end{equation}

如果我们希望改装后的电压表量程是原来满偏电压的$n$倍,即$n=U/U_g$,则分压电阻

\begin{equation}
R=(n-1)R_g~.
\end{equation}

此时电压表的内阻

\begin{equation}
R_V=R_g+R=nR_g~.
\end{equation}

\subsubsection{改装电流表}

与改装电压表类似,为了增大电流量程,就需要并联一个电阻进行分流,这个并联的电阻叫做\textbf{分流电阻}。

设改装后的电流表量程为$I$,分流电阻为$R$,则有

\begin{equation}
I=\frac{Ug}{R_g}+\frac{U_g}{R}=\frac{R_g+R}{R}I_g~.
\end{equation}

需要分流的电阻为

\begin{equation}
R=\frac{I_gR_g}{I-I_g}~.
\end{equation}

如果我们希望改装后的电流表量程是原来满偏电流的$n$倍,即$n=I/I_g$,则分流电阻

\begin{equation}
R=\frac{R_g}{n-1}~.
\end{equation}

此时电流表的内阻

\begin{equation}
R_A=\frac{R_gR}{R_g+R}=\frac{R_g}{n}~.
\end{equation}
