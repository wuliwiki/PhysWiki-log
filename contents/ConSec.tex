% 圆锥曲线与圆锥(高中)
% keys 椭圆|双曲线|抛物线|圆锥|圆锥曲线
% license Xiao
% type Tutor

\begin{issues}
\issueDraft
\end{issues}

\pentry{椭圆\nref{nod_Elips3}, 抛物线\nref{nod_Para3}, 双曲线\nref{nod_Hypb3}, 右手定则\nref{nod_RHRul}, 平面旋转矩阵\nref{nod_Rot2D}}{nod_177c}




椭圆的存在如此普遍,以至于早在在公元前3世纪的古希腊时期,在没有坐标、没有代数、没有计算器的情况下,数学家\textbf{阿波罗尼奥斯(Apollonius of Perga)}就在《圆锥曲线论》中对它进行了研究。

阿波罗尼奥斯之前的古人相信天体运行是完美的圆。但随着时间推移,人们观察到了不那么完美的现象:行星轨迹有“逆行”,日月运行有周期波动,日月食不是完全等间距出现等等。这些“不完美”让人开始思考:也许圆不是唯一的完美曲线。在这样的气氛下,阿波罗尼奥斯提出圆锥曲线是一种比圆更丰富的几何家族,是一种可能的“自然运动路径”。


阿波罗尼乌斯受欧几里得《几何原本》的影响,希望将所有圆锥曲线系统化,像欧几里得那样构建一个逻辑完备的“曲线几何宇宙”。


\subsection{圆锥视角下的圆锥曲线}

圆锥曲线之所以叫做圆锥曲线,是因为它们可以由平面截取圆锥面得到(\autoref{fig_ConSec_1})。 然而由于这涉及较为繁琐的计算,所以初学圆锥曲线时我们往往先介绍更简单的定义,例如 “\enref{圆锥曲线的极坐标方程}{Cone}” 中的定义, 或者直接在 $x$-$y$ 直角坐标系中使用二次方程定义(见预备知识)。 以下我们来证明双圆锥被平面截出的任意曲线都是圆锥曲线。

\begin{figure}[ht]
\centering
\includegraphics[width=6cm]{./figures/4be0db073edf3f93.png}
\caption{圆锥的有限截面是一个椭圆(来自 Wikipedia)} \label{fig_ConSec_1}
\end{figure}

双圆锥面如\autoref{fig_ConSec_2} 所示。 在直角坐标系 $x$-$y$-$z$ 中, 为了方便我们使用顶角(两条母线的最大夹角)为 $\pi/2$ 的圆锥
\begin{figure}[ht]
\centering
\includegraphics[width=6cm]{./figures/cab7feed48b7007c.png}
\caption{\autoref{eq_ConSec_4} 表示的双圆锥面(修改自 Wikipedia)} \label{fig_ConSec_2}
\end{figure}
其方程为
\begin{equation}\label{eq_ConSec_4}
z_1^2 = x_1^2 + y_1^2~.
\end{equation}
对其他顶角的圆锥, 我们只需要把 $z$ 轴缩放一下即可。

我们可以再列出一个一般的平面方程与\autoref{eq_ConSec_4} 联立得到方程组, 但这样解出来的曲线将与 $x$-$y$ 平面未必平行。 所以更方便的办法是先把圆锥旋转一下, 再用某个和 $x$-$y$ 平面平行的平面 $z = z_0$ 去截出曲线。 这样就方便化为圆锥曲线的标准方程。 关于 $y$ 轴的\enref{旋转变换}{Rot2DT}为\footnote{\autoref{eq_ConSec_1} 和\autoref{eq_ConSec_2} 可以表示为 $3\times 3$ 的\enref{三维旋转矩阵}{Rot3D}。}
\begin{equation}\label{eq_ConSec_1}
\pmat{x_1\\z_1} = \pmat{\cos\theta & -\sin\theta\\ \sin\theta & \cos\theta}\pmat{x\\z}~,
\end{equation}
\begin{equation}\label{eq_ConSec_2}
y_1 = y~.
\end{equation}
代入\autoref{eq_ConSec_4} 得
\begin{equation}\label{eq_ConSec_3}
(\sin\theta\cdot x + \cos\theta\cdot z)^2 = (\cos\theta\cdot x - \sin\theta\cdot z)^2 + y^2~.
\end{equation}
这相当于把圆锥关于 $y$ 轴用\enref{右手定则}{RHRul}旋转了 $\theta$。 当 $\theta \ne \pi/4$ 时, 化成椭圆或双曲线的标准方程(\autoref{eq_Elips3_3}  \autoref{eq_Hypb3_4})
\begin{equation}
\frac{(x - \tan2\theta \cdot z)^2}{(z/\cos2\theta)^2} + \frac{y^2}{z^2/\cos2\theta} = 1~.
\end{equation}
长半轴、短半轴和离心率分别为
\begin{equation}
a = \frac{z}{\cos2\theta}~,
\qquad
b = \frac{z}{\sqrt{\abs{\cos2\theta}}}~,
\qquad
e = \sqrt{2}\sin\theta~.
\end{equation}
当 $\theta < \pi/4$ 时, 式中 $\cos2\theta > 0$, 就得到了椭圆($e < 1$), 反之则得到双曲线。

当 $\theta = \pi/4$, \autoref{eq_ConSec_3} 化为抛物线($e = 1$)的标准方程(\autoref{eq_Cone_9})
\begin{equation}
y^2 = 2zx~.
\end{equation}
