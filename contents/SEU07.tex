% 东南大学 2007 年 考研 量子力学
% license Usr
% type Note

\textbf{声明}:“该内容来源于网络公开资料,不保证真实性,如有侵权请联系管理员”

\subsection{一.}
设质量为 $m$ 的粒子在一维 $\delta(x)$ 势场

\[
V(x) = -V_0 \delta(x)~
\]
中运动,其中 $V_0 > 0$。试求:

\begin{enumerate}
    \item 粒子处于束缚态时的能级和对应的态函数;$\qquad $(15 分)
    \item 若粒子以能量 $E > 0$ 左入射上述势场,计算粒子的透射系数。 $\qquad $ (10 分)
\end{enumerate}
\subsection{二.}
 一个二维振子,其哈密顿量为

\[
H = \frac{1}{2} (p_x^2 + p_y^2) + \frac{1}{2} (1 + \beta xy)(x^2 + y^2)~
\]

其中 $\hbar = 1$ 和 $\beta \ll 1$。试求:

\begin{enumerate}
    \item 当 $\beta = 0$ 时,振子的能级和波函数; $\qquad $ (10 分)
    \item 当 $\beta \neq 0$ 时,第一激发态的能级修正(计算至一级微扰)。 $\qquad $ (15 分)
\end{enumerate}
\subsection{三.}
试利用测不准关系估算类氢原子中电子的基态能量.$\qquad $ (20 分)
\subsection{四.}
设一个自旋为1,电荷为 $e$ 的粒子处于磁场 $\vec{B} = B\vec{e}_z$ 中,其哈密顿量为

\[
H = -\vec{\mu} \cdot \vec{B} = \frac{e}{mc} \vec{B} \cdot \vec{S}~
\]

其中自旋为1的三个自旋矩阵为 (在 $S^2, S_z$ 表象)

\[
S_x = \frac{\hbar}{\sqrt{2}}
\begin{pmatrix}
0 & 1 & 0 \\
1 & 0 & 1 \\
0 & 1 & 0
\end{pmatrix}
, \quad
S_y = \frac{\hbar}{\sqrt{2}}
\begin{pmatrix}
0 & -i & 0 \\
i & 0 & -i \\
0 & i & 0
\end{pmatrix}
, \quad
S_z = \hbar
\begin{pmatrix}
1 & 0 & 0 \\
0 & 0 & 0 \\
0 & 0 & -1
\end{pmatrix}~
\]

设时间 $t = 0$ 时粒子的自旋在 $x$ 轴上的投影为 $+\hbar$。试求 $t > 0$ 时:

\begin{enumerate}
    \item 粒子的自旋态 $\chi(t)$; $\qquad $ (12 分)
    \item 粒子自旋在 $x$ 轴投影仍然为 $+\hbar$ 的几率; $\qquad $ (6 分)
    \item 粒子自旋 $(S_x, S_y, S_z)$ 的平均值。 $\qquad $ (12 分)
\end{enumerate}
