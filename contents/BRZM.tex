% 路德维希·玻尔兹曼(综述)
% license CCBYSA3
% type Wiki

本文根据 CC-BY-SA 协议转载翻译自维基百科\href{https://en.wikipedia.org/wiki/Ludwig_Boltzmann}{相关文章}。

\begin{figure}[ht]
\centering
\includegraphics[width=6cm]{./figures/ed66f21add0c5482.png}
\caption{1902年的玻尔兹曼} \label{fig_BRZM_1}
\end{figure}
\textbf{路德维希·爱德华·玻尔兹曼}(Ludwig Eduard Boltzmann,1844年2月20日-1906年9月5日)是奥地利的物理学家和哲学家。他的最大成就包括统计力学的发展和热力学第二定律的统计解释。1877年,他提出了当前的熵定义:\(S = k_{\rm B}\ln\Omega \)其中,Ω是系统能量等于宏观系统能量的微观状态数,解释为衡量系统统计无序度的一个指标。马克斯·普朗克将常数 \( k_B \) 命名为玻尔兹曼常数。

统计力学是现代物理学的基石之一。它描述了宏观观测(如温度和压力)如何与围绕平均值波动的微观参数相关。它将热力学量(如比热容)与微观行为联系起来,而在经典热力学中,唯一可用的方式是为不同材料测量并列出这些量。
\subsection{传记}  
\subsubsection{童年与教育}  
尔兹曼出生在维也纳的郊区厄尔德贝格(Erdberg),来自一个天主教家庭。他的父亲路德维希·乔治·玻尔兹曼(Ludwig Georg Boltzmann)是一名税务官员。他的祖父从柏林迁至维也纳,是一位钟表制造商,而玻尔兹曼的母亲凯瑟琳·保尔恩芬德(Katharina Pauernfeind)则来自萨尔茨堡。玻尔兹曼在家中接受教育,直到十岁才开始正式上学,之后在上奥地利州的林茨市读高中。15岁时,玻尔兹曼的父亲去世。

1863年起,玻尔兹曼在维也纳大学学习数学和物理学。他于1866年获得博士学位,并于1869年获得讲授资格(venia legendi)。玻尔兹曼与物理学研究所所长约瑟夫·斯特凡(Josef Stefan)密切合作,正是斯特凡将玻尔兹曼引入了麦克斯韦的研究成果。
\subsubsection{学术生涯} 
1869年,玻尔兹曼在25岁时,凭借约瑟夫·斯特凡的推荐信,[9] 被任命为格拉茨大学(位于施蒂利亚省)数学物理学全职教授。1869年,他在海德堡与罗伯特·本森(Robert Bunsen)和莱奥·凯尼茨贝格(Leo Königsberger)合作工作了几个月,随后在1871年与古斯塔夫·基尔霍夫(Gustav Kirchhoff)和赫尔曼·冯·亥姆霍兹(Hermann von Helmholtz)在柏林合作。1873年,玻尔兹曼加入维也纳大学,担任数学教授,并在此工作直到1876年。
\begin{figure}[ht]
\centering
\includegraphics[width=8cm]{./figures/2f19a04ecc7fbc4a.png}
\caption{1887年,路德维希·玻尔兹曼与格拉茨的同事们:(站立,左起)能斯特(Nernst)、斯特赖因茨(Streintz)、阿伦纽斯(Arrhenius)、海克(Hiecke);(坐着,左起)奥林格(Aulinger)、艾廷斯豪森(Ettingshausen)、玻尔兹曼(Boltzmann)、克莱门奇奇(Klemenčič)、豪斯曼宁格(Hausmanninger)。} \label{fig_BRZM_2}
\end{figure}
1872年,在女性尚未被允许进入奥地利大学之前,玻尔兹曼遇到了亨丽埃特·冯·艾根特勒(Henriette von Aigentler),她是一位有志成为数学和物理学教师的年轻女性,居住在格拉茨。她曾被拒绝非正式旁听讲座的许可。玻尔兹曼支持她提出上诉,最终她获得了成功。1876年7月17日,路德维希·玻尔兹曼与亨丽埃特结婚,他们育有三位女儿:亨丽埃特(1880年)、伊达(1884年)和埃尔泽(1891年);以及一个儿子,阿图尔·路德维希(1881年)。[10]玻尔兹曼回到格拉茨,担任实验物理学教席。在格拉茨的学生中,有斯万特·阿伦纽斯(Svante Arrhenius)和瓦尔特·能斯特(Walther Nernst)。[11][12] 他在格拉茨度过了14个快乐的年头,并且在那里发展了他关于自然的统计学概念。

玻尔兹曼于1890年被任命为德国巴伐利亚州慕尼黑大学的理论物理学教席。

1894年,玻尔兹曼继承了他的导师约瑟夫·斯特凡(Joseph Stefan)的职位,成为维也纳大学的理论物理学教授。[13]
\subsubsection{最后的岁月与去世}
玻尔兹曼在最后的岁月里投入了大量精力为自己的理论辩护。[14] 他与一些维也纳的同事关系不和,尤其是恩斯特·马赫(Ernst Mach),后者在1895年成为了哲学与科学史教授。那一年,乔治·赫尔姆(Georg Helm)和威廉·奥斯特瓦尔德(Wilhelm Ostwald)在吕贝克的一次会议上提出了他们的能量学观点。他们认为,能量而非物质才是宇宙的主要组成部分。在这场辩论中,玻尔兹曼的立场得到了其他物理学家的支持,尤其是那些支持他原子理论的物理学家。[15] 1900年,玻尔兹曼应威廉·奥斯特瓦尔德的邀请前往莱比锡大学。奥斯特瓦尔德为玻尔兹曼提供了物理学教授的职位,这个职位在古斯塔夫·海因里希·维德曼去世后空缺。由于健康原因,马赫退休后,玻尔兹曼于1902年返回维也纳。[14] 1903年,玻尔兹曼与古斯塔夫·冯·艾舍里奇(Gustav von Escherich)和埃米尔·穆勒(Emil Müller)共同创立了奥地利数学学会。他的学生包括卡尔·普里布拉姆(Karl Přibram)、保罗·埃伦费斯特(Paul Ehrenfest)和莉泽·迈特纳(Lise Meitner)。[14]

在维也纳,玻尔兹曼教授物理学,同时也讲授哲学。玻尔兹曼的自然哲学讲座非常受欢迎,并引起了相当大的关注。他的第一次讲座获得了巨大的成功。尽管选择了最大的讲堂,听众还是站满了楼梯。由于玻尔兹曼哲学讲座的巨大成功,皇帝邀请他参加宫殿的接待[具体时间待定]。[16]

1905年,玻尔兹曼应邀在加利福尼亚大学伯克利分校的夏季学期举办了一系列讲座,他在一篇受欢迎的文章《一位德国教授的埃尔多拉多之旅》中描述了这一经历。[17]

1906年5月,玻尔兹曼的精神状况恶化,院长在信中将其描述为“严重的神经衰弱”。这一症状表明他可能患有今天所诊断的双相情感障碍。[14][18] 四个月后,他在1906年9月5日自杀,死亡方式是上吊。那时他正与妻子和女儿在杜伊诺(位于特里斯特附近,彼时属奥地利)度假。[19][20][21][18] 他被安葬在维也纳的中央公墓(Zentralfriedhof)。他的墓碑上刻有玻尔兹曼的熵公式:\( S = k \cdot \log W \)[14]
\subsection{哲学}
玻尔兹曼的气体分子运动论似乎预设了原子和分子的现实性,但几乎所有德国哲学家以及许多科学家,如恩斯特·马赫和物理化学家威廉·奥斯特瓦尔德,都不相信它们的存在。[22] 玻尔兹曼通过原子论者詹姆斯·克拉克·麦克斯韦尔(James Clerk Maxwell)的一篇论文《气体的动力学理论插图》接触到了分子理论,文中将温度描述为依赖于分子速度,从而将统计学引入物理学。这启发了玻尔兹曼接受原子论并扩展该理论。[23]

玻尔兹曼写了许多哲学论文,如《关于无生命自然中过程的客观存在问题》(1897年)。他是一个现实主义者。[24] 在他的《关于叔本华学说》一文中,玻尔兹曼将自己的哲学称为唯物主义,并进一步说道:“唯心主义主张只有自我存在,所有的观念都是自我产生的,试图从中解释物质。而唯物主义则从物质的存在出发,试图从物质中解释感觉。”[25]
\subsection{物理学}
玻尔兹曼最重要的科学贡献是在基于热力学第二定律的气体分子运动论方面。这一点非常重要,因为牛顿力学没有区分过去和未来的运动,而鲁道夫·克劳修斯(Rudolf Clausius)发明熵来描述第二定律时,基于的是分子层面的分散或扩散,使得未来具有单向性。玻尔兹曼在25岁时接触到詹姆斯·克拉克·麦克斯韦尔(James Clerk Maxwell)关于气体分子运动论的工作,麦克斯韦尔假设温度是由分子碰撞引起的。麦克斯韦尔使用统计学创建了分子动能分布的曲线,玻尔兹曼在此基础上澄清并发展了动理论和熵的思想,基于统计原子理论,创建了麦克斯韦-玻尔兹曼分布,作为气体中分子速度的描述。[26] 是玻尔兹曼推导出了第一个方程,用于描述麦克斯韦尔和他自己所创建的概率分布的动态演化。[27] 玻尔兹曼的关键洞察是,分散发生是由于分子“状态”增加的统计概率。玻尔兹曼超越了麦克斯韦尔,他将自己的分布方程不仅应用于气体,还应用于液体和固体。玻尔兹曼还在1877年的论文中进一步扩展了他的理论,超越了卡诺(Carnot)、克劳修斯、麦克斯韦尔和凯尔文勋爵,证明了熵是由热、空间分离和辐射共同贡献的。[28] 麦克斯韦-玻尔兹曼统计学和玻尔兹曼分布仍然是经典统计力学基础中的核心内容。它们也适用于不需要量子统计的其他现象,并为温度的意义提供了深刻的洞察。

他多次尝试解释热力学第二定律,这些尝试涵盖了多个领域。他尝试了赫尔姆霍兹的单周期模型,[29][30] 像吉布斯那样的纯集合方法,像遍历理论那样的纯机械方法,组合论证,Stoßzahlansatz等。[31]
\begin{figure}[ht]
\centering
\includegraphics[width=6cm]{./figures/d5509ffade35b901.png}
\caption{玻尔兹曼1898年提出的I2分子图,显示了原子“敏感区域”(α,β)的重叠。} \label{fig_BRZM_3}
\end{figure}
自从约翰·道尔顿(John Dalton)在1808年,詹姆斯·克拉克·麦克斯韦尔(James Clerk Maxwell)在苏格兰和乔西亚·威拉德·吉布斯(Josiah Willard Gibbs)在美国的发现之后,大多数化学家都认同玻尔兹曼对原子和分子的信仰,但直到几十年后,物理学界的大部分学者才接受这一信念。玻尔兹曼与当时德国最权威的物理学期刊的主编长期争执,该主编拒绝让玻尔兹曼将原子和分子视为其他什么东西,而只是方便的理论构造。玻尔兹曼去世仅几年后,佩朗(Perrin)对胶体悬浮液(1908-1909)的研究,基于爱因斯坦1905年的理论研究,确认了阿伏伽德罗常数和玻尔兹曼常数的数值,从而使世界确信这些微小粒子确实存在。

引用普朗克的话:“熵和概率之间的对数联系首次由L.玻尔兹曼在他的气体分子运动理论中提出。”[32] 这个著名的熵公式是:[33]
\[
S = k_{\mathrm{B}} \ln W~
\]
其中 \( k_{\mathrm{B}} \) 是玻尔兹曼常数,\( \ln \) 是自然对数,\( W \)(德语词“Wahrscheinlichkeit”,意思是“概率”)是宏观状态发生的概率[34],或者更准确地说,是与系统宏观状态对应的可能微观状态的数量——即通过赋予不同的分子位置和动量,系统的(不可观察的)“方式”在(可观察的)热力学状态下可以实现的数量。玻尔兹曼的范式是一个由N个相同粒子组成的理想气体,其中 \( N_i \) 是处于第i个微观条件(位置和动量范围)的粒子数。 \( W \) 可以通过以下排列公式来计算:
\[
W = N! \prod_{i} \frac{1}{N_i!}~
\]
其中 \( i \) 遍历所有可能的分子条件,\( ! \) 表示阶乘。分母中的“修正”项考虑了处于相同条件下不可区分的粒子。

玻尔兹曼也可以被视为量子力学的先驱之一,因为他在1877年提出,物理系统的能级可能是离散的,尽管玻尔兹曼将此作为一种数学工具,并没有物理上的意义。[35]

玻尔兹曼熵公式的一个替代方案是克劳德·香农(Claude Shannon)在1948年提出的信息熵定义。[36] 香农的定义旨在用于通信理论,但也适用于所有领域。当所有概率相等时,它会简化为玻尔兹曼的表达式,但当然也可以在概率不相等时使用。它的优点在于,能够直接得到结果,无需使用阶乘或斯特林近似。然而,类似的公式早在玻尔兹曼的工作中就已出现,吉布斯(Gibbs)也明确提出过类似公式(参见参考文献)。
\subsection{玻尔兹曼方程} 
\begin{figure}[ht]
\centering
\includegraphics[width=6cm]{./figures/eba1b61e086d05c8.png}
\caption{玻尔兹曼的雕像位于维也纳大学主楼的庭院拱廊中} \label{fig_BRZM_4}
\end{figure}
玻尔兹曼方程是为了描述理想气体的动力学而发展起来的。其形式为:
\[
\frac{\partial f}{\partial t} + v \frac{\partial f}{\partial x} + \frac{F}{m} \frac{\partial f}{\partial v} = \frac{\partial f}{\partial t}\Big|_{\text{collision}}~
\]
其中,\( f \) 表示给定时刻单粒子位置和动量的分布函数(参见麦克斯韦-玻尔兹曼分布),\( F \) 是作用在粒子上的力,\( m \) 是粒子的质量,\( t \) 是时间,\( v \) 是粒子的平均速度。

该方程描述了单粒子相空间中密度分布云的位移和动量分布的时间和空间变化。(参见哈密顿力学。)左边的第一个项表示分布函数随时间的显式变化,第二项给出空间上的变化,第三项描述了作用于粒子的任何力的影响。方程右边则表示碰撞的影响。

原则上,给定适当的边界条件,上述方程可以完全描述气体粒子集合的动力学。这个一阶微分方程看起来 deceptively 简单,因为 \( f \) 可以表示任意单粒子的分布函数。此外,作用在粒子上的力直接依赖于速度分布函数 \( f \)。然而,玻尔兹曼方程的积分非常困难。大卫·希尔伯特(David Hilbert)曾花费多年时间尝试解决它,但没有真正的成功。

玻尔兹曼假设的碰撞项是近似的。然而,对于理想气体,玻尔兹曼方程的标准查普曼-恩斯科格(Chapman–Enskog)解法是非常准确的。只有在冲击波条件下,才有可能导致理想气体的错误结果。

玻尔兹曼多年来一直尝试通过他的气体动力学方程(即著名的H定理)“证明”热力学第二定律。然而,在制定碰撞项时,他做出的关键假设是“分子混乱”假设,这一假设破坏了时间反演对称性,而这是任何能够暗示第二定律的前提条件。正是仅凭这种概率假设,玻尔兹曼才看似成功,因此他与洛施密特(Loschmidt)及其他人之间关于洛施密特悖论的长期争论最终以他的失败告终。

最终,在1970年代,E. G. D. 科恩(E. G. D. Cohen)和J. R. 多夫曼(J. R. Dorfman)证明,将玻尔兹曼方程扩展到高密度的系统(即进行系统的幂级数展开)在数学上是不可能的。因此,密集气体和液体的非平衡统计力学更多地集中在格林-库博关系、波动定理和其他方法上。
\subsection{热力学第二定律作为无序法则}
\begin{figure}[ht]
\centering
\includegraphics[width=6cm]{./figures/878c0a1e9d391e67.png}
\caption{玻尔兹曼的墓地位于维也纳中央公墓,墓上有雕像和熵公式} \label{fig_BRZM_5}
\end{figure}
热力学第二定律或“熵定律”是无序法则(或者动态有序状态是“无限不可能”的)的观点,源于玻尔兹曼对热力学第二定律的理解。

特别是,玻尔兹曼试图将其简化为一个随机碰撞函数,或者是由机械粒子随机碰撞所引发的概率法则。沿袭麦克斯韦的思路,玻尔兹曼将气体分子模型化为在盒子里碰撞的台球,并指出每次碰撞后,非平衡速度分布(同速且同向运动的分子组)将变得越来越无序,最终导致宏观均匀状态和最大微观无序状态,即最大熵状态(其中宏观均匀性对应于所有场势或梯度的消失)。他认为,第二定律因此仅仅是机械碰撞粒子世界中无序状态最为可能的结果。由于无序状态的可能性远大于有序状态,因此系统几乎总是会处于最大无序状态——即具有最大可接入微观状态数的宏观状态(例如,处于平衡状态的盒子中的气体)——或朝这个状态发展。玻尔兹曼总结道,动态有序状态,即分子“以相同速度和相同方向”运动的状态,是“最不可能的情况……一种无限不可能的能量配置”。

玻尔兹曼的伟大成就之一,是证明了热力学第二定律仅仅是一个统计事实。能量的逐渐无序化类似于初始有序的扑克牌在反复洗牌过程中逐渐变得无序,就像扑克牌经过数不清的洗牌后会重新回到原始顺序一样,整个宇宙终有一天也会通过纯粹的偶然恢复到它最初的状态(这个关于宇宙终结的乐观结论在尝试估计它自发发生之前所需的时间时显得有些微弱)。熵增加的趋势似乎让热力学初学者感到困惑,但从概率理论的角度来看,它是很容易理解的。考虑两个普通的骰子,六点朝上。当骰子被摇动后,两个六点朝上的概率很小(1/36);因此,可以说骰子的随机运动(就像分子因为热能而发生的混乱碰撞)使得不太可能的状态变成了更可能的状态。对于数百万个骰子,就像热力学计算中涉及的数百万个原子一样,所有骰子都显示六点的概率变得如此微不足道,以至于系统必须进入一个更可能的状态。
\subsection{遗产与对现代科学的影响}
路德维希·玻尔兹曼对物理学和哲学的贡献对现代科学产生了深远的影响。他在统计力学和热力学方面的开创性工作为物理学中一些最基本的概念奠定了基础。例如,马克斯·普朗克在其黑体辐射理论中量化共振器时,使用玻尔兹曼常数来描述系统的熵,并最终在1900年得出了他的公式。然而,玻尔兹曼的工作在他生前并未立即被广泛接受,他也遭遇了一些同时代人的反对,尤其是在原子和分子存在性的争议上。尽管如此,他的思想的有效性和重要性最终得到了认可,并且这些思想已成为现代物理学的基石。接下来,我们将探讨玻尔兹曼遗产的一些方面,以及他对各个科学领域的影响。
\subsubsection{原子理论与原子和分子的存在}
玻尔兹曼的气体动力学理论是最早尝试通过个别原子和分子的行为来解释宏观性质(如压力和温度)的一种理论。尽管许多化学家早已接受了原子和分子的存在,但广泛的物理学界花了一些时间才接受这一观点。玻尔兹曼与一位著名德国物理学期刊编辑之间长时间的争论,凸显了这一观点最初的抵制。

直到一些实验,例如让·佩朗的胶体悬浮液研究,确认了阿伏伽德罗常数和玻尔兹曼常数的数值后,原子和分子的存在才被广泛接受。玻尔兹曼的气体动力学理论在证明原子和分子的现实性以及解释气体、液体和固体中的各种现象方面起了关键作用。
\subsubsection{统计力学与玻尔兹曼常数}
玻尔兹曼开创的统计力学将宏观观测与微观行为联系起来。他对热力学第二定律的统计解释是一个重大成就,他提供了目前熵的定义:\(S = k_{\rm{B}} \ln \Omega \)其中 \( k_{\rm{B}} \) 是玻尔兹曼常数,\( \Omega \) 是对应于给定宏观态的微观态的数量。

马克斯·普朗克后来将这个常数命名为玻尔兹曼常数,以表彰玻尔兹曼在统计力学方面的贡献。玻尔兹曼常数现在已成为物理学以及许多科学学科中的基本常数。
\subsubsection{玻尔兹曼方程与现代应用}
由于玻尔兹曼方程在解决稀薄或稀疏气体问题中具有实用性,它已被应用于许多不同的技术领域。例如,它被用于计算航天飞机在高层大气中的再入过程。[43] 它也是中子传输理论的基础,并且在半导体中的离子传输中也有应用。[44][45]
\subsection{量子力学的影响}
Boltzmann在统计力学方面的工作为理解具有大量自由度的系统中粒子的统计行为奠定了基础。在他1877年的论文中,他将物理系统的离散能级作为数学工具,并展示了同样的方法可以应用于连续系统。这可以看作是量子力学发展的先驱之一。[46] 一位Boltzmann的传记作家指出,Boltzmann的方法“为普朗克铺平了道路。”[47]  

能级量子化成为量子力学中的基本假设,导致了量子电动力学和量子场论等突破性理论的发展。因此,Boltzmann关于能级量子化的早期见解对量子物理的发展产生了深远的影响。
\subsection{著作}
\begin{itemize}
\item 《与远距离作用理论的关系,静电学的特殊情况,静态流动与感应》(德文)。第2卷。莱比锡:Johann Ambrosius Barth出版社,1893年。
\item 《范德瓦尔斯理论,含有复合分子的气体,气体解离,结语》(德文)。第2卷。莱比锡:Johann Ambrosius Barth出版社,1896年。
\item 《单原子分子气体理论,其分子尺寸相较于平均自由程趋近于零》(德文)。第1卷。莱比锡:Johann Ambrosius Barth出版社,1896年。
\item 《静止、均匀、各向同性体的基本方程分类》(德文)。第1卷。莱比锡:Johann Ambrosius Barth出版社,1908年。
\item 《气体理论讲座》(法文)。巴黎:Gauthier-Villars出版社,1922年。
\end{itemize}
\begin{figure}[ht]
\centering
\includegraphics[width=6cm]{./figures/331308c722043f7e.png}
\caption{《气体理论讲座》第一卷和第二卷(1896-1898年)} \label{fig_BRZM_6}
\end{figure}
\begin{figure}[ht]
\centering
\includegraphics[width=6cm]{./figures/1108aea425cc2d1d.png}
\caption{《气体理论讲座》第一卷和第二卷的标题页(1896-1898年)} \label{fig_BRZM_7}
\end{figure}
\begin{figure}[ht]
\centering
\includegraphics[width=6cm]{./figures/8ffe6935adbdae93.png}
\caption{《气体理论讲座》第一卷和第二卷的目录(1896-1898年)} \label{fig_BRZM_8}
\end{figure}
\begin{figure}[ht]
\centering
\includegraphics[width=6cm]{./figures/b11cda6c02981ac6.png}
\caption{《气体理论讲座》第一卷和第二卷的导言(1896-1898年)} \label{fig_BRZM_9}
\end{figure}
\subsection{奖项与荣誉} 
1885年,他成为奥地利帝国科学院的成员;1887年,他成为格拉茨大学的校长。1888年,他被选为瑞典皇家科学院的会员,1899年成为英国皇家学会的外籍会员(ForMemRS)。许多事物以他的名字命名。
\subsection{参见}  
\begin{itemize}
\item 热力学  
\item 统计力学  
\item 玻尔兹曼脑
\end{itemize}
\subsection{参考文献}  
\begin{enumerate}
\item "Fellows of the Royal Society". 伦敦:皇家学会。原文存档于2015年3月16日。  
\item "Boltzmann". 《牛津英语词典》(在线版)。牛津大学出版社。doi:10.1093/OED/6830903157。(需要订阅或参与机构会员资格)。  
\item "Boltzmann constant". Merriam-Webster.com 字典。Merriam-Webster。  
\item Klein, Martin (1970) [1768]. "Boltzmann, Ludwig". 收录于Preece, Warren E.(编辑)《大英百科全书》(精装版)。第3卷(为1970年世博会纪念版)。芝加哥:威廉·本顿出版社。第893页。ISBN 0-85229-135-3。  
\item Partington, J.R. (1949), 《高级物理化学教科书》,第1卷,基本原理,气体的性质,伦敦:Longmans, Green and Co.,第300页。  
\item Gibbs, Josiah Willard (1902). 《统计力学的基础原理》。纽约:查尔斯·斯克里布内尔儿子公司。  
\item Simmons, John; Simmons, Lynda (2000). 《科学100人》。肯辛顿出版社,第123页。ISBN 978-0-8065-3678-1。  
\item James, Ioan (2004). 《著名物理学家:从伽利略到汤川秀树》。剑桥大学出版社,第169页。ISBN 978-0-521-01706-0。  
\item Južnič, Stanislav (2001年12月). "Ludwig Boltzmann and the First Student of Physics and Mathematics of Slovene Descent" [路德维希·玻尔兹曼与第一位斯洛文尼亚裔物理学和数学学生]。《Kvarkadabra》(斯洛文尼亚语)(第12期)。2012年2月17日检索。  
\item Fasol, Gerhard. "Ludwig Boltzmann biography (1844年2月20日-1906年9月5日)"。Ludwig Boltzmann网站。2024年5月20日检索。
\item Jäger, Gustav; Nabl, Josef; Meyer, Stephan (1999年4月). "Three Assistants on Boltzmann". 《Synthese》, 119(1–2): 69–84. doi:10.1023/A:1005239104047. S2CID 30499879. 保罗·埃伦费斯特(1880–1933)与能斯特、阿伦纽斯和迈特纳一起,必须被认为是玻尔兹曼最杰出的学生之一。  
\item "Walther Hermann Nernst". 原文存档于2008年6月12日。沃尔特·赫尔曼·能斯特曾参加路德维希·玻尔兹曼的讲座。  
\item "路德维希·玻尔兹曼:维也纳的新的理论物理学的发明者"。www.iqoqi-vienna.at。2024年10月15日检索。  
\item Cercignani, Carlo (1998). 《路德维希·玻尔兹曼:信任原子的男人》。牛津大学出版社。ISBN 978-0-19-850154-1  
\item Max Planck (1896). "Gegen die neure Energetik". 《物理年鉴》。57(1): 72–78。Bibcode:1896AnP...293...72P. doi:10.1002/andp.18962930107.  
\item 《玻尔兹曼方程:理论与应用》,E. G. D. Cohen, W. Thirring(编辑),Springer科学与商业媒体,2012年  
\item 玻尔兹曼, 路德维希 (1992年1月1日). "一位德国教授的埃尔多拉多之行"。《今日物理》, 45(1): 44–51。Bibcode:1992PhT....45a..44B. doi:10.1063/1.881339. ISSN 0031-9228.  
\item Nina Bausek 和 Stefan Washietl (2018年2月13日). "科学中的悲剧性死亡:路德维希·玻尔兹曼——一个陷入混乱的头脑"。Paperpile。2020年4月26日检索。  
\item Muir, Hazel, 《尤里卡!科学的伟大思想家及其关键突破》,第152页,ISBN 1-78087-325-5  
\item 玻尔兹曼, 路德维希 (1995). "结论"。收录于Blackmore, John T.(编辑)《路德维希·玻尔兹曼:他的晚年生活与哲学,1900–1906》第二卷。Springer,第206-207页。ISBN 978-0-7923-3464-4.  
\item 在玻尔兹曼去世后,弗里德里希(“弗里茨”)哈森厄尔成为他在维也纳物理学教授职位的继任者。
\item Bronowski, Jacob (1974). 《世界之内的世界》。 《人类的崛起》。Little Brown & Co. 第265页。ISBN 978-0-316-10930-7.  
\item Nancy Forbes, Basil Mahon (2019). 《法拉第、麦克斯韦与电磁场》。第11章。ISBN 978-1633886070。[需要完整引文]  
\item Cercignani, Carlo. 《路德维希·玻尔兹曼:信任原子的男人》。ISBN 978-0198570646。[需要完整引文]  
\item Cercignani, Carlo (2008). 《路德维希·玻尔兹曼:信任原子的男人》(再版)。牛津:牛津大学出版社。第176页。ISBN 978-0-19-850154-1.  
\item Ludwig Boltzmann, 《气体理论讲座》,由Stephen G. Brush翻译,《翻译者序言》,1968年。  
\item Penrose, Roger. "前言"。收录于Cercignani, Carlo, 《路德维希·玻尔兹曼:信任原子的男人》,ISBN 978-0198570646.
\item Boltzmann, Ludwig (1877). 由Sharp, K.; Matschinsky, F.翻译。“论机械热力学第二基本定理与关于热平衡条件的概率计算之间的关系”。《帝国科学院会议报告》。数学-自然科学类。第二部分,第LXXVI期,76:373–435。维也纳。再版于《科学论文集》,第II卷,再版42,第164–223页,巴特出版社,莱比锡,1909年。Entropy 2015, 17, 1971–2009. doi:10.3390/e17041971  
\item Príncipe, João (2014),de Paz, María; DiSalle, Robert (编),“亨利·庞加莱:机械解释的地位与统计力学的基础”,收录于《庞加莱,科学哲学家:问题与视角》,《西安大略大学哲学系列:科学哲学》第79卷,荷兰多德雷赫特:Springer出版社,第127–151页,doi:10.1007/978-94-017-8780-2_8,hdl:10174/13352,ISBN 978-94-017-8780-2,检索日期:2024年5月28日  
\item Klein, Martin J. (1974),Seeger, Raymond J.; Cohen, Robert S. (编),“玻尔兹曼、单车与机械解释”,收录于《科学哲学的哲学基础》,《波士顿科学哲学研究系列》第11卷,荷兰多德雷赫特:Springer出版社,第155–175页,doi:10.1007/978-94-010-2126-5_8,ISBN 978-90-277-0376-7,检索日期:2024年5月28日  
\item Uffink, Jos (2022),“玻尔兹曼在统计物理中的工作”,收录于Zalta, Edward N. (编),《斯坦福哲学百科全书》(2022年夏季版),斯坦福大学形而上学研究实验室,检索日期:2024年5月28日  
\item Max Planck,第119页  
\item 熵的概念由鲁道夫·克劳修斯于1865年引入。他是第一个阐述热力学第二定律的人,他说:“熵总是增加的。”  
\item Pauli, Wolfgang (1973). 《统计力学》。剑桥:MIT出版社。ISBN 978-0-262-66035-8,第21页  
\item Boltzmann, Ludwig (1877). 由Sharp, K.; Matschinsky, F.翻译。“论机械热力学第二基本定理与关于热平衡条件的概率计算之间的关系”。《帝国科学院会议报告》。数学-自然科学类。第二部分,第LXXVI期,76:373–435。维也纳。再版于《科学论文集》,第II卷,再版42,第164–223页,巴特出版社,莱比锡,1909年。Entropy 2015, 17, 1971–2009. doi:10.3390/e17041971
\item “《信息的数学理论》由Claude E. Shannon著作”。cm.bell-labs.com。原文存档于2007年5月3日。  
\item Maxwell, J. (1871). 《热力学理论》。伦敦:Longmans, Green & Co.  
\item Boltzmann, L. (1974). 《热力学第二定律》。普及文献,文章3,1886年5月29日帝国科学院正式会议演讲,收入《路德维希·玻尔兹曼,理论物理学与哲学问题》,S. G. Brush(翻译)。波士顿:Reidel出版社。(原著发表于1886年)  
\item Boltzmann, L. (1974). 《热力学第二定律》。第20页  
\item 《Collier百科全书》,第19卷,Phyfe至Reni,“物理学”,David Park编,第15页  
\item 《Collier百科全书》,第22卷,Sylt至乌拉圭,热力学,Leo Peters编,第275页  
\item A. Douglas Stone,《爱因斯坦与量子》,第一章“绝望的行为”。2015年。  
\item Neunzert, H., Gropengießer, F., Struckmeier, J. (1991). 《玻尔兹曼方程的计算方法》。收录于Spigler, R.(编)《应用与工业数学》,数学与应用系列,第56卷,Springer出版社,荷兰多德雷赫特。doi:10.1007/978-94-009-1908-2_10  
\item 《半导体与半导体器件的高级理论 数值方法与仿真》/ Umberto Ravaioli [PDF文档] [链接]  
\item 《玻尔兹曼传输方程解的概述:中子、光子与电子在笛卡尔几何中的应用》,Barbara D. do Amaral Rodriguez,Marco Tullio Vilhena,2009年国际核能大西洋会议 - INAC 2009,巴西里约热内卢,2009年9月27日至10月2日,巴西核能协会-ABEN ISBN 978-85-99141-03-8  
\item Sharp, K.; Matschinsky, F. 翻译《路德维希·玻尔兹曼论文《论机械热力学第二基本定理与关于热平衡条件的概率计算之间的关系》》。《帝国科学院会议报告》。数学-自然科学类。第II部分,LXXVI期,1877,第373-435页(维也纳,1877,76:373-435)。再版于《科学论文集》,第II卷,再版42,第164-223页,巴特出版社,莱比锡,1909年。Entropy 2015, 17, 1971-2009. [DOI链接](https://doi.org/10.3390/e17041971) [文章链接](https://www.mdpi.com/1099-4300/17/4/1971)  
\item Cercignani, Carlo, 《路德维希·玻尔兹曼:信任原子的男人》,第12.3章《黑体辐射》,2006年,ISBN 978-0198570646.
\end{enumerate}
\subsection{进一步阅读}
\begin{itemize}
\item Roman Sexl & John Blackmore (编),《路德维希·玻尔兹曼——选集》,(《路德维希·玻尔兹曼全集,第8卷》),Vieweg,布伦瑞克,1982年。
\item John Blackmore (编),《路德维希·玻尔兹曼——他的晚年与哲学,1900–1906年,第1卷:文献史》,Kluwer,1995年。ISBN 978-0-7923-3231-2
\item John Blackmore,《路德维希·玻尔兹曼——他的晚年与哲学,1900–1906年,第2卷:哲学家》,Kluwer,荷兰多德雷赫特,1995年。ISBN 978-0-7923-3464-4
\item John Blackmore (编),《路德维希·玻尔兹曼——动荡的天才作为哲学家》,收录于《Synthese》,第119卷,第1期和第2期,1999年,1-232页。
\item Blundell, Stephen; Blundell, Katherine M. (2006). 《热物理学概念》,牛津大学出版社,第29页。ISBN 978-0-19-856769-1
\item Boltzmann, Ludwig, 《路德维希·玻尔兹曼——生活与书信》,Walter Hoeflechner 编,学术出版社。奥地利格拉茨,1994年。
\item Brush, Stephen G. (编与翻译),《玻尔兹曼,气体理论讲座》,加利福尼亚大学出版社,1964年
\item Brush, Stephen G. (编),《动理论》,纽约:Pergamon出版社,1965年
\item Brush, Stephen G. (1970)。 《玻尔兹曼》,收录于Charles Coulston Gillispie(编)《科学传记词典》。纽约:Scribner。ISBN 978-0-684-16962-0
\item Brush, Stephen G. (1986). 《我们称之为热的运动:气体动理论的历史》,阿姆斯特丹:North-Holland。ISBN 978-0-7204-0370-1
\item Cercignani, Carlo (1998). 《路德维希·玻尔兹曼:信任原子的男人》,牛津大学出版社。ISBN 978-0-19-850154-1
\item Darrigol, Olivier (2018). 《原子、力学与概率:路德维希·玻尔兹曼的统计力学》,牛津大学出版社。ISBN 978-0-19-881617-1
\item Ehrenfest, P. & Ehrenfest, T. (1911) 《统计观点在力学中的概念基础》,收录于《数学科学百科全书》卷IV,第二部分(F. Klein和C. Müller编)。莱比锡:Teubner,第3-90页。翻译为《力学中的统计方法的概念基础》,纽约:康奈尔大学出版社,1959年。ISBN 0-486-49504-3
\item Everdell, William R (1988). 《连续性问题与现代主义的起源:1870–1913》。欧洲思想史,9(5):531–552。doi:10.1016/0191-6599(88)90001-0
\item Everdell, William R (1997). 《首批现代人》。芝加哥:芝加哥大学出版社。ISBN 9780226224800
\item Gibbs, Josiah Willard (1902). 《统计力学中的基础原理》,特别针对热力学的理性基础发展。纽约:查尔斯·斯克里布内尔公司。
\item Johnson, Eric (2018). 《焦虑与方程:理解玻尔兹曼的熵》,麻省理工学院出版社。ISBN 978-0-262-03861-4
\item Klein, Martin J. (1973). 《玻尔兹曼统计思想的发展》,收录于E.G.D. Cohen 和 W. Thirring(编)《玻尔兹曼方程:理论与应用》,Acta Physica Austriaca Supplement 10。维也纳:Springer,第53-106页。ISBN 978-0-387-81137-6
\item Lindley, David (2001). 《玻尔兹曼的原子:引发物理学革命的大辩论》,纽约:自由出版社。ISBN 978-0-684-85186-0
\item Lotka, A. J. (1922). 《进化的能量学贡献》,《美国国家科学院学报》8(6):147-151。Bibcode:1922PNAS....8..147L。doi:10.1073/pnas.8.6.147。PMC 1085052。PMID 16576642
\item Meyer, Stefan (1904). 《为庆祝路德维希·玻尔兹曼六十岁生日的纪念文集》,J.A. Barth出版。
\item Planck, Max (1914). 《热辐射理论》。P. Blakiston Son & Co. 英文翻译由Morton Masius,第二版《热辐射》翻译,Dover出版社再版(1959)与(1991)。ISBN 0-486-66811-8
\item Sharp, Kim (2019). 《熵与计数的道:统计力学与热力学第二定律简明介绍》(《物理学Springer简报》)。Springer Nature。ISBN 978-3030354596
\item Tolman, Richard C. (1938). 《统计力学原理》,牛津大学出版社。再版:Dover出版社(1979)。ISBN 0-486-63896-0
\end{itemize}
\subsection{外部链接}
\begin{itemize}
\item Uffink, Jos (2004). "玻尔兹曼在统计物理学中的工作",《斯坦福哲学百科全书》。检索于2007年6月11日。
\item 《路德维希·玻尔兹曼——无序的天才》(YouTube)
\item O'Connor, John J.; Robertson, Edmund F.,《路德维希·玻尔兹曼》,《MacTutor数学历史档案》,圣安德鲁斯大学。
\item Ruth Lewin Sime,《丽泽·迈特纳:物理学中的一生》第1章:维也纳的少女时代,讲述了丽泽·迈特纳对玻尔兹曼的教学和事业的回忆。
\item Eftekhari, Ali, "路德维希·玻尔兹曼(1844–1906)"。讨论玻尔兹曼的哲学观点,包含许多引用。
\item Rajasekar, S.; Athavan, N. (2006年9月7日)。《路德维希·爱德华·玻尔兹曼》。arXiv:physics/0609047。
\item 路德维希·玻尔兹曼,数学家家谱项目
\item Weisstein, Eric Wolfgang (编),《玻尔兹曼,路德维希(1844–1906)》,《ScienceWorld》。
\end{itemize}