% 刚体的静力平衡
% keys 静力平衡|合外力|刚体|合力矩
% license Xiao
% type Tutor

\pentry{角动量定理\nref{nod_AMLaw}}{nod_c2f8}

\footnote{参考 Wikipedia \href{https://en.wikipedia.org/wiki/Mechanical_equilibrium}{相关页面}和 \cite{新力}。}在惯性系中, 如果刚体所受的所有合外力与合外\enref{力矩}{Torque}都为零, 则我们说它处于\textbf{静力平衡(static equilibrium)}。 其中合外力(矩)是指所有施加在刚体上的力(矩)的矢量和。

\begin{theorem}{刚体的静力平衡}
若一个刚体处于静力平衡, 那么它将保持静止或者做以下两种运动的组合: 1. 质心做匀速运动, 2. 绕质心做定轴匀速转动。
\end{theorem}
至于定理中的哪种情况会发生, 取决于初始时刚体的状态: 若初始时刚体开始静止, 那么受力平衡条件下它将保持静止, 否则保持初始时的平动或/和转动。 注意当合外力为零时, 合外力矩与参考点(参考系)的选取无关(\autoref{eq_Torque_5}~\upref{Torque})。 在非惯性系中, 若加入惯性力的修正, 该结论仍然成立。 

\textbf{证明}:

1. 合力: 刚体合外力为零时刚体\enref{动量守恒}{PLaw}, 而动量等于 “\enref{质心的动量” }{SysMom}
$\bvec p_c = M_c \bvec v_c$,所以质心做匀速运动或不动。

2. 合力矩: 刚体合外力矩为零时,其\enref{角动量守恒}{AMLaw},而刚体的角动量等于质心的角动量 $\bvec L_c =\bvec r_c\cross \bvec p_c$ 加上质心系中的角动量(\autoref{eq_AngMom_3}~\upref{AngMom})。 当质心匀速直线运动或不动时 $\bvec L_c$ 不变,所以质心系中刚体的角动量也不变,所以刚体绕质心做匀速转动或不转动。 证毕。

\begin{example}{轻杆三力平衡}
如图, 一个长度为 $L$ 质量不计的细杆, 两端受力分别为 $\bvec F_1, \bvec F_3$, 中间受力为 $\bvec F_2$。
\addTODO{图, 选取不同受力点}
\end{example}

\addTODO{吊桥例题, 见 EP1 20201021, 缆绳受力与重物位置的关系。}

\begin{example}{}\label{ex_RBSt_1}
如\autoref{fig_RBSt_1}, 一个质量为 $m$ 的线轴被斜挂在墙上, 线轴与墙面的摩擦系数为 $\mu$,线轴的大圆半径为 $R$, 小圆半径为 $r$, 求当角 $\alpha$ 满足什么条件时, 线轴才能不滑落。
\begin{figure}[ht]
\centering
\includegraphics[width=5cm]{./figures/3e7795769ba273ff.pdf}
\caption{线轴的平衡} \label{fig_RBSt_1}
\end{figure}

我们先来看线轴受哪几个力:重力 $mg$, 绳的拉力 $T$, 墙的支持力 $N$ 和摩擦力 $f$。 由摩擦系数的定义和刚体平衡条件可得
\begin{equation}
\begin{cases}
f \leqslant \mu N~, & \text{(摩擦系数)}\\
N - T\sin\alpha = 0~, & \text{(水平方向受力平衡)}\\
T\cos\alpha + f - mg = 0~, & \text{(竖直方向受力平衡)}\\
Tr - fR = 0~. & \text{(力矩平衡)}
\end{cases}
\end{equation}
其中最后一条力矩平衡是以圆心为原点计算力矩, 虽然原则上我们可以取任意点计算力矩, 但取在圆心计算最为简单。 除了 $\alpha$ 我们有三个未知数 $T, f, N$, 用以上三条等式恰好可以把这三个未知数消去, 可得关于 $\alpha$ 的不等式
\begin{equation}
\sin\alpha \geqslant \frac{r}{\mu R}~.
\end{equation}

一个有趣的地方在于, 不等式中没有出现质量 $m$。 事实上, 我们不使用那条含有 $mg$ 的等式也可以顺利解出答案。
\end{example}

\begin{example}{二人抬物}
\begin{figure}[ht]
\centering
\includegraphics[width=7cm]{./figures/c1435d940d164801.pdf}
\caption{二人高低抬物} \label{fig_RBSt_2}
\end{figure}
两个人分别通过一个受力点保持一个刚体的静力平衡, 若已知质心的位置和两个受力点的位置(三者在同一竖直平面上), 试分析谁出的力更大。 分为两种情况讨论: 1. 每个人只允许在竖直方向上出力, 2. 允许任意方向的力。 3. 在第二问中, 什么条件下第一个人出的合力模长最小?

1. 令重心为坐标原点, 两个受力点的坐标为 $(x_1, y_1)$, $(x_2, y_2)$, 如果两个受力点只有竖直方向的力(向上为正), 那么受力平衡得
\begin{equation}
F_1 + F_2 = Mg~.
\end{equation}
以重物的质心为坐标原点, 则重力力矩为零, 力矩平衡条件为
\begin{equation}
F_1 x_1 + F_2 x_2 = 0~,
\end{equation}
解得
\begin{equation}
F_1 = \frac{Mg x_2}{x_2 - x_1} ~,\qquad F_2 = \frac{-Mg x_1}{x_2 - x_1}~.
\end{equation}
可见力 $\abs{F_i}$ 和 $\abs{x_i}$ 成反比, 重心离谁的水平距离越近, 谁的受力就越大。 注意该解要求 $x_1 \ne x_2$, 下同。

作为一个具体的例子, 若二人搬的是均匀长方体, 由图可知, 若搬箱子底部, 则 $F_1 > F_2$, 若拉箱子顶部, 则 $F_1 < F_2$。 (图未完成)

2. 再来看一般情况, 每个人施加的力除了竖直分量还有水平分量。 令水平的力为 $F_1', F_2'$ (向右为正), 由水平方向和竖直方向的受力平衡和力矩平衡得
\begin{equation}
\begin{aligned}
&F_1' + F_2' = 0\\
&F_1 + F_2 = Mg\\
&F_1 x_1 + F_2 x_2 - F_1' y_1 - F_2' y_2 = 0~.
\end{aligned}
\end{equation}
这里四个未知的力只有三条约束方程, 所以有一个\enref{自由度}{DoF}。 以 $F_1'$ 为参数, 解得
\begin{equation}\label{eq_RBSt_1}
\begin{aligned}
F_1 = \frac{Mg x_2 + F_1'(y_2 - y_1)}{x_2 - x_1}~,\\
F_2 = \frac{-Mg x_1 - F_1'(y_2 - y_1)}{x_2 - x_1}~.
\end{aligned}
\end{equation}
我们发现, 由于 $F_1'$ 可取任意实数, 所以无论两个受力点在哪里, 只要 $y_1 \ne y_2$ 总能通过调节 $F_1'$ 使 $\abs{F_1} > \abs{F_2}$, 即第一人的合力 $\sqrt{F_1^2 + F_1'^2}$ 大于第二人, 或者即第一人的合力小于第二人。

3. 那么, $F_1'$ 为多少时, 第一人的合力最小呢? 合力模长的平方为 $F_1^2 + F_1'^2$, 把\autoref{eq_RBSt_1} 代入并令
\begin{equation}
\dv{F_1'}\qty(F_1^2 + F_1'^2) = 0~,
\end{equation}
得
\begin{equation}
F_1' = -\frac{Mgx_2 (y_2 - y_1)}{(x_2-x_1)^2 + (y_2 -y_1)^2}~,
\end{equation}
\begin{equation}
F_1 = \frac{Mg x_2(x_2-x_1)}{(x_2-x_1)^2 + (y_2 -y_1)^2}~.
\end{equation}
最小合力模长为
\begin{equation}
\sqrt{F_1^2 + F_1'^2} = \frac{Mg x_2}{\sqrt{(x_2 - x_1)^2 + (y_2-y_1)^2}}~.
\end{equation}
也就是说当 $y_1 < y_2$ 时, 第一个人为了减小自己的合力, 需要提供一个向左的水平力。
\end{example}

