% 对数函数(复数)
% keys 对数函数|复变函数
% license Xiao
% type Tutor

\begin{issues}
\issueDraft
\end{issues}

\pentry{指数函数(复数)\nref{nod_CExp}, 函数(高中)\nref{nod_functi}}{nod_fd52}

\footnote{参考 Wikipedia \href{https://en.wikipedia.org/wiki/Complex_logarithm}{相关页面}。}我们先给出\enref{复数}{CplxNo}对数函数的一种定义
\begin{definition}{对数函数(复数)}
对数函数 $\log z$ 是从任意非零复数到复数的映射 $f:\mathbb C^*\to \mathbb C$, 定义为
\begin{equation}
\log z = \log\abs{z} + \I \arg z~.
\end{equation}
其中 $\arg(z)$ 是 $z$ 的辐角, 值域为 $(-\pi, \pi]$。
\end{definition}
这样定义的函数值叫做 $\log$ 函数的\textbf{主值(principal value)}。 注意 $\log 0$ 在复数域中没有定义,也有时候说 $\log 0 = -\infty$(注意 $\pm\infty \notin \mathbb C$)。

结合复数\enref{指数函数}{CExp}的定义, 恒有
\begin{equation}
\exp(\log z) \equiv z \qquad (z \in \mathbb C^*)~.
\end{equation}
\begin{equation}
\log \E^z \equiv z \qquad (\Re[z]\in\mathbb R\ \text{ and }\ \Im[z]\in(-\pi, \pi])~.
\end{equation}
所以在有限的范围内, $\exp z$ 和 $\log z$ 互为反函数。

现在我们也可以用 $\log z$ 来表示复数的辐角
\begin{equation}
\arg z = \Im\log z~.
\end{equation}
