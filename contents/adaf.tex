% 一维薛定谔方程的格林函数
% keys adaf

\begin{issues}
\issueDraft
\end{issues}

考虑一个自由粒子的系统,其哈密顿量可以写成:$H=\frac{p^2}{2m}$,其中$p$为粒子的动量,$m$为粒子的质量。我们可以通过求解该系统的薛定谔方程来得到其格林函数。

薛定谔方程为:$i\hbar\frac{\partial}{\partial t}\psi(x,t)=H\psi(x,t)$。将哈密顿量代入可得:$i\hbar\frac{\partial}{\partial t}\psi(x,t)=-\frac{\hbar^2}{2m}\frac{\partial^2}{\partial x^2}\psi(x,t)$。

我们考虑在$t=0$时,将粒子从位置$x_0$处的波包释放,粒子的初速度为$v_0$。那么波函数可以表示为:$\psi(x,0)=\frac{1}{\sqrt{2\pi}}e^{ikx}$,其中$k=\frac{mv_0}{\hbar}$。为了求解格林函数,我们需要将该波函数演化到任意时刻$t$下的波函数。

我们可以使用格林函数来描述波函数的演化过程。对于自由粒子系统,其格林函数可以表示为:$G(x,t;x_0,0)=\frac{1}{2\pi}\int_{-\infty}^{+\infty}dk e^{i(kx-\frac{\hbar k^2}{2m}t)}e^{-ikx_0}$。该式子的物理意义是:在$t=0$时,在$x_0$处施加一个$\delta$函数源,粒子的位置分布可以表示为$\psi(x,0)$。那么在$t$时刻,粒子的位置分布可以表示为$\int dx_0 G(x,t;x_0,0)\delta(x_0)$。

我们可以对$G(x,t;x_0,0)$进行求解,得到其形式为:$G(x,t;x_0,0)=\frac{1}{\sqrt{2\pi i\hbar t}}e^{i\frac{m(x-x_0)^2}{2\hbar t}}$。于是,我们可以得到任意时刻$t$下粒子在位置$x$处的波函数为:$\psi(x,t)=\int dx_0 G(x,t;x_0,0)\psi(x_0,0)$。代入初态波函数,得到:$\psi(x,t)=\frac{1}{\sqrt{2\pi i\hbar t}}\int_{-\infty}^{+\infty}dx_0 e^{i\frac{m}{2\hbar t}[(x-x_0)-\frac{v_0t}{m}]^2}e^{ikx_0}$。
