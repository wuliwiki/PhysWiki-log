% SLISC 的 band 矩阵类
% keys 带对角矩阵|对角线|数据结构|SLISC|C++
% license Xiao
% type Tutor

\begin{issues}
\issueDraft
\end{issues}

\pentry{数据结构:带对角矩阵\nref{nod_BanDmt}, SLISC 的密矩阵类\nref{nod_SliMat}}{nod_8549}

SLISC 把 \enref{BLAS}{BLAS} 中的带对角矩阵封装成了 \verb`Cband` 类(列主序。 行主序暂时没有实现)。 其中矩阵元储存于 \verb`m_a` 中, \verb`m_N0` 是行数, \verb`m_a.n1()` 是列数。 \verb`m_Nup, m_Nlow` 分别是对角线上方和下方非零副对角线的条数, \verb`m_a` 的第 \verb`m_idiag` 行则是对角线所在的行。
\begin{lstlisting}[language=cpp]
class CbandDoub
{
public:
    Long m_N0;
    Long m_Nup;
    Long m_Nlow;
    Long m_idiag;
    CmatDoub m_a;

    CbandDoub();
    CbandDoub(Long_I N0, Long_I N1, Long_I Nup, Long_I Nlow,
              Long_I lda = -1, Long_I idiag = -1);
    Doub * p();
    const Doub * p() const;
    Doub operator()(Long_I i, Long_I j) const;
    Doub &ref(Long_I i, Long_I j);
    Long n0() const;
    Long n1() const;
    Long size() const;
    Long nup() const;
    Long nlow() const;
    Long lda() const;
    Long idiag() const;
    CmatDoub &cmat();
    const CmatDoub &cmat() const;
    DcmatDoub band();
    DcmatDoub_c band() const;
    DvecDoub diag();
    DvecDoub_c diag() const;
    void resize(Long_I lda, Long_I N1);
    void resize(Long_I N0, Long_I N1, Long_I Nup,
                Long_I Nlow, Long_I lda = -1, Long_I idiag = -1);
    void reshape(Long_I N0, Long_I Nup, Long_I Nlow, Long_I idiag = -1);
    void shift(Long_I idiag);
};
\end{lstlisting}
