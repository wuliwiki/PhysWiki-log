% 工程热力学
% license CCBYSA3
% type Wiki

(本文根据 CC-BY-SA 协议转载自原搜狗科学百科对英文维基百科的翻译)

工程热力学是\textbf{热力学}的一个分支。热力学是研究能量及其相互转换规律的科学,即是一门关于\textbf{能量}的普遍学说。 能量是对运动的普遍概括和高度抽象,因此,热力学理论也具有高度的抽象性和概括性。作为面向工程应用的工程热力学,它所涉及的是工程上应用最为广泛的两种能量:\textbf{机械能}和\textbf{热能}。因此,工程热力学是研究热能与机械能及其相互转换规律的一门工程科学。工程热力学的最终目的,就是提高机械能和热能之间相互转换的效率, 以消耗最少的热能,获得最多的机械能,或者以花费最少的机械能,获得最多的热能。

\subsection{工程热力学发展史}
人类对热能的使用由来已久,火是人类最早使用的工具之一。但是,由于人类并没有对热能形成有效系统的科学研究方法,现代的热力学理论发展,只是近300年的事情。

1679年, 巴本 (Denis Papin, 1647-1712)研制出一种完全密封的炊锅:蒸煮器。水在里面煮沸后产生的蒸气压使沸点升高,高温使食物极易煮烂。1690年,在蒸煮器的基础上,巴本制成了第一台带活塞的蒸汽机。这是一个单缸活塞式蒸汽机,汽缸底部放有少量的水,用汽缸被加热时所产生的蒸汽推动活塞至顶端,再将热源撤除,里面的蒸汽必定冷凝形成真空,于是汽缸在大气压力的作用下下落,这个过程可以提供动力。

1698年,英国工程师萨佛利(Thomas Savery, 1650-1715)发明了蒸汽泵。与巴本的蒸汽机不同,它没有活塞,因为他的直接目的是抽水。1705年,英国工程师纽可门(Thomas Newcomen, 1663-1729)制造出纽可门蒸汽机。纽可门蒸汽机的创造性在于:为了提高冷凝速率,他在汽缸里面装了一个冷水喷射器,大大的提高了热效率。

1769年,瓦特 (James Watt, 1736-1819)造出了第一台瓦特蒸汽机的样机,并在其中引入了冷凝器。其实早在1765年,瓦特就产生了一个设想:蒸汽机内部应该有冷,热两个容器,让汽缸始终是热的,负责做功,让另一个容器始终是冷的,负责使蒸汽冷凝,产生真空,也就是所谓的冷凝器。在瓦特蒸汽机中,冷凝器与汽缸之间用一个可调节阀门相连,高温蒸汽注入汽缸是阀门关上,做功后打开阀门,正气就会立刻被引入冷凝器冷却,之后在冷凝器和汽缸内均形成真空。活塞在大气压力作用下做功,之后关上阀门,重新将冷凝器抽成真空,重复前一过程。1781年,瓦特改变了蒸汽机只能直线做功的状态,用一个齿轮装置将活塞的直线往复式运动转换为轮轴的旋转运动。1782年,他又进一步设计出了双向汽缸,蒸汽轮流从活塞的两端进入,使热效率又增加了一倍。蒸汽机引发的第一次工业革命为工程热力学的诞生奠定了物质基础和需求基础。

作为19世纪自然科学领域的三大发现之一,能量转换与守恒定律(另外两大发现是细胞学说及进化论)证明了:自然界的一切物质都具有能量。能量不可以被创造,也不可能被消灭,而只能在一定条件下从一种形态转换为另一种形态。在转换过程中,能量的总和保持不变。热力学第一定律就是能量转换与守恒定律在热力学上的应用。

1850年,德国物理学家克劳修斯 (Rudolph Clausius, 1822-1888)指出:一个自动运作的机器,不可能把热从低温物体移到高温物体而不发生任何变化。这就是热力学第二定律。与此同时,克劳修斯还提出了熵的概念。

1905年,能斯特(H.W. Nernst, 1864-1941)提出当温度趋近于绝对零度的时候, 凝聚系的熵趋于相等 (能斯特热定理)。普朗克 (M.Planck,1858-1947)在能斯特的基础上,利用统计理论指出:各种凝聚系,在绝对零度时,熵为零。这就是热力学第三定律。热力学三大定律的发现与验证,是工程热力学的绝对基础。

\subsection{热能动力及应用}
热能在工程工业方面的应用极为广泛,比如蒸汽动力装置,内燃机,燃气轮机,航空发动机,斯特林发动机,制冷机和热泵,能源利用与环境保护等。

\subsection{参考资料}
动力机械及工程热物理:工程热力学. 西北工业大学出版社. 2006. p. 1.