% 拉普拉斯算符
% 梯度|散度|拉普拉斯方程

\pentry{梯度、梯度定理\upref{Grad}, 散度、散度定理\upref{Divgnc}}

\footnote{参考 Wikipedia \href{https://en.wikipedia.org/wiki/Laplace_operator}{相关页面}.}我们令一个标量函数 $u(x, y, z)$ 的\textbf{梯度的散度}为它的\textbf{拉普拉斯(Laplacian)}, 合成的算符(类比复合函数)叫做\textbf{拉普拉斯算符}, 记为 $\laplacian$ 或 $\Delta$.
\begin{equation}
\Delta u = \laplacian u = \div (\grad u) = \pdv[2]{u}{x} + \pdv[2]{u}{y} + \pdv[2]{u}{z}
\end{equation}
也可以记
\begin{equation}
\begin{aligned}
\laplacian &= \Nabla \vdot \Nabla = \qty(\uvec x\pdv{x} + \uvec y\pdv{y} + \uvec z\pdv{z})^2\\
&= \pdv[2]{u}{x} + \pdv[2]{u}{y} + \pdv[2]{u}{z}
\end{aligned}
\end{equation}
这些定义也容易拓展到 $N = 1, 2, \dots$ 元函数上.

柱坐标系中的拉普拉斯算符见\autoref{CylNab_eq4}~\upref{CylNab}, 球坐标中的拉普拉斯算符见\autoref{SphNab_eq4}~\upref{SphNab}.

\subsection{在物理中的应用}
在二维波动方程\upref{Wv2D}中, 薄膜上一个小面源的受力正比于薄膜形状 $u(x, y)$ 在该处的拉普拉斯. 结合牛顿第二定律, 就得到了二维波动方程
\begin{equation}
\laplacian u - \frac{\sigma}{\lambda}\pdv[2]{u}{t} = 0
\end{equation}

在静电学中, 电势 $V$ 的梯度是电场的负, 而电场的散度是电荷密度 $\rho$ 除以电介质常数 $\epsilon_0$, 这样就得到了静电场的泊松方程\upref{EPoiEQ}
\begin{equation}
\laplacian V = -\frac{\rho}{\epsilon_0}
\end{equation}
由于万有引力也服从平方反比定律, 所以同理引力势(引力势能除以质量) $V_G$ 和质量密度 $\rho_m$ 也有类似的关系
\begin{equation}
\laplacian V_G = 4\pi G \rho_m
\end{equation}

在热力学中, 对一块均匀的各向同性的介质, 温度和能量密度 $u$ 成正比, 温度的梯度和能流密度成正比, 而能流密度的散度和能量密度的变化率成正比, 于是有热传导方程\upref{heatc}
\begin{equation}
\laplacian u = \kappa \pdv{u}{t}
\end{equation}
