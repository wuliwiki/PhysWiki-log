% 厦门大学 2006 年 考研 量子力学
% license Usr
% type Note

\textbf{声明}:“该内容来源于网络公开资料,不保证真实性,如有侵权请联系管理员”

\subsection{(20 分)简述(每小题5分)}
(1)什么是玻色(Bose)子和费米(Fermi)子?简要介绍玻色子和费米子的主要特性;

(2)正常塞受(Zeeman)效应及其解释:

(3)解释能级简并的概念并指出其起因:

(4)什么是跃迁选择定则?简单解释其起因.
\subsection{(10 分)}
对低速运动的一维自由粒子,指出下列推导过程中的错误所在:

由 $E = h \nu, \\ \nu = \frac{v}{\lambda}, \\ p = \frac{h}{\lambda}$ 和 $ p = mv$,

得$E = h \frac{v}{\lambda} = \frac{h}{\lambda} v = p v = mv^2 = 2 \cdot \frac{1}{2} mv^2 = 2E$.
\subsection{(20分)}
假设质量为 $m$ 的粒子在一维无限深势阱
\[V(x) = \begin{cases}    0, & 0<x<a \\\\   \infty, & x<0,x> a.    \end{cases}\text{中运动,}~\]
(1)求粒子的能量本征值及相应的本征函数:

(2)若粒子在势阱中的状态由波函数 $\Psi(x)=Ax(a-x)$描写,$A$为归一化系数,求粒子能量的平均值.
\subsection{(20分)}
粒子作一维运动时,常将 $p_x$ 简写为 $p$。设 $F(x, p)$ 是 $x, p$ 的整函数,即 
$$F(x, p) = \sum_{m,n=0}^{\infty} C_{nm} x^m p^n~$$
证明:

1. $[x, p^n] = i \hbar n p^{n-1}$;

2. $[x, F] = i \hbar \frac{\partial F}{\partial p}$。
\subsection{(20 分)}
设备系处于 $\psi = C_1 Y_{11} + C_2 Y_{20}$ 状态($Y_{11}, Y_{20}$ 为球谐函数),
且 $\psi$ 已归一化,即 $|C_1|^2 + |C_2|^2 = 1$,求:

(1) $L_z$(轨道角动量的 $z$ 分量)的可能测值及平均值;

(2) $L^2$(轨道角动量平方)的可能测值、相应的几率及平均值;

(3) $L_x$(轨道角动量的 $x$ 分量)的可能测值。
\subsection{(20 分)}
氢原子的“圆轨道”(指$l=n-1$的态)的径向波函数
$$R_{n,n-1}(r) = C_n r^{n-1} \exp\left( -\frac{r}{na} \right)~$$
式中 $C_n$ 为归一化常数,$a$ 为玻尔(Bohr)半径,试计算

1. 平均半径 $\langle r \rangle$;

2. 涨落 $\Delta r = \sqrt{\langle r^2 \rangle - \langle r \rangle^2}$。
\subsection{(20 分)}
设有一个定域电子,受到沿 $x$ 方向均匀磁场 $B$ 的作用,哈密顿量(不考虑轨道运动)可表示为
\[H = \frac{eB}{mc} S_x = \hbar \omega \sigma_x~\]
式中 \(\omega = \frac{eB}{2mc}\),\(\sigma\) 为泡利 (Pauli) 矩阵,\(S\) 为电子的自旋。

设 \(t = 0\) 时电子自旋向上(\(S_z = \frac{\hbar}{2}\)),求 \(t > 0\) 时 \(S_z\) 的平均值。
\subsection{(20 分)}
考虑耦合谐振子, $H = H_0 + H'$  ,而

$$H_0 = - \frac{\hbar^2}{2 \mu} \left( \frac{\partial^2}{\partial x_1^2} + \frac{\partial^2}{\partial x_2^2} \right) + \frac{1}{2} \mu \omega^2 \left( x_1^2 + x_2^2 \right)~$$
$$H' = -\lambda x_1 x_2 \quad (\lambda \\ \text{为常数,刻画耦合强度})~$$

(1)求出$H_0$的本征值及能级简并度;

(2)在弱耦合的情况下,以第一激发态为例,用简并微扰论计算 $H'$对能级的影响(一级近似).
[附] 谐振子的能量本征函数$\psi(x)$满足
$$x \psi_n (x) = \frac{1}{\sqrt{2 \alpha}} \left( \sqrt{n} \psi_{n-1}(x) + \sqrt{n+1} \psi_{n}(x) \right) \quad (\text{式中} \\ \alpha = \sqrt{\mu \omega / \hbar} )~$$