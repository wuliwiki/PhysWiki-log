% 卡方分布
% license Xiao
% type Tutor

\begin{issues}
\issueDraft
\end{issues}

\subsection{卡方分布}
\begin{figure}[ht]
\centering
\includegraphics[width=7cm]{./figures/b10b7e60de40503f.pdf}
\caption{卡方概率密度函数(\autoref{eq_Chi2_1}), 来自 Wikipedia} \label{fig_Chi2_1}
\end{figure}

\footnote{参考 Wikipedia \href{https://en.wikipedia.org/wiki/Chi-squared_distribution}{相关页面 1} 和\href{https://en.wikipedia.org/wiki/Pearson's_chi-squared_test}{相关页面 2}。}卡方分布为(可以认为 $x<0$ 时函数值为零)
\begin{equation}\label{eq_Chi2_1}
f_k(x) = \frac{1}{2^{k/2}\Gamma(k/2)}x^{k/2-1}\E^{-x/2} \qquad (x > 0)~
\end{equation}
期望值 $k$, 方差 $2k$。

累计分布函数为(Matlab 的 \verb|gammainc(x/2,k/2)|)
\begin{equation}
F_k(x) = \int_0^{x} f_k(t) \dd{t} = \frac{\gamma(k/2, x/2)}{\Gamma(k/2)}~,
\end{equation}
其中 $\gamma$ 为下不完全 $\Gamma$ 函数\upref{IncGam}。

\subsection{检验是否符合指定一维分布}
若要检验 $N$ 次试验中, 某分布是否符合指定分布。 把随机变量划分成 $k$ 个区间, 落入每个区间的数量是 $f_i$, $\sum f_i = N$。 指定分布在每个区间的概率为 $p_i$, 那么 $f_i/N$ 应该接近于 $p_i$, 可以用以下函数判断有多接近
\begin{equation}
\sum_{i=1}^k C_i \qty(\frac{f_i}{N} - p_i)^2~,
\end{equation}
Pearson's cumulative test statistic 令 $C_i = N/p_i$, 于是
\begin{equation}
\chi^2 = \sum_{i=1}^k \frac{N}{p_i} \qty(\frac{f_i}{N} - p_i)^2 = \sum_{i=1}^k \frac{f_i^2}{Np_i} - N~.
\end{equation}
当 $N$ 足够大, 上式的值近似服从 $\chi^2(k-1)$ 分布。

对于\textbf{显著水平(significance level)} $\alpha$, 当 $\chi_{\alpha}^2(k-1)$ 时就接受假设。 其中 $\chi_{\alpha}^2(k-1)$ 满足
\begin{equation}
1 - F_{k-1}(\chi_{\alpha+}^2(k-1)) = \int_{\chi_{\alpha}^2(k-1)}^\infty f_{k-1}(x) \dd{x} = \alpha~.
\end{equation}
假设被拒绝的显著水平越小, 说明假设越不可能成立。

$\alpha$ 被称为 $p$ 值。

\begin{figure}[ht]
\centering
\includegraphics[width=10cm]{./figures/261cf4889b27dfff.png}
\caption{临时例题(待删除)} \label{fig_Chi2_2}
\end{figure}

\subsection{检验两个变量的独立性}
\begin{equation}
\chi^2 = \sum_{i=1}^{k} \sum_{j=1}^{l} \frac{N}{p_ip_j} \qty(\frac{f_{ij}}{N} - p_ip_j)^2 = \sum_{i=1}^{k} \sum_{j=1}^{l} \frac{f_{ij}^2}{Np_ip_j} - N~.
\end{equation}
当 $N$ 足够大, 上式的值近似服从 $\chi^2[(k-1)(l-1)]$ 分布。
\addTODO{补充详细}
