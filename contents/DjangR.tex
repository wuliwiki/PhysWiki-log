% 用 Python Django REST 搭建 Web API 笔记

\pentry{Python 虚拟环境 venv 笔记\upref{PyVenv}}

\begin{issues}
\issueDraft
\end{issues}

\begin{itemize}
\item Flask 是一个更加轻量级的 web api 库, 一个教程见\href{https://programminghistorian.org/en/lessons/creating-apis-with-python-and-flask}{这里}, 包括什么是 web api。
\item 用 django 做一个简单的 web form 见\href{https://www.geeksforgeeks.org/how-to-create-a-form-using-django-forms/}{这里}(以及它的预备知识\href{https://www.geeksforgeeks.org/django-project-mvt-structure}{Django 的 MVT 结构}, \href{https://www.geeksforgeeks.org/how-to-create-a-basic-project-using-mvt-in-django/}{创建 MVT 基本工程}和\href{https://www.geeksforgeeks.org/how-to-create-an-app-in-django/}{创建 app})。
\item django 基于 \textbf{MVT (Model-View-Template)} 的\textbf{设计模式(design pattern)}。
\item \textbf{Model}: 整个应用的数据结构和数据库。
\item \textbf{View}: 渲染用户界面(HTML/CSS/Javascript)。
\item \textbf{Template}: HTML 模板, 用一些特殊语法指定如何加载动态内容。
\item Django 主要用于后端(虽然提供少量前端支持), 前端可以用 React, Vue 等框架实现。
\item \href{https://www.django-rest-framework.org/}{Django REST framework(DRF)} 是基于 django 的用于搭建 web api 的工具。
\item \href{https://www.django-rest-framework.org/tutorial/quickstart/}{Django REST quick start tutorial}。 下面我们按照这个教程来做。
\item 创建文件夹 \verb|tutorial| 并进入。
\item 【可选】创建虚拟环境\upref{PyVenv} \verb|python3 -m venv env|, 然后 \verb|source env/bin/activate| 开启虚拟环境
\item 在虚拟环境中 \verb|pip install django djangorestframework|, 大概需要安装 9MB 的包。
\item 如果没有用虚拟环境, 可能需要 \verb|export PATH=$PATH:/home/addis/.local/bin| 把 \verb|pip| 的可执行文件目录添加到路径。
\item 当然应该也可以在 conda 中安装虚拟环境并安装。
\item 创建新的 django project(注意后面有一点): \verb|django-admin startproject tutorial .| 名为 tutorial, 该命令会创建同名文件夹
\item 运行 \verb|cd tutorial; django-admin startapp quickstart; cd ..|。 会生成如\autoref{DjangR_fig1} 的 \verb|quickstart| 目录。
\begin{figure}[ht]
\centering
\includegraphics[width=5cm]{./figures/DjangR_1.png}
\caption{quickstart 的目录结构} \label{DjangR_fig1}
\end{figure}
\item \verb|python manage.py migrate|
\item \verb|python manage.py createsuperuser --email admin@example.com --username admin| 交互创建密码
\item 创建文件 \verb|tutorial/quickstart/serializers.py|, 插入代码…(见 \href{https://github.com/MacroUniverse/django-test}{GitHub})
\item 替换文件 \verb|tutorial/quickstart/views.py| 的代码…
\item 替换文件 \verb|tutorial/urls.py| 的代码…
\item 在文件 \verb|vim tutorial/settings.py| 最后插入代码… 并在 \verb|INSTALLED_APPS| 列表的最后插入一行…
\item 现在启动 api \verb|python manage.py runserver|。 将在 localhost 端口运行。
\item 现在就可以用 curl 测试我们的 web api 了: \verb|curl -H 'Accept: application/json; indent=4' -u admin:password123 http://127.0.0.1:8000/admin/ -o ret.html|
\item 或者直接访问 \verb|http://127.0.0.1:8000/users/|
\end{itemize}
