% 点集拓扑学(综述)
% license CCBYSA3
% type Wiki

本文根据 CC-BY-SA 协议转载翻译自维基百科\href{https://en.wikipedia.org/wiki/General_topology}{相关文章}。

\begin{figure}[ht]
\centering
\includegraphics[width=10cm]{./figures/16b10920f52e4ba5.png}
\caption{拓扑学家正弦曲线,这是点集拓扑学中的一个有用例子。它是连通的,但不是路径连通的。} \label{fig_DJTP_1}
\end{figure}
在数学中,一般拓扑(或称点集拓扑)是拓扑学的一个分支,主要研究拓扑学中使用的基本集合论定义和构造。它是大多数其他拓扑学分支的基础,包括微分拓扑、几何拓扑和代数拓扑。

点集拓扑中的基本概念是连续性、紧致性和连通性:
\begin{itemize}
\item 连续函数直观上是将相邻的点映射到相邻的点。
\item 紧致集是指可以被有限多个任意小的集合覆盖的集合。
\item 连通集是指不能被分成两个彼此远离的部分的集合。
\end{itemize}
“附近”、“任意小”和“远离”这些术语都可以通过使用开集的概念来精确定义。如果我们改变“开集”的定义,就会改变连续函数、紧致集和连通集的定义。对于“开集”的每一种定义选择,都称为一种拓扑。具有拓扑的集合称为拓扑空间。

度量空间是拓扑空间中的一个重要类别,在这些空间中,可以为集合中的点对定义一个实数的非负距离,也称为度量。具有度量简化了许多证明,且许多最常见的拓扑空间都是度量空间。
\subsection{历史}
一般拓扑学起源于多个领域,其中最重要的包括:
\begin{itemize}
\item 对实数线子集的详细研究(曾被称为点集的拓扑;这种用法现已过时)
\item 流形概念的引入
\item 在功能分析的早期阶段对度量空间,特别是赋范线性空间的研究。
\end{itemize}
一般拓扑学在大约1940年形成了现在的形式。可以说,它几乎囊括了连续性直觉的所有内容,以一种技术上足够的形式,能够应用于数学的任何领域。
\subsection{集合上的拓扑}
设 $X$ 是一个集合,$\tau$ 是 $X$ 的子集族。如果 $\tau$ 满足以下条件,则称 $\tau$ 是 $X$ 上的拓扑:\(^\text{[1][2]}\)
\begin{enumerate}
\item 空集和 $X$ 本身是 $\tau$ 的元素
\item $\tau$ 的元素的任意并集是 $\tau$ 的元素
\item $\tau$ 的有限多个元素的任意交集是 $\tau$ 的元素
\end{enumerate}
如果 $\tau$ 是 $X$ 上的拓扑,则对偶 $(X, \tau)$ 称为拓扑空间。可以使用符号 $X_\tau$ 来表示 $X$ 上带有特定拓扑 $\tau$ 的集合。

$\tau$ 的成员称为 $X$ 中的开集。如果 $X$ 的一个子集的补集在 $\tau$ 中(即其补集是开集),则称该子集是闭集。$X$ 的一个子集可以是开集、闭集、两者(闭开集)或都不是。空集和 $X$ 本身总是既是闭集又是开集。
\subsubsection{拓扑的基}
一个拓扑空间 $X$ 及其拓扑 $T$ 的基 $B$ 是 $T$ 中开集的一个集合,使得 $T$ 中的每个开集都可以写成 $B$ 中元素的并集。\(^\text{[3][4]}\)我们说基 $B$ 生成了拓扑 $T$。基是有用的,因为拓扑的许多性质可以简化为关于生成该拓扑的基的陈述——并且许多拓扑可以通过定义生成它们的基来最容易地定义。
\subsubsection{子空间与商空间}
每个拓扑空间的子集都可以赋予子空间拓扑,其中开集是大空间的开集与子集的交集。对于任何索引族的拓扑空间,可以赋予积空间积拓扑,该拓扑由在投影映射下,各个因子的开集的逆像生成。例如,在有限积中,积拓扑的基由所有开集的积组成。在无限积中,还需要额外的要求:在一个基本开集中的所有投影中,除了有限个投影之外,其他的投影都是整个空间。

商空间定义如下:如果 $X$ 是一个拓扑空间,$Y$ 是一个集合,并且 $f : X \to Y$ 是一个满射函数,则 $Y$ 上的商拓扑是那些在 $f$ 下具有开逆像的 $Y$ 的子集。换句话说,商拓扑是使得 $f$ 连续的最细拓扑。商拓扑的一个常见例子是当在拓扑空间 $X$ 上定义一个等价关系时。此时,映射 $f$ 是到等价类集合的自然投影。
\subsubsection{拓扑空间的例子}
给定的集合可以具有多种不同的拓扑。如果给一个集合赋予不同的拓扑,它就被视为一个不同的拓扑空间。

\textbf{离散拓扑与平凡拓扑}

任何集合都可以赋予离散拓扑,在该拓扑中,所有子集都是开集。在这种拓扑中,唯一收敛的序列或网是那些最终恒定的序列或网。此外,任何集合也可以赋予平凡拓扑(也叫不可分拓扑),在这种拓扑中,只有空集和整个空间是开集。在这种拓扑中的每个序列和网都收敛到空间的每一点。这个例子表明,在一般的拓扑空间中,序列的极限不一定是唯一的。然而,通常拓扑空间必须是豪斯多夫空间,在这种空间中极限点是唯一的。

\textbf{余有限拓扑与余可数拓扑}

任何集合都可以赋予余有限拓扑,在这种拓扑中,开集是空集和补集是有限集的集合。这是任何无限集合上的最小的 $T_1$ 拓扑。

任何集合也可以赋予余可数拓扑,在这种拓扑中,如果一个集合是空集或者它的补集是可数的,则该集合被定义为开集。当集合是不可数时,这种拓扑在许多情况下作为反例。

\textbf{实数和复数的拓扑}

在 $\mathbf{R}$(实数集)上有许多方法可以定义拓扑。标准的实数拓扑是由开区间生成的。所有开区间的集合构成了该拓扑的基(或基底),这意味着每个开集都是从基中某些集合的并集。特别地,这意味着一个集合是开集,如果在该集合中的每个点周围都存在一个非零半径的开区间。更一般地,欧几里得空间 $\mathbf{R}^n$ 也可以赋予拓扑。在 $\mathbf{R}^n$ 的通常拓扑中,基本开集是开球。类似地,复数集 $\mathbf{C}$ 和 $\mathbf{C}^n$ 也有一个标准拓扑,其中基本开集是开球。

实数线也可以赋予下限拓扑。在这里,基本开集是半开区间 $[a, b)$。这种拓扑在实数集 $\mathbf{R}$ 上比上面定义的欧几里得拓扑要细致;如果一个序列在这种拓扑中收敛到一个点,当且仅当它在欧几里得拓扑中从上方收敛到该点。这个例子表明,一个集合可以有许多不同的拓扑定义。

\textbf{度量拓扑}

每个度量空间都可以赋予度量拓扑,其中基本开集是由度量定义的开球。这是任何赋范向量空间上的标准拓扑。在有限维向量空间中,对于所有范数,这个拓扑是相同的。

\textbf{更多例子}

\begin{itemize}
\item 在任何给定的有限集合上都存在众多拓扑。这类空间被称为有限拓扑空间。有限空间有时用于提供关于拓扑空间的一般猜想的例子或反例。
\item 每个流形都有一个自然的拓扑,因为它是局部欧几里得的。类似地,每个单纯形和每个单纯复形从$\mathbf{R}^n$中继承一个自然的拓扑。
\item 扎里斯基拓扑在一个环或代数簇的谱上按代数方式定义。在$\mathbf{R}^n$或$\mathbf{C}^n$上,扎里斯基拓扑的闭集是多项式方程组的解集。
\item 线性图具有一种自然的拓扑,它概括了图形中顶点和边的许多几何方面。
\item 在泛函分析中,许多线性算子集合被赋予了通过指定某个特定函数序列何时收敛到零函数来定义的拓扑。
\item 任何局部域都有一个原生拓扑,并且可以扩展到该域上的向量空间。
\item 谢尔宾斯基空间是最简单的非离散拓扑空间。它与计算理论和语义学有重要的关系。
\item 如果 $\Gamma$ 是一个序数,则集合 $\Gamma = [0, \Gamma)$ 可以赋予由区间 $(a, b)$,[0, b) )和 $(a, \Gamma)$ 生成的顺序拓扑,其中 $a$ 和 $b$ 是 $\Gamma$ 的元素。
\end{itemize}
\subsection{连续函数}
连续性是通过邻域来表示的:如果且仅如果对于任何 $f(x)$ 的邻域 $V$,存在 $x$ 的一个邻域 $U$,使得 $f(U) \subseteq V$,那么函数 $f$ 在点 $x \in X$ 处是连续的。直观地说,连续性意味着无论 $V$ 多么"小",总会存在一个包含 $x$ 的邻域 $U$,使得 $f$ 在 $U$ 上的像包含在 $V$ 内,并且 $f$ 的像包含 $f(x)$。这与以下条件是等价的:$Y$ 中的开(闭)集的原像在 $X$ 中是开(闭)集。在度量空间中,这一定义等价于在分析中常用的 $\varepsilon$-$\delta$ 定义。

一个极端的例子:如果集合$X$赋予离散拓扑,那么到任何拓扑空间$T$的所有函数
$$
f \colon X \rightarrow T~
$$
都是连续的。另一方面,如果$X$赋予不可离散拓扑,并且空间$T$ 至少是$T_0$空间,那么唯一的连续函数是常函数。反过来,任何值域是不可离散的函数都是连续的。
\subsubsection{替代定义}
存在几种等价的拓扑结构定义,因此也有几种等价的方式来定义连续函数。

\textbf{邻域定义}

基于逆像的定义通常不容易直接使用。以下准则通过邻域来表达连续性:如果函数 $f$ 在某点 $x \in X$ 连续,当且仅当对于 $f(x)$ 的任意邻域 $V$,都存在 $x$ 的邻域 $U$,使得 $f(U) \subseteq V$。直观上,连续性意味着无论 $V$ 变得多么“小”,总有一个包含 $x$ 的邻域 $U$,使得其映射落在 $V$ 内。

如果 $X$ 和 $Y$ 是度量空间,那么考虑以 $x$ 和 $f(x)$ 为中心的开球的邻域系统,而不是所有邻域,这与在度量空间中的 δ-ε 连续性定义是等价的。然而,在一般的拓扑空间中,并没有邻近或距离的概念。

然而需要注意的是,如果目标空间是豪斯多夫空间,仍然有 $f$ 在 $a$ 处连续当且仅当当 $x$ 趋近于 $a$ 时,$f(x)$ 的极限是 $f(a)$ 这一点成立。在孤立点处,每个函数都是连续的。

\textbf{序列和网}

在多个上下文中,空间的拓扑通常通过极限点来方便地指定。在许多情况下,这是通过指定一个点何时是序列的极限来实现的,但对于某些在某种意义上过于大的空间,也指定一个点何时是由一个有向集索引的更一般的点集的极限,这些点集被称为网。\(^\text{[5]}\)一个函数只有在它将序列的极限映射到序列的极限时才是连续的。在前一种情况下,极限的保持也是足够的;在后一种情况下,一个函数可能保持所有序列的极限,但仍然不是连续的,保持网的极限是必要且充分的条件。

详细来说,一个函数 $f: X \rightarrow Y$ 如果每当 $X$ 中的序列 $(x_n)$ 收敛于极限 $x$ 时,序列 $(f(x_n))$ 收敛于 $f(x)$,则称该函数是序列连续的。\(^\text{[6]}\)因此,序列连续的函数“保持序列的极限”。每个连续函数都是序列连续的。如果 $X$ 是一个首数可数空间并且满足可数选择公理,那么反之亦然:任何保持序列极限的函数都是连续的。特别地,如果 $X$ 是度量空间,序列连续性和连续性是等价的。对于非首数可数空间,序列连续性可能严格弱于连续性。(对于这两种性质等价的空间称为序列空间。)这促使我们在一般拓扑空间中考虑网而非序列。连续函数保持网的极限,事实上,这一特性表征了连续函数。

\textbf{闭包算子定义}

与指定拓扑空间的开子集不同,拓扑也可以通过闭包算子(记作 cl)来确定,闭包算子将任意子集 $A \subseteq X$ 映射到其闭包,或者通过内核算子(记作 int)来确定,内核算子将任意子集 $A$ 映射到其内核。在这种情况下,一个函数
$$
f: (X, \mathrm{cl}) \to (X', \mathrm{cl'})~
$$
在拓扑空间之间是连续的,当且仅当对于所有的子集 $A \subseteq X$,有
$$
f(\mathrm{cl}(A)) \subseteq \mathrm{cl'}(f(A)).~
$$
也就是说,给定 $X$ 中任意一个元素 $x$ 属于某个子集 $A$ 的闭包,$f(x)$ 属于 $f(A)$ 的闭包。这等价于以下要求,对于 $X'$ 中所有的子集 $A'$,有
$$
f^{-1}(\mathrm{cl'}(A')) \supseteq \mathrm{cl}(f^{-1}(A')).~
$$
此外,
$$
f: (X, \mathrm{int}) \to (X', \mathrm{int'})~
$$
是连续的,当且仅当对于 $X$ 中的任何子集 $A$,有
$$
f^{-1}(\mathrm{int'}(A)) \subseteq \mathrm{int}(f^{-1}(A)).~
$$
\subsubsection{性质}
如果 $f: X \to Y$ 和 $g: Y \to Z$ 是连续的,那么它们的复合函数 $g \circ f: X \to Z$ 也是连续的。如果 $f: X \to Y$ 是连续的,并且
\begin{itemize}
\item $X$ 是紧致的,那么 $f(X)$ 是紧致的。
\item $X$ 是连通的,那么 $f(X)$ 是连通的。
\item $X$ 是路径连通的,那么 $f(X)$ 是路径连通的。
\item $X$ 是 Lindelöf 的,那么 $f(X)$ 是 Lindelöf 的。
\item $X$ 是可分的,那么 $f(X)$ 是可分的。
\end{itemize}
对于固定集合 $X$,可能的拓扑是部分有序的:如果拓扑 $\tau_1$ 比另一个拓扑 $\tau_2$ 更粗(记作 $\tau_1 \subseteq \tau_2$),那么当且仅当每个在 $\tau_1$ 下是开的子集也是在 $\tau_2$ 下开的子集时,拓扑 $\tau_1$ 就是比 $\tau_2$ 更粗的。然后,恒等映射
$$
\text{id}_X: (X, \tau_2) \to (X, \tau_1)~
$$
当且仅当 $\tau_1 \subseteq \tau_2$ 时是连续的(参见拓扑的比较)。更一般地,连续函数
$$
(X, \tau_X) \to (Y, \tau_Y)~
$$
如果将拓扑 $\tau_Y$ 替换为更粗的拓扑,和/或将拓扑 $\tau_X$ 替换为更细的拓扑,仍然保持连续性。
\subsubsection{同胚映射}
与连续映射的概念对称的是开映射,其特征是开集的像也是开集。事实上,如果一个开映射 $f$ 存在逆函数,那么该逆函数是连续的;如果一个连续映射 $g$ 存在逆函数,那么该逆函数是开映射。对于两个拓扑空间之间的一个双射映射 $f$,其逆映射 $f^{-1}$ 不一定是连续的。一个双射连续映射,如果其逆映射也是连续的,则称该映射为同胚映射。

如果一个连续双射的定义域是紧致空间,而它的值域是 Hausdorff 空间,那么该映射是同胚映射。
\subsubsection{通过连续函数定义拓扑}
给定一个函数
$$
f \colon X \rightarrow S,~
$$
其中 $X$ 是一个拓扑空间,$S$ 是一个集合(没有指定拓扑),则 $S$ 上的最终拓扑定义为:令 $S$ 的开集为那些子集 $A \subseteq S$,使得 $f^{-1}(A)$ 在 $X$ 中是开集。如果 $S$ 已经有一个现有拓扑,那么 $f$ 对应于该拓扑是连续的,当且仅当现有拓扑比 $S$ 上的最终拓扑更粗。由此,最终拓扑可以被描述为使得 $f$ 连续的最细拓扑。如果 $f$ 是满射,则这个拓扑与通过 $f$ 定义的等价关系下的商拓扑在规范上是等同的。

对偶地,对于从集合 $S$ 到拓扑空间 $X$ 的函数 $f$,$S$ 上的初始拓扑的基是由那些形如 $f^{-1}(U)$ 的开集构成,其中 $U$ 在 $X$ 中是开集。如果 $S$ 已经有一个现有拓扑,则 $f$ 对应于该拓扑是连续的,当且仅当现有拓扑比 $S$ 上的初始拓扑更细。由此,初始拓扑可以被描述为使得 $f$ 连续的最粗拓扑。如果 $f$ 是单射,则这个拓扑与 $S$ 在 $X$ 中作为子集所拥有的子空间拓扑在规范上是等同的。

集合 $S$ 上的拓扑由所有连续函数$S \rightarrow X$到所有拓扑空间 $X$ 的类唯一确定。对偶地,可以对映射$X \rightarrow S$应用类似的概念。
\subsection{紧致集}
正式地,拓扑空间 $X$ 被称为紧致的,如果其每一个开覆盖都有一个有限子覆盖。否则,它被称为非紧致。明确地,这意味着对于每一个任意的集合
$$
\{ U_{\alpha} \}_{\alpha \in A}~
$$
的 $X$ 的开子集,满足
$$
X = \bigcup_{\alpha \in A} U_{\alpha},~
$$
存在 $A$ 的一个有限子集 $J$,使得
$$
X = \bigcup_{i \in J} U_i.~
$$
一些数学分支,如代数几何,通常受到法国布尔巴基学派的影响,使用术语“拟紧致”来表示一般概念,并将“紧致”这一术语保留给既是豪斯多夫空间又是拟紧致的拓扑空间。紧致集有时被称为紧致体(compactum),复数形式为紧致体(compacta)。

实数集 $R$ 中每个有限长度的闭区间都是紧致的。更进一步地,在 $R^n$ 中,一个集合当且仅当它是闭的且有界时才是紧致的。(见海涅-博雷尔定理)

每个紧致空间的连续映像都是紧致的。

豪斯多夫空间的紧致子集是闭集。

每个从紧致空间到豪斯多夫空间的连续双射必然是同胚。

每个紧致度量空间中点的序列都有一个收敛子序列。

每个紧致的有限维流形都可以嵌入某个欧几里得空间 $R^n$。
\subsection{连通集}
一个拓扑空间 $X$ 如果是两个不相交的非空开集的并集,则称其为不连通的。否则,$X$ 被称为连通的。一个拓扑空间的子集,如果在其子空间拓扑下是连通的,则称该子集是连通的。一些作者将空集(具有唯一拓扑的空集)排除在连通空间之外,但本文不遵循这种做法。

对于一个拓扑空间 $X$,以下条件是等价的:
\begin{enumerate}
\item $X$ 是连通的。
\item $X$ 不能被分割为两个不相交的非空闭集。
\item $X$ 中唯一既是开集又是闭集(即连通集)的子集是 $X$ 和空集。
\item $X$ 中唯一边界为空的子集是 $X$ 和空集。
\item $X$ 不能表示为两个非空分离集的并集。
\item 从 $X$ 到 $\{0,1\}$(这个空间赋予离散拓扑)的唯一连续函数是常数函数。
\item 实数集 $R$ 中的每个区间都是连通的。
\end{enumerate}
连通空间的连续映像是连通的。
\subsubsection{连通分支}
非空拓扑空间的最大连通子集(按包含关系排序)称为该空间的连通分支。任何拓扑空间$X$的分支构成$X$的一个划分:它们是互不相交的、非空的,并且它们的并集就是整个空间。每个分支都是原空间的闭子集。因此,在它们的数量有限的情况下,每个分支也是开子集。然而,如果它们的数量是无限的,则不一定如此;例如,有理数集的连通分支是单点集,而这些单点集并不是开集。

设$\Gamma_x$为拓扑空间 $X$ 中点 $x$ 的连通分支,且 $\Gamma_x'$ 为包含 $x$ 的所有开闭集的交集(称为 $x$ 的准分支)。那么,$\Gamma_x \subset \Gamma_x'$当且仅当 $X$ 是紧Hausdorff空间或局部连通时,二者相等。
\subsubsection{非连通空间}
一个空间,如果它的所有分支都是单点集,则称其为完全非连通的。与此性质相关,空间 $X$ 被称为完全分离的,如果对于 $X$ 中的任何两个不同元素 $x$ 和 $y$,存在不相交的开邻域 $U$ 和 $V$,使得 $X$ 是 $U$ 和 $V$ 的并集。显然,任何完全分离的空间都是完全非连通的,但反之则不成立。例如,取两个有理数集 $\mathbf{Q}$ 的副本,并将它们在除零外的每个点上识别起来。由此得到的空间,采用商拓扑,是完全非连通的。然而,考虑零的两个副本,可以发现该空间并不是完全分离的。事实上,它甚至不是Hausdorff空间,且完全分离的条件严格强于Hausdorff空间的条件。
\subsubsection{路径连通集}
\begin{figure}[ht]
\centering
\includegraphics[width=8cm]{./figures/8dd00b0ccfcb65a6.png}
\caption{这个 $\mathbf{R}^2$ 的子空间是路径连通的,因为可以在空间中的任何两个点之间画出一条路径。} \label{fig_DJTP_2}
\end{figure}
路径从点 $x$ 到点 $y$,在拓扑空间 $X$ 中,从点 $x$ 到点 $y$ 的路径是一个从单位区间 $[0,1]$ 到 $X$ 的连续函数 $f$,其中 $f(0) = x$ 且 $f(1) = y$。
$X$ 的路径分量是 $X$ 下的一个等价类,等价关系使得如果从 $x$ 到 $y$ 存在路径,则 $x$ 与 $y$ 是等价的。
如果 $X$ 中的点之间存在一条路径将任意两点连接起来,则称 $X$ 是路径连通的(或称为路径可连接的或 0-连通的);即,$X$ 至多有一个路径分量。许多作者会排除空集的情况。

路径连通与连通性,每一个路径连通的空间都是连通的。然而,反过来并不总是成立:并不是所有连通的空间都是路径连通的。例如,扩展的长线 $L^*$ 和拓扑学正弦曲线就是连通但不路径连通的空间。

然而,实数线 $R$ 的子集当且仅当是路径连通时才是连通的;这些子集是 $R$ 的区间。此外,$R^n$ 或 $C^n$ 的开子集当且仅当是路径连通时才是连通的。另外,对于有限拓扑空间,连通性和路径连通性是相同的。
\subsection{空间的积}
给定 $X$ 为
$$
X := \prod_{i \in I} X_i,~
$$
即由索引集$i \in I$定义的拓扑空间$X_i$的笛卡尔积,并且定义了标准投影$p_i : X \to X_i$,则 $X$上的积拓扑定义为使所有投影$p_i$连续的最粗拓扑(即具有最少开放集的拓扑)。积拓扑有时也被称为 Tychonoff 拓扑。

积拓扑中的开放集是由形如
$$
\prod_{i \in I} U_i,~
$$
其中每个 $U_i$ 在 $X_i$ 中是开放集,并且仅有有限个 $U_i \neq X_i$ 的集合的并(有限的或无限的)。特别地,对于有限积(特别是两个拓扑空间的积),$X_i$ 的基元素的积给出了积空间
$$
\prod_{i \in I} X_i~
$$
的基。

$X$ 上的积拓扑是由形如 $p_i^{-1}(U)$ 的集合生成的拓扑,其中 $i \in I$ 且 $U$ 是 $X_i$ 的开放子集。换句话说,集合 $\{p_i^{-1}(U)\}$ 形成了 $X$ 上拓扑的一个子基。如果一个子集是开放的,那么它当且仅当是形如 $p_i^{-1}(U)$ 的有限个集合交集的(可能是无限个)并。$p_i^{-1}(U)$ 有时被称为开放圆柱体,它们的交集是圆柱集。

一般而言,每个 $X_i$ 的拓扑的积形成了所谓的箱子拓扑(box topology)上的基。通常,箱子拓扑比积拓扑更精细,但对于有限积,它们是相同的。

与紧致性相关的是 Tychonoff 定理:紧空间的(任意)积是紧的。
\subsection{分离公理}
这些名称在一些数学文献中有不同的含义,如《分离公理的历史》所解释的那样;例如,“normal”和“T4”的含义有时会互换,类似地,“regular”和“T3”也会有这种情况,等等。许多概念也有多个名称;然而,第一个列出的名称通常最不容易产生歧义。

这些公理中的大多数都有不同的定义,但含义相同;这里给出的定义遵循一个一致的模式,涉及上一部分中定义的各种分离概念。其他可能的定义可以在各个单独的条目中找到。

在以下所有定义中,$X$ 仍然是一个拓扑空间。
\begin{itemize}
\item $X$ 是 $T_0$ 或 Kolmogorov 空间,如果$X$中的任何两个不同的点是拓扑可区分的。(在分离公理中,通常有一种公理版本要求是 $T_0$,另一种则不要求是 $T_0$。)
\item $X$ 是 $T_1$ 或可达空间或 Fréchet 空间,如果 $X$ 中的任何两个不同的点是可分离的。因此,$X$ 是 $T_1$ 当且仅当它既是 $T_0$ 也是 $R_0$。(尽管你可以说 $T_1$ 空间、Fréchet 拓扑,或假设拓扑空间 $X$ 是 Fréchet,但在此语境中避免使用 Fréchet 空间一词,因为在泛函分析中有完全不同的 Fréchet 空间概念。)
\item $X$ 是 Hausdorff 空间,或 $T_2$ 空间或分离空间,如果 $X$ 中的任何两个不同的点可以通过邻域分离。因此,$X$ 是 Hausdorff 空间当且仅当它既是 $T_0$ 也是 $R_1$。一个 Hausdorff 空间必须也是 $T_1$。
\item $X$ 是 $T_{2\frac{1}{2}}$ 空间,或 Urysohn 空间,如果$X$ 中的任何两个不同的点可以通过闭邻域分离。一个 $T_{2\frac{1}{2}}$ 空间必须也是 Hausdorff 空间。
\item $X$ 是正则的,或 $T_3$ 空间,如果它是 $T_0$,并且对于 $X$ 中的任意点 $x$ 和闭集 $F$,只要 $x$ 不属于 $F$,它们可以通过邻域分离。(实际上,在一个正则空间中,任何这样的 $x$ 和 $F$ 也可以通过闭邻域分离。)
\item $X$ 是 Tychonoff 空间,或 $T_{3\frac{1}{2}}$ 空间,完全 $T_3$ 空间,或完全正则空间,如果它是 $T_0$,并且对于 $X$ 中的任意点 $x$ 和闭集 $F$,只要 $x$ 不属于 $F$,它们可以通过连续函数分离。
\item $X$ 是正常的,或 $T_4$ 空间,如果它是 Hausdorff 空间,并且对于 $X$ 中的任何两个不相交的闭集,它们可以通过邻域分离。(实际上,一个空间是正常的当且仅当任何两个不相交的闭集可以通过连续函数分离;这就是 Urysohn 引理。)
\item $X$ 是完全正常的,或 $T_5$ 空间,或完全 $T_4$ 空间,如果它是 $T_1$,并且对于 $X$ 中的任何两个分离的集合,它们可以通过邻域分离。一个完全正常的空间必须也是正常的。
\item $X$ 是完美正常的,或 $T_6$ 空间,或完美 $T_4$ 空间,如果它是 $T_1$,并且对于 $X$ 中的任何两个不相交的闭集,它们可以通过连续函数精确分离。一个完美正常的 Hausdorff 空间必须也是完全正常的 Hausdorff 空间。
\item Tietze 延拓定理:在一个正常空间中,任何定义在闭子空间上的连续实值函数都可以扩展为定义在整个空间上的连续映射。
\end{itemize}
\subsection{计数公理}
计数公理是某些数学对象(通常是某个类别中的对象)的一个属性,要求存在一个具有特定属性的可数集,而没有这个公理,这些集可能不存在。

拓扑空间中的重要计数公理:
\begin{itemize}
\item 序列空间:如果每个收敛到集合中某个点的序列最终都属于该集合,则该集合是开集。
\item 第一计数空间:每个点都有一个可数的邻域基(局部基)。
\item 第二计数空间:拓扑有一个可数的基。
\item 可分空间:存在一个可数的稠密子空间。
\item Lindelöf 空间:每个开覆盖都有一个可数的子覆盖。
\item $\sigma$-紧空间:存在一个可数的由紧空间构成的覆盖。
\end{itemize}
关系:
\begin{itemize}
\item 每个第一计数空间都是序列空间。
\item 每个第二计数空间都是第一计数、可分和Lindelöf空间。
\item 每个$\sigma$-紧空间都是Lindelöf空间。
\item 每个度量空间都是第一计数空间。
\item 对于度量空间,第二计数性、可分性和Lindelöf性质是等价的。
\end{itemize}
\subsection{度量空间}
一个度量空间是一个有序对$(M, d)$,其中$M$ 是一个集合,$d$ 是 $M$ 上的一个度量,即一个函数
$d: M \times M \to \mathbb{R}$
对于任意 $x, y, z \in M$,满足以下条件:
\begin{enumerate}
\item $d(x, y) \geq 0$ (非负性),
\item $d(x, y) = 0$ 当且仅当 $x = y$ (不可分性),
\item $d(x, y) = d(y, x)$ (对称性),
\item $d(x, z) \leq d(x, y) + d(y, z)$ (三角不等式)。
\end{enumerate}
函数 $d$ 也称为距离函数,简称为距离。通常,如果上下文明确,$d$ 可以省略,直接写 $M$ 来表示度量空间。

每个度量空间都是准紧的和豪斯多夫的,因此是正常的。

度量化定理提供了一个拓扑可以由度量生成的必要和充分条件。
\subsection{Baire类别定理}
Baire类别定理表示:如果 $X$ 是一个完备的度量空间或局部紧豪斯多夫空间,那么每一个计数多个无处稠密集的并集的内涵都是空集。

任何Baire空间的开子空间本身也是一个Baire空间。
\subsection{主要研究领域}
\begin{figure}[ht]
\centering
\includegraphics[width=10cm]{./figures/2866143c4e10c817.png}
\caption{三次佩亚诺曲线构造的迭代,其极限是一个填充空间的曲线。佩亚诺曲线在连续体理论中被研究,连续体理论是一般拓扑学的一个分支。} \label{fig_DJTP_3}
\end{figure}
\subsubsection{连续体理论}
连续体(复数形式为continua)是一个非空紧致连通度量空间,或者更少见的是,紧致连通Hausdorff空间。连续体理论是拓扑学的一个分支,致力于研究连续体。这些对象在几乎所有的拓扑学和分析学领域中频繁出现,它们的性质足够强大,以产生许多“几何”特征。
\subsubsection{动力系统}
拓扑动力学关注空间及其子空间在时间上受到连续变化时的行为。许多与物理学和其他数学领域相关的例子包括流体动力学、台球和流形上的流动。分形几何中的分形的拓扑特征、复动力学中的朱利亚集和曼德博集,以及微分方程中的吸引子,通常对于理解这些系统至关重要。
\subsubsection{无点拓扑}
无点拓扑(也称为无点或无点式拓扑)是一种避免提及点的拓扑学方法。‘无点拓扑’这一名称来源于约翰·冯·诺依曼。\(^\text{[9]}\) 无点拓扑的思想与“仅区域拓扑”密切相关,在这种拓扑中,区域(集合)被视为基础对象,而无需明确提及底层的点集。
\subsubsection{维数理论}
维数理论是一般拓扑学的一个分支,研究拓扑空间的维数不变量。
\subsubsection{拓扑代数}
一个拓扑代数 $A$ 在拓扑域 $K$ 上是一个拓扑向量空间,并且配备了一个连续的乘法映射
$$
\cdot : A \times A \longrightarrow A~
$$
$$
(a,b) \longmapsto a \cdot b~
$$
使得它成为 $K$ 上的代数。如果一个单位元的结合拓扑代数是一个拓扑环。

这个术语是由David van Dantzig创造的;它出现在他1931年博士论文的标题中。
\subsubsection{度量化理论}
在拓扑学及相关数学领域,度量空间是一个与度量空间同胚的拓扑空间。即,拓扑空间 $(X, \tau)$ 被称为度量化的,如果存在一个度量
$$
d: X \times X \to [0, \infty)~
$$
使得由 $d$ 诱导的拓扑就是 $\tau$。度量化定理是给出拓扑空间可度量化的充分条件的定理。
\subsubsection{集合论拓扑学}
集合论拓扑学是一个结合了集合论和一般拓扑学的学科。它关注的是独立于 Zermelo–Fraenkel 集合论 (ZFC) 的拓扑问题。一个著名的问题是正常 Moore 空间问题,这是一般拓扑学中的一个问题,曾是广泛研究的主题。最终,正常 Moore 空间问题的答案被证明与 ZFC 无关。
\subsection{另见}
\begin{itemize}
\item 一般拓扑学中的例子列表
\item 一般拓扑学术语表,包含详细定义
\item 一般拓扑学主题列表,包含相关条目
\item 拓扑空间类别
\end{itemize}
\subsection{参考文献}
\begin{enumerate}
\item Munkres, James R. 《拓扑学》。第二卷。Upper Saddle River: Prentice Hall, 2000.
\item Adams, Colin Conrad, 和 Robert David Franzosa. 《拓扑学导论:纯粹与应用》。Pearson Prentice Hall, 2008.
\item Merrifield, Richard E.; Simmons, Howard E. (1989). 《化学中的拓扑方法》。纽约: John Wiley & Sons. 第16页. ISBN 0-471-83817-9. 2012年7月27日检索. 定义:一个拓扑空间(X,T)的子集集合B称为T的基,如果每个开集都可以表示为B中成员的并集.
\item Armstrong, M. A. (1983). 《基础拓扑学》。Springer. 第30页. ISBN 0-387-90839-0. 2013年6月13日检索. 假设我们在集合X上有一个拓扑,并且有一个集合β,使得每个开集都是β中成员的并集。那么,β称为拓扑的基...
\item Moore, E. H.; Smith, H. L. (1922). "限制的一般理论"。《美国数学杂志》。44 (2): 102–121. doi:10.2307/2370388. JSTOR 2370388.
\item Heine, E. (1872). "函数论的元素"。《纯粹与应用数学杂志》。74: 172–188.
\item Maurice Fréchet在其作品《关于功能分析的一些问题》中介绍了度量空间,Rendic. Circ. Mat. Palermo 22 (1906) 1–74.
\item R. Baire. 《实变量函数》。Ann. di Mat., 3:1–123, 1899.
\item Garrett Birkhoff, VON NEUMANN AND LATTICE THEORY, John Von Neumann 1903-1957, J. C. Oxtoley, B. J. Pettis, American Mathematical Soc., 1958, 第50-55页.
\end{enumerate}
\subsection{进一步阅读}
一些关于一般拓扑学的标准书籍包括:
\begin{itemize}
\item Bourbaki, 《拓扑学概论》(Topologie Générale),ISBN 0-387-19374-X.
\item Kelley, John L. (1975) [1955]. 《一般拓扑学》(General Topology)。数学研究生教材系列。第27卷(第二版)。纽约:Springer-Verlag. ISBN 978-0-387-90125-1. OCLC 1365153.
\item Stephen Willard, 《一般拓扑学》(General Topology),ISBN 0-486-43479-6.
\item James Munkres, 《拓扑学》(Topology),ISBN 0-13-181629-2.
\item George F. Simmons, 《拓扑学与现代分析导论》(Introduction to Topology and Modern Analysis),ISBN 1-575-24238-9.
\item Paul L. Shick, 《拓扑学:点集与几何》(Topology: Point-Set and Geometric),ISBN 0-470-09605-5.
\item Ryszard Engelking, 《一般拓扑学》(General Topology),ISBN 3-88538-006-4.
\item Steen, Lynn Arthur; Seebach, J. Arthur Jr. (1995) [1978], 《拓扑学中的反例》(Counterexamples in Topology)(1978年版的Dover再版),柏林,纽约:Springer-Verlag, ISBN 978-0-486-68735-3, MR 0507446.
\item O.Ya. Viro, O.A. Ivanov, V.M. Kharlamov 和 N.Yu. Netsvetaev, 《初等拓扑学:习题集》(Elementary Topology: Textbook in Problems),ISBN 978-0-8218-4506-6.
arXiv学科代码为math.GN。
\end{itemize}
\subsection{外部链接}
有关一般拓扑学的媒体资料,见维基共享资源(Wikimedia Commons)。
