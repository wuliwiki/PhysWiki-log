% 高斯消元法程序
% 线性代数|线性方程组|高斯消元
% 未完成:考虑写一个程序把解表示出来,包括唯一解,无穷多解,无解的情况

\pentry{线性方程组高斯消元法\upref{GAUSS}}

这里给出一个简单的程序演示高斯消元法的基本步骤. 注意在实际应用中, 我们解线性方程组一般使用 Matlab 的反斜杠算符:\lstinline|x=A \ y| ,其中 $\mat A$ 是系数矩阵,$\bvec y$ 是常数列,$\bvec x$ 是未知量.

\Code{GaussEli}

与“高斯消元法\upref{GAUSS}” 中的步骤略有不同的是, 该程序在处理每一行 \x{A(ii,:)}  时都会试图做一个行交换使系数 \x{A(ii,q(ii))} 的绝对值尽可能大. 这是为了减小数值误差: 试想如果 \x{A(ii,q(ii))} 的解析值为 0, 但由于数值误差, 计算出来是一个很小的数(例如 \x{1e-15}), 那么用其消元时就可能需要将第 \x{ii} 行乘以一个很大的系数(例如 \x{1e15})再加到另一行上, 导致程序不稳定.

用\autoref{GAUSS_ex2}\upref{GAUSS} 中的增广矩阵测试程序如下:
\begin{lstlisting}[language=MatlabCom]
>> A = [1 1 -1 1; 2 2 -2 1; 1 1 0 2; 2 2 -1 5];
>> y = [3; 7; 3; 4];
>> [A1, q] =GaussEli([A,y],true); q
     1     1    -1     1     3
     2     2    -2     1     7
     1     1     0     2     3
     2     2    -1     5     4

交换: r_1 <-> r_2
     2     2    -2     1     7
     1     1    -1     1     3
     1     1     0     2     3
     2     2    -1     5     4

消元: -0.5 * r_1 + r_2
    2.0000    2.0000   -2.0000    1.0000    7.0000
         0         0         0    0.5000   -0.5000
    1.0000    1.0000         0    2.0000    3.0000
    2.0000    2.0000   -1.0000    5.0000    4.0000

消元: -0.5 * r_1 + r_3
    2.0000    2.0000   -2.0000    1.0000    7.0000
         0         0         0    0.5000   -0.5000
         0         0    1.0000    1.5000   -0.5000
    2.0000    2.0000   -1.0000    5.0000    4.0000

消元: -1 * r_1 + r_4
    2.0000    2.0000   -2.0000    1.0000    7.0000
         0         0         0    0.5000   -0.5000
         0         0    1.0000    1.5000   -0.5000
         0         0    1.0000    4.0000   -3.0000

交换: r_2 <-> r_3
    2.0000    2.0000   -2.0000    1.0000    7.0000
         0         0    1.0000    1.5000   -0.5000
         0         0         0    0.5000   -0.5000
         0         0    1.0000    4.0000   -3.0000

消元: -1 * r_2 + r_4
    2.0000    2.0000   -2.0000    1.0000    7.0000
         0         0    1.0000    1.5000   -0.5000
         0         0         0    0.5000   -0.5000
         0         0         0    2.5000   -2.5000

交换: r_3 <-> r_4
    2.0000    2.0000   -2.0000    1.0000    7.0000
         0         0    1.0000    1.5000   -0.5000
         0         0         0    2.5000   -2.5000
         0         0         0    0.5000   -0.5000

消元: -0.2 * r_3 + r_4
    2.0000    2.0000   -2.0000    1.0000    7.0000
         0         0    1.0000    1.5000   -0.5000
         0         0         0    2.5000   -2.5000
         0         0         0         0         0
         
q =
     1     3     4     0
\end{lstlisting}
