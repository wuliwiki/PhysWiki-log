% 实二次型
% keys 实二次型|惯性定律|惯性指数
% license Xiao
% type Tutor


\pentry{二次型的规范型\nref{nod_GuaOQu}}{nod_1869}
\subsection{实二次型}
在\enref{二次型的规范型}{GuaOQu}一节里,已经知道,在任意域 $\mathbb F$ 上的二次型都具有规范型(或对角型)\autoref{def_QuaFor_1}~\upref{QuaFor}。一般来说,规范型是二次型最简单的形式。但是,在域为实数域,即 $\mathbb F=\mathbb R$ 时,可以让规范型\autoref{eq_GuaOQu_1}~\upref{GuaOQu}
\begin{equation}
q({x})=\lambda_1 x_1^2+\cdots+\lambda_r x_r^2~
\end{equation}
的所有系数 $\lambda_i$ 均为 $\pm 1$。
\begin{definition}{实二次型}
若二次型 $q$ 所配备的矢量空间 $V$ 定义在实数域 $\mathbb R$ 上,则 $q$ 称为\textbf{实二次型}。
\end{definition}
适当置换基底矢量,可认为前 $s$ 个系数 $\lambda_1,\cdots,\lambda_s$ 是正的,而其余的系数是负的。进行替换
\begin{equation}
\begin{aligned}
&x'_i=\sqrt{\lambda_i}x_i\;&(1\leq i\leq s)~,\\
&x'_i=\sqrt{-\lambda_i}x_i\;&(s+1\leq i\leq r)~,\\
&x'_i=x_i\; &(r+1\leq i\leq n)~.
\end{aligned}
\end{equation}
 即得
 \begin{equation}
 q({x})=\sum_{i=1}^{s}{x'_i}^2-\sum_{i=s+1}^{r} {x'_{i}}^{2}~.
 \end{equation}
 而若对于有理数域 $\mathbb Q$,在 $\sqrt{\lambda_i}$ 为无理数时并不能作这样的简化。
 
\begin{definition}{标准型}
称可以按公式
\begin{equation}\label{eq_RQuaF_1}
q( x)=\sum_{i=1}^{s}{x_i}^2-\sum_{i=s+1}^{r} {x_{i}}^{2}~
\end{equation}
计算值的二次型有\textbf{标准型}。
\end{definition}
由上面的讨论立刻得
\begin{theorem}{}\label{the_RQuaF_1}
实矢量空间 $V$ 上的所有二次型 $q$ 均可化为标准型。
\end{theorem}
\subsection{惯性定理}
\begin{theorem}{惯性定理}
实二次型 $q$ 的标准型\autoref{eq_RQuaF_1} 中的整数 $r$ 和 $s$,$s\leq r\leq n$ 仅依赖于 $q$,即与规范基底的选择无关。
\end{theorem}
\textbf{证明:} 由于 $r$ 不变,故只需证明 $s$ 不变。

设另有一基底 $( e'_1,\cdots, e'_n)$,在其上 $q$ 具有标准型
\begin{equation}\label{eq_RQuaF_2}
q( x)=\sum_{i=1}^{t}{x'_i}^2-\sum_{i=t+1}^{r} {x'_{i}}^{2}~
\end{equation}
不是一般性,令 $t<s$。
对子空间
\begin{equation}
L=\langle  e_1,\cdots, e_s\rangle_\mathbb{R},\quad L'=\langle  e'_{t+1},\cdots, e'_n\rangle_\mathbb{R}~,
\end{equation}
因为 $\mathrm{dim}\;(L+L')\leq s+n-t\leq n=$,那么
\begin{equation}
\begin{aligned}
\mathrm{dim}\;(L\cap L')&=\mathrm{dim}\; L+\mathrm{dim}\; L'-\mathrm{dim}\;(L+L')\\
&\geq s+(n-t)-n=s-t> 0~.
\end{aligned}
\end{equation}
可见,存在一非零矢量 $ a\in(L\cap L')$:
\begin{equation}
 0\neq a=\sum_{i=1}^s a_i e_i=\sum_{i=t+1}^n a'_i  e'_i~.
\end{equation}
由\autoref{eq_RQuaF_1} 
\begin{equation}\label{eq_RQuaF_3}
q( a)=\sum_{i=1}^s a_i^2>0~.
\end{equation}
由\autoref{eq_RQuaF_2} 
\begin{equation}\label{eq_RQuaF_4}
q( a)=-\sum_{i=t+1}^n a_i^2\leq 0~.
\end{equation}
\autoref{eq_RQuaF_3} \autoref{eq_RQuaF_4} 联立得出矛盾。因此,只能是 $s=t$。

\textbf{证毕!}
\begin{definition}{惯性指数}\label{def_RQuaF_1}
称实二次型的秩为\textbf{惯性指数},数 $s$ 为\textbf{正惯性指数}, $r-s$ 为\textbf{负惯性指数}。
\end{definition}

\textbf{注意:}对于复二次型,即 $\mathbb F=\mathbb C$ 的情形,正负惯性指数失去任何意义,因为它的规范型完全可以做成所有的 $\lambda_i$ 都为 $1$ (或 $-1$)。
