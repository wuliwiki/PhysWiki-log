% 随机变量的数字特征(高中)
% 高中|随机变量|数字特征

\begin{issues}
\issueDraft
\end{issues}

\pentry{离散型随机变量(高中)\upref{HsDRV}}
\subsection{数学期望}
一般地,设一个离散型随机变量 $X$ 所有可能的值是 $x_1,x_2,\cdots,x_n$,这些值对应的概率是 $p_1,p_2,\cdots,p_n$,则
\begin{equation}
E(X) = x_1p_1 + x_2p_2 + \cdots + x_np_n
\end{equation}
叫做这个\textbf{离散型随机变量 $X$ 的均值}或\textbf{数学期望}(简称\textbf{期望}).

离散型随机变量的数学期望刻画了这个离散型随机变量的平均取值水平.

由数学期望的定义可以知道,若随机变量 $X$ 服从参数为 $p$ 的二点分布,则
\begin{equation}
E(X) = 1 \cdot p + 0 \cdot(1 - p) = p
\end{equation}

设离散型随机变量 $X$ 服从参数为 $n$ 和 $p$ 的二项分布,由 $X$ 的分布列
\begin{equation}
P(X = k) = C_n^kp^kq^{n-k},k=0,1,2,\cdots ,n
\end{equation}
和数学期望的定义式可得
\begin{equation}
\begin{aligned}
E(X) &= 0\cdot C_n^0p^0q^n+1\cdot C_n^1p^1q^{n-1}+\cdots +k\cdot C_n^kp^kq^{n-k}+\cdots +n\cdot C_n^np^nq^0 \\
&= np(C_{n-1}^0p^0q^{n-1}+C_{n-1}^1p^1q^{n-1}+\cdots +C_{n-1}^{k-1}p^{k-1}q^{(n-1)-(k-1)}+\cdots+C_{n-1}^{n-1}p^{n-1}q^0) \\
&= np(p+q)^{n-1} \\
&= np
\end{aligned}
\end{equation}

若离散型随机变量 $X$ 服从参数为 $N,M,n$的超几何分布,则
\begin{equation}
E(X) = \frac{nM}{N}
\end{equation}
相关证明在教材(人教B版)附录部分,不要求掌握