% 唯一析因环
% license Xiao
% type Tutor

% 唯一分解环|唯一分解整环|唯一析因|唯一分解|因式分解|因子分解|UFD|整环

\pentry{真因子树\nref{nod_FctTre}}{nod_f258}

唯一析因环,顾名思义,就是每个元素都有唯一的不可约因式分解。这个定义用真因子树的语言来描述颇为方便。

\begin{definition}{唯一析因环}
对于整环 $R$,如果它具有\textbf{有限析因性}和\textbf{唯一析因性}\footnote{见\textbf{真因子树}文章的\autoref{def_FctTre_1}~\upref{FctTre}。},那么称其为一个\textbf{唯一析因环(unique factorization domain)},或译作\textbf{唯一分解整环},常简称UFD。
\end{definition}


唯一析因环的好处显而易见,每个元素都可以唯一对应一种素元素分解,比如每个非平凡整数都可以表示为素数的乘积,且这种乘积是唯一的。

\begin{example}{正面例子}
\begin{itemize}
\item 整数环 $\mathbb{Z}$ 是唯一析因环。
\item 域 $\mathbb{F}$ 上的多项式环 $\mathbb{F}[x]$ 是唯一析因环。
\end{itemize}
\end{example}

唯一析因环的概念脱胎于一个经典的错误。柯西等人曾以为自己证明了\textbf{费马大定理},而事实上他们的证明依赖了一个直觉上成立的假设,用现代语言来说就是“所有的环都是唯一析因环”。然而很可惜,存在不唯一析因的环,这也成了此类证明中的命门。

\begin{example}{反面例子}
我们给出一些不是唯一析因环的例子。
\begin{itemize}
\item 环 $\mathbb{Z}[\sqrt{-2}]=\{a+b\sqrt{-5}|a, b\in\mathbb{Z}\}$ 存在有多种因式分解方式的元素。比如说,$6$ 可以分解为 $\qty(1+\sqrt{-5})\times \qty(1-\sqrt{-5})$,也可以分解为 $2\times 3$,这两种分解都已不可再分。
\item 复数域上的全体\textbf{整函数}构成的环不是UFD,因为存在可以无限析因的元素\footnote{本例取自维基百科\href{https://en.wikipedia.org/wiki/Unique_factorization_domain}{相关页面}。}:$\sin{\pi z=\pi z\prod\limits_{i=1}^\infty(1-\frac{z^2}{i^2})}$。
\end{itemize}
\end{example}

\addTODO{关于整函数的概念,缺少“全纯函数”文章作为预备知识。}


是否所有形如$\mathbb{Z}[\sqrt{-n}]$($n$为正整数)的环都不是唯一析因环呢?答案是否定的,比如$\mathbb{Z}[\I]$和$\mathbb{Z}[\sqrt{-2}]$等就是唯一析因环。不过对于这两个环,证明它们是欧几里得环更加方便,由此可知它们是唯一析因环,详情请参见\autoref{ex_EuRing_1}~\upref{EuRing}。



给定一个整环,如何判断它是否是唯一析因环呢?有限析因性通常是容易判断的,如$\mathbb{Z}[\sqrt{-2}]$中,任意元素$a+b\sqrt{-2}$的真因子之模长一定小于$\abs{a+b\sqrt{-2}}$,这就足以判断有限析因性\footnote{当然,我们已经知道$\mathbb{Z}[\sqrt{-2}]$不是唯一析因环。}——但唯一析因性通常并不容易,因此需要介绍判定唯一析因环的条件。




\subsection{判定唯一析因环:素元素}



如果一个整环$R$满足有限析因性,但不满足唯一析因性,会发生什么情况?以$\mathbb{Z}[\sqrt{-2}]$为例,$6=2\times 3=(2+\sqrt{-2})\times(2-\sqrt{-2})$,但是$2\nmid(2\pm\sqrt{-2})$,故不可约元素$2$不是素元素\footnote{注意由\autoref{the_FctTre_2}~\upref{FctTre},素元素必是不可约元素,而这个反例说明不可约元素不一定是素元素。}。

但是,如果不可约元素一定是素元素呢?应用有限析因时素元素的等价定义(\autoref{the_FctTre_1}~\upref{FctTre})和不可约元素的定义(真因子树的末端),容易证明此时$R$中任意给定元素的所有真因子树都具有相同的末端,即唯一析因性。事实上,这就是判定唯一析因环的一种条件:



\begin{definition}{}\label{def_UFD_1}
给定整环$R$,如果$R$中所有不可约元素都是素元素,则称$R$满足\textbf{素性条件}。
\end{definition}


\begin{theorem}{}

给定整环$R$,则“$R$满足\textbf{有限析因性}和\textbf{素性条件}”当且仅当“$R$是\textbf{唯一析因环}”。

\end{theorem}


\textbf{证明}:

\textbf{必要性}:

已知$R$有限析因了,只需要证明此时$R$有唯一析因性。

由素性条件的\autoref{def_UFD_1}  和此时素元素的等价定义(\autoref{the_FctTre_1}~\upref{FctTre}),可知如果$t\in R$在$r\in R$的某棵真因子树的末端,则它必在$r$的每一棵真因子树的末端,因此$r$的所有真因子树的末端元素都相同,从而得证唯一析因性。

\textbf{充分性}:

由于$R$是唯一析因环,故已知$R$有限析因。任取不可约元素$x\in R$,如果$x\mid r$,则$x$在$r$的每一棵真因子树上。由\autoref{the_FctTre_1}~\upref{FctTre}得证$x$是素元素。

\textbf{证毕}。







\subsection{判定唯一析因环:最大公因子}



满足有限析因性的环$R$中,如果没有唯一析因性,还有另外一个疑难:难以判断最大公因子。


\begin{definition}{最大公因子}

给定整环$R$。对于$a, b\in R$,如果$x\in R$是$a$和$b$的公因子,且$a$和$b$的任意公因子都是$x$的因子,那么称$x$是$a, b$的\textbf{最大公因子(greatest common divisor)},记为$\opn{gcd}(a, b)$。

对于任意多的元素,也可以定义它们的最大公因子为,使得所有公因子都是其因子的公因子,同样用符号$\opn{gcd}$表示,如$\opn{gcd}(a_1, a_2, a_3, \cdots)$。

\end{definition}


在$\mathbb{Z}[\sqrt{-2}]$上考虑$a=4+2\sqrt{-2}$和$b=6$,则它们的公因子集合为$\{2, 2+\sqrt{-2}\}$,但这两个公因子都不是彼此的因子,从而$a, b$没有最大公因子。

观察$a, b$的构造可以发现,我们是取$6$的两个不同的不可约分解,然后在两个分解中各取一个对方没有的元素相乘得到$a$,也就是说这种反例的构造依赖于\textbf{没有}唯一析因性。

事实上,这是另一种判定唯一析因环的条件:



\begin{definition}{}
如果整环$R$的任意两个元素之间都有最大公因子,则称$R$满足\textbf{最大公因子条件}。
\end{definition}


\begin{theorem}{}
给定整环$R$。则“$R$满足\textbf{有限析因性}和\textbf{最大公因子条件}”当且仅当“$R$是唯一析因环”。
\end{theorem}


\textbf{证明}:

\textbf{必要性}:

反设$R$不是唯一析因环,则存在$r\in R$,它有两棵末端不相同的真因子树。分别取这两棵树独有的末端(即对方没有的)$x$和$y$,则$r$和$xy$之间没有最大公因子\footnote{$x, y$不可约,故$r$和$xy$的公因子集合就是$\{x, y\}$;又因为$x, y$分别是两棵树独有的元素,故二者不相伴。}。

\textbf{充分性}:

由于$R$是唯一析因环,故任意元素的不可约分解(真因子树的末端)是确定的,不随分解方式变化而变化。对于$a, b\in R$,各自取它们的唯一不可约分解,给$a$的分解结果编号,按顺序,先取第一个不可约因子,看它是否等于$b$的某个不可约因子。如果没有,则看$a$的下一个不可约因子;如果有,则把这两个相同的因子都取出备用,然后看$a$的下一个不可约因子是否等于$b$的某个\textbf{剩下的}不可约因子。以此类推,直到按顺序遍历$a$的全体有限个不可约因子。

遍历后,把过程中取出的$a$的不可约因子乘起来,结果即为$\opn{gcd}(a, b)$。

\textbf{证毕}。








































