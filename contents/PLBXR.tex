% 泡利不相容原理(综述)
% license CCBYSA3
% type Wiki

本文根据 CC-BY-SA 协议转载翻译自维基百科\href{https://en.wikipedia.org/wiki/Pauli_exclusion_principle}{相关文章}。

\begin{figure}[ht]
\centering
\includegraphics[width=6cm]{./figures/5713325387bd5c08.png}
\caption{沃尔夫冈·泡利在1929年哥本哈根的一次讲座中。[1] 沃尔夫冈·泡利提出了泡利不相容原理。} \label{fig_PLBXR_1}
\end{figure}

在量子力学中,泡利不相容原理(德语:Pauli-Ausschlussprinzip)指出,在遵循量子力学定律的系统中,两个或多个具有半整数自旋(即费米子)的相同粒子不能同时占据相同的量子态。奥地利物理学家沃尔夫冈·泡利于1925年首先为电子提出了这一原理,随后在1940年通过自旋-统计定理将其推广至所有费米子。

对于原子中的电子,不可违背原理可以表述如下:在一个多电子原子中,不可能有两个电子的所有四个量子数取值都相同。这四个量子数分别是:主量子数 \( n \),角量子数 \( \ell \),磁量子数 \( m_\ell \),以及自旋量子数 \( m_s \)。例如,如果两个电子处于同一个轨道中,那么它们的 \( n \)、\( \ell \) 和 \( m_\ell \) 值相等。在这种情况下,它们的自旋量子数 \( m_s \) 值必须不同。由于自旋量子数 \( m_s \) 仅能取 \( +1/2 \) 或 \( -\frac{1}{2} \),因此其中一个电子必须具有 \( m_s = +\frac{1}{2} \),另一个必须具有 \( m_s = -\frac{1}{2} \)。