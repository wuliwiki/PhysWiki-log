% 刚体的转动 转动惯量

% 10,min
%刚体的绕轴转动 转动惯量
% 未完成:图片
\pentry{角动量定理\upref{AMLaw}}

设刚体绕光滑轴转动.这里令轴的方向为 $z$,假设轴光滑,则轴对刚体可施加 $x, y$ 两个方向的力矩,却不能施加 $z$ 方向的力矩(或者说任何 z 方向的力矩都被轴的反力矩抵消).所以根据角动量定理, 角动量的 $z$ 分量守恒.

对于单个质点,${L_z} = \left( {\vec r \cross \vec p} \right) \vdot \uvec z$. 首先把质点的位矢在水平方向和竖直方向分解, $\vec r = {\vec r_z} + {\vec r_ \bot }$. 由于 $\vec p$ 一直沿水平方向, 根据叉乘的几何定义, ${\vec r_z} \cross \vec p$ 也是沿水平方向, 只有 ${\vec r_ \bot } \cross \vec p$ 沿 $z$ 方向.另外, 在圆周运动中, 半径始终与速度垂直, 所以 ${\vec r_ \bot }$ 始终与 $\vec p$ 垂直.得出结论
\begin{equation}
{L_z} = \left| {{{\vec r}_ \bot }} \right|\left| {\vec p} \right| = m{r_ \bot }v = mr_ \bot ^2\omega 
\end{equation}
若把刚体分成无数小块, 每小块的质量分别为 $m_i$, 离轴的距离 ${r_{ \bot i}} = \sqrt {x_i^2 + y_i^2} $, 则刚体的角动量 $z$ 分量为
\begin{equation}
{L_z} = \omega \sum\limits_i {{m_i}r_{ \bot i}^2} 
\end{equation}
用积分写成
\begin{equation}
{L_z} = \omega \int {r_ \bot ^2dm} = \omega \int {r_ \bot ^2\rho {\kern 1pt} dV} 
\end{equation}

定义刚体的\textbf{绕轴转动惯量}为
\begin{equation}
{\rm{I}} = \int {r_ \bot ^2dm} 
\end{equation}
则刚体沿轴方向的角动量为
\begin{equation}
{L_z} = I\omega 
\end{equation}
 

