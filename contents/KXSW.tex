% 柯西-施瓦茨不等式(综述)
% license CCBYNCSA3
% type Wiki

本文根据 CC-BY-SA 协议转载翻译自维基百科 \href{https://en.wikipedia.org/wiki/Cauchy\%E2\%80\%93Schwarz_inequality}{相关文章}。

柯西–施瓦茨不等式(也称为柯西–布尼亚科夫斯基–施瓦茨不等式)是对内积空间中两个向量的内积绝对值的一个上界,其上界由这两个向量范数的乘积给出。它被认为是数学中最重要且应用最广泛的不等式之一。

向量的内积可以用于描述有限和(通过有限维向量空间)、无穷级数(通过序列空间中的向量)以及积分(通过希尔伯特空间中的向量)。柯西于1821年首次发表了关于求和形式的不等式。相应的积分形式的不等式由布尼亚科夫斯基于1859年发表,赫尔曼·施瓦茨于1888年发表了积分形式的现代证明。
\subsection{不等式的表述}
柯西–施瓦茨不等式指出,对于内积空间中的任意两个向量$\mathbf{u}$和$\mathbf{v}$,都有:
$$
|\langle \mathbf{u}, \mathbf{v} \rangle|^2 \leq \langle \mathbf{u}, \mathbf{u} \rangle \cdot \langle \mathbf{v}, \mathbf{v} \rangle~
$$
其中,$\langle \cdot , \cdot \rangle$ 表示内积运算。例如,实数或复数的点积就是常见的内积形式。每一个内积都对应一个欧几里得 $\ell_2$ 范数,也叫做“标准范数”或“诱导范数”,记作 $|\mathbf{u}|$,其定义为:
$$
|\mathbf{u}\| := \sqrt{\langle \mathbf{u}, \mathbf{u} \rangle},~
$$
其中 $\langle \mathbf{u}, \mathbf{u} \rangle$ 总是一个非负实数(即使内积是复值的)。对上述不等式两边取平方根,就可以得到柯西–施瓦茨不等式更常见的形式,用范数表示为:
$$
|\langle \mathbf{u}, \mathbf{v} \rangle| \leq |\mathbf{u}| \cdot |\mathbf{v}|~
$$
当且仅当 $\mathbf{u}$ 和 $\mathbf{v}$ 线性相关时,上述不等式取等号。\(^\text{[8][9][10]}\)
\subsection{特殊情形}
\subsubsection{Sedrakyan 引理 —— 正实数情形}
Sedrakyan 不等式,又称为 Bergström 不等式、Engel 形式、Titu 引理(或 T2 引理),表述如下:对于实数 $u_1, u_2, \dots, u_n$ 和正实数 $v_1, v_2, \dots, v_n$,有:
$$
\frac{(u_1 + u_2 + \cdots + u_n)^2}{v_1 + v_2 + \cdots + v_n} \leq \frac{u_1^2}{v_1} + \frac{u_2^2}{v_2} + \cdots + \frac{u_n^2}{v_n},~
$$
或者用求和符号表示为:
$$
\left( \sum_{i=1}^{n} u_i \right)^2 \bigg/ \sum_{i=1}^{n} v_i \leq \sum_{i=1}^{n} \frac{u_i^2}{v_i}.~
$$
这个不等式是柯西–施瓦茨不等式的直接推论,具体地,可以将其看作是在欧几里得空间 $\mathbb{R}^n$ 中对向量点积应用柯西–施瓦茨不等式得到的。

方法是令:
$$
u_i' = \frac{u_i}{\sqrt{v_i}}, \quad v_i' = \sqrt{v_i},~
$$
将其代入向量内积后即可得出上述不等式。这种形式在处理分式型不等式(尤其是分子为完全平方形式)时尤其有用。
\subsubsection{$R^2$ —— 平面}
\begin{figure}[ht]
\centering
\includegraphics[width=8cm]{./figures/099935da4974289a.png}
\caption{欧几里得平面单位圆中的柯西–施瓦茨不等式} \label{fig_KXSW_1}
\end{figure}
实向量空间 $\mathbb{R}^2$ 表示二维平面。它也是二维欧几里得空间,其中的内积就是点积。若$\mathbf{u} = (u_1, u_2),\quad \mathbf{v} = (v_1, v_2)$
则柯西–施瓦茨不等式变为:
$$
\langle \mathbf{u}, \mathbf{v} \rangle^2 = (\|\mathbf{u}\| \|\mathbf{v}\| \cos \theta)^2 \leq \|\mathbf{u}\|^2 \|\mathbf{v}\|^2,~
$$
其中 $\theta$ 是向量 $\mathbf{u}$ 和 $\mathbf{v}$ 之间的夹角。

上述形式也许是最容易理解此不等式的方式,因为余弦的平方最大为 1,当且仅当两个向量方向相同或相反时达到最大值。该不等式也可用向量坐标 $u_1$、$u_2$、$v_1$ 和 $v_2$ 表示为:
$$
(u_1 v_1 + u_2 v_2)^2 \leq (u_1^2 + u_2^2)(v_1^2 + v_2^2),~
$$
其中等号成立的充要条件是向量 $(u_1, u_2)$ 与向量 $(v_1, v_2)$ 共线(同向或反向),或其中一个为零向量。
\subsubsection{$R^n$:n 维欧几里得空间}
在具有标准内积(即点积)的欧几里得空间 $\mathbb{R}^n$ 中,柯西–施瓦茨不等式变为:
$$
\left( \sum_{i=1}^{n} u_i v_i \right)^2 \leq \left( \sum_{i=1}^{n} u_i^2 \right) \left( \sum_{i=1}^{n} v_i^2 \right).~
$$
在此情形下,柯西–施瓦茨不等式可以仅使用初等代数证明,其关键是注意到右侧与左侧之差为:
$$
\frac{1}{2} \sum_{i=1}^{n} \sum_{j=1}^{n} (u_i v_j - u_j v_i)^2 \geq 0,~
$$
或者通过考虑如下关于 $x$ 的二次多项式:
$$
(u_1 x + v_1)^2 + \cdots + (u_n x + v_n)^2 = \left( \sum_i u_i^2 \right) x^2 + 2 \left( \sum_i u_i v_i \right) x + \sum_i v_i^2.~
$$
由于该多项式恒为非负,因此它至多有一个实根,因而其判别式小于等于零,即:
$$
\left( \sum_i u_i v_i \right)^2 - \left( \sum_i u_i^2 \right) \left( \sum_i v_i^2 \right) \leq 0.~
$$
这就得出了柯西–施瓦茨不等式。
\subsubsection{$C^n$:n维复空间}
若 $\mathbf{u}, \mathbf{v} \in \mathbb{C}^n$,其中$\mathbf{u} = (u_1, \ldots, u_n), \quad \mathbf{v} = (v_1, \ldots, v_n)$(其中 $u_1, \ldots, u_n \in \mathbb{C}$,$v_1, \ldots, v_n \in \mathbb{C}$),并且若向量空间 $\mathbb{C}^n$ 上的内积定义为标准复数内积:
$$
\langle \mathbf{u}, \mathbf{v} \rangle := u_1 \overline{v_1} + \cdots + u_n \overline{v_n},~
$$
其中上划线表示复共轭,那么柯西–施瓦茨不等式可以更明确地写为:
$$
\left| \langle \mathbf{u}, \mathbf{v} \rangle \right|^2 = \left| \sum_{k=1}^n u_k \overline{v_k} \right|^2 \leq \langle \mathbf{u}, \mathbf{u} \rangle \cdot \langle \mathbf{v}, \mathbf{v} \rangle = \left( \sum_{k=1}^n u_k \overline{u_k} \right) \left( \sum_{k=1}^n v_k \overline{v_k} \right) = \left( \sum_{j=1}^n |u_j|^2 \right) \left( \sum_{k=1}^n |v_k|^2 \right).~
$$
即:
$$
\left| u_1 \overline{v_1} + \cdots + u_n \overline{v_n} \right|^2 \leq \left( |u_1|^2 + \cdots + |u_n|^2 \right) \left( |v_1|^2 + \cdots + |v_n|^2 \right).~
$$
\subsubsection{$L^2$ 空间}
对于平方可积复值函数所构成的内积空间,有如下不等式成立:
$$
\left| \int_{\mathbb{R}^n} f(x)\, \overline{g(x)}\, dx \right|^2 \leq \left( \int_{\mathbb{R}^n} |f(x)|^2\, dx \right) \left( \int_{\mathbb{R}^n} |g(x)|^2\, dx \right).~
$$
这个不等式是 Hölder 不等式的一个特例。
翻译如下:
\subsection{应用}
\subsubsection{分析学中的应用}
在任意内积空间中,**三角不等式**可以由柯西–施瓦茨不等式推导出来,推导如下:
$$
\|\mathbf{u} + \mathbf{v} \|^2 = \langle \mathbf{u} + \mathbf{v}, \mathbf{u} + \mathbf{v} \rangle
= \|\mathbf{u}\|^2 + \langle \mathbf{u}, \mathbf{v} \rangle + \langle \mathbf{v}, \mathbf{u} \rangle + \|\mathbf{v}\|^2~
$$
由于
$$
\langle \mathbf{v}, \mathbf{u} \rangle = \overline{\langle \mathbf{u}, \mathbf{v} \rangle}~
$$
所以:
$$
\|\mathbf{u} + \mathbf{v} \|^2 = \|\mathbf{u}\|^2 + 2\operatorname{Re} \langle \mathbf{u}, \mathbf{v} \rangle + \|\mathbf{v}\|^2~
$$
应用柯西–施瓦茨不等式(CS)得:
$$
\leq \|\mathbf{u}\|^2 + 2|\langle \mathbf{u}, \mathbf{v} \rangle| + \|\mathbf{v}\|^2 
\leq \|\mathbf{u}\|^2 + 2\|\mathbf{u}\|\|\mathbf{v}\| + \|\mathbf{v}\|^2 
= (\|\mathbf{u}\| + \|\mathbf{v}\|)^2~
$$
对两边取平方根,得到三角不等式:
$$
\|\mathbf{u} + \mathbf{v} \| \leq \|\mathbf{u} \| + \|\mathbf{v} \|~
$$
此外,柯西–施瓦茨不等式还可用于证明:内积是关于其自身诱导的拓扑结构下的连续函数。\(^\text{[11][12]}\)
\subsubsection{几何学}
柯西–施瓦茨不等式使我们能够将“两向量之间的角度”这一概念扩展到任意实内积空间中,其定义为:\(^\text{[13][14]}\)
$$
\cos \theta_{\mathbf{u}\mathbf{v}} = \frac{\langle \mathbf{u}, \mathbf{v} \rangle}{\|\mathbf{u}\| \|\mathbf{v}\|}.~
$$
柯西–施瓦茨不等式证明了该定义是合理的,因为右侧表达式的值始终位于区间 $[ -1, 1 ]$ 之间,这就为我们将(实)Hilbert 空间视为欧几里得空间的推广提供了正当性。
在复内积空间中,也可以使用这个不等式来定义角度,不过需要取右侧表达式的绝对值或实部,\(^\text{[15][16]}\)—— 这种方法在从量子保真度中提取度量时尤其常见。
\subsubsection{概率论}
设 $X$ 和 $Y$ 为随机变量,则协方差不等式为:\(^\text{[17][18]}\)
$$
\operatorname{Var}(X) \geq \frac{\operatorname{Cov}(X, Y)^2}{\operatorname{Var}(Y)}.~
$$
我们可以将随机变量的乘积的期望定义为内积:
$$
\langle X, Y \rangle := \operatorname{E}(XY),~
$$
在这种定义下,柯西–施瓦茨不等式变为:
$$
|\operatorname{E}(XY)|^2 \leq \operatorname{E}(X^2)\operatorname{E}(Y^2).~
$$
为了利用柯西–施瓦茨不等式证明协方差不等式,设:$\mu = \operatorname{E}(X), \quad \nu = \operatorname{E}(Y)$
则有:
$$
\begin{aligned}
|\operatorname{Cov}(X, Y)|^2 
&= |\operatorname{E}[(X - \mu)(Y - \nu)]|^2 \\
&= |\langle X - \mu, Y - \nu \rangle|^2 \\
&\leq \langle X - \mu, X - \mu \rangle \cdot \langle Y - \nu, Y - \nu \rangle \\
&= \operatorname{E}[(X - \mu)^2] \cdot \operatorname{E}[(Y - \nu)^2] \\
&= \operatorname{Var}(X) \cdot \operatorname{Var}(Y),
\end{aligned}~
$$
其中 $\operatorname{Var}$ 表示方差,$\operatorname{Cov}$ 表示协方差。
\subsection{证明}
除了下述方法外,柯西–施瓦茨不等式还有许多不同的证明方式。[19][5][7] 在查阅其他资料时,读者常会遇到两个混淆点:第一,部分作者将内积符号 $\langle \cdot, \cdot \rangle$ 定义为对第二个变量是线性的(而非第一个变量);
第二,有些证明只在数域为实数 $\mathbb{R}$ 时成立,不适用于复数域 $\mathbb{C}$。[20]本节将给出以下定理的两种证明方式:
