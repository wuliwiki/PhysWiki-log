% 简单的偏振电磁波

\begin{issues}
\issueDraft
\end{issues}

\pentry{真空中的平面电磁波\upref{VcPlWv}}
\footnote{本文参考了周磊教授的《电动力学》课程及讲义}

\begin{figure}[ht]
\centering
\includegraphics[width=5cm]{./figures/PLREMW_1.png}
\caption{沿z轴传播的电磁波只有x,y分量} \label{PLREMW_fig1}
\end{figure}

让我们想象一束沿$z$轴传播的电磁波。由于电磁波是横波,所以$E_z=0$。为简明起见,我们假定电场$x,y$两个分量的振幅相同,且$E_x$分量的相位因子为0.\footnote{重要的是分量间的相位差,而不是具体的初相位}此时,电场的波函数就可以写为
\begin{equation}
\bvec E = 
\begin{pmatrix}
E_{0} \cos(kz - \omega t)\\
E_{0} \cos(kz - \omega t+\varphi_{0})\\
0\\
\end{pmatrix}
\end{equation}

根据$\varphi_0$的取值,电磁波也就呈现不同的偏振类型。这让我们联想到利萨茹曲线\upref{Lissaj}。

\subsubsection{$\varphi_0=n\pi, n=0,\pm1,\pm2,...$:线偏振}
\begin{figure}[ht]
\centering
\includegraphics[width=10cm]{./figures/PLREMW_2.png}
\caption{线偏振} \label{PLREMW_fig2}
\end{figure}

\subsubsection{$\varphi_0=\frac{\pi}{2}n, n=\pm1,\pm3,\pm5,...$:圆偏振}
\begin{figure}[ht]
\centering
\includegraphics[width=10cm]{./figures/PLREMW_3.png}
\caption{圆偏振,\href{https://www.geogebra.org/m/hj6qsfdu}{一个可动的模型}(站外链接)} \label{PLREMW_fig3}
\end{figure}

\subsubsection{其余情况:椭圆偏振}
\begin{figure}[ht]
\centering
\includegraphics[width=10cm]{./figures/PLREMW_4.png}
\caption{椭圆偏振} \label{PLREMW_fig4}
\end{figure}
