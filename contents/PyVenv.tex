% Python 虚拟环境 venv 笔记

\begin{issues}
\issueDraft
\end{issues}

\subsection{安装}
\begin{itemize}
\item 安装: \verb|pip3 install virtualenv|
\item 在 ubuntu 上还需要安装 \verb|apt install python?.?-venv|
\end{itemize}

\subsection{使用}
\begin{itemize}
\item \verb|venv| 类似于 \verb|conda| 的虚拟环境。
\item 要在当前路径创建新环境, 用 \verb|python3 -m venv 环境名|, 然后会出现一个名为 \verb|环境名| 的文件夹
\item 运行 \verb|source 环境名/bin/activate| 进入刚创建的虚拟环境。 此时 bash 的行首会标示当前的环境名。
\item 此时 \verb|pip3 install ...| 就可以在当前环境安装一些包。
\item \verb|pip3 list| 会发现当前几乎没有安装的包。 也可以安装新包。
\item \verb|deactivate| 退出虚拟环境。 再用 \verb|pip3 list| 就会看到系统默认的包。
\end{itemize}
