% 强相互作用
% license CCBYSA3
% type Wiki

(本文根据 CC-BY-SA 协议转载自原搜狗科学百科对英文维基百科的翻译)

在粒子物理学中,\textbf{强相互作用}是针对\textbf{强核力}(也称为\textbf{强力},\textbf{核强力},或\textbf{色力})的机制,并且是四个已知的基本相互作用中的一个,其他三个是电磁相互作用,弱相互作用和引力。在10−15m(1 fm)的范围内,强力大约是电磁相互作用力的137倍,是弱相互作用力的100万倍,是引力的1038倍。[1]强力将大多数普通物质聚集在一起,因为它将夸克限制在强子粒子中,例如质子和中子。此外,强大的力将中子和质子结合在一起,形成原子核。一般的质子或中子的大部分质量是强力场能量的结果;单个夸克只提供大约1\%的质子质量。

强相互作用在两个范围内都可以观察到,并由两种力载体介导。在更大的尺度上(大约1到3 fm ),是将质子和中子(核子)结合在一起以形成原子的原子核的力(由介子携带)。在较小的尺度上(小于约0.8 fm,核子的半径尺度),它是将夸克保持在一起形成质子、中子和其他强子粒子的力(由胶子携带)。[2]在后一种情况下,它通常被称为色力。强力本身具有如此高的强度,以至于被强力束缚的强子可以产生新的大质量粒子。因此,如果强子被高能粒子撞击,它们会产生新的强子,而不是发射自由移动的辐射(胶子)。这种强作用力的性质叫做夸克禁闭,它阻止了强力的自由“发射”:实际上,产生大量的粒子喷注。

在原子核中,同样强的相互作用力(将夸克结合在核子内)也将质子和中子结合在一起形成原子核。由于这种能力,它被称为核力(或剩余强力)。因此质子和中子之间强相互作用的残留物也将核子结合在一起。[2]因此,剩余强力与它在束缚核子内的夸克时遵循着完全不同的核子之间的距离依赖行为。不同核子在核力结合能上的差异决定了核聚变和核裂变。核聚变是太阳和其他恒星的主要能量来源。核裂变允许放射性元素及其同位素的衰变,尽管它通常由弱相互作用介导。人工地,与核力相关的能量部分地体现在了核能和核武器上,都是以铀或者钚为基础的裂变武器和聚变武器,如氢弹。[3][4]

强相互作用是由称为胶子的无质量粒子交换介导的,这些粒子在夸克、反夸克和其他胶子之间起作用。胶子被认为通过一种叫做色荷的电荷与夸克和其他胶子相互作用。色荷类似于电磁荷,但它有三种类型(红绿蓝)而不是一种,这导致了不同类型的力,具有不同的行为规则。这些规则在量子色动力学理论(QCD)中有详细描述,QCD是夸克-胶子相互作用的理论。

\subsection{历史}
在20世纪70年代之前,物理学家不确定原子核是如何结合在一起的。众所周知,原子核由质子和中子组成,质子带正电荷,而中子是电中性的。根据当时对物理学的理解,正电荷会相互排斥,带正电荷的质子应该会导致原子核分裂。然而,原子核分裂的现象从未被观察到。这需要新物理学来解释这一现象。

质子之间存在相互的电磁排斥,因此提出了一个有更强的吸引力来解释原子核是如何结合的假设。这种假设的力被称为强力,这被认为是一种基本力,决定质子和中子是怎么构成原子核的。

后来发现质子和中子不是基本粒子,而是由称为夸克的组成粒子组成。核子之间的强大吸引力是将夸克束缚在一起形成质子和中子的更基本的力的副作用。量子色动力学理论解释说夸克带有所谓的色荷,尽管它与可见颜色无关。[5]与色荷不同的夸克相互吸引是由于强相互作用力,介导这一过程的粒子被称为胶子。

\subsection{强力的行为}
\begin{figure}[ht]
\centering
\includegraphics[width=10cm]{./figures/547a2ec03e42fe96.png}
\caption{强相互作用的基本耦合,从左到右:胶子辐射、胶子分裂和胶子自耦合。} \label{fig_QXHZY_1}
\end{figure}
使用强烈 这个词是因为强相互作用是四种基本力中“最强的”。在距离1飞米 ($1  fm = 10^{15}\text{岁}$ 米)或更小,其强力大约是电磁相互作用力的137倍,是弱相互作用力的106倍,是引力的1038倍。

量子色动力学 (QCD)描述了这种强大的力,它是粒子物理标准模型的一部分。数学上,QCD是非阿贝尔群规范理论,基于一个叫做SU(3) 的局部(规范)对称群。

强相互作用的力载体粒子是胶子,一种无质量的玻色子。与电磁学中的中性光子不同,胶子带有色荷。夸克和胶子是唯一携带不消失的色荷的基本粒子,因此它们相互之间只参与强相互作用。强力是胶子与其他夸克和胶子粒子相互作用的表达。

QCD中的所有夸克和胶子都通过强大的力相互作用。相互作用的强度由强耦合常数来参数化。该强度由粒子的规范色荷(一种群论性质)来修改。

强作用力作用在夸克之间。与所有其他力(电磁力、弱力和重力)不同,强力不会随着夸克对之间距离的增加而减弱。达到极限距离(大约是强子的大小)后,它的强度保持在大约10,000 牛顿(N) ,无论夸克之间的距离有多远。[6]随着夸克之间间距的增加,添加到这对夸克中的能量在原来的两对夸克之间产生了新的配对夸克;因此不可能产生单独的夸克。解释是,在10,000牛的力下所做的功足以在很短的相互作用距离内产生粒子-反粒子对。将两个夸克拉开所需要的加到系统中的能量,将产生一对新夸克,它们将与原始夸克配对。在量子色动力学中,这种现象被称为夸克禁闭;因此只能观察到强子,而不能观察到单个的自由夸克。所有寻找自由夸克的实验都失败了,这被认为是这种现象的证据。

高能碰撞中涉及的基本夸克和胶子粒子是不能直接观察到的。这种相互作用产生可以观察到的新生强子的喷注。作为质能守恒的一种表现,当足够的能量沉积到夸克-夸克键中时,这些强子就产生了,就像在粒子加速器实验中,一个质子中的夸克被另一个撞击质子的快速夸克撞击一样。然而,夸克胶子等离子体已经观察到了。[7]

\subsection{剩余强力}
\begin{figure}[ht]
\centering
\includegraphics[width=6cm]{./figures/bbc4b46793528fcc.png}
\caption{在质子和中子之间的核力(或剩余强力)的相互作用的动画。小的彩色双环是胶子,可以看到胶子将质子和中子结合在一起。这些胶子也持有夸克/反夸克的组合,称为介子。因此即使在无色强子之间也能传递剩余的一部分强作用力。反色图如这张图表所示。更大的版本点击这里} \label{fig_QXHZY_2}
\end{figure}
并非宇宙中的每一个夸克都以上述距离独立的方式吸引每一个其他夸克。夸克禁闭意味着,强作用力只在成对夸克之间作用,而在束缚夸克的集合中(强子),夸克的净色荷实质上会抵消,导致力的作用受到限制。因此,夸克(强子)的集合几乎不带色荷,除了抵消不太完美之外,这些强子之间几乎没有强作用力。剩余力(如下所述)称为\textbf{剩余强力}。这种残余力随着距离迅速减小,因此是非常短的距离(实际上是几$fm$)。它表现为“无色”强子之间的力,有时被称为\textbf{强核力}或者简称为核力。

强力的残余效应被称为核力。核力在被称为强子的物质,如介子和重子之间起作用。这种间接作用的“残余强力”传递形成虚数一部分的胶子π和ρ 介子,它反过来在核子之间传递将原子核(氕以上)保持在一起的力。

因此,剩余强力是将夸克结合在一起形成质子和中子的强力的微小剩余。介于中子和质子之间,这种力量要弱得多,因为它们大多被中和了,类似地,中性原子之间的电磁相互作用力(范德华力)比让电子和原子核结合在一起形成原子的电磁相互作用力要弱得多。[8]

不像强力本身,剩余强力,确实在强度上减弱了,事实上它随着距离的增加而迅速减弱。这种减少大约是距离的负指数幂,尽管没有简单的表达式来表示;如汤川势。吸引剩余力随距离的迅速减小和作用在原子核内质子之间的排斥电磁力的不太快的减小,导致了较大原子核的不稳定性,例如所有具有原子序数大于82的原子(元素导致)中。

尽管核力比强相互作用本身弱,但它的能量仍然很高:跃迁产生γ射线。原子核的质量与单个核子的质量显著不同。这种质量缺陷是由于与核力相关的势能。质量的不同主导着核聚变和核裂变的差异。

\subsection{大一统}
所谓的大统一理论(GUT)旨在将强相互作用和弱电相互作用描述为单个力的各个方面,类似于电磁相互作用和电弱相互作用。弱相互作用由格拉肖-温伯格-萨拉姆模型转化为弱电相互作用。强相互作用有一个属性叫做渐近自由性,其中强力的强度在较高的能量(或温度)下减小。理论推导,在它的强度和弱电相互作用相等时,其能量等于大统一能量。然而,大统一理论至今尚未成功地描述这一过程,大统一仍然是一个物理学中未解决的问题。

如果GUT是正确的,那么在大爆炸之后和宇宙的弱电时代期间,弱电与强电分离。因此,据猜测,大一统纪元在此之前已经存在。

\subsection{参考文献}
[1]
^相互作用的相对强度随距离而变化。例如,参见马特·斯特拉斯勒的论文《已知力量的力量》。.

[2]
^四种力量:强相互作用杜克大学天体物理学系网站.

[3]
^关于结合能:参见结合能,质量缺陷,毛绒大象物理教育网站,retr 2012-07-01.

[4]
^关于结合能:参见第四章核过程,强大的力量,M. Ragheb 1/27/2012,伊利诺伊大学.

[5]
^Feynman, R.P. (1985). QED: The Strange Theory of Light and Matter. Princeton University Press. p. 136. ISBN 978-0-691-08388-9. The idiot physicists, unable to come up with any wonderful Greek words anymore, call this type of polarization by the unfortunate name of 'color', which has nothing to do with color in the normal sense..

[6]
^弗里奇,作品引用,p。 164.作者指出,不同颜色夸克之间的力在彼此相距很小的距离后,在任何距离都保持不变,并且等于需要增加一吨,即1000吨 公斤 × 9.8 m/s = ~10,000 N..

[7]
^"Quark-gluon plasma is the most primordial state of matter". About.com Education. Archived from the original on 2017-01-18. Retrieved 2017-01-16..

[8]
^Fritzsch, H. (1983). Quarks: The Stuff of Matter. Basic Books. pp. 167–168. ISBN 978-0-465-06781-7..