% ponderomotive 能量
% 偶极子近似|简谐运动|动能

\pentry{偶极子近似\upref{DipApr}}

\footnote{参考 \cite{Bransden} Chap.15.3}\textbf{Ponderomotive Energy} 定义为在平面电磁波(偶极子近似\upref{DipApr},磁场为零)中做简谐运动的带电粒子一个周期内的平均能量。
\begin{equation}\label{eq_Ponder_1}
U_p = \frac{q^2 \mathcal E_0^2}{4m\omega^2}
\end{equation}
$\mathcal E_0$ 是电场, $\omega$ 是激光频率。

\subsubsection{推导}
令电场为
\begin{equation}
\mathcal E(t) = \mathcal E_0 \sin(\omega t + \phi)~.
\end{equation}
点电荷 $q$ 在电场中的加速度为
\begin{equation}
a = \frac{q\mathcal E_0}{m} \sin(\omega t + \phi)~.
\end{equation}
速度为
\begin{equation}
v = -\frac{q\mathcal E_0}{m\omega} \cos(\omega t + \phi) + v_0~.
\end{equation}
平均动能为
\begin{equation}
\overline{E_k} = \frac{1}{2}m \overline{v^2} = \frac{q^2\mathcal E_0^2}{4m\omega^2} + \frac{1}{2}mv_0^2~.
\end{equation}
所以当电子做简谐振动, 即 $v_0 = 0$ 时的平均动能就是 $U_p$。 如果是简谐振动和平移的叠加, 就要多加上平移的动能。
