% 真空中的平面电磁波
% 麦克斯韦方程组|电磁波|平面波|波动方程

\pentry{电场波动方程\upref{EWEq}}
%感觉可以合并掉?

\footnote{参考 \cite{GriffE} 相关章节与周磊教授的\href{http://fdjpkc.fudan.edu.cn/d200927/2009/0314/c8569a14801/page.htm}{讲义}.}

\begin{figure}[ht]
\centering
\includegraphics[width=13cm]{./figures/VcPlWv_1.pdf}
\caption{平面电磁波的电磁场分布. 注意该例子中,电场矢量与 $x, y$ 坐标无关, 并占据整个空间(图片来自维基百科).\href{https://www.geogebra.org/m/xhYwXSsH}{一个可动的模型}(站外链接)} \label{VcPlWv_fig1}
\end{figure}

\subsection{波函数}
$\bvec E, \bvec B$均为矢量场\upref{Vfield},即电场$\bvec E$或磁场$\bvec B$均是关于空间与时间的矢量函数 $\bvec E = \bvec E(\bvec r, t), \bvec B = \bvec B(\bvec r, t)$.

完全类似于机械波,电磁波的波函数的通解可以写为一组平面简谐波的线性组合.

\subsubsection{三角形式}
平面电磁波为
\begin{align}\label{VcPlWv_eq3}
&\bvec E(\bvec r, t) = \bvec E_0 \cos(\bvec k\vdot \bvec r - \omega t + \varphi_0)\\
&\bvec B(\bvec r, t) = \bvec B_0 \cos(\bvec k \vdot \bvec r - \omega t + \varphi_0)
\end{align}

$\bvec E$的各分量即为
$\bvec E = 
\begin{pmatrix}
E_{x0} \cos(\bvec k\vdot \bvec r - \omega t+ \varphi_{x0})\\
E_{y0} \cos(\bvec k\vdot \bvec r - \omega t+ \varphi_{y0})\\
E_{z0} \cos(\bvec k\vdot \bvec r - \omega t+ \varphi_{z0})\\
\end{pmatrix}
$

请注意,$\bvec E$各分量的初相位可以不同,并且之间没有直接的约束关系.这一点没有很好地反映在\autoref{VcPlWv_eq3} 中.

\subsubsection{复指数形式}
或写为复数形式\upref{PWave}.复数形式的波函数更便于求导、相乘等计算,并在某些情况下是必不可少的数学工具(但也更难懂).最终“实际存在的场”是他的实数部分 $\bvec E = \Re (\tilde {\bvec E})$

\begin{align}
\tilde {\bvec E} = \tilde {\bvec E_0} \E^{\I(\bvec k \cdot \bvec r - \omega t)}\\
\tilde {\bvec B} = \tilde {\bvec B_0} \E^{\I(\bvec k \cdot \bvec r - \omega t)}\\
\end{align}

$\tilde {\bvec E_0}$表明$\tilde {\bvec E_0}$是一个复矢量,即它的各分量是一个复数.$\tilde {\bvec E_0}$包括了初相位信息,写为分量形式即为$
\tilde {\bvec E_0} = 
\begin{pmatrix}
E_{x0} \E^{\I\varphi_{x0}}\\
E_{y0} \E^{\I\varphi_{x0}}\\
E_{z0} \E^{\I\varphi_{x0}}\\
\end{pmatrix}
$.同理,$\tilde {\bvec E}$写为分量形式即为 
$\tilde {\bvec E} = 
\begin{pmatrix}
E_{x0} \E^{\I(\bvec k \cdot \bvec r - \omega t + \varphi_{x0})}\\
E_{y0} \E^{\I(\bvec k \cdot \bvec r - \omega t + \varphi_{y0})}\\
E_{z0} \E^{\I(\bvec k \cdot \bvec r - \omega t + \varphi_{z0})}\\
\end{pmatrix}
$

\subsection{电磁波基本结论}
%需要check一下结论...
\subsubsection{波速}
波速等于真空中的光速 $c$, 且
\begin{equation}\label{VcPlWv_eq1}
c = \frac{1}{\sqrt{\epsilon_0\mu_0}} = 299,792,458 \Si{m/s}
\end{equation}

\subsubsection{波的性质}
\begin{itemize}
\item $\bvec k$被称为波矢,他的方向是电磁波的传播方向.
\item  “色散关系”:$k^2=\frac{\omega^2}{c^2}$.
\item $k=\frac{2\pi}{\lambda}$, $\omega=\frac{2\pi}{T}, c=\frac{\lambda}{T}$
\item 电磁波是横波:即在电磁场传播方向$\bvec k$上没有电场、磁场的分量.$\bvec k \cdot \bvec E = 0$ (如果 $\bvec E_0$ 中存在平行于 $\bvec k$ 的分量, 那么 $\div \bvec E \ne 0$, 违反麦克斯韦方程,所以二者必须垂直)
\end{itemize}

\subsubsection{电场与磁场}
电磁波中电场与磁场二者不是相互独立的,而是互相紧密关联.在一点处的电场与磁场满足:
\begin{equation}
\tilde {\bvec B_0} = \frac{1}{\omega} \bvec k \times \tilde {\bvec E_0} 
\end{equation}
由此,可推导出以下结论:
\begin{itemize}
\item $\bvec E, \bvec B, \bvec k$ 互相垂直,且成右手螺旋关系.见\autoref{VcPlWv_fig1} 右侧
\item $\bvec E \cdot \bvec B = 0,\bvec E \times \bvec B \parallel \bvec k$,$\bvec k \times \bvec E \parallel \bvec B, ...$.
\item $\abs{E_0}=c\abs{B_0}$
\end{itemize}
可见,只要知道了$\bvec E$,就能很快确定同一处的$\bvec B$.

部分推导可见 “电场波动方程\upref{EWEq}”

\subsubsection{能量密度}
\pentry{电场的能量\upref{EEng}, 磁场的能量\upref{BEng}}
任意一点的瞬时能量密度为
\begin{equation}\label{VcPlWv_eq2}
\rho_E = \frac{1}{2}\qty(\epsilon_0 E^2 + \frac{B^2}{\mu_0}) = \epsilon_0 E^2
\end{equation}
虽然磁场“数值上”小于电场强度;但能量上,电场和磁场的能量相同,各贡献总能量一半. 

(一个周期内的)平均能量密度
\begin{equation}
\bar \rho_E = \frac{1}{2} \epsilon_0 E^2
\end{equation}

平均能流密度(光强)为
\begin{equation}
I = c \bar \rho_E = \frac12 c\epsilon_0 E_0^2
\end{equation}
另一个基于坡印廷矢量的推导见\autoref{EBS_ex1}~\upref{EBS}%, 也可以认为瞬时能流密度等于能量密度乘以波速 $c$, 对于简谐波,需要除以二得平均值.


