% 复旦大学 2004 量子真题
% license Usr
% type Note

\textbf{声明}:“该内容来源于网络公开资料,不保证真实性,如有侵权请联系管理员”


(1) 质量为 $m$ 的粒子处在宽度为 $a$ 的一维无限深势阱中,设在时刻 $t=0$ 粒子的状态为 $\Phi(0) = a_1 \varphi_1 + a_2 \varphi_2 + a_3 \varphi_3 + a_4 \varphi_4$,$\varphi_i (i=1,2,3,4)$ 是能量为 $E_i$ 时一维无限深势阱的归一化本征函数,$a_1, a_2, a_3, a_4$ 是已知的常数,求:

   \begin{enumerate}
     \item 在时刻 $t=0$ 时,测量能量,结果小于 $3 \pi^2 \hbar^2 / ma^2$ 的几率
     \item 在时刻 $t=0$ 时,能量 $E$ 和 $E^2$ 的平均值
     \item 时刻为 $t$ 时的波函数 $\Phi(t)$
     \item 如果在 $\Phi$ 态测量能量,所得结果为 $8 \pi^2 \hbar^2 / ma^2$,问测量后粒子处在何种状态?(30分)
    \end{enumerate}

   (2)  设氢原子处在 $R_{21} Y_{1,-1}$ 态,求:

    \begin{enumerate}
     \item  势能 $V = -e^2 / r$ 的平均值
     \item  $\mathbf{L}$ 为轨道角动量,求符号 $\langle \mathbf{L}, \mathbf{L}^2, \mathbf{L}_z \rangle$ 的平均值 $\langle L, L^2, L_z^2 \rangle$
    \end{enumerate}


已知 $R_{21} = \frac{1}{2 \sqrt{6 a_0}} r e^{-r / 2a_0}, Y_{1,-1} = \sqrt{\frac{3}{8 \pi}} \sin \theta e^{-i \varphi}, a_0$ 为波尔半径 (30分)

   (3) 一质量为 $m$ 的粒子在三维势场 $V = \frac{1}{2} k (x^2 + y^2 + z^2 + \lambda xy)$ 中运动,式中 $k$ 是常数,$\lambda$ 为小量

a) 用微扰论求基态能量至二级修正 (30分) 
b) 用简并微扰论求相对于第一激发态的能级至一级修正值 (30分)

(4) 两个自旋为 $1/2$ 的粒子组成的体系由哈密顿量 
$$H = A (S_{1z} + S_{2z}) + B \vec{S_1}~, \cdot \vec{S_2}$$
描述,其中 $\vec{S_1} \vec{S_2}$~, 分别是两个粒子的自旋,$S_{1z}, S_{2z}$ 是它们的 $z$ 分量,$A, B$ 为常数,求该哈密顿量的所有能级  (35分)


(5) 考虑两个具有同样频率 $\omega_0$ 的振子,哈密顿量为
$$ H_1 = h \omega_0 a_1^\dagger a_1, \quad H_2 = h \omega_0 a_2^\dagger a_2~, $$
记 $H_1, H_2$ 相应于本征值 $n_1 h \omega_0$ 和 $n_2 h \omega_0$ 的本征态为 $|n_1, n_2 \rangle$,零点能可略去。在两个振子有相互作用后,其哈密顿量为
$$  H = h \omega_0 a_1^\dagger a_1 + h \omega_0 a_2^\dagger a_2 + g a_1^\dagger a_2 + g a_2^\dagger a_1 = \begin{pmatrix} a_1^\dagger & a_2^\dagger \end{pmatrix} \begin{pmatrix} h \omega_0 & g \\\\ g & h \omega_0 \end{pmatrix} \begin{pmatrix} a_1 \ a_2 \end{pmatrix}~, $$ 
$g$ 为耦合常数。因为有相互作用,故 $|n_1, n_2 \rangle$ 不是 $H$ 的本征态
$g$ 为耦合常数。因为有相互作用,故 $|n_1, n_2 \rangle$ 不是 $H$ 的本征态
\begin{itemize}
    \item [(a)] 求 $H$ 的本征值 (提示:可考虑矩阵 $\begin{pmatrix} h \omega_0 & g  g & h \omega_0 \end{pmatrix}$ 对角化) 
    \item [(b)] 设体系在 $t=0$ 时,处在 $|n_1=1, n_2=0 \rangle$ 态,求 $t>0$ 时体系的态失服 
    \item [(c)] 设在 $t=0$ 时,体系出现在 $|n_1=0, n_2=1 \rangle$ 态的几率 (25分)

\end{itemize}