% 引力红移
% keys 引力红移
% license Xiao
% type Tutor

\pentry{万有引力、引力势能\nref{nod_Gravty},牛顿运动定律、惯性系\nref{nod_New3}}{nod_6246}
\addTODO{增加爱因斯坦等效原理的词条}

爱因斯坦的弱等效原理指出,引力质量 $m_g$ 等于惯性质量 $m_i$,所以根据万有引力定律和牛顿运动定律 $\bvec F=-m_g \grad \Phi = m_i \bvec a$ 可以得到 
\begin{equation}
\bvec a=\bvec g \equiv -\nabla\Phi~.
\end{equation}
其中 $\Phi$ 为引力势能,$\bvec g$ 为重力加速度。根据万有引力公式可以表达为
\begin{equation}
\Phi(\bvec x)=-\sum_{i} \frac{G m^{(i)}}{|\bvec x^{(i)}-\bvec x|}~.
\end{equation}

爱因斯坦等效原理说的另一件事是,在一个足够小的区域内(引力几乎是均匀的),无法通过任何局域的实验探测出引力的存在。例如站在地面上的人受到地面向上的支持力,它等效于一个以加速度 $-\bvec g$ 加速上升的电梯参考系。而在地球上空自由下落的电梯,电梯内部环境可以视为一个惯性系,其中的一切运动学规律可以用狭义相对论给出。

爱因斯坦等效原理也指出了只有通过观察与相邻局域惯性系之问的相对加速度,才能够判断是否有引力存在。

爱因斯坦等效原理的一个重要结果是引力红移现象,表明了光从引力势能低处向高处走的过程中波长会红移,在经过引力区域时会发生偏转。
\subsection{利用能量守恒定律推导引力红移}
设一个高度为 $h$ 的电梯,位于一个重力加速度为 $\bvec g$ 的均匀引力场中。电梯的顶部为 $A$,底部为 $B$,$z_A=h,z_B=0$。

设 $B$ 处一个粒子湮灭为一系列光子,这些光子从 $B$ 点出发,到达 $A$ 点重新合为一个粒子。
根据爱因斯坦质能关系式,粒子的质量 $m$ 乘以 $c^2$ 就等同于光子的能量之和,设 $A$ 处粒子的质量为 $m$,
那么根据能量守恒定律,$B$ 点处粒子产生的光子的能量之和将变为
\begin{equation}
E_B=E_A+mgh = mc^2+mgh~.
\end{equation}
因此光子的频率发生了红移。定义红移因子为光子波长的增量
\begin{equation}
z = \frac{\lambda_A-\lambda_B}{\lambda_B}~.
\end{equation}
那么
\begin{equation}\label{eq_grared_3}
1+z=\frac{\omega_B}{\omega_A} = \frac{\lambda_A}{\lambda_B} = \frac{mc^2+mgh}{mc^2} = 1+\frac{gh}{c^2}~.
\end{equation}
对于不满足 $gh\ll c^2$ 的强引力场,红移因子的计算要考虑到广义相对论的效应,因此结果会比上式更复杂。

\subsection{利用爱因斯坦等效原理推导引力红移}
设有一个高度为 $h$ 的电梯,以 $\bvec a=-\bvec g$ 的加速度从静止开始向上逐渐升高。电梯的速度远小于光速,因此可以忽略相对论效应带来的尺缩效应,忽略 $O(v^2/c^2)$ 级别的高阶小量。下面的所有计算都只保留到 $gh/c^2$ 和 $v/c$ 级别的小量。

电梯的顶部为 $A$,底部为 $B$,它们的高度随时间变化为
\begin{equation}
z_A(t)=\frac{1}{2}gt^2+h,\quad z_B(t)=\frac{1}{2}gt^2~.
\end{equation}
在 $t=0$ 时刻, $B$ 点向上发出一个脉冲,在 $t=t_1$ 时刻传到 $A$ 点,那么
\begin{equation}\label{eq_grared_1}
ct_1 = z_A(t_1)-z_B(0)=\frac{1}{2}g t_1^2+h ~.
\end{equation}
在 $t=\Delta t_B$ 时刻, $B$ 点又一次向上发出脉冲,在 $t=t_1+\Delta t_A$ 时刻传到 $A$ 点,那么
\begin{equation}\label{eq_grared_2}
\begin{aligned}
c (t_1+\Delta t_A - \Delta t_B) &= z_A(t_1+\Delta t_A) - z_B(\Delta t_B)~,
\\
&=\frac{1}{2} g[(t_1+\Delta t_A)^2-\Delta t_B^2] +h
~.
\end{aligned}
\end{equation}
将 \autoref{eq_grared_1} 和 \autoref{eq_grared_2} 相减,得到
\begin{equation}
c(\Delta t_A-\Delta t_B)=\frac{1}{2}g(\Delta t_A^2-\Delta t_B^2)+gt_1 \Delta t_A~.
\end{equation}
在 $\Delta t_A,\Delta t_B$ 为小量时,高阶小量 $\Delta t_A^2,\Delta t_B^2$ 可以忽略,且由  $c t_1\approx h$,可以得到。
\begin{equation}
\frac{\Delta t_A-\Delta t_B}{\Delta t_A} = \frac{g}{c}t_1=\frac{gh}{c^2}~.
\end{equation}
$\Delta t$ 与脉冲的频率成反比,因此
\begin{equation}
\frac{\Delta t_B}{\Delta t_A} = \frac{\omega_A}{\omega_B} = 1-\frac{gh}{c^2}~.
\end{equation}
保留 $gh/c^2$ 的最低阶,两边取倒数,得到
\begin{equation}
\frac{\omega_B}{\omega_A} = 1+\frac{gh}{c^2}~.
\end{equation}
这得到了和 \autoref{eq_grared_3} 相同的结果。要注意的是这里计算了保留到最低阶的计算结果,忽略了 $o(gh/c^2),o(v/c)$ 的高阶小量。因此对于强引力场,即加速度较大的参考系,红移因子的计算会更复杂。