% 重力、重量
% 万有引力|重力

\pentry{万有引力\upref{Gravty}, 圆周运动的向心力\upref{CentrF}}

一般来说, 重力的定义并不明确. 有的地方直接把一个质点在某点的\textbf{万有引力(gravity)}定义为重力. 但另一些情况下把弹簧秤的度数定义为重力(例如在地面参考系). 为了区分, 我们建议不要使用 “重力”, 而是直接将前者称为(万有)引力, 后者称为\textbf{重量(weight)}或者\textbf{视重}. 英语中, “重力” 并没有单独对应的词汇.

什么情况下引力会和重量不同? 答案是当质点(在惯性系中)具有加速度时. 最常见的例子就是地球表面虽然可以近似为惯性系, 但严格来说却不是, 所以相对地面静止的一点在惯性系中具有加速度. 如果只考虑地球的自转产生的加速度\footnote{地球绕太阳系公转的加速度远小于自转, 一般可忽略不计, 有兴趣的读者可以自行计算.}, 那么地表任意一个相对静止的点\footnote{除了两个极点}在(相对地轴静止的)惯性系中都会做圆周运动, 它的加速度等于向心加速度.

\subsubsection{惯性系中的分析}
为了方便, 我们先选取与地轴相对静止的惯性系分析. 当一个相对地表静止的质点挂在弹簧秤上达到平衡时, 它所受的引力 $\bvec G$ 和弹簧对它的拉力 $\bvec T$(或者台秤的支持力) 的合力提供圆周运动的向心力\upref{CentrF}.
\begin{equation}\label{Weight_eq1}
\bvec F_c = \bvec G + \bvec T
\end{equation}
所以不妨定义重量矢量为弹簧拉力的逆矢量
\begin{equation}
\bvec F_w = -\bvec T = \bvec G - \bvec F_c
\end{equation}
那么重量, 也就是弹簧秤的示数大小就是模长 $F_w = \abs{\bvec F_w}$.

可以看出, 当质点无加速度($\bvec F_c = \bvec 0$)时, 重量矢量与引力相等.

(未完成: 计算一下重力加速度)

\subsubsection{地表参考系中的分析}
\pentry{离心力\upref{Centri}}
非惯性系中, 保持平衡的条件同样是合力为零, 但合力要包括惯性力. 在地表参考系, 惯性力就是地球自转的离心力 $\bvec F'_c$.
\begin{equation}
\bvec G + \bvec T + \bvec F'_c = \bvec 0
\end{equation}
由于离心力刚好与向心力相反, 即 $\bvec F'_c = -\bvec F_c$, 代入后仍然可以得到\autoref{Weight_eq1}. 注意离心力并不真的存在, 只是一个数学上的 “把戏”, 或者叫做惯性力.
