% 南京理工大学 普通物理 B(845)模拟五套卷 第四套
% license Usr
% type Note

\textbf{声明}:“该内容来源于网络公开资料,不保证真实性,如有侵权请联系管理员”

\subsection{一、 填空题 I(24 分,每空 2 分)}
1.一质点作圆周运动,设半径为 $R$,运动方程$s=v_0t-\frac{1}{2}bt^2$ ,其中 $s$ 为弧长,$v_0$为初速度,$b$ 为常数。则任一时刻 $t$ 质点的法向加速度为_____,切向加速度为__________。

2. 质量为 $4.25Kg$ 的质点,在合力$F=5i-3j(N)$的作用下由静止从原点运动到$r=5i-3j(m)$时,合力所做的功为_________;此时质点的运动速度大小为_______________。

3. 花样滑冰运动员绕通过自身的竖直轴转动,开始时两臂伸开,转动惯量为 $J$,角速度为$\omega$ ,然后她将两臂收回,使转动惯量减少为 $J/2$,这时她转动的角速度变为____________。

4. 质量为 2kg 的质点,按方程 $x=0.2\sin[5t-(\pi/6)]$沿着 $x$ 轴振动,则 $t=0$ 时,作用于质点的力的大小为__________;作用于质点的力的最大值为________,此时质点的位置________。

5. 设平面简谐波沿 $x$ 轴传播时在 $x=0$ 处发生反射 ,反射波的表达式为$y_2=A\cos [2\pi(vt-x/\lambda)+\pi/2]$,已知反射点为一自由端,则由入射波和反射波形成驻波波节的位置坐标为__________。

6. 如图,真空中一长为$L$ 的均匀带电细直杆,总电量为 $q$,则在直杆延长线上
距杆一端距离为 $d$ 的 $P$ 点的电场强度为___________。
\begin{figure}[ht]
\centering
\includegraphics[width=6cm]{./figures/f20e00a1183b6526.png}
\caption{} \label{fig_NJUD4_1}
\end{figure}
7. 一气缸内储有 $10mol$ 单原子分子理想气体,在压缩过程中,外力做功 $209J$,
气体温度升高 $1K$,则气体内能的增量$\Delta E$ 为________$J$,吸收的热量 $Q $____________$J$。
\subsection{二、 填空题 I(24 分,每空 2 分)}
1. 迈克耳孙干涉仪可用来测量单色光的波长,当 $M_2$ 反射镜移动距离$d=0.3220mm$ 时,测得某单色光的干涉条纹移过 $N=1204$ 条,则该单色光的波长为_______。

2. 一个 50 匝的半径 $R=5.0cm$ 的通电线圈处于 $B=1.5(T)$的均匀磁场中,线圈中电流 $I=0.2A$,则该线圈中的磁矩大小为________,当线圈的磁矩与外磁场方向的夹角从 0 转到 $\pi$时,外磁场对线圈所作的功为___________。

3. 波长为 $680nm$ 的平行光垂直地照射在 $12cm$ 长的两块玻璃片上,两玻璃片一边相互接触,另一边被厚 $0.048mm$ 的纸片隔开,则在这 $12cm$ 内呈现的明条纹数为____________。

4. 一电子在 $B=2\times10^{-3}T$ 的磁场中沿半径为 $R=2\times10^{-2}m$、螺距为 $h=5\times10-2m$ 的螺旋运动,如图所示,则磁场方向为__________,电子速度大小为________。
\begin{figure}[ht]
\centering
\includegraphics[width=6cm]{./figures/eec581e562920fb9.png}
\caption{} \label{fig_NJUD4_2}
\end{figure}
5. 有一半径为 $R$ 的水平圆转台,可绕通过其中心的竖直固定光滑轴转动,转动
惯量 $J$,开始时转台以匀角速度$\omega_0$ 转动,此时有一质量为 $m$ 的人站在转台中心,
随后人沿着半径向外跑去,当人到达离转轴为 $r$ 处,转台的角速度为_________。

6. 在速度 $v=$_________情况下,粒子的动量等于非相对论动量的两倍;在速度
$v=$__________的情况下,粒子的动能等于它的静止能量。

7. 已知某一频率的光子的能量为 $E$,则其动能为_________,质量为________,动量为____________。
\subsection{三、(12 分)}
一质量为 $m$的小球,由顶端沿质量为 $M$ 的圆弧形木槽自静止下滑,设圆弧形槽的半径为 $R$(如图所示),忽略所有摩擦,求:\\
(1) 小球刚离开圆弧形槽时,圆弧形槽的速度;\\
(2) 此过程中,圆弧形槽对小球所做的功。
\subsection{四、(12 分)}
以理想气体为工作物质的热机循环, 如图所示 , 试证明其效率为
\begin{figure}[ht]
\centering
\includegraphics[width=6cm]{./figures/31cdc8bb0da122a4.png}
\caption{} \label{fig_NJUD4_3}
\end{figure}
\subsection{五、(12 分)}
半径分别为 $R_1$ 和 $R_2(R_2>R_1)$的两个同心导体薄球壳,分别带有电荷 $Q_1$和 $Q_2$,今将内球壳用细导线与远处半径为 $r$ 的导体球相连,如图所示,导体球原来不带电,试求相连后导体球所带电荷$q$。
\begin{figure}[ht]
\centering
\includegraphics[width=6cm]{./figures/e005175e43ff66f9.png}
\caption{} \label{fig_NJUD4_4}
\end{figure}
\subsection{六、(12 分)}
有一闭合回路由半径为 $a$ 和 $b$ 的两个同心共面半圆连接而成。如图,其上均匀分布线密度为$\lambda$的电荷,当回路以匀角速度$\omega_0$绕过 $O$ 点垂直于回路平面的轴转动时,求圆心 $O$ 点处的磁感应强度的大小。
\begin{figure}[ht]
\centering
\includegraphics[width=6cm]{./figures/787df127df3551c0.png}
\caption{} \label{fig_NJUD4_5}
\end{figure}
\subsection{七、(14 分)}
如图所示,一长直导线通有电流 $I$,其旁共面地放置一匀质金属梯形线框$abcda$,已知:$da=ab=bc=L$,两斜边与下底边夹角均为 60°,$d$ 点与导线相距 $l$。今线框从静止开始自由下落 $H$ 高度,且保持线框平面与长直导线始终共面,求:\\
(1)下落高度为 $H$ 的瞬间,线框中的感应电流为多少?\\
(2) 该瞬时线框中电势最高处与电势最低处之间的电势差为多少 ?
\begin{figure}[ht]
\centering
\includegraphics[width=6cm]{./figures/8612bc031c7623ae.png}
\caption{} \label{fig_NJUD4_6}
\end{figure}
\subsection{八、(14 分)}
如图所示,$S_1$ 与 $S_2$ 为两相干波源,相距$\frac{1}{4}\lambda$ ,且 $S_1$ 较 $S_2$ 位相超前 $0.5\pi$,如
果两波在 $S_1S_2$ 连线方向上的强度相同均为 $I0$ 且不随距离变化,求:\\
(1) $S_1S_2$ 连线上 $S_1$ 外侧各点处合成波的强度;\\
(2) $S_1S_2$ 连线上 $S_2$ 外侧各点处合成波的强度。
\begin{figure}[ht]
\centering
\includegraphics[width=6cm]{./figures/abf056fc737f1fae.png}
\caption{} \label{fig_NJUD4_7}
\end{figure}
\subsection{九、(13 分)}
用一个每毫米有 500 条缝的衍射光栅观察钠光谱线$(589nm)(1nm=10^{-9}m)$,设平行光以入射角 30°入射到光栅上,问最多能观察到第几级谱线?
\subsection{十、(13 分)}
某加速器将质子加速到 $76GeV(1eV=1.6\times10^{-19}J)$的动能,试求:
(1) 加速后质子的质量;(2)加速后质子的动量。