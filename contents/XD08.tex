% 厦门大学 2008 年硕士入学考试物理试题
% keys 厦门大学|考研|物理|2008年
% license Copy
% type Tutor


\textbf{声明}:“该内容来源于网络公开资料,不保证真实性,如有侵权请联系管理员”


\textbf{科目代码:615}

\begin{enumerate}
\item 一质量为$ m$ 的小球,从高出水面为$h$的$ A $点自由下落:已知小球在空气中所受的阻力可以忽略不计,在水中受到的粘滞阻力与小球的运动速度成正比:$f=kv$($k$为常数),它在水中受到的浮力为$ B$,如果以小球恰好垂直落入水中时为计时起点(t=0),试求小球在水中任一时刻$t$的沉降速度及其终极速度。
\item 一质量为 $m$,长度为$l$的匀质细杆,铅直地放置在光滑的水平桌面上,杆自静止倒下,求杆与竖直方向夹角为$\theta$时:\\
(1)细杆质心的速度;\\
(2)杆的转动角速度;\\
(3)地面对杆的支持力。
\begin{figure}[ht]
\centering
\includegraphics[width=6cm]{./figures/1959e79d3b5641fa.png}
\caption{} \label{fig_XD08_1}
\end{figure}
\item 容器中储存有$2mol $压强为$ p=6.9*10^5Pa$,温度$t=127$℃的氧气,求:\\
(1)单位体积的分子数;\\
(2)分子间的平均距离;\\
(3)一个分子的平均平动动能;\\
(4)系统的内能。
\item 有一除底部外都是绝热的气简,被一位置固定的导热板隔成相等的两部分$A$和 $B$,如图所示,$A,B$ 中分别盛有 $1mol$氮气和氨气。今将$112J$的热量缓慢地由底部传给气体,设活塞上的压强始终保持为 $latm$,求\\
(1)$A$ 部和 $B$ 部气体温度的改变量及系统对外所作的功(设导热板的热容量可忽略不计);\\
(2)将位置固定的导热板换成可自由滑动的导热板,重复上述讨论;\\
(3)将位置固定的导热板换成可自由滑动的绝热板,重复上述讨论。
\begin{figure}[ht]
\centering
\includegraphics[width=6cm]{./figures/2f2198082cc53989.png}
\caption{} \label{fig_XD08_2}
\end{figure}
\item 均匀带电球壳,其体电荷密度为$\rho$,球壳的内、外半径分别为$R_1$和$R_2$,,如图所示。试求:\\
(1)空间中不同区域的电场强度矢量$\bar E(r)$的分布;\\
(2)球心的电势。
\begin{figure}[ht]
\centering
\includegraphics[width=6cm]{./figures/d094283bd00426b3.png}
\caption{} \label{fig_XD08_3}
\end{figure}
\item 在一无限长半径为$R$的圆柱形导体内,有一半径为$r$的圆形空腔,空腔中心轴线与圆柱形导体的中心轴线平行,两轴线之间相距为$a$。假设导体内通有均匀分布的电流,电流密度为$\bar \delta$,如图所示。试求:\\
(1)导体中心轴线口点处的磁感应强度;\\
(2)空腔中任一点$P$处的磁感应强度。
\begin{figure}[ht]
\centering
\includegraphics[width=6cm]{./figures/639053d313bac18a.png}
\caption{} \label{fig_XD08_4}
\end{figure}
\item 如图,一匝数为 $N$,面积为$ S$,电阻为$ R$的矩形线圈在匀强磁场$ B$中以角速度$\omega$作匀速转动,$t=0$时线圈平面与磁场方向垂直。求$t$时刻:\\
(1)线圈中的感应电动势和感应电流;\\
(2)为维持线圈匀速转动,作用在线圈上的外力矩为多少?
\begin{figure}[ht]
\centering
\includegraphics[width=6cm]{./figures/9b299bd4ed800219.png}
\caption{} \label{fig_XD08_5}
\end{figure}
\item 振幅为$A$,频率为$v$,波长为$\lambda$ 的平面简谐波入射到$P$点反射,以后形成驻波。设反射点存在半波损失,在$t=0$ 时刻位于O点的质元处在平衡位置且向$y$轴负方向运动。求\\
(1)驻波的波函数;\\
(2)在离反射点$P$处$\lambda/6$ 的质点的振动表达式。
\begin{figure}[ht]
\centering
\includegraphics[width=6cm]{./figures/65fa4cb7b6d3a747.png}
\caption{} \label{fig_XD08_6}
\end{figure}
\item 波长为 $5000A $和 $5200A$的钠光投射于光栅常数为$ 0.002$ 厘米的衍射光栅上。在光栅后面置一焦距为2米的凸透镜把光会聚在屏上。试求:\\
(1)这钠双线的第一级谱线间的线距离;\\
(2)这钠双线同级谱线间的线距离随谱线级数的增加是如何变化的?
\item 由一尼科耳棱镜透射出来的平面偏振光,垂直射到一块石英的四分之一波片上,光的偏振面和波片光轴成30°角,然后又经过第二尼科耳,它和第一尼科耳夹角为60°,和波片主轴夹角也是 30°,试求:\\
(1)从四分之一波片出来的寻常光及非寻常光的光强之比;\\
(2)设入射光强为$I_0$,透过第二个尼科耳镜后的光强度为多少?
\end{enumerate}