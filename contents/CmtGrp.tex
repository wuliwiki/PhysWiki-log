% 换位子群
% keys 换位子|commutator|导子群|交换子|交换子群|群论|正规子群
% license Xiao
% type Tutor

\pentry{正规子群和商群\nref{nod_NormSG}}{nod_82ce}


\begin{definition}{换位子}\label{def_CmtGrp_1}
给定群 $G$ 中的任意元素 $a$ 和 $b$,记元素 $[a, b]=a^{-1}b^{-1}ab$,称之为为一个$G$上的\textbf{换位子(commutator)},或者\textbf{导子}。
\end{definition}

给\autoref{def_CmtGrp_1} 中的概念取换位子这一名称,是因为它作用在两个元素的乘积上可以交换其乘积顺序:$(ab)(b^{-1}a^{-1}ba)=ba$。它可以用来更细致地刻画群的交换性,而不仅仅是粗糙的“交换 / 不交换”的描述。具体来说,我们需要用\textbf{换位子群}来描述群的交换性:

\begin{definition}{换位子群}

给定群$G$及其子群$H$和$K$,称
\begin{equation}
[H, K] = \langle \{[h, k]\mid h\in H, k\in K\} \rangle~
\end{equation}
为$H$和$K$的\textbf{换位子群(commutator subgroup)}或\textbf{导子群(derived subgroup)}。特别地,称$[G, G]$为$G$的换位子群。

\end{definition}

注意,$[H, K]$并不是换位子构成的集合,而是由这个集合生成的子群。换句话说,换位子群中存在不是换位子的元素。

\begin{example}{换位子群中不是换位子的元素}

考虑由四个元素生成的\textbf{自由群}\upref{FreGrp}$\opn{F}_4=\langle x, y, z, w \rangle$,则显然$[x, y][z, w]\in [\opn{F}_4, \opn{F}_4]$,但这个元素本身并不是$\opn{F}_4$的换位子。

\end{example}

\begin{theorem}{}\label{the_CmtGrp_2}
给定群 $G$及其子群$H, K$,则 $[H, K]\triangleleft G$。
\end{theorem}

\textbf{证明}:

换位子群中的元素都可以写成换位子相乘的形式:$x_1x_2x_3\cdots x_n$,其中各 $x_i$ 为换位子$[h_i, k_i]$,$h_i\in H, k_i\in K$。因此,要证明 $g^{-1}[H, K]g\subseteq [H, K]$,只需要证明 $g^{-1}x_1g$ 是一个$H$和$K$的换位子即可,这样 $g^{-1}x_1x_2x_3\cdots x_ng=g^{-1}x_1gg^{-1}x_2gg^{-1}x_3gg^{-1}\cdots gg^{-1}x_ng$ 也是$H$和$K$的换位子的乘积,故在换位子群中。

对于任意 $g\in G$,任取$h\in H, k\in K$,则
\begin{equation}
\begin{aligned}
g^{-1}[h, k]g ={}& g^{-1}h^{-1}k^{-1}hkg\\
={}& g^{-1}h^{-1}gg^{-1}k^{-1}gg^{-1}hgg^{-1}kg\\
={}& \qty(g^{-1}hg)^{-1}\qty(g^{-1}kg)^{-1}\qty(g^{-1}hg)\qty(g^{-1}kg)
\end{aligned}
~
\end{equation}
也是一个换位子。

\textbf{证毕}。




换位子群刻画交换性的方式由以下\autoref{the_CmtGrp_1} 描述:

\begin{theorem}{}\label{the_CmtGrp_1}
设 $N$ 是群 $G$ 的正规子群,则商群 $G/N$ 交换当且仅当 $G^{(1)}\subseteq N$。
\end{theorem}

\textbf{证明}:

任取 $x, y\in G$,则 $G^{(1)}\subseteq N$ 当且仅当 $(xy)^{-1}yx\in N$。

而 $(xy)^{-1}yx\in N$ 等价于 $(xy)^{-1}yxN=N$,这又等价于 $yxN=xyN$,也就是商群运算的交换性。

% 先设 $G/N$ 交换。于是对于任意 $x, y\in G$,都有 $xyN=yxN$,即 $(xy)^{-1}yx\in N$,故任意换位子都在 $N$ 中,故换位子群必在 $N$ 中。

% 再设 $G^{(1)}\subseteq N$,那么

\textbf{证毕}。

\autoref{the_CmtGrp_1} 表明,一个群 $G$ 如果不交换,我们也可能找到一个正规子群使其商群为交换群,好比是抹去了一些细节之后陪集的运算体现出交换性。当然,\textbf{必须}抹去的细节越多,我们就可以说 $G$ 的交换性越差;没有抹去的必要,那 $G$ 就是完全交换的。而这个必须抹去的细节,就是换位子群。换位子群越大,则 $G$ 的交换性越低;换位子群只含单位元,那么 $G$ 本身就是交换的。












