% SQL 基础笔记

\begin{issues}
\issueDraft
\end{issues}

\begin{itemize}
\item 本文参考 w3schools
\item SQL(Structured Query Language) 是操作数据库的语言
\item MySQL 是具体的管理数据库的软件, 是 **RDBMS (relational database management system)
\item 开发者需要掌握 SQL 语言来使用 MySQL. 不同的 RDBMS 基本都是用 SQL 语言, 虽然有微小差异, 但都兼容 SQL standard.
\item dump 操作常常用于备份数据库(而不是直接拷贝数据库文件). 数据库文件一般是加密的, 但是 dump 不会加密.
\item 一个 database 里面可以有许多 table, 每个 field 就是 table 的一列, 每个 record 就是一行
\item 显示数据库 \verb`SHOW DATABASES;`
\end{itemize}

\subsection{SQL Basics}
\begin{itemize}
\item 大部分操作都通过 **SQL statement** 完成, 如 \verb`SELECT * FROM 表名;`
\item 命令可以随意换行,用分号结尾
\item SQL 关键字不区分大小写,但一般关键词用大写
\item 数据中的字符串用单引号,大部分数据库双引号也行,字符串区分大小写
\item 一些最重要的关键字如
\item \verb`SELECT` - extracts data from a database
\item \verb`UPDATE` - updates data in a database
\item \verb`DELETE` - deletes data from a database
\item \verb`INSERT INTO` - inserts new data into a database
\item \verb`CREATE DATABASE` - creates a new database
\item \verb`ALTER DATABASE` - modifies a database
\item \verb`CREATE TABLE` - creates a new table
\item \verb`ALTER TABLE` - modifies a table
\item \verb`DROP TABLE` - deletes a table
\item \verb`CREATE INDEX` - creates an index (search key)
\item \verb`DROP INDEX` - deletes an index
\end{itemize}

\subsubsection{SELECT}
\begin{itemize}
\item \verb`SELECT column1, column2, ... FROM table_name;` 获取的数据表叫做 **result-set**. 如果要选取所有 column, 用 \verb`*` 即可 e.g. \verb`SELECT * FROM Customers;` 显示名为 Customers 的表格中全部内容
\item \verb`SELECT DISTINCT column1, column2, ... FROM table_name;` 仅列出不完全相同的行, 例如 \verb`SELECT DISTINCT Country FROM Customers;`
\item \verb`SELECT COUNT(DISTINCT Country) FROM Customers;` 显示表格中有多少不同的国家. (这个命令在 Microsoft Access 里面没用)
\end{itemize}

\subsubsection{WHERE}
\begin{itemize}
\item \verb`SELECT column1, column2, ... FROM table_name WHERE condition;`
\item \verb`SELECT * FROM Customers WHERE Country='Mexico';`
\item \verb`SELECT * FROM Customers WHERE CustomerID=1;`
\item \verb`WHERE` 后面可以是 \verb`=`, \verb`>`, \verb`<`, \verb`>=`, \verb`<=`, \verb`<>`(不等于,有时候 \verb`!=` 也行), \verb`BETWEEN` \verb`LIKE` \verb`IN`(在几个可能的值之中)
\item \verb`SELECT * FROM Products WHERE Price BETWEEN 50 AND 60;`
\item \verb`WHERE` 后面的条件可以用 \verb`AND`, \verb`OR`, \verb`NOT`
\item \verb`SELECT column1, column2, ... FROM table_name WHERE NOT condition;`
\item \verb`SELECT * FROM Customers WHERE City='Berlin' OR City='München';`
\end{itemize}

\subsubsection{ORDER BY}
\begin{itemize}
\item 用 \verb`ORDER BY` 来进行排序 \verb`SELECT column1, column2, ... FROM table_name ORDER BY column1, column2, ... ASC|DESC;`
\item \verb`SELECT * FROM Customers ORDER BY Country DESC;`
\end{itemize}

\subsubsection{INSERT INTO}
\begin{itemize}
\item \verb`INSERT INTO table_name (column1, column2, column3, ...) VALUES (value1, value2, value3, ...);` 插入行, 如果每列都插入, 那么 \verb`(column1, column2,...)` 可以省略
\item 对所有列, \verb`INSERT INTO table_name VALUES (value1, value2, value3, ...);`
\item 第一列总是数字(无论 field 叫做什么)? 插入时可以指定,  如果不指定会自动更新. 不能指定已经存在的数字.
\item 如果某列时 optional 的, 那么插入一行时可以不输入这列的信息, 这列的值就是 \verb`NULL`
\item \verb`SELECT column_names FROM table_name WHERE column_name IS NULL;`
\item \verb`SELECT column_names FROM table_name WHERE column_name IS NOT NULL;`
\end{itemize}

\subsubsection{UPDATE}
\begin{itemize}
\item 改变已经存在的 record. where 可以选择一条或多条 record
\item 如果不用 \verb`WHERE`, 所有行的指定列都会被更新
\item \verb`UPDATE table_name SET column1 = value1, column2 = value2, ... WHERE condition;`
\item \verb`UPDATE Customers SET ContactName = 'Alfred Schmidt', City= 'Frankfurt' WHERE CustomerID = 1;`
\item \verb`UPDATE Customers SET ContactName='Juan' WHERE Country='Mexico';`
\end{itemize}

\subsubsection{DELETE}
\begin{itemize}
\item \verb`DELETE FROM table_name WHERE condition;`
\item e.g. \verb`DELETE FROM Customers WHERE CustomerName='Alfreds Futterkiste';`
\item \verb`DELETE FROM table_name;`
\item e.g. \verb`DELETE FROM Customers;`
\end{itemize}
