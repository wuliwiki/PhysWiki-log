% 粒子的小群分类(量子场论)
% keys 小群分类
% license Usr
% type Tutor

\begin{issues}
\issueMissDepend
\end{issues}

\subsection{庞加莱群}
庞加莱群是一个极大的对称性。具体的,将一个粒子 Lorentz 变换后平移,即:
\begin{equation}
	x^\mu \rightarrow (x')^\mu = \Lambda_{\nu}^\mu x^\nu + a^\mu ~.
\end{equation}
也可以理解为整体是一个\textbf{\textbf{非齐次 Lorentz 变换}}。其中 $\Lambda_{\nu}^\mu$ 是齐次部分,平移 $a^\mu$ 造成非齐次。熟知的,要求 $\Lambda_{\nu}^\mu$ 保度规,即:
\begin{equation}
	\Lambda_{\sigma}^\mu \Lambda_{\rho}^\nu \eta^{\sigma \rho} = \eta^{\mu \nu} ~.
\end{equation}
对于 $\Lambda$ 对应的 Lorentz 变换构成 Lorentz 群。保度规直接要求 $\det \Lambda = \pm 1$,同时再根据 $\Lambda^0_0$ 的符号可以对\textbf{\textbf{Lorentz 群}}进行分类(并不是庞加莱群)。

\subsection{庞加莱代数}
下面考虑 Lie 代数,也就是无穷小变换以及矩阵表示相关。关注一个无穷小变换:
\begin{equation}
	\Lambda_\nu^\mu = \delta_\nu^\mu + \omega_\nu^\mu , ~ a^\mu = \varepsilon^\mu ~,
\end{equation}
其中 $\omega^\mu_\nu$ 和 $\varepsilon^\mu$ 都是小量。带入对度规的变换的要求会得到
\begin{equation}
	\eta_{\rho\sigma} = \eta_{\mu\nu}(\delta^\mu_\rho + \omega^\mu_\rho) (\delta^\nu_\sigma + \omega^\nu_\sigma) = \eta_{\rho\sigma} + \omega_{\sigma\rho} + \omega_{\rho\sigma} + \mathcal O(\omega^2) ~.
\end{equation}
可以发现必须要求 $\omega_{\rho \sigma}$ 是\textbf{\textbf{反对称的}},也就是 $\omega_{\rho \sigma} = -\omega_{\sigma \rho}$。

考虑庞加莱对称性诱导的变换算符 $\Psi \to \Psi' = U(\Lambda, a) \Psi$\footnote{可以考虑为,$U(?)$ 是作用到希尔伯特空间上的。},并满足 $U(\Lambda', a') U(\Lambda, a) = U(\Lambda' \Lambda, \Lambda' a + a')$。变换算符可以根据对称性进行展开:
\begin{equation}
	U(1+\omega, \varepsilon) = 1 + \frac{1}{2} \mathbf{i} \omega_{\rho \sigma} J^{\rho \sigma} - \mathbf{i} \varepsilon_\rho P^\rho + \cdots ~~
\end{equation}
为了让其是幺正的,必须要求 $J$ 和 $P$ 都是 Hermit 的\footnote{即 $J^{\rho \sigma \dagger} = J^{\rho \sigma}, P^{\rho \dagger} = P^\rho$}。同时由于 $\omega$ 的反对称性,$J$ 也应当是反对称的。

考虑对 $U(1+\omega, \varepsilon)$ 进行庞加莱变换: \footnote{$U^{-1}(\Lambda, a) = U(\Lambda^{-1}, -\Lambda^{-1}a)$}
\begin{equation}
	U(\Lambda, a) U(1+\omega, \varepsilon) U^{-1}(\Lambda, a) ~,
\end{equation}
代入可以得到 
\begin{equation}
	\begin{aligned}
		U(\Lambda, a) J^{\rho \sigma} U^{-1}(\Lambda, a) &= \Lambda_{\mu}^\rho \Lambda_\nu ^\sigma(J^{\mu\nu} - a^\mu P^\nu + a^\nu P^\mu) \\ 
		U(\Lambda, a) P^\mu U^{-1}(\Lambda, a) &= \Lambda_{\nu}^\mu P^\nu ~~
	\end{aligned}~~
\end{equation}

若令这一步的庞加莱变换(实际上是 Lorentz 变换)也成为一个无穷小变换就得到了\textbf{\textbf{庞加莱代数}}:\footnote{值得一提,笔者的推导是在东海岸度规下进行的。}
\begin{equation}
	[P^\mu, P^\nu] = 0 ~,
\end{equation}
\begin{equation}
	\mathbf{i} [P^\mu, J^{\rho \sigma}] = \eta^{\mu \rho} P^\sigma - \eta^{\mu \sigma} P^\rho ~,
\end{equation}
\begin{equation}
	\mathbf{i} [J^{\mu\nu}, J^{\rho \sigma}] = \eta^{\nu \rho} J^{\mu \sigma} + \eta^{\sigma \nu} J^{\rho \mu} - \eta^{\mu \rho} J^{\nu \sigma} - \eta^{\sigma \mu} J^{\rho \nu} ~.
\end{equation}
熟知的物理意义,$P^0$ 是能量(哈密顿量),$P^{1}, P^2, P^3$ 是动量,角动量分别是 $J^{23}, J^{31}, J^{12}$,以及推促(boost)对应 $J^{01}, J^{02}, J^{03}$。

\subsection{粒子是庞加莱群的表示}
所谓群的表示,下面予以简单介绍。
\begin{definition}{群的表示}
	设 $G$ 是一个群,$V$ 是一个矢量空间,$\rho: G \to V$ 使得 $\forall g \in G$, $\rho(g)$ 是 $V$ 上的一个线性变换;同时 $\rho$ 保群乘法,即 $\forall g_1, g_2 \in G, \rho(g_1 g_2) = \rho(g_1) \rho(g_2)$。可以理解为 $\rho$ 将群元表示为矩阵。
\end{definition}

若考虑一个粒子有态 $\Phi_{\sigma}$,$\sigma$ 是描述粒子状态的所有指标,可以是 $p^\mu$、$J^\mu$、旋量指标等等。若对这个态进行庞加莱变换,物理过程上考虑、这个粒子应当还是原来的粒子,而且仅是对各个不同态进行叠加:
\begin{equation}
	U(\Lambda, a) \Phi_\sigma = \sum_{\rho }\mathcal M_{\sigma \rho} \Phi_{\rho} ~.
\end{equation}
此处就将群元 $U(\Lambda, a)$ 表示为了一个矩阵 $\mathcal M_{\sigma \rho}$,故说粒子是庞加莱群的一个表示。

回顾\textbf{\textbf{不可约表示}}是将矩阵拆成分块矩阵,仅在对角线上有元素,显然粒子应该是一个不可约表示,才能“自己”变成“自己”,而不产生“杂化”。

同时物理上很容易想到粒子应该是一个幺正的表示,故说:

\textbf{\textbf{粒子是庞加莱群的幺正不可约表示}}



\subsection{对粒子分类}
对粒子进行分类,就需要找出来一组庞加莱群的幺正不可约表示。为此就需要找到一组好的基来进行分解,同时让抽象的 $\Phi_\mu$ 有实际意义。从物理上,不妨尝试用动量 $P^\mu$ 先来标记粒子。考虑一组动量的本征态 
\begin{equation}
	P^\mu \Phi_{p, \sigma} = p^\mu \Phi_{p, \sigma} ~,
\end{equation}
此处 $p^\mu$ 就是粒子真实的 $4$-动量。将庞加莱群的变换分开看,分别考虑平移和 Lorentz 变换。先看平移:\footnote{这里直接运用了 Lie 群是对应 Lie 代数的指数映射关系。}
\begin{equation}
	U(1, a) \Phi_{p, \sigma} = e^{-\mathbf{i} p \cdot a} \Phi_{p, \sigma} ~,
\end{equation}
然后是一个齐次的 Lorentz 变换 $U(\Lambda) = U(\Lambda, 0)$,给出一个 $\Lambda p$ 的本征矢:
\begin{equation}
	P^\mu U(\Lambda) \Phi_{p, \sigma} = U(\Lambda) [U^{-1}(\Lambda) P^\mu U(\Lambda)] \Phi_{p, \sigma} ~, 
\end{equation}
其中可以进行代换中括号内为 $(\Lambda^{-1})^\mu_{\nu} P^\nu$,也就是给出 
\begin{equation}
	P^\mu U(\Lambda) \Phi_{p, \sigma} = \Lambda_\nu^\mu p^\nu U(\Lambda) \Phi_{p, \sigma} ~,
\end{equation}
从而 $U(\Lambda) \Phi_{p, \sigma}$ 必须是 $\Phi_{\Lambda p, \sigma'}$ 的线性组合才可以,即 
\begin{equation}
	U(\Lambda) \Phi_{p, \sigma} = \sum_{\rho} C_{\rho \sigma} \Phi_{\Lambda p, \rho} ~.
\end{equation}
现在考虑选取合适的基 $\Phi_{p, \sigma}$ 使得表示矩阵 $C_{\rho \sigma}$ 分块对角化,也就形成了一个不可约表示。

熟知的,有质量粒子和无质量粒子之间是无法通过 Lorentz 变换得到的,为此选取两个代表的动量\footnote{沿用 Weinberg 的记号,时间维在最后。}
\begin{equation}
	k_1^\mu = (0, 0, 0, M); ~ k_2^\mu = (0, 0, \kappa, \kappa) ~.
\end{equation}
然而现在并不能区分“两个自由粒子”(比如两个粒子的质心系)和单粒子态(含束缚态)。为此考虑按住粒子的动量不变,考虑 $\sigma$ 指标的变化。

具体过程,先找一个“标准动量” $k^\mu$,例如 $k^\mu = (0, 0, 0, M)$,然后考虑使得 $k^\mu$ 不变的变换:$k^\mu = W^\mu_\nu k^\nu$,是\textbf{\textbf{Wigner 的“小群”}}(保这一个标准动量不变,也正是所谓“迷向子群”)。

对于一个粒子,例如刚才的 $k^\mu$,小群就成为 $\text{SO}(3)$,可以约化为“自旋” $j$ 的标记;而例如对于两个粒子,体现为 $\sigma$ 连续,要先约化到每个粒子的动量 $p$ 再按 $\sigma$ 展开。

最后给出分类结果:\footnote{其中 $\text{ISO}(2)$ 是 $2$ 维欧氏空间的等距变换群,由两个平动和一个转动生成元给出。}
\begin{table}[ht]
\centering
\caption{小群分类结果}\label{tab_WigCla1}
\begin{tabular}{|c|c|c|c|}
\hline
     &  & 标准 $k^\mu$ & 小群 \\
    \hline
    (a) & $p^2 = -M^2 < 0, p^0 > 0$ & $(0, 0, 0, M)$ & $\text{SO}(3)$ \\
    \hline
    (b) & $p^2 = -M^2 < 0, p^0 < 0$ & $(0, 0, 0, -M)$ & $\text{SO}(3)$ \\
    \hline
    (c) & $p^2 = 0, p^0>0$ & $(0, 0, \kappa, \kappa)$ & $\text{ISO}(2)$ \\
    \hline
    (d) & $p^2 = 0, p^0<0$ & $(0, 0, \kappa, -\kappa)$ & $\text{ISO}(2)$ \\ 
    \hline
    (e) & $p^2 = N^2 > 0$ & $(0, 0, N, 0)$ & $\text{SO}(2, 1)$ \\
    \hline
    (f) & $p^\mu = 0$ & $(0, 0, 0, 0)$ & $\text{SO}(3, 1)$ \\
    \hline
\end{tabular}
\end{table}

\footnote{其中结果只有 (a)、(c)、(f) 是物理的,(f) 是真空。}显然对于任意一个该类下的粒子的动量,可以通过该类的标准动量经过某个线性变换得到:$p^\mu = L^\mu_\nu k^\nu$。

\subsection{分类结果分析}
显然对于任意一个粒子的态 $\Phi_{p, \sigma}$,可以通过 $L(p)$ 将标准动量的态 $\Phi_{k, \sigma}$ 变换到 $\Phi_{p, \sigma} = U[L(p)] \Phi_{k,\sigma}$(未归一化)。

另外若考虑一般动量态的 Lorentz 变换与标准动量的小群变换的关系:
\begin{equation}
	U(\Lambda) \Phi_{p, \sigma} = U[L(\Lambda p)] (U[L^{-1} (\Lambda p) \Lambda L(p)] \Phi_{k, \sigma}) ~~
\end{equation}
令 $w^\mu_\sigma(\Lambda, p) = [L^{-1}(p)]^\mu_\nu \Lambda^\nu_\rho L(p)^\rho_\sigma$\footnote{其中,$L(p)$ 将 $k$ 变为 $p$,$\Lambda$ 将 $p$ 变为 $\Lambda p$,$L(\Lambda p)$ 将 $k$ 变为 $\Lambda p$,故这个 $w$(即 Wigner 转动)将 $k$ 沿 $k \to p \to \Lambda p \to k$ 的路径“转动一圈”,是一个小群变换(即在小群内的)。},则:
\begin{equation}
	U(w) \Phi_{k, \sigma} = \sum_{\sigma'} D_{\sigma' \sigma}(w) \Phi_{k, \sigma'} ~.
\end{equation}
这是 by definition 的,显然应当是个线性变换。现在,通过 $w$,可以通过 Wigner 转动将 $\Lambda p$ 与 $p$ 联系:\footnote{仍未归一化。}
\begin{equation}
	\begin{aligned}
		U(\Lambda) \Phi_{p, \sigma} 
		&= U[\Lambda L(p)] \Phi_{k, \sigma} \\
		&= U[L(\Lambda p)] (U[L^{-1}(\Lambda p) \Lambda L(p)] \Phi_{k, \sigma}) \\
		&= U[L(\Lambda p)] (U[w(\Lambda, p)] \Phi_{k, \sigma}) \\
		&= U[L(\Lambda p)] \sum_{\sigma '} D_{\sigma ' \sigma}[w(\Lambda, p)] \Phi_{k, \sigma'} \\
		&= \sum_{\sigma'} D_{\sigma' \sigma} D\left(w(\Lambda, p)\right) \Phi_{\Lambda p, \sigma'} ~.
	\end{aligned} ~~
\end{equation}

\subsection{正质量粒子}
在正质量粒子前提下,小群是 $\text{SO}(3)$,$D^{(j)}_{\sigma' \sigma}$ 是三维转动群的自旋为 $j$ 的不可约表示\footnote{熟知三维转动群有不等价的不可约表示,用自旋区分。}:

对于无穷小转动 $R_{ik} = \delta_{ik} + \Theta_{ik}$\footnote{其中 $\Theta$ 反对称。},有:
\begin{equation}
	D_{\sigma' \sigma}^{(j)} (1+\Theta) = \delta_{\sigma' \sigma} + \frac{\mathbf{i}}{2} \Theta_{ik} \left(J_{ik}^{(j)}\right)_{\sigma' \sigma} ~.
\end{equation}
其中 $J^{(j)}$ 满足\footnote{取沿着第三方向的转轴}:
\begin{equation}
	\left(J^{(j)}_{23} \pm \mathbf{i} J_{31}^{(j)}\right)_{\sigma' \sigma} = \left(J_1^{(j)} \pm \mathbf{i} J_2^{(j)}\right) = \delta_{\sigma', \sigma \pm 1} \sqrt{(j \mp \sigma)(j \pm \sigma + 1)} ~.
\end{equation}
\footnote{这即是上升下降算符乘以归一化因子},以及第三方向上
\begin{equation}
	\left(J_{12}^{(j)}\right)_{\sigma' \sigma} = \left(J^{(j)}_3\right)_{\sigma' \sigma} = \sigma \delta_{\sigma' \sigma} ~.
\end{equation}

\subsection{无质量粒子}
这会麻烦一些,要找到让标准动量 $(0, 0, \kappa, \kappa)$ 不变的小群是无穷小变换的哪些组合。

首先,显然可以沿 $z$ 轴转 $\theta$ 角,$U(w) - 1$ 就可以包含 $\mathbf{i} \theta J_3$\footnote{$J_3$ 是沿 $z$ 轴转动的转动生成元}。

其次,若可以沿 $x$ 或 $y$ 转 $\alpha$ 角,当 $\alpha$ 是无穷小转动时,不妨取 $\kappa = 1$,$(0, 0, 1, 1)$ 变为 $(\alpha, 0, 1, 1)$;而若这时候再在 $x$ 方向 boost:$(0, 0, 1, 1)$ 变为 $(-\alpha, 0, 1, 1)$,组合恰好抵消。就可令 $A = J_2 + \mathbf{i} K_1$,保标准动量。

类似的,令 $B = -J_1 + K_2$,也保标准动量。故可写(无穷小变换形式下) $U(w(\theta, \alpha, \beta)) = 1 + \mathbf{i} \alpha A + \mathbf{i} \beta B + \mathbf{i} \theta J_3$。由 $J$ 和 $K$ 的 Lie 代数可以写出:$[J_3, A] = iB$,$[J_3, B] = -iA$,$[A, B] = 0$。

现在考虑 $D_{\sigma' \sigma}$,由于 $A$ 与 $B$ 对易,可以同时对角化 $A$ 和 $B$;取他们的共同本征态:$A \Phi_{k, a, b, \sigma} = a \Phi_{k, a, b, \sigma}$ 且 $B \Phi_{k, a, b, \sigma} = b \Phi_{k, a, b, \sigma}$。然而实验上未发现任何 $a \neq 0$ 或 $b \neq 0$ 的粒子\footnote{若有不等于 $0$ 的,沿这方向的转动会有多余自由度。有相关研究,比如 1302.1198,以及相关文章。}

现在设 $a=b=0$,并用 $J_3$ 的本征态标记粒子:$J_3 \Phi_{k, \sigma} = \sigma \Phi_{k, \sigma}$\footnote{这就是螺旋度 helicity,是角动量在动量方向上的投影}。立刻可以写:
\begin{equation}
	U(w) \Phi_{k, \sigma} = e^{\mathbf{i} \theta \sigma} \Phi_{k, \sigma} ~.
\end{equation}
此时 $D_{\sigma' \sigma} = \exp(\mathbf{i} \theta \sigma) \delta_{\sigma' \sigma}$。

一些例子:$\sigma = \pm 1$ 可以由空间反演联系为同一粒子,是光子;$\sigma = \pm 2$ 将会是引力子;$\sigma = \pm 1/2$ 是在 Higgs 机制前无质量的费米子。此外,拓扑考虑下,$\sigma$ 仅能取整数或半整数。