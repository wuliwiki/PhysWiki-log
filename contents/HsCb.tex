% 组合(高中)
% 高中|排列

\pentry{排列(高中)\upref{HsPm}}

\subsection{定义}
一般地,从 $n$ 个不同元素中,任取出 $m(m\leq n)$ 个元素并成一组,叫做从 $n$ 个元素中任取 $m$ 个元素的\textbf{一个组合(combination)}.

从 $n$ 个不同元素中,任取 $m(m\leq n)$个元素的所有组合的个数,叫做从 $n$ 个不同元素中,任意取出 $m$ 个元素的\textbf{组合数(number of combination)},用符号 $C_n^m$ 表示.

我们可以从排列来推导组合,在组合中认为${a,b}$和${b,a}$是等效的,在排列中认为两者是不等的,现在我们有$n$个元素,我们要从中选取$m$个元素,即$C_n^m$.
我们先根据组合的知识可以知道 $A_n^m$ 表示 $n$ 个元素,从中选取 $m$ 个再对选中的 $m$ 个元素进行全排,我们就可以得到如下公式
\begin{equation}
A_n^m = C_n^m A_m^m
\end{equation}
对等式进行变换,可得
\begin{equation}\label{HsCb_eq1}
C_n^m = \frac {A_n^m}{A_m^m}
\end{equation}
对于这个公式,我们可理解为,从排列中排除组合中认为等效的组合.
我们将\autoref{HsCb_eq1} 展开,可得
\begin{equation}\label{HsCb_eq2} 
C_n^m = \frac{n(n -1) \cdots(n -m + 1)}{m(m-1)\cdots 1}
\end{equation}
进一步变换,可得
\begin{equation}\label{HsCb_eq3}
C_n^m = \frac{n!}{m!(n-m)!}
\end{equation}

\subsection{性质}

我们将\autoref{HsCb_eq3} 中的$m$用$n-(n-m) $代换,可得组合的\textbf{性质1}
\begin{equation}
C_n^m = \frac{n!}{(n -m)![n-(n-m)]!} = C_n^{n-m}
\end{equation}
对于性质1我们可以用一种直观的方式理解,到我们取m个元素时,剩余的元素本身就是取$n-m$个元素时的组合

当我们的总元素个数从$n$变为$n$变为$n+1$时,我们可以分两类,一类为不包含新元素的组合,一类为包含新元素的组合,由分类加法技术原理,可得组合的\textbf{性质2}
\begin{equation}
C_{n + 1}^m = C_n^m + C_n^{m -1}
\end{equation}
当然我们也可以对性质2进行数学证明,
\begin{equation}
\begin{aligned}
C_n^m + C_n^{m - 1} &= \frac{n!}{m!(n-m)!} + \frac{n!}{(m - 1)!(n - m + 1)!}\\
&= \frac{n!(n - m + 1)! + n!\cdot m}{m!(n - m + 1)!}\\
&= \frac{n![(n - m + 1)] + m}{m!(n - m + 1)!}\\
&= \frac{(n + 1!)}{m!(n - m + 1)!}\\
&= C_{n + 1}^m
\end{aligned}
\end{equation}