% 库珀对
% license CCBYSA3
% type Wiki

(本文根据 CC-BY-SA 协议转载自原搜狗科学百科对英文维基百科的翻译)

\begin{figure}[ht]
\centering
\includegraphics[width=10cm]{./figures/b24f79a35ae4d3c4.png}
\caption\label{fig_KBD_1}
\end{figure}

在凝聚态物理学中,\textbf{库柏对}或者\textbf{BCS}对美国物理学家利昂·库珀在1956年首次描述了一对电子(或其他费米子)在低温下以某种方式结合在一起。[1]库珀指出,金属中电子之间任意小的吸引力都会导致电子成对状态的能量低于费米能,这意味着该对是绑定的。在传统中超导体,这种吸引力是由于电子–声子互动。库柏对状态是超导的原因,如BCS理论开发人约翰·巴丁,利昂·库珀,和约翰·施里弗为此他们分享了1972年诺贝尔奖。[2]

虽然库珀配对是一种量子效应,但配对的原因可以从简化的经典解释中看出。[2][3]金属中的电子通常表现为自由粒子。由于其他电子是负电子,所以电子被排斥费用,但它也吸引了积极的一面离子组成金属刚性晶格的。这种吸引力扭曲了离子晶格,使离子稍微向电子移动,增加了附近晶格的正电荷密度。这种正电荷可以吸引其他电子。在长距离下,由于离子位移引起的电子之间的吸引力可以克服由于负电荷引起的电子排斥,并使它们配对。严格的量子力学解释表明,这种效应是由于电子–声子声子是带正电荷晶格的集体运动。[4]

配对相互作用的能量相当弱,约为$10^{-3}$eV ,热能可以很容易地破坏成对。因此,只有在低温下,在金属和其他衬底中,才有大量的电子处于库珀对中。

一对中的电子不一定靠得很近;因为相互作用是长距离的,配对的电子可能仍有数百个纳米分开。这个距离通常大于平均电极间距离,因此许多库珀对可以占据相同的空间。[5]电子有旋转-1⁄2,所以它们是费米子,但是库柏对的总自旋是整数(0或1),所以它是一个复合玻色子。这意味着波函数在粒子交换下是对称的。因此,与电子不同,允许多个库珀对处于相同的量子态,这是超导现象的原因。

BCS理论也适用于其他费米子系统,如氦-3 。事实上,库珀配对是氦-3在低温下的超流动性现象的原因。最近还证明了库柏对可以由两个玻色子组成。[6]这里,配对由光学晶格中的纠缠支持。

\subsection{与超导性的关系}
一个物体中所有库珀对“凝聚”成相同的基态的趋势是超导的特殊性质的原因。

库珀最初只考虑金属中形成孤立对的情况。当考虑许多电子对形成的更现实的状态时,正如在完整的BCS理论中阐明的,人们发现电子对在允许的电子能量状态的连续光谱中打开了一个间隙,这意味着系统的所有激发必须具有一定的最小能量。这兴奋间隙导致超导性,因为电子的散射等小激发是被禁止的。[7]该间隙是由于感受到吸引力的电子之间的多体效应而出现的。

R.小奥格(A. Ogg,Jr .)第一次提出电子可以作为材料中晶格振动耦合的对。[8] [9]超导体中观察到的同位素效应表明了这一点。同位素效应表明,含较重离子(不同核同位素)的材料具有较低的超导转变温度。这可以用库珀对理论来解释:较重的离子更难被电子吸引和移动(库珀对是如何形成的),这导致对的结合能更小。

库珀对理论非常普遍,不依赖于特定的电子-声子相互作用。凝聚态物质理论家提出了基于其他吸引相互作用的配对机制,如电子–激子相互作用或电子–等离子体相互作用。目前,在任何材料中都没有观察到这些其他配对相互作用。

应该指出的是,库珀配对不涉及单个电子配对形成“准玻色子”。成对的状态在能量上是有利的,电子优先进出这些状态。这是约翰·巴丁做出的一个细微的区别:

"配对电子的想法,虽然不完全准确,但抓住了它的感觉."[10]

这里涉及的二阶相干性的数学描述由杨给出。[11]

\subsection{参考文献}
[1]
^Cooper, Leon N. (1956). "Bound electron pairs in a degenerate Fermi gas". Physical Review. 104 (4): 1189–1190. Bibcode:1956PhRv..104.1189C. doi:10.1103/PhysRev.104.1189..

[2]
^Nave, Carl R. (2006). "Cooper Pairs". Hyperphysics. Dept. of Physics and Astronomy, Georgia State Univ. Retrieved 2008-07-24..

[3]
^Kadin, Alan M. (2005). "Spatial Structure of the Cooper Pair". Journal of Superconductivity and Novel Magnetism. 20 (4): 285. arXiv:cond-mat/0510279. doi:10.1007/s10948-006-0198-z..

[4]
^Fujita, Shigeji; Ito, Kei; Godoy, Salvador (2009). Quantum Theory of Conducting Matter. Springer Publishing. pp. 15–27. ISBN 978-0-387-88211-6..

[5]
^Feynman, Richard P.; Leighton, Robert; Sands, Matthew (1965). Lectures on Physics, Vol.3. Addison–Wesley. pp. 21–7, 8. ISBN 0-201-02118-8..

[6]
^玻色子库珀对.

[7]
^Nave, Carl R. (2006). "The BCS Theory of Superconductivity". Hyperphysics. Dept. of Physics and Astronomy, Georgia State Univ. Retrieved 2008-07-24..

[8]
^Ogg, Richard A. (1 February 1946). "Bose-Einstein Condensation of Trapped Electron Pairs. Phase Separation and Superconductivity of Metal-Ammonia Solutions". Physical Review. American Physical Society (APS). 69 (5–6): 243–244. doi:10.1103/physrev.69.243. ISSN 0031-899X..

[9]
^小普尔,查尔斯·P,《凝聚态物理百科词典》(学术出版社,2004年),第576页.

[10]
^Bardeen, John (1973). "Electron-Phonon Interactions and Superconductivity". Written at New York. In H. Haken and M. Wagner. Cooperative Phenomena. Berlin, Heidelberg: Springer Berlin Heidelberg. p. 67. doi:10.1007/978-3-642-86003-4_6. ISBN 978-3-642-86005-8..

[11]
^Yang, C. N. (1 September 1962). "Concept of Off-Diagonal Long-Range Order and the Quantum Phases of Liquid He and of Superconductors". Reviews of Modern Physics. American Physical Society (APS). 34 (4): 694–704. doi:10.1103/revmodphys.34.694. ISSN 0034-6861..