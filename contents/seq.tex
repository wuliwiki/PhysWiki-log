% 数列
% keys 数列
% license Xiao
% type Tutor

\begin{issues}
\issueTODO
\issueDraft
% \issueOther{定位是什么?和数列的概念与函数特性(高中)\upref{HsSeFu}区别在哪?}
\end{issues}

%\subsection{数列的定义}
\begin{definition}{数列}
我们定义\textbf{数列}(或\textbf{序列}),是从正整数集 $\mathbb{N}$  到实数集 $\mathbb{R}$ 的一个函数 $f\ :\ \mathbb{N}\rightarrow \mathbb{R}$。
\end{definition}
\begin{figure}[ht]
\centering
\includegraphics[width=12cm]{./figures/3d5e094561513874.png}
\caption{数列} \label{fig_seq_1}
\end{figure}

可以将一个数列看做是按照一定顺序排列的一列数:
\begin{equation}
x_1=f(1),\ x_2=f(2),\ \cdots,\ x_n=f(n),\ \cdots~
\end{equation}
通常将这个数列记为 $\{x_n\}$,其中 $x_n$ 称为\textbf{通项}。

要注意,作为\textbf{集合}和作为\textbf{数列}的$\{x_n\}$不同,比如数列中可以有$x_1=x_2$等情况,但集合中没有两个相同的元素。

例如:可以将 $[0,1)$ 中的有理数排成一个数列:
\begin{equation}
0,\frac{1}{2},\frac{1}{3},\frac{2}{3},\frac{1}{4},\frac{3}{4},\frac{1}{5}\cdots~
\end{equation}

一个有趣的事实是,可以通过某种方式将全体有理数排成一个数列,但不能将区间 $[a,b]\ (a<b)$ 内的全体实数排成一个数列。因此我们称有理数集是\textbf{可数集},实数集是\textbf{不可数集}。可数集的意思是,这个集合中的元素可以被排成一个数列;而不可数集就表示做不到。

\begin{example}{}
思考:为什么实数集是不可数集?

考虑 $[0,1)$ 之间的实数,用“无限长”的二进制数表示。假设可以排成数列: 
\begin{equation}
\begin{aligned}
x_1&=0.\boldsymbol 0001000...\\
x_2&=0.0\boldsymbol 110100...\\
x_3&=0.01\boldsymbol 01100...\\
x_4&=0.001\boldsymbol 0000...\\
x_5&=0.0000\boldsymbol 001...\\
x_6&=0.10000\boldsymbol 01...\\
\cdots
\\
y&=0.101111...
\end{aligned}~
\end{equation}
则可以构造新的实数 $y$,使得小数点后第 $i$ 位与 $x_i$ 的第 $i$ 位不同。那么新构造的这个实数,不出现在这个数列中。
矛盾!因此实数集是不可数集。
\end{example}

​		

​		
