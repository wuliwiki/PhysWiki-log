% 量子气体(巨正则系宗)
% 统计力学|巨正则系综|玻色子|费米子|巨配分函数|配分函数

\pentry{近独立子系\upref{depsys},巨正则系综法\upref{MCEsb}}
\begin{issues}
\issueDraft
\end{issues}

假设 $N$ 个玻色子(不可区分)之间没有互相作用, 每个粒子具有能级 $\varepsilon_i$ (这里先假设每个粒子都只有 $m$ 个能级而不是无穷多个, 最后再说明 $m \to \infty $ 时成立)。化学势为 $\mu $,  则巨分配函数为
\begin{equation}\ali{
\Xi & = \sum_{N=0}^\infty \sum_{\{n_i\} }^*  \E^{(N\mu - \varepsilon_i)\beta} 
 = \sum_{N=0}^\infty  \sum_{\{n_i\}}^* {z^N}\exp(- \beta \sum_i^m n_i \varepsilon_i)   \\
& = \sum_{N=0}^\infty  \sum_{\{ {n_i}\} }^*  \qty(\prod_{i=1}^\infty z^{n_i}) \qty(\prod_{i=1}^\infty (\E^{-\beta \varepsilon_i})^{n_i})
= \sum_{N=0}^\infty  \sum_{\{n_i\}}^* \prod_{i=1}^\infty (z \E^{-\beta\varepsilon_i})^{n_i}~.
}\end{equation}
其中 $\sum_{\{n_i\} }^ *  {} $ 求和时, 由于同一个能级上的玻色子数量不限, 限制条件仅为 $\sum_i^m {n_i}  = N$ (对于费米子, 由于不相容原理, 要求 $\sum_i^m {n_i}  = N$ 以及 ${n_i} \leqslant 1$ )。

巨正则系综的最大优势就是可以利用关系
\begin{equation}
\sum_{N=0}^\infty \sum_{\{n_i\}}^ *  (\dots)  = \sum_{n_1 = 0}^\infty  \sum_{n_2=0}^\infty \dots \sum_{m = 0}^\infty  (\dots)~.
\end{equation}
于是
\begin{equation}\ali{
\Xi & = \sum_{N = 0}^\infty \sum_{\{ {n_i}\} }^ *  \prod_{i=1}^\infty {(z{\E^{ - \beta {\varepsilon_i}}})}^{n_i}
 = \sum_{n_1=0}^\infty \sum_{{n_2} = 0}^\infty \dots \sum_{n_m=0}^\infty \prod_{i=1}^\infty {(z{\E^{ - \beta {\varepsilon_i}}})}^{n_i}\\
& = \sum_{n_1=0}^\infty (z\E^{-\beta \varepsilon_1})^{n_1} \sum_{n_2=0}^\infty (z\E^{ - \beta\varepsilon_2})^{n_2} \dots \sum_{n_m = 0}^\infty (z \E^{-\beta\varepsilon_m})^{n_m}\\
& = \prod_{i = 1}^m \sum_{n_i}^\infty (z \E^{-\beta\varepsilon_i})^{n_i} ~.
}\end{equation}
