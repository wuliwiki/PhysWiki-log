% 基本群的计算
%基本群|Seifert-van Kampen定理|van Kampen定理|莫比乌斯带
\pentry{基本群\upref{HomT3},可缩空间\upref{HomT2},群的自由积\upref{FrePrd}}

%基本完成

虽然把各阶同伦群都考虑进去以后可以很详细地刻画空间的伦型,但是高阶同伦群大多极其难计算。本节简单介绍一阶同伦群,即基本群的计算方法。

\subsection{基本群计算定理}

我们列举一些方便用于计算基本群的定理如下。

\begin{theorem}{积空间的基本群}
给定拓扑空间 $X$ 和 $Y$,则 $\pi_1(X\times Y)=\pi_1(X)\times\pi_1(Y)$。
\end{theorem}
%需要附证明吗?直观来看这个定理很显然,证明无非就是严格把直觉描述清楚。

\begin{theorem}{Seifert-van Kampen定理}
设拓扑空间 $X$ 可以被它的两个开集 $U_1$ 和 $U_2$ 覆盖,即 $X=U_1\cup U_2$;若 $U_1$,$U_2$ 和 $U_1\cap U_2$ 都是道路连通空间,取 $U_1\cap U_2$ 中一个点作为基点来构造各空间的基本群。设 $f_i:U_1\cap U_2\rightarrow U_i$ 为恒等嵌入,即 $\forall x\in U_1\cap U_2, f_i(x)=x\in U_i$。如果用 $f_i$ 来定义 $\pi_1(U_1\cap U_2)$ 到 $\pi_1(U_i)$ 上的同态,那么有:$\pi_1(X)=\pi_1(U_1)*_{\pi_1(U_1\cap U_2)}\pi_1(U_2)$。
\end{theorem}

Seifert-van Kampen定理难以简洁表达,不过如果能充分理解\textbf{群的自由积}\upref{FrePrd},应该容易理解该定理。不过,我们可以考虑该定理的弱化版本,此版本用处也很广泛:

\begin{theorem}{弱化版Seifert-van Kampen定理}
设拓扑空间 $X$ 可以被它的两个开集 $U_1$ 和 $U_2$ 覆盖,并且 $U_1\cap U_2$ 是单连通的\autoref{def_SmpCn_1}~\upref{SmpCn},那么有 $\pi_1(X)=\pi_1(U_1)*\pi_1(U_2)$。
\end{theorem}

除此之外,我们在\textbf{可缩空间}\upref{HomT2}中提到的形变收缩也能用于大大简化基本群的计算。

\begin{theorem}{形变收缩核的同伦}\label{the_HomT5_1}
给定拓扑空间 $X$。如果 $f:X\rightarrow A\in X$ 是一个形变收缩,且 $A$ 是其收缩核,那么 $X\cong A$。
\end{theorem}

证明很简单,由形变收缩的定义可知,$f:X\rightarrow A$ 和 $1_X:A\rightarrow X$ 是彼此的同伦逆;$1_X$ 是 $X$ 到自身的恒等映射,由于此处将其限制在 $A$ 上,也可以记为 $1_A=1_X|_A$。

\autoref{the_HomT5_1} 很好地展示了同胚和同伦的区别:如果存在一个形变收缩,把拓扑空间压成更低维度的情况,比如将烟卷压成圆环,那么收缩前后的空间是同伦的,但它们由于维度不同,就不会同胚。

\begin{theorem}{锥空间的基本群}
设 $X$ 是任意非空的拓扑空间,则 $\pi_1(\widetilde{C}X)=\{e\}$,即为只有一个元素的平凡群。
\end{theorem}

锥空间的基本群总是平凡群,也就是说所有回路都是保基点同伦的。这是因为,首先锥空间一定是道路连通空间,因此基点可以任意选择,不妨选为锥顶点;其次,任何一条回路都可以通过各点沿着锥空间的 $I$ 分量连续地收缩到锥顶点上,从而和恒等于锥顶点的回路同伦。



\subsection{基本群计算实例}

\pentry{复数\upref{CplxNo},覆叠空间\upref{CovTop}}

$S^1$ 是用于构建许多拓扑空间的伦型的原料,因此严格证明 $S^1$ 的基本群非常重要。\autoref{ex_HomT5_2} 提供了严格证明的大体思路,限于篇幅,只能由有兴趣的读者自行补充细节了。

\begin{example}{$S^1$ 的基本群}\label{ex_HomT5_2}
将 $S^1$ 看成复平面上的单位圆 $\{\E^{2\pi\I t}\in\mathbb{C}|t\in\mathbb{R}\}=\{x+y\I\in\mathbb{C}|x=\cos{2\pi t}, y=\sin{2\pi t}\}$。取通常的实度量空间 $\mathbb{R}$,建立映射 $p:\mathbb{R}\rightarrow S^1$,其中 $p(t)=\E^{2\pi\I t}$,则 $p$ 是一个覆叠映射。

把 $S^1$ 看成 $\mathbb{R}$ 的商拓扑空间,其中等价关系 $\sim$ 为:$x\sim y\iff \abs{x-y}\in\mathbb{Z}$,即把实数轴绕到周长为 $1$ 的圆上。对于圆周上任意一个点 $\E^{2\pi\I t_0}$,可以取典范邻域 $U_{t_0}=\{\E^{2\pi\I t}|t\in(t_0-1/4, t_0+1/4)\}$。这里 $1/4$ 的选择是任意的,换成任何小于 $1/2$ 的正数也可以,我们只需要用该典范邻域来说明接下来定义的提升映射 $\tilde{f}$ 是唯一的。

取 $S^1$ 的基点为 $p(0)=1$,设 $S^1$ 中有一条道路 $f:I\rightarrow S^1$。我们可以把 $f$\textbf{提升}为 $\mathbb{R}$ 中的道路 $\tilde{f}:I\rightarrow\mathbb{R}$,使得 $f=p\circ\tilde{f}$。如果取定 $\tilde{f}(0)=0$,那么这种提升是唯一的\footnote{这是因为通过 $f$ 可以确定 $\tilde{f}$ 的导函数,从而能确定 $\tilde{f}$ 本身,最多只相差一个积分常数,即起点的位置。规定起点的位置是 $0$ 后,这个提升映射就是唯一的了。}。这样,我们就可以通过唯一的提升,把道路 $f$ 都表示为 $\tilde{f}$。这样做的好处是,$\tilde{f}$ 的图像容易画出来。

任意回路 $f:I\rightarrow S^1$ 表示为 $\tilde{f}:I\rightarrow\mathbb{R}$ 后,必然是 $I\times\mathbb{R}$ 平面上,从点 $(0, 0)$ 出发,结束于 $(1, f(1))$ 的一段连续函数,其中 $f(1)$ 是一个整数\footnote{因为 $f$ 要构成回路,$f=p\cdot\tilde{f}$,而只有整数能被 $p$ 映射到 $S^1$ 的基点上。}。在这个例子的情况下,函数 $g$ 和 $f$ 保基点同伦当且仅当 $\tilde{g}$ 的图像也是从点 $(0, 0)$ 出发,结束于 $(1, f(1))$ 的连续函数。

有了以上约定,我们就可以把 $S^1$ 中的道路积表示如图。

\begin{figure}[ht]
\centering
\includegraphics[width=8cm]{./figures/87993e3fa6160f9e.pdf}
\caption{$S^1$ 上的回路 $f_1$ 和 $f_2$,以及它们的回路积 $f_1*f_2$。} \label{fig_HomT5_1}
\end{figure}

记按照以上约定所得到的 $\tilde{f}(1)$ 为回路 $f$ 的\textbf{度数},记为 $\opn{deg}{f}$。由\autoref{fig_HomT5_1} 易知,保基点同伦的回路都有相同的度数,并且 $\opn{deg}{f_1*f_2}=\opn{deg}{f_1}+\opn{deg}{f_2}$。

因此,容易得出一维球面的基本群:$\pi_1(S^1)=\mathbb{Z}$。


\end{example}

例子中的讨论是较为严谨的思路,实际上可以直观地把 $\pi_1(S^1)$ 中的各元素(回路类)看成是顺时针或逆时针绕过整个圆周 $n$ 周后回到基点的回路所构成的类,其中 $n$ 是一个整数,因此基本群同构于整数加群。比如说,如果选定逆时针为正方向,那么顺时针旋转 $2$ 圈后回到基点的回路类就对应于 $-2$。


\begin{example}{和 $S^1$ 有关的空间中的基本群}
由于 $\pi_1(S^1)=\mathbb{Z}$,结合本节所述定理,我们可以轻松计算如下空间的基本群:
\begin{itemize}
\item 甜甜圈空间可以表示为 $S^1\times S^1$,因此其基本群为 $\pi_1(S^1\times S^1)=\mathbb{Z}\times\mathbb{Z}$。
\item 高维甜甜圈 $S^1\times\cdots\times S^1$ 的基本群是 $\mathbb{Z}\times\cdots\times\mathbb{Z}$。
\item 二阶圈图 $S^1\vee S^1$(\autoref{def_Topo9_1}~\upref{Topo9})的基本群是 $\mathbb{Z}*\mathbb{Z}$。
\item 高阶圈图 $S^1\vee\cdots\vee S^1$ 的基本群是 $\mathbb{Z}*\cdots*\mathbb{Z}$。
\end{itemize}

\end{example}

\begin{example}{莫比乌斯带的基本群}\label{ex_HomT5_1}
莫比乌斯带 $M$ 可以通过强形变收缩(\autoref{def_HomT2_1}~\upref{HomT2})来同伦于其中轴线 $S^1$,因此莫比乌斯带的基本群和 $S^1$ 一样,都是 $\mathbb{Z}$。

理解莫比乌斯带基本群的难点是看出中轴线是其强形变收缩核,为了方便描述如何构造对应的强形变收缩,我们把莫比乌斯带看成射影平面挖去中心的结果,如%\autoref{fig_HomT5_9}
所示。

% \begin{figure}[ht]
% \centering
% \includegraphics[width=10cm]{./figures/Topo7_9.pdf}
% \caption{从射影平面中间挖去一个洞得到莫比乌斯带示意图。详见\textbf{商拓扑}\upref{Topo7}词条。} \label{fig_HomT5_9}
% \end{figure}

由于射影平面本身可以看成一个圆盘 $B^2$ 的商拓扑空间,我们同样可以把 $M$ 看成圆盘 $B^2$ 挖去中心后再取商拓扑,也就是看成圆环的商拓扑空间。不失一般性地,把这个圆环看成半径为 $2$ 的圆和半径为 $1$ 的圆之间的部分。取映射 $f:M\times I\rightarrow M$,其中对于任意 $(x, y)\in M, t\in I$,都有 $f((x, y), t)= (1+t/(1-\sqrt{x^2+y^2}))(x, y)$,则 $f$ 是一个强形变收缩,其收缩核就是圆环的外边缘;取商拓扑可见,这个外边缘正是莫比乌斯带的中轴线。
\end{example}

\begin{exercise}{莫比乌斯带的强形变收缩}
验证\autoref{ex_HomT5_1} 中的 $f$ 是同伦,进而证明它确实是强形变收缩。提示:先考虑未将圆环外边缘的对径点粘合时的情况,再对比考虑粘合后的效果。
\end{exercise}

\begin{exercise}{}
将莫比乌斯带看成一个射影平面挖去中心后的空间。如果将沿着外边缘走了半圈的回路类\footnote{走了半圈就回到了起点,因为对径点粘在一起了。}记为 $a\in\pi_1(M)$,那么绕着中心缺口顺时针旋转一周的回路类是 $\pi_1(M)$ 中的哪个元素?

答案是 $a^2$。
\end{exercise}



