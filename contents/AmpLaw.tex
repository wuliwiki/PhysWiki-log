% 安培环路定律(静磁学)
% keys 安培环路定理|毕奥—萨伐尔
% license Xiao
% type Tutor

\pentry{斯托克斯定理\nref{nod_Stokes}}{nod_9ea5}

\footnote{参考 \cite{GriffE} 以及 Wikipedia \href{https://en.wikipedia.org/wiki/Amp\%C3\%A8re's_circuital_law}{相关页面}。}静磁学问题% (链接未完成)
中, 在空间中选取一环路(称为\textbf{安培环路}) 并定义一个正方向, 那么磁感应强度在该环路上的线积分\upref{IntL}等于穿过环路的总电流(电流的正方向由右手定则\upref{RHRul} 判断)乘以真空中的磁导率 $\mu_0$。
\begin{equation}\label{eq_AmpLaw_1}
\oint \bvec B \vdot \dd{\bvec r} = \mu_0 I~.
\end{equation}
这就是\textbf{安培环路定律(Ampere's circuital law)}。 在许多中文教材中, 它常被写作 “安培环路定理”, 但由于它是经典电动力学的基本假设之一, 所以应该叫做定律。另外, law 的翻译也的确是定律而不是定理(theorem)。

\textbf{微分形式}:根据斯托克斯公式(\autoref{eq_Stokes_1}~\upref{Stokes}), 可以把\autoref{eq_AmpLaw_1} (积分形式)记为微分形式
\begin{equation}\label{eq_AmpLaw_2}
\curl \bvec B = \mu_0 \bvec j~,
\end{equation}
微分形式和积分形式是完全等价的。

静磁学问题的要求: 第一, 电流密度\upref{Idens}的空间分布不随时间变化。 这点可以从 “不存在瞬时作用” 理解, 假设某时刻电流突然从 0 变为某个值, 由于电磁场传播需要一定时间, 环路上不可能瞬间出现磁场。 第二, 空间中不能有变化的电场, 因为变化的电场也会产生涡旋磁场\upref{DisCur}, 改变环路积分的结果。 安培环路定律可以由位移电流拓展为 “广义安培环路定律\upref{DisCur}”, 以支持非静磁学的情况。

\begin{example}{无限长直导线的磁场}\label{ex_AmpLaw_1}
\begin{figure}[ht]
\centering
\includegraphics[width=10cm]{./figures/8244446845264e86.pdf}
\caption{无限长直导线的磁场} \label{fig_AmpLaw_2}
\end{figure}

\begin{figure}[ht]
\centering
\includegraphics[width=8cm]{./figures/a8d330f7ec6534aa.pdf}
\caption{上图的一个截面。可见,磁场围绕电流“打转”} \label{fig_AmpLaw1}
\end{figure}

若导线的电流为 $I$, 在其周围作一个半径为 $r$ 的安培环路, 由对称性, 环路上任意一点的磁感应强度大小相同且沿正方向。 所以\autoref{eq_AmpLaw_1} 等于
\begin{equation}
2\pi r B = \mu_0 I~,
\end{equation}
所以磁感应强度大小的分布为
\begin{equation}
B(r) = \frac{\mu_0}{2\pi} \frac Ir~.
\end{equation}
这与使用毕奥—萨伐尔定律(\autoref{ex_BioSav_1}~\upref{BioSav}) 得出的结论一致。
\end{example}

\begin{example}{无限长螺线管的磁场}\label{ex_AmpLaw_2}
\begin{figure}[ht]
\centering
\includegraphics[width=5cm]{./figures/9aa7dd17d3b2decc.pdf}
\caption{无限长螺线管} \label{fig_AmpLaw_1}
\end{figure}
圆柱形均匀缠绕的螺线管单位长度匝数为 $n$ 沿螺线管的轴线方向取一个长方形回路, 根据对称性, 垂直于轴线的方向不会有任何磁场。 所以现在螺线管外面的部分可以任意伸缩, 如果伸到无穷远, 则磁场为零。 所以环路积分完全由内部的平行边贡献
\begin{equation}
BL = \mu_0 I_{tot} = \mu_0 nLI~.
\end{equation}
所以外部磁场为零, 内部磁场为匀强。
\begin{equation}
B = \mu_0 nI~.
\end{equation}
在实际情况中, 如果螺线管比较细长, 那我们仍然可以近似认为它的内部为匀强磁场。
\end{example}

\subsection{安培定理与毕奥—萨伐尔定律}
\pentry{毕奥—萨伐尔定律\nref{nod_BioSav}, 旋度的逆运算\nref{nod_HlmPr2}}{nod_dbb8}

在静磁学条件下, 若已知\textbf{电流密度分布}, 毕奥—萨伐尔定律\upref{BioSav}可以让我们通过空间中的电流直接计算处任意位置的磁场 $\bvec B$, 可以证明该磁场对任意曲面满足安培环路定律(\autoref{eq_AmpLaw_1} 和\autoref{eq_AmpLaw_2} )。 反之可以证明, 符合安培环路定律的磁场 $\bvec B'$ 不是唯一的, 它可以等于毕奥—萨伐尔定律算出的磁场 $\bvec B$ 叠加一个任意无旋磁场 $\bvec h$。 而由磁场的高斯定律\upref{MagGau}, $\bvec h$ 只能是调和场\upref{HarmF}。 所以严格来说安培定理与毕奥—萨伐尔定律并不完全等价, 前者还要加上一个适当的边界条件(例如无穷远处磁场为零), 才能使 $\bvec h$ 恒为零。
