% 南京理工大学 2009 年 研究生入学考试试题 普通物理(B)
% license Usr
% type Note

\textbf{声明}:“该内容来源于网络公开资料,不保证真实性,如有侵权请联系管理员”

\subsection{一、填空题(每题 2 分,共 30 分)}
1. 一半径为 $a$ 的金属球带电 $Q$,其周围充满介电常数为$\xi$ 的均匀无限大介质,则金属球内部的电场能量为__________,金属球外内部的电场能量为______________。

2. 一无限长载流直线 $I$ 与一载流矩形回路共面,其尺寸如图所示,则载流线圈受到的力矩大小为___________;电流 $I$ 激发的磁场通过回路的磁通量________。

3. 互感系数的物理意义是________。

4. 油轮漏油$(n=1.2)$入海,在海面上形成一大片油膜,如有人在膜厚为 480nm的油膜上空的飞机上垂直往下看,能看到 $\lambda=$_________$nm$ 的光;如有人在 45 度方向往油膜看,又能看到 $\lambda=$___________$nm$ 的光。(海水的折射率为 1.33)

5. 一部分偏振光由线偏振光和自然光组成,让该部分偏振光经过一可旋转的线偏振片后,得到 $Imax/Imin=5/2$,则该部分偏振光中线偏振光的比例为________;如用自然光通过,则 $Imax/Imin=$______________。

6. 宽度为 $a$ 的一维无限深势阱中,粒子的波函数为:$\psi_\pi(x)=\sqrt{\frac{2}{a}}\sin\frac{n \pi}{a}x$ ,则该粒子在势阱中出现的几率密度表达式为 $P=$______________,若 $n=2$ 时,粒子在$x=$________________处出现的概率最大。

7. 电子的静止质量是 $m_0$,当电子以 $v=0.8c$ 的速度运动时,它的运动动能为____________,总能量为________________。

8. 在氢原子光谱的莱曼系$(n=1)$中,最短波长为_______$nm$,最长波长为_______$nm$。
\subsection{二、填空题(每空 2 分,共 30 分)}
1. 一质点作直线运动,运动方程为 $x=3+3t2-t3(t>0)$(SI 制),则该质点 $t=1s$
时,$v=$___________;在 $t=$_______时,质点开始作减速直线运动。

2. 有一孤立球形天体绕过球心的自转轴转动,其初始转动惯量为 $I0$,初始转动到那个能为 $Ek0$。若干年后由于自身收缩致使其转动惯量减少为 $I0/2$,则此刻其自转角速度大小为_____________,其转动动能的改变量$\Delta Ek=$___________。

3. 一质量为 $m$、长为 $l$ 的均匀直尺,立于桌面 $O$ 点,直尺可绕 $O$ 点转动。该系统对 $O$ 轴的转动惯量 $I=$__________,直尺在竖直位置由静止开始转动,直尺刚到地之前的角速度 大小为___________,此时直尺的角加速度 $\beta$ 大小为_____________。

4. 某种理想气体的密度为 $\rho$,摩尔质量为 $\mu$,其最概然(最可几)速率为 $v\rho$,则该气体的分子数密度 $n=$_________,其压强 $P=$______________。

5. t=27°C时,$1mol$ 氧气分子的内能为______________;t=27°C时,$1mol$ 氧气分子的平均总动能为_____________。

6. 一倔强系数为 $k$ 的轻质弹簧,一端系以质量为 $m$ 的小球,此系统振动时,振动频率是_________。若将弹簧截成相同的两端,取其一段系住小球$m$,则系统的振动频率是_________。

7. 一半径为 $R$ 的导体球的球心为 $O$ 点,电量为 $q$,在距 $O$ 点 $2R$ 处的电场强度$E=$___________,在距 $O$ 点 $R/2$ 处的电势 $U=$_______。
\subsection{三、(12 分)}
水平桌面上有一半径为 $R$、质量为 $m$ 的均质圆盘,圆盘与桌面间摩擦系数为 $\mu$,圆盘可绕通过中心且垂直于桌面的轴转动。若刚开始转动时,圆盘转动角速度大小为$\omega_0$ ,求:(1)转动时圆盘收到的摩擦力矩;(2)圆盘转动时的角加速度;(3)圆盘停止转动前共转了多少圈?
\subsection{四、(12 分)}
推导绝热指数为 $ \gamma $ 的理想气体的绝热过程的一个过程方程。
\subsection{五、(10 分)}
同一媒质中的两波源 $A,B$,相距为 $AB=30m$,它们的振幅相同,频率都是 $100Hz$,相位差为 $\pi$,波速为 $400m/s$,试求 $A,B$ 连线上因干涉而静止的各点的位置。
\subsection{六、(10 分)}
一厚度为 $2a$ 的无限大平板内,均匀地分布着正电荷,体密度为$\rho$,试求平板层内外的电场强度的分布。
\subsection{七、(12 分)}
无限大螺线管半径为 $R$,管内磁场的变化率 $dB/dt=k(k>0)$,导体棒 $ab=bc=ca$,试求:(1)导体棒中感应电动势的大小和方向;(2)$a$ 点的感生电场的大小和方向。
\subsection{八、(12 分)}
已知一光栅的光栅常数 $d=6.0\times10-4cm$,当用波长为 $600nm$ 的单色光垂直垂直照射在光栅上时,发现第 4 级缺级(第一个缺级),透镜的焦距$f=1m$,求:(1)光栅上狭缝的宽度 $a$;(2)屏幕上所呈现的全部明纹的级数和条数。
\subsection{九、(12 分)}
一电子用静电场加速后,其动能为 $1.53MeV$,求:(1)该电子的德布罗意波长;(2)该电子通过直径为 $1nm$ 的圆孔的第一暗环的衍射角大小。
\subsection{十、(10 分)}
一根长直圆筒形(内外半径分别为 $R1,R2$)导线均匀载有电流$I$,求:(1)磁感应强度的分布;(2)这导线内部的磁场能量密度