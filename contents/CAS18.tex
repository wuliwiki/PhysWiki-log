% 中国科学院 2018 年考研普通物理
% keys 中国科学院|考研|普通物理

\begin{issues}
\issueTODO
\begin{enumerate}
\item 试题内容未补全
\item 未画图
\end{enumerate}

\end{issues}


\subsection{选择题}

\begin{enumerate}

\item 某时刻的弦波如图所示,在此弦段中,振动动能最大的部位为\\

A. A处$\quad$
B. B处$\quad$
C. C处$\quad$
C. A和C处$\quad$
\begin{figure}[ht]
\centering
\includegraphics[width=5cm]{./figures/a7c98e7c5840c321.png}
\caption{选择题1图示} \label{fig_CAS18_1}
\end{figure}
\item 以下关于质点系描述正确的是\\

A. 质点系质心运动只与只与质点系所受合内力有关,质点系的内力不可能改变质点系的总动能。\\
B. 质点系质点运动只与质点系所受合外力有关,质点系的内力可以改变质点系的总动能。\\
C. 质点系质点运动只与质点所受合内力有关,质点系的内力可以改变质点系的总动能。\\
D. 质点系质点运动只与质点系所受合外力有关,质点系的内力不可以改变质点系的总动能。\\

\item 

\item 

\item 

\item 关于磁化电流与传导电流,下面说法不正确的是\\

A。 磁化电流是大量分子电流统计平均的宏观效果,传导电流是电荷迁移的结果\\
B. 磁化电流和传导电流都能产生磁场\\
C. 磁化电流和传导电流都能产生焦耳热\\
C. 磁化电流产生的磁场服从安培环路定律\\

\item 一电路如下图所示,两电阻大小均为 $R$ ,电感 $L$ ,电源电动势 $\epsilon$ 开关 $S$ 闭合后电感 $L$ 上的电流 $i$ 随时间的变化关系为\\

A. $i = \frac{\epsilon}{2R}$\\
B. $i = \frac{\epsilon}{R}e^{-\frac{2R}{L}t}$\\
C. $i = \frac{\epsilon}{2R}e^{-\frac{R}{L}t}$\\
C. $i = \frac{\epsilon}{2R}(1-e^{-\frac{R}{L}t})$\\
\begin{figure}[ht]
\centering
\includegraphics[width=5cm]{./figures/89068eaf9fad3fd0.png}
\caption{选择题7图示} \label{fig_CAS18_2}
\end{figure}
\item 原子态 $^{1}D_{1}$ 的能级在磁感应强度 $\bvec{B}$ 的弱磁场中分裂成多少子能级?\\

A. 3个 $\quad$
B. 2个 $\quad$
C. 5个 $\quad$
C. 4个 $\quad$
\end{enumerate}

\subsection{简答题}

\begin{enumerate}

\item 所谓的二体问题,是指两个指点只有相互作用力,不受外力,对于二体问题,试推导一个质点相对于另一个质点的运动学方程。\\
对比 $\bvec{F} = \mu \bvec{a}$ 形式,请写出 $\mu$ 的表达式( $\mu$ 代表二体约化质量)

\item 圆盘上有一圈带正电的带点球,问中间的螺线管断电时,圆盘是否会转动,如何转,为什么。

\item 牛顿环,凸透镜和平面介质的折射率分别为 $n_{1}$ $n_{3}$ ,中间为空气,此时牛顿环中心为暗纹。若在中间注入一种液体,折射率在凸透镜和平面介质中间。$n_{1}>n_{2}>n_{3}$ ,此时中心为什么纹,说明原因。

\end{enumerate}

\subsection{ }
太阳系某一小行星的抛物线轨道方程可表述为 $y^{2} = 4Cx$ ,太阳位于焦点 $x = C$ ,$y = 0$ 处,将太阳质量记为 $M$ ,求:\\
(1) 小行星在抛物线顶点处的速率;\\
(2) 小行星在抛物线顶点处的曲率半径;\\
(3) 小行星在抛物线顶点处的角动量与其质量的比值。\\

\subsection{ }
狐狸沿半径为 $R$ 的圆轨道以速率 $v$ 奔跑,在狐狸出发的同时,猎犬从圆心出发,以相同的速率追击狐狸。在追击过程中,圆心、狐狸和猎犬始终连成一条直线,以圆心 $O$ 为原点,从 $O$ 点到狐狸初始位置连线为极轴,建立极坐标系。\\
(1) 导出猎犬速度矢量与径向位置 $r$ 的关系;\\
(2) 导出猎犬加速度矢量与径向位置 $r$ 的关系;\\
(3) 确定猎犬的轨道方程。

\subsection{ }
如图所示,电子感应加速器是利用变化的磁场 $\bvec{B}$ 所产生的感应电场(漩涡电场) $\bvec{E}$ 来加速电子,同时磁场本身又维持电子在固定的圆轨道上运动,设开始时,磁感应强度 $B$ 为0,电子的初速度为0;变化的磁场 $\bvec{B}$ 经过一段时间加速电子后,电子轨道处的的磁感应强度大小为 $B_{0}$ ,电子轨道内的平均磁感应强度大小为 $\bvec{B} = $