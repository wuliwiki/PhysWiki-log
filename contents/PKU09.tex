% 北京大学 2009 年 考研 量子力学
% license Usr
% type Note

\textbf{声明}:“该内容来源于网络公开资料,不保证真实性,如有侵权请联系管理员”


1.两个非全同粒子在一维谐振子势中的波函数、能级。知道 $t = 0$ 时的初态,求 $t$ 时刻处于最大对称的概率。 (**表示记不清楚)

2.一维方势阱, $0 < x < a$ 处 $V(x) = V_0$,其余地方 $V(x) = 0$。一粒子从 $x > a$ 区域向左碰去,求透射的概率。


3.①在 $L_z$ 表象中求 $L_x (L = 1)$ 的本征值、本征态。
②在 $L_z = 1$ 的态下求 $L_x = 0$ 或 $1$ 的概率。

4. ① 某势阱,求基态的波函数数和能量。 ② 开始处于 $E = (1/2) h \omega$,求在 $H'$ 作用下,仍处于 $E = (5/2) h \omega$ 的概率。

5.一个立方体形状的势场。

6.氮原子在微扰作用 $H' = e \cdot z \cdot \delta(t)$ 作用下跃迁到各激发态的概率之和。