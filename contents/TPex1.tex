% 双指针法运用
% license Usr
% type Tutor

\pentry{双指针算法\nref{nod_TP}}{nod_56e7}

\begin{issues}
\issueDraft 请不要占用他人文章不作为
缺少一些图片解释,待有缘人吧,例2有时间再更
\end{issues}

\begin{example}{快慢指针——链表中点}
给定一个头结点为 \textsl{h} 的非空单链表,返回链表的中间结点。

如果有两个中间结点,则返回第二个中间结点。

\textsl{ex.1}

输入:[1,2,3,4,5] 

输出:此列表中的结点 3 (序列化形式:[3,4,5])
返回的结点值为 3 。

\textsl{ex.2}

输入:[1,2,3,4,5,6]

输出:此列表中的结点 4 (序列化形式:[4,5,6])
\end{example}

\subsection{解}
\subsubsection{方法一:}
数组

我们最常做的就是遍历法了,简单,暴力,将遍历到的元素依次放入数组 \textsl{A }中。如果我们遍历到了 \textsl{n }个元素,那么链表及数组的长度也为 \textsl{n},对应的中间节点即为 \textsl{A[n/2]}。

下面整理的为几种语言的部分代码:
\begin{lstlisting}[language=python]
def middlenode ():
    A = [h]
        while A[-1].next:
            A.append(A[-1].next)
    return A[len(A) // 2]

\end{lstlisting}
python如上
\begin{lstlisting}[language=cpp]
public:
    ListNode* middlenode(ListNode* head) 
    {
        vector<ListNode*> A = {h};
        while (A.back()->next != NULL)
            A.push_back(A.back()->next);
        return A[A.size() / 2];
    }
\end{lstlisting}
C++见上

时间复杂度:O(N),其中 N 是给定链表中的结点数目。

空间复杂度:O(N),即数组 A 用去的空间。
\subsubsection{方法二:}
单指针法

方法一可以进行空间优化,比如省去数组 A。

我们可以对链表进行两次遍历。第一次遍历时,我们统计链表中的元素个数 N;第二次遍历时,我们遍历到第 N/2 个元素(链表的首节点为第 0 个元素)时,返回该元素即可。

相当于,方法一我们拿刻度瓶装,方法二我们用数豆子的方法
\begin{lstlisting}[language=python]


def middlenode():
    n, a = 0, h
    while a:
        n += 1
        a = a.next #第一次
    b, a = 0, h
    while b < n // 2:
        b += 1
        a = a.next #第二次
    return a

\end{lstlisting}
python如上
\begin{lstlisting}[language=java]

    public ListNode middlenode() 
    {
        int n = 0;
        ListNode a = h;
        while (a != null) 
        {
            ++n;
            a = a.next;
        }
        int b = 0;
        a = h;
        while (b < n / 2) 
        {
            ++b;
            a = a.next;
        }
        return a;
    }

\end{lstlisting}
java如上

时间复杂度:O(N),其中 N 是给定链表的结点数目。

空间复杂度:O(1),常数空间存放变量和指针。

\subsubsection{方法三:}
快慢指针法

继续优化,用两个速度不同的指针 slow 与 fast 一起遍历链表。速度不同产生位移之差,于是,我们让慢爷爷一次走一步,快小伙一次走两步。不妨试试,你会发现,当fast 走到末尾时,slow 恰在中间。

所以:
\begin{lstlisting}[language=python]
def middlenode():
    slow = fast = h
    while fast and fast.next:
        slow = slow.next
        fast = fast.next.next
    return slow
\end{lstlisting}
python
\begin{lstlisting}[language=cpp]
public:
    ListNode* middlenode(ListNode* h) 
    {
        ListNode* slow = h;
        ListNode* fast = h;
        while (fast != NULL && fast->next != NULL) 
        {
            slow = slow->next;
            fast = fast->next->next;
        }
        return slow;
    }

\end{lstlisting}
c++
\begin{lstlisting}[language=java]
public ListNode middlenode() 
    {
        ListNode slow = h, fast = h;
        while (fast != null && fast.next != null) {
            slow = slow.next;
            fast = fast.next.next;
        }
        return slow;
    }

\end{lstlisting}
java
