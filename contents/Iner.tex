% 非惯性参考系、惯性力
% keys 非惯性系|惯性系|牛顿第二定律|惯性力
% license Xiao
% type Tutor

\pentry{牛顿第二定律\nref{nod_New3}, 加速度的坐标系变换\nref{nod_AccTra}}{nod_17cf}

\footnote{参考 Wikipedia \href{https://en.wikipedia.org/wiki/Fictitious_force}{相关页面}。}若一个质点 $m$ 某时刻在惯性系 $S$ 中的加速度为 $\bvec a_{S}$, 在另一个非惯性系 $S'$ 中的加速度为 $\bvec a_{S'}$, 则可假设质点受到一个\textbf{惯性力(inertial force 或 fictitious force)}
\begin{equation}\label{eq_Iner_1}
\bvec f = m( \bvec a_{S'} - \bvec a_{S} )~,
\end{equation}
使得\enref{牛顿运动定律}{New3}在非惯性系 $S'$ 中仍然成立。惯性力作为一个数学工具,既没有施力物体,也不是真实的力。另外,惯性力取决于参考系的选取甚至质点的运动,真实的受力不依赖于参考系。

\begin{example}{加速的电梯}\label{ex_Iner_1}
在向上加速的电梯中,电梯给人的支持力大于人的重力。这在地面参考系(惯性系)中的解释是,电梯给人的支持力除了要抵消人的重力,还要提供额外的向上的力使人产生向上的加速度。但是人从直觉上认为自己所处的是惯性系,符合牛顿运动定律。那么唯一合理的解释就是自己被施加了额外的向下的力。 为了使自己保持“静止” 或“受力平衡”,电梯会给人一个额外的支持力。
\end{example}

\begin{example}{转弯的车}
车向左转时人感觉到向右的“离心力”。同样,这一现象在地面参考系中解释,是车为了使人具有向左的向心加速度,给人一个向左的力,同时人对车施加一个向右的反作用力。然而,车中的人直觉上认为自己所处的是惯性系,符合牛顿运动定律,那么唯一合理的解释就是自己受到了向右的“离心力”。为了使人保持“静止” 或“受力平衡”,车必须给人一个向左的反作用力。
\end{example}

从这两个例子可以看出,如果要使牛顿运动定律在非惯性系中也成立,则需要假设一些力的存在,即惯性力。人的直觉总会假设自己的参考系是惯性系,这就解释了为什么日常生活中我们常说离心力却不说向心力。用离心力来描述现象并没有错,这只是从更符合直觉的非惯性系的角度来分析而已。

\subsection{平动非惯性系}
假设某个\textbf{非惯性系} $S'$ 相对于\textbf{惯性系} $S$ 没有旋转只有平移,且 $t$ 时刻的\textbf{相对加速度}为 $\bvec a(t)$。 这样, $S'$ 中的任何一个静止点相对于 $S$ 系的加速度都是 $\bvec a(t)$。 设 $S'$ 系中有一质点 $m$, 相对于 $S'$ 系的加速度为 $\bvec a_{S'} (t)$, 那么在惯性系中质点的加速度为
\begin{equation}\label{eq_Iner_2}
\bvec a_{S} = \bvec a_{S'} + \bvec a~,
\end{equation}
即\textbf{绝对加速度}\footnote{这里把绝对加速度定义为任意\textbf{惯性系}中的测得质点的加速度。由于不同惯性系之间的相对加速度为零,绝对加速度与惯性系的选择无关。}等于相对加速度加牵连加速度。

在惯性系 $S$ 中运用牛顿第二定律得质点的真实受力为(注意真实受力不随参考系变化)
\begin{equation}\label{eq_Iner_3}
\bvec F = m \bvec a_{S} = m(\bvec a + \bvec a_{S'} )~.
\end{equation}
非惯性系中的观察者如果假设自己的参考系中牛顿定律仍然成立,并假设存在惯性力 $\bvec f$, 对于某个质点有
\begin{equation}\label{eq_Iner_4}
\bvec F + \bvec f = m \bvec a_{S'}~,
\end{equation}
所以
\begin{equation}\label{eq_Iner_5}
\bvec f = m \bvec a_{S'} - \bvec F = m(\bvec a_{S'} - \bvec a_{S}) =  - m\bvec a~.
\end{equation}
这说明,平动非惯性系中任何一个物体受到的惯性力大小与质量和非惯性系的加速度的乘积成正比,方向与相对加速度方向相反。

在\autoref{ex_Iner_1} 中, 电梯的参考系就是一个平动的非惯性系。 如果人站在一个弹簧秤上, 秤的示数将等于人的体重加上惯性力 $ma$(电梯向上加速时 $a$ 取正值)。 这用惯性力来解释,就是静止的人受到与电梯加速度方向相反的惯性力,大小等于 $ma$(注意在电梯参考系中人没有加速度,所以是“受力平衡”的)。

\subsection{一般非惯性系}
非平动参考系相对于惯性系除了平移运动还可能做相对旋转运动。 对于非平动参考系, \autoref{eq_Iner_2} 并不满足(即使把 $\bvec a$ 看做位置和时间的函数也不行), 以后在“\enref{科里奥利力}{Corio}” 中我们会推导 $\bvec a_{S}$ 和 $\bvec a_{S'}$ 间的具体关系。 所以以上对任何非惯性系都成立的结论只有
\begin{equation}
\bvec F = m \bvec a_{S}~,
\end{equation}
\begin{equation}
\bvec F + \bvec f = m \bvec a_{S'}~,
\end{equation}
得
\begin{equation}
\bvec f = m( \bvec a_{S'} - \bvec a_{S})~.
\end{equation}
所以在最一般的情况下,\textbf{质点的惯性力等于质量乘以它在两参考系中的加速度之差}。

\subsubsection{计算惯性力的坐标法}
\begin{enumerate}
\item 求出所选惯性系和非惯性系之间的坐标变换和逆变换关系(具体表达式可以用\enref{罗德里格旋转公式}{RotA}或者\enref{四元数}{QuatN}得到,如果问题是二维的,直接用\enref{平面旋转变换}{Rot2DT}即可)
\begin{equation}
\leftgroup{
x &= f_1( a,b,c,t )\\
y &= f_2( a,b,c,t )\\
z &= f_3( a,b,c,t )}~,
\qquad
\leftgroup{
a &= g_1( x,y,z,t )\\
b &= g_2( x,y,z,t )\\
c &= g_3( x,y,z,t )}~.
\end{equation}
\item 对 $x,y,z$ 求时间的二阶\enref{全导数}{TotDer},得出惯性系中质点的加速度 $\bvec a_{S}$。
\begin{equation}
\leftgroup{
\ddot x &= \dv[2]{t} f_1( a,b,c,t )\\
\ddot y &= \dv[2]{t} f_2( a,b,c,t )\\
\ddot z &= \dv[2]{t} f_3( a,b,c,t )}~
\end{equation}

\item 把 $S$ 系中的加速度矢量\footnote{上标 $T$ 表示\enref{矩阵转置}{Mat},将行矢量变为列矢量。这是为了排版方便。} $\pmat{\ddot x &\ddot y &\ddot z}\Tr$ 用 $g_1, g_2, g_3$ 逆变换到 $S'$ 系中表示,
\begin{equation}\label{eq_Iner_6}
\pmat{\ddot x\\ \ddot y\\ \ddot z}_{S'} = \pmat{ g_1(\ddot x,\ddot y,\ddot z, t) \\ g_2(\ddot x,\ddot y,\ddot z, t) \\ g_3(\ddot x,\ddot y,\ddot z, t) } -
\pmat{ g_1(0,0,0,t)\\g_2(0,0,0,t)\\g_3(0,0,0,t) }~.
\end{equation}

\item 质点受到的惯性力为
\begin{equation}\label{eq_Iner_9}
\bvec f = m ( \bvec a_{S'} - \bvec a_{S} ) = 
m\pmat{ \ddot a\\ \ddot b\\ \ddot c} - m\pmat{\ddot x\\ \ddot y\\ \ddot z}_{S'}~
\end{equation}
\end{enumerate}
注意\autoref{eq_Iner_6} 中的坐标变换是必须的,因为 $\pmat{\ddot a & \ddot b & \ddot c}\Tr$ 是 $S'$ 系中的矢量坐标, $\pmat{\ddot x & \ddot y & \ddot z}\Tr$ 是 $S$ 系中的矢量坐标,直接相减没有几何意义。必须把 $\pmat{\ddot x & \ddot y & \ddot z}\Tr$ 变换到 $S'$ 系中表示,或者把 $\pmat{\ddot a & \ddot b & \ddot c}\Tr$ 变换到 $S$ 系中表示才可相减。由于惯性力是非惯性系中的假想力,所以第一种方式比较方便。
