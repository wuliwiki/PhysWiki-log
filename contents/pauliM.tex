% 泡利矩阵
% keys 泡利矩阵|自旋算符
% license Xiao
% type Tutor
\pentry{幺正变换\upref{Unitar},爱因斯坦求和约定\upref{EinSum}}
\begin{definition}{泡利矩阵}\label{def_pauliM_1}
泡利矩阵为 $2\times 2$ 幺正厄米矩阵:
\begin{equation}
\sigma_1 = \begin{pmatrix}
0 & 1\\
1 & 0
\end{pmatrix},\quad 
\sigma_2 = \begin{pmatrix}
0 & -i\\
i & 0
\end{pmatrix},\quad 
\sigma_3 = 
\begin{pmatrix}
1 & 0\\
0 & -1
\end{pmatrix}
~.
\end{equation}
有时也定义零号泡利矩阵为单位矩阵:
\begin{equation}
\sigma_0 = \pmat{1&0\\0&1}~.
\end{equation}

\end{definition}

\subsection{泡利矩阵的性质}
泡利矩阵是 $2\times 2$ 幺正厄米矩阵,它们有以下的性质
\begin{equation}
\sigma_i\cdot \sigma_i = \begin{pmatrix}1&0\\0&1\end{pmatrix},\quad \sigma_i = {\sigma_i}^\dagger~.
\end{equation}
即每个泡利矩阵的平方都是单位矩阵。又可以得到 $\sigma_i \cdot  \sigma_i{}^\dagger = I$,所以泡利矩阵也是幺正矩阵。

泡利矩阵有以下性质:
\begin{theorem}{}
\begin{equation}\label{eq_pauliM_1}
\sigma_a\sigma_b = \delta_{ab}\sigma_0 + \mathrm{i}\epsilon_{abc}\sigma_c~.
\end{equation}
\end{theorem}
由这条性质可以推出
\begin{lemma}{}
\begin{equation}
[\sigma_a,\sigma_b] = \mathrm{i}2\epsilon_{abc}\sigma_c,\quad \{\sigma_a,\sigma_b\}=2\delta_{ab}\sigma^0~.
\end{equation}
\end{lemma}
对 \autoref{eq_pauliM_1} 两边求迹,还可以得到
\begin{lemma}{}
\begin{equation}
\mathrm{tr}[\sigma_a \sigma_b]= 2\delta_{ab} ~
\end{equation}
\end{lemma}
