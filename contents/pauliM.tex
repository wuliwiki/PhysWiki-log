% 泡利矩阵
% keys 泡利矩阵|自旋算符
% license Xiao
% type Tutor



\pentry{幺正变换\nref{nod_Unitar},爱因斯坦求和约定\nref{nod_EinSum}}{nod_fdf8}

\begin{definition}{泡利矩阵}\label{def_pauliM_1}
泡利矩阵$\hat{\sigma}_i$为 $2\times 2$ 幺正厄米矩阵,与自旋矩阵的关系为:
\begin{equation}
S_i=\frac{\hbar}{2}\hat{\sigma}_i~.
\end{equation}
\end{definition}
在$\hat{\sigma}_z$表象下,泡利矩阵的一般形式为:
\begin{equation}
\hat{\sigma}_1 = \begin{pmatrix}
0 & 1\\
1 & 0
\end{pmatrix},\quad 
\hat{\sigma}_2 = \begin{pmatrix}
0 & -i\\
i & 0
\end{pmatrix},\quad 
\hat{\sigma}_3 = 
\begin{pmatrix}
1 & 0\\
0 & -1
\end{pmatrix}
~.
\end{equation}
有时也定义零号泡利矩阵为单位矩阵:
\begin{equation}
\hat{\sigma}_0 = I=\pmat{1&0\\0&1}~.
\end{equation}

根据定义,我们可以直接得到泡利矩阵的对易关系:
\begin{equation}
[\hat{\sigma}_i,\hat{\sigma}_j]=2\mathrm {i}\epsilon ^{\,\,\, k}_{ij}\hat{\sigma}_k,\quad\{\hat{\sigma}_i,\hat{\sigma}_j\}=2\delta_{ij}I~,
\end{equation}
其中$i,j,k\in\{1,2,3\}$。$\delta_{ij}$是克罗内克函数,$\epsilon ^{\,\,\, k}_{ij}$是列维-奇维塔符号(Levi-Civita symbol),由对易运算的反对称性可知,这是一个关于下标的反对称张量。

\subsection{泡利矩阵的其他性质}
由泡利矩阵的一般形式得:
\begin{equation}
\hat{\sigma}_i\cdot \hat{\sigma}_i = \begin{pmatrix}1&0\\0&1\end{pmatrix}~,
\end{equation}
即每个泡利矩阵的平方都是单位矩阵,则 $\hat{\sigma}_i \cdot  \hat{\sigma}_i{}^\dagger = I$,所以泡利矩阵也是幺正矩阵。
\begin{exercise}{}
证明在别的角动量分量表象下,我们依然有:$\hat{\sigma}_i^2=I$
\end{exercise}

由泡利矩阵的对易关系可得:
\begin{theorem}{}
\begin{equation}\label{eq_pauliM_1}
\hat{\sigma}_i\hat{\sigma}_j = \delta_{ij}I + \epsilon ^{\,\,\, k}_{ij}\hat{\sigma}_k~.
\end{equation}
\end{theorem}
Proof.

$i=j$时,显然成立。
$i\neq j$时,利用$\epsilon ^{\,\,\, k}_{ij}=-\epsilon ^{\,\,\, k}_{ji}$及对易关系得证。


\begin{exercise}{}
证明:
$\mathrm{tr}[\hat{\sigma}_i \hat{\sigma}_j]= 2\delta_{ij}$。
\end{exercise}

每个泡利矩阵都有 $\pm 1$ 两个本征矢量,归一化后为:
\begin{equation}
\begin{aligned}
\phi_{x+} &= \frac{\sqrt 2}{2}\pmat{1\\1}, &\phi_{x-} &= \frac{\sqrt 2}{2} \pmat{1\\-1}~;\\
\phi_{y+} &= \frac{\sqrt 2}{2}\pmat{1 \\ i}, &\phi_{y-} &= \frac{\sqrt 2}{2} \pmat{1 \\ -i} ~;\\
\phi_{z+} &= \pmat{1 \\ 0}, &\phi_{z-} &= \pmat{0 \\ 1}~.
\end{aligned}
\end{equation}

\subsection{泡利矩阵与四元数}
特殊酉群$SU(2)$可以看作\textbf{实数域}上的四维线性空间,可以验证泡利矩阵为该线性空间的一组基——通过泡利矩阵的任意线性组合,我们可以得到该线性空间的任意元素,也即$SU(2)$上的任意元素。四元数与$SU(2)$是同构的,一般同构方式为以下两种:
\begin{equation}
\begin{aligned}
1\rightarrow I,\,\bvec{\mathrm i}\rightarrow \mathrm i \hat{\sigma}_3,\,\bvec j\rightarrow \mathrm i\hat{\sigma}_2,\,\bvec k\rightarrow \mathrm i\hat{\sigma}_1.\\
1\rightarrow I,\,\bvec{\mathrm i}\rightarrow -\mathrm i\hat{\sigma}_1,\,\bvec j\rightarrow -\mathrm i\hat{\sigma}_2,\,\bvec k\rightarrow -\mathrm i\hat{\sigma}_3~.
\end{aligned}
\end{equation}

\begin{exercise}{}
验证泡利矩阵保四元数乘法。
\end{exercise}