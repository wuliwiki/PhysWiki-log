% 天津大学 2015 年考研量子力学答案
% keys 考研|天津大学|量子力学|2015|答案
% license Xiao
% type Tutor

\begin{issues}
\issueDraft
\issueTODO
\end{issues}

\subsection{ }
由归一化条件 $\displaystyle (\sqrt{\frac{1}{3}}A)^{2} + (\sqrt{\frac{2}{3}}A)^{2} = 1 $ 可得到 $A=1$。故归一化函数为:\\

$\displaystyle \psi(x) = \sqrt{\frac{1}{3}}\phi_{210}(x)+\sqrt{\frac{2}{3}} \phi_{310}(x)$ \\

\subsection{ }
\begin{enumerate}
\item 由题可得:
\begin{equation}
\begin{aligned}
\left[L^{2},L_{x}\right] =& [ L^{2}_{x} + L^{2}_{y} + L^{2}_{z} , L_{x}] \\
=& 0 + [L^{2}_{y},L_{x}] + [L^{2}_{z},L_{x}] \\
=& 0 - L_y[L_y,L_x]+[L_y,L_x]L_y \\
=& 0 - i\hbar L_y L_z - i\hbar L_z L_y + i\hbar L_z L_y + i\hbar L_y L_z \\
=& 0~.
\end{aligned}
\end{equation}
同理可得:
\begin{equation}
\begin{aligned}
\left[\hat{L}_+,\hat{L}_z\right]Y_{lm}(\theta ,\phi) =& \left[\hat{L}_{x}+i\hat{L}_{y} ,\hat{L}_z \right]Y_{lm}(\theta ,\phi) \\
=& \left[\hat{L}_x ,\hat{L}_z \right]Y_{lm}(\theta ,\phi) + i\left[\hat{L}_x ,\hat{L}_z \right]Y_{lm}(\theta ,\phi) \\
=& -\hbar \hat{L}_{+}Y_{lm}(\theta ,\phi)~,
\end{aligned}
\end{equation}

\begin{equation}
\begin{aligned}
\left[\hat{L}_{-},\hat{L}_{z}\right]Y_{lm}(\theta ,\phi) =& \left[\hat{L}_{x}-i\hat{L}_{y} ,\hat{L}_z \right]Y_{lm}(\theta ,\phi) \\
=& \left[\hat{L}_x ,\hat{L}_z \right]Y_{lm}(\theta ,\phi) - i\left[\hat{L}_x ,\hat{L}_z \right]Y_{lm}(\theta ,\phi) \\
=& \hbar \hat{L}_{-}Y_{lm}(\theta ,\phi)~,
\end{aligned}
\end{equation}

\begin{equation}
\begin{aligned}
\left[\hat{L}_{+},\hat{L}_{-}\right]Y_{lm}(\theta ,\phi) =& \left[\hat{L}_{x} + i\hat{L}_{y} ,\hat{L}_{x}-i\hat{L}_{y} \right]Y_{lm}(\theta ,\phi) \\
=& 2\hbar \hat{L}_{z}Y_{lm}(\theta ,\phi)~.
\end{aligned}
\end{equation}

\item 答:%经典物理不能解释黑体辐射,光电效应等效应,普朗克提出能量量子化后,解决了经典物理不能解释的黑体辐射;在普朗克的启发下,爱因斯坦引入了光量子的概念,解决了经典物理不能解释的光电效应,并由康普顿效应证实光具有粒子性。
\item 由题可得:
\begin{equation}
\begin{aligned}
e^{i \rho_{y} \partial} =& \sum_{n=0}^{\infty} \frac{(i \rho_y \partial)^{n}}{n!} \\
=& \sum (i\rho_y \partial - \frac{i\rho_y \partial^3}{3!} + \frac{i\rho_y \partial^5}{5!} + \dots) + \sum (1 - \frac{\partial^2}{2!} + \frac{\partial^4}{4!} + \dots) \\
=& i\rho_y \sin{\partial} + \cos{\partial} ~.
\end{aligned}
\end{equation}
故 $B=\sin{\partial},A=\cos{\partial}$。
\end{enumerate}
\subsection{ }
由 $E_n = (n+\frac{1}{2})\hbar \omega$ 可得 $E_0 = \frac{1}{2}\hbar \omega ,E_1 = \frac{3}{2}\hbar \omega$ 将波函数归一化可得 $A=\frac{1}{\sqrt{1+x^2}}$。
\begin{enumerate}
\item $t$ 时刻的波函数为:
\begin{equation}
\psi(x,t)=A\psi_{0}(x)e^{-\frac{iE_{0}t}{\hbar}} + Ax\psi_{1}(x)e^{-\frac{iE_{1}t}{\hbar}}~.
\end{equation}
\item 坐标的平均值为:
\begin{equation}
\begin{aligned}
\bar{x} =& \mel{\psi}{x}{\psi} \\
=& \frac{A^{2}}{\alpha} \braket{\psi_{0} + x\psi_{1}}{\sqrt{\frac{1}{2}}\psi_{1} +\sqrt{\frac{1}{2}}x\psi_{0} + \psi_{2} } \\
=& \frac{A^{2}}{\alpha} (\sqrt{\frac{1}{2}}x + \sqrt{\frac{1}{2}}x) \\
=& \frac{\sqrt{2}A^{2}x}{\alpha}~.
\end{aligned}
\end{equation}
\item 能量的平均值为:
\begin{equation}
\begin{aligned}
\bar{E} =& \frac{1}{1+x^2}\vdot \frac{1}{2}\hbar \omega + \frac{x}{1+x^2}\vdot \hbar \omega \\
=& \frac{\hbar \omega (3x+1)}{2(x+1)}~.
\end{aligned}
\end{equation}
系统是定态,力学量平均值不随时间变化。
\end{enumerate}
\subsection{ }
根据题意有:
\begin{equation}
\begin{aligned}
\psi^{(0)}_{n} =& \sqrt{\frac{2}{a}} \sin{\frac{n \pi x}{a}} \\
E^{(0)}_{n} =& \frac{n^{2} \pi^{2} \hbar^{2}}{2ma^{2}}~. \\
\end{aligned}
\end{equation}
将 $H' = -q \epsilon x$ 是为微扰。因此能量的一级修正为:
\begin{equation}
\begin{aligned}
E^{(1)}_{n} =& \int \psi_{n}^{(0)*} (-q\epsilon) \psi_{n}^{(0)} \dd{x} \\
=& -\frac{2q\epsilon}{a} \int^{a}_{0} x \sin[2](\frac{n \pi x}{a}) \dd{x} \\
=& -\frac{q\epsilon}{a} \vdot \frac{a^{2}}{2} \\
=& -\frac{q \epsilon a}{2}~.
\end{aligned}
\end{equation}
又因为:
\begin{equation}
\begin{aligned}
H'_{mn} =& \int \psi_{m}^{(0)*} (-q\epsilon) \psi_{n}^{(0)} \dd{x} \\
=& -\frac{2q\epsilon}{a} \int^{a}_{0} x \sin{\frac{m \pi x}{a}} \sin{\frac{n \pi x}{a}} \dd{x} \\
=& -\frac{2q\epsilon}{a} \int^{a}_{0} x \frac{1}{2} \left[\cos{\frac{(m-n)\pi x}{a}} - \cos{\frac{(m+n)\pi x}{a}} \right] \dd{x} \\
=& -\frac{q\epsilon}{\pi} \left[\frac{1}{m-n}\sin{\frac{(m-n)\pi x}{a}} \Big|^{a}_{0} - \frac{1}{m+n}\sin{\frac{(m+n)\pi x}{a}} \Big|^{a}_{0} \right] \\
=& -\frac{q\epsilon}{\pi} \left[\frac{1}{m-n}\sin{(m-n)\pi x} - \frac{1}{m+n}\sin{(m+n)\pi x}  \right]~,
\end{aligned}
\end{equation}
所以能量的二级修正为:
\begin{equation}
E^{(2)}_{n} = \sum_{m}' \frac{\left|H'_{mn} \right|^{2}}{E^{(0)}_{m} - E^{(0)}_{n}}~.
\end{equation}
波函数的一级修正为:
\begin{equation}
\psi^{(1)}_{n} = \sum_{m}'\frac{H'_{mn}}{E^{(0)}_{n} - E^{(0)}_{m}} \psi^{(0)}_{m}~.
\end{equation}

\subsection{ }
\begin{enumerate}
\item 由题可得 $\hat{H} = g \bvec{S}_{1}\vdot \bvec{S}_{2} = \frac{1}{2}g(S^{2}-S^{2}_{1}-S^{2}_{2}) = \frac{1}{2}gs(s+1)\hbar^{2}-\frac{3}{2}\hbar^2 $。

自旋波函数为:
\begin{align}
& \chi^{s}_{11} = \chi_{\frac{1}{2}}(s_{1z})\chi_{\frac{1}{2}}(s_{2z})~, \\
& \chi^{s}_{1-1} = \chi_{\frac{1}{2}}(s_{1z})\chi_{-\frac{1}{2}}(s_{2z})~, \\
&\chi^{s}_{10} = \frac{1}{\sqrt{2}} \left[ \chi_{\frac{1}{2}}(s_{1z})\chi_{-\frac{1}{2}}(s_{2z}) + \chi_{-\frac{1}{2}}(s_{1z})\chi_{\frac{1}{2}}(s_{2z})  \right]~, \\
&\chi^{A}_{00} = \frac{1}{\sqrt{2}} \left[ \chi_{\frac{1}{2}}(s_{1z})\chi_{-\frac{1}{2}}(s_{2z}) - \chi_{-\frac{1}{2}}(s_{1z})\chi_{\frac{1}{2}}(s_{2z})  \right]~.
\end{align}
能级为:
\begin{align}
&E_{11} =\frac{g\hbar^{2}}{2}~,  \\
&E_{1-1} =\frac{g\hbar^{2}}{2} ~, \\
&E_{10} =\frac{g\hbar^{2}}{2}  ~,\\
&E_{00} =-\frac{3g\hbar^{2}}{2}  ~.
\end{align}
$S=1$ 时为三重简并,$S=0$ 时为非简并。
\item 设磁场 $B$ 沿 $x$ 方向,$\bvec{\mu} = g \bvec{S}_{x} $.在 $S_{z}$ 表象中:$S_{x} = \frac{\hbar}{2} \bmat{0&1\\1&0}$。\\
所以:
\begin{equation}
\hat{H} = -\bvec{\mu}\bvec{B} = -\frac{gB\hbar}{2} \bmat{0&1\\1&0}~.
\end{equation}
设粒子自旋波函数为:$\psi_{n} = \pmat{c_{1}\\c_{2}} $,又因为本征方程:$\hat{H}\Psi = E\Psi $,所以可得:\\
\begin{equation}
-\frac{gB\hbar}{2} \bmat{0&1\\1&0}\bmat{c_{1}\\c_{2}} = E\bmat{c_{1}\\c_{2}}~.
\end{equation}
解本征方程得:
\begin{align}
E = \frac{gB\hbar}{2} ~,\qquad \psi_{1} = \frac{1}{\sqrt{2}} \bmat{1\\-1} ~,\\
E =  -\frac{gB\hbar}{2}~, \qquad \psi_{1} = \frac{1}{\sqrt{2}} \bmat{1\\1}~.
\end{align}
能级简并完全消除。
\end{enumerate}

