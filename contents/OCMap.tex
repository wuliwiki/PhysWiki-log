% 开映射和闭映射
% keys 开映射|闭映射
% license Usr
% type Tutor

\pentry{拓扑空间\nref{nod_Topol}}{nod_dab5}
拓扑空间中,映射是\enref{连续}{Topo1}的当且将当对像空间的每一开集的原象是\enref{开集}{Topol},或者像空间的每一闭集的原象是闭集。即若 $(X,\mathcal T_X),(Y,\mathcal T_Y)$ 是两个拓扑空间, $f:\mathcal X\rightarrow\mathcal Y$ 是连续的当且将当每一 $O_Y\in\mathcal T_Y$,$f^{-1}(O_Y)\in\mathcal T_X$。然而,我们可以问:在连续映射下,开集的像是否一定是开集?闭集的象是否是闭集?一些例子告诉我们,一般情况下回答是否定的。这就引出了开映射和闭映射的概念。而对理解这两个概念本身来说,我们无需知道连续映射的定义。

\subsection{连续映射下开(闭)集的像不是开(闭)集}
\begin{example}{}
考虑半开区间 $X=[0,1)$ 到圆周 $O=\{(x,y)|x^2+y^2=1,x,y\in\mathbb R\}$ 的映射:
\begin{equation}
f(x)=(\cos(2\pi x),\sin(2\pi x)).~
\end{equation}
其中 $X$ 的开集为 $[0,x),x\in (0,1)$ 和普通的开区间,$O$ 的开集是不包括端点的圆周上的弧(当然还有它们的并和交)。
那么任一 $O$ 的开集 $\{(\cos(2\pi x),\sin(2\pi x))|x\in(a,b),0<a<b<1\}$ 对应 $X$ 的 $(a,b)$,
,显然这是 $X$ 的开集,于是映射 $f$ 是连续的。然而 $f([1/2,1))$ 对应 $O$ 的角度在 $[\pi,2\pi)$ 的弧,其是 $O$ 的非闭集,而 $[1/2,1)$ 是 $X$ 闭集(因为其关于 $X$ 的补集是开集 $[0,1/2)$)。

同样的,$f$ 将 $X$ 的开集 $[0,1/2)$ 映到 $O$ 的角度在 $[0,\pi)$ 的非开集。
\end{example}

因此,一般的连续映射并不会将开(闭)集映射为开(闭)集,因此我们可以在拓扑空间中定义一种将开(闭)集映射为开(闭)集的映射,并给它们专有的名字。

\subsection{开映射和闭映射}

\begin{definition}{开映射,闭映射}
设 $f$ 是拓扑空间之间的映射,若 $f$ 把任一开集映到开集,则称 $f$ 是\textbf{开}的。若 $f$ 把任一闭集映到闭集,则称 $f$ 是\textbf{闭}的。
\end{definition}

由开映射和连续映射的定义,可以知道,开映射的逆是连续的。因此,尽管不能保证一一的连续映射的逆是连续的,但是我们可以专门研究那种一一的连续映射的逆是连续的那些映射,这样的映射称为\enref{同胚映射}{Topo1}。显然,同胚映射首先是一个开映射,也是一个闭映射。




