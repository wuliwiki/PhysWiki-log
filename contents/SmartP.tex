% C++ 的智能指针笔记
% license Xiao
% type Note

\begin{issues}
\issueDraft
\end{issues}

\subsection{unique\_ptr}

\begin{lstlisting}[language=cpp]
// 创建(无法指定删除器,默认用 delete 释放)
auto ptr = make_unique<int>(42); // 等效于 unique_ptr<int> ptr(new int(42));
unique_ptr<int> ptr2 = std::move(ptr1); // 转移所有权
int* rawPtr = ptr.release(); // 释放所有权并返回裸指针(被管理的对象不会被释放)
int* rawPtr = ptr.get(); // 返回裸指针(不释放所有权)
ptr.reset(); // 手动释放对象
ptr.reset(ptr3); // 释放当前对象并管理新的对象

// 自定义删除器类
struct Deleter {
    void operator()(int* ptr) {
        cout << "Deleting resource: " << *ptr << endl;
        delete ptr;
    }
};

// 用法
unique_ptr<int, Deleter> ptr(new int(42)); // 默认使用 Deleter() 新建删除器
unique_ptr<int, Deleter> ptr2(new int(42), Deleter()); // 手动提供删除器
\end{lstlisting}

\begin{itemize}
\item \href{https://en.cppreference.com/w/cpp/memory/unique_ptr}{文档}。
\item \verb`unique_ptr<类> p(类 *类指针)`, 其中 \verb`类指针` 可以由 \verb`new 类(构造参数...)` 产生,指向一个需要 \verb`delete 类指针` 的对象。
\item \verb`unique_ptr<类[]> p(类 *类指针)`, 其中 \verb`类指针` 可以由 \verb`new 类[数量]` 产生,指向一个需要 \verb`delete[] 类指针` 的对象。
\item 注意 \verb`unique_ptr<int[]> p(new int[10])` 不能写成 \verb`unique_ptr<int[]> p = new int[10]` 会导致编译错误因为类型不对。
\item \verb`auto ptr = std::make_unique<MyClass>(arg1, arg2, ...);` 等效于 \verb`auto ptr = std::unique_ptr<MyClass>(new MyClass(arg1, arg2, ...));`。
\item \verb`auto arr_ptr = std::make_unique<MyClass[]>(5);` 可以创建长度为 5 的 int 数组。
\item \verb`std::make_unique<int[10]>()` 这种用法是不支持的。
\item 不允许 “复制赋值” 或 “复制构造”,要传给函数可以用 \verb`p` 的引用或指针。允许 \verb`std::move()`。
\item \verb`unique_ptr<类,删除器类> p(类指针, 删除器)`。 \verb`删除器` 是 \verb`删除器类` 的一个对象(例如 C 函数指针)。
\item \verb`删除器类` 或 \verb`删除器` 可以省略, 省略后默认用对应的 \verb`delete 类指针` 或 \verb`delete[] 类数组`,实际上是 \verb`std::default_delete<T>`(对数组进行了特化)。
\item \verb`删除器` 需要可以像 \verb`void 函数名(类 *p)` 一样使用。 它几乎可以是任何 callable(例如具有 \verb`operator()` 的对象)
\item \verb`unique_ptr` 的 destructor 会调用 \verb`删除器(类指针)` 释放内存。
\item 一般指针的操作 \verb`*, ->, []`, 也可以对 \verb`unique_ptr` 使用。
\item \verb`p.get()` 返回 \verb`类指针`。 \verb`p.get_deleter()` 返回 \verb`删除器` 的 ref。
\item \verb`p.reset(新指针)` 调用原来对象的删除器, 并管理新的对象。 \verb`p.reset()` 释放对象。
\item \verb`if(p)` 或 \verb`bool(p)` 可以判 \verb`p` 是否在管理一个对象。 例如默认构造函数产生 \verb`p`。 或者 \verb`p.reset()` 后的就会是 \verb`false`。
\item 如果用同一个指针来初始化两个不同的 \verb`unique_ptr`,那么将造成重复 delete 错误,行为未定义,这也是为什么要尽量用 \verb`make_unique`
\end{itemize}


\subsection{shared\_ptr}
\begin{itemize}
\item 可以复制, 但只有所有关联的 \verb`shared_ptr` 都释放所有权, 管理的对象才会被删除器释放。
\item \verb`p.use_count()` 返回有几个 \verb`shared_ptr` 正在使用。
\item \verb`p.reset()` 和 \verb`p.reset(p3)` 用法同 \verb`unique_ptr`
\item \verb`std::shared_ptr<int> ptr(new int(42), custom_deleter);`
\item \verb`shared_ptr<void>` 可以装任何指针
\item 一个被 \verb`shared_ptr` 管理的对象会有一个 \textbf{control block} 即\textbf{控制块},它所有的 \verb`shared_ptr` 都会有个指针指向这个 block。 里面记录了该对象的 deleter, allocator, 分别有几个 \verb`shared_ptr` 和 \verb`weak_ptr`(也就是 \verb`shared_count` 和 \verb`weak_count`)
\item 当 \verb`MyClass* raw = ptr.get();` 可以返回裸指针,但不可以手动把它释放或使用它创建新的 \verb`shared_ptr`。
\item 不要用一个裸指针创建多个 \verb`shared_ptr`。\verb`p2 = p` 即可,copy constructor 也行。
\item \verb`auto p = make_shared<int>();`
\item \verb`auto p_arr = make_shared<int[]>(10);` 需要 C++17
\end{itemize}

\subsection{weak\_ptr}
\begin{itemize}
\item 可以从 \verb`shared_ptr`, 或者 \verb`weak_ptr` 初始化
\item \verb`p.lock()` 创建出管理对象的一个新的 \verb`shared_ptr`
\item \verb`p.expired()` 可以检查对象是否已经被删除。
\item 可用于防止循环依赖导致的内存泄漏, 例如:
\begin{lstlisting}[language=cpp]
#include <iostream>
#include <memory>

class Node {
public:
    shared_ptr<Node> next;
    weak_ptr<Node> prev; // weak_ptr to avoid circular reference

    ~Node() {
        cout << "Node destroyed" << endl;
    }
};

int main() {
    shared_ptr<Node> node1 = make_shared<Node>();
    shared_ptr<Node> node2 = make_shared<Node>();

    node1->next = node2;
    // This would create a circular reference if prev were a shared_ptr
    node2->prev = node1;

    cout << "node1 use count: " << node1.use_count() << endl;  // 1
    cout << "node2 use count: " << node2.use_count() << endl;  // 2

    return 0;
}
\end{lstlisting}
\item 上面提到的 \verb`shared_count` 和 \verb`weak_count` 中, 前者控制被管理的对象是否销毁。 后者控制控制块是否被销毁。
\item 上面例子中避免循环依赖也可以使用裸指针,但罗指针无法判断是否对象已被销毁所以不太安全。
\item \verb`shared_count` 和 \verb`weak_count` 都是线程安全的,可以在多个线程中任意改变。
\item \verb`weak_ptr` 不提供 \verb`get()` 成员,无法直接返回裸指针。 因为这是线程不安全的,因为有可能 \verb`get()` 调用以后、裸指针使用以前,其他线程会把对象销毁。 所以必须要先用 \verb`lock()` 返回一个 \verb`shared_ptr`,把这个对象锁住,再用 \verb`shared_ptr` 返回裸指针才能安全使用。 这也是 \verb`lock()` 名字的由来:把对象锁住不让它释放。
\end{itemize}
