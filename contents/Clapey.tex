% 克拉伯龙方程
% keys 相变|相平衡|饱和蒸气压方程
% license Xiao
% type Tutor

\begin{issues}
\issueDraft
\end{issues}

\pentry{相变平衡条件\nref{nod_PhEquv}}{nod_732b}

\subsection{克拉伯龙方程}
单元系两相平衡共存时,必须满足\autoref{eq_PhEquv_3}  的 $T^\alpha=T^\beta=T,P^\alpha=P^\beta=P,\mu^\alpha(T,P)=\mu^\beta(T,P)$ 三个条件。其中相平衡条件中摩尔化学势 $\mu$ 可以看作是温度和压强的函数。此时,相变温度 $T$ 对应着压强 $P$(在压强 $P$ 的环境下,相变温度为 $T$)。

现在考虑系统的温度和压强有改变量 $\dd T,\dd P$,要使得两相仍处于平衡条件,就有

\begin{align}
&\mu^\alpha(T,P)=\mu^\beta(T,P)~,\\
&\mu^\alpha(T+\dd T,P+\dd P)=\mu^\beta(T+\dd T,P+\dd P)~.
\end{align}

由于化学势就是摩尔吉布斯函数,所以 $\dd\mu=-S_m\dd T+V_m\dd P$($S_m$ 为摩尔熵,$V_m$ 是摩尔体积)。由 $\dd \mu^\alpha=\dd \mu^\beta$ 可以推出
\begin{equation}
\frac{\dd P}{\dd T}=\frac{S^\beta_m-S^\alpha_m}{V^\beta_m-V^\alpha_m}~.
\end{equation}

在实验上熵是不能直接测量的,但我们知道在可逆过程中 $\Delta Q=T\Delta S$。考虑 $1\rm mol$ 物质从 $\alpha$ 相转变到 $\beta$ 相所吸收的\textbf{相变潜热},由于相变时物质温度不变,有 $L=T(S_m^\beta-S_m^\alpha)$。可以得到\textbf{克拉伯龙方程}:

\begin{equation}\label{eq_Clapey_1}
\frac{\dd P}{\dd T}=\frac{L}{T(V^\beta_m-V^\alpha_m)}~.
\end{equation}
\begin{figure}[ht]
\centering
\includegraphics[width=10cm]{./figures/dbbfe7aaec096192.png}
\caption{水的三相图} \label{fig_Clapey_1}
\end{figure}

相平衡曲线的斜率通常是正的,但也存在例外:从水的三相图中看到,水的熔化线斜率 $\dd p/\dd T<0$。而水的摩尔体积比冰的摩尔体积小,密度比冰大,代入克拉伯龙方程确实能得到熔化曲线斜率小于 $0$ 的结果。

\subsection{饱和蒸气压方程}

由克拉伯龙方程可以得出在气相 $\beta$ 与凝聚相(液相或固相)$\alpha$ 之间的相变方程,可以得到饱和蒸气压与温度的关系,也就是\textbf{饱和蒸气压方程}。现在做粗略的近似,如果将气相看作理想气体,那么由 \autoref{eq_Clapey_1} 可得
\begin{equation}
\frac{1}{P}\frac{\dd P}{\dd T}=\frac{L}{RT^2}~.
\end{equation}

再做更粗糙的近似,将相变潜热 $L$ 认为是与温度无关。那么可以积分得:
\begin{equation}\label{eq_Clapey_2}
\ln P=-\frac{L}{RT}+A~.
\end{equation}

\subsection{二级相变的爱伦费斯特方程}

爱伦费斯特(Ehrenfest)试图对相变进行分类:因为 $S_m=\frac{\partial \mu}{\partial T},V_m=\frac{\partial \mu}{\partial P}$,爱氏将一级相变概括为化学势连续、但化学势的一级偏导数在两相存在突变的相变。对与这一类相变,$S_m^\alpha\neq S_m^\beta,V_m^\alpha\neq V_m^\beta$,克拉伯龙方程适用。。

但在气液通过临界点的转变、铁磁顺磁的转变等过程中,既没有 $S_m$ 的突变(也就是说不存在相变潜热),又没有 $V_m$ 的突变。在这些过程中,化学势的一级偏导数连续,但化学势的二级偏导数不连续。爱氏将它归类为二级相变。

\begin{align}
&C_{P,m}=T\left(\frac{\partial S_m}{\partial T}\right)=-T\frac{\partial^2 \mu}{\partial T^2}~,\\
&\alpha=\frac{1}{V_m}\left(\frac{\partial V_m}{\partial T}\right)_P
=\frac{1}{V_m}\frac{\partial^2\mu}{\partial T\partial P}~,\\
&\kappa_T=-\frac{1}{V_m}\left(\frac{\partial V_m}{\partial P}\right)_T
=-\frac{1}{V_m}\frac{\partial^2\mu}{\partial P^2}~.
\end{align}

对于二级相变,容易得出压强与温度的新变化关系,\textbf{爱伦费斯特方程}:

\begin{equation}
\frac{\dd P}{\dd T}=\frac{\alpha^{(2)}-\alpha^{(1)}}{\kappa_T^{(2)}-\kappa_T^{(1)}}=\frac{C_{P,m}^{(2)}-C_{P,m}^{(1)}}{TV_m(\alpha^{(2)}-\alpha^{(1)})}~.
\end{equation}
