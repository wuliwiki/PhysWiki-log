% CLion 笔记
% license Xiao
% type Note

\begin{issues}
\issueDraft
\end{issues}

本文假设我们在 linux 桌面环境(或 Windows 的 WSL)下使用 CLion。

\subsubsection{简单的 cmake 文件}
如果你只用 \verb|Makefile| 不想迁移到 \verb|cmake| 也没关系,你只需要写一个很简单的 wrapper,就可以用它作为 CLion 的工程文件。 一个简单的例子:
\begin{lstlisting}[language=cmake]
# just a wrapper for Makefile for CLion
cmake_minimum_required(VERSION 3.10)
project(FEDVR-TDSE)

# ==== wrapper of Makefile ===
add_custom_target(my_targ1
  COMMAND make -f make/all.mak opt_debug=true main1.x
  WORKING_DIRECTORY "${CMAKE_CURRENT_SOURCE_DIR}"
)

add_custom_target(my_targ2
  COMMAND make -f make/all.mak opt_debug=true main2.x
  WORKING_DIRECTORY "${CMAKE_CURRENT_SOURCE_DIR}"
)


# ==== dummy settings to make CLion happy ===
set(CMAKE_CXX_STANDARD ${opt_std})
set(CMAKE_CXX_STANDARD_REQUIRED ON)
set(CMAKE_CXX_EXTENSIONS OFF)

file(GLOB ALL_HEADERS
    "${PROJECT_SOURCE_DIR}/headers/*.h")

message(STATUS "${ALL_HEADERS}")

add_executable(dummy1
    ${PROJECT_SOURCE_DIR}/main1.cpp ${ALL_HEADERS})

add_executable(dummy2
    ${PROJECT_SOURCE_DIR}/main2.cpp ${ALL_HEADERS})
\end{lstlisting}

\subsubsection{cmake 环境}
\begin{itemize}
\item \verb|Settings > Build,Execution,Development > Toolchains| 选择编译环境, 如 WSL, Mingw, Docker, SSH 等。 会检测 CMake 的版本以及 GDB 的版本。
\item \textbf{SSH} 设置中甚至可以自动读取 \verb|~/.ssh/config| 中的 host 设置, 但是第一次编译时可能要稍微花点时间把源码传到 remote host (目录如 \verb|/tmp/tmp.TH5zt7WrwC|)。 另外仍然不会运行 \verb|~/.bashrc|, 需要在 CLion 中手动添加环境变量。
\item \textbf{MinGW}(放弃) 设置时选择自己安装的 MSYS2 中的 \verb|mingw64.exe| 却提示说找不到文件。 又试了用 CLion 自带的 mingw, CMake 说找不到 \verb|/usr/bin/g++|, 放弃。
\item 下面的 CMake 面板里面的齿轮按钮可以设置 \verb|cmake|。 也可以用 \verb|Settings > Build,Execution,Development > CMake| 进入。 包括 Build options \verb|-- -j 16|, 运行 \verb|cmake| 时的环境变量(例如 \verb|CPATH, LIBRARY_PATH|)。
\item \textbf{Remote Development} 在 File 菜单中选这个可以用 SSH 直接连到 Linux 服务器, 在服务器的文件系统中打开项目并调试。 第一次使用会自动在服务器下载安装 backend, 在本地安装 client。
\item Linux/WSL 环境下 CLion 通过 \verb|sh| 运行 CMake, 而且应该是新开的 session,所以还是不要试图通过脚本设置环境变量了
\end{itemize}

\subsubsection{运行环境}
\begin{itemize}
\item 上面菜单栏的 Select Run Debug Configuration 可以设置运行时的 working dir, arguments, input file, 环境变量(例如 \verb|LD_LIBRARY_PATH|)。
\item 注意如果你在 \verb|~/.bashrc| 中设置的环境变量,如果用菜单栏打开 CLion,并不会生效。 所以此时在命令行中启动 CLion 即可省略手动设置。
\item 为了方便在命令行启动 clion, 可以在 \verb|~/.bashrc| 中添加 \verb|alias clion="sh /opt/clion-*/bin/clion.sh"|。
\end{itemize}

\subsection{设置}
\begin{itemize}
\item 粘贴的时候会自动 indent 有时候比较难受, 在 \verb|Editor > General > Smart Keys > Reformat on paste| 里面设置。
\end{itemize}
