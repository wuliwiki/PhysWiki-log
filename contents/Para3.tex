% 抛物线
% keys 极坐标系|直角坐标系|圆锥曲线|抛物线
% license Xiao
% type Tutor

\pentry{圆锥曲线的极坐标方程\nref{nod_Cone}}{nod_7c17}

\subsection{用直角坐标方程定义抛物线}
我们已经知道用焦点和准线如何定义抛物线和其他圆锥曲线(\autoref{eq_Cone_5}), 抛物线的离心率 $e = 1$, 所以极坐标方程为
\begin{equation}\label{eq_Para3_1}
r = \frac{p}{1 - \cos \theta }~.
\end{equation}
以与极坐标系相同的原点建立直角坐标系, 要把以上方程变到直角坐标系中, 将$r = \sqrt{x^2 + y^2}$,$\cos \theta  = x/\sqrt{x^2 + y^2}$ 代入得
\begin{equation}
\sqrt{x^2 + y^2}  = p + x~.
\end{equation}
两边平方并化简得到
\begin{equation}
y^2 = 2p \qty(x + \frac p2)~.
\end{equation}
把双曲线沿 $x$ 轴正方向移动 $p/2$, 可得标准抛物线方程
\begin{equation}\label{eq_Para3_2}
y^2 = 2px~,
\end{equation}
所以抛物线的\textbf{焦距}(焦点到端点)为 $f = p/2$。 与椭圆和双曲线不同的是, 所有的抛物线的形状都相似(形状相同, 大小不同), 这是因为抛物线有固定的离心率(离心率决定圆锥曲线的形状, 焦距或准线决定大小)。

\subsection{另一种定义}
这就是 “\enref{圆锥曲线的极坐标方程}{Cone}” 中对抛物线的定义。
\begin{figure}[ht]
\centering
\includegraphics[width=4.2cm]{./figures/c89771dd2fef516e.pdf}
\caption{抛物线的定义} \label{fig_Para3_1}
\end{figure}

在 $x$ 轴正半轴作一条与准线平行的直线 $L$, 则抛物线上一点 $P$ 到其焦点的距离 $r$ 与 $P$ 到 $L$ 的距离之和不变。

如\autoref{fig_Para3_1}, 要证明由焦点和准线定义的抛物线满足该性质, 只需过点 $P$ 作从准线到直线 $L$ 的垂直线段 $AB$, 由于 $r$ 等于线段 $PA$ 的长度, 所以 $r$ 加上 $PB$ 的长度等于 $AB$ 的长度, 与 $P$ 的位置无关。 证毕。

\subsection{用圆锥截面定义抛物线}
抛物线之所以叫做圆锥曲线, 是因为它们可以由平面截取双圆锥面得到, 详见 “\enref{圆锥曲线和圆锥}{ConSec}”。

\subsection{端点的曲率半径}
\pentry{平面曲线的曲率和曲率半径(简明微积分)\nref{nod_curvat}}{nod_1813}
%\addTODO{按照定义推导一下曲率半径为 $p$}
抛物线顶点处的曲率半径为$p$.

我们使用极坐标曲率半径公式
$$
\rho = \frac{(r^2 + \dot r^2)^{3/2}}{r^2 + 2\dot r^2 - r\ddot r}~.
$$
与抛物线的极坐标方程:
$$
r = \frac{p}{1 - \cos \theta }~.
$$

在抛物线顶点处,我们有
$$
r|_{\theta = \pi} = \frac{p}{1 - \cos \theta} = \frac{p}{2}~,
$$
$$
r' |_{\theta = \pi} = -\frac{p}{(1 - \cos \theta)^2} \sin(\theta) = 0~.
$$
$r'' |_{\theta = \pi}$使用导数的定义会更好做,这也是高数中求解复杂导数的一个技巧:
$$
\begin{aligned}
r''|_{\theta = \pi}  &= \lim_{\theta \to \pi} \frac{r'(\theta) - r'(\pi)}{\theta - \pi}\\
&=\lim_{\theta \to \pi} \frac{-\frac{p}{(1 - \cos \theta)^2} \sin(\theta)}{\theta - \pi}\\
&=-\frac{p}{4} \lim_{\theta \to \pi} \frac{\sin(\theta)}{\theta - \pi}\\
&=\frac{p}{4}~.\\
\end{aligned}
$$

那么
$$
\begin{aligned}
\rho &= \frac{(r^2 + \dot r^2)^{3/2}}{r^2 + 2\dot r^2 - r\ddot r}\\
&=\frac{p^3/8}{p^2/4 - p/2*p/4}\\
&=\frac{p^3/8}{p^2/8}\\
&=p~.\\
\end{aligned}
$$
