% 阿瑟·凯利(综述)
% license CCBYSA3
% type Wiki

本文根据 CC-BY-SA 协议转载翻译自维基百科\href{https://en.wikipedia.org/wiki/Arthur_Cayley}{相关文章}。

\begin{figure}[ht]
\centering
\includegraphics[width=6cm]{./figures/54015c4f6823f2b0.png}
\caption{} \label{fig_Cayley_1}
\end{figure}
阿瑟·凯利(Arthur Cayley,FRS,1821年8月16日-1895年1月26日)是英国数学家,主要从事代数方面的研究。他帮助创立了现代英国纯数学学派,并在剑桥大学三一学院担任教授长达35年。

他提出了如今被称为凯利-哈密顿定理的观点——即每个方阵都是其自身特征多项式的根,并验证了2阶和3阶矩阵的情况。[1]他是第一个定义抽象群概念的人,抽象群是满足某些运算规律的集合,[2]区别于艾瓦里斯特·伽罗瓦(Évariste Galois)对置换群的定义。在群论中,凯利表、凯利图以及凯利定理都以他命名,而在组合数学中,也有凯利公式以纪念他。
\subsection{早年生活}  
阿瑟·凯利于1821年8月16日出生在英国伦敦的里士满。他的父亲亨利·凯利是航空工程师乔治·凯利的远亲,来自约克郡的一个古老家族,并作为商人定居在俄罗斯圣彼得堡。他的母亲是玛丽亚·安托尼娅·道蒂(Maria Antonia Doughty),威廉·道蒂的女儿。根据一些作家的说法,她是俄罗斯人,但她父亲的名字表明她有英格兰血统。他的兄弟是语言学家查尔斯·巴戈特·凯利。阿瑟在圣彼得堡度过了他生命的最初八年。1829年,他的父母定居在伦敦的布莱克希斯,阿瑟在那里上了一所私立学校。

14岁时,他被送到国王学院学校。年轻的凯利喜欢复杂的数学问题,学校的校长注意到他在数学方面的天赋,并建议父亲不要让儿子从事商业工作,而是送他去剑桥大学学习。
\subsection{教育}  
17岁时,凯利开始在剑桥大学三一学院学习,在那里他在希腊语、法语、德语、意大利语以及数学方面表现出色。此时,分析学会的事业已取得胜利,剑桥数学杂志由格雷戈里(Gregory)和罗伯特·莱斯利·埃利斯(Robert Leslie Ellis)创办。20岁时,凯利向该杂志投稿了三篇论文,这些论文的主题受到了他阅读约瑟夫·路易·拉格朗日的《解析力学》和拉普拉斯的一些著作的启发。

凯利在剑桥的导师是乔治·皮科克(George Peacock),他的私人教练是威廉·霍普金斯(William Hopkins)。他通过赢得高级学者(Senior Wrangler)和第一个史密斯奖学金(Smith's Prize)顺利完成了本科课程。接下来,他计划获得硕士学位,并通过竞争考试赢得奖学金。[3]他继续在剑桥大学居住了四年,在此期间,他接收了一些学生,但他的主要工作是为数学杂志准备28篇论文。
\subsection{法律事业}  
由于奖学金的任期有限,他必须选择一个职业;像德·摩根一样,凯利选择了法律,并于1846年4月20日以24岁的年龄被录取为林肯律师协会(Lincoln's Inn)的成员。[4]他专攻不动产转让。在参加律师资格考试期间,他曾前往都柏林聆听威廉·罗文·汉密尔顿关于四元数的讲座。

他的朋友J.J. 西尔维斯特(J. J. Sylvester),比他大五岁,曾在剑桥大学与他为同学,当时是一名常驻伦敦的精算师;他们常一起在林肯律师协会的院子里散步,讨论不变量和协变理论。在这十四年间,凯利创作了大约二三百篇论文。[5]
\subsection{教授职位}  
大约在1860年,剑桥大学的卢卡谢恩数学教授席位(牛顿的讲座)被新的萨德莱尔教授职位所补充,该职位是由萨德莱夫人遗赠的基金资助的,42岁的凯利成为第一位担任此职的人。他的职责是“阐明和教授纯数学的原理,并致力于该科学的进步。”他放弃了收入丰厚的法律事业,选择了一个 modest 的工资,但他从未后悔这一决定,因为它使他能够将精力投入到自己最喜爱的事业中。他立即结婚并定居在剑桥,并且(与哈密尔顿不同)享受着非常幸福的家庭生活。曾经,大学时期的朋友西尔维斯特表达了对凯利宁静家庭生活的羡慕,而未婚的西尔维斯特则终其一生不得不与世界作斗争。

最初,萨德莱尔教授的薪水仅足够在学年中的一个学期讲课,但1886年的大学财政改革为其讲座提供了资金,使得讲座可以扩展至两个学期。在许多年里,他的课程仅有几位完成考试准备的学生参加,但改革后,出席人数约为十五人。他通常会讲解他当前的研究课题。

至于他在数学科学进步方面的职责,他出版了一系列漫长而富有成效的论文,涵盖了纯数学的各个领域。他还成为了许多国内外学会的常驻评审,负责评估数学论文的价值。

1872年,他被授予三一学院荣誉院士称号,三年后成为正式院士,担任有薪职位。大约在这个时候,他的朋友们为他捐款,送上了一幅肖像画。麦克斯韦写了一篇致辞,赞扬凯利的主要著作,包括《n维解析几何章节》;《行列式理论》;《矩阵理论论文》;《偏斜曲面的论文,也叫做螺旋曲面》;以及《三次螺旋曲线的描绘》。[6] 

除了在代数方面的工作外,凯利还对代数几何作出了基础性的贡献。凯利和萨尔蒙发现了立方曲面上的27条直线。凯利构造了所有投影三维空间曲线的周空间。[7]他创立了有规律曲面的代数几何理论。他在组合数学方面的贡献包括通过生成函数的开创性使用,计算了\(n\)个标记顶点上的\(n^{n-2}\)棵树。

1876年,他出版了《椭圆函数论》一书。他对女性大学教育运动表现出浓厚的兴趣。在剑桥,第一所女子学院是吉尔顿学院和纽纳姆学院。在吉尔顿学院的早期,他直接参与了教学工作,并且在许多年里,他担任了纽纳姆学院理事会主席,对学院的进展一直保持着浓厚的兴趣,直至最后。

1881年,他收到了当时麦克斯韦担任剑桥大学数学教授的约翰斯·霍普金斯大学(位于巴尔的摩)的邀请,邀请他进行一系列讲座。他接受了邀请,并于1882年上半年在巴尔的摩讲授关于阿贝尔函数和θ函数的主题。

1893年,凯利成为荷兰皇家艺术与科学院的外籍会员。[8]
\subsection{英国学会会长}  
1883年,凯利成为了英国科学促进会的会长。会议在英格兰北部的南港举行。由于会长的致辞是会议中的重要公众活动之一,并吸引了具有一般文化背景的听众,因此通常会尽量避免使用过于技术性的内容。凯利(1996年)以《纯数学的进展》为主题进行致辞。
\subsection{《论文集》}  
1889年,剑桥大学出版社开始出版他的论文集,这让他十分欣慰。他亲自编辑了其中七卷四开本,尽管当时正饱受严重的内疾折磨。他于1895年1月26日去世,享年73岁。他的葬礼在三一学院教堂举行,英国的顶尖科学家们出席了仪式,甚至来自俄罗斯和美国的官方代表也前来悼念。  

其余的论文由他的继任者、萨德利里安数学教授安德鲁·福赛斯编辑,总共十三卷四开本,收录了967篇论文。他的研究成果至今仍被广泛使用,仅在21世纪就已被200多篇数学论文引用。  

凯莱至生命最后一刻仍钟爱小说阅读和旅行。他对绘画和建筑尤为喜爱,并练习水彩画,这在绘制数学图示时偶尔派上用场。
\subsection{遗产}
凯莱被安葬在剑桥的米尔路公墓。  

由洛厄斯·凯托·狄金森(Lowes Cato Dickinson)于1874年所绘的凯莱肖像以及由威廉·朗梅德(William Longmaid)于1884年所绘的肖像,现收藏于剑桥大学三一学院。

许多数学术语以凯莱的名字命名,包括:  
\begin{itemize}
\item 凯莱定理(Cayley's theorem)  
\item 凯莱-汉密尔顿定理(Cayley–Hamilton theorem)——线性代数中的重要定理  
\item 凯莱-巴克拉赫定理(Cayley–Bacharach theorem)  
\item 格拉斯曼-凯莱代数(Grassmann–Cayley algebra)  
\item 凯莱-门格行列式(Cayley–Menger determinant)  
\item 凯莱图(Cayley diagrams)——用于机械工程中寻找共轭连杆机构  
\item 凯莱-迪克森构造(Cayley–Dickson construction)  
\item 凯莱代数(八元数)(Cayley algebra, Octonion)  
\item 凯莱图(Cayley graph)  
\item 凯莱数(Cayley numbers)  
\item 凯莱六次曲线(Cayley's sextic)  
\item 凯莱表(Cayley table)  
\item 凯莱-珀瑟算法(Cayley–Purser algorithm)  
\item 凯莱公式(Cayley's formula)  
\item 凯莱-克莱因度量(Cayley–Klein metric)  
\item 凯莱-克莱因双曲几何模型(Cayley–Klein model of hyperbolic geometry)  
\item 凯莱Ω过程(Cayley's Ω process)  
\item 凯莱曲面(Cayley surface)  
\item 凯莱变换(Cayley transform)  
\item 凯莱结点三次曲面(Cayley's nodal cubic surface)  
\item 凯莱直纹三次曲面(Cayley's ruled cubic surface)  
\item 月球上的凯莱陨石坑(Cayley crater)——因此衍生出**凯莱地层**(Cayley Formation),一个以该陨石坑命名的地质单元  
\item 凯莱捕鼠器(Cayley's mousetrap)——一种纸牌游戏  
\item 凯莱数(Cayleyan)  
\item 沙勒-凯莱-布里尔公式(Chasles–Cayley–Brill formula)  
\item 超行列式(Hyperdeterminant)  
\item 奎皮安(Quippian)  
\item 四面体曲面(Tetrahedroid)  
\end{itemize}
\subsection{参考书目}  
\begin{itemize}
\item 凯莱,阿瑟(Arthur Cayley)(2009)[1876],《椭圆函数初等论》(*An Elementary Treatise on Elliptic Functions*),康奈尔大学图书馆,ISBN 978-1-112-28006-1,MR 0124532。  
\item 凯莱,阿瑟(Arthur Cayley)(2009)[1889],《数学论文集》(*The Collected Mathematical Papers*),剑桥数学文库(共14卷),剑桥大学出版社,ISBN 978-1-108-00507-4,存档。  
\item 凯莱,阿瑟(Arthur Cayley)(1894),《复式记账原理》(*The Principles of Book-Keeping by Double Entry*),剑桥大学出版社。
\end{itemize}
\subsection{参见} 
\begin{itemize}
\item 以阿瑟·凯莱命名的事物列表(List of things named after Arthur Cayley)
\end{itemize}  
\subsection{参考文献}  
\begin{enumerate}
\item 凯莱(1858),“关于矩阵理论的论文”("A Memoir on the Theory of Matrices"),《伦敦皇家学会哲学汇刊》(Philosophical Transactions of the Royal Society of London),第148卷,第24页:“我已经在下一个最简单的情形,即阶数为3的矩阵中验证了该定理……但我认为没有必要在一般情况下,即任何阶的矩阵,都进行正式证明。”  
\item 凯莱(1854),“关于群论,取决于符号方程θⁿ = 1”("On the theory of groups, as depending on the symbolic equation θⁿ = 1"),《哲学杂志》第四系列,第7卷(42):40–47。然而,也请参见对该定义的批评:MacTutor:《抽象群概念》(The abstract group concept)。  
\item "Cayley, Arthur (CLY838A)",剑桥校友数据库(A Cambridge Alumni Database),剑桥大学。  
\item The Records of the Honorable Society of Lincoln's Inn, 第II卷,入学登记(1420-1893),伦敦:林肯律师学院(Lincoln's Inn),1896年,第226页。  
\item 福赛斯,安德鲁·拉塞尔(1901),“凯莱,阿瑟”("Cayley, Arthur"),载于李·悉尼(Sidney Lee)编,《国家人物传记词典》(Dictionary of National Biography,第一增补卷),伦敦:史密斯、埃尔德公司(Smith, Elder & Co.)。  
\item “致凯莱肖像基金委员会”("To the Committee of the Cayley Portrait Fund"),1874年。  
\item A. 凯莱,《数学论文集》(Collected Mathematical Papers),剑桥(1891),第4卷,446−455页。  
W. V. D. 霍奇 和 D. 佩多,《代数几何方法》(*Methods of Algebraic Geometry*),剑桥(1952),第2卷,第388页。  
\item “A. 凯莱(1821 - 1895)”,荷兰皇家艺术与科学院(Royal Netherlands Academy of Arts and Sciences),2016年4月19日检索。  
\item “剑桥大学三一学院”("Trinity College, University of Cambridge"),BBC《你的画作》(Your Paintings),2014年5月11日存档,2018年2月12日检索。  
\end{enumerate}
\subsection{来源} 
\begin{itemize}
\item 凯莱,阿瑟(Arthur Cayley)(1996)[1883],“英国科学协会主席演讲” ("Presidential address to the British Association"),收录于埃瓦尔德,威廉(William Ewald)(编),《从康德到希尔伯特:数学基础文献集》(From Kant to Hilbert: a source book in the foundations of mathematics),第I、II卷,牛津科学出版物,克拉伦登出版社·牛津大学出版社,页542–573,ISBN 978-0-19-853271-2,MR 1465678。该文也被收录于《数学论文集》第11卷。  
\item 克里利,托尼(Tony Crilly)(1995),“维多利亚时代的数学家:阿瑟·凯莱(1821–1895)”("A Victorian Mathematician: Arthur Cayley (1821–1895)"),《数学公报》(The Mathematical Gazette),第79卷(485),数学协会(The Mathematical Association),页259–262,doi:10.2307/3618297,ISSN 0025-5572,JSTOR 3618297,S2CID 188006684。  
\item 克里利,托尼(Tony Crilly)(2006),《阿瑟·凯莱:维多利亚时代的数学桂冠》(Arthur Cayley. Mathematician laureate of the Victorian age),约翰·霍普金斯大学出版社,ISBN 978-0-8018-8011-7,MR 2284396。  
\item 麦克法兰,亚历山大(Alexander Macfarlane)(2009)[1916],《十九世纪十位英国数学家的讲座》(Lectures on Ten British Mathematicians of the Nineteenth Century),数学专著系列(Mathematical monographs),第17卷,康奈尔大学图书馆,ISBN 978-1-112-28306-2(完整文本可在Project Gutenberg获取)。  
\end{itemize}
\subsection{外部链接}
\begin{itemize}
\item 奥康纳,约翰·J.;罗伯逊,埃德蒙·F.,“阿瑟·凯莱” (Arthur Cayley),麦克图尔数学史档案 (MacTutor History of Mathematics Archive),圣安德鲁斯大学。  
\item 数学家世系计划 (Mathematics Genealogy Project) 中的阿瑟·凯莱。  
\item 魏斯坦,埃里克·沃尔夫冈(编),“凯莱,阿瑟(1821–1895)” (Cayley, Arthur (1821–1895)),科学世界 (ScienceWorld)。  
\item 阿瑟·凯莱写给罗伯特·哈雷的信(1859–1863),可在线获取,存于理海大学 (Lehigh University) 的“我仍然存在:书信、手稿与文献数字档案” (I Remain: A Digital Archive of Letters, Manuscripts, and Ephemera)。  
\item 萨尔蒙,乔治(1883年9月20日),“科学名人. XXII.—阿瑟·凯莱” (Science Worthies. XXII.—Arthur Cayley),《自然》(*Nature*),第28卷,481–485页,doi:10.1038/028481a0。  
\item 斯科特,夏洛特·安加斯(1895年),“阿瑟·凯莱:生于1821年8月16日,卒于1895年1月26日” (*Arthur Cayley. Born August 16th, 1821. Died January 26th, 1895*),《美国数学学会通报》(Bull. Amer. Math. Soc.),第1卷(6):133–141,doi:10.1090/s0002-9904-1895-00261-x,MR 1557369。  
\end{itemize}