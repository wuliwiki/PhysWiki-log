% Julia 中的函数
% keys Julia 中的函数
% license Xiao
% type Tutor

本文授权转载自郝林的 《Julia 编程基础》。 原文链接:\href{https://github.com/hyper0x/JuliaBasics/blob/master/book/ch12.md}{第 12 章 函数与方法}。

\subsubsection{12.2 Julia 中的函数}

在 Julia 中,函数实际上也属于一种值,同时也是该语言中的第一等公民(first-class citizen)。因此,它们可以被赋给变量和字段,以及作为其他函数的参数值和结果值。甚至,它们还可以是匿名的。

Julia 中的函数的含义和作用都与数学中的定义比较贴近。你还记得定义函数的那种简洁形式吗?如:

\begin{lstlisting}[language=julia]
greet() = print("Hello World!")
\end{lstlisting}

这与数学中定义函数的方式是多么的相似啊!不过,这个名叫\verb|greet|的函数并不是一个纯粹的函数。因为它有副作用,会向标准输出打印内容。

Julia 虽然允许函数拥有副作用,但是它有一个非常好的惯用法,同时也是一项很值得遵从的编程最佳实践,那就是:拥有副作用的函数的名称会以\verb|!|为后缀。反过来说,名称不以\verb|!|为后缀的函数就不会有副作用。因此,函数名称中的\verb|!|也就成为了一种标志。它向我们表明了当前函数的特殊行为方式。我们在之前已经介绍过不少这样的函数,如:\verb|append!|、\verb|broadcast!|、\verb|delete!|、\verb|get!|、\verb|merge!|、\verb|sort!|,等等。

Julia 语言的标准库以及很多第三方库都会严格地遵循这种惯用法。不过,其中也有一些拥有副作用的函数的名称并没有以\verb|!|结尾,比如我们在前面提到的\verb|print|。这种函数很少,而且都是很特别的。因为它们本身就是为副作用而存在的。并且,它们在名称上往往也有着非常鲜明的示意。这才保证了使用者不会把它们与其他的没有副作用的函数相混淆,从而避免了困惑和错用。总之,这样的惯用法是非常值得借鉴的。我们也应该尽可能地遵循它们。

另一方面,Julia函数的用途是很广泛的。Julia中的大多数操作符都是由函数实现的,只不过它们支持了某些特殊的语法罢了。我们在前面的章节中使用过的操作符\verb|+|、\verb|-|、\verb|*|、\verb|/|,以及\verb|==|和\verb|===|等等,其实都是函数。例如,在Julia的标准库中定义了这样一个名称为\verb|+|的函数:

\begin{lstlisting}[language=julia]
+(x::Float64, y::Float64) = add_float(x, y)
\end{lstlisting}

这个函数会被应用在两个浮点数相加的操作当中:

\begin{lstlisting}[language=julia]
julia> 1.2 + 2.3
3.5

julia> @which 1.2 + 2.3
+(x::Float64, y::Float64) in Base at float.jl:401

julia> +(1.2, 2.3)
3.5

julia> @which +(1.2, 2.3)
+(x::Float64, y::Float64) in Base at float.jl:401

julia>
\end{lstlisting}

可以看到,不论是\verb|1.2 + 2.3|还是\verb|+(1.2, 2.3)|,它们使用的实际上都是同一种操作方式。

除此之外,在索引表达式(如\verb|array[1]|)、选择表达式(如\verb|user.name|)以及数组拼接操作背后的也都是函数。我们刚刚用到的宏\verb|@which|可以让它们都原形毕露。

这里需要明确一下,我们确实可以把前面那个名称为\verb|+|的定义笼统地称为函数定义。但是,确切地说,它定义的是一个衍生方法,而且是基于泛化函数\verb|+|的且针对于\verb|Float64|类型的参数的衍生方法。

你应该已经对衍生方法这个概念不陌生了。我们在前面几章中都有所提及,而且还一起编写过几个简单的衍生方法。不过,我在本章的后面还会对衍生方法进行一次系统化的阐释,以便让你能够深入地理解它们。

下面,我们先从基本的函数编写方式讲起。
