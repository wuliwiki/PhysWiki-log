% 费罗贝尼乌斯定理

\pentry{切空间\upref{tgSpa}}

在微分几何中, \textbf{费罗贝尼乌斯定理(Frobenius' theorem)} 给出了一阶拟线性偏微分方程组可积的充分必要条件。 它有许多重要的几何推论。

\subsection{对合分布与可积性}
设 $M$ 是 $n$ 维微分流形。 一个\textbf{光滑 $k$ 维分布} (smooth $k$-dimensional distribution, 注意这不是分析学意义下代表广义函数的分布) $\mathfrak{D}$ 是指切丛 $TM$ 的 $k$ 维光滑子丛。 等价地说, 这表示在任何一点 $p\in M$ 都给出 $T_pM$ 的 $k$ 维子空间 $\mathfrak{D}_p$, 而且 $\mathfrak{D}_p$ 光滑地依赖于 $p$。

维数为 $k$ 的光滑分布 $\mathfrak{D}$ 称为\textbf{对合的(involutive)}, 如果对于 $\mathfrak{D}$ 的任意两个光滑截面 $X,Y$, $[X,Y]$ 也还是 $\mathfrak{D}$ 的截面。 

\begin{theorem}{费罗贝尼乌斯定理}
设 $\mathfrak{D}$ 是 $M$ 上的 $k$ 维对合分布。 则在任意一点 $p\in M$, 都有局部坐标系 $\{x^i\}$ 使得 $\mathfrak{D}$ 在该点处由坐标向量 $\partial_1,\cdots ,\partial_k$ 张成。 等价地说, 在任意一点 $p$ 都有一过点 $p$ 的 $k$ 维子流形 $N_p$ 使得 $\mathfrak{D}$ 恰为 $N_p$ 的切丛。
\end{theorem}
这样的分布称为\textbf{可积的(integrable)} 。 由此在小邻域内得到的子流形称为分布的\textbf{积分子流形(integral submanifold)}, 积分子流形的族称为 $M$ 局部的一个\textbf{正则叶理(regular foliation)}。 它将流形 $M$ 在局部上划分成了许多相互不交叠的叶片。

定理的证明大意如下。 不妨设 $M$ 是 $\mathbb{R}^n$ 中坐标原点的小邻域。 取这小邻域里 $\mathfrak{D}$ 的局部标架 $X_1,\cdots ,X_k$, 使得 $X_i(0)=\partial_i|_0$ (这总可以通过适当的旋转和缩放做到), 并定义映射 $\Pi: M\to \mathbb{R}^k$ 为
$$
\Pi(x^1,\cdots ,x^n)=(x^1,\cdots ,x^k).
$$
则对于 $p\in M$ 和 $p$ 处的切向量 $v=\sum_{i=1}^nv^i\partial_i|_p$, 有
$$
d\Pi(p)(v)=v^1\partial_1|_{\Pi(p)}+...+v^k\partial_k|_{\Pi(p)}.
$$
于是切丛的子丛 $\text{ker}d\Pi$ 的秩是 $n-k$, 且在原点的小邻域内与 $\mathfrak{D}$ 互补。 故 $(d\Pi|_{\mathfrak{D}})$ 实际上是从子丛 $\mathfrak{D}$ 到切丛 $T\Pi(M)$ 的同构, 因此 $(d\Pi|_{\mathfrak{D}})^{-1}$ 在丛 $T\Pi(M)=\text{span}(\partial_1,\cdots ,\partial_k)$ 上是良好定义的, 而且其像正是 $\mathfrak{D}$。 命 $V_i=(d\Pi|_{\mathfrak{D}})^{-1}\partial_i$, 则诸 $V_i$ 都是 $\mathfrak{D}$ 的截面, 根据对合性质, $[V_i,V_j]$ 也是 $\mathfrak{D}$ 的截面, 而且
$$
d\Pi[V_i,V_j]=[\partial_i,\partial_j]=0.
$$
于是 $[V_i,V_j]=0$, 因此它们实际上是坐标向量场。 因此 $\mathfrak{D}$ 是由坐标向量场张成的。

\subsection{等价表述}
设 $\mathfrak{D}$ 是 $k$ 维光滑分布。 如果微分 $p$-形式 $\omega$ 作用在 $\mathfrak{D}^{\otimes p}$ 的任何截面上都得到零, 则称 $\omega$ 消没 $\mathfrak{D}$。 显然两个消没 $\mathfrak{D}$ 的微分形式的外积仍然消没 $\mathfrak{D}$。 所有消没 $\mathfrak{D}$ 的微分形式组成 $M$ 的外微分代数的子代数, 称为 $\mathfrak{D}$ 的消没子(annihilator)。 根据外微分的计算法则, 可见 $\mathfrak{D}$ 为对合分布当且仅当对于任意消没 $\mathfrak{D}$ 的 $\omega$, $d\omega$ 仍旧消没 $\mathfrak{D}$。 

在流形 $M$ 上给定 $n-k$ 个线性无关的微分形式 $\omega_{k+1},...,\omega_{n}$ 后, 方程组
$$
\omega_{l}=0,\,l=+1,...,n-k
$$
称为一个\textbf{普法夫系 (Pfaffian system)}, 它确定了一个 $k$ 维分布, 即所有被 $\omega_{k+1},...,\omega_{n}$ 消没的子空间给出的分布。 因此, 费罗贝尼乌斯定理有下列等价形式:
\begin{theorem}{}
设 $\omega_{k},...,\omega_{n}$ 是 $M$ 上线性无关的微分形式, 则普法夫系
$$
\omega_{l}=0,\,l=k+1,...,n-k
$$
给出可积的 $k$ 维分布当且仅当每个 $d\omega_l$ 皆由 $\omega_{k+1},...,\omega_{n}$ 生成, 或等价地,
$$
d\omega_{l}\wedge\omega_{k+1}\wedge...\wedge\omega_{n}=0,\,\forall l.
$$
这称为费罗贝尼乌斯条件。
\end{theorem}

\subsection{应用举例}
费罗贝尼乌斯定理在微分几何中有广泛的应用。 大致来说, 它是"偏导数可交换"这一事实的推广: 偏微分方程组受制于偏导数可交换这一条件, 因此可积的偏微分方程组必须满足由此导出的必要条件, 而费罗贝尼乌斯定理断言这些条件也是可积的充分条件。

考虑 $n$ 维自变量 $x=(x^1,...,x^n)$ 的 $m$ 个未知函数 $u=(u^1,...,u^m)$ 的拟线性偏微分方程组:
\begin{equation}\label{eq_FrobTh_1}
\frac{\partial u^\alpha}{\partial x^i}=f_i^\alpha(x,u),\,1\leq\alpha\leq m;1\leq i\leq n.
\end{equation}
这里 $f_i^\alpha(x,z)$ 是已知的函数。 为了确定这个方程组的解, 需要的是一个“初始条件”, 即指定 $u$ 在某 $x_0$ 处的值。 方程的可解性问题如下:

\textbf{设有开集 $U\subset\mathbb{R}^n$, $V\subset\mathbb{R}^m$, 函数组 $f_i^\alpha(x,z)$ 定义在 $U\times V$ 上。 则对于 $x_0\in U$ 和 $z_0\in V$, 是否存在定义在 $U$ 上, 取值于 $V$ 中, 且满足\autoref{eq_FrobTh_1} 和初始条件 $u(x_0)=z_0$ 的函数组 $u=(u^1,...,u^m)$?}

如果 $u=u(x),u(x_0)=z_0$ 是这方程组的解, 那么根据偏导数的交换性, 显然可得
\begin{equation}\label{eq_FrobTh_2}
\frac{\partial f_i^\alpha}{\partial x^j}+\sum_{\beta=1}^mf_j^\beta\frac{\partial f_i^\alpha}{\partial z^\beta}
=\frac{\partial f_j^\alpha}{\partial x^i}+\sum_{\beta=1}^mf_i^\beta\frac{\partial f_j^\alpha}{\partial z^\beta},\forall i,j,\alpha
\end{equation}
在点 $(x_0,z_0)$ 处成立。 因此, 欲使得\autoref{eq_FrobTh_1} 对于任何初值都可积, 则\autoref{eq_FrobTh_2} 必须在 $U\times V$ 中处处成立。 显然\autoref{eq_FrobTh_2} 处处成立当且仅当向量场
$$
X_i=\frac{\partial}{\partial x^i}+\sum_{\alpha=1}^mf_i^\alpha(x,z)\frac{\partial}{\partial z^\alpha},\,i=1,...,n
$$
两两对易。 然而另一方面, $u=u(x)$ 是\autoref{eq_FrobTh_1} 的解当且仅当子流形 $\{(x,u(x)):x\in U\}$ 是 $n$ 维分布 $\text{span}(X_1,...,X_n)$ 的积分子流形。 因此, 根据费罗贝尼乌斯定理, 诸 $X_i$ 两两对易实为\autoref{eq_FrobTh_1} 对于任何初值均可积的充要条件。 故\autoref{eq_FrobTh_2} 在 $U\times V$ 上处处成立是\autoref{eq_FrobTh_1} 对于任何初值均可积的充要条件。

如果用普法夫系来表示则更简单。 命
\begin{equation}\label{eq_FrobTh_3}
\omega^\alpha:=dz^\alpha-\sum_{i=1}^nf_i^\alpha dx^i,1\leq\alpha\leq m.
\end{equation}
则\autoref{eq_FrobTh_1} 等价于 $U\times V$ 上的普法夫系 $\omega^\alpha=0,1\leq\alpha\leq m$,或者直接计算可见微分1-形式 $\{\omega^\alpha\}_{\alpha=1}^m$ 消没的那些切向量恰是 $m$ 维分布 $\text{span}\{X_i\}_{i=1}^n$。 \autoref{eq_FrobTh_2} 是这个普法夫系给出 $U\times V$ 上可积分布的充要条件, 而通过简单的外微分运算可看出它进一步等价于
$$
d\omega^\alpha=0,1\leq\alpha\leq m~.
$$
在整个 $U\times V$ 上恒成立。用这个办法来判定可积性非常方便。