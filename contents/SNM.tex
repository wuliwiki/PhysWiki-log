% 势能面
% license CCBYSA3
% type Wiki

(本文根据 CC-BY-SA 协议转载自原搜狗科学百科对英文维基百科的翻译)

势能面用某些参数,通常是原子的位置来描述一个系统(如原子集合)的能量。表面可以将能量定义为一个或多个坐标的函数;如果只有一个坐标,该表面称为势能曲线或能量剖面。莫尔斯/长程潜力就是一个例子。

使用地形图的类比是有帮助的:对于具有两个自由度(例如两个键长)的系统,能量的值(类比:陆地的高度)是两个键长的函数(类比:地面上位置的坐标)。[1]

势能面的概念适用于化学和物理等领域,特别是这些学科的理论分支。它可以用于理论上探索由原子组成的结构的性质,例如,找到分子的最小能量形状或计算化学反应的速率。

\subsection{数学定义和计算}
一组原子的几何位置可以用向量r来描述,向量$r$的元素代表原子的位置。向量$r$可以是原子笛卡尔坐标的集合,或者也可以是原子间距离和角度的集合。

给定r,能量作为位置的函数,$E(r)$是所有感兴趣的$r$的$E(r)$值。利用引言中的地形图类比,E给出了“能量地形图”的高度,从而产生了势能面的概念。

为了研究使用势能面作为原子位置函数的化学反应,有必要计算每个感兴趣的原子排列的能量。计算特定原子排列的能量的方法在计算化学文章中有很好的描述,这里的重点将是找到$E(r)$的近似值,以产生精细的能量位置信息。

对于非常简单的化学系统,或者当对原子间相互作用进行简化近似时,有时可以使用作为原子位置函数的能量的解析导出表达式。一个例子是伦敦-艾林-波兰尼-佐藤潜力[2][3][4] 作为三个氢-氢距离的函数的系统氢+ H2。

对于更复杂的系统来说,计算特定原子排列的能量对于表面的大规模表示来说通常太过昂贵而不可行。对于这些系统,一种可能的方法是仅计算势能面上的一组简化的点,然后使用计算成本更低的插值方法,例如谢泼德插值,来填充间隙。[5]

\subsection{应用}
势能面是一个概念工具,有助于分析分子几何和化学反应动力学。一旦在势能面上评估了必要的点,就可以根据能量相对于位置的一阶导数和二阶导数对这些点进行分类,这两个导数分别是梯度和曲率。静止点(或零梯度点)具有物理意义:能量最小值对应于物理稳定的化学物质,鞍点对应于过渡态,即反应坐标上的最高能量点(它是连接化学反应物和化学产物的最低能量路径)。

\subsection{吸引和排斥表面}
通过比较活化络合物中的键长相对于反应物和产物的键长,化学反应的势能面可以分为吸引或排斥。[6][7]对于A+B→C→A+B+C型反应,新形成的A–B键的键长延伸定义为$R^*_{AB} = R_{AB} - R^0_{AB}$,其中$R_{AB}$是过渡态的A–B键长,产物分子中的$R^0_{AB}$。类似地,对于在反应中断裂的键,$R^*_{BC} = R_{BC} - R^0_{BC}$,其中$R^0_{BC}$指的是反应物分子。[8]

对于放热反应,如果$R^*_{AB} > R^*_{BC}$,势能面被分类为有吸引力的(或早期下坡),因此当反应物彼此接近时达到转变状态。在转变状态之后,A–B键的长度继续减少,因此将多余的反应能转化为A–B键的形成能。[8][9] 一个例子是鱼叉反应K + Br$_2$ → K–Br + Br,其中反应物的结合过程导致类似K$\cdots$Br$\cdots$Br的活化络合物。[8] 产物分子的振动激发群可以通过近红外化学发光来检测。[10][11]

相比之下,反应H + Cl$_2$ → HCl + Cl的势能面是排斥性的(或后期下坡),因为当产物分离时,$R^*_{HCl} < R^*_{ClCl}$并达到转变状态。[^8][^9] 对于原子A(这里是H和Cl原子的反应),反应能量主要在作为产物的平移动能释放。对于一个反应,如H + H$_2$ → H + H$_2$,其原子A比B和C重,这些结合能量主要被转化为振动和平移,即使势能面是排斥的。

对于吸热反应,表面的类型决定了最能有效引发反应的能量类型。反应物的平移能量最有效地诱导与吸引表面的反应,而振动激发对与排斥表面的反应更有效。[8] 作为后一种情况的例子,对于相同的HCl,总能量, $F + HCl(v=1) \rightarrow Cl + HF$ 的速度大约是$F + HCl(v=0) \rightarrow Cl + HF$ 的五倍。[13]

\subsection{历史}
化学反应势能面的概念是法国物理学家勒内·马尔塞林在1913年首次提出的。[14] 亨利·艾林和迈克尔·波拉尼在1931年为H+ H2反应提出了势能面的第一个半经验计算。艾林在1935年用势能面计算过渡态理论中的反应速率常数。

\subsection{参考文献}
[1]
^Potential-energy (reaction) surface in Compendium of Chemical Terminology, 2nd ed. (the "Gold Book"). Compiled by A. D. McNaught and A. Wilkinson. Blackwell Scientific Publications, Oxford (1997).

[2]
^Sato, S. (1955). "A New Method of Drawing the Potential Energy Surface". Bulletin of the Chemical Society of Japan. 28 (7): 450. doi:10.1246/bcsj.28.450."On a New Method of Drawing the Potential Energy Surface". The Journal of Chemical Physics. 23 (3): 592. 1955. Bibcode:1955JChPh..23..592S. doi:10.1063/1.1742043..

[3]
^Keith J. Laidler, Chemical Kinetics (3rd ed., Harper & Row 1987) p.68-70 ISBN 0-06-043862-2.

[4]
^Steinfeld J.I., Francisco J.S. and Hase W.L. Chemical Kinetics and Dynamics (2nd ed., Prentice-Hall 1998) p.201-2 ISBN 0-13-737123-3.

[5]
^Moving least-squares enhanced Shepard interpolation for the fast marching and string methods, Burger SK1, Liu Y, Sarkar U, Ayers PW, J Chem Phys. 2009 130(2) 024103. doi: 10.1063/1.2996579..

[6]
^Attractive potential-energy surface in Compendium of Chemical Terminology, 2nd ed. (the "Gold Book"). Compiled by A. D. McNaught and A. Wilkinson. Blackwell Scientific Publications, Oxford (1997).

[7]
^Repulsive potential-energy surface in Compendium of Chemical Terminology, 2nd ed. (the "Gold Book"). Compiled by A. D. McNaught and A. Wilkinson. Blackwell Scientific Publications, Oxford (1997).

[8]
^Keith J. Laidler, Chemical Kinetics (3rd ed., Harper & Row 1987) p.461-8 ISBN 0-06-043862-2.

[9]
^Steinfeld J.I., Francisco J.S. and Hase W.L. Chemical Kinetics and Dynamics (2nd ed., Prentice-Hall 1998) p.272-4 ISBN 0-13-737123-3.

[10]
^Steinfeld J.I., Francisco J.S. and Hase W.L. Chemical Kinetics and Dynamics (2nd ed., Prentice-Hall 1998) p.263 ISBN 0-13-737123-3.

[11]
^Atkins P. and de Paula J. Physical Chemistry (8th ed., W.H.Freeman 2006) p.886 ISBN 0-7167-8759-8.

[12]
^Here v is the vibratonal quantum number..

[13]
^Atkins P. and de Paula J. Physical Chemistry (8th ed., W.H.Freeman 2006) p.889-890 ISBN 0-7167-8759-8.

[14]
^Computational Chemistry: Introduction to the Theory and Applications of Molecular and Quantum Mechanics Errol G. Lewars, 2nd ed. (Springer 2011) p.21 ISBN 978-9048138616.
