% 高锟
% license CCBYSA3
% type Wiki

(本文根据 CC-BY-SA 协议转载自原搜狗科学百科对英文维基百科的翻译)

\textbf{查尔斯·高锟爵士} GBM KBE FRS [1][2][3] (1933年11月4日至2018年9月23日)是一位物理学家和电气工程师,他率先在电信领域开发和使用光纤。20世纪60年代,高锟发明了多种方法将玻璃纤维和激光结合起来传输数字数据,这为互联网的发展奠定了基础。

被誉为“宽带教父” 、“光纤之父”、[4][5][6] 和“光纤通信之父”[7] 的高锟因“在光纤通信中光传输方面的突破性成就”获得2009年诺贝尔物理学奖。[8]

高锟出生于中国上海,是香港永久居民[9] ,并在英国和美国拥有公民身份。

\subsection{早期生活和教育}
高锟于1933年出生于中国上海,[10] 他的祖籍在金山附近,[10] 当时是一个独立的行政区。[11] 他和哥哥在家庭教师的指导下学习中国古典文学。[12][10] 他还在上海法租界的一所国际学校学习英语和法语,[13] 这所学校是由包括蔡元培在内的一些先驱的中国教育家创办的。[14]

高锟的家人于1948年移居台湾,随后又移居英属香港,[10][15] 1952年,他在圣约瑟夫学院完成了中学教育(香港中学会考,HKCEE的前身)。[16][17]他在伍尔维奇理工学院(现在的格林威治大学) 完成了电气工程本科学习,获得了工程学士学位。[10]

之后,他继续研究,并于1965年在伦敦大学获得电气工程博士学位,当时他是伦敦大学学院哈罗德·巴洛教授的旁听生,在英国哈洛的标准电信实验室(STL)工作,该实验室是标准电话和电缆的研究中心。[18] 正是在那里,高锟在亚历克·里维斯的管理下,作为一名工程师和研究员,与乔治·霍克汉姆并肩作战,完成了他的第一项开创性工作。

\subsubsection{1.1 祖先和家庭}
高锟的父亲高君湘[10] 是一名律师, 1925年从密歇根大学法学院获得法学博士学位。[19] 他是中国苏州大学(当时在上海)比较法学院的教授。[20][21]

他的祖父高燮是晚清时期的学者、诗人、艺术家[12] 和南方社会的领军人物。[22] 包括高旭、姚光在内的几位作家和高增也都是高锟的近亲。

他父亲的堂兄弟是天文学家高平子[12][23] (高平子环形山就是以他的名字命名的[24])。高锟的弟弟高铻是华盛顿特区美国天主教大学的土木工程师和荣誉退休教授。他的研究领域是流体力学。[25]

毕业后,高锟在伦敦遇到了他未来的妻子黄美芸,他们都在伦敦的标准电话电缆公司当工程师。[10][26] 黄美芸是英国华裔。[10] 他们于1959年在伦敦结婚,[10][27] 有两个孩子,一个儿子和一个女儿,[27] 他们都在加利福尼亚的硅谷生活和工作。[28][29][26] 根据高锟的自传,高锟是天主教徒,而他的妻子参加英国圣公会。[10]

\subsection{学术生涯}
\subsubsection{2.1 光纤和通信}
\begin{figure}[ht]
\centering
\includegraphics[width=6cm]{./figures/a28d89f54d895c5f.png}
\caption{一束用于光通信的二氧化硅玻璃纤维,这是全球事实标准。高锟还首次公开表示,高纯二氧化硅玻璃是一种理想的长距离光通信材料。[1]} \label{fig_GK_1}
\end{figure}
20世纪60年代,在位于埃塞克斯哈洛的标准电信实验室(STL),高锟和他的同事在实现光纤作为电信媒介方面做了开创性的工作,证明现有光纤的高损耗是由玻璃中的杂质引起的,而不是技术本身的根本问题。[30]

1963年,当高锟首次加入光通信研究团队时,他记录总结了当时的背景情况[31] 和可用技术,并确定了涉及的关键人物。[31] 最初,高锟在安东尼·卡尔鲍伊克(Toni Karbowiak)的团队工作,他在亚历克·里维斯手下研究通信光波导。高锟的任务是研究纤维衰减,为此他从不同的纤维制造商那里收集样本,并仔细研究大块玻璃的性能。高锟的研究初步让他相信是材料中的杂质导致了这些纤维的高光损失。[32] 那年晚些时候,高锟被任命为标准电信实验室光电研究小组的组长。[33] 他于1964年12月接管了标准电信实验室的光通信项目,因为他的导师卡尔鲍伊克(Karbowiak)去了澳大利亚悉尼新南威尔士大学(UNSW)电气工程学院,担任通信教授。[34]

虽然高锟接替卡尔鲍伊克担任光通信研究经理,但他立即决定放弃卡尔博维克的计划(薄膜波导),与同事乔治·霍克汉姆(George Hockham)一起全面改变了研究方向。[32][34] 他们不仅考虑了光学物理,还考虑了材料特性。1966年1月,高锟在伦敦向英国电机工程师学会首次提交了研究结果,并于7月与乔治·霍克汉(1964-1965年与高锟合作)一起发表了进一步的研究成果。[35] 这项研究首先从理论上提出用玻璃纤维实现光通信,所描述的思想(特别是结构特征和材料)在很大程度上是当今光纤通信的基础。

1965年,[33][36] 霍克汉姆和高锟得出结论,玻璃光衰减的基本极限低于20 dB/km(分贝/km,是信号在一定距离内衰减的量度),这是光通信的关键阈值。[37] 然而,在测定时,光纤通常表现出高达1000分贝/公里甚至更高的光损耗。这一结论开启了寻找低损耗材料和适合达到这一标准的合适纤维的激烈竞争。

高锟和他的新团队(成员包括戴维、琼斯和赖特)通过测试各种材料来努力达成这个目标。他们精确测量了不同波长的光在玻璃和其他材料中的衰减。在此期间,高锟指出高纯度的熔融石英(二氧化硅)是光通信的理想候选。高锟还指出,玻璃材料的杂质是玻璃纤维内部光传输急剧衰减的主要原因,而不是像当时许多物理学家认为的像散射这样的基本物理效应,这种杂质可以被去除。这导致了高纯度玻璃纤维的全球研究和生产。[38] 当高先生第一次提出这种玻璃纤维可以用于远距离信息传输,并且可以代替那个时代用于电信的铜线时,他的想法被广泛地怀疑;后来人们才意识到高锟的想法彻底改变了整个通信技术和行业。[39]

他还在光通信的工程和商业实现的早期阶段发挥了主导作用。 1966年春天,高锟前往美国,但未能引起贝尔实验室(Bell Labs)的兴趣,当时贝尔实验室是STL在通信技术方面的竞争对手。[40] 他后来去了日本并获得了支持。[40] 高锟参观了许多玻璃和聚合物工厂,与包括工程师、科学家、商人在内的许多人讨论了玻璃纤维制造的技术和改进。1969年,高锟和琼斯测量了体积熔融石英的固有损耗为4 dB/km,这是超透明玻璃可能存在的第一个证据。贝尔实验室开始认真考虑光纤。[40]

高锟开发了玻璃纤维波导的重要技术和配置,并为满足民用和军用应用要求的不同纤维类型和系统设备以及光纤通信外围支持系统的开发做出了贡献。 20世纪70年代中期,他在玻璃纤维疲劳强度方面做了开创性的工作。 当被任命为第一位ITT执行科学家时,高锟推出了“兆比特技术”计划,解决信号处理的高频极限,因此高锟也被称为“兆比特技术概念之父”。[41] 高锟已经发表了100多篇论文,获得了30多项专利, 其中包括防水高强度纤维(与马克拉德公司合作)。[42]

在光纤开发的早期阶段,高锟已经强烈地倾向于使用单模进行长距离光通信,而不是使用多模系统。他的设想后来被采纳,现在几乎被完全应用。[38][43]高锟也是现代海底通信电缆的梦想家,并在很大程度上推动了这一想法。他在1983年预测,世界海洋将遍布光纤,比这种跨海洋光纤电缆首次投入使用提前了五年。[44]

阿里·贾文引进稳定的氦氖激光器和高锟发现光纤损耗特性在现在被认为是光纤通信发展的两个重要里程碑。[34]

\subsubsection{2.2 之后的工作}
高锟于1970年加入香港中文大学(CUHK)成立了电子系,后来成为电子工程系。在此期间,高锟是读者,然后是香港中文大学的电子系教授;他建立了电子专业的本科生和研究生课程,并管理他的第一批学生毕业。在他的领导下,香港中文大学成立了教育学院和其他新的研究机构。他于1974年回到美国国际电视电报公司(当时卫星电视公司的母公司),在弗吉尼亚州罗诺克工作,先是担任首席科学家,后来担任工程总监。1982年,他成为第一位国际电视电报公司执行科学家,常驻在康涅狄格州的高级技术中心。[45] 在那里,他担任耶鲁大学特鲁姆布尔学院的兼职教授和研究员。1985年,高锟在西德SEL研究中心呆了一年。1986年,高锟担任国际电话电报公司研究公司董事。

他是最早研究香港填海造地对环境影响的人之一,并于1972年在爱丁堡举行的英联邦大学协会(ACU)会议上陈述了他的第一份相关研究报告。[45]

高锟在1987年至1996年期间担任香港中文大学副校长。[46] 自一1991年起,高锟成为香港精电国际有限公司独立非执行董事及审核委员会成员。[47][48] 1993年至1994年,他担任东南亚高等教育机构协会主席。[49] 1996年,高锟向耶鲁大学捐款,并设立了高琨基金研究基金,以支持耶鲁在亚洲的研究、研究和创新项目。[50] 该基金目前由耶鲁大学东亚和东南亚研究委员会管理。[51]。1996年从香港中文大学退休后,高锟在伦敦帝国理工学院电气与电子工程系休了六个月的假;从1997年到2002年,他还在同一系担任客座教授。[52]

高先生曾担任香港能源咨询委员会主席及委员两年,并于2000年7月15日退休。[53][54] 高锟先生是香港创新科技顾问委员会成员,于2000年4月20日获委任。[55] 2000年,高锟参与创立了位于香港数码港的独立学校基金会学院。[56] 他在2000年是ISF的创始主席,并于2008年12月从ISF董事会卸任。[56] 高锟是2002年在台湾台北举行的电气电子工程师协会全球电信展的主讲人。2003年,高锟被任命为国立台湾大学电子工程与计算机科学学院电子研究所的教授。 高锟随后担任香港电信咨询公司环球科技服务有限公司的董事长兼首席执行官。他是ITX服务有限公司的创始人、董事长兼首席执行官。自2003年至2009年1月30日,高锟担任独立非执行董事及未来媒体审计委员会委员。[57][58]

\subsection{荣誉和奖项}
高锟获得了许多荣誉和奖项,其中最著名的是诺贝尔物理学奖。他的奖项包括:
\subsubsection{3.1 荣誉}
\begin{itemize}
\item 1993年:英帝国高级勋爵士。[59]
\item 2010年:英帝国最高级巴思爵士。[60][60]
\item 2010年:香港特别行政区大紫荆勋章。[61]
\end{itemize}
\subsubsection{3.2 社会和学术认可}
\begin{itemize}
\item 美国电气和电子工程师学会终身研究员(1979年选举)[61]
\item 英国工程技术学院研究员
\item 1997年当选为皇家学会会员[62]
\item 英国皇家工程学院研究员(1989年选举)
\item 美国马可尼学会会员(1985年选举)
\item 香港工程科学院名誉研究员(1994年选举)和前院长[63]
\item 香港电脑学会杰出研究员(1989年选举)[64][65]
\item 香港工程师学会荣誉院士(1994年选举)[66]
\item 台北中央研究院院士(1992年选举)[67]
\item 美国光学学会会员[68]
\item 奥地利欧洲科学院成员
\item 美国国家工程学院成员(1990年选举)
\item 瑞典皇家工程科学院外国成员(1988年选举)
\item 北京中国科学院外籍成员(1996年选举)
\item 耶鲁大学特兰布尔学院研究员
\item 伦敦大学玛丽女王荣誉研究员[69]
\item 香港中文大学名誉教授(1996年获委任)[70]
\item 北京大学名誉教授,北京(1995年任命)
\item 北京清华大学名誉教授(1995年任命)
\item 北京国际经济贸易大学名誉教授(1995年任命)
\item 北京邮电大学名誉教授(1995年任命)
\item 台北国立台湾大学特别委任讲座教授(2003年委任)[71]
\item 香港城市大学电子工程系荣誉教授(1997-2002)[71]
\item 香港城市大学终身名誉教授(2002年1月1日任命)[71]
\item 澳门科学技术委员会顾问[72]
\end{itemize}

\subsubsection{3.3 荣誉学位}
\begin{figure}[ht]
\centering
\includegraphics[width=6cm]{./figures/91c161c6f56e5ce4.png}
\caption{亚历山大·格雷厄姆·贝尔,电信先驱,伦敦大学学院校友,1876年被授予美国第一个电话专利。90年后的1966年,高锟和霍克汉姆在《光纤通信》上发表了他们开创性的文章。高锟也是伦敦大学学院校友,并于一九八五年获颁著名的IEEE亚历山大·格雷厄姆·贝尔奖章。2010年,高锟被伦敦大学学院授予荣誉博士学位。} \label{fig_GK_2}
\end{figure}
\begin{itemize}
\item 英属香港香港香港中文大学荣誉理学博士[73] (1985年)
\item 英国苏塞克斯大学理学博士[73] (1990年)
\item 台湾国立交通大学工程学博士(1990年)[74][75]
\item 日本创价大学荣誉博士学位(1991年)
\item 英国格拉斯哥大学工程博士(1992年)
\item 英国达勒姆大学荣誉学院(1994年)[76]
\item 澳大利亚格里菲斯大学大学博士(1995年)
\item 意大利帕多瓦大学“电信工程”荣誉学位(1996年10月18日)[77]
\item 英国赫尔大学理学博士(1998年)[78]
\item 美国耶鲁大学理学博士(1999年)[79]
\item 英国格林威治大学荣誉理学博士(2002年)[80]
\item 美国普林斯顿大学理学博士(2004年)[80]
\item 加拿大多伦多大学荣誉法学博士学位(2005年6月16日)[81]
\item 北京邮电大学荣誉博士(2007年)
\item 英国伦敦大学学院荣誉科学博士学位(2010年)[82]
\item 英国斯特拉斯克莱德大学荣誉学位(2010年9月24日)[83]
\item 中华人民共和国香港大学荣誉理学博士(2011年)[84]
\end{itemize}

\subsubsection{3.4 奖项}
\begin{figure}[ht]
\centering
\includegraphics[width=6cm]{./figures/4ec7c56b12f53066.png}
\caption{古列尔莫·马可尼,无线通信的先驱,获得了1909年诺贝尔物理学奖的一半。2009年,在马可尼获诺贝尔奖100周年,高锟因其在光纤领域的开创性工作而获得了该奖项的一半。高锟还于1985年获得马可尼奖,并是马可尼学会的会员。} \label{fig_GK_3}
\end{figure}
高锟把他的大部分奖牌捐给了香港中文大学。[59]
\begin{itemize}
\item 1976年:美国陶瓷学会莫雷奖。
\item 1977年:美国富兰克林研究所斯图亚特·百龄坛奖章。[59]
\item 1978年:英国等级信托基金等级奖。
\item 1978年:IEEE莫里斯·李伯曼纪念奖。引文:“通过发现、发明和开发玻璃纤维波导的材料、技术和配置,使光频率下的通信变得实用,特别是通过在大块玻璃中的仔细测量,识别和证明硅玻璃可以提供实用通信系统所需的必要的低光损耗”。
\item 1979年:瑞典爱立信国际奖。[59]
\item 1980年:美国亚足联金奖。
\item 1981年:美国南加州经济、社会和文化委员会成就奖。
\item 1983年:美国亚洲研究所USAI成就奖。[59]
\item 1985年:美国电气工程师协会亚历山大·格雷厄姆·贝尔奖章。[59]
\item 1985年:美国马可尼基金会马可尼国际科学家奖。
\item 1985年:意大利热那亚市哥伦布奖章。
\item 1986年:美国国际经济学院-美国年度奖国际经济学院成就奖。[85]
\item 1987年:日本通信和计算机促进基金会C & C奖。
\item 1989年:英国电气工程师学会法拉第奖章。[59]
\item 11989年:美国物理学会詹姆斯·麦克格罗德新材料奖。引文:“对材料研究和开发的贡献,开发了实用的低损耗光纤,这是光通信技术的基石之一”。[86]
\item 1992年:SPIE学会金奖。[87]
\item 1995年:英国世界工程组织联合会工程卓越金奖。[59]
\item 1996年:英国皇家工程学院菲利普亲王奖章;“他的开创性工作导致了光纤的发明,并在工程和商业实现方面发挥了领导作用;他对香港高等教育作出的卓越贡献。[59]
\item 1996年:意大利帕多瓦市。[59]
\item 1996年:第12届日本奖。引文:“对宽带、低损耗光纤通信的开创性研究”。
\item 1998年:英国国际教育学院国际演讲奖章。[59][88]
\item 1999年:美国查尔斯·斯塔克德雷珀奖(罗伯特·莫伊雷尔和约翰·麦克切斯尼共同获奖)。[59]
\item 2001年:香港千禧杰出工程师奖。[59]
\item 2006年:HKIE金奖,香港HKIE(香港工程师学会)。[89]
\item 2009年:瑞典诺贝尔物理学奖(该奖的1/2)。引文:“关于光在光纤中传输的开创性成就”。[90]
\item 2009年:IEEE光子学会牌匾。[91]
\item 2010年(2月27日):杰出科学技术奖,2010年度亚裔美国工程师奖,美国亚奥伊2010。[92]
\item 2010年(3月27日):2009/2010世界中国大奖,凤凰卫视,香港。[93][94]
\item 2010年(4月8日/9日):美国旧金山华裔杰出奖。[95]
\item 2014年2月20日:FTTH运营商奖和个人奖[96]
\end{itemize}

\subsubsection{3.5 命名}
\begin{figure}[ht]
\centering
\includegraphics[width=6cm]{./figures/fe9b0f6eacd03883.png}
\caption{香港科学园的标志性礼堂自2009年12月30日起以高锟的名字命名。} \label{fig_GK_4}
\end{figure}
\begin{itemize}
\item 3463号高锟,发现于1981年,1996年以高锟的名字命名。
\item 1996年(11月7日):香港中文大学科学中心北翼被命名为高锟大楼。[70]
\item 2009年(12月30日):香港科学园的地标性礼堂以高锟-高锟礼堂命名。[97][98]
\item 2010年(3月18日):高琨教授广场,独立学校基金会学院的广场。[99]
\item 2014年(9月):高琨爵士美国联合技术公司(现称BMAT STEM学院)开幕。[100]
\end{itemize}

\subsubsection{3.6 其他奖项}
\begin{itemize}
\item 伦敦科学博物馆展出。
香港事务顾问(1994年5月至1997年6月30日)。[101][102]
\item 1999年:世纪亚洲,科学和技术。[103][103]
\item 2002年:香港星岛创新科技类年度领袖。[59]
\item 2002年10月21日:入选工程名人堂,50周年发行,电子设计。[104][105]
\item 2008年1月3日:参加英国文化委员会在香港成立60周年庆典。[106][107]
\item 2009年11月4日:荣誉公民身份和美国加利福尼亚州山景城的高锟博士日。[108]
\item 2009年:香港年度人物。[109]
\item 2009年亚洲十大成就——第7名。[110]
\item 2010年(2月):美国100人委员会获奖者。[95]
\item 2010年OFC/NFOEC会议于3月23日至25日在美国加利福尼亚州圣地亚哥举行。[111][112]
\item 2010年5月14日至15日:在中国上海举行的第19届年度无线和光通信会议(WOCC,2010年)上,为高教授举行了两场会议。[113][114]
\item 2010年5月22日:入选2010年上海世博会纪念品档案。[115]
\item 2010年年中:香港特别行政区香港通用邮票小型张(第一号)。[116]
\item 2011年3月25日:英国埃塞克斯郡哈洛市揭开蓝色匾。[117]
\item 2014年11月4日:高锟的生日那天被定为FTTH全球联盟委员会给我纤维日。[118]
\end{itemize}

\subsection{晚年与死亡}
高锟的国际旅行让他认为他属于世界而不是任何国家。[119][120] 2010年,高锟夫妇在一封公开信中澄清道:“查尔斯在香港念高中,在这里教书,他是香港中文大学的副校长,后来也在这里退休。所以他是香港人。”[121]

制陶是中国的传统手工艺,也是高锟的爱好。高锟还喜欢看武侠小说。[122]

2009年10月6日,高锟因其在光纤中的光传输和光纤通信研究中的贡献而被授予诺贝尔物理学奖,[123] 他说,“我简直说不出话来,从未想过会获得这样的荣誉。”[124][124] 高锟的妻子格温告诉媒体,在向美国政府纳税后,奖金将主要用于查尔斯的医疗费用。[125] 2010年,查尔斯·高和格温·高建立了查尔斯·高阿尔茨海默病基金会,以提高公众对该疾病的认识,并为患者提供支持。

高锟从2004年初起就患有阿尔茨海默氏症,有语言障碍,但识别人或地址没有问题。[126] 高锟的父亲也患有同样的疾病。从2008年开始,为了住在他的孩子和孙儿附近,他从香港搬去美国加州山景城。[127]

2016年,高锟失去了保持平衡的能力。在痴呆症末期,他由妻子照顾,不打算靠生命补给维持生存,也不打算给自己做心肺复苏术。[127] 高锟先生于2018年9月23日在香港布拉德伯里临终关怀医院去世,享年84岁。[128][129][130][131]

\subsection{笔记}
a:  Kao's major task was to investigate light-loss properties in materials of optic fibers, and determine whether they could be removed or not. Hockham's was investigating light-loss due to discontinuities and curvature of fibre.   

b:  Some sources show around 1964, [132] [133] for example, " By 1964, a critical and theoretical specification was identified by Dr. Charles K. Kao for long-range communication devices, the 10 or 20 dB of light loss per kilometer standard." from Cisco Press. [132]    

c:  In 1980, Kao was awarded the Gold Medal from American Armed Forces Communications and Electronics Association, " for contribution to the application of optical fiber technology to military communications". [134]    

d:  In the United States National Academy of Engineering Membership Website, Kao's country is indicated as People's Republic of China. [134]    

e:   OFC/NFOEC – Optical Fiber Communication Conference and Exposition/National Fiber Optic Engineers Conference [134]

\subsection{参考文献}
[1]
^Charles K. Kao was elected in 1990 as a member of National Academy of Engineering in Electronics, Communication & Information Systems Engineering for pioneering and sustained accomplishments towards the theoretical and practical realization of optical fibre communication systems..

[2]
^"- Royal Society"..

[3]
^"The Fellowship – List of Fellows". Raeng.org.uk. Retrieved October 26, 2009..

[4]
^dpa (October 6, 2009). "PROFILE: Charles Kao: 'father of fibre optics,' Nobel winner". Earthtimes. Retrieved November 30, 2009..

[5]
^Record control number (RCN):31331 (October 7, 2009). "'Father of Fibre Optics' and digital photography pioneers share Nobel Prize in Physics". Europa (web portal). Archived from the original (cfm) on January 25, 2008. Retrieved November 30, 2009..

[6]
^Bob Brown (Network World) (October 7, 2009). "Father of fiber-optics snags share of Nobel Physics Prize". cio.com.au. Retrieved November 30, 2009..

[7]
^"Prof. Charles K Kao speaks on the impact of IT in Hong Kong". The Open University of Hong Kong. January 2000. Retrieved December 24, 2009..

[8]
^The Nobel Prize in Physics 2009. Nobel Foundation. October 6, 2009. Retrieved October 6, 2009..

[9]
^高锟. 香港百人 (in 粤语, 中文, and 英语). Asia Television. 2011..

[10]
^Kao, Charles K. (2013) [original Chinese translation published in 2005]. 潮平岸阔——高锟自传 [A Time And A Tide: Charles K. Kao ─ A Memoir] (autobiography). Translated by 许迪锵 ("First" ed.). Joint Publishing (Hong Kong). ISBN 978-962-04-3444-0..

[11]
^历史沿革. Government of Jinshan District, Shanghai. Retrieved 27 September 2018..

[12]
^范彦萍 (10 October 2009). 诺贝尔得主高锟的堂哥回忆:他儿时国学功底很好 [Interview of Kao's cousin]. Youth Daily. Shanghai. Retrieved October 9, 2009 – via eastday.com..

[13]
^高锟. 杰出华人系列 (documentary and oral history) (in 粤语, 中文, and 英语). Radio Television Hong Kong. 2000. Event occurs at 12:00 to 13:00. Retrieved 27 September 2018..

[14]
^陶家骏 (June 1, 2008). "Archived copy" 著名女教育家陶玄 [Famous Female Educator Tao Xuan]. 绍兴县报 [Shaoxing County News]. Archived from the original on March 13, 2012. Retrieved October 9, 2009.CS1 maint: Archived copy as title (link).

[15]
^“光纤之父”高锟离世 享年84岁 (16:56). Online instant news section. Ming Pao. Hong Kong: Media Chinese International. 23 September 2018. Retrieved 27 September 2018..

[16]
^"How Hong Kong's public exam system evolved for secondary school pupils". 2018-09-29..

[17]
^"【高锟病逝】展览怀缅光纤之父 会考证书曝光数学只攞Credit"..

[18]
^"Prof Charles K. Kao". Department of Electronic & Electrical Engineering. University College London. Archived from the original on 14 September 2010. Retrieved 27 September 2018..

[19]
^University of Michigan Law School: Alphabetical List with Year of Law School Graduates.

[20]
^高君湘_法律学人_雅典学园..

[21]
^中国近代法律教育与中国近代法学. Archived from the original on July 8, 2011..

[22]
^参加南社纪念会姓氏录 [List of Nan Society member]. 南社研究网 [Research of Nan Society]. Archived from the original on November 21, 2008. Retrieved October 8, 2009..

[23]
^高平子先生简介. 青岛天文网--中国科学院紫金山天文台青岛观象台/青岛市天文爱好者协会. February 8, 2006. Archived from the original on July 7, 2011. Retrieved October 8, 2009..

[24]
^"Lunar Crater Statistics". NASA. Archived from the original on August 13, 2009. Retrieved October 8, 2009..

[25]
^高锟个人简历 [The biography of Charles K. Kao]. chinanews.com.cn. October 6, 2009. Retrieved October 9, 2009..

[26]
^光纤与爱情——高锟一生的实验. Ming Pao. Hong Kong. March 4, 2000. Retrieved October 7, 2009 – via networkchinese.com..

[27]
^高锟履历 [resume of Kao Kuen]. Wen Wei Po. Hong Kong. 7 October 2009. Retrieved 27 September 2018..

[28]
^高锟. 杰出华人系列 (documentary and oral history) (in 粤语, 中文, and 英语). Radio Television Hong Kong. 2000. Event occurs at around 20:00. Retrieved 27 September 2018..

[29]
^"Draper Prize". draper.comg. Archived from the original on February 14, 2010. Retrieved November 4, 2009. "Charles Kao is credited for first publicly proposing the possibility of practical telecommunications using fibers in the 1960s.".

[30]
^Montgomary, Jeff D. (March 22, 2002). "Chapter 1 – History of Fiber Optics". In DeCusatis, Casimer. Fiber optic data communication: technological trends and advances (1st ed.). Academic Press. 1.3.1. Long Road to Low-Loss Fiber (pp. 9–16). ISBN 978-0-12-207891-0..

[31]
^"Charles Kao's Notes made in 1963 – Set A". March 23, 2016..

[32]
^Jeff Hecht. "A Short History of Fiber Optics". Archived from the original on June 13, 2010. Retrieved October 8, 2010..

[33]
^"Communication pioneers win 2009 physics Nobel". IET. October 7, 2009. Archived from the original on October 13, 2009. Retrieved October 28, 2009..

[34]
^"Fiber Types in Gigabit Optical Communications" (PDF). Cisco Systems, USA. April 2008. Retrieved November 3, 2009..

[35]
^Kao, K. C.; Hockham, G. A. (1966). "Dielectric-fibre surface waveguides for optical frequencies". Proc. IEE. 113 (7): 1151–1158. doi:10.1049/piee.1966.0189..

[36]
^Maryanne C. J. Large; Leon Poladian; Geoff Barton; Martijn A. van Eijkelenborg. (2008). Microstructured Polymer Optical Fibres. Springer. ISBN 978-0-387-31273-6. Page 2.

[37]
^"Chapter 1.1 – The Evolution of Fibre Optics" (PDF). Archived from the original (PDF) on August 31, 2011. Retrieved October 28, 2009..

[38]
^"2009 Nobel Prize in Physics – Scientific Background: Two revolutionary optical technologies – Optical fiber with high transmission" (PDF). Nobelprize.org. October 6, 2009. Archived from the original (PDF) on November 22, 2009. Retrieved December 4, 2009..

[39]
^1999 Charles Stark Draper Award Presented "Kao, who was working at ITT's Standard Telecommunications Laboratories in the 1960s, theorized about how to use light for communication instead of bulky copper wire and was the first to publicly propose the possibility of a practical application for fibre-optic telecommunication.".

[40]
^"A Fiber-Optic Chronology (by Jeff Hecht)". Archived from the original on June 13, 2010. Retrieved November 3, 2009..

[41]
^Technology of Our Times: People and Innovation in Optics and Optoelectronics (SPIE Press Monograph Vol. PM04), by Frederick Su; SPIE Publications (July 1, 1990); ISBN 0-8194-0472-1, ISBN 978-0-8194-0472-5. Page 82–86, Terabit Technology, by Charles K. Kao..

[42]
^"Water resistant high strength fibers (United States Patent 4183621)" (PDF). January 15, 1980 [date filed: December 29, 1977]. Retrieved November 1, 2009..

[43]
^"Guiding light". May 1989. Archived from the original (PDF) on December 16, 2009. Retrieved December 4, 2009..

[44]
^"1, A Global Footprint" (PDF). Building the Global Fiber Optics Superhighway (Free Abstract). Springer USA. May 8, 2007. ISBN 978-0-306-46505-5. Retrieved November 3, 2009. ISBN 978-0-306-46979-4 (Online).

[45]
^"The father of optical fiber – Narinder Singh Kapany/Prof. C. K. Kao" (in 中文 and 英语). networkchinese.com. Retrieved October 8, 2009..

[46]
^CUHK Handbook Archived 12月 9, 2008 at the Wayback Machine.

[47]
^"Annual Report 2002, Varitronix International Limited" (PDF). Varitronix International Ltd. April 3, 2003. Retrieved November 1, 2009..

[48]
^"精电国际有限公司" (PDF) (in 中文 and 英语). 精电国际有限公司. 2004. Retrieved November 1, 2009..

[49]
^"President of ASAIHL". ASAIHL. Archived from the original on July 4, 2015. Retrieved November 1, 2009..

[50]
^"Kao Gift Will Help Build Ties Between Asia and Yale". Yale Bulletin and Calendar, News Stories. June 24 – July 22, 1996. Archived from the original on June 11, 2009. Retrieved November 30, 2009..

[51]
^"Fellowships and research support" (php). The Councils on East Asian and Southeast Asian Studies at Yale University. Retrieved November 30, 2009..

[52]
^"Research Awards and Honours". Imperial College London Department of Electric and Electronic Engineering. 2009. Retrieved December 24, 2009..

[53]
^"Appointment of Chairman and Members of the Energy Advisory Committee". Hong Kong Government. August 11, 2000. Retrieved November 3, 2009..

[54]
^"EPD – Advisory Council on the Environment". Environmental Protection Department, The Government of Hong Kong SAR. April 28, 2006. Retrieved November 3, 2009..

[55]
^"The Council of Advisors on Innovation & Technology appointed" (PDF). The Government of Hong Kong SAR. April 20, 2000. Archived from the original (PDF) on July 22, 2011. Retrieved November 3, 2009..

[56]
^"Founding Chairman receives 2009 Nobel Prize for Physics" (php). The ISF Academy. Retrieved November 1, 2009..

[57]
^壹传媒(00282)高锟辞任独立非执董及审核委员,黄志雄接任. jrj.com.cn. July 2, 2009. Retrieved November 1, 2009..

[58]
^中研院士高锟 勇夺物理奖. Apple Daily. Taiwan. October 7, 2009. Retrieved November 1, 2009..

[59]
^"Medals Donated to CUHK by Professor Kao". The Chinese University of Hong Kong. Retrieved December 24, 2009..

[60]
^"2010 Queen's Birthday Honours List" (PDF). The London Gazette. June 12, 2010. Supplement No.1 B23. Retrieved June 12, 2010..

[61]
^"306 people to receive honours". The Government of Hong Kong SAR. July 1, 2010. Retrieved July 1, 2010.[失效连结].

[62]
^"Fellows of the Royal Society". London: Royal Society. Archived from the original on 2015-03-16..

[63]
^高锟:厚道长者 毕生追求 (shtm). news.sciencenet.cn (科学网·新闻). 2009-10-14. Retrieved July 11, 2010..

[64]
^"Membership – Hong Kong Computer Society Annual Report 2008-2009". Hong Kong Computer Society. Archived from the original on July 21, 2011. Retrieved October 26, 2009..

[65]
^"List of Distinguished Fellows". The Hong Kong Computer Society. Archived from the original (asp) on May 7, 2010. Retrieved May 21, 2010..

[66]
^"The HKIE – News". The Hong Kong Institute of Engineers (HKIE). October 7, 2009. Archived from the original (asp) on July 21, 2011. Retrieved July 19, 2010..

[67]
^"中央研究院院士"..

[68]
^"OSA Nobel Laureates". Optical Society of America (OSA). Archived from the original (aspx) on October 29, 2009. Retrieved October 26, 2009..

[69]
^"e-Newsletter, Alumni at Queen Mary, University of London". Qmw.ac.uk. Retrieved October 26, 2009.[永久失效连结].

[70]
^高锟校长荣休志念各界欢送惜别依依, a September 1996 article from the Chinese University of Hong Kong alumni website (中文).

[71]
^"Charles K. Kao, NTU's former chair professor by special appointment, wins the Nobel Prize in Physics". National Taiwan University. Archived from the original on July 19, 2011. Retrieved November 1, 2009..

[72]
^XinhuaNet News: Macao chief congratulates Nobel Prize winner Charles Kao.

[73]
^"Honorary Professors and Emeritus Professors". Chinese University of Hong Kong. n.d. Archived from the original on 20 July 2011. Retrieved 27 September 2018..

[74]
^国立交通大学 公共事务委员会 名誉博士名单 (php). National Chiao Tung University. Retrieved October 26, 2009..

[75]
^校史 – 国立交通大学时期|民国六十八年(一九七九)以后. National Chiao Tung University (NCTU). Archived from the original on March 26, 2010. Retrieved October 26, 2009..

[76]
^"Honorary Degrees" (PDF). Retrieved October 26, 2009..

[77]
^Università degli Studi di Padova – Honoris causa degrees Archived 9月 5, 2009 at the Wayback Machine.

[78]
^"Honorary graduates 2 – University of Hull". Archived from the original on December 19, 2016..

[79]
^"Yale Honorary Degree Recipients". Archived from the original on May 21, 2015..

[80]
^"meantimealumni Spring 2005" (PDF). University of Greenwich. Archived from the original (PDF) on October 9, 2011. Retrieved October 7, 2009..

[81]
^"Engineering a World of Possibilities" (PDF). University of Toronto Applied Science & Engineering. Spring 2006. Retrieved October 26, 2009.[永久失效连结].

[82]
^"UCL Fellows and Honorary Fellows announced". June 17, 2010. Retrieved June 19, 2010..

[83]
^"Honorary degree for broadband pioneer". September 24, 2010. Archived from the original on September 30, 2010. Retrieved September 27, 2010..

[84]
^"HKU Honorary Graduates - Graduate Detail" (Press release). The University of Hong Kong. 2011 [circa]. Retrieved 25 September 2018..

[85]
^"CIE-USA ANNUAL AWARDS" (PDF) (in 英语 and 中文). CIE-USA. 2007. Archived from the original (PDF) on July 25, 2011. Retrieved April 2, 2010..

[86]
^"Prize Recipient"..

[87]
^"Gold Medal Award - SPIE"..

[88]
^News from the Institution of Electrical Engineer (PDF). IEE. June 1998. Archived from the original (PDF) on July 6, 2011. Retrieved November 3, 2009..

[89]
^The Hong Kong Institute of Engineers: Press Releases – 香港工程师学会荣誉大奖、会长特设成就奖及杰出青年工程师奖2006 (The HKIE Gold Medal Award, the President's Award & Young Engineer of the Year Award 2006)[永久失效连结].

[90]
^"The Nobel Prize in Physics 2009". NobelPrize.org. Retrieved 2018-09-25..

[91]
^"Research Highlights". IEEE Photonics Society. Archived from the original on July 27, 2011. Retrieved October 16, 2010..

[92]
^"美洲中国工程师学会2010年工程奖章得奖名单出炉(2/27)" (asp) (in 中文 and 英语). AAEOY. 2010-02-23. Retrieved February 23, 2010..

[93]
^华裔科学家高锟荣获影响世界华人大奖. Xinhua News Agency. March 11, 2010. Archived from the original on March 16, 2010. Retrieved March 11, 2010..

[94]
^华裔科学家高锟荣获影响世界华人大奖 (shtml). Phoenix Television. 2010-03-11. Retrieved March 11, 2010..

[95]
^Jane Leung Larson (February 2010). "2009 Nobel Laureate Charles Kao among Committee of 100 Honorees in San Francisco". Committee of 100. Retrieved March 14, 2010..

[96]
^"Vodafone and Sir Charles Kao recognised in FTTH Awards 2014" (PDF). FTTH Council Europe. 20 Feb 2014. Retrieved 28 Jan 2015..

[97]
^"香港两座建筑物将以高锟及饶宗颐名字命名(图) (Two landmark buildings in Hong Kong are named after Charles K. Kao and Rao Zongyi (with photos))" (shtml). 凤凰资讯 (Ifeng News). December 30, 2009. Retrieved January 3, 2009..

[98]
^"Hong Kong to name building after Nobel laureate Charles Kao". chinaview.cn. December 31, 2009. Archived from the original on November 4, 2012. Retrieved January 3, 2009..

[99]
^The ISF Academy Newsletter 2009/10 March, 2010 Issue 3 (PDF) (in 英语). Independent Schools Foundation Academy. March 2010..

[100]
^"Sir Charles Kao UTC". Archived from the original on July 14, 2014..

[101]
^A chat with vice-chancellor Kao, by Midori Hiraga.

[102]
^The Standard: The day Nobel winner lost mic Archived 6月 4, 2011 at the Wayback Machine.

[103]
^Erickson, Jim; Chung, Yulanda (December 10, 1999). "Asian of the Century, Charles K. Kao". Asiaweek. Archived from the original on 2002-07-21. Retrieved December 24, 2009..

[104]
^"Electronic Design, 50th Anniversary Issue". Electronic Design. October 21, 2002. Retrieved May 21, 2010..

[105]
^"ED Hall of Fame 2002 INDUCTEES" (PDF). Electronic Design. October 21, 2002. Retrieved May 21, 2010..

[106]
^"Enter the Creative Dragon Feature" (PDF). AlumniNews London Business School. January–March 2009. Retrieved May 21, 2010..

[107]
^"British Council Celebrates 60 Years in Hong Kong" (PDF). Hong Kong: British Council. January 3, 2008. Archived from the original (PDF) on June 6, 2011. Retrieved May 21, 2010..

[108]
^"City Press Release: Mountain View Honors Dr. Charles Kao for Being Awarded the 2009 Nobel Prize in Physics". Office of the City Manager, Mountain View, California. October 27, 2009. Archived from the original (asp) on February 29, 2012. Retrieved January 3, 2009..

[109]
^"Nobel laureate Charles Kao is named Hong Kong's Person of Year". Earthtimes. January 4, 2010. Retrieved January 3, 2009..

[110]
^Evangeline Cafe (December 30, 2009). "The top 10 Asian achievements of 2009". Northwest Asian Weekly. Retrieved January 3, 2009..

[111]
^"OFC/NFOEC 2010 To Be Dedicated To Nobel Laureate Charles Kao" (mvc). Photonics Online. January 15, 2010. Retrieved January 20, 2009..

[112]
^"OFC/NFOEC 2010 Announces Plenary Session Speaker Lineup". Yahoo! Finance. January 21, 2010. Retrieved January 20, 2009.[永久失效连结].

[113]
^"The 19th Annual Wireless and Optical Communications Conference (WOCC 2010)". WOCC 2010. 2010. Archived from the original on April 17, 2010. Retrieved May 26, 2010..

[114]
^"Archived copy" 康宁公司在华开展光纤发明40周年庆祝活动. 美通社(亚洲). May 18, 2010. Archived from the original on July 22, 2011. Retrieved May 26, 2010.CS1 maint: Archived copy as title (link).

[115]
^《世界百位名人谈上海世博》首发. Xinhua News Agency. May 23, 2010. Archived from the original on June 21, 2010. Retrieved May 26, 2010..

[116]
^"Hongkong Post Stamps – Hong Kong Stamps". Hongkong Post. Archived from the original on March 30, 2010. Retrieved Apr 8, 2010..

[117]
^"Harlow Nobel Prize winner to be commemorated in town centre". HarlowStar. March 25, 2011. Retrieved April 29, 2011..

[118]
^"Gimme Fibre Day - 4 November". Fibre to the Home Council Europe..

[119]
^高锟. 杰出华人系列 (documentary and oral history) (in 粤语, 中文, and 英语). Radio Television Hong Kong. 2000. Event occurs at around 38:00. Retrieved 27 September 2018. 我对每一个国家,每一个种族感情都差不多。。。。。。我是以人为主,不是以国家或种族为主。。。。。。我变成了世界中间的一部份,不是任何国家的一部份。.

[120]
^Kao, Charles; Kao, May Wan (13 October 2009). "Professor and Mrs Charles K. Kao wish to express their gratitude to their friends, all staff, students and alumni at CUHK, members of the media, and the people of Hong Kong, by the following Open Letter" (Press release). Chinese University of Hong Kong. Archived from the original on 16 October 2009. Retrieved 30 September 2018. Charles Kao was born in Shanghai, China, did his primary research in 1966 at Standard Telecommunication Laboratories (STL) in Harlow, UK, followed through with work in the USA at ITT, over the following 20 years, to develop fiber optics into a commercial product and finally came to CUHK, Hong Kong in 1987 to pass on his knowledge and expertise to a new generation of students and businessmen. Charles really does belong to the world!.

[121]
^Kao, Charles K.; Kao, May Wan (5 February 2010). "Message from Prof. and Mrs. Charles K. Kao (5 February 2010)" (Press release). Chinese University of Hong Kong. Archived from the original on 27 December 2010. Retrieved 1 October 2018..

[122]
^QQ.com News 记者探访"光纤之父"高锟:顽皮慈爱的笑.

[123]
^"Physics 2009". Nobelprize.org. Retrieved October 26, 2009..

[124]
^Editor: Zhang Pengfei (October 7, 2009). "Nobel Prize winner Charles Kao says never expected such honor" (shtml). CCTV. Retrieved November 30, 2009.CS1 maint: Extra text: authors list (link).

[125]
^"○九教育大事(二) 高锟获迟来的诺奖". Sing Tao Daily (in Chinese). HK Yahoo! Archive. January 2, 2010. Archived from the original on January 7, 2010.CS1 maint: Unrecognized language (link).

[126]
^Ifeng.com: 港媒年初传高锟患老年痴呆症 妻称老人家记性差.

[127]
^Mesher, Kelsey (October 15, 2009). "The legacy of Charles Kao". Mountain View Voice. Retrieved November 30, 2009..

[128]
^Chiu, Peace; Singh, Abhijit; Lam, Jeffie (23 September 2018). "Hong Kong mourns passing of Nobel Prize winner and father of fiber optics, Charles Kao, 84". South China Morning Post. Hong Kong. Retrieved 23 September 2018..

[129]
^诺奖得主光纤之父高锟逝世 慈善基金:最后心愿助脑退化病人. Ming Pao. Hong Kong: Media Chinese International. 24 September 2018. Retrieved 25 September 2018..

[130]
^"In memory of Sir Charles K. Kao (1933-2018)" (Press release). Hong Kong: Charles K. Kao Foundation for Alzheimer’s Disease. 23 September 2018. Retrieved 25 September 2018..

[131]
^Ives, Mike (September 24, 2018). Written at Hong Kong. "Charles Kao, Nobel Laureate Who Revolutionized Fiber Optics, Dies at 84". New York Times. Tiffany May contributed reporting. New York City. Retrieved 26 September 2018..

[132]
^Vivek Alwayn (April 23, 2004). "Fiber-Optic Technologies – A Brief History of Fiber-Optic Communications". Cisco Press. Retrieved December 4, 2009..

[133]
^Mary Bellis. "The Birth of Fiber Optics". inventors.about.com. Retrieved December 15, 2009..

[134]
^"Charles Kuen Kao" (PDF). Archived from the original (PDF) on August 14, 2011. Retrieved October 28, 2009..