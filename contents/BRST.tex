% BRST 量子化
% 弦论|量子化|李代数|时空|鬼场|开弦|物理态|无鬼定理

弦理论的量子化方案有三种:协变量子化,光锥量子化和BRST量子化。三种方案的优劣比较如下:
\begin{itemize}
\item 协变量子化:洛伦兹不变能明显表现出来。有鬼。时空维数是26维难以证明。
\item 光锥量子化:洛伦兹不变不再明显。没有鬼。时空维数是26维容易证明。
\item BRST量子化:洛伦兹不变能明显表现出来。有鬼。时空维数是26维容易证明。
\end{itemize}
\subsubsection{BRST算符}
首先我们来考虑李代数。考虑算符 $K_i$,这些算符满足如下的李代数
\begin{equation}
[K_i,K_j] = f_{ij}{}^k K_k~.
\end{equation}
其中 $f_{ij}{}^k$ 被称作理论的结构常数。这些结构常数满足
\begin{equation}
f_{ij}{}^m f_{mk}{}^i + f_{jk}{}^m f_{mi}{}^l+f_{ki}{}^m f_{mj}{}^l = 0 ~. 
\end{equation}
现在我们引入两个鬼场,记作 $b_i$ 和 $c_j$,它们满足反对易关系
\begin{equation}\label{eq_BRST_5}
\{ c^i, b_j \} = \delta^i_j~.
\end{equation}
我们现在回忆一下,一个场 $\phi(z,\bar z)$ 在共形变换 $z\rightarrow w(z)$ 下具有如下变换的时候
\begin{equation}
\phi(z,\bar z) = \bigg( \frac{\partial w}{\partial z} \bigg)^h \bigg( \frac{\partial w}{\partial \bar z} \bigg)^{\bar h} \phi (w,\bar w)~.
\end{equation}
我们就说,这个场具有共形维度 $(h,\bar h)$. $b$ 和 $c$ 场的共形维度分别是 $2$ 和 $-1$. 我们现在来用 $b$ 和 $c$ 这两个鬼场以及 $K_i$ 来构造两个算符。第一个是
\begin{equation}
Q = c^i K_i - \frac{1}{2} f_{ij}{}^k c^i c^j b_k~.
\end{equation}
我们假设 $Q = Q^\dagger$。$Q$ 满足如下性质
\begin{equation}
Q^2 = 0~.
\end{equation}
这个关系也可以写作 $\{Q,Q\}=0$。我们把BRST的算符记作 $Q$ 是为了暗示它是这个系统的守恒荷。我们把这叫做BRST荷。

第二个完全由鬼场组成的算符叫做鬼场算符U。它的表达式如下
\begin{equation}\label{eq_BRST_1}
U = c^i b_i~.
\end{equation}
这个算符具有整数的特征值。如果李代数的维度是 $n$,那么 $U$ 的特征值就是 $0,\ldots ,n$。如果一个态 $|\Psi\rangle$ 满足 $U|\Psi\rangle=m|\Psi\rangle$,我们就说这个态具有鬼数 $m$。$Q$ 能够把鬼数提升1.
\begin{equation}
U (Q|\Psi\rangle) = (1+m) (Q|\Psi\rangle)~. 
\end{equation}
\subsubsection{BRST不变态}
一个态 $|\Psi\rangle$ 如果能被BRST算符湮灭掉,那这个态就被称为BRST不变态。
\begin{equation}
Q|\Psi\rangle = 0 \rightarrow |\Psi\rangle \text{是 BRST 不变的}~.
\end{equation}
BRST不变态是一个理论的物理态。因为 $Q^2=0$,我们可以得知任何一个满足如下条件的态 $|\Psi\rangle$ 都是BRST不变的
\begin{equation}
|\Psi\rangle = Q |\chi\rangle \neq 0~.
\end{equation}
因为
\begin{equation}
Q|\Psi\rangle = Q^2 |\chi\rangle = 0~.
\end{equation}
我们把满足 $|\Psi\rangle=Q|\chi\rangle$ 的态叫做零态。假设 $|\phi\rangle$ 是物理态,那么 $Q|\phi\rangle=0$. 注意到
\begin{equation}
\langle\phi|\Psi\rangle = \langle\phi|(Q|\chi\rangle) = (\langle | Q) |\chi\rangle = 0~.
\end{equation}
因此,一个物理态和一个零态之间的振幅总是为0.现在我们知道了,对态 $|\phi\rangle$ 加上一个零态 $|\Psi\rangle = Q|\chi\rangle$ 之后,仍然和 $|\phi\rangle$ 等价。
\begin{equation}
|\phi\rangle \text{等价于} |\phi\rangle+Q|\chi\rangle~.
\end{equation}
如果一个态 $|\Psi\rangle$ 具有鬼数0,我们有
\begin{equation}\label{eq_BRST_2}
U|\Psi\rangle = 0~.
\end{equation}
从\autoref{eq_BRST_1} 以及\autoref{eq_BRST_2},我们可以得知 $b_k|\Psi\rangle = 0$.一个态 $|\Psi\rangle$ 如果满足 $b_k|\Psi\rangle = 0$,也就是说这个态被 $b$ 湮灭,这样一个态不能被 $c$ 湮灭,因为
\begin{equation}
U = \sum_i c^i b_i = \sum_i \delta^i_i - b_i c^i = n - \sum_i b_i c^i~.
\end{equation}
我们可以得知
\begin{equation}
U|\Psi\rangle = \bigg( n - \sum_i b_i c^i \bigg) | \Psi\rangle = n|\Psi\rangle - \sum_i b_i c^i |\Psi\rangle
\end{equation}
对于鬼数为0的态我们有如下结论
\begin{equation}
Q|\Psi\rangle = \sum_i c^i K_i |\Psi\rangle~.
\end{equation}
这说明对于一个鬼数为零的态,我们有
\begin{equation}
K_i|\Psi\rangle = 0~.
\end{equation}
我们可以说:一个BRST不变而且具有鬼数0的态也会在生成元是 $K_i$ 的对称性下保持不变。如果一个态具有鬼数0,这告诉我们这个态不是一个鬼态,因此我们避免了负的概率。

\subsubsection{弦理论里面的BRST}
我们现在选取共形规范 $h_{\alpha\beta} = \eta_{\alpha\beta}$. 在这个情形下,能量动量张量具有一个全纯的部分 $T_{zz}(z)$ 和一个反全纯的部分 $T_{\bar z\bar z}(\bar z)$. $T_{zz}(z)$ 有如下展开式
\begin{equation}
T_{zz}(z) = \sum_{m=-\infty}^{\infty} \frac{L_{m}}{z^{m+2}}~.
\end{equation}
$T_{zz}(z)T_{ww}(w)$ 的算子积展开如下
\begin{equation}\label{eq_BRST_3}
T_{zz}(z)T_{ww}(w) = \frac{D/2}{(z-w)^4} - \frac{2T_{ww}(w)}{(z-w)^2} - \frac{\partial_wT_{ww}(w)}{z-w}~.
\end{equation}
为了简便起见,我们省略了因子 $l_s^2$。我们对鬼场进行如下定义
\begin{equation}
b(z)c(w) = \frac{1}{z-w},\quad \bar b (\bar z) \bar c (\bar w) = \frac{1}{\bar z- \bar w} ~.
\end{equation} 
接下来我们写下鬼场的能量动量张量 $T_{\rm gh}(z)$.它的表达式如下
\begin{equation}
T_{\rm gh} = -2 b(z) \partial_z c(z) - \partial_z b(z) c(z)~.
\end{equation}
利用鬼场的能量动量张量以及弦的能量动量张量,我们推出了如下的BRST流
\begin{equation}
j(z) = c(z) \bigg( T_{zz} (z) + \frac{1}{2} T_{\rm gh}(z) \bigg) = c(z) T_{zz} (z) + c(z) \partial_z c(z) b(z)~.
\end{equation}
BRST荷由下式给定
\begin{equation}
Q = \int \frac{dz}{2\pi i} j(z)~.
\end{equation}
中心荷来自算子积展开\autoref{eq_BRST_3} 的第一项,也就是
\begin{equation}
\frac{D/2}{(z-w)^4}~.
\end{equation}
这一项也叫做共形反常,因为它让代数不闭合。所以我们想要去掉这一项。我们考虑总的能量动量张量,它是弦的能量动量张量和鬼的能量动量张量之和。
\begin{equation}
T = T_{zz} (z) + T_{\rm gh} (z)~.
\end{equation}
鬼的能量动量张量如下
\begin{equation}\label{eq_BRST_4}
T_{\rm gh}(z) T_{\rm gh} (w) = \frac{-13}{(z-w)^4} - \frac{2T_{\rm gh} (w) }{(z-w)^2} - \frac{\partial_w T_{\rm gh}(w)}{z-w}
\end{equation}
如果时空维度是26维,那么鬼场的能量动量张量的算子积展开\autoref{eq_BRST_4} 的第一项能够把共形反常给消掉。这一结论是直接由BRST荷的 $Q^2=0$ 推导出来的。

\subsubsection{BRST变换}
我们在光锥坐标系下,定义鬼场 $c$ 和反鬼场 $b$. 其中 $c$ 由 $c^+$,$c^-$ 两部分组成。$b$ 由 $b_{++}$,$b_{--}$ 两部分组成。 我们引入鬼场的能量动量张量 $T^{\rm gh}_{\pm\pm}$
\begin{equation}
\begin{aligned}
T^{\rm gh}_{++} &  = i (2 b_{++}\partial_+ c^+ +\partial_+ b_{++}c^+)~. \\
T^{\rm gh}_{--} &  = i (2 b_{--}\partial_- c^- +\partial_- b_{--}c^-)~.
\end{aligned}
\end{equation}
BRST变换归纳如下
\begin{equation}
\begin{aligned}
\delta X^\mu & = i \epsilon (c^+\partial_+ + c^- \partial_- ) X^\mu~. \\
\delta c^{\pm} & = \pm i \epsilon (c^+\partial_++c^-\partial_-) c^{\pm} ~. \\
\delta b_{\pm\pm} & = \pm i \epsilon (T_{\pm\pm} +T^{\rm gh}_{\pm\pm})~. 
\end{aligned}
\end{equation}
鬼场的作用量是
\begin{equation}
S_{\rm gh} = \int d^2\sigma (b_{++} \partial_- c^+ + b_{--} \partial_+ c^- ) ~.
\end{equation}
从作用量出发我们可以推出如下运动方程
\begin{equation}
\partial_- b_{++} = \partial_+ b_{--} = \partial_- c^+  = \partial_+ c^- = 0~.
\end{equation}
接下来我们写下鬼场的模式展开
\begin{equation}
\begin{aligned}
& c^+ = \sum_n \bar c_n e^{-i n (\tau+\sigma)} ~, \quad c^- = \sum_n c_n e^{-in(\tau-\sigma)} ~, \\
& b_{++} = \sum_n \bar b_n e^{-i n (\tau+\sigma)}~, \quad b_{--} = \sum_n b_n e^{-in(\tau-\sigma)}
\end{aligned}
\end{equation}
这些模式满足如下的反对易关系
\begin{equation}
\{b_m,c_n\} = \delta_{m+n,0}~.
\end{equation}
并且 $\{b_m,b_n\}=\{c_m,c_n\}=0$.鬼场的Virasoro算符是用这些模式定义出来的。使用正规序展开,它们是
\begin{equation}
L_m^{\rm gh} = \sum_n (m-n):b_{m+n}c_{-n}:~, \quad \bar L_m^{\rm gh} = \sum_n (m-n) :\bar b_{m+n}\bar c_{-n}:~. 
\end{equation}
总的Virasoro算符定义如下
\begin{equation}
L^{\rm tot} = L_m + L^{\rm gh}_m - a \delta_{m,0}~.
\end{equation}
我们可以证明,总的Virasoro算符满足下面的对易关系
\begin{equation}
[L^{\rm tot}_m,L^{\rm tot}_n] = (m-n) L^{\rm tot}_{m+n} + A(m) \delta_{m+n,0}~. 
\end{equation}
右边的 $A(m)$ 项是Anormaly。它的表达式如下
\begin{equation}
A(m) = \frac{D}{12} m (m^2-1) + \frac{1}{6} (m-13 m^3) + 2 a m~. 
\end{equation}
为了让反常消失,我们取 $D=26,a=1$. BRST流由下式给出
\begin{equation}
j = c T+ \frac{1}{2} : c T^{\rm gh} : + \frac{3}{2} \partial^2 c~.
\end{equation}
BRST荷由下面的模式展开给出
\begin{equation}
Q = \sum_n c_n L_{-n} + \frac{1}{2} \sum_{m,n} (m-n) :c_m c_n b_{-m-n}: -c_0~.
\end{equation}
我们可以计算 $Q^2$,结果如下
\begin{equation}
\begin{aligned}
Q^2&  = \frac{1}{2} \sum ( [L^{\rm tot}_m,L^{\rm tot}_n] - (m-n) L^{\rm tot}_{m+n} ) c_{-m} c_{-n} \\
& \simeq \frac{1}{12} (D-26)~.
\end{aligned}
\end{equation}
$Q^2=0$ 要求时空维数必须是26维($D=26$)。回到原始的BRST方法,使用经典的Virasoro代数
\begin{equation}
[L_m,L_n]=(m-n)L_{m-n}~.
\end{equation}
我们可以得到结构常数的表达式如下
\begin{equation}
f^k_{mn} = (m-n) \delta_{m+n,k}~.
\end{equation}
现在我们来看弦理论里面物理的谱是如何构造出来的。我们先看开弦的例子。这些态是用鬼真空态构造出来的。我们把鬼真空态称作 $|\chi\rangle$.这个态被所有的正的鬼模湮灭。如果 $n>0$ 我们有
\begin{equation}
b_n|\chi\rangle = c_n|\chi\rangle = 0~.
\end{equation}
鬼场的零模是一个特殊情况。它们可以用来构造理论的物理态。使用对易关系,我们可以推出零模满足如下方程
\begin{equation}
\{b_0,c_0\} = 1~.
\end{equation}
由\autoref{eq_BRST_5} 我们可知 $b_0^2=c_0^2=0$.我们也要求对于物理态 $|\Psi\rangle$,我们有 $b_0|\Psi\rangle = 0$。现在,我们可以从鬼态的零模出发,构造一个两态的系统。这两个基矢被记作 $|\uparrow\rangle$ 和 $|\downarrow\rangle$。它们满足
\begin{equation}
\begin{aligned}
& b_0|\downarrow\rangle = c_0 |\uparrow\rangle = 0 \\
& b_0|\uparrow\rangle = |\downarrow\rangle~,\quad c_0|\downarrow\rangle = |\uparrow\rangle~.
\end{aligned} 
\end{equation}
我们选取 $|\downarrow\rangle$ 作为鬼真空态。我们可以把这个态与动量态 $|k\rangle$ 做张量积, 可以得到 $|\downarrow,k\rangle$. 为了得到物理的态,我们把BRST荷 $Q$ 作用在上面。我们可以证明
\begin{equation}
Q|\downarrow,k\rangle = (L_0-1)c_0|\downarrow,k\rangle~.
\end{equation}
$Q|\downarrow,k\rangle=0$ 的要求给出了在壳条件 $L_0-1=0$.这个是快子态。第一激发态如下
\begin{equation}
|\Psi\rangle = (\varsigma\cdot\alpha_{-1}+\xi_1c_{-1}+\xi_2b_{-1})|\downarrow,k\rangle~.
\end{equation}
$\xi_1$ 和 $\xi_2$ 是常数,$\varsigma_\mu$ 是一个有26个分量的矢量。因为第一激发态是无质量的。所以它有24个独立的分量。为了去掉多余的自由度,我们创造一个满足 $Q|\psi\rangle = 0$ 的物理态。我们可以证明
\begin{equation}
Q|\psi\rangle = 2 (p^2 c_0+(p\cdot \varsigma) c_{-1} + \xi_2 p\cdot \alpha_{-1} ) | \downarrow, k\rangle~.
\end{equation}
$Q|\psi\rangle = 0$ 对参数进行了一些限制。有26个正模的态和2个负模的态。我们可以引入一些约束来去掉负模的态。第一个约束是取 $p\cdot\varsigma = 0$ 以及 $\xi_1=0$.这可以把负模的态去掉。同时 $p^2=0$,这告诉我们这是一个无质量的态。我们有量纲零模的态
\begin{equation}
k_\mu\alpha^\mu_{-1}|\downarrow,k\rangle~,\quad c_{-1} |\downarrow,k\rangle~.
\end{equation}
这些态跟物理的态是正交的。去掉它们给了我们一个24自由度的态。

\subsubsection{无鬼定理}
无鬼定理的叙述如下:
如果时空维数是 $D=26$,那么负模的态会被自动消去。

