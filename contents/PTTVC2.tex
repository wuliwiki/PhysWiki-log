% 相变的热力学量变化

\pentry{相简介(热力学)\upref{PHS}}

\subsection{相变的热力学量变}
\footnote{本文参考了刘俊吉等人的《物理化学》}
\begin{figure}[ht]
\centering
\includegraphics[width=8 cm]{./figures/9f9b0c7adf0109a9.pdf}
\caption{即使压力与温度相同,液态水与水蒸气的各种热力学性质还是不同} \label{fig_PTTVC2_1}
\end{figure}

我们先假定系统中只有一种物质。在相简介(热力学)\upref{PHS} 中我们已经知道物质的性质与物质的相态有关。比如说,即使压力与温度相同,液态水与水蒸气的各种热力学性质还是不同。
$$
\begin{aligned}
U_g(p,T) &\ne U_l(p,T)\\
S_g(p,T) &\ne S_l(p,T)\\
H_g(p,T) &\ne H_l(p,T)\\
&...
\end{aligned}
$$
其中$U_g(p,T)$指气态水的内能、$U_l(p,T)$指液态水的内能等。

根据状态量与路径无关的性质\upref{StaPro},我们可以大胆放心地定义等温等压的两种相态下,热力学量的差值:
$$
\begin{aligned}
U_g(p,T) &= U_l(p,T) + \Delta ^ g_l U (p,T)\\
S_g(p,T) &= S_l(p,T) + \Delta ^ g_l S (p,T)\\
H_g(p,T) &= H_l(p,T) + \Delta ^ g_l H (p,T)\\
&...\\
\end{aligned}
$$
其中,$\Delta ^ g_l U (p,T)$就代表了气态水与液态水的内能之差等。
\begin{figure}[ht]
\centering
\includegraphics[width=8 cm]{./figures/06648f2703e4d249.pdf}
\caption{$\Delta ^ g_l U (p,T)$的另一种含义} \label{fig_PTTVC2_2}
\end{figure}
从更有物理意义的角度而言,这个差值还意味着,等温等压条件下系统相变前后,系统内能的变化量。在一些国内教材的符号规范中,$\Delta^g_l$的上标$g$代表相变后的末相气相,下标$l$代表相变前的初始相液相。

显然,由于$U$是与温度、压力、物质种类有关的量,因此$\Delta ^ g_l U (p,T)$也是与温度、压力、物质种类有关的量。其余的热力学量变也是如此。

你可能迫不及待地想问,那么如何计算相变前后的$\Delta U (p,T)$?很遗憾的是,这是经典热力学力所不能及的。在经典热力学中,这些量都是由实验测定的。不过,根据“设计路径”\upref{Hess}的思路,如果已经测得了某一状态$(p_0,T_0)$下的 $\Delta U (p_0,T_0)$,那么可以计算另一状态$(p_1,T_1)$下的$\Delta U (p_1,T_1)$。其余相变热力学量变也是如此。

\subsection{常关注的相变热力学量变}
实操中更常用的量是等温等压条件下相变的焓变、熵变与Gibbs变等。各种物质的这类参数一般可以在厚厚的物化手册或一些在线数据库中找到。
$$
\Delta H, \Delta  S, \Delta G
$$

\begin{example}{相变潜热}
为什么相变焓是一个重要的热力学量?我们回顾焓的含义:在等温等压相变、无非体积功过程中,有
$$
\Delta H = \delta q - \delta w_{others} = \delta q
$$
也就是说,这种情况下系统的焓变直接代表着系统的热效应。因此,相变焓也称为“相变潜热”。
\end{example}

由相变焓与相变熵可以计算 相变的Gibbs自由能变
$$
\Delta G = \Delta H - T \Delta S
$$
Gibbs变之所以重要,是因为Gibbs判据\upref{GibbsG}仍然成立,并且与化学势高度相关。此处先讨论相变的Gibbs判据:在等温等压情况下,只有当$\Delta G<0$时,相变才会发生。在刚好发生相变的临界情况(也称“可逆相变”),有
$$
\Delta G = 0 \Rightarrow T = \frac{\Delta H}{\Delta S}
$$
或者
$$
\Delta S = \frac{\Delta H}{T}
$$
由于 $\Delta H$, $\Delta S$等其实也与$T$有关,因此严格地说这个等式的含义更复杂一些。

\subsection{摩尔相变热力学量变}
\pentry{摩尔量与偏摩尔量\upref{ParMol}}
由于状态量的广延性质,更常用的是摩尔热力学量变,即在变化量上除以物质的量。符号上补充$m$下标\footnote{在一些课本中,$m$下标被省略,$\Delta X$始终指相应的摩尔量变。}。例如,摩尔焓变:
$$
\Delta H_m = \Delta H / n
$$
那么
$$
H_g = H_l + n \cdot \Delta^g_l H_m
$$

摩尔相变热力学量变 更有用的地方在于计算摩尔量:例如,某一$(T,p)$下,纯气体的摩尔焓是$h^*_g$, 纯液体的摩尔焓是$h^*_l$,那么二者之间由摩尔相变焓关联:$h^*_g = h^*_l + \Delta^g_l H_m$。这是具体运用相变平衡条件\upref{PhEquv} 来确定系统状态时,必不可少的知识。

