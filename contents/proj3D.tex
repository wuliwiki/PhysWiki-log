% 三维投影
% 立体几何|投影|透视|画图

\begin{issues}
\issueTODO
\end{issues}

\footnote{参考 Wikipedia \href{https://en.wikipedia.org/wiki/3D_projection}{相关页面}。}当我们在平面上画三维物体时, 我们需要某种投影算法把物体上的每个点对应到平面上的一点。 以下介绍两种常用的方法, 一种是\textbf{平行投影(parallel projection)}, 另一种是\textbf{透视投影(perspective projection)}。

% 首先来比较两个图(图未完成: 左图是长方体的平行投影, 右图是长方体的透视投影)
% 参考 https://construct3.ideas.aha.io/ideas/C3-I-754

\subsection{平行投影}
顾名思义, 平行投影是指在空间中指定一个方向(如图未完成), 将三维物体上的每一点沿着该方向投影到与该方向垂直的平面上。 工程制图中的正视图, 侧视图等都属于平行投影。 这种投影的特点是, 空间中的任意两条平行线的投影仍然是平行线。 平行投影是线性的, 即一个线段若伸长若干倍, 那么它的投影也会按照同样的比例伸长。

\subsection{透视投影}
当人眼或相机观察一个三维物体时, 使用的是透视投影。
\begin{figure}[ht]
\centering
\includegraphics[width=8cm]{./figures/a64ee2d20c8505ba.pdf}
\caption{透视投影示意图} \label{fig_proj3D_1}
\end{figure}

我们在平面后方取一个固定点 $F$ 称为\textbf{焦点}, 要把物体上任意一点 $P$ 投影到屏幕(对于眼睛而言,这是我们“看到的图像”)上, 先作直线 $PF$, 该直线与平面的交点 $P'$ 就是投影后的点。 

%我们在平面上建立直角坐标系, 把平面上离焦点最近的点定义为平面坐标的原点。 
\begin{figure}[ht]
\centering
\includegraphics[width=8cm]{./figures/4bdc370703837bc8.pdf}
\caption{“近大远小”:尽管两个正方体事实上是一样大的,但由于左侧的正方体离屏幕更远,因此看起来更小} \label{fig_proj3D_2}
\end{figure}

透视投影不是线性的,且会使物体的投影呈现“近大远小”, 但仍可以保证直线的投影仍然为直线。由于人眼适应的是透视投影,因此运用透视投影作画,会让画面更显真实。

为什么透视投影要这样定义? 我们先思考一种非常简单的相机, 即小孔成像相机。 根据光的直线传播, 物体上的某点发出的(或反射的)光线只有经过小孔才能投影到孔后面的平面。 如果按照这个模型, 我们在计算透视投影时应该把焦点定义在屏幕之前, 然而这么做有一个缺点就是投影后平面上的像是倒像。 所以为了方便起见, 我们保持焦点不变, 但是把屏幕平移到焦点之前(焦距也不变), 容易看出这样做的唯一改变就是把倒像变为正像, 即点 $P'$ 的两个平面坐标分别取相反数。 % (未完成)需要一张图比较焦点在前和焦点在后

如果我们将相机上的小孔改为焦距为 $f$ 的小凸透镜, 并假设物距远大于焦距, 那么根据成像公式, 凸透镜和平面(即底片)的距离就是焦距 $f$。 根据凸透镜成像原理, 物体一点在底片上的像仍然会过凸透镜的中点% 图未完成
, 所以使用凸透镜的相机和使用小孔成像的相机得到的投影是相同的。 至于人眼, 人眼成像的原理和相机基本一致, 虽然视网膜并不是一个平面, 得到的成像是扭曲的, 但大脑在处理图象是会自动纠正这种扭曲(例如我们看到的直线仍然是直的)。

\subsection{计算方法}
无论是哪种投影, 我们通常建立两个坐标系, 一个是\textbf{世界坐标系\upref{Worcod}(world frame)} $S$, 一个是\textbf{相机坐标系}\upref{CamMdl}\textbf{(camera frame)} $S'$。 一个基本的问题就是将一个系里面的坐标变换到另一个系中的坐标。 这可以通过空间旋转矩阵\upref{Rot3D}, 再加上一个平移完成(平移的矢量是两个坐标系原点之间的位移)。
\begin{figure}[ht]
\centering
\includegraphics[width=12cm]{./figures/86e90b1b4fe89c52.pdf}
\caption{转换坐标系的方法相当于平移并旋转所有的顶点} \label{fig_proj3D_4}
\end{figure}
把所有要投影的点 $P_i$ ($i = 1, \dots, N$) 的在世界系中的坐标(假设已知)记为一个 3 行 $N$ 列矩阵, 矩阵的第 $i$ 列就是列矢量 $\bvec P_i = (x_i, y_i, z_i)\Tr$。 将 3 乘 3 的旋转矩阵左乘该矩阵, 再给每行加上常数进行平移即可。 完成后, 我们就得到了相机系中的坐标矩阵, 每一列是 $\bvec P_i' = (x_i', y_i', z_i')\Tr$

\subsubsection{平行投影}
当我们做平行投影时, 可以令相机系的 $x$-$y$ 平面为投影平面, 投影方向为 $-\uvec z$。 即 $(x_i', y_i', z_i')$ 投影后为 $(x_i', y_i')$。

\subsubsection{透视投影}
当我们做透视投影时, 可以令原点为焦点, $(0, 0, f)$ 为平面的中点, 平面与 $z$ 轴垂直。 相机系中的某点 $(x_i', y_i', z_i')$ 投影后变为 $(x_i' f/z_i', y_i' f/z_i')$, 即每个分量乘以 $f/z_i'$, 使得 $z$ 分量等于 $f$。

\begin{figure}[ht]
\centering
\includegraphics[width=10 cm]{./figures/32317c69e6902a20.pdf}
\caption{透视投影ABC} \label{fig_proj3D_3}
\end{figure}
或者更讲人话的,如 \autoref{fig_proj3D_3} 所示,我们依据计算机图形学的惯例\footnote{\textsl{不同人有不同的习惯},因此你的参考教材、软件可能使用了其他的约定。},以焦点为原点,$-z$方向垂直指入屏幕,建立坐标系。

显而易见,透视投影其实是一个简单的相似三角形问题。假设物体的一个顶点$P$的坐标为$(x,y,z)$,它在屏幕上对应点$P'$的坐标为$(x',y')$\footnote{由于屏幕是二维的,所以目前而言$P'$的$z$坐标$z'$是无关紧要的。不过,为了以后进一步处理形状的遮挡顺序(显然,靠近屏幕的形状会遮住后面的形状),我们会定义$z'$坐标。}。先观察$y$坐标,显然:
$$\frac{y}{z}=\frac{y'}{n}$$
即
$$y'=\frac{n}{z}y$$
同理
$$x'=\frac{n}{z}x$$

总结一下,透视投影是一个这样的过程:
$$P: (x,y,z) \to P': \left( \frac{n}{z}x, \frac{n}{z}y \right)$$
真是简单的出乎意料!

\addTODO{另外开一篇文章分享平行投影和透视投影的 Matlab 代码}

\subsection{3D 艺术画}
见 3D 艺术画\upref{art3D}。
