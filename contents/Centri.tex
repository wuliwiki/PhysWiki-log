% 离心力
% keys 惯性力|圆周运动|旋转|惯性系|参考系
% license Xiao
% type Tutor

\pentry{匀速圆周运动的加速度\upref{CMAD},惯性力\upref{Iner}, 连续叉乘的化简\upref{TriCro}}
令参考系 $abc$ 和 $xyz$ 的 $c$ 轴和 $z$ 轴始终重合。其中 $xyz$ 是惯性系, $abc$ 以恒定的角速度 $\omega$ 绕 $z$ 轴逆时针转动。求 $abc$ 系中一个质量为 $m$ 的静止质点所受的惯性力, 即\textbf{离心力(centrifugal force)}。

令质点的坐标 $(a,b,c)$ 离 $c$ 轴的距离为 $r_\bot = \sqrt{a^2 + b^2}$, 对应的径向矢量为 $\bvec r_\bot = (a,b,0)\Tr$。 在 $xyz$ 系中,质点做匀速圆周运动,相对于 $xyz$ 系的加速度(用 $abc$ 系的坐标表示)为
\begin{equation}
\bvec a_{xyz} =  - \omega ^2 \bvec r_\bot =  - \omega ^2 \pmat{a\\b\\0}_{abc}~,
\end{equation}
质点相对于 $abc$ 系静止,相对加速度为零
\begin{equation}
\bvec a_{abc} = \bvec 0~.
\end{equation}
所以由惯性力\upref{Iner} 中的结论,惯性力为
\begin{equation}\label{eq_Centri_3}
\bvec f = m(\bvec a_{abc} - \bvec a_{xyz}) = \omega^2\bvec r_\bot = m\omega ^2\pmat{a\\b\\0}_{abc}~.
\end{equation}
注意离心力向外,与直觉相符。 注意这个结论只适用于质点相对于 $abc$ 系静止的情况,若有相对运动,则惯性力除了离心力,还会有一项科里奥利力\upref{Corio}。

若转轴取任意方向 $\uvec\omega$, 由\autoref{eq_CMAD_9}~\upref{CMAD} 得(这里使用了连续叉乘的化简\upref{TriCro})
\begin{equation}
\bvec a_{xyz} = \bvec\omega\cross (\bvec\omega\cross\bvec r)~,
\end{equation}
\begin{equation}\label{eq_Centri_5}
\bvec f = -m\bvec\omega\cross (\bvec\omega\cross\bvec r)~.
\end{equation}
