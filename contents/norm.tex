% 向量空间上的范数
% license Xiao
% type Tutor

% 移动自 NormV,原作者 addis; DTSIo

\textbf{范数(norm)}可以看作\enref{几何向量}{GVec}的模长在一般向量空间上的拓展。
\begin{definition}{范数、赋范空间}\label{def_NormV_1}
设 $X$ 是实数或复数域上的向量空间。 $X$ 上的范数是满足如下条件的非负函数 $\norm{\cdot}: X \to \mathbb{R}$:
\begin{enumerate}
\item $\norm{x} \geqslant 0$ (正定),
\item $\norm{x} = 0$ 当且仅当 $x = 0$,
\item $\norm{\lambda x} = |\lambda|\norm{x}$,
\item $\norm{x_1+x_2} \leqslant \norm{x_1} + \norm{x_2}$ (三角不等式)。
\end{enumerate}
如果一个向量空间中定义了范数, 我们就把它称为\textbf{赋范向量空间(normed vector space)},简称\textbf{赋范空间(normed space)}。
\end{definition}

一个线性空间上可能可以定义许多个范数。两个范数是\textbf{等价的}如果它们导出的度量是强等价的。

\begin{example}{有限维空间上的范数}
设 $p\geq 1$。 定义 $\mathbb R^N$ 或 $\mathbb C^N$ 空间(即 $N$ 维实数或复数列向量空间) 的 \textbf{$p$-范数}为
\begin{equation}
\norm{ x}_p = \qty(\sum_{i=1}^N \abs{x_i}^p)^{1/p}~.
\end{equation}
物理中常见的是 \textbf{2-范数}, 也叫\textbf{欧几里得范数(Euclidean norm)} 即
\begin{equation}
\norm{ x}_2 = \sqrt{\abs{x_1}^2 + \abs{x_2}^2 + \dots+|x_N|^2}~.
\end{equation}
它是由内积
$$
\langle x,y\rangle=\sum_{i=1}^Nx_i\bar y_i~
$$
诱导的。

在极限 $p \to \infty$ 之下, 绝对值最大的 $x_i$ 对求和的贡献将远大于其他分量, 所以可定义\textbf{无穷范数(infinity norm)}为
\begin{equation}
\norm{\bvec x}_\infty = \max \qty{\abs{x_i}}~.
\end{equation}

除此之外, 有限维实或复线性空间上还可以定义许多种不同的范数。 不过, 有限维实或复线性空间上的任意两个范数必然彼此等价。 它们都给出空间上唯一的一个自然拓扑(即使得所有线性泛函(线性函数)均连续的最疏松的\enref{拓扑}{Topol})。
% 在任何范数之下, 有限维实或复线性空间都是\enref{巴拿赫空间}{banach}。
\addTODO{举例“定义许多种不同的范数”,或者说明它们不存在}

\end{example}

