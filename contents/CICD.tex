% GitLab CI/CD 笔记
% license Usr
% type Note

\verb`.gitlab-ci.yml` 文件用于设置 CI/CD

环境变量
\begin{lstlisting}[language=none]
# 设置环境变量
variables:
  MY_NAME: "codegen"
  MY_LIB_VERSION: "3.5.5"
  PACKAGE_VERSION: $CI_COMMIT_TAG

# 如果这些条件不满足,将直接跳过整个 pipeline(下面的东西不会执行)
workflow:
  rules:
    - if: ... && ...
    - if: ...
      when: never
    - if: ...
    - if: ...


# 定义不同阶段(按照顺序)
stages:
  - build
  - upload
  - deploy

# 一个 build 阶段的 job,(每个阶段可以有多个 job,例如编译到不同平台)
build-ubuntu22.04: # job 的名称
  stage: build # 指定执行的阶段
  image: runner/builtin:ubuntu22.04 # docker 容器
  rules: # 该 job 运行的触发条件
    - if: $CI_PIPELINE_SOURCE == 'merge_request_event' # merge request 时
    - if: $CI_COMMIT_BRANCH == 'master' # 如果是 master 分支
    - if: $CI_COMMIT_TAG # 如果 commit 有 tag
    - when: manual # 即使条件满足,也要手动确认才会运行
  script: # 运行的代码(相当于依次粘贴到命令行运行)
    - .build-tools/build.sh ubuntu22.04
  artifacts:
    expire_in: 1 day
    paths:
      - build/*.zip
  tags:
    - linux
\end{lstlisting}

\subsection{如果需要从 Package Registry 下载依赖}
可以在 \verb`.gitlab-ci.yml` 脚本中从 api 下载安装包(例如使用 \verb`curl`):
\begin{lstlisting}[language=none]
variables:
  PACKAGE_REGISTRY_URL: "${CI_API_V4_URL}/projects/${CI_PROJECT_ID}/
    packages/generic/仓库名/${CI_COMMIT_TAG}/文件名"
\end{lstlisting}
其中 \verb`项目id` 可以在仓库首页右上角的三个点里面复制。
