% 开弦的谱
% license Usr
% type Tutor


\subsection{光锥规范}
现在终于可以来讨论弦的具体形式。用光锥规范来消除 (diff $\times$ Weyl) 冗余。光锥规范虽然隐藏了理论的协变性,但是通往谱的最快路径,可以揭示重要特征\footnote{例如弦的最轻激发态是无质量粒子,维度必须于 $D=26$。}。这里的过程我们仅关注结果而非细节推导。使用协变的共形规范的方法会在之后讨论。

定义光锥坐标系:纵向坐标 $x^{\pm} = 2^{-1/2} (x^0 \pm x^1)$; 横向坐标 $x^i, i = 2, 3, \dots, D-1$。时空坐标将使用小写 $x^\mu$,而世界面上相关的坐标仍用大写 $X^\mu(\sigma, \tau)$。光锥坐标下度规是:
\begin{itemize}
	\item 内积:$a^\mu b_\mu = -a^+ b^- - a^- b^+ + a^i b^i$;
	\item 指标升降:$a_- = -a^+$、$a_+ = -a^-$、$a_i = a^i$。
\end{itemize}

令 $x^+$ 与世界面上 $\tau$ 相等,则相应的 $p^-$ 会成为能量,可以看到 $x^-$ 与 $p^+$ 将和 $x^i$ 与 $p^i$ 一样扮演坐标与动量的角色。

\subsection{与全息理论}
此外,良好的光锥规范下,Virasoro 约束里的 $X^\pm$ 可以和横向的 $X^i$ 解耦,变为:
\begin{equation}
	\left(\dot X^i \pm X^i~'\right)^2 = 2 (\dot X^+ \pm X^+~') (\dot X^- \pm X^-~') ~.
\end{equation}
而可以化出
\begin{equation}
	\dot X^- \pm X^-~' = \frac{1}{2 X^+/\tau} \left(\dot X^i \pm X^i~'\right)^2 ~.
\end{equation}
这指出,弦的横向自由度 $X^i$ 才是真正的独立自由度,纵向自由度 $X^-$ 可以由横向自由度 $X^i$ 完全确定。也就是说弦的运动少了一个自由度,这与\textbf{全息原理}\footnote{也就是一个 $D+1$ 维的引力系统只需要 $D$ 个独立的自由度。}有关。

\subsection{点粒子分析}
考虑对于一个点粒子,世界线上 $X^+(\tau) = \tau$,作用量 
\begin{equation}
	S'_{pp} = \frac{1}{2} \int \dd \tau \left(-2 \eta^{-1} \dot X^- + \eta^{-1} \dot X^i \dot X^i - \eta m^2\right) ~,
\end{equation}
此时共轭动量
\begin{equation}
	p_\mu = \frac{\partial \mathcal L}{\partial \dot X^\mu} ~,
\end{equation}
也就是 $p_- = -\eta^{-1}$,而 $p_i = \eta^{-1} \dot X^i$。回顾度规,可以写出哈密顿量:
\begin{equation}
	H = p_- \dot X^- + p_i \dot X^i - \mathcal L = \frac{1}{2 p^+} (p^i p^i + m^2) ~.
\end{equation}
因为 $X^+$ 非动力学变量,因而此处并未出现 $p_+ \dot X^+$ 项。同时 $\dot\eta$ 也并未出现,以及 $\eta^{-1} = - p_-$,不能处理为独立的正则坐标\footnote{我们在之后对弦的分析中也将会看到这一点,只不过 $\eta$ 变为了 $\gamma_{\sigma \sigma}$。}。

简单分析就可以发现,选择恰当的规范下,$H = -p_+ = p^-$。

\subsection{开弦的分析}
还是考虑 $-\infty < \tau < \infty$,$0 \le\sigma \le l$。取三个条件:
\begin{equation}
	X^+ = \tau,~~  \partial_\sigma \gamma_{\sigma \sigma} = 0, ~~\det\gamma_{ab} = -1 ~,
\end{equation}
来选定一个好的规范以消除冗余\footnote{规范是有自由度的,Polyakov 作用量有大自由度,选定好的规范(gauge fix)可以消除冗余自由度而大大简化计算。}。

在选定了 $X^+ = \tau$ 后,发现 $f = \gamma_{\sigma \sigma}/\sqrt{-\det \gamma_{ab}}$ 的换系是 $f' \dd \sigma' = f \dd \sigma$。为此定义不变长度 $\dd l = f \dd \sigma$。定义一个点的 $\sigma$ 坐标正比于从 $\sigma = 0$ 到该店的不变长度,比例系数由归一化决定。由于 $f = \dd l / \dd \sigma$ 独立于 $\sigma$ 而依赖于 $\tau$,可以做一个 Weyl 变换来满足 $\det \gamma_{ab} = -1$,而 $f$ 是 Weyl 不变的,$\partial_\sigma f$ 仍是 $0$,也就意味着 $\partial_\sigma \gamma_{\sigma \sigma}=0$ 自动满足。

解出度规 $\gamma$ 仅有两个自由度,而写出度规的逆 
\begin{equation}
	\left[\begin{matrix}
		\gamma^{\tau\tau} & \gamma^{\tau\sigma} \\
		\gamma^{\sigma \tau} & \gamma^{\sigma \sigma }
	\end{matrix}\right] = \left[\begin{matrix}
		-\gamma_{\sigma \sigma}(\tau) & \gamma_{\tau \sigma}(\tau, \sigma) \\
		\gamma_{\tau \sigma}(\tau, \sigma) & \gamma_{\sigma\sigma}^{-1}(\tau)(1 - \gamma_{\tau \sigma}^2(\tau, \sigma))
	\end{matrix}\right] ~.
\end{equation}

此时,Polyakov 拉氏量是 
\begin{equation}
	\begin{aligned}
		L = -\frac{1}{4 \pi \alpha'} \int_0^l \dd \sigma [&\gamma_{\sigma \sigma}(2 \partial_\tau x^- - \partial_\tau X^i \partial_\tau X^i)\\
		-2 & \gamma_{\sigma \tau}(\partial_\sigma Y^- - \partial_\tau X^i \partial_\sigma X^i) \\
		+&\gamma_{\sigma \sigma}^{-1}(1-\gamma_{\tau \sigma}^2) \partial_\sigma X^i \partial_\sigma X^i] ~.
	\end{aligned}
\end{equation}

其中,将 $X^-(\tau, \sigma)$ 拆分为了质心部分 $x^-(\tau)$ 以及 $Y^-(\tau, \sigma)$:
\begin{equation}
	\begin{aligned}
		x^-(\tau) &= \frac{1}{l} \int_0^l \dd \sigma X^-(\tau, \sigma) \\
		Y^-(\tau, \sigma) &= X^-(\tau, \sigma) - x^-(\tau) ~.
	\end{aligned}
\end{equation}

$Y^-$ 是非动力学项,即并不出现在时间导数中。其作用类似拉格朗日乘子,约束 $\partial_\sigma \gamma_{\tau \sigma} = 0$。此外,在现在选定的规范下,开弦的纽曼边界条件变为
\begin{equation}
	\gamma_{\tau \sigma} \partial_\tau X^\mu - \gamma_{\tau \tau} \partial_\sigma X^\mu = 0, ~~ \text{at} ~ \sigma = 0, l ~.
\end{equation}

对于 $\mu = +$,给出在 $\sigma = 0, l$ 处 $\gamma_{\tau \sigma} = 0$,而 $\partial_{\sigma} \gamma_{\tau \sigma} = 0$,因而 $\gamma_{\tau \sigma}$ 在处处都是 $0$。

此外,对于 $\mu = i$,给出在 $\sigma = 0, l$ 时 $\partial_\sigma X^i = 0$。


下面考虑现在改写拉氏量为 
\begin{equation}
	L = -\frac{l}{2\pi \alpha'} \gamma_{\sigma \sigma} \partial_\tau x^- + \frac{1}{4\pi\alpha'} \int_0^l \dd \sigma (\gamma_{\sigma \sigma} \partial_\tau X^i \partial_\tau X^i - \gamma_{\sigma \sigma}^{-1} \partial_\sigma X^i \partial_{\sigma} X^i) ~.
\end{equation}

从而 $x^-$ 的共轭动量是\footnote{正如粒子中的 $\eta$,$\gamma_{\sigma \sigma}$ 也是动量,而非坐标。}
\begin{equation}
	p_- = -p^+ = \frac{\partial L}{\partial(\partial_\tau x^-)} = -\frac{l}{2\pi \alpha'} \gamma_{\sigma \sigma } ~.
\end{equation}

此外,$X^i(\tau, \sigma)$ 的共轭动量密度是 
\begin{equation}
	\Pi^i = \frac{\delta L}{\delta(\partial_\tau X^i)} = \frac{1}{2\pi \alpha'} \gamma_{\sigma \sigma}\partial_\tau X^i = \frac{p^+}{l} \partial_\tau X^i ~.
\end{equation}

计算出哈密顿量:
\begin{equation}
	H = p_- \partial_\tau x^- - L + \int_0^l \dd \sigma \Pi_i \partial_\tau X^i = \frac{l}{4 \pi \alpha' p^+} \int_0^l \dd \sigma\left(2\pi \alpha' \Pi^i \Pi^i + \frac{1}{2\pi \alpha'} \partial_\sigma X^i \partial_\sigma X^i\right) ~.
\end{equation}
\footnote{其实就是 $(D-2)$ 个自由场 $X^i$ 的哈密顿量,而 $p^+$ 是守恒量。}给出运动方程:
\begin{equation}
	\begin{matrix}
		\partial_\tau x^- = \pdv{H}{p_-} = \frac{H}{p^+}, & \partial_\tau p^+ = \pdv{H}{x^- } = 0 ~,\\
		\partial_\tau X^i = \frac{\delta H}{\delta \Pi^i} = 2\pi\alpha' c \Pi^i, & \partial_\tau \Pi^i = -\frac{\delta H}{\delta X^i} = \frac{c}{2\pi \alpha'} \partial_\sigma^2 X^i ~.
	\end{matrix}
\end{equation}
其中 $c = l/(2\pi\alpha' p^+)$。得到波动方程 $\partial_\tau^2 X^i = c^2 \partial_\sigma^2 X^i$。

实际上,$X^\pm$ 也满足波动方程,而 $X^-$ 的计算量较大。