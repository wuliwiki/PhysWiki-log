% 矩阵指数
% 矩阵|矩阵指数|特征值|特征子空间|特征分解

\begin{issues}
\issueMissDepend
\end{issues}


\subsection{定义}
实数域上的指数函数 $\E^x$ 可以进行Maclaulin展开:\begin{equation}
\E^x=\sum\limits_{n=0}^\infty \frac{x^n}{n!}
\end{equation}

展开式使得我们只需要用 $x$ 的幂就可以表示指数 $\E^x$.我们把这一点应用到矩阵中,就可以用方阵的幂来定义出矩阵的指数:

\begin{definition}{矩阵指数}
给定方阵 $\bvec{M}$,定义
\begin{equation}\label{MatExp_eq2}
\E^{\bvec{M}}=\sum_{n=0}^\infty \frac{\bvec{M}^n}{n!}
\end{equation}
并称之为矩阵 $\bvec{M}$ 的\textbf{指数(matrix exponential)}. 其中对于任意方阵 $\bvec{M}$,都有 $\bvec{M}^0=\bvec{I}$,$\bvec{I}$ 是单位矩阵.
\end{definition}

矩阵指数在常微分方程中非常常用,是用来解线性齐次方程组的利器.一个矩阵的指数本身还是一个矩阵.

\subsection{矩阵指数的性质}

\subsubsection{相似变换的统一}

由\textbf{过渡矩阵}\upref{TransM}可知,如果矩阵 $\bvec{M}$ 在某基下表示一个线性变换,那么当基按过渡矩阵 $\bvec{Q}$ 改变时,同一个线性变换的矩阵表示就变为 $\bvec{Q}^{-1}\bvec{M}\bvec{Q}$.在原基下,$\E^{\bvec{M}}$ 可以表示另一个线性变换,而它在 $\bvec{Q}$ 下的变换是
\begin{equation}\label{MatExp_eq1}
\bvec{Q}^{-1}\E^{\bvec{M}}\bvec{Q}=\E^{\bvec{Q}^{-1}\bvec{M}\bvec{Q}}
\end{equation}

也就是说,$\E^{\bvec{M}}$ 所表示的变换,在基变换的时候,其矩阵表示的变换相当于给 $\bvec{M}$ 变换后再取矩阵指数.这意味着我们也可以定义线性变换的指数——也可以反过来说,这是因为我们可以定义线性变换的指数,方式也是使用Maclaulin级数.

事实上,如果 $\mathcal{T}_i$ 表示若干线性变换,我们可以用映射的复合来定义线性变换的乘法:那么对于任意向量 $\bvec{v}$,$\mathcal{T}^n_i(\bvec{v})=\mathcal{T}_i(\mathcal{T}^{n-1}_i(\bvec{v}))$,其中 $\mathcal{T}_i^1=\mathcal{T}_i$.类似地,也可以定义线性变换的加法:$(\mathcal{T}_1+\mathcal{T}_2)(\bvec{v})=\mathcal{T}_1(\bvec{v})+\mathcal{T}_2(\bvec{v})$.这样,有了乘法和加法,就可以计算线性变换的级数了,而Maclaulin级数就可以定义为其指数:
\begin{equation}
\E^\mathcal{T}=\sum\limits_{n=0}^\infty \frac{\mathcal{T}^n}{n!}
\end{equation}
其中 $\mathcal{T}^0$ 是恒等变换,对应单位矩阵.

\autoref{MatExp_eq1} 意味着,如果 $\bvec{M}$ 是 $\mathcal{T}$ 在某基下的矩阵表示,那么 $\E^\mathcal{T}$ 在该基下的矩阵表示就是 $\E^{\bvec{M}}$.

\subsubsection{运算性质}

设 $\bvec{M}, \bvec{N}\in \opn{gl}(n, \mathbb{F})$,$a, b\in\mathbb{F}$,则容易得出以下性质:

如果 $\bvec{MN}=\bvec{NM}$,那么我们有 $\E^{\bvec{M}}\E^{\bvec{N}}=\E^{\bvec{M}+\bvec{N}}$.

$\E^{(\bvec{M}\Tr)}=(\E^{\bvec{M}})\Tr$,$\E^{(\bvec{M}^\dagger)}=({\E^{\bvec{M}}})^\dagger$.

\begin{theorem}{矩阵指数的行列式与矩阵的迹}\label{MatExp_the2}
对于 $\bvec{M}\in\opn{gl}(n, \mathbb{F})$,有 $\abs{\E^{\bvec{M}}}=\E^{\opn{tr}(\bvec{M})}$.即:矩阵指数的行列式,等于矩阵迹的指数.
\end{theorem}

\textbf{证明}:

我们只需要考虑上三角矩阵 $\bvec{M}$ 的情况即可,因为任何矩阵总可以通过相似变换变成上三角矩阵.此时,$\bvec{M}$ 的迹就是主对角元素之和,而 $\bvec{M}^k$ 的第 $i$ 个主对角元素都是 $\bvec{M}$ 的第 $i$ 个主对角元素的 $k$ 次方.

如果只看主对角元素,那么可以记 $\bvec{M}$ 为 $(m_1, m_2,\cdots,m_n)$,其中各 $m_i$ 是 $\bvec{M}$ 的第 $i$ 个元素.类似地,$\bvec{M}^k$ 就可以记为 $(m_1^k, m_2^k,\cdots,m_n^k)$.代入矩阵指数的定义式,可得 $\E^{\bvec{M}}$ 的对角线元素为 $(\E^m_1, \E^m_2,\cdots, \E^m_n)$.

由于上三角矩阵的乘积还是上三角矩阵,可知 $\E^{\bvec{M}}$ 是上三角矩阵,因此 $\abs{\E^{\bvec{M}}}=\E^m_1\times\E^m_2\times\cdots\times\E^m_n=\E^{\E^m_1+\E^m_2+\cdots+\E^m_n}=\E^{\opn{tr}(\bvec{M})}$.

\textbf{证毕}.





\begin{theorem}{矩阵指数求导}\label{MatExp_the1}
矩阵 $\E^{\mat Mt}$ 是一个关于实变量 $t$ 的函数,则
\begin{equation}\label{MatExp_eq3}
\frac{\dd}{\dd t}\E^{\mat Mt}=\mat M\E^{\mat Mt}
\end{equation}
其中求导定义为对每个矩阵元单独求导的结果.
\end{theorem}

\autoref{MatExp_the1} 的形式和 $\frac{\dd}{\dd t}\E^{at}=a\E^{at}$ 是一样的,它们也共享同一个证明,我们留作习题:

\begin{exercise}{}
根据\autoref{MatExp_eq2} 的定义,注意 $\mat M$ 是常数矩阵,证明\autoref{MatExp_eq3} .
\end{exercise}








