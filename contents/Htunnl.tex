% 氢原子隧道电离
% license Xiao
% type Tutor

\pentry{薛定谔方程\nref{nod_TDSE}, 隧道效应}{nod_59f4}

\footnote{参考 Wikipedia \href{https://en.wikipedia.org/wiki/Tunnel_ionization}{相关页面}。}本文使用\enref{原子单位制}{AU}。 在 $E \ll 1$ 的恒定电场下, 氢原子的电离率(单位时间的概率)为
\begin{equation}
\omega = \frac{4}{\abs{E}} \exp[-\frac{2}{3\abs{E}}]~.
\end{equation}

注意若给氢原子的哈密顿算符添加恒定电场项 $\bvec E \vdot \bvec r$, 那么该系统将不存在严格的束缚态。 因为当 $\bvec r$ 和 $\bvec E$ 夹角大于 $90^\circ$ 且当 $r\to\infty$ 时 $\bvec E \vdot \bvec r \to -\infty$ (\autoref{fig_Htunnl_1} )。 所以隧道电离不存在严格的\textbf{阈值(threshold)},理论上任何强度的恒定电场都会产生隧道电离。

\begin{figure}[ht]
\centering
\includegraphics[width=12cm]{./figures/128f221645eb7eff.png}
\caption{隧道电离示意图} \label{fig_Htunnl_1}
\end{figure}

假设氢原子开始时处于某束缚态且无外电场, 能量为 $E_n$, 经过一段时间后外电场出现(图中向左), 这时在原子核的右边就可能会出现一个势垒, 局部势能大于 $E_n$, 但随着 $r$ 增大势能最终小于 $E_n$。 这时根据隧道效应,波函数会以一定的速率穿过该势垒, 单位时间穿过势垒的波函数的概率就叫做\textbf{电离率(ionization rate)}。 电场越弱, 势垒越高越宽, 电离率就越小。

\begin{figure}[ht]
\centering
\includegraphics[width=12cm]{./figures/291176c1eb20e85c.png}
\caption{氢原子隧道电离的数值模拟(见“\enref{球坐标系中的定态薛定谔方程}{RadSE}”)} \label{fig_Htunnl_2}
\end{figure}
\autoref{fig_Htunnl_2} 中画出了氢原子隧道电离的数值模拟结果, 电场方向向上, 强度为 0.1 au, 图中的数值代表波函数模长的平方, 也就是电子在空间中出现的概率(colorbar 使用 log10)。

相反, 若电场太强, 使得右边的势能突起比 $E_n$ 要小, 那么同样不存在隧道效应, 而是直接电离。 这种电离十分迅速。

另见\enref{ADK 电离率}{ADKrat},Keldish 模型(未完成)。
