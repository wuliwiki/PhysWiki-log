% 公理化集合论
% keys 集合|公理|公理系统
% license Usr
% type Tutor
\begin{issues}
\issueTODO
\end{issues}

\pentry{集合\nref{nod_Set}}{nod_dc33}
\subsection{产生原因}
通常来说,我们都采取朴素集合论的观点,认为集合是一个最基本的数学概念,不需要严格的定义。但是,\textbf{罗素悖论(antinomy of Russell)}使得这一观念受到挑战。罗素悖论可以叙述为:已知对于任何一个集合\footnote{这里把集合也看成元素。},都能判定自己是否属于这个集合。所以,我们可以将不属于自己的集合收集起来构成一个集合。因此就应该存在一个集合 $\mathcal{A}=\{A|A\notin A\}$,其中 $A$ 是一个集合。但是,如果认为 $\mathcal{A}\in\mathcal{A}$,那么根据定义,$\mathcal{A}\notin\mathcal{A}$;反过来,如果认为 $\mathcal{A}\notin\mathcal{A}$,根据定义又有 $\mathcal{A}\in\mathcal{A}$。这就构成一个悖论。

罗素悖论有许多通俗版本:
\begin{example}{理发师悖论}
一个小镇里的理发师乐于助人,他说:“我会给所有不给自己理发的人理发。”

请问:理发师是否可以给自己理发?
\end{example}

罗素悖论的存在使人们认识到,朴素集合论并不像他们想象中的那么严谨。为了解决罗素悖论,人们利用公理对集合进行一系列限制,从而使得在新的公理系统中无法构造出罗素悖论中的集合。

\subsection{ZF公理系统}
ZF 公理系统中认为所有的元素都可以看作是集合。区分它们可以用一个简单的方法,看“$\in$” “$\notin$”符号,左边应看作元素,右边应看作集合。

在不同的书上 ZF 公理系统的这些公理可能有不同的顺序和表述,如果学校考试需要请以具体的教科书为准。

下面是 ZF 公理系统的全部公理\footnote{由于 ZF 公理系统认定所有的元素都是集合,所以下面存在一些容易引起混淆的地方,比如集族也是集合。}:

\textbf{本体论承诺(存在公理)} $\exists x: x=x.$ 这告诉我们,存在一个集合。

\textbf{公理 0(空集存在公理)} 表述为:$\exists A,  \forall x: \neg ( x \in A).$ 即存在一个集合使得没有集合是它的元素。这是可以由分离公理模式证明的,故将其列为公理 0,而在实际的 ZF 公理系统中是没有这条公理的,这作为定理或真命题的形式存在。

\textbf{公理 1(外延公理)} 一个集合完全由其元素决定。如果两集合所有元素相等,则这两个集合相等。表述为:给定任意集合 $A$ 和集合 $B$,$A=B$ 当且仅当:给定任意集合中的元素 $x$,$x \in A$ 当且仅当 $x \in B$。这又被成为容积公理。

\textbf{公理 2(无序对公理)} 对于任意两个集合 $x,y$,存在一个集合 $A$,使得对于任意 $w\in A$,$w=x$ 或 $w=y$。这又被成为配对公理。

\textbf{公理 3(并集公理)}对于任意一个集合 $A$,存在一个集合 $B$ 使 $x\in B$ 当且仅当存在一个集合 $y\in A$ 使 $x\in y$。

\textbf{公理 4(幂集公理)} 对任意集合 $x$,存在集合 $y$,使得 $y$ 中有集合 $z$ (表述为 $z \in y$)当且仅当对于 $z$ 的所有元素 $w$,$w\in x$。

\textbf{公理 5(无穷公理)} 存在归纳集。(存在一个集合,空集是其元素,且对其任意元素 $x$,$x^+ = x\cup \{x\}$ 也是其元素)也就是说,存在一集合 $x$,它有无穷多元素。

考虑皮亚诺公理,有时候也会把无穷公理表述为:
存在一个集合 $\mathbb{N}$,使得 $\varnothing\in\mathbb{N}$,且对任意 $x\in \mathbb{N}$,$x\cup\{x\}\in\mathbb{N}$。

把这个集合命名为 $\mathbb{N}$ 是因为根据皮亚诺公理,它就是自然数集合。

\textbf{公理 6(正则公理)}对于任意非空集合 $A$,存在一个 $x\in A$ 使 $x\cap A=\varnothing$。

这里空集的存在性由空集存在公理保证。

接下来的公理实际上是无数条公理的结合\footnote{一阶逻辑禁止量化谓词,所以只能算是无数条公理。},被称为公理模式:

\textbf{公理 7(分离公理模式)}对任意集合 $A$ 和任意对 $A$ 的元素有定义的逻辑谓词 $P(z)$,存在集合 $B$ ,使 $z\in B$ 当且仅当 $z\in A$ 而且 $P(z)$ 为真。

也就是说,我们可以构造一个集合$\{z\in A | P(z)\}$。这在一些书上被称为分离公理(模式),或是\textbf{内涵公理},值得注意的是,这是一个公理模式,是可以由其他公理推导出的。但是这是必不可少的,因为在“直觉主义”的视角下,排中律(也就是:要么 $A$ 成立、要么非 $A$ 成立)并不成立。实际在证明过程中由于排中律而引起问题的地方在于需要证明:非空集合必有元素。

这定理的证明需要由空集存在公理保证。但如果将分离公理模式作为公理加入到公理化体系中,那么空集存在公理可以由分离公理模式证明得出,从而可以不需要空集存在公理。

\textbf{公理 8(替换公理模式)} 对任意集合 $A$ 和任意对 $A$ 的元素有定义的(逻辑)公式$F(z)$,存在集合 $B$,使 $y\in B$ 当且仅当存在 $z\in A$ 而且 $F(z)=y$. 这又被称为置换公理模式。也就是说,由 $F(z)$ 定义的某函数的定义域在一集合 $A$ 中时,他的值域可以被限定在另一集合 $B$ 中。

\subsubsection{罗素悖论的解决}
\begin{theorem}{}
在 ZF 公\label{the_SetAxi_1}理系统中,不存在所有集合的集合。
\end{theorem}

本定理是正则公理的直接推论。

证明:假如这个集合存在,那么利用分离公理模式,我们可以构造所有非空集合的集合,设这个集合为 $A$ ,那么对任意 $x\in A$ , 由于存在集合 $y\in x$,故至少有 $y\in x\cap A$,也就是说 $x\cap A\neq\varnothing$。这与正则公理矛盾。证毕。

也可以用下面证明的引理:不存在以自身为元素的集合。于是所有的集合不能构成一个集合,否则其将以自己为元素,这样就解决了该命题。

\begin{lemma}{}\label{lem_SetAxi_1}
不存在一个集合 $A$ ,使得它的唯一元素为 $A$ 本身。即不存在以自身为元素的集合。
\end{lemma}

证明:由于 $\varnothing$ 无元素,仅考虑非空集合。设存在一个非空集 $A$ 使得 $A \in A$,则由无序对公理可以得到存在集合 $\{A\}$,他仅有一个元素 $A$,由正则公理 $A\cap \{A\} = \varnothing = \varnothing$,而由 $A \in \{A\}, A \in A$ 得到 $A \in (A \cap \{A\})$,矛盾!

故不存在以自身为元素的集合。

\begin{theorem}{}\label{the_SetAxi_2}
对于任何集合 $A$,$A\notin A$。
\end{theorem}

证明:如不然,利用分离公理模式可以构造一个 \autoref{lem_SetAxi_1} 所述的集合。

\begin{theorem}{}
不存在集合$\{A|A\notin A\}$.
\end{theorem}

由 \autoref{the_SetAxi_1} 和 \autoref{the_SetAxi_2} 可得。

\subsection{选择公理和 ZFC 公理系统}

选择公理(Axiom of Choice,又称 AC 公理)的出现源于罗素曾经提到过的以下问题:

现有无穷多个人,每个人都有无穷多只袜子,我们要从每个人手中各挑选出一只袜子,而这些袜子是没有差别的。

在没有选择公理之前,对于这些没有差别的袜子,我们无法保证有一种选法(虽然这是违反直觉的)。所以有了选择公理的出现:

对于任意一个集族 $\mathscr X$:

$$\forall \mathscr X \left[\varnothing \not \in \mathscr X \Rightarrow \exists f: \mathscr X \rightarrow \bigcup \mathscr X, \forall A \in \mathscr X: \left(f(A) \in A\right)\right] ~.$$

需要指出的是,空集是作为 $\mathscr X$ 的一个子集出现的,而不是元素。


AC 与上面所述的 ZF $1-8$ 一同构成了 ZFC 公理系统。

AC 与良序定理在 ZF 的前提下等价,二者选其中之一即可组成 ZFC 公理系统。

AC 还有等价的表述形式:

\begin{enumerate}
\item 给定由相互不交的非空集合组成的任何集合,存在一个集合,它与每个非空集合恰好有一个公共元素。
\item 设 $\mathscr X$ 为一个由非空集合所组成的集合。那么可以从每一个在 $\mathscr X$ 中的集合中,都选择一个元素和其所在的集合配成\textbf{有序对}来组成一个新的集合。
\end{enumerate}

\subsection{基数}
集合的基数可以理解为对集合内元素数量的一种衡量。对于一个集合 $\mathscr A$,用符号 $\left|\mathscr A\right|$ 表示。

具体的,对于有限集,其基数定义为这个集合内的元素的数量。

对于无限集而言,首先需要定义基数的比较方式:

对于两个集合 $\mathscr P$ 与 $\mathscr Q$:
\begin{itemize}
\item 若存在双射 $\mathscr f: \mathscr P \rightarrow \mathscr Q$,则称他们有相同基数,即 $|\mathscr P| = |\mathscr Q|$,又将其称为 $\mathscr P$ 与 $\mathscr Q$ 等势;
\item 若存在单射 $\mathscr g: \mathscr P \rightarrow \mathscr Q$,则说 $|\mathscr P| \le |\mathscr Q|$。
\end{itemize}

特别的,对于任意基数 $\mathscr X$,总存在一个比它更大的基数 $\mathscr X^+$,因此也存在一个\textbf{恰巧比它大}的基数 $\mathscr X^+_0$,而这个基数 $\mathscr X^+_0$ 与原基数 $\mathscr X$ 间不存在其他基数。称这个基数为它的后继基数。

我们定义阿列夫-0($\aleph_0$)为自然数集的基数。

之后的每个后继基数定义为:$\aleph_{n+1}=2^{\aleph_n}$。


\subsection{连续统假设}
称 $\mathscr A$ 为可数集或可列集,当且仅当 $|\mathscr A| \le \aleph_0$。

连续统假设说的是,实数集与自然数集不等势,且实数集的基数为 $\aleph_1$。也就是可列集基数和实数基数之间没有别的基数。
