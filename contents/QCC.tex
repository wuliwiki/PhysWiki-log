% 量类的延拓
% keys 量类|延拓|矢量空间
% license Xiao
% type Tutor

%\begin{issues}
%\issueTODO
%\end{issues}

\pentry{量类和单位\nref{nod_QCU}}{nod_bc91}
在物理中,不少量类(这里指用单位测量量类中的量所得的数值)在通常情况下或不能取负值,或不连续,或只在某一范围取值。比如质量、面积、体积、密度、电容等不会取负值;电荷取值具有量子化性质;引力常量 $\boldsymbol{G}$ 只有一个等等。为建立一套完整的量纲理论,我们需要对量类进行延拓。在此之前,我们先给出一些理由,以便理解。
\subsection{量类延拓的理由}
某些量类对于一些数值是“没有物理意义”的,比如质量非负,然而,“没有物理意义”的说法其实是非常含混的。比如对速率量类,在相对论里其数值不能超过光速,但若限于牛顿力学,任何速率都是允许的。再如温度,我们知道绝对零度不可达到,但总应将绝对零度视作温度量类 $\tilde{\boldsymbol{T}}$ 的元素,否则“绝对零度不可达到”将意义不明。绝对零度下的温度虽然“没有物理意义”,但不妨纳入温度量类 $\tilde{\boldsymbol{T}}$ 这个集合中,只需将其理解成在具体的问题中不出现而已。现在,“温度量类 $\tilde{\boldsymbol{T}}$” 指由开尔文(作为单位)的任意实数倍组成的集合。同样,对于量子化的量类,只需将其取值的性质改为可以连续,那些“没有物理意义”的值只需理解为在具体问题中不会出现或观察不到。更如,对于引力常量量类 $\tilde{\boldsymbol{G}}$ ,人们过去一直认为其是常量,但现在,越来越多的人相信,在宇宙演化的历史长河中,引力常量 $\boldsymbol{G}$ 是在非常缓慢地改变着的(由狄拉克于1937年率先提出)。这一“常量不常”的现象对其它若干物理常量也适用。

当然,对量类进行延拓的最直接理由是建立其对应的数学结构,以便进行各种运算。
\subsection{量类的最大延拓}
\begin{definition}{}
量类 $\tilde{\boldsymbol{Q}}$ 的最大延拓是指:以$\tilde{\boldsymbol{Q}}$ 的任一单位测量延拓后的量类的所有元素,所得实数取遍整个实数集 $\mathbb{R}$.
\end{definition}
今后不加说明,“量类”一词(及其符号$\tilde{\boldsymbol{Q}}$)都指最大延拓的集合,它包含反映正、负状态的两个子集。
\begin{definition}{}
设 $\tilde{\boldsymbol{Q}}$ 是最大延拓的量类,则
\begin{enumerate}
\item $\tilde{\boldsymbol{Q}}_{\text{正}}$ 是$\tilde{\boldsymbol{Q}}$ 的子集,以任一单位测 $\tilde{\boldsymbol{Q}}_{\text{正}}$ 的所有元素的得数能取遍开区间 $(0,\infty)$;
\item $\tilde{\boldsymbol{Q}}_{\text{负}}$ 是$\tilde{\boldsymbol{Q}}$ 的子集,以任一单位测 $\tilde{\boldsymbol{Q}}_{\text{负}}$ 的所有元素的得数能取遍开区间 $(-\infty,0)$
\end{enumerate}
称 $\tilde{\boldsymbol{Q}}_{\text{正}}$ 和$\tilde{\boldsymbol{Q}}_{\text{负}}$ 分别为量类 $\tilde{\boldsymbol{Q}}$ 的\textbf{正半轴} 和 \textbf{负半轴}。
\end{definition}
\subsection{量类是1维矢量空间}
现在,我们可给出量类的加法、数乘和零元的定义。
\subsubsection{加法}
\begin{figure}[ht]
\centering
\includegraphics[width=8cm]{./figures/eacfa5bdb923f0f2.pdf}
\caption{米尺测量木块的长} \label{fig_QCC_1}
\end{figure}
如图,用米尺测量两个同样的长方体木块(左图),其长为 $1\boldsymbol{m}$,后将两个木块拼接在一起(右图),其长为 $2\boldsymbol{m}$。对同样的情况,若两木块长分别为 $\alpha ,\beta$(以 $\boldsymbol{m}$ 为单位),则拼接在一起的木块长为 $\alpha+\beta$。这提示我们可对同类量的加法进行如下定义
\begin{definition}{}
设 $\boldsymbol{Q_1},\boldsymbol{Q_2}\in \tilde{\boldsymbol{Q}}$ ,任选单位 $\hat{\boldsymbol{Q}}\in\tilde{\boldsymbol{Q}}_{\text{正}}$ ,有 $\boldsymbol{Q_1}=Q_1\hat{\boldsymbol{Q}}, \boldsymbol{Q_2}=Q_2\hat{\boldsymbol{Q}}$ ,则 $\boldsymbol{Q_1}+\boldsymbol{Q_2}\in\tilde{\boldsymbol{Q}}$,其定义为
\begin{equation}
\boldsymbol{Q_1}+\boldsymbol{Q_2}:=(Q_1+Q_2)\hat{\boldsymbol{Q}}~.
\end{equation}
\end{definition}
\begin{example}{}
试证明:$\boldsymbol{Q_1}+\boldsymbol{Q_2}$ 与所选单位无关。即对于任选单位 $\hat{\boldsymbol{Q}},\hat{\boldsymbol{Q'}}\in\tilde{\boldsymbol{Q}}$,所得结果分别记为 $\boldsymbol{Q_1}+\boldsymbol{Q_2}$ 和 $\boldsymbol{Q_1}'+\boldsymbol{Q_2}'$ ,则
\begin{equation}
\boldsymbol{Q_1}+\boldsymbol{Q_2} =\boldsymbol{Q_1}'+\boldsymbol{Q_2}'~.
\end{equation}
 该结论由\autoref{the_QCU_3}~\upref{QCU} 直接得到。
\end{example}
\subsubsection{数乘}
\autoref{the_QCU_2}~\upref{QCU} 使得我们能够在量类当中定义数乘。
\begin{definition}{}
设 $\boldsymbol{Q}\in\tilde{\boldsymbol{Q}}$,任选单位 $\hat{\boldsymbol{Q}}$ 便有 $\boldsymbol{Q}=Q\hat{\boldsymbol{Q}}$。任一实数 $\alpha$ 与 $\boldsymbol{Q}$ 的\textbf{数乘} $\alpha\boldsymbol{Q}\in\tilde{\boldsymbol{Q}}$ 定义为
\begin{equation}
\alpha\boldsymbol{Q}:=(\alpha Q)\hat{\boldsymbol{Q}}~.
\end{equation}
\end{definition}
\subsubsection{零元}
\begin{definition}{}
$\tilde{\boldsymbol{Q}}$ 的元素称为\textbf{零元},记作 $\bvec{0}$ ,若存在 $\boldsymbol{Q_0}\in\tilde{\boldsymbol{Q}}$ 使得用 $\boldsymbol{Q_0}$ 测该元素得数为0.
\end{definition}
容易证明,用任一非零元素 $\boldsymbol{Q}\in\tilde{\boldsymbol{Q}}$ 测 $\tilde{\boldsymbol{Q}}$ 的零元得数都是0.
\begin{example}{}
试证明:每个量类 $\tilde{\boldsymbol{Q}}$ 的零元是唯一的。
\textbf{证明:}设存在两个零元 $\bvec{0}$ 和 $\bvec{0'}$,则存在 $\boldsymbol{Q_0},\boldsymbol{Q_0'}$ 使得 
\begin{equation}
\bvec{0}=0\boldsymbol{Q_0}~,\quad \bvec{0'}=0\boldsymbol{Q_0'}~.
\end{equation}
于是
\begin{equation}
\bvec{0'}=0\boldsymbol{Q_0'}=0\frac{\boldsymbol{Q_0'}}{\boldsymbol{Q_0}}\boldsymbol{Q_0}=0\boldsymbol{Q_0}=\boldsymbol{0}~.
\end{equation}
\end{example}
可以验证,这样定义的加法、数乘、零元满足\enref{矢量空间}{LSpace}的定义。所以每个量类都是(实数域上的)一个矢量空间。又因为它只有一个线性独立的元素(不妨就选取作为单位的量),即只有一个基矢,所以是1维矢量空间。
