% 狄拉克方程的非相对论近似
% keys 狄拉克方程|泡利方程
% license Xiao
% type Tutor

\pentry{狄拉克方程\nref{nod_qed4}}{nod_415f}

\enref{狄拉克方程}{qed4}是描述相对论性自由电子的方程\autoref{eq_qed4_9} :(这里我们没有采用自然单位制,所以需要带上 $\hbar,c$)
\begin{equation}
\begin{aligned}
&i\hbar \pdv{t} \psi = H\psi~,\\
&H=c\bvec \alpha\cdot \bvec p+mc^2\beta~,
\end{aligned}
\end{equation}
其中 $\bvec p=-i\hbar\nabla$;$\bvec \alpha,\beta$ 是四维矩阵代数中的元素,满足一定的反对易关系。更常见地,上式也可以写成\autoref{eq_qed4_22}  的形式:
\begin{equation}
i\qty(\gamma^\mu \partial_\mu-\frac{mc}{\hbar})\psi(x)=0~,
\end{equation}
其中 $\partial_\mu=\qty(\pdv{(ct)},\pdv{x},\pdv{y},\pdv{z})$。

\subsection{自由电子狄拉克方程的非相对论近似}

下面我们将从狄拉克方程出发,得到它的非相对论近似。设平面波解
\begin{equation}
\psi=\pmat{\varphi\\\chi}\exp(-imc^2t/\hbar)~,
\end{equation}
将它代入狄拉克方程。这里不妨采用 $\bvec \alpha,\beta$ 的标准表示\autoref{eq_qed4_6},则有
\begin{equation}\label{eq_DiracB_1}
\begin{aligned}
&i\hbar\pdv{t} \varphi=c\bvec \sigma\cdot \bvec p \chi~,\\
&i\hbar\pdv{t} \chi = c\bvec \sigma\cdot \bvec p \varphi -2mc^2 \chi ~.
\end{aligned}
\end{equation}
注意上面的第二行式子中,由于在非相对论极限下 $\pdv{}{t} \chi$ 相比等式右边带 $c$ 的分量可以略去,所以有近似等式
\begin{equation}\label{eq_DiracB_4}
\chi\approx \frac{1}{2mc} \bvec \sigma\cdot \bvec p \varphi~.
\end{equation}
将它代入\autoref{eq_DiracB_1} 的第一行,可以得到
\begin{equation}\label{eq_DiracB_2}
i\hbar\pdv{t} \varphi=\frac{1}{2m}(\bvec \sigma\cdot \bvec p)^2 \varphi~.
\end{equation}
可以将此式与\autoref{eq_scheq2_3}  进行对比,可以发现两种方式推导出的非相对论性自旋 1/2 粒子的波函数是一致的。

\subsection{电磁场中狄拉克方程的非相对论近似}
\pentry{电磁场中的狄拉克方程\nref{nod_DiracE}}{nod_bfab}
缓变外场中狄拉克方程可以写为\autoref{eq_DiracE_4}  的形式:
\begin{equation}
(i\gamma^\mu \partial_\mu -\frac{q}{\hbar}\gamma^\mu A_\mu - \frac{mc}{\hbar})\psi(x)=0~.
\end{equation}
在高斯单位制或者洛伦兹-亥维赛单位制下讨论其非相对论极限下的方程,可以令 $A_\mu=(\phi,-\bvec A/c)$。那么这相当于于将前面的自由狄拉克方程作以下的替换:
\begin{equation}
i\hbar \partial_t \rightarrow i\hbar\partial_t -q\phi, i\hbar\partial_x\rightarrow i\hbar\partial_x - \frac{q}{c}A_x~.
\end{equation}
或者说,用能量动量算符来表达:
\begin{equation}
\hat H\rightarrow \hat H-q\phi,\bvec{\hat P}\rightarrow \hat {\bvec P}-\frac{q}{c}\bvec A~.
\end{equation}
因此很容易将 \autoref{eq_DiracB_2} 推广到电磁场中狄拉克方程的非相对论极限:
\begin{equation}\label{eq_DiracB_3}
i\hbar\pdv{t} \varphi = \frac{1}{2m}\qty[\bvec \sigma \cdot\qty(\bvec {\hat P} - \frac{q}{c}\bvec A)]^2 \varphi + q\phi  \varphi~,
\end{equation}
这里 $\hat{\bvec P}=-i\hbar\nabla$。继续推导则可以得到\enref{泡利方程}{Pauli}。
