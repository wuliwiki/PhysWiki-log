% 静电场的环路定理

\pentry{电势、电势能\upref{QEng}}

根据静电场的保守性质\upref{QEng},电场力做的功与具体路径无关,而只与始末态有关。
\begin{figure}[ht]
\centering
\includegraphics[width=5cm]{./figures/ELECLD_1.pdf}
\caption{电荷做环路运动一周,电场力做功之和为零} \label{ELECLD_fig1}
\end{figure}

设想一电荷沿某一环路运动一圈,由于始末态相同,因此电场力做功为零。
$$
W = \oint q\bvec E \cdot \dd \bvec l = 0
$$
即
\begin{equation}
\oint \bvec E \cdot \dd \bvec l = 0
\end{equation}
此即为静电场的环路性质。

根据斯托克斯定理\upref{Stokes},该定理还可以写为
$$\int \curl \bvec E \cdot \dd \bvec s = 0$$
由于对任意环路均成立,因此
\begin{equation}
\curl \bvec E = 0
\end{equation}

注意,该定理只适用于静电场情况。若存在变化的磁场等,则需要考虑电磁感应\upref{FaraEB}。