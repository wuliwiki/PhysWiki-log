% 栈
% 栈|数据结构|C++

栈是一种“先进后出”的数据结构.也是计算机实现递归和基本结构,栈只有一端可以进出元素,这一段被称为“栈顶”,另一端被称为“栈底”.往栈中插入元素被称为”进栈“,往栈中删除元素被称为“出栈”.
C++ 的 STL 已经帮助我们实现好了栈,一般情况我们可以直接使用 STL 库里的栈.

栈的常用操作:
\begin{enumerate}
\item 向栈顶插入一个数 $x$;
\item 从栈顶弹出一个数;
\item 判断栈是否为空;
\item 查询栈顶元素.
\end{enumerate}

C++ STL
\begin{lstlisting}[language=cpp]
stack<int> stk;
stk.push(x);     // 在栈顶插入一个数 x
stk.pop();       // 弹出栈顶元素, 但不返回其值
if (stk.empty()) // 果栈为空则返回 true, 否则返回 false;
cout << stk.top() << endl;   // 输出栈顶元素, 但不删除该元素

// 还有如下几个操作
stk.size();        // 返回栈中元素的个数

\end{lstlisting}

但我们在这里详细的讲一下如何使用数组模拟栈.

定义一个数组 $\mathtt{stk}$ 用于模拟栈,再定义一个变量表示栈顶,初始化为 $-1$.

\begin{lstlisting}[language=cpp]
string op;
cin >> op;

int tt = -1;
if (op == "push")
{
    int x;
    cin >> x;
    stk[ ++ tt] = x;   // 向栈顶添加一个数
} 
if (op == "query") cout << stk[tt] << endl;
if (op == "pop") tt -- ;
if (op == "empty") cout << (tt ? "NO" : "YES") << endl;  // 栈为空则输出 NO,否则为 YES
\end{lstlisting}

以上就是栈的基本操作了.