% 粒子物理学(综述)
% license CCBYSA3
% type Wiki

本文根据 CC-BY-SA 协议转载翻译自维基百科\href{https://en.wikipedia.org/wiki/Particle_physics}{相关文章}。

粒子物理学或高能物理学是研究构成物质和辐射的基本粒子和力的学科。该领域还研究从基本粒子到质子和中子尺度的粒子组合,而研究质子和中子组合的学科称为核物理学。

宇宙中的基本粒子在标准模型中被分类为费米子(物质粒子)和玻色子(传递力的粒子)。费米子有三代,然而普通物质仅由第一代费米子构成。第一代包括构成质子和中子的上夸克和下夸克,以及电子和电子中微子。已知由玻色子介导的三种基本相互作用是电磁相互作用、弱相互作用和强相互作用。

夸克不能单独存在,而是形成强子。含有奇数个夸克的强子称为重子,而含有偶数个夸克的强子称为介子。两个重子,质子和中子,构成了普通物质的大部分质量。介子是不稳定的,最长寿命的介子也只有几微秒。介子发生在由夸克组成的粒子之间的碰撞后,例如在宇宙射线中快速运动的质子和中子。介子也可以在回旋加速器或其他粒子加速器中产生。

粒子有对应的反粒子,具有相同的质量,但电荷相反。例如,电子的反粒子是正电子。电子带负电,正电子带正电。这些反粒子理论上可以形成一种对应的物质形式,称为反物质。一些粒子,如光子,是它们自身的反粒子。

这些基本粒子是量子场的激发,这些量子场也支配着它们的相互作用。解释这些基本粒子和场及其动力学的主导理论被称为标准模型。引力与当前粒子物理学理论的统一尚未解决;许多理论已经试图解决这个问题,如环量子引力理论、弦理论和超对称理论。

实际粒子物理学是研究这些粒子在放射性过程和粒子加速器中的行为,如大型强子对撞机。理论粒子物理学则是在宇宙学和量子理论的背景下研究这些粒子。这两者是紧密相关的:希格斯玻色子是由理论粒子物理学家提出的,并通过实际实验确认了它的存在。
\subsection{历史}
\begin{figure}[ht]
\centering
\includegraphics[width=10cm]{./figures/f59246dd62ba0e0e.png}
\caption{盖革–马斯登实验观察到,当α粒子撞击金箔时,一小部分α粒子发生了剧烈的偏转。} \label{fig_Partic_1}
\end{figure}
所有物质基本上由基本粒子构成的思想至少可以追溯到公元前6世纪。[1] 在19世纪,约翰·道尔顿通过对化学计量学的研究得出结论,认为自然界的每种元素都由一种独特的粒子组成。[2] “原子”一词来源于希腊语“atomos”,意为“不可分割”,此后该词一直用来表示化学元素中最小的粒子,但物理学家后来发现,原子实际上并不是自然界的基本粒子,而是由更小的粒子(如电子)组成的聚合物。20世纪初,核物理学和量子物理学的研究导致了1939年莉泽·迈特纳(基于奥托·哈恩的实验)证明了核裂变,并且汉斯·贝特在同一年发现了核聚变;这两项发现也促成了核武器的发展。贝特1947年计算的兰姆位移被认为“开启了粒子物理学现代时代的道路”。[3]

在1950年代和1960年代,随着越来越高能量的粒子束碰撞,发现了种类繁多的粒子,这种现象被非正式地称为“粒子动物园”。詹姆斯·克罗宁和瓦尔·费奇的CP破坏等重要发现提出了物质和反物质不平衡的新问题。[4] 在1970年代标准模型的制定之后,物理学家澄清了粒子动物园的来源。大量粒子的出现被解释为(相对较小数量的)更基本粒子的组合,并被框架化在量子场理论的背景下。这种重新分类标志着现代粒子物理学的开始。[5][6]
\subsection{标准模型}  
目前所有基本粒子的分类状态由标准模型解释,该模型在1970年代中期获得广泛接受,此时实验确认了夸克的存在。标准模型描述了强相互作用、弱相互作用和电磁相互作用,并通过介质规范玻色子来实现这些相互作用。规范玻色子的种类包括八种胶子、W⁻、W⁺和Z玻色子,以及光子。标准模型还包含24种基本费米子(12种粒子及其对应的反粒子),这些费米子是所有物质的组成部分。最后,标准模型还预测了一种名为希格斯玻色子的玻色子的存在。2012年7月4日,欧洲核子研究组织(CERN)的“大型强子对撞机”实验的物理学家宣布,他们发现了一种新粒子,其行为与预期的希格斯玻色子相似。

目前的标准模型包含61种基本粒子。这些基本粒子可以组合成复合粒子,解释了自1960年代以来发现的数百种其他粒子种类。标准模型与迄今为止进行的几乎所有实验测试一致。然而,大多数粒子物理学家认为,标准模型是对自然的一个不完整描述,且更为基本的理论有待发现(参见万物理论)。近年来,中微子质量的测量提供了与标准模型的首个实验偏差,因为标准模型中认为中微子没有质量。
\subsection{亚原子粒子}
\begin{figure}[ht]
\centering
\includegraphics[width=10cm]{./figures/3f36f74204fb82e2.png}
\caption{} \label{fig_Partic_2}
\end{figure}
现代粒子物理学研究的重点是亚原子粒子,包括原子成分,如电子、质子和中子(质子和中子是由夸克组成的复合粒子,称为重子),这些粒子通过放射性过程和散射过程产生;这些粒子包括光子、中微子、缪子以及各种异域粒子。迄今为止观察到的所有粒子及其相互作用几乎可以完全通过标准模型来描述。

粒子的动态也由量子力学支配;它们表现出波粒二象性,在某些实验条件下表现出粒子特性,而在其他条件下则表现出波动特性。更技术化的说法是,它们通过量子态矢量描述,这些矢量位于希尔伯特空间中,量子场论也对其进行了处理。遵循粒子物理学家的惯例,术语“基本粒子”用于描述那些根据当前理解,被假定为不可分割且不由其他粒子组成的粒子。
\subsubsection{夸克和轻子}
\begin{figure}[ht]
\centering
\includegraphics[width=6cm]{./figures/2176313ce5a825b6.png}
\caption{一个费曼图,展示了β−衰变,显示一个中子(n, udd)转化为一个质子(p, udu)。"u" 和 "d" 是上夸克和下夸克,"e−" 是电子,"νe" 是电子反中微子。} \label{fig_Partic_3}
\end{figure}
普通物质由第一代夸克(上夸克、下夸克)和轻子(电子、电子中微子)构成。[13] 夸克和轻子统称为费米子,因为它们具有半整数的量子自旋(−1/2,1/2,3/2等)。这使得费米子遵循泡利不相容原理,即任何两个粒子不能占据相同的量子态。[14] 夸克具有分数电荷(−1/3或2/3),而轻子具有整数电荷(0或−1)。[15][16] 夸克还具有色荷,这个色荷是任意标记的,与实际的光色没有关联,通常标记为红色、绿色和蓝色。[17] 由于夸克之间的相互作用会储存能量,而当夸克足够远离时,这些能量可以转化为其他粒子,因此夸克无法独立观测到。这种现象称为色禁闭。[17]

已知夸克有三代(上夸克和下夸克、奇夸克和魅夸克、顶夸克和底夸克),轻子也有三代(电子及其中微子、μ子及其中微子、τ子及其中微子),有强有力的间接证据表明不存在第四代费米子。[18]
\subsubsection{玻色子}  
玻色子是基本相互作用的媒介或载体,如电磁相互作用、弱相互作用和强相互作用。[19] 电磁相互作用通过光子(光的量子)进行媒介作用。[20]: 29–30  弱相互作用由W玻色子和Z玻色子媒介。[21] 强相互作用由胶子媒介,胶子能够将夸克连接在一起形成复合粒子。[22] 由于上述的颜色禁闭,胶子从未被独立观察到。[23] 希格斯玻色子通过希格斯机制赋予W玻色子和Z玻色子质量[24]——而胶子和光子预计是无质量的。[23] 所有玻色子具有整数量子自旋(0和1),并且可以处于相同的量子态。[19]
\subsubsection{反粒子和颜色电荷}  
前述大多数粒子都有对应的反粒子,它们组成了反物质。普通粒子具有正的轻子数或重子数,而反粒子则具有这些数值的负值。[25] 对应的反粒子和粒子的属性大多数是相同的,只有少数属性是相反的;例如,电子的反粒子——正电子,具有相反的电荷。为了区分反粒子和粒子,通常在符号上方加上正负号。例如,电子和正电子分别用 \( e^- \) 和 \( e^+ \) 表示。[26] 然而,如果粒子的电荷为0(与反粒子的电荷相等),反粒子则在符号上方加一条线。因此,电子中微子是 \( \nu_e \),而其反中微子则是 \( \overline{\nu_e} \)。当粒子和反粒子相互作用时,它们会发生湮灭并转化为其他粒子。[27] 有些粒子,例如光子或胶子,没有反粒子。[citation needed]

夸克和胶子还具有颜色电荷,这影响强相互作用。夸克的颜色电荷被称为红色、绿色和蓝色(尽管粒子本身没有物理颜色),而反夸克的颜色电荷则被称为反红、反绿和反蓝。[17] 胶子可以具有八种颜色电荷,这些颜色电荷是夸克相互作用形成复合粒子的结果(规范对称性 SU(3))。[28]
\subsubsection{复合的}
\begin{figure}[ht]
\centering
\includegraphics[width=6cm]{./figures/b8cc42153e6bbe27.png}
\caption{一个质子由两个上夸克和一个下夸克组成,这些夸克通过胶子相互连接。夸克的颜色电荷也可见。} \label{fig_Partic_4}
\end{figure}
原子核中的中子和质子是重子——中子由两个下夸克和一个上夸克组成,质子由两个上夸克和一个下夸克组成。[29] 重子由三个夸克组成,而介子则由两个夸克组成(一个是正常的,一个是反夸克)。重子和介子统称为强子。强子内部的夸克受强相互作用支配,因此遵循量子色动力学(颜色电荷)。束缚状态下的夸克必须使其颜色电荷保持中性,或者“白色”,类似于混合原色的比喻。[30] 更为奇特的强子可能有其他类型、排列或数量的夸克(四夸克、五夸克)。[31]

原子由质子、中子和电子组成。[32] 通过改变普通原子内部的粒子,可以形成奇异原子。[33] 一个简单的例子是氢-4.1,其中一个电子被缪子取代。[34]
\subsubsection{假设的}  
引力子是一种假设的粒子,能够介导引力相互作用,但尚未被探测到或与当前的理论完全统一。[35] 许多其他假设的粒子也已被提出,以解决标准模型的局限性。特别地,超对称粒子旨在解决层次问题,轴子解决强CP问题,而各种其他粒子则被提出用来解释暗物质和暗能量的起源。
\subsection{实验室}
世界主要的粒子物理实验室包括:
\begin{itemize}
\item \textbf{布鲁克海文国家实验室}(美国纽约长岛):其主要设施是相对论重离子对撞机(RHIC),该设施碰撞重离子,如金离子和极化质子。它是世界上第一个重离子对撞机,也是世界上唯一的极化质子对撞机。[36][37]
\item \textbf{布德科尔核物理研究所}(俄罗斯诺沃西比尔斯克):其主要项目包括自2006年运营的电子-正电子对撞机VEPP-2000,[38] 和1994年开始实验的VEPP-4。[39] 早期的设施包括第一个电子-电子束对撞机VEP-1,该设备于1964至1968年间进行实验;电子-正电子对撞机VEPP-2,运营时间为1965至1974年;以及其继任者VEPP-2M,[40] 在1974至2000年间进行实验。[41]
\item \textbf{欧洲核子研究中心(CERN)}(法瑞边界,日内瓦附近,瑞士)。其主要项目是大强子对撞机(LHC),该对撞机在2008年9月10日进行了首次束流循环,并且现在是世界上能量最高的质子对撞机。自开始碰撞铅离子以来,它也成为了世界上能量最高的重离子对撞机。早期的设施包括大型电子-正电子对撞机(LEP),该对撞机在2000年11月2日停止运行,并被拆除,为LHC的建设腾出空间;超级质子同步加速器(SPS),该加速器目前作为LHC的预加速器,并用于固定靶实验。[42]
\item \textbf{德意志电子同步加速器(DESY)}(德国,汉堡)。其主要设施曾为哈德龙电子环装置(HERA),该装置将电子和正电子与质子碰撞。[43] 目前,加速器综合体主要集中在利用PETRA III、FLASH和欧洲XFEL产生同步辐射。
\item 费米国家加速器实验室(Fermilab)(美国,伊利诺伊州,巴塔维亚)。其主要设施直到2011年为Tevatron,这是一个质子和反质子对撞机,直到2009年11月29日LHC超过其能量,成为地球上能量最高的粒子对撞机。[44]
\item 高能物理研究所(IHEP)(中国,北京)。IHEP管理着中国多个主要粒子物理设施,包括北京电子-正电子对撞机II(BEPC II)、北京谱仪(BES)、北京同步辐射源(BSRF)、杨坝井国际宇宙射线观测站(位于西藏)、大亚湾反应堆中微子实验、中国散裂中子源、硬X射线调制望远镜(HXMT)、加速器驱动次临界系统(ADS)以及江门地下中微子实验(JUNO)。[45]
\item \textbf{高能加速器研究机构(KEK)}(日本,筑波)。这是多个实验的所在地,例如K2K实验(一个中微子振荡实验)和Belle II实验(一个测量B介子CP破坏的实验)。[46]
\item \textbf{SLAC国家加速器实验室}(美国,加利福尼亚州,门洛帕克)。其2英里长的线性粒子加速器自1962年开始运营,是许多电子和正电子碰撞实验的基础,直到2008年。自那时起,该线性加速器被用于线性相干光源X射线激光以及先进加速器设计研究。SLAC的工作人员继续参与全球多个粒子探测器的开发和建设。[47]
\end{itemize}
\subsection{理论}
理论粒子物理学试图发展模型、理论框架和数学工具,以理解当前的实验并为未来的实验做出预测(另见理论物理学)。目前,理论粒子物理学中有几个主要的相互关联的努力。

一个重要的分支是试图更好地理解标准模型及其测试。理论学家对碰撞机和天文实验中的可观测量做出定量预测,这些预测与实验测量一起用于提取标准模型的参数,从而减少不确定性。这项工作探讨了标准模型的极限,因此拓展了对自然基本构成部分的科学理解。由于量子色动力学中计算高精度量的困难,这些努力充满挑战。一些从事这一领域的理论学家使用微扰量子场论和有效场论的工具,称自己为现象学家[引用需要]。另一些则利用格点场论,称自己为格点理论家。

另一个重要的努力是模型构建,模型构建者开发关于标准模型之外的物理学(在更高能量或更小距离下)的理论。这项工作通常由层次问题驱动,并受现有实验数据的约束[48][49]。它可能涉及超对称、替代希格斯机制、额外空间维度(例如兰德尔–孙德伦模型)、前子理论、这些理论的组合或其他想法。消失维度理论是一种粒子物理学理论,建议具有更高能量的系统具有较少的维度[50]。

理论粒子物理学中的第三个主要努力是弦理论。弦理论家试图通过构建一个基于小弦和膜而非粒子的理论,来构造量子力学和广义相对论的统一描述。如果该理论成功,它可能被视为“万物理论”或“TOE”[51]。

此外,理论粒子物理学还有其他领域的工作,从粒子宇宙学到圈量子引力[引用需要]。
\subsection{实际应用}
原则上,所有物理学(以及由此发展出的实际应用)都可以通过研究基本粒子来推导出来。实际上,即使“粒子物理学”仅指“高能原子撞击器”,许多在这些开创性研究过程中开发的技术后来也在社会中得到了广泛应用。粒子加速器被用于生产医学同位素用于研究和治疗(例如,用于PET成像的同位素),或者直接用于外部束放射治疗。超导体的发展也得到了粒子物理学的推动。万维网和触摸屏技术最初是在CERN开发的。粒子物理学的其他应用还包括医学、国家安全、工业、计算、科学和劳动力发展,展示了粒子物理学对社会做出的广泛且日益增长的有益贡献。
\subsection{未来}
为了探索标准模型之外的物理学,主要的努力包括为CERN提议的未来圆形对撞机(Future Circular Collider)[53],以及美国的粒子物理学项目优先排序小组(P5),该小组将更新2014年的P5研究,推荐进行深地下中微子实验(Deep Underground Neutrino Experiment)等其他实验。
\subsection{另见}
\begin{itemize}
\item 粒子物理学与表示理论
\item 原子物理学
\item 天文学
\item 高压
\item 国际高能物理学会议
\item 量子力学简介
\item 粒子物理学中的加速器列表
\item 粒子列表
\item 磁单极子
\item 微型黑洞
\item 数论
\item 共振(粒子物理学)
\item 高能物理中的自洽原理
\item 非广延自洽热力学理论
\item 标准模型(数学形式化)
\item 斯坦福物理信息检索系统
\item 粒子物理学时间轴
\item 无粒子物理学
\item 四夸克
\item 路径重要性
\item 国际光子学、电子学和原子碰撞会议
\end{itemize}
\subsection{参考文献}
\begin{enumerate}
\item "Fundamentals of Physics and Nuclear Physics" (PDF). 从原始版本存档于2012年10月2日。检索于2012年7月21日。
\item Grossman, M. I. (2014). "John Dalton and the London Atomists". 《伦敦皇家学会笔记与记录》, 68 (4): 339–356. doi:10.1098/rsnr.2014.0025. PMC 4213434.
\item Brown, Gerald Edward; Lee, Chang-Hwan (2006). *Hans Bethe and His Physics*. 新加坡: 世界科学出版社. p. 161. ISBN 978-981-256-609-6.
\item "Antimatter". 2021年3月1日. 从原始版本存档于2018年9月11日。检索于2021年3月12日。
\item Weinberg, Steven (1995–2000). *The quantum theory of fields*. 剑桥: 剑桥大学出版社. ISBN 978-0521670531.
\item Jaeger, Gregg (2021). "The Elementary Particles of Quantum Fields". *Entropy*. 23 (11): 1416. Bibcode:2021Entrp..23.1416J. doi:10.3390/e23111416. PMC 8623095. PMID 34828114.
\item Baker, Joanne (2013). *50 quantum physics ideas you really need to know*. 伦敦. pp. 120–123. ISBN 978-1-78087-911-6. OCLC 857653602.
\item Nakamura, K. (2010年7月1日). "Review of Particle Physics". *《核物理与粒子物理学杂志》* 37 (7A): 1–708. Bibcode:2010JPhG...37g5021N. doi:10.1088/0954-3899/37/7A/075021. hdl:10481/34593. PMID 10020536.
\item Mann, Adam (2013年3月28日). "Newly Discovered Particle Appears to Be Long-Awaited Higgs Boson". *Wired Science*. 从原始版本存档于2014年2月11日。检索于2014年2月6日。
\item Braibant, S.; Giacomelli, G.; Spurio, M. (2009). *Particles and Fundamental Interactions: An Introduction to Particle Physics*. 施普林格. pp. 313–314. ISBN 978-94-007-2463-1. 从原始版本存档于2021年4月15日。检索于2020年10月19日。
\item "Neutrinos in the Standard Model". T2K 合作组. 从原始版本存档于2019年10月16日。检索于2019年10月15日。
\item Terranova, Francesco (2021). *A Modern Primer in Particle and Nuclear Physics*. 牛津大学出版社. ISBN 978-0-19-284524-5.
\item Povh, B.; Rith, K.; Scholz, C.; Zetsche, F.; Lavelle, M. (2004). "Part I: Analysis: The building blocks of matter". *Particles and Nuclei: An Introduction to the Physical Concepts* (第4版). 施普林格. ISBN 978-3-540-20168-7. 从原始版本存档于2022年4月22日。检索于2022年7月28日。普通物质完全由第一代粒子组成,即u和d夸克,以及电子及其中微子。
\item Peacock, K. A. (2008). *The Quantum Revolution*. Greenwood Publishing Group. p. 125. ISBN 978-0-313-33448-1.
\item Quigg, C. (2006). "Particles and the Standard Model". 见G. Fraser (编辑). *The New Physics for the Twenty-First Century*. 剑桥大学出版社. p. 91. ISBN 978-0-521-81600-7.
\item Serway, Raymond A.; Jewett, John W. (2013年1月1日). *Physics for Scientists and Engineers, Volume 2*. Cengage Learning. ISBN 978-1-285-62958-2.
\item Nave, R. "The Color Force". HyperPhysics. 乔治亚州立大学物理与天文学系. 从原始版本存档于2018年10月7日。检索于2009年4月26日。
\item Decamp, D. (1989). "Determination of the number of light neutrino species". *Physics Letters B*. 231 (4): 519–529. Bibcode:1989PhLB..231..519D. doi:10.1016/0370-2693(89)90704-1. hdl:11384/1735.
\item Carroll, Sean (2007). *Guidebook. Dark Matter, Dark Energy: The dark side of the universe*. The Teaching Company. 第二部分,p. 43. ISBN 978-1598033502. … 玻色子:一种力载体粒子,与物质粒子(费米子)相对。玻色子可以无限堆叠在一起。例子包括光子、胶子、引力子、弱玻色子和希格斯玻色子。玻色子的自旋始终是整数:0, 1, 2, 等等……
\item "Role as gauge boson and polarization" §5.1 in Aitchison, I. J. R.; Hey, A. J. G. (1993). *Gauge Theories in Particle Physics*. IOP Publishing. ISBN 978-0-85274-328-7.
\item Watkins, Peter (1986). *Story of the W and Z*. 剑桥: 剑桥大学出版社. p. 70. ISBN 9780521318754. 从原始版本存档于2012年11月14日。检索于2022年7月28日。
\item Nave, C. R. "The Color Force". HyperPhysics. 乔治亚州立大学物理系. 从原始版本存档于2018年10月7日。检索于2012年4月2日。
\item Debrescu, B. A. (2005). "Massless Gauge Bosons Other Than The Photon". *Physical Review Letters*. 94 (15): 151802. arXiv:hep-ph/0411004. Bibcode:2005PhRvL..94o1802D. doi:10.1103/PhysRevLett.94.151802. PMID 15904133. S2CID 7123874.
\item Bernardi, G.; Carena, M.; Junk, T. (2007). "Higgs bosons: Theory and searches" (PDF). *Review: Hypothetical particles and Concepts*. Particle Data Group. 从原始版本存档(PDF)于2018年10月3日。检索于2022年7月28日。
\item Tsan, Ung Chan (2013). "Mass, Matter, Materialization, Mattergenesis and Conservation of Charge". *International Journal of Modern Physics E*. 22 (5): 1350027. Bibcode:2013IJMPE..2250027T. doi:10.1142/S0218301313500274. 物质守恒指的是重子数A和轻子数L的守恒,A和L是代数数。正的A和L与物质粒子相关,负的A和L与反物质粒子相关。所有已知的相互作用都保持物质守恒。
\item Raith, W.; Mulvey, T. (2001). *Constituents of Matter: Atoms, Molecules, Nuclei and Particles*. CRC Press. 第777–781页. ISBN 978-0-8493-1202-1.
\item "Antimatter". Lawrence Berkeley National Laboratory. 从原始版本存档于2008年8月23日。检索于2008年9月3日。
\item Peskin, M. E.; Schroeder, D. V. (1995). *An Introduction to Quantum Field Theory* 第三部分. Addison–Wesley. ISBN 978-0-201-50397-5.
\item Munowitz, M. (2005). *Knowing*. Oxford University Press. 第35页. ISBN 0195167376.
\item Schumm, B. A. (2004). *Deep Down Things*. Johns Hopkins University Press. 第131–132页. ISBN 978-0-8018-7971-5.
\item Close, F. E. (1988). "Gluonic Hadrons". *Reports on Progress in Physics*. 51 (6): 833–882. Bibcode:1988RPPh...51..833C. doi:10.1088/0034-4885/51/6/002. S2CID 250819208.
\item Kofoed, Melissa; Miller, Shawn (July 2024). *Introductory Chemistry*.
\item §1.8, *Constituents of Matter: Atoms, Molecules, Nuclei and Particles*, Ludwig Bergmann, Clemens Schaefer, and Wilhelm Raith, Berlin, Germany: Walter de Gruyter, 1997, ISBN 3-11-013990-1.
\item Fleming, D. G.; Arseneau, D. J.; Sukhorukov, O.; Brewer, J. H.; Mielke, S. L.; Schatz, G. C.; Garrett, B. C.; Peterson, K. A.; Truhlar, D. G. (28 January 2011). "Kinetic Isotope Effects for the Reactions of Muonic Helium and Muonium with H2". *Science*. 331 (6016): 448–450. Bibcode:2011Sci...331..448F. doi:10.1126/science.1199421. PMID 21273484. S2CID 206530683.
\item Sokal, A. (22 July 1996). "Don't Pull the String Yet on Superstring Theory". *The New York Times*. Archived from the original on 7 December 2008. Retrieved 26 March 2010.
\item Harrison, M.; Ludlam, T.; Ozaki, S. (March 2003). "RHIC project overview". *Nuclear Instruments and Methods in Physics Research Section A: Accelerators, Spectrometers, Detectors and Associated Equipment*. 499 (2–3): 235–244. Bibcode:2003NIMPA.499..235H. doi:10.1016/S0168-9002(02)01937-X. Archived from the original on 15 April 2021. Retrieved 16 September 2019.
\item Courant, Ernest D. (December 2003). "Accelerators, Colliders, and Snakes". *Annual Review of Nuclear and Particle Science*. 53 (1): 1–37. Bibcode:2003ARNPS..53....1C. doi:10.1146/annurev.nucl.53.041002.110450. ISSN 0163-8998.
\item "index". Vepp2k.inp.nsk.su. Archived from the original on 29 October 2012. Retrieved 21 July 2012.
\item "The VEPP-4 accelerating-storage complex". V4.inp.nsk.su. Archived from the original on 16 July 2011. Retrieved 21 July 2012.
\item "VEPP-2M collider complex" (in Russian). Inp.nsk.su. Archived from the original on 3 December 2013. Retrieved 21 July 2012.
\item "The Budker Institute of Nuclear Physics". English Russia. 21 January 2012. Archived from the original on 28 June 2012. Retrieved 23 June 2012.
\item "Welcome to". Info.cern.ch. Archived from the original on 5 January 2010. Retrieved 23 June 2012.
\item "Germany's largest accelerator centre". Deutsches Elektronen-Synchrotron DESY. Archived from the original on 26 June 2012. Retrieved 23 June 2012.
\item "Fermilab | Home". Fnal.gov. Archived from the original on 5 November 2009. Retrieved 23 June 2012.
\item "IHEP | Home". ihep.ac.cn. Archived from the original on 1 February 2016. Retrieved 29 November 2015.
\item "Kek | High Energy Accelerator Research Organization". Legacy.kek.jp. Archived from the original on 21 June 2012. Retrieved 23 June 2012.
\item "SLAC National Accelerator Laboratory Home Page". Archived from the original on 5 February 2015. Retrieved 19 February 2015.
\item Gagnon, Pauline (14 March 2014). "Standard Model: a beautiful but flawed theory". Quantum Diaries. Retrieved 7 September 2023.
\item "The Standard Model". CERN. Retrieved 7 September 2023.
\item Corbion, Ashley (22 March 2011). "The vanishing dimensions of the Universe". Astra Materia. Retrieved 21 May 2013.
\item Wolchover, Natalie (22 December 2017). "The Best Explanation for Everything in the Universe". The Atlantic. Archived from the original on 15 November 2020. Retrieved 11 March 2022.
\item "Fermilab | Science at Fermilab | Benefits to Society". Fnal.gov. Archived from the original on 9 June 2012. Retrieved 23 June 2012.
\item "Muon Colliders Hold a Key to Unraveling New Physics". www.aps.org. Retrieved 17 September 2023.
\end{enumerate}