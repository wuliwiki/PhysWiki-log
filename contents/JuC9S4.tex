% Julia 数组的基本要素
% 数组 基本要素

本文授权转载自郝林的 《Julia 编程基础》。 原文链接:\href{https://github.com/hyper0x/JuliaBasics/blob/master/book/ch09.md}{第 9 章 容器:数组(上)}。


\subsubsection{9.4 数组的基本要素}

当我们拿到一个数组,首先应该去了解它的元素类型、维数和尺寸。在 Julia 中,这些信息都由专门的函数提供。函数\verb|eltype|可以获取到一个数组的元素类型,函数\verb|ndims|用于获取一个数组的维数。\verb|length|函数用于获得一个数组的元素总数量。而若要想获得数组在各个维度上的长度,我们就需要使用\verb|size|函数。

\verb|size|函数有一个必选的参数\verb|A|,代表目标数组。它还有一个可选的参数\verb|dim|,代表维度的序号。在调用\verb|size|函数的时候,如果我们只为\verb|A|指定了参数值,那么该函数就会返回一个元组。这个元组会依次地包含该数组在各个维度上的长度。但倘若我们同时给定了\verb|dim|的值,那么它就只会返回对应的那个长度了。例如:

\begin{lstlisting}[language=julia]
julia> array2d = [[1,2,3,4,5] [6,7,8,9,10] [11,12,13,14,15]
 [16,17,18,19,20] [21,22,23,24,25] [26,27,28,29,30]]
5×6 Array{Int64,2}:
 1   6  11  16  21  26
 2   7  12  17  22  27
 3   8  13  18  23  28
 4   9  14  19  24  29
 5  10  15  20  25  30

julia> size(array2d)
(5, 6)

julia> size(array2d, 2)
6

julia> eltype(array2d), ndims(array2d), length(array2d)
(Int64, 2, 30)

julia> 
\end{lstlisting}

我使用数组值的一般表示法创建了一个 5 行 6 列的数组\verb|array2d|。这个数组拥有两个维度,其元素类型是\verb|Int64|。之所以表达式\verb|size(array2d)|的求值结果为\verb|(5, 6)|,是因为该数组在第一个维度和第二个维度上的长度分别是\verb|5|和\verb|6|。实际上,我们用\verb|5|乘以\verb|6|就可以得到这个二维数组的元素总数量\verb|30|。