% 氢原子 FEDVR TDSE 笔记

\subsection{关于 FEDVR}
Gauss-Lobatto 积分中令采样点(包括两个端点)的个数为 $N$, 如果被积函数是 $2N-3$ 阶多项式($f(x) = x^{2n-3} + \dots$), 则积分没有误差。 每一个 FEDVR 基底都是 $N-1$ 阶多项式, 所以两个基底乘积的积分是精确的。

为了将基底归一化, 非边界点对应的基底要除以对应的根号权重, bridge function 积分时可以分成两段, 所以是除以“两个权重之和开根号”。

把第 $i$ 个有限元的第 $j$ 个归一化的基底记为 $u_{ij}(x)$, 坐标记为 $x_{ij}$ 权重记为 $w_{ij}$, $[-1,1]$ 区间的归一化基底记为 $f_j(t)$, 权重记为 $w_{0j}$。 令 $a_i$ 是第 $i$ 个 FE 长度的一半, $b_i$ 是第 $i$ 个 FE 的终点坐标。 注意 Gauss-Lobatto 积分是对称的
\begin{equation}
t_j = -t_{N-j+1} \qquad w_{0j} = w_{0,N-j+1}
\end{equation}
$x$ 和 $t$ 间有线性关系
\begin{equation}
x_{ij} = a_i t_j + b_i
\end{equation}
\begin{equation}
x = a_i t + b_i \qquad (x_{i1} \les x \les x_{iN})
\end{equation}
$f_j$ 基底的最大值为
\begin{equation}
f_j(t_j) = \frac{1}{\sqrt{w_{0j}}}
\end{equation}
归一化的 $u_{ij}$ 基底和 $f_j$ 基底间的关系为
\begin{equation}
u_{ij}(x) = \leftgroup{
&\frac{1}{\sqrt{a_i}} f_j\qty(\frac{x-b_i}{a_i})  \qquad &&( 1 < j < N) \\
& \frac{1}{\sqrt{a_i+a_{i+1}}}\times\leftgroup{
& f_N\qty(\frac{x-b_i}{a_i}) \quad &&(x \les  x_{iN})\\
& f_1\qty(\frac{x-b_{i+1}}{a_{i+1}}) \quad &&(x \ges x_{iN})
}  \quad &&(j = N) 
} \end{equation}
所以 $u_{ij}$ 基底的最大值为
\begin{equation}
u_{ij}(x_{ij}) = \leftgroup{
&\frac{1}{\sqrt{a_i w_{0j}}} \qquad  &&( 1 < j < N) \\
& \frac{1}{\sqrt{(a_i+a_{i+1}) w_{0N}}} \qquad &&(j = N) 
}\end{equation}
对应的权重为
\begin{equation}
w_{ij} = \leftgroup{
&a_i w_{0j} \qquad &&( 1 < j < N) \\
&(a_i + a_{i+1}) w_{0N} \qquad &&( j = N)\\
} \end{equation}
这个权重可用于对整个 FEDVR 区间的积分。

\subsection{关于 Kinetic Energy Matrix}

假设所有的 FEDVR 基底都是正交归一化的, 就可以开始计算动能矩阵的矩阵元了。 先来看一维的情况 , 假设边界条件为零, 积分的上下限为整个 FEDVR 的范围, 使用分部积分得
\begin{equation}\ali{
K_{ij} &= -\frac12\mel{u_i}{\dv[2]{x}}{u_j}% 未完成: 线性代数需要说一下这个怎么来的, 必须要用正交归一基底
= -\frac12 \int_{x_1}^{x_2} u_i(x) u_j''(x) \dd{x}\\
&= -\frac12 \eval{u_i(x) u'_j(x)}_{x_1}^{x_2} + \frac12 \int_{x_1}^{x_2} u'_i(x) u_j'(x) \dd{x}\\
&= \frac12 \int_{x_1}^{x_2} u'_i(x) u_j'(x) \dd{x}
}\end{equation}
可以看出 $\mat K$ 是实对称矩阵。 这个积分可以精确地用求和表示, 而且只有同一个 FE 里面的 $u_i(x), u_j(x)$ 才能使积分不为零(bridge function 属于两个 FE), 所以就得到了几乎是 block diagonallized 的矩阵, 只是每一个 block 左上角和右下角的一个矩阵元与其他 block 重叠。 这些矩阵元就是 bridge function 对应的。

那如何计算基底函数的导数呢? 可以通过计算勒让德插值多项式以及它们的导数得到
\begin{equation}\ali{
&f_i'(t_j) =\frac{1}{\sqrt{w_{0i}}} \times\\
&\leftgroup{
&\frac{t_j-t_1}{t_i-t_1} \frac{t_j-t_2}{t_i-t_2} \dots \frac{1}{t_i-t_j} \dots \frac{t_j-t_{i-1}}{t_i-t_{i-1}}\frac{t_j-t_{i+1}}{t_i-t_{i+1}} \dots \frac{t_j-t_N}{t_i-t_N} \qquad &&(i \ne j) \\
& \frac{1}{t_i-t_1} + \dots + \frac{1}{t_i-t_{i-1}} + \frac{1}{t_i-t_{i+1}} + \dots \frac{1}{t_i-t_N} \quad &&(i = j)
}} \end{equation}
代入得
\begin{equation}
u'_{ij}(x_{ij'}) = \frac{1}{a_i^{3/2}} f'_j\qty(t_{j'})  \qquad ( 1 < j < N)
\end{equation}
\begin{equation}
u'_{iN}(x_{i j'}) = \frac{1}{a_i\sqrt{a_i+a_{i+1}}} f'_N\qty(t_{j'})
\end{equation}
\begin{equation}
u'_{iN}(x_{i+1, j'}) = \frac{1}{a_{i+1}\sqrt{a_i+a_{i+1}}} f'_1\qty(t_{j'})
\end{equation}
注意 $u'_{iN}$ 在 $x_{iN}$ 处的左右导数不相等。

现在就可以求动能矩阵的矩阵元了(由于是对称矩阵, 我们只列出半边), 先看不含 bridge function 的矩阵元
\begin{equation} \ali{
K_{(im), (in)} &=  \frac12 \int_{x_{i1}}^{x_{iN}} u'_{im}(x) u'_{in}(x) \dd{x} \\
&= \frac12 \sum_k  a_i w_{0k} u'_{im}(x_{ik}) u'_{in}(x_{ik}) \\
&= \frac{1}{a_i^2} \sum_k w_{0k} f'_m(t_k) f'_n(t_k)
\qquad (1 < m \les n < N)
} \end{equation}
可见动能矩阵的每一个 sub-block 的内部都是成正比的。

再看含 bridge function 的矩阵元
\begin{equation} \ali{
K_{(im), (iN)} &= \frac12 \int_{x_{i1}}^{x_{iN}} u'_{im}(x) u'_{iN}(x) \dd{x} \\
&= \frac{1}{2a_i^{3/2} \sqrt{a_i + a_{i+1}}} \sum_k w_{0k} f'_m(t_k) f'_N(t_k)
\qquad (1< m < N)
} \end{equation}

\begin{equation} \ali{
K_{(iN), (iN)} &= \frac12 \int_{x_{i1}}^{x_{iN}} u'_{iN}(x)^2 \dd{x} + \frac12 \int_{x_{i+1,1}}^{x_{i+1,N}} u'_{i+1,1}(x)^2 \dd{x}\\
&= \frac{1}{2(a_i + a_{i+1})} \sum_k w_{0k} \qty[\frac{1}{a_i} f'_N(t_k)^2 + \frac{1}{a_{i+1}} f'_1(t_k)^2]
} \end{equation}

\begin{equation} \ali{
K_{(iN), (i+1,n)} &= \frac12 \int_{x_{i+1,1}}^{x_{i+1,N}} u'_{i,N}(x) u'_{i+1,n}(x) \dd{x}\\
&= \frac{1}{2 a_{i+1}^{3/2} \sqrt{a_i + a_{i+1}}} \sum_k w_{0k} f'_1(t_k) f'_n(t_k)
\qquad (1< n < N)
} \end{equation}

\begin{equation} \ali{
K_{(iN), (i+1,N)} &= \frac12 \int_{x_{i+1,1}}^{x_{i+1,N}} u'_{i,N}(x) u'_{i+1,N}(x) \dd{x}\\
&= \frac{1}{2a_{i+1}\sqrt{(a_i + a_{i+1})(a_{i+1}+a_{i+2})}} \sum_k w_{0k} f'_1(t_k) f'_N(t_k)
} \end{equation}
