% 薛定谔方程的分离变量法
% keys 量子力学|薛定谔方程|分离变量|哈密顿算符
% license Xiao
% type Tutor

\begin{issues}
\issueDraft
\end{issues}


\pentry{张量积空间\nref{nod_DirPro}}{nod_7211}
如果定态薛定谔方程
\begin{equation}\label{eq_SEsep_2}
H \ket{\Psi} = E \ket{\Psi}~
\end{equation}
中的状态 $\ket{\Psi}$ 可以看作是两个小空间的张量积空间\footnote{也可以是 $N > 2$ 个小空间的张量积空间, 以下结论类比可得}, 且总哈密顿 $H$ 可以分解为
\begin{equation}\label{eq_SEsep_1}
H = H_1 \otimes I + I \otimes H_2~
\end{equation}
的形式。 那么我们先分别解出两个子空间的定态薛定谔方程
\begin{equation}
\begin{cases}
H_1 \ket{\Psi_{1,i}} = E_{1,i} \ket{\Psi_{1,i}}\\
H_2 \ket{\Psi_{2,j}} = E_{2,j} \ket{\Psi_{2,j}}~,
\end{cases}
\end{equation}
这两组解分别构成了两个小空间的一组完备的正交归一基底。 我们也可以证明 $\ket{\Psi_{1,i}} \otimes \ket{\Psi_{2,j}}$ 满足张量积空间中的薛定谔方程(\autoref{eq_SEsep_2})。 代入得
\begin{equation}
\begin{aligned}
(H_1 \otimes I + I \otimes H_2) \ket{\Psi_{1,i}} \otimes \ket{\Psi_{2,j}} &= (H_1 \ket{\Psi_{1,i}}) \otimes \ket{\Psi_{2,j}} +  \ket{\Psi_{1,i}} \otimes (H_2 \ket{\Psi_{2,j}})\\
&= (E_{1,i} + E_{2, j}) \ket{\Psi_{1,i}} \otimes \ket{\Psi_{2,j}}~.
\end{aligned}
\end{equation}
证毕。

事实上, 这就是偏微分方程的分离变量法。 两个小空间既可以是同一个粒子在两个不同方向上的状态空间, 例如二维无限深势阱\upref{ISW2D}), 也可以是两个不同粒子的状态空间例如双粒子无限深势阱。%\addTODO{链接}
