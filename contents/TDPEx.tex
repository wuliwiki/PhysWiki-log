% 几种含时微扰
% keys 含时微扰|微扰理论|量子力学|脉冲
% license Xiao
% type Tutor

\pentry{含时微扰理论\upref{TDPT}}

令初态到末态能量差对应的光子频率为
\begin{equation}
\omega_{fi} = \frac{E_f - E_i}{\hbar}~,
\end{equation}

\begin{equation}
S = \frac{1}{\I\hbar} \int_{-\infty}^{+\infty} \mel{f}{\Q H'(t)}{i} \E^{\I\omega_{fi}} \dd{t}~.
\end{equation}
当 $\mel{f}{H'(t)}{i} = W_{fi} g(t)$ 时
\begin{equation}
P_{fi} = \abs{S}^2 = \frac{\abs{W_{fi}}^2}{\hbar^2} \abs{\int_{-\infty}^{+\infty} g(t) \E^{\I\omega_{fi} t} \dd{t}}^2~.
\end{equation}

\subsection{瞬时脉冲}
令 $g(t) = \delta(t-t_0)$, 则
\begin{equation}
\abs{\int_{-\infty}^{+\infty} g(t) \E^{\I\omega_{fi} t} \dd{t}}^2
= \abs{\int_{t_0-\epsilon}^{t_0+\epsilon} \delta(t-t_0) \E^{\I\omega_{fi} t} \dd{t}}^2
= 1~,
\end{equation}
代入得
\begin{equation}
P_{fi} = \frac{\abs{W_{fi}}^2}{\hbar^2}~.
\end{equation}

\subsection{方形脉冲}
\begin{equation}
g(t) = \leftgroup{
&1 \qquad (0 < t < \Delta t)\\
&0 \qquad (\text{其他})~,
}\end{equation}

\begin{equation}\begin{aligned}
\abs{\int_{-\infty}^{+\infty} g(t) \E^{\I\omega_{fi}t} \dd{t}}^2
&= \abs{\int_{t_1}^{t_2} \E^{\I\omega_{fi}t} \dd{t}}^2
= \abs{\frac{\E^{\I\omega_{fi}t_2} - \E^{\I\omega_{fi}t_1}}{\I\omega_{fi}}}^2\\
&= \frac{\sin^2[\omega_{fi}(t_2-t_1)/2]}{[\omega_{fi}(t_2-t_1)/2]^2} (t_2-t_1)^2 \\
&= \Delta t^2 \sinc^2[\omega_{fi}\Delta t/2]~,
\end{aligned}\end{equation}
概率为
\begin{equation}\label{eq_TDPEx_2}
P_{fi} = \frac{\abs{W_{fi}}^2}{\hbar^2} \Delta t^2 \sinc^2[\omega_{fi}\Delta t/2]~.
\end{equation}
于瞬时脉冲相比,主要跃迁到附近的 $E_2$ 能级。且时间越长能量变化越小。

注意当 $\Delta t \to \infty$ 时, $\Delta t\sinc^2(\Delta t\cdot x) \to \pi\delta(x)$ (\autoref{ex_Delta_2}~\upref{Delta})。 所以当脉冲较长时, 末态能量约等于初态能量。

\subsection{简谐振动乘以方形脉冲}
\begin{equation}
g(t) = \leftgroup{
&\E^{\pm\I\omega t} &&(0 < t < \Delta t)\\
&0 &&(\text{其他})~,
}\end{equation}
与上面的推导类似,结果为
\begin{equation}\label{eq_TDPEx_4}
c_i(t) = -\frac{W_{fi}}{\hbar} \frac{\E^{\I(\omega_{ij} \pm \omega)t} - 1}{\omega_{ij} \pm \omega}~,
\end{equation}

\begin{equation}
P_{fi} = \abs{c_i(t)}^2 = \frac{\abs{W_{fi}}^2}{\hbar^2} \Delta t^2 \sinc^2[(\omega_{fi}\pm\omega)\Delta t/2]~.
\end{equation}
这说明,跃迁倾向于增加能量 $\hbar\omega$,时间越长,就越靠近 $\hbar\omega$。要注意真实的简谐微扰往往是 $\cos(\omega t)$, 分解为两项积分后,会有干涉效应,结果较为复杂。但若 $\omega \gg \omega_{fi}$ 时可以忽略干涉项。

\subsection{简谐振动乘以方形脉冲}
\begin{equation}
g(t) = \leftgroup{
&\cos(\omega t) &&(0 < t < \Delta t)\\
&0 &&(\text{其他})~.
}\end{equation}
由于积分关于 $g(t)$ 是线性的, 可以将 $\cos(\omega t)$ 拆分为两个指数函数
\begin{equation}
c_i(t) = -\frac{W_{fi}}{2\hbar} \qty[\frac{\E^{\I(\omega_{ij} + \omega)t} - 1}{\omega_{ij} + \omega} + \frac{\E^{\I(\omega_{ij} - \omega)t} - 1}{\omega_{ij} - \omega}]~.
\end{equation}
作为一个近似, 如果 $\abs{\omega - \omega_{ij}} \ll \omega$ 那么第一项可以忽略不计(\autoref{eq_TDPEx_4} 取负号), 所以
\begin{equation}\label{eq_TDPEx_1}
P_{fi} = \abs{c_i(t)}^2 = \frac{\abs{W_{fi}}^2}{4\hbar^2} \Delta t^2 \sinc^2[(\omega_{fi}-\omega)\Delta t/2]~.
\end{equation}
当 $\omega = \omega_{fi}$ 时, 跃迁概率和时间平方成正比。

相反, 如果 $\abs{\omega + \omega_{ij}} \ll \omega$, 那么第二项忽略不计(\autoref{eq_TDPEx_4} 取正号), 所以
\begin{equation}\label{eq_TDPEx_3}
P_{fi} = \abs{c_i(t)}^2 = \frac{\abs{W_{fi}}^2}{4\hbar^2} \Delta t^2 \sinc^2[(\omega_{fi}+\omega)\Delta t/2]~.
\end{equation}
当\autoref{eq_TDPEx_1} 和\autoref{eq_TDPEx_3} 的两个峰都比较窄且不重叠, 那么 $P_{fi}$ 也可以写成它们之和。
