% 复合命题(高中)
% keys 命题|高中|且|或|非|条件|量词|逆否|复合
% license Usr
% type Tutor

\pentry{命题\nref{nod_HsLogi}}{nod_acc5}

由原子命题通过逻辑连接词组合、量词限定变量和条件连接等方式形成的命题,称为\textbf{复合命题}(compound proposition)。

\subsection{逻辑连接词}

逻辑连接词分为:且、或、非。

\subsubsection{且}

\textbf{且}(and,也称为\textbf{同}、\textbf{合取})记号为$A\land B$。

\subsubsection{或}

\textbf{或}(or,也称为\textbf{或者}、\textbf{析取})记号为$A\lor B$。

\subsubsection{非}

\textbf{非}(not,也称为\textbf{否定})记号为$\lnot A$。

\subsection{量词命题}

描述关于某些变量的通用性质或存在性质的命题,称为\textbf{量词命题}(quantified propositions)。原本开放命题中真值根据变量取值确定,量词命题使用\textbf{量词}(quantifiers)来限定开放命题中的变量,使得量词命题具有明确的真值。以“$x$是一个偶数”为例,如果用“所有”来限定$x$,即“所有的自然数$x$都是偶数”,是假命题,而用“存在”来限定$x$,即“存在一个自然数$x$是偶数”,则是真命题。

注意,如“任意$m>n$”等单独用量词来限定变量的语句是没有意义的,量词需要在命题中使用。

\subsubsection{全称量词}

在陈述中表达所述事物的全体的含义时,使用的量词称为\textbf{全称量词}(universal quantifier),记作$\forall$,可以读作“任意”、“所有”、“每一个”等。形如“集合$M$中所有的元素$x$,都满足性质$P(x)$”的命题称为\textbf{全称量词命题}或称其\textbf{恒成立}(universal proposition),记作:


\begin{equation}
\forall x\in M,P(x).~
\end{equation}

\subsubsection{存在量词}

在陈述中表达所述事物的个体、部分或特例的含义时,使用的量词称为\textbf{存在量词}(existential quantifier,也称作\textbf{特称量词},Particular Quantifier),记作$\exists$,可以读作“存在”、“有”、“至少有一个”等。形如“集合$M$中存在某个元素$x$满足性质$P(x)$”的命题称为\textbf{存在量词命题},记作:

\begin{equation}
\exists x\in M,P(x).~
\end{equation}

\subsection{条件命题}

\textbf{条件命题}(conditional proposition或\textbf{蕴含命题},implication)用于表达两个命题之间的条件关系,通常形如“若 $p$ ,则 $q$ ” ,由一个前件(antecedent)$p$和一个后件(consequent)$q$组成,通过“如果……那么……”(if…then…)的形式来连接。条件命题在计算机领域广泛使用。

\subsubsection{充分与必要}

“若 $p$ ,则 $q$ ”表示如果命题 $p$ 为真,那么命题 $q$ 也为真。条件命题通常用符号$p\Rightarrow q$表示。

如果这个条件命题成立,则$p$称为$q$的充分条件,反过来看则$q$称为$p$的必要条件。
当且仅当


\subsubsection{逆与否}

逆命题、否命题、逆否命题

区分命题的否定和否命题。
