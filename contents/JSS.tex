% 计算机科学(综述)
% license CCBYSA3
% type Wiki

本文根据 CC-BY-SA 协议转载翻译自维基百科\href{https://en.wikipedia.org/wiki/Computer_science}{相关文章}。

计算机科学是关于计算、信息和自动化的研究。[1][2][3] 计算机科学涵盖了从理论学科(如算法、计算理论和信息理论)到应用学科(包括硬件和软件的设计与实现)等多个领域。[4][5][6]

算法和数据结构是计算机科学的核心。[7] 计算理论研究的是计算的抽象模型和可以通过这些模型解决的普遍问题。密码学和计算机安全领域涉及研究安全通信的手段以及防止安全漏洞的技术。计算机图形学和计算几何学关注图像的生成。编程语言理论探讨了描述计算过程的不同方式,数据库理论则关注数据存储库的管理。人机交互研究人类和计算机交互的接口,软件工程则专注于开发软件的设计和原则。操作系统、网络和嵌入式系统等领域研究复杂系统的原理和设计。计算机架构描述了计算机组件和计算机操作设备的构建方式。人工智能和机器学习旨在合成目标导向的过程,如人类和动物中出现的问题解决、决策制定、环境适应、规划和学习过程。在人工智能领域,计算机视觉旨在理解和处理图像和视频数据,而自然语言处理则专注于理解和处理文本和语言数据。

计算机科学的根本问题是确定什么可以被自动化,什么不能。[2][8][3][9][10] 图灵奖通常被认为是计算机科学领域的最高荣誉。[11][12]
\subsection{历史}
\begin{figure}[ht]
\centering
\includegraphics[width=6cm]{./figures/50908b7fe6abac5b.png}
\caption{戈特弗里德·威廉·莱布尼茨(Gottfried Wilhelm Leibniz,1646-1716)发展了二进制数系统的逻辑,并被称为“计算机科学的创始人。[13]} \label{fig_JSS_1}
\end{figure}
计算机科学的最早基础早于现代数字计算机的发明。像算盘这样的计算固定数值任务的机器,自古以来就存在,用于帮助进行乘法和除法等计算。执行计算的算法也自古就有,甚至在复杂计算设备的发展之前就已存在。[16]

威廉·席卡德(Wilhelm Schickard)于1623年设计并制造了第一台工作的机械计算器。[17] 1673年,戈特弗里德·莱布尼茨(Gottfried Leibniz)展示了一种数字机械计算器,称为“阶梯计算机”(Stepped Reckoner)。[18] 莱布尼茨可能被认为是第一位计算机科学家和信息理论家,原因有很多,其中之一是他记录了二进制数系统的使用。1820年,托马斯·德·科尔马(Thomas de Colmar)发明了简化版的算术仪器——第一台足够强大且可靠的计算机器,能够在日常办公环境中使用,从而开启了机械计算器行业。[note 1] 查尔斯·巴贝奇(Charles Babbage)于1822年开始设计第一台自动机械计算器——差分机(Difference Engine),这最终启发了他设计第一台可编程的机械计算机——分析机(Analytical Engine)。[19] 他于1834年开始开发这台机器,并且“不到两年,他就勾画出了现代计算机的许多显著特征”。[20] “一个关键步骤是采用了源自贾卡尔织布机的打孔卡片系统”[20],使得这台机器具有了无限的可编程性。[note 2] 1843年,在翻译关于分析机的法国文章时,阿达·洛夫莱斯(Ada Lovelace)在她的多篇附注中写下了一个计算伯努利数的算法,这被认为是历史上首个专门为计算机实施的已发表算法。[21] 大约在1885年,赫尔曼·霍勒里斯(Herman Hollerith)发明了制表机,这种机器使用打孔卡片来处理统计信息;最终,他的公司成为了IBM的一部分。在巴贝奇之后,虽然未曾了解他之前的工作,珀西·卢德盖特(Percy Ludgate)于1909年发表了[22]历史上仅有的两个机械分析机设计中的第二个。1914年,西班牙工程师莱昂纳多·托雷斯·奎韦多(Leonardo Torres Quevedo)发表了《自动化论文》[23],并设计了一台理论上的电机计算机,灵感来自巴贝奇,计划通过只读程序来控制这台机器。该论文还介绍了浮点算术的概念。[24][25] 1920年,为庆祝算术仪器发明100周年,托雷斯在巴黎展示了电机算术仪器原型,这台原型演示了电机分析机的可行性,[26] 可通过键入命令并自动打印结果。[27] 1937年,在巴贝奇的梦想百年后,霍华德·艾肯(Howard Aiken)说服了IBM,IBM当时生产各种打孔卡片设备并涉足计算器业务[28],开发了他的大型可编程计算机——ASCC/哈佛马克I,基于巴贝奇的分析机,后者本身就使用卡片和中央计算单元。当机器完成后,一些人将其誉为“巴贝奇的梦想成真”。[29]
\begin{figure}[ht]
\centering
\includegraphics[width=6cm]{./figures/c26717478c951a6d.png}
\caption{查尔斯·巴贝奇有时被称为“计算机之父。[14]} \label{fig_JSS_2}
\end{figure}
在1940年代,随着像阿塔纳索夫–贝里计算机(Atanasoff–Berry Computer)和ENIAC等新型更强大的计算机器的出现,“计算机”一词开始指代这些机器,而非其人类前身。[30] 随着计算机可以用于的不仅仅是数学计算,计算机科学的领域扩展到研究一般的计算问题。1945年,IBM在纽约市哥伦比亚大学成立了沃森科学计算实验室。位于曼哈顿西区的一座翻修过的兄弟会大厦成为了IBM第一个专注于纯科学的实验室。该实验室是IBM研究部的前身,今天,IBM在全球各地都设有研究设施。[31] 最终,IBM与哥伦比亚大学之间的紧密关系对于新科学学科的诞生起到了关键作用,哥伦比亚大学于1946年开设了第一门计算机科学学分课程。[32] 计算机科学开始在1950年代和1960年代初期作为一个独立的学术学科逐渐确立。[33][34] 世界上第一个计算机科学学位课程——剑桥大学计算机科学文凭课程,于1953年在剑桥大学计算机实验室开设。美国的第一个计算机科学系于1962年在普渡大学成立。[35] 自从实用计算机问世以来,计算的许多应用已成为独立的研究领域。

另见:计算历史和信息学历史