% 电磁场中的类氢原子
% keys 氢原子|类氢原子|电磁场|规范
% license Xiao
% type Tutor

\begin{issues}
\issueDraft
\end{issues}

\pentry{长度规范和速度规范\nref{nod_LVgaug}}{nod_3d7d}

\footnote{本文使用原子单位}假设原子核不动(无限核质量近似), 使用库仑规范得
\begin{equation}
\varphi = \frac{Z}{r}~,
\end{equation}
其中 $Z < 0$ 是核电荷数。

式?变为
\begin{equation}
H = H_0 + H_I = -\frac{1}{2m} \laplacian +  \frac{Z}{r} - \frac{\I}{m} \bvec A \vdot \Nabla + \frac{1}{2m} \bvec A^2~,
\end{equation}
其中含时哈密顿算符 $H_I$ 是后两项。

以下我们使用偶极近似, 则 $\bvec A$ 只是时间的函数。
