% 微分方程 $y^{(N)}=f(x)$
% 高阶导数|常微分方程|积分

\pentry{高阶导数\upref{HigDer}, 常微分方程\upref{ODE}}
在常微分方程\upref{ODE}中,我们曾提到形如 $y^{(N)}=f(x)$ 的 $N$ 阶微分方程可通过 $N$ 次积分求得通解。为了之后引入记号方法\upref{Sign}方便,本节将具体讨论该方程。

方程
\begin{equation}\label{eq_ynfx_1}
y^{(N)}=f(x)
\end{equation}
的通解具有形式
\begin{equation}\label{eq_ynfx_7}
y=y_1+\sum_{i=0}^{N}C_ix^i~.
\end{equation}
其中, $y_1$ 是\autoref{eq_ynfx_1} 的任意一个特殊解。

特殊地, 若\autoref{eq_ynfx_1} 满足下列零初始条件
\begin{equation}\label{eq_ynfx_4}
y|_{x=x_0}=0,\quad y'|_{x=x_0}=0\quad\cdots \quad y^{(N-1)}|_{x=x_0}=0~.
\end{equation}
则\autoref{eq_ynfx_1} 满足初始条件\autoref{eq_ynfx_4} 的解 $y_1(x)$ 为
\begin{equation}\label{eq_ynfx_5}
y_1(x)=\frac{1}{(N-1)!}\int_{x_0}^x(x-t)^{N-1}f(t)\dd t~.
\end{equation}
方程\autoref{eq_ynfx_1} 的通解则为
\begin{equation}
y(x)=\frac{1}{(N-1)!}\int_{x_0}^x(x-t)^{N-1}f(t)\dd t+\sum_{i=0}^{N-1}C_ix^i~.
\end{equation}
\subsection{证明}
先来求方程\autoref{eq_ynfx_1} 的一般积分公式。设 $y_1(x)$ 是方程\autoref{eq_ynfx_1} 的任何一个解,就是
\begin{equation}\label{eq_ynfx_2}
y_1^{(N)}=f(x)~,
\end{equation}
引入未知函数 $z$
\begin{equation}\label{eq_ynfx_3}
y=y_1(x)+z~.
\end{equation}
代入\autoref{eq_ynfx_1} ,得到一个关于 $z$ 的方程
\begin{equation}
y_1^{(N)}+z^{(N)}=f(x)~.
\end{equation}
根据恒等式\autoref{eq_ynfx_2} 
\begin{equation}
z^{(N)}=0~.
\end{equation}
这意味着函数 $z$ 是具有任何常系数的 $N$ 次多项式
\begin{equation}
z=\sum_{i=0}^{N-1}C_ix^i~,
\end{equation}
于是\autoref{eq_ynfx_3} 给出方程\autoref{eq_ynfx_1} 的一般积分
\begin{equation}
y=y_1+\sum_{i=0}^{N}C_ix^i~.
\end{equation}
即方程\autoref{eq_ynfx_1} 的一般积分是这方程的任何一个特殊解与具有任意常系数的 $N-1$ 次多项式之和。

这样,剩下的任务就是证明方程\autoref{eq_ynfx_1} 满足初始条件\autoref{eq_ynfx_4} 的解为\autoref{eq_ynfx_5}。为此,将 $y(x)$ 写成余项具有积分形式的泰勒公式(引用未完成)
\begin{equation}
y(x)=\sum_{i=0}^{N}\frac{y^{(i)}(x_0)}{i!}\qty(x-x_0)^{i}+\frac{1}{(N-1)!}\int_{x_0}^{x}(x-t)^{N-1}y^{(N)}(t)\dd t~,
\end{equation}
代入初始条件\autoref{eq_ynfx_4} 和\autoref{eq_ynfx_1} 到上式,即得\autoref{eq_ynfx_5} .

正如所说的,方程\autoref{eq_ynfx_1} 的解可通过 $N$ 次积分求得,那么满足初值条件\autoref{eq_ynfx_4} 的解为
\begin{equation}\label{eq_ynfx_6}
y=\int_{x_0}^x\dd x\int_{x_0}^x\dd x\cdots\int_{x_0}^{x}\dd x\int_{x_0}^{x}f(x)\dd x~.
\end{equation}
这样,\autoref{eq_ynfx_5} 给出了 $N$ 次积分\autoref{eq_ynfx_6} 的一次积分形式的表达式。
