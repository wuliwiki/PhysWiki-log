% 幂级数与解析函数
% keys 泰勒级数|幂级数|收敛域|收敛半径|解析函数
% license Xiao
% type Tutor

\pentry{ 泰勒公式\upref{Tayl}}{nod_e891}

按照泰勒公式, 一个在定义域内无穷次可微的实函数在任何一点都可以用它的泰勒级数的部分和进行逼近:
$$
\begin{aligned}
f(x)=&f(x_0)+\frac{f'(x_0)}{1!}(x-x_0)+\frac{f''(x_0)}{2!}(x-x_0)^2+\cdots\\
&+\frac{f^{(n)}(x_0)}{n!}(x-x_0)^n+o((x-x_0)^n)~.
\end{aligned}
$$
然而, 如果执意要将右端的有限和扩展为无穷级数, 那么立刻就会出现两个问题: 这个级数收敛吗? 如果收敛, 它能够收敛到左端吗?

一般来讲, 答案都是否定的。 

\begin{theorem}{博雷尔 (Borel) 定理}
设 $\{a_n\}$ 是任意复数序列。 则存在一个光滑函数 $f:(-1,1)\to\mathbb{C}$, 使得 $f^{(n)}(0)=a_n$.
\end{theorem}
博雷尔定理告诉我们: 任何复数序列都能够成为某个光滑函数在某一点处的导数值序列。 由此构成的泰勒级数
$$
\sum_{n=0}^\infty\frac{a_n}{n!}x^n~,
$$
当然可能发散, 比如取 $a_n=(n!)^2$~.

即便泰勒级数是收敛的, 它也不一定能够收敛到被展开的函数。 一个典型的例子是
$$
f(x)=\exp\left(-\frac{1}{x^2}\right)~,
$$
这里补充定义 $f(0)=0$. 直接计算可以看出 $f^{(n)}(0)=0$, 所以它的泰勒级数恒为零。 因此, $f(x)$ 的泰勒级数部分和同 $f(x)$ 自身的偏差永远是 $f(x)$ 本身。

\begin{exercise}{}
用归纳法证明: 对于正整数 $n$, 有一个多项式 $P_n$ 使得
$$
f^{(n)}(x)=\exp\left(-\frac{1}{x^2}\right)P_n\left(\frac{1}{x}\right)~.
$$
例如, 
$$
f'(x)=\frac{2}{x^3}\exp\left(-\frac{1}{x^2}\right)~.
$$
由此证明 $f^{(n)}(0)=0$.
\end{exercise}

因此, 泰勒级数收敛到自身的函数其实是十分特殊的。 这样的函数被称为实解析函数 (real analytic function). 它与复解析函数的联系十分紧密。

\subsection{幂级数}
在复数域上, 形如
\begin{equation}
\sum_{n=0}^\infty c_n(z-a)^n~.
\end{equation}
的级数称为\textbf{幂级数(power series)}, 这里 $c_n$ 皆为复数, 未定元 $z$ 一般也视为复数。 

\begin{theorem}{幂级数的收敛域}
如果幂级数在某点 $z_0\neq a$ 处收敛, 那么它一定在开圆盘 $|z-a|<|z_0-a|$ 上绝对收敛且内闭一致收敛。
\end{theorem}

证明很简单: 如果 $\sum_{n=0}^\infty c_n(z-a)^n$ 在 $z=z_0$ 时收敛, 那么 $c_n(z_0-a)^n\to0$, 从而有一 $M$ 使得 $|c_n(z_0-a)^n|\leq M$ 对任何 $n$ 都成立。 故对于任何固定的 $0<q<1$, 当 $|z-a|<q|z_0-a|$ 时就有
$$
\begin{aligned}
\sum_{n=0}^\infty |c_n(z-a)^n|
&=\sum_{n=0}^\infty |c_n(z_0-a)^n|\frac{|z-a|^n}{|z_0-a|^n}\\
&<\sum_{n=0}^\infty Mq^n~.
\end{aligned}
$$
于是幂级数诸项绝对值由收敛的几何级数控制。

由此可见, 幂级数的收敛域或者只是一个点 $a$, 或者至少包含某个以 $a$ 为圆心的开圆盘。 这样的开圆盘中最大者叫做幂级数的\textbf{收敛圆 (disk of convergence)}, 其半径称为\textbf{收敛半径 (radius of convergence)}。 幂级数的收敛半径由柯西-阿达玛公式\upref{CHF}给出。 在收敛圆的边界上, 无法作出幂级数是否收敛的一般论断, 例如幂级数 $\sum_{n=0}^\infty z^n$ 在点 $z=\pm1$ 处皆发散, 但其和函数 $1/(1-z)$ 在 $z=1$ 时为奇异, 在 $z=-1$ 时表现正常。

\subsection{幂级数的运算}
幂级数的四则运算与一般级数的四则运算无异。

\begin{theorem}{幂级数的微分与积分}
幂级数在进行逐项微分与逐项积分后, 其收敛圆不变。 因此, 幂级数的和函数可以在收敛圆内逐项微分, 也可以逐项积分。 幂级数的和函数在收敛圆内是无穷可微的。
\end{theorem}
这是柯西-阿达玛公式\upref{CHF}的直接推论。 

\subsection{解析函数}
由收敛幂级数表示的复变函数称为\textbf{解析函数 (analytic function)}。 它与用复可微性定义的\textbf{全纯函数 (holomorphic function)}\upref{CauRie} 是等价的对象, 尽管这个事实的证明并不平凡 (需要用到柯西积分公式)。 如果限制自变量取实数, 那么得到的是\textbf{实解析函数 (real analytic function)}。 等价地, 实解析函数是泰勒级数收敛到其自身的函数。 有如下定理:

\begin{theorem}{}
开区间 $I$ 上的实函数 $f$ 为解析函数, 当且仅当对于 $I$ 的任何闭子区间 $K$, 都有常数 $M_K$ 使得
$$
\max_{x\in K}|f^{(n)}(x)|\leq n!M_K^n~.
$$
\end{theorem}

证明是直接的计算: 如果 $f$ 满足此条件, 那么可以估算出其泰勒展开式的余项趋于零; 反过来, 如果 $f$ 由收敛幂级数表征, 那么可以估计其逐项微分得到的幂级数的上界而得到 $f$ 的高阶导数所满足的条件。

这个定理说明: 实解析函数的高阶导数随着其阶数的提升不能增长得太快。 举例来说, 开头提到的函数 $f(x)=\E^{-1/x^2}$ 在任何不包含原点的开区间上是解析函数。 在原点处, $f(x)$ 的各阶导数都是零, 但是对于任何正数 $M$ (不管有多么大), 都可以找到趋于零的序列 $x_n$, 使得对于充分大的 $n$ 有
$$
|f^{(n)}(x_n)|>n!M^n~.
$$
因此 $f$ 在任何包含原点的开区间上都不是解析函数。 用复变函数论的语言, $z=0$ 是函数 $\E^{-1/z^2}$ 的本性奇点。
