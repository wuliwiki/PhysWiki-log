% 队列
% keys 队列|数据结构|C++

队列是一种 “先进先出” 的数据结构。队列在左端(队头)弹出元素,在右端(队尾)插入元素。C++ STL 中的 \verb|queue| 实现了队列。队列通常是实现广度优先搜索(BFS)的数据结构。队列还有多种变体,如两端都能插入和弹出元素的\textbf{双端队列}(C++ STL \verb|deuqe|),还有给元素赋予优先级,具有最高优先级的元素最先弹出,等价于一个堆的\textbf{优先队列}(C++ STL \verb|priority_queue|)。

队列通常有四种基本操作:

\begin{enumerate}
\item 向队尾插入一个数 $x$;
\item 从队头弹出一个数;
\item 判断队列是否为空;
\item 查询队头元素。
\end{enumerate}

C++ STL:

\begin{lstlisting}[language=cpp]
queue<int> q;
int hh = 0, tt = -1;  // hh 为队头,tt 为队尾

q.push(x);  // 在队尾插入新元素
q.pop();    // 删除队头元素但不返回其值
q.empty();  // 如果队列为空返回 true,否则返回 false
q.front();  // 返回队头元素的值,但不删除该元素

// 还有如下几个操作
q.back();   // 返回队列尾元素的值,但不删除该元素
q.size();   // 返回队列中元素的个数

\end{lstlisting}

我们这里也是着重讲一下如何用数组实现队列。

定义一个数组 $q$ 用于模拟队列,定义 $\mathtt{hh}$ 为队头,初始化为 $0$,$\mathtt{tt}$ 为队尾,初始化为 $-1$。

\begin{lstlisting}[language=cpp]
string op;
cin >> op;
if (op == "push") 
{
    int x;
    cin >> x;
    q[ ++ tt] = x;
} 
// 如果 hh <= tt 说明队列不为空
else if (op == "empty") cout << (hh <= tt ? "NO" : "YES") << endl;
else if (op == "query") cout << q[hh] << endl;
else if (op == "pop") hh ++ ;
\end{lstlisting}

以上就是队列的基本操作。