% 查尔斯·巴贝奇(综述)
% license CCBYSA3
% type Wiki

本文根据 CC-BY-SA 协议转载翻译自维基百科\href{https://en.wikipedia.org/wiki/Charles_Babbage}{相关文章}。

\begin{figure}[ht]
\centering
\includegraphics[width=6cm]{./figures/2f69cba1b5206a32.png}
\caption{巴贝奇在1860年} \label{fig_CRSBQ_1}
\end{figure}
查尔斯·巴贝奇(Charles Babbage,1791年12月26日-1871年10月18日)是一位英国博学家。[1] 他是一位数学家、哲学家、发明家和机械工程师,巴贝奇提出了数字可编程计算机的概念。[2]

巴贝奇被一些人认为是“计算机之父”。[2][3][4][5] 他被认为发明了第一台机械计算机——差分机,这为更复杂的电子设计奠定了基础,尽管现代计算机的所有基本思想都可以在他的分析机中找到,该机器是通过一个明确借鉴自雅卡尔织机的原则来编程的。[2][6] 除了计算机相关工作外,巴贝奇在他1832年出版的《制造与机械经济学》一书中还涉及了广泛的兴趣领域。[7] 他是伦敦社交圈中的重要人物,并且以其举办的周六晚会而闻名,被认为将“科学晚会”从法国引入英国。[8][9] 他在其他领域的多样工作使他被描述为其世纪中“最杰出”的博学家之一。[1]

巴贝奇虽然未能完成许多设计的成功工程实现,包括他的差分机和分析机,但他在计算机理念的提出上依然是一个重要人物。他未完成的部分机械装置如今被展示在伦敦的科学博物馆。1991年,一台根据原始设计图纸构建的功能性差分机完成了建造。按照19世纪可以实现的公差制造,最终完成的差分机成功运转,证明了巴贝奇的机器本应能够正常工作。


巴贝奇的出生地存在争议,但根据《牛津国家传记词典》,他最有可能出生在英国伦敦沃尔沃思路的44号克罗斯比街。[10] 在拉科姆街与沃尔沃思路交汇处有一块蓝色纪念牌,纪念这一事件。[11]

在《泰晤士报》对巴贝奇的讣告中,他的出生日期为1792年12月26日;但随后一位侄子写信表示,巴贝奇出生在一年之前,即1791年。伦敦纽宁顿圣玛丽教区的注册簿显示巴贝奇于1792年1月6日接受洗礼,支持他出生于1791年的说法。[12][13][14]
\subsection{早年生活}
\begin{figure}[ht]
\centering
\includegraphics[width=6cm]{./figures/c3131f4177c06ff7.png}
\caption{查尔斯·巴贝奇肖像(约1820年)} \label{fig_CRSBQ_9}
\end{figure}
巴贝奇是本杰明·巴贝奇和贝齐·普拉姆利·蒂普的四个孩子之一。他的父亲本杰明·巴贝奇是伦敦弗利特街普雷德银行的创始人之一,该银行与威廉·普雷德合伙,于1801年创办。[15] 1808年,巴贝奇一家搬到了东泰恩茅斯的老罗登斯宅邸。大约在八岁时,巴贝奇因患上威胁生命的高烧,被送到埃克塞特附近的阿尔菲顿乡村学校就读。短时间内,他曾在南德文托特尼斯的国王爱德华六世文法学校就读,但由于健康原因,他不得不转回私人家教。[16]

随后,巴贝奇进入了位于米德尔塞克斯恩菲尔德贝克街的霍尔姆伍德学院,学习数学,该校由史蒂芬·弗里曼牧师主持。[17] 学院里有一个图书馆,这激发了巴贝奇对数学的热爱。离开学院后,巴贝奇又请了两位私人家教。第一位家教是一位来自剑桥附近的牧师;通过这位家教,巴贝奇结识了查尔斯·西门和他的福音派追随者,但家教的内容并不适合他。[18] 他被带回家,回到托特尼斯的学校继续学习,那个时候他大约16或17岁。[19] 第二位家教是一位牛津的导师,在他的指导下,巴贝奇掌握了足够的古典学知识,并被剑桥大学录取。
\subsection{在剑桥大学}
\begin{figure}[ht]
\centering
\includegraphics[width=6cm]{./figures/35eadd5224acad78.png}
\caption{约1850年的巴贝奇} \label{fig_CRSBQ_8}
\end{figure}
Babbage 于 1810 年 10 月到达剑桥大学三一学院。[20] 他在当时的数学某些领域已经是自学成才;[21] 他曾阅读过罗伯特·伍德豪斯、约瑟夫·路易·拉格朗日和玛丽亚·盖塔纳·阿涅西的著作。因此,他对大学提供的标准数学教学感到失望。[10]

1812 年,Babbage、约翰·赫歇尔、乔治·皮科克和几位朋友共同成立了分析学会;他们也与爱德华·赖安关系密切。[22] 作为学生,Babbage 还是其他社团的成员,如“幽灵俱乐部”,该社团致力于调查超自然现象,以及“解救者俱乐部”,该社团旨在将其成员从精神病院中解救出来,如果有人被送入精神病院的话。[23][24]

1812 年,Babbage 转学到剑桥大学的彼得豪斯学院。[20] 他是那里最顶尖的数学家,但未能以优异的成绩毕业。相反,他在 1814 年获得了无需考试的学位。他曾在初步的公开辩论中为一篇被认为是亵渎性的论文辩护,但目前尚不清楚这一事实是否与他没有参加考试有关。[10]
\subsection{剑桥大学之后} 
考虑到他的声誉,Babbage 很快取得了进展。他于 1815 年在皇家学会讲授天文学,并在 1816 年当选为皇家学会会员。[25] 然而,毕业后,他申请职位未果,事业发展有限。1816 年,他曾申请海利伯里学院的教职,获得了詹姆斯·艾弗里和约翰·普莱费尔的推荐信,但最终败给了亨利·沃尔特。[26] 1819 年,Babbage 和赫歇尔访问了巴黎和阿尔库伊学会,拜访了法国的著名数学家和物理学家。[27] 同年,Babbage 在皮埃尔·西蒙·拉普拉斯的推荐下申请了爱丁堡大学的教授职位,但该职位最终授予了威廉·沃拉斯。[28][29][30]

Babbage 与赫歇尔一起研究了阿拉戈旋转的电动力学,并于 1825 年发表了相关研究。他们的解释只是过渡性的,后来由迈克尔·法拉第接手并加以扩展。现今这些现象已成为涡电流理论的一部分,而 Babbage 和赫歇尔未能抓住统一电磁理论的某些线索,他们仍然局限于安培的力学定律。[31]

Babbage 购买了乔治·巴雷特的精算表,巴雷特于 1821 年去世,留下未出版的工作,并在 1826 年通过《各种人寿保险机构的比较视角》对该领域进行了调查。[32] 他对此兴趣的起因是一个筹建保险公司的项目,该项目受到弗朗西斯·贝利的推动,并于 1824 年提出,但未能付诸实施。[33] Babbage 确实为这一计划计算了精算表,使用了自 1762 年起的“公正公司”死亡率数据。[34]

在整个这一时期,Babbage 在父亲的支持下生活,尽管他的父亲对他 1814 年的早婚态度颇为尴尬:他与爱德华·赖安一起娶了惠特莫尔姐妹。他在伦敦的玛丽尔本建立了家庭,并养育了众多子女。[35] 1827 年父亲去世后,Babbage 继承了一笔大财产(大约 10 万英镑,相当于今天的 1090 万英镑或 1500 万美元),使他变得经济独立。[10] 由于妻子在同年去世,他开始了旅行生活。在意大利,他遇到了托斯卡纳的大公利奥波德二世,预示着他之后将访问皮埃蒙特。[25] 1828 年 4 月,他在罗马,依赖赫歇尔来管理差分机项目时,得知自己成为了剑桥大学的教授,而这一职位是他三次未能获得的职位(分别在 1820 年、1823 年和 1826 年)[36]。
\subsubsection{皇家天文学会}  
\begin{figure}[ht]
\centering
\includegraphics[width=6cm]{./figures/ce04ddc804b98a90.png}
\caption{差分机的一部分} \label{fig_CRSBQ_7}
\end{figure}
Babbage 在 1820 年为创立皇家天文学会发挥了重要作用,该会最初名为伦敦天文学会。[37] 它的最初目标是将天文学计算简化为更标准的形式,并传播数据。[38] 这些目标与 Babbage 关于计算的理念密切相关,1824 年他因此获得了该会的金奖勋章,表彰他“发明了一个用于计算数学和天文表格的引擎”。[39]

Babbage 希望通过机械化克服表格中的错误的动机,在 1834 年由狄俄尼修斯·拉德纳(Dionysius Lardner)在《爱丁堡评论》中提到时已成为常见话题(在 Babbage 的指导下)。[40][41] 这些发展的背景仍然存在争议。Babbage 自述的差分机起源故事始于天文学会希望改进《航海年鉴》一书的愿望。Babbage 和赫歇尔被要求监督一个试点项目,重新计算该表格的某部分数据。在拿到结果后,发现了不一致之处。这发生在 1821 或 1822 年,也是 Babbage 提出机械计算想法的契机。[42] 《航海年鉴》的问题如今被描述为英国科学中的一种分裂遗产,这种分裂源于对已故约瑟夫·班克斯爵士(Sir Joseph Banks)的态度,班克斯于 1820 年去世。[43]

Babbage 与他的朋友托马斯·弗雷德里克·科尔比(Thomas Frederick Colby)研究了建立现代邮政系统的需求,得出结论认为应该设立统一的邮资费率。这个想法最终在 1839 年和 1840 年通过推出统一的四便士邮政系统和后来的统一便士邮政系统得以实施。[44] 科尔比是该学会创始小组的成员之一。[45] 他还负责爱尔兰的测量工作。赫歇尔和 Babbage 也参与了该测量工作中一次著名的行动——重新测量福伊尔湖基线。[46]
\subsubsection{英国拉格朗日学派}
分析学会最初不过是一个本科生的挑衅。在这一时期,它取得了一些更为实质性的成就。1816 年,Babbage、赫歇尔和皮科克共同出版了从法语翻译过来的西尔维斯特·拉克鲁瓦(Sylvestre Lacroix)讲义,这本讲义当时是最先进的微积分教材。[47]

在微积分领域提到拉格朗日,标志着现在所称的正规幂级数的应用。英国数学家大约从 1730 年到 1760 年期间就已经使用了这些方法。重新引入后,它们不仅仅作为微分学中的符号表示被应用。它们开启了函数方程(包括差分机基础的差分方程)和微分方程的算子(D-模)方法的研究。差分方程和微分方程的类比通过符号变化将Δ变为D,将“有限”的差异变为“无穷小”。这些符号化的方向成为了运算微积分,并推动到收益递减的程度。科西的极限概念被排除在外。[48] 伍德豪斯早已建立了这个第二个“英国拉格朗日学派”,并将泰勒级数作为正规化处理。[49]

在这个背景下,函数组合的表达变得复杂,因为链式法则不仅仅适用于二阶及更高阶导数。伍德豪斯在 1803 年就已知道这个问题,他从路易·弗朗索瓦·安托万·阿尔博戈斯特那里得到了如今称为法·迪·布鲁诺公式的东西。事实上,这个方法早在 1697 年就已为亚伯拉罕·德·莫伊夫(Abraham De Moivre)所知。赫歇尔对这一方法印象深刻,Babbage 也知道这一方法,后来 Ada Lovelace 认为它与分析机相兼容。[50] 在 1820 年之前的这段时间,Babbage 集中精力研究一般的函数方程,并抵制传统的有限差分法和阿尔博戈斯特的方法(在该方法中,Δ 和 D 通过指数映射的简单加法关系联系)。但通过赫歇尔,他在迭代问题上受到了阿尔博戈斯特思想的影响,即将一个函数与自身组合,可能会多次组合。[49] 在他于《哲学会刊》(Philosophical Transactions)上发表的关于函数方程的重要论文中(1815/6),Babbage 说他的出发点是加斯帕尔·蒙日(Gaspard Monge)的工作。[51]
\subsection{学术生涯} 
从 1828 年到 1839 年,Babbage 担任剑桥大学卢卡斯数学教授。他不是一位传统的常驻教授,也不太关心教学责任,在这段时间里,他写了三本专题书籍。1832 年,他被选为美国艺术与科学学院的外籍荣誉会员。[52] Babbage 与同事们的关系并不融洽:乔治·比德尔·艾里(George Biddell Airy),他作为卢卡斯数学教授的前任,认为应该对 Babbage 不热衷讲授的态度做出反应。Babbage 曾计划在 1831 年讲授政治经济学。他的改革方向旨在使大学教育更加包容,大学在研究方面做得更多,课程大纲更加广泛,且更加关注应用;但威廉·惠威尔(William Whewell)认为这个计划不可接受。Babbage 与理查德·琼斯(Richard Jones)之间的争论持续了六年。[54] 他最终并未进行讲授。[55]

在这段期间,Babbage 尝试进入政治领域。西蒙·谢弗(Simon Schaffer)写道,他在 1830 年代的观点包括解散英国国教、扩大选举权和将制造商纳入利益相关者。[56] 他曾两次作为候选人竞选芬斯伯里区的国会议员。1832 年,他在五名候选人中排名第三,在双议员选区中落后约 500 票,原因是两位其他改革派候选人,托马斯·沃克利(Thomas Wakley)和克里斯托弗·坦普尔(Christopher Temple),分裂了选票。[57][58] 在回忆录中,Babbage 讲述了这次选举如何让他与塞缪尔·罗杰斯(Samuel Rogers)建立了友谊:他的兄弟亨利·罗杰斯原本打算再次支持 Babbage,但几天后去世。[59] 1834 年,Babbage 在四名候选人中排最后一位。[60][61][62] 1832 年,Babbage、赫歇尔和艾弗里被任命为皇家格尔夫勋章骑士,但他们随后并未被授予骑士爵士头衔,因此不能使用“Sir”这一前缀,尽管赫歇尔后来被封为男爵。[63]
\subsubsection{“衰退派”、学术团体与英国科学促进协会(BAAS)}
\begin{figure}[ht]
\centering
\includegraphics[width=6cm]{./figures/a4dcfcd9459609cb.png}
\caption{《致汉弗里·戴维爵士的信》,1822年} \label{fig_CRSBQ_6}
\end{figure}
Babbage 此时开始成为一位激进的辩论者。其一位传记作者指出,他的所有著作中都包含了“运动性质的元素”。然而,他的《科学衰退及其一些原因的反思》(1830)尤为突出,以其尖锐的攻击而闻名。该书旨在改善英国科学,特别是推翻戴维·吉尔伯特(Davies Gilbert)作为皇家学会会长的职务,因为 Babbage 希望改革该学会。[64] 该书的写作源于他的一种愤怒,当时 Babbage 希望成为皇家学会的 junior secretary(助理秘书),而赫歇尔担任 senior secretary(高级秘书),但由于他与汉弗里·戴维(Humphry Davy)的对立,他未能成功。[65] 迈克尔·法拉第(Michael Faraday)对此进行了回应,由格里特·莫尔(Gerrit Moll)撰写《论英国科学的所谓衰退》(1831)。[66] 然而,Babbage 在皇家学会的影响力有限,吉尔伯特同年被选为苏塞克斯公爵(Duke of Sussex)继任会长,选举平淡无奇。另一方面,作为一个广泛的宣言,他的《衰退》迅速促成了英国科学促进协会(BAAS)的成立,该协会于 1831 年成立。[66]

《机械杂志》在 1831 年将 Babbage 的追随者称为“衰退派”(Declinarians)。该杂志以不友善的语气指出,大卫·布鲁斯特(David Brewster)在《季刊评论》(Quarterly Review)上写文章,并且暗示 Babbage 和布鲁斯特都曾接受过公共资助。[67]

在当时关于统计学(即数据收集)以及如今所谓的统计推断的辩论中,BAAS 在其统计学分会中(该分会部分也受惠于惠威尔)选择了数据收集。该分会是第六个分会,于 1833 年成立,Babbage 担任会长,约翰·艾略特·德林克沃特(John Elliot Drinkwater)担任秘书。统计学会随后也应运而生。[68][69][70] Babbage 是该学会的公众形象,得到了理查德·琼斯和罗伯特·马尔萨斯的支持。[71]
\subsubsection{《机械与制造业的经济学》}
\begin{figure}[ht]
\centering
\includegraphics[width=6cm]{./figures/f286e97ce9384082.png}
\caption{《机器与制造业的经济学》,1835年} \label{fig_CRSBQ_5}
\end{figure}
Babbage 发表了《机械与制造业的经济学》(On the Economy of Machinery and Manufactures,1832),讨论了工业生产的组织问题。这是运筹学的早期具有影响力的著作之一。[73] 约翰·伦尼(John Rennie the Younger)在 1846 年向土木工程师学会(Institution of Civil Engineers)介绍制造业时,主要提到了百科全书中的调查,而 Babbage 的这本书最初是《大百科全书》(Encyclopædia Metropolitana)中的一篇文章,伦尼正是在这篇文章中提到了它,并与约翰·法雷(John Farey Jr.)、彼得·巴洛(Peter Barlow)和安德鲁·尤尔(Andrew Ure)的相关著作一起出现。[74] 在《机械应用于制造业和机械艺术的一般原理论文》(An Essay on the General Principles Which Regulate the Application of Machinery to Manufactures and the Mechanical Arts,1827)中,Babbage 提出了机器的图式分类,结合对工厂的讨论,构成了本书的第一部分。第二部分则讨论了制造业的“家庭与政治经济”问题。[75]

这本书卖得很好,并迅速进入了第四版(1836年)。[76] Babbage 将他的著作描述为主要基于他在国内外工厂中的实际观察。它在初版时并未意图讨论政治经济学的更深层次问题;但第二版(1832 年末)则包含了三章新内容,其中包括一章关于计件工资(piece rate)的讨论。[77] 书中还包含了关于工厂合理设计和利润分配的观点。[78]
\subsubsection{Babbage原理 } 
在《机械与制造业的经济学》中,描述了现在所称为“Babbage原理”的内容。该原理指出,通过更精细的劳动分工,可以带来商业上的优势。正如Babbage本人所提到的,这一原理在1815年梅尔基奥雷·乔亚(Melchiorre Gioia)的著作中已有出现。[79] 这个术语由哈里·布拉弗曼(Harry Braverman)在1974年提出。[80] 相关的表述包括菲利普·萨尔甘特·弗洛伦斯(Philip Sargant Florence)的“倍数原理”(principle of multiples)和“过程平衡”(balance of processes)。[81][82]

Babbage指出,熟练工人通常会花费部分时间执行低于自己技能水平的任务。如果劳动过程可以在多个工人之间分配,则通过将高技能任务分配给高成本工人,并将其他任务限制给低薪工人,劳动成本可能会降低。[83] 他还指出,培训或学徒制可以被视为固定成本;但是,通过他所提出的标准化任务方法,可以实现规模效益,因此再次支持工厂制度。[84] 他对人力资本的看法仅限于最小化培训成本回收的时间周期。[85]
\subsubsection{出版}  
\begin{figure}[ht]
\centering
\includegraphics[width=6cm]{./figures/c7faa744c37e9fb1.png}
\caption{巴贝奇的机器部件符号,来自他在《用符号表示机器作用的方法》(1827)中对‘机械符号’的解释,这些符号是他为理解差分机的工作而发明的,并且对解析机的构思产生了影响。} \label{fig_CRSBQ_4}
\end{figure}
该著作的另一个重要方面是它对图书出版成本结构的详细分析。Babbage从出版商的角度采取了不受欢迎的立场,揭示了这一行业的盈利性。[86] 他甚至公开指名道姓,揭露了该行业的限制性商业行为的组织者。[87] 二十年后,他参加了约翰·查普曼(John Chapman)主办的会议,继续反对书商协会(Booksellers Association),该协会依然是一个卡特尔。[88]
\subsubsection{影响}  
有人写道,“亚瑟·杨(Arthur Young)对农业的影响,正如查尔斯·巴贝奇(Charles Babbage)对工厂参观和机械的影响。”[89] Babbage的理论被认为对1851年大博览会的布局产生了影响,[90] 他的观点对当时的乔治·朱利叶斯·保莱特·斯科普(George Julius Poulett Scrope)产生了强烈影响。[91] 卡尔·马克思(Karl Marx)认为,工厂制度的生产力来源正是劳动分工与机械的结合,继承了亚当·斯密(Adam Smith)、Babbage和尤尔(Ure)的观点。[92] 马克思从Babbage的理论中吸取了并与斯密不同的地方,认为劳动分工的动机在于追求盈利而非提高生产力,并指出它对贸易概念的影响。[93]

约翰·罗斯金(John Ruskin)更进一步,完全反对Babbage所代表的制造业模式。[94] Babbage的观点也影响了约翰·斯图尔特·密尔(John Stuart Mill)的经济思维。[95] 乔治·霍利奥克(George Holyoake)认为,Babbage对利润分配的详细讨论具有实质性,延续了罗伯特·欧文(Robert Owen)和查尔斯·富里叶(Charles Fourier)的传统,尽管当时需要有一位仁慈的工业领袖关注这一问题,然而却被忽视了。[96]

查尔斯·巴贝奇的星期六晚间聚会从1828年持续到1840年代,是当时著名科学家、作家和贵族的重要聚会场所。Babbage因其广受欢迎的星期六晚会而被誉为引入“科学聚会”的人,这一形式源自法国。[8][9]

Babbage与尤尔的作品在1830年以法文翻译出版;《机械与制造业的经济学》于1833年由埃德瓦尔·比奥(Édouard Biot)翻译成法语,同年由戈特弗里德·弗里德伯格(Gottfried Friedenberg)翻译成德语。[98] 法国工程师兼工业组织作家莱昂·拉兰(Léon Lalanne)受Babbage影响,同时也受经济学家克劳德·吕西安·贝热里(Claude Lucien Bergery)影响,将问题归结为“技术”问题。[99] 威廉·杰文斯(William Jevons)将Babbage的“劳动经济学”与他在1870年的劳动实验相联系。[100] Babbage原理在弗雷德里克·温斯洛·泰勒(Frederick Winslow Taylor)的科学管理理论中也被视为一个固有的假设。[101]

玛丽·埃弗里斯特·布尔(Mary Everest Boole)声称,通过她的叔叔乔治·埃弗里斯特(George Everest),印度思想,特别是印度逻辑,对Babbage、她的丈夫乔治·布尔(George Boole)以及奥古斯都·德·摩根(Augustus De Morgan)产生了深远影响:

想想Babbage、De Morgan和乔治·布尔这三位男士的强烈印度化,会对1830-1865年间的数学氛围产生什么样的影响。它在产生矢量分析(Vector Analysis)和用于物理科学研究的数学方法方面起到了怎样的作用?[102]
\subsubsection{自然神学}
\begin{figure}[ht]
\centering
\includegraphics[width=6cm]{./figures/6acb4f015d693f5c.png}
\caption{《第九版布里奇沃特论述中的插图,展示了一类代数曲线的参数族,获得孤立的实数点》} \label{fig_CRSBQ_3}
\end{figure}
1837年,作为对《布里奇沃特论集》八篇著作的回应,Babbage出版了他的《第九篇布里奇沃特论集》,标题为《论上帝在创造中的能力、智慧与善良》。在这本书中,Babbage支持了当前辩论中的均变论(uniformitarianism)。[103] 他倾向于接受一种创造的观念,其中上帝赋予的自然法则主宰一切,从而不需要持续的“设计”行为。[104]

这本书是一本自然神学著作,包含了赫歇尔(Herschel)与查尔斯·莱尔(Charles Lyell)之间相关信件的摘录。[105] Babbage提出了这样的观点:上帝拥有全能和远见,可以作为一位神圣的立法者来创造世界。在这本书中,Babbage探讨了科学与宗教之间的关系解读;一方面,他坚持认为“圣经的话语与自然的事实之间并不存在致命的冲突”;另一方面,他写道,《创世纪》一书不应当从科学术语的角度被字面理解。针对那些认为科学和宗教相冲突的人,Babbage写道:“他们所设想的矛盾不可能真实存在,并且只要摩西的见证未曾受到指责,我们就有理由相信我们感官的见证。”[106]

《第九篇布里奇沃特论集》在《创造的自然历史迹象》(Vestiges of the Natural History of Creation)一书中被广泛引用。[107] 书中明确将Babbage的计算机与物种转变的理论相类比,认为物种的转变可能是预先编程的。[108]

乔纳·甘瑞(Jonar Ganeri),《印度逻辑》一书的作者,认为Babbage可能受到了印度思想的影响;其中一种可能的途径是通过亨利·托马斯·科尔布鲁克(Henry Thomas Colebrooke)。[109] 玛丽·埃弗雷斯特·布尔(Mary Everest Boole)认为,Babbage是在1820年代通过她的叔叔乔治·埃弗雷斯特(George Everest)接触到印度思想的:

“大约在1825年左右,[埃弗雷斯特]来到了英格兰,停留了两三年,并与赫歇尔(Herschel)和当时还很年轻的Babbage建立了快速且终生的友谊。我希望任何公平的数学家都能阅读Babbage的《第九篇布里奇沃特论集》,并将其与他在英国的同时代人作品进行对比;然后问问自己,Babbage关于曲线奇异点(第八章)的特殊观念究竟是来源于欧洲神学,还是印度形而上学?哦!当时的英国教士是多么痛恨Babbage的这本书啊!”[102]
\subsubsection{宗教观点}
Babbage在基督教的新教形式中长大,家人将正统的崇拜形式灌输给了他。[110] 他解释道:

“我优秀的母亲教会了我每日和夜晚的祈祷形式;我的父母双方都没有偏执和不容忍,也没有像我年轻时那样让我感到厌恶的、不敬且过于熟悉的称呼上帝的方式。”[111]

他在年轻时拒绝了《亚他那修信经》,认为其为“直接的矛盾”,并在此期间受到了塞缪尔·克拉克(Samuel Clarke)关于宗教的作品,特别是《神的存在与属性》(1704)的强烈影响。后来,Babbage得出结论:“基督教的真正价值不在于推测性[神学]...而在于...它所宣扬和执行的仁慈与善意教义,这些教义不仅是为了人类本身,也为了所有能够感知痛苦或幸福的生物。”[112]

在他的自传《哲学家的生活片段》(1864)中,Babbage写了一整章关于宗教的话题,他在其中确定了三种神圣知识的来源:[113]
\begin{enumerate}
\item 先验或神秘经验
\item 启示
\item 对创造者之工作进行考察
\end{enumerate}
他基于设计论的观点表示,研究自然界的作品是更具吸引力的证据,并且是促使他积极宣称上帝存在的原因。[114][115] 提倡自然神学时,他写道:

“在创造者的工作中,我们拥有一个坚实的基础,可以在此基础上建立一个开明的信仰体系。人类越是研究调控物质宇宙的法则,越会确信宇宙的各种形式源自几个简单原则的作用……创造者的作品,始终展现在我们的感官面前,给予我们活生生的、永恒的见证,证明了他的力量与善良,远远超过了通过人类见证传递的任何证据。人类的见证在每次传递的过程中变得越来越微弱,而每一次对全能者作品的探索,都会让我们对他的智慧、善良和力量有更崇高的认识。”[116]

像塞缪尔·文斯(Samuel Vince)一样,Babbage也为相信神迹辩护。[117] 反驳大卫·休谟(David Hume)之前提出的反对意见,Babbage主张相信神的行为,他说:“我们不能仅仅凭借自己狭窄的经验来衡量事件的可信度或不可信度,也不能忘记有一种超越自然法则的神圣能量。”[118] 他提到人类经验的局限性,表达道:“我们在神迹中看到的只是一个对我们观察而言是新的效果,其原因被隐藏起来。原因可能超出了我们的观察范围,因此也超出了我们熟悉的自然范畴;但这并不意味着事件违反了任何自然法则。人类的观察限于非常狭窄的范围,假设人类的力量范围即是自然界的限度,这是一种傲慢。”[119]
\subsection{晚年}
\begin{figure}[ht]
\centering
\includegraphics[width=6cm]{./figures/1e1663cbeb508c1b.png}
\caption{《插图伦敦新闻》(1871年11月4日)} \label{fig_CRSBQ_2}
\end{figure}
英国学会(BAAS)在很大程度上是模仿了1822年成立的德国自然研究者大会(Deutsche Naturforscher-Versammlung)。该学会拒绝浪漫主义科学以及形而上学,开始确立科学与文学的分界,以及专业人士与业余爱好者的分隔。作为BAAS中‘瓦特派’的成员,特别是由詹姆斯·瓦特(James Watt)之子代表的瓦特派,巴贝奇与工业界保持紧密联系。他希望在相同的方向上更快地前进,并对其会员中更为绅士化的部分没有太多兴趣。事实上,他认同一种推测性的历史观,认为工业社会是人类发展的巅峰(这一观点他与赫歇尔共享)。1838年与罗德里克·默奇森(Roderick Murchison)发生冲突后,他退出了进一步的参与。年末,他辞去了卢卡西安教授职务,也结束了与剑桥大学的威胡尔(Whewell)之争。他的兴趣变得更加集中,专注于计算和计量学,以及国际间的联系。
\subsubsection{计量学计划}
巴贝奇宣布的一个项目是列出所有物理常数(称为“自然常数”,这一术语本身就是一种新词),然后编纂一部包含数值信息的百科全书。他是“绝对测量”领域的先驱之一。其思想继承了约翰·克里斯蒂安·波根多夫(Johann Christian Poggendorff)的观点,并于1832年向布鲁斯特提到过这些想法。该项目涉及19个常数类别,伊恩·哈金(Ian Hacking)认为这些类别部分反映了巴贝奇“古怪的热情”。巴贝奇的论文《自然与艺术常数表》于1856年由史密森学会重新印刷,并附加了一段说明,指出阿诺德·亨利·吉约(Arnold Henry Guyot)的物理学表“将成为本文所提重要工作的组成部分”。

精确测量对于机床的发展也至关重要。在这一领域,巴贝奇被视为先驱之一,与亨利·莫德斯雷(Henry Maudslay)、威廉·塞勒斯(William Sellers)和约瑟夫·惠特沃斯(Joseph Whitworth)并列。
\subsubsection{工程师与发明家}
通过皇家学会,巴贝奇结识了工程师马克·布鲁内尔(Marc Brunel)。正是通过布鲁内尔,巴贝奇认识了约瑟夫·克莱门特(Joseph Clement),从而接触到了他在制造业工作中观察到的工匠。1830年,巴贝奇为伊桑巴德·金斯敦·布鲁内尔(Isambard Kingdom Brunel)引荐了与拟议中的布里斯托尔与伯明翰铁路(Bristol & Birmingham Railway)的接触。1838年左右,他进行了一项研究,证明了布鲁内尔的“大西部铁路”使用的宽轨道在铁路上具有优势。

1838年,巴贝奇发明了“前导器”(也叫“撞牛器”),即安装在机车前部的金属框架,用于清除铁轨上的障碍物;他还制造了一个测力车。他的长子本杰明·赫歇尔·巴贝奇(Benjamin Herschel Babbage)曾在布鲁内尔的铁路上担任工程师,后来于1850年代移民到澳大利亚。

巴贝奇还发明了一个眼底镜,并将其交给托马斯·沃顿·琼斯(Thomas Wharton Jones)进行测试。然而,琼斯忽视了这一发明。该设备直到赫尔曼·冯·亥姆霍兹(Hermann von Helmholtz)独立发明之后才开始被广泛使用。
\subsubsection{密码学}
巴贝奇在密码学方面取得了显著成果,尽管这些成果在他去世后一个世纪仍未被人们所知。字母频率是巴贝奇列出项目中的第18类。后来,约瑟夫·亨利(Joseph Henry)在缺乏事实依据的情况下,为对其的兴趣进行了辩护,认为它与可移动活字的管理相关。

早在1845年,巴贝奇就破解了他侄子亨利·霍利尔(Henry Hollier)提出的一个密码挑战,并在这个过程中发现了关于基于维吉尼尔表的密码的一些新知识。具体来说,他意识到使用一个关键词对明文进行加密会使密文受到模运算的影响。在1850年代的克里米亚战争期间,巴贝奇破解了维吉尼尔的自动密钥密码以及今天称为维吉尼尔密码的较弱密码。他的发现被作为军事机密保密,并没有公开。相反,这一成果的荣誉被归功于普鲁士步兵军官弗里德里希·卡西斯基(Friedrich Kasiski),他在几年后做出了相同的发现。然而,巴贝奇在1854年发表了解析一个维吉尼尔密码的文章,这篇文章之前曾在《艺术学会杂志》上发布过。在1855年,巴贝奇还在同一杂志上发表了一封简短的信件《密码书写》。然而,直到1985年,他的优先权才得以确立。
\subsubsection{公共害物}
巴贝奇参与了针对公共害物的广为宣传但不受欢迎的运动。他曾统计过一家工厂的所有破损玻璃窗,并在1857年发布了《玻璃窗破损原因的相对频率表》:在464块破损的窗玻璃中,14块是由于“醉汉、女人或男孩”造成的。

巴贝奇对普通人(即“暴民”)的反感包括在1864年写下《街头噪音观察》,并统计了80天内的165个“噪音源”。他特别讨厌街头音乐,尤其是街头拉风琴的人,他在多个场合对他们进行抨击。以下是他的一段典型引用:

“很难估计成千上万的人所遭受的痛苦,以及那些知识工作者因时间被浪费在风琴手和其他类似噪音上而遭受的经济损失。”

巴贝奇在这场运动中并不孤单。该运动的支持者之一是议员迈克尔·托马斯·巴斯(Michael Thomas Bass)。

在1860年代,巴贝奇还发起了反对滚铁环的运动。他指责玩滚铁环的孩子将铁环滚到马的腿下,导致骑手摔下并且常常让马骨折。巴贝奇因此在这件事上获得了某种恶名,在1864年的下议院辩论中,他被指控“发动了一场针对流行的‘点球’游戏和滚铁环的十字军运动。
\subsection{计算先锋}
\begin{figure}[ht]
\centering
\includegraphics[width=6cm]{./figures/7fc3d91079e79d88.png}
\caption{查尔斯·巴贝奇的差分机(#1)的一部分,由他的儿子亨利·普雷沃斯特·巴贝奇(Henry Prevost Babbage,1824–1918)在查尔斯去世后组装,使用的是查尔斯实验室中找到的零件。位于英国剑桥的惠普尔科学史博物馆。} \label{fig_CRSBQ_10}
\end{figure}
巴贝奇的机器是最早的机械计算机之一。它们未能最终完成,主要是由于资金问题和个人冲突,特别是与皇家天文学家乔治·比戴尔·艾里(George Biddell Airy)之间的矛盾。

巴贝奇指导了几台蒸汽驱动的机器的建造,这些机器取得了一定的成功,表明计算可以实现机械化。在十多年里,他获得了政府的资助,金额达到17,000英镑,但最终财政部对他失去了信心。

尽管巴贝奇的机器机械且笨重,但它们的基本架构与现代计算机相似。数据和程序内存是分开的,操作是基于指令的,控制单元能够进行条件跳转,机器还拥有独立的输入/输出单元。
\subsubsection{数学表格的背景}
在巴贝奇的时代,印刷的数学表格是由人工计算员计算出来的;换句话说,就是手工完成的。这些表格对导航、科学、工程以及数学至关重要。然而,在抄写和计算过程中经常会发生错误。

在剑桥,巴贝奇意识到这一过程的易错性,并看到了将机械化引入其管理的机会。他自己在回顾通向机械计算的道路时提到过一个特别的时刻:

“1812年,我坐在分析学会的房间里,看着一张对数表,我知道这张表满是错误。那时,我想到通过机械计算所有的表格函数。法国政府采用了一种新方法制作了几张表格。三四个数学家决定如何计算这些表格,再有六七个数学家将计算过程分解成简单的步骤,最终的工作只涉及加法和减法,由80个只懂这两种算术运算的计算员完成。这是第一次,批量生产的理念被应用到算术运算中,巴贝奇意识到,普通计算员的工作完全可以由机械取代,而机械将更快、更可靠。”

七年后,巴贝奇再次对数学表格计算的问题产生兴趣。他熟悉法国官方发起的加斯帕尔·德·普罗尼(Gaspard de Prony)项目及其实施中的问题。拿破仑战争结束后,科学界的人际联系恢复了:1819年,查尔斯·布莱格登(Charles Blagden)在巴黎考察停滞的普罗尼项目的印刷工作,并为皇家学会争取支持。在1820年代和1830年代的著作中,巴贝奇详细提到过普罗尼的项目。
\subsubsection{差分机}
\begin{figure}[ht]
\centering
\includegraphics[width=8cm]{./figures/2e8d3cde868a4e43.png}
\caption{科学博物馆的差分机2号,依据巴贝奇的设计建造} \label{fig_CRSBQ_12}
\end{figure}
巴贝奇在1822年开始设计他所称的差分机,用来计算多项式函数的值。它的目的是自动计算一系列值。通过使用有限差分法,可以避免需要进行乘法和除法运算。

为了制作差分机的原型,巴贝奇在1823年邀请了约瑟夫·克莱门特(Joseph Clement)来实施设计。克莱门特的工作标准非常高,但他的机床设计特别复杂。根据当时的商业惯例,他可以为这些机床的建造收费,并且机床的所有权也归他所有。大约在1831年,巴贝奇与克莱门特因费用问题发生了争执。

原型的一些零件至今保存在牛津的科学史博物馆。这个原型后来演变成了“第一台差分机”。它仍未完成,完成的部分现存于伦敦的科学博物馆。这台第一台差分机将由大约25,000个零件组成,重达15短吨(约13,600千克),高达8英尺(约2.4米)。尽管巴贝奇为该项目获得了充足的资金,但它始终未能完成。后来(1847–1849年),他为改进版本“差分机2号”制作了详细的图纸,但没有获得英国政府的资助。最终,这一设计在1989至1991年间根据巴贝奇的计划和19世纪的制造公差被建造完成,并在伦敦的科学博物馆进行了首次计算,结果达到了31位数字。

九年后,即2000年,科学博物馆完成了巴贝奇为差分机设计的打印机。
\begin{figure}[ht]
\centering
\includegraphics[width=6cm]{./figures/a5f840b5d924e322.png}
\caption{巴贝奇差分机的一部分} \label{fig_CRSBQ_11}
\end{figure}

\textbf{完成的模型}

科学博物馆根据巴贝奇的差分机2号设计建造了两台差分机。其中一台由博物馆拥有,另一台则由科技亿万富翁内森·迈尔沃尔德(Nathan Myhrvold)拥有,并于2008年5月10日在加利福尼亚州山景城的计算机历史博物馆展出。这两台已建成的模型并非复制品。
\subsubsection{解析机}
\begin{figure}[ht]
\centering
\includegraphics[width=8cm]{./figures/ac08b046327a97cd.png}
\caption{解析机的磨盘部分,带有打印机制,由查尔斯·巴贝奇建造,现展示于伦敦科学博物馆} \label{fig_CRSBQ_13}
\end{figure}
在第一次尝试制造差分机失败后,巴贝奇开始设计一个更复杂的机器——解析机(Analytical Engine)。他聘用了C. G. 贾维斯(C. G. Jarvis),贾维斯曾在克莱门特(Clement)处担任制图员。

解析机标志着从机械算术到完全成熟的通用计算的过渡。巴贝奇作为计算机先驱的地位,主要建立在解析机的基础上。

其主要创新在于,解析机将使用打孔卡片进行编程:该机器计划使用雅可卡(Jacquard)的打孔卡片循环来控制机械计算器,计算器可以将前面计算的结果作为输入。解析机还计划采用一些现代计算机中使用的特性,包括顺序控制、分支和循环。它在原则上将是第一个具备图灵完备性的机械设备。巴贝奇从1837年到1840年为解析机编写了一系列程序。第一个程序于1837年完成。解析机不是一台单一的物理机器,而是一系列设计,巴贝奇在他的生命中不断改进这些设计,直到1871年去世。
\subsubsection{艾达·洛夫莱斯与意大利追随者}
\begin{figure}[ht]
\centering
\includegraphics[width=8cm]{./figures/19d436846b397133.png}
\caption{解析机的部分展示,1843年,位于此幅刻画伦敦国王学院乔治三世博物馆图像的中央左侧} \label{fig_CRSBQ_14}
\end{figure}
艾达·洛夫莱斯(Ada Lovelace)在巴贝奇开发解析机期间与他通信,并被认为是开发了一个算法,使得解析机能够计算伯努利数列。尽管有洛夫莱斯亲笔书写的文献证据,一些学者对这些想法究竟有多少是洛夫莱斯自己的表示质疑。由于这一成就,她常被称为第一个计算机程序员;尽管那时还没有发明编程语言。

洛夫莱斯还翻译并撰写了支持该项目的文献。她描述解析机通过打孔卡片进行编程时写道:“我们可以最恰当地说,解析机编织代数模式,就像雅可卡织布机编织花卉和叶子一样。”

巴贝奇于1840年应乔凡尼·普拉纳(Giovanni Plana)邀请访问都灵。普拉纳在1831年开发了一个模拟计算机,作为万年历。在1840年,巴贝奇在都灵进行了关于解析机的唯一公开讲解和演讲。1842年,查尔斯·惠特斯通(Charles Wheatstone)找到洛夫莱斯,邀请她翻译路易吉·梅纳布雷亚(Luigi Menabrea)的一篇论文,梅纳布雷亚曾在巴贝奇的都灵讲座中做了记录;巴贝奇还请她加入一些自己的内容。作为都灵翻译员的福尔图纳托·普兰迪(Fortunato Prandi)是一名意大利流亡者,还是朱塞佩·马志尼(Giuseppe Mazzini)的追随者。
\subsubsection{瑞典追随者} 
佩尔·乔治·谢茨(Per Georg Scheutz)在1830年写了关于差分机的文章,并开始了自动计算的实验。1834年,在阅读了拉德纳(Lardner)在《爱丁堡评论》中的文章后,他开始了自己的项目,怀疑巴贝奇最初的计划是否能实现。他与儿子埃德瓦尔德·谢茨(Edvard Scheutz)共同推动了这个项目。另一个瑞典的差分机是马丁·维伯格(Martin Wiberg)于1860年设计的。
\subsubsection{遗产} 
2011年,英国的研究人员提出了一项数百万英镑的项目,“计划28”(Plan 28),旨在构建巴贝奇的解析机。由于巴贝奇的设计计划不断完善并且从未完成,他们打算让公众参与这个项目,并通过众包的方式分析应建造什么样的机器。这个机器将拥有相当于675字节的内存,时钟频率约为7赫兹。他们希望能够在巴贝奇去世150周年时(2021年)完成这项工作。

微电子机械系统(MEMS)和纳米技术的进展推动了最近机械计算领域的高科技实验。建议的优势包括能够在高辐射或高温环境下运行。这些现代机械计算版本在《经济学人》杂志的“千年末期”特刊中得到了强调,特刊标题为“巴贝奇的最后一笑”。

由于与该城镇的联系,巴贝奇在2007年被选中出现在托特尼斯五英镑纸币上。2015年,巴贝奇的形象出现在新设计的英国护照中的英国文化图标部分。
\subsection{家庭和子女} 
1814年7月25日,巴贝奇与乔治安娜·惠特莫尔(Georgiana Whitmore)结婚,乔治安娜是英国议员威廉·沃尔里奇-惠特莫尔(William Wolryche-Whitmore)的妹妹,婚礼在德文郡泰恩茅斯的圣迈克尔教堂举行。婚后,夫妻俩住在什罗普郡的达德马斯顿大厅(Dudmaston Hall),巴贝奇在这里设计了中央供暖系统,之后于1815年搬到了伦敦的德文郡街5号。

查尔斯和乔治安娜共有八个孩子,但只有四个孩子存活至成年,分别是本杰明·赫歇尔(Benjamin Herschel)、乔治安娜·惠特莫尔(Georgiana Whitmore)、杜戈尔德·布罗姆赫德(Dugald Bromhead)和亨利·普雷沃斯特(Henry Prevost)。乔治安娜于1827年9月1日在伍斯特去世,恰好与巴贝奇的父亲、他们的第二子(也名为查尔斯)以及新生儿亚历山大同年去世。
\begin{itemize}
\item 本杰明·赫歇尔·巴贝奇(Benjamin Herschel Babbage,1815–1878)
\item 查尔斯·惠特莫尔·巴贝奇(Charles Whitmore Babbage,1817–1827)
\item 乔治安娜·惠特莫尔·巴贝奇(Georgiana Whitmore Babbage,1818–1834年9月26日)
\item 爱德华·斯图尔特·巴贝奇(Edward Stewart Babbage,1819–1821)
\item 弗朗西斯·摩尔·巴贝奇(Francis Moore Babbage,1821–????)
\item 杜戈尔德·布罗姆赫德(Bromhead?)·巴贝奇(Dugald Bromhead Babbage,1823–1901)
\item 亨利·普雷沃斯特·巴贝奇(Henry Prevost Babbage,1824–1918)
\item 亚历山大·福布斯·巴贝奇(Alexander Forbes Babbage,1827–1827)
\end{itemize}
巴贝奇的最小的存活儿子亨利·普雷沃斯特·巴贝奇(1824–1918)根据父亲的设计,为差分机一号(Difference Engine No. 1)制作了六个小型示范模型,其中一个被送到了哈佛大学,并被哈佛马克I计算机的先驱霍华德·艾肯(Howard H. Aiken)发现。亨利·普雷沃斯特于1910年设计的解析机“米尔”(Mill)曾在达德马斯顿大厅展出,现在则在伦敦的科学博物馆展出。
\subsection{死亡}
巴贝奇在伦敦玛丽勒本的多塞特街1号住了超过40年,直到1871年10月18日去世,享年79岁。他被安葬在伦敦的肯萨尔绿地公墓。根据霍斯利的说法,巴贝奇死于“肾功能不全,继发于膀胱炎”。他曾拒绝接受爵士头衔和男爵封号,并且主张反对世袭贵族制度,倾向于终身贵族制度。
\subsubsection{尸检报告} 
1983年,查尔斯·巴贝奇的尸检报告被发现,并由他的曾曾孙后出版。原始报告的副本也可供查阅。巴贝奇的大脑一半被保存在伦敦皇家外科学院的亨特博物馆,另一半则陈列在伦敦的科学博物馆。
\begin{figure}[ht]
\centering
\includegraphics[width=6cm]{./figures/ec6310c6bdf20531.png}
\caption{查尔斯·巴贝奇的大脑陈列在伦敦的科学博物馆。} \label{fig_CRSBQ_15}
\end{figure}
\subsubsection{纪念碑}
有一块黑色纪念牌,纪念巴贝奇在伦敦多赛特街1号度过的40年时光。以巴贝奇命名的地点、机构和其他事物包括:
\begin{itemize}
\item 月球上的巴贝奇陨石坑[200]
\item 查尔斯·巴贝奇研究所,位于明尼苏达大学,是一个信息技术档案和研究中心[201]
\item 巴贝奇河瀑布,加拿大育空地区[202]
\item 查尔斯·巴贝奇奖,年度计算机奖项[203]
\item 英国铁路在1990年代以查尔斯·巴贝奇命名了一辆机车[204]
\item 巴贝奇岛,西澳大利亚[205]
\item 普利茅斯大学的巴贝奇大楼,该校计算机学院位于此[206]
\item 为GEC 4000系列小型计算机设计的巴贝奇编程语言[207]
\item 《经济学人》科学与技术博客中的“巴贝奇”[208]
\item 曾经的连锁零售计算机和视频游戏商店“巴贝奇”(现为GameStop)以他命名[209]
\end{itemize}
\subsection{在小说与电影中}
巴贝奇经常出现在蒸汽朋克作品中,并被称为这一类型的标志性人物。其他包含巴贝奇角色的作品包括:
\begin{itemize}
\item 2008年短片《巴贝奇》(Babbage),该片在2008年戛纳电影节展映,2009年成为海登电影节的决赛选片,并在2009年HollyShorts电影节及其他国际电影节上放映。影片展示了巴贝奇在一场晚宴上的情景,客人们讨论他的生活与工作。
\item 悉尼·帕杜亚(Sydney Padua)创作的《洛夫莱斯与巴贝奇的惊险冒险》(The Thrilling Adventures of Lovelace and Babbage),这是一部漫画形式的另类历史作品,讲述了巴贝奇和洛夫莱斯成功构建分析机的故事。该作品大量引用了洛夫莱斯、巴贝奇及其同时代人的著作。
\item 凯特·比顿(Kate Beaton),《Hark! A Vagrant》网络漫画的漫画家,曾创作过一篇专门描绘查尔斯和乔治亚娜·巴贝奇的漫画。
\item 《神秘博士》剧集《间谍突袭,第二部分》(Season 12, episode 2)中,查尔斯·巴贝奇和艾达·戈登作为角色出现在剧情中,协助博士解决困境,时间设定为1834年。
\end{itemize}
\subsection{出版物}
\begin{figure}[ht]
\centering
\includegraphics[width=6cm]{./figures/ee3ba06c2be7387e.png}
\caption{《关于M. Arago重复进行的实验,研究旋转过程中各种物质表现出的磁性》,1825年} \label{fig_CRSBQ_16}
\end{figure}
\begin{itemize}
\item Babbage, Charles (1825). Account of the repetition of M. Arago's experiments on the magnetism manifested by various substances during the act of rotation. 伦敦:William Nicol。
\item Babbage, Charles (1826). A Comparative View of the Various Institutions for the Assurance of Lives. 伦敦:J. Mawman。
\item Babbage, Charles (1830). Reflections on the Decline of Science in England, and on Some of Its Causes. 伦敦:B. Fellowes。
\item Abstract of a paper entitled Observations on the Temple of Serapis at Pozzuoli. 伦敦:Richard Taylor,1834。
\item Babbage, Charles (1835). On the Economy of Machinery and Manufactures(第4版)。伦敦:Charles Knight。
\item Babbage, Charles (1837). The Ninth Bridgewater Treatise, a Fragment. 伦敦:John Murray。 (2009年由剑桥大学出版社重印,ISBN 978-1-108-00000-0。)
\item Babbage, Charles (1841). Table of the Logarithms of the Natural Numbers from 1 to 108000. 伦敦:William Clowes and Sons。 (LOCOMAT网站包含该表格的重建版。)
\item Babbage, Charles (1851). The Exposition of 1851. 伦敦:John Murray。
\item Laws of mechanical notation. 1851。
\item Babbage, Charles (1864). Passages from the Life of a Philosopher. 伦敦:Longman。
\item Babbage, Charles (1989). Hyman, Anthony (编)。 Science and Reform: Selected Works of Charles Babbage. 剑桥大学出版社,ISBN 978-0-521-34311-4。
\item abbage, Charles (1989) [1815]. Charles Babbage's Lectures On Astronomy. 伦敦。[217]
\end{itemize}
\subsection{参见}
\begin{itemize}
\item 巴贝奇的同余  
\item IEEE计算机学会查尔斯·巴贝奇奖  
\item 计算机科学先驱名单
\end{itemize}
\subsection{注释}
\begin{enumerate}
\item Whalen, Terence (1999). *Edgar Allan Poe and the Masses: The Political Economy of Literature in Antebellum America*. Princeton University Press. p. 254. ISBN 978-0-691-00199-9. Retrieved 18 April 2013.
\item Copeland, B. Jack (18 December 2000). "The Modern History of Computing". *The Modern History of Computing* (Stanford Encyclopedia of Philosophy). Stanford Encyclopedia of Philosophy. Metaphysics Research Lab, Stanford University. Retrieved 1 March 2017.
\item Halacy, Daniel Stephen (1970). *Charles Babbage, Father of the Computer*. Crowell-Collier Press. ISBN 978-0-02-741370-0.
\item "Charles Babbage Institute: Who Was Charles Babbage?". cbi.umn.edu.
\item Swade, Doron (2002). *The Difference Engine: Charles Babbage and the Quest to Build the First Computer*. Penguin. ISBN 9780142001448.
\item Newman, M.H.A. (1948). "General Principles of the Design of All-Purpose Computing Machines". *Proceedings of the Royal Society of London, Series A*, 195 (1042): 271–274. Bibcode:1948RSPSA.195..271N. doi:10.1098/rspa.1948.0129. S2CID 64893534.
\item Goldstine, Herman H. (1972). *The Computer: from Pascal to von Neumann*. Princeton University Press. ISBN 0-691-02367-0.
\item Secord, James A. (2007). "How Scientific Conversation Became Shop Talk". *Transactions of the Royal Historical Society*, 17: 129–156. doi:10.1017/S0080440107000564. ISSN 0080-4401. S2CID 161438144.
\item Collier, Bruce; MacLachlan, James H. (1998). *Charles Babbage and the Engines of Perfection*. Oxford Portraits in Science. New York: Oxford University Press. ISBN 978-0-19-508997-4.
\item Swade, Doron. "Babbage, Charles". *Oxford Dictionary of National Biography* (online ed.). Oxford University Press. doi:10.1093/ref:odnb/962. (Subscription or UK public library membership required.)
\item 1140 Plaque # 1140 on Open Plaques
\item Hyman, Anthony (1985). *Charles Babbage: Pioneer of the Computer*. Princeton University Press. p. 5. ISBN 978-0-691-02377-9.
\item Moseley, Maboth (1964). *Irascible Genius, The Life of Charles Babbage*. Chicago: Henry Regnery. p. 29.
\item "The Late Mr. Charles Babbage, F.R.S". *The Times*. UK.
\item Members Constituencies Parliaments Surveys. "Praed, William (1747–1833), of Tyringham, Bucks. and Trevethoe, nr. St. Ives, Cornwall". *History of Parliament Online*. Retrieved 7 June 2014.
\item Moseley 1964, p. 39.
\item "Reverend Stephen Freeman's Schools Ponders End". 74 – *London Metropolitan Archives: City of London*. UK: The National Archives. Retrieved 9 August 2021.
\item Hyman, Anthony (1985). *Charles Babbage: Pioneer of the Computer*. Princeton University Press. p. 17. ISBN 978-0-691-02377-9.
\item Collier, Bruce; MacLachlan, James (2000). *Charles Babbage: And the Engines of Perfection*. Oxford University Press. p. 11. ISBN 978-0-19-514287-7.
\item "Babbage, Charles (BBG810C)". *A Cambridge Alumni Database*. University of Cambridge.
\item Leedham-Green, E. S. (1996). *A Concise History of the University of Cambridge*. Cambridge University Press. p. 142. ISBN 978-0-521-43978-7.
\item Wilkes (2002) p.355.
Hofstadter, Douglas R. (2000) [1979]. *Gödel, Escher, Bach: an Eternal Golden Braid*. Penguin Books. p. 726.
\item "Charles Babbage's Computer Engines". Archived from the original on 30 April 2013. Retrieved 13 March 2012.
\item Essinger, James (2007). *Jacquard's Web*. Oxford University Press. pp. 59, 98. ISBN 978-0-19-280578-2.
\item Flood, Raymond; Rice, Adrian; Wilson, Robin (2011). Mathematics in Victorian Britain. Oxford University Press. p. 145. ISBN 978-0-19-960139-4.
\item George Green: Mathematician and Physicist, 1793–1841: The Background to His Life and Work. SIAM. 2001. p. 255 note 19. ISBN 978-0-89871-463-0.
\item Hahn, Roger (2005).Pierre Simon Laplace: 1749–1827; a Determined Scientist. Harvard University Press. pp. 295 note 34. ISBN 978-0-674-01892-1. Retrieved 8 May 2013.
\item Panteki, Maria. "Wallace, William". *Oxford Dictionary of National Biography (online ed.). Oxford University Press. doi:10.1093/ref:odnb/28545. (Subscription or UK public library membership required.)
\item The Edinburgh magazine, and literary miscellany, a new series of The Scots magazine. 1819. p. 369.
\item Williams, L. Pearce (1965).Michael Faraday. Da Capo Press. pp. 170–172. ISBN 978-0-306-80299-7.
\item Brown, Robert. "Barrett, George". Oxford Dictionary of National Biography (online ed.). Oxford University Press. doi:10.1093/ref:odnb/1523. (Subscription or UK public library membership required.)
\item Collier, Bruce; MacLachlan, James (2000).Charles Babbage: And the Engines of Perfection. Oxford University Press. pp. 29–30. ISBN 978-0-19-514287-7.
\item Hyman, Anthony (1985).Charles Babbage: Pioneer of the Computer. Princeton University Press. p. 59. ISBN 978-0-691-02377-9.
\item James (25 February 2010).Remarkable Engineers. Cambridge University Press. p. 45. ISBN 978-1-139-48625-5. Retrieved 26 April 2013.
\item Knox, Kevin C. (6 November 2003).From Newton to Hawking: A History of Cambridge University's Lucasian Professors of Mathematics*. Cambridge University Press. pp. 242, 258–272. ISBN 978-0-521-66310-6. Retrieved 26 April 2013.
\item "Babbage biography".History.mcs.st-and.ac.uk. Retrieved 21 December 2017.
\item Campbell-Kelly, Martin (2003).The History of Mathematical Tables: From Sumer to Spreadsheets. Oxford University Press. p. 8. ISBN 978-0-19-850841-0.
\item Buxton, Harry Wilmot; Tomash, Erwin (1988).Memoir of the Life and Labours of the Late Charles Babbage Esq., F.R.S. Cambridge: MIT Press. p. 77. ISBN 978-0-262-02269-9.
\item Flood, Raymond; Rice, Adrian; Wilson, Robin (2011). Mathematics in Victorian Britain. Oxford University Press. p. 34. ISBN 978-0-19-162794-1.
\item Maas, Harro (2005).William Stanley Jevons and the Making of Modern Economics. Cambridge University Press. p. 201. ISBN 978-0-521-82712-6.
\item Wise, M. Norton (1997).The Values of Precision. Princeton University Press. p. 320. ISBN 978-0-691-01601-6.
\item Robson, Eleanor; Stedall, Jacqueline (18 December 2008).The Oxford Handbook of the History of Mathematics*. Oxford University Press. p. xxxiv. ISBN 978-0-19-160744-8. Retrieved 25 April 2013.
\item Hyman, Anthony (1985).Charles Babbage: Pioneer of the Computer. Princeton University Press. p. 115. ISBN 978-0-691-02377-9.
\item Hyman, Anthony (1985).Charles Babbage: Pioneer of the Computer. Princeton University Press. p. 45. ISBN 978-0-691-02377-9.
\item Proceedings.Royal Institution of Great Britain. 1858. p. 518.
\item Constructing a Bridge: An Exploration of Engineering Culture, Design, and Research in Nineteenth-century France and America. MIT Press. 1997. p. 110. ISBN 978-0-262-11217-8.
\item Guicciardini, Niccolò (2003).The Development of Newtonian Calculus in Britain, 1700–1800. Cambridge University Press. p. 138. ISBN 978-0-521-52484-1.
\item Dov M. Gabbay; John Woods (2008).British Logic in the Nineteenth Century*. Elsevier. pp. 403–404. ISBN 978-0-08-055701-4.
\item Craik 2005, pp. 122–3.
\item Gray, Jeremy J.; Hunger Parshall, Karen (2011).Episodes in the History of Modern Algebra (1800–1950). American Mathematical Society. p. 19. ISBN 978-0-8218-7257-4.
\item "Book of Members, 1780–2010: Chapter B" (PDF).American Academy of Arts and Sciences*. Archived (PDF) from the original on 18 June 2006. Retrieved 28 April 2011.
\item George Biddell Airy#Biography
\item Kevin C. Knox (6 November 2003).From Newton to Hawking: A History of Cambridge University's Lucasian Professors of Mathematics*. Cambridge University Press. pp. 284–285. ISBN 978-0-521-66310-6. Retrieved 19 April 2013.
\item Gavin Budge et al. (editors),The Dictionary of Nineteenth-Century British Philosophers (2002), Thoemmes Press (two volumes), article Babbage, Charles, p. 35.
\item Kevin C. Knox (6 November 2003).From Newton to Hawking: A History of Cambridge University's Lucasian Professors of Mathematics*. Cambridge University Press. p. 243. ISBN 978-0-521-66310-6. Retrieved 25 April 2013.
\item Members Constituencies Parliaments Surveys."historyofparliamentonline.org, Middlesex County 1820–1832". Historyofparliamentonline.org. Retrieved 7 June 2014.
\item Sylvanus Urban (1838).The Gentleman's Magazine. p. 659.
\item Charles Babbage (1864).Passages from the Life of a Philosopher. Longman, Green, Longman, Roberts, & Green. pp. 190–191. Retrieved 1 May 2013.
\item Crowther, J. G. (1968).Scientific Types. London: Barrie & Rockliff. p. 266. ISBN 978-0-248-99729-4.
\item Crowther, J. G. (1968).Scientific Types*. London: Barrie & Rockliff. p. 266. ISBN 978-0-248-99729-4.
\item Hyman, Anthony (1985).Charles Babbage: Pioneer of the Computer*. Princeton, New Jersey: Princeton University Press. pp. 82–87. ISBN 978-0-691-08303-2.
\item Moseley 1964, pp. 120–121 – Note some confusion as to the dates.
\item Shaw, William Arthur (1906).The Knights of England: A Complete Record from the Earliest Time to the Present Day of the Knights of All the Orders of Chivalry in England, Scotland, and Ireland, and of Knights Bachelors, Incorporating a Complete List of Knights Bachelors Dubbed in Ireland*. vol. 2, London: Sherratt and Hughes.
\item Anthony Hyman (1985).Charles Babbage: Pioneer of the Computer. Princeton University Press. p. 88. ISBN 978-0-691-02377-9.
\item James (25 February 2010).Remarkable Engineers. Cambridge University Press. p. 47. ISBN 978-1-139-48625-5. Retrieved 27 April 2013.
\item Richard Holmes (2008).The Age of Wonder. Pantheon Books. pp. 437–440. ISBN 978-0-375-42222-5.
\item The Mechanics' Magazine, Museum, Register, Journal, and Gazette. M. Salmon. 1831. pp. 373–374.
\item Mary Poovey (1998).A History of the Modern Fact: Problems of Knowledge in the Sciences of Wealth and Society. University of Chicago Press. p. 309. ISBN 978-0-226-67526-8.
\item Dov M. Gabbay; John Woods (2008).British Logic in the Nineteenth Century*. Elsevier. p. 164. ISBN 978-0-08-055701-4.
\item Patricia Jones (1979).Population Malthus. Routledge and Kegan Paul. p. 445. ISBN 978-0-7100-0266-2.
\item James P. Henderson (1996).Early Mathematical Economics: William Whewell and the British Case. Rowman & Littlefield. p. 33. ISBN 978-0-8476-8201-0.
\item Michael Lingren (1990).Glory and Failure: The Difference Engines of Johann Müller, Charles Babbage and Georg and Edvard Scheutz*. MIT Press. p. 54. ISBN 978-0-262-12146-0.
\item Gavin Budge et al. (editors).The Dictionary of Nineteenth-Century British Philosophers (2002), Thoemmes Press (two volumes), article Babbage, Charles, p. 39.
\item Charles Manby (1846).Minutes of the Proceedings of the Institution of Civil Engineers. p. 57.
\item M. Lucertini; Ana Millán Gasca; Fernando Nicolò (2004).Technological Concepts and Mathematical Models in the Evolution of Modern Engineering Systems. Springer. p. 30. ISBN 978-3-7643-6940-8.
\item Mauro F. Guillén (15 October 1994).Models of Management: Work, Authority, and Organization in a Comparative Perspective. University of Chicago Press. p. 207. ISBN 978-0-226-31036-7. Retrieved 18 April 2013.
\item Maxine Berg (1982).The Machinery Question and the Making of Political Economy 1815–1848. CUP Archive. p. 181. ISBN 978-0-521-28759-3.
\item Bruce E. Kaufman (2004).The Global Evolution of Industrial Relations: Events, Ideas and the IIRA. Academic Foundation. p. 66. ISBN 978-81-7188-544-2.
\item Julio Segura; Carlos Rodríguez Braun (2004).An Eponymous Dictionary of Economics: A Guide to Laws and Theorems Named After Economists. Edward Elgar Publishing. p. 13. ISBN 978-1-84542-360-5. Retrieved 18 April 2013.
\item Work.Polity. 2011. p. 39. ISBN 978-0-7456-4678-7.
\item P. Sargant Florence (1953).The Logic of British and American Industry: A Realistic Analysis of Economic Structure and Government*. Routledge. p. 51. ISBN 978-0-415-31350-6.
\item The Theory of the Growth of the Firm [Electronic book]. Oxford University Press. 1995. p. 68. ISBN 978-0-19-828977-7.
\item Giddens, Anthony; Held, David (1982).Classes, Power and Conflict: Classical and Contemporary Debates. University of California Press. p. 155. ISBN 978-0-520-04627-6. Retrieved 18 April 2013.
\item Guang-Zhen Sun (2005).Readings in the Economics of the Division of Labor: The Classical Tradition. World Scientific. p. 10. ISBN 978-981-270-127-5.
\item Pere Mir Artigues; Josep González Calvet (2007).Funds, Flows and Time: An Alternative Approach to the Microeconomic Analysis of Productive Activities*. Springer. p. 72 note 15. ISBN 978-3-540-71291-6.
\item James A. Secord (2000).Victorian Sensation: The Extraordinary Publication, Reception, and Secret Authorship of Vestiges of the Natural History of Creation. University of Chicago Press. p. 52. ISBN 978-0-226-74410-0.
\item Mary Poovey (1995).Making a Social Body: British Cultural Formation, 1830–1864. University of Chicago Press. p. 193 note 43. ISBN 978-0-226-67524-4.
\item Rosemary Ashton (2000).G. H. Lewes: An Unconventional Victorian*. Pimlico. p. 128. ISBN 978-0-7126-6689-3.
\item Ashworth, William J. (1996). "Memory, Efficiency, and Symbolic Analysis: Charles Babbage, John Herschel, and the Industrial Mind." Isis. 87 (4): 629–653. doi:10.1086/357650. JSTOR 235196. S2CID 143404822.
\item Peter H. Hoffenberg (2001).An Empire on Display: English, Indian, and Australian Exhibitions from the Crystal Palace to the Great War. University of California Press. p. 178. ISBN 978-0-520-21891-8.
\item Maxine Berg (1982).The Machinery Question and the Making of Political Economy 1815–1848. CUP Archive. p. 127. ISBN 978-0-521-28759-3.
\item Ákos Róna-Tas (1997).The Great Surprise of the Small Transformation: The Demise of Communism and the Rise of the Private Sector of Hungary*. University of Michigan Press. p. 20. ISBN 978-0-472-10795-7.
\item Ben Fine; Alfredo Saad-Filho; Marco Boffo (2012).The Elgar Companion to Marxist Economics*. Edward Elgar Publishing. p. 190. ISBN 978-1-78100-122-6.
\item Sympathy of Things: Ruskin and the Ecology of Design. V2 Publishing. 2011. pp. 52–53. ISBN 978-90-5662-827-7.
\item Patrice L. R. Higonnet; David Saul Landes; Rosovsky Henry (1991).Favorites Or Fortune: Technology, Growth, and Economic Development Since the Industrial Revolution*. Harvard University Press. p. 401. ISBN 978-0-674-29520-9.
\item George Jacob Holyoake,The History of Co-operation in England: its literature and its advocates, vol. 2 (1879), pp. 228–232; archive.org.
\item Thierry Pillon; François Vatin (2003).Traité de sociologie du travail. Octarès. p. 164.
\item Anthony Hyman (1985).Charles Babbage: Pioneer of the Computer*. Princeton University Press. p. 122. ISBN 978-0-691-02377-9.
\item François Vatin (May 2007).Morale industrielle et calcul économique dans le premier XIXème siècle: L'économie industrielle de Claude-Lucien Bergery (1787–1863) (in French). Editions L'Harmattan. p. 123. ISBN 978-2-296-17103-9.
\item Philip Mirowski (1994).Natural Images in Economic Thought: Markets Read in Tooth and Claw. Cambridge University Press. p. 206. ISBN 978-0-521-47884-7.
\item Yeheskel Hasenfeld; Andrew Delano Abbott (1992).Human Services as Complex Organizations*. SAGE Publications. p. 211. ISBN 978-0-8039-4065-9.
\item Boole, Mary Everest (1931)."Indian Thought and Western Science in the Nineteenth Century." In Cobham, E.M.; Dummer, E.S. (eds.). Boole, Mary Everest "Collected Works". London: Daniel. pp. 947–967.
\item Genesis and Geology: A Study in the Relations of Scientific Thought, Natural Theology, and Social Opinion in Great Britain, 1790–1850. Harvard University Press. 1996. p. 247. ISBN 978-0-674-34481-5.
\item John G. Trapani (2004).Truth Matters: Essays in Honor of Jacques Maritain. CUA Press. p. 109. ISBN 978-0-9669226-6-0.
\item Note I, in darwin-online.org.uk, Babbage, Charles. 1838. The Ninth Bridgewater Treatise*. 2d edn. London: John Murray.
\item Babbage, Ninth Bridgewater Treatise. Chapter V. *Further View of the Same Subject*. The Victorian Web.
\item James A. Secord (2000).Victorian Sensation: The Extraordinary Publication, Reception, and Secret Authorship of Vestiges of the Natural History of Creation*. University of Chicago Press. p. 107. ISBN 978-0-226-74410-0.
\item Robert Chambers (15 August 1994).Vestiges of the Natural History of Creation and Other Evolutionary Writings. University of Chicago Press. pp. xvi–xvii. ISBN 978-0-226-10073-9. Retrieved 19 April 2013.
\item Jonar Ganeri (2001).Indian Logic: A Reader Routledge. p. 7. ISBN 978-0-7007-1329-5.
\item Passages from the Life of a Philosopher (1864). p. 8.
\item Babbage 1864 
\item Passages from the Life of a Philosopher (1864), pp. 404–405  
\item Passages from the Life of a Philosopher (1864), p. 396  
\item Smithsonian Institution. (1846). Annual report of the Board of Regents of the Smithsonian Institution. Washington Government Printing Office. Smithsonian Institution  
\item Passages from the Life of a Philosopher (1864), The Athanasian Creed, p. 403:  
\item "In the course of my inquiries, I met with the work upon the Trinity, by Dr. Samuel Clarke. This I carefully examined, and although very far from being satisfied, I ceased from further inquiry. This change arose probably from my having acquired the much more valuable work of the same author, on the Being and Attributes of God. This I studied, and felt that its doctrine was much more intelligible and satisfactory than that of the former work. I may now state, as the result of a long life spent in studying the works of the Creator, that I am satisfied they afford far more satisfactory and more convincing proofs of the existence of a supreme Being than any evidence transmitted through human testimony can possibly supply."  
\item Passages from the Life of a Philosopher (1864), pp. 396–402  
Pickover, Clifford A. (2009). *The Math Book: From Pythagoras to the 57th Dimension, 250 Milestones in the History of Mathematics.
\item Sterling Publishing Company, Inc. p. 218  
\item Babbage, Charles. (1864) Passages from the Life of a Philosopher, Appendix: "Miracles", p. 488  
\item Babbage, Charles. (1864) *Passages from the Life of a Philosopher*, Appendix: "Miracles", pp. 487–488  
\item Hook, Diana H.; Jeremy M. Norman; Michael R. Williams (2002). *Origins of Cyberspace: A Library on the History of Computing, Networking, and Telecommunications*. Norman Publishing. pp. 161, 165. ISBN 978-0-930405-85-4.  
\item Andrew Ede; Lesley B. Cormack (2012). *A History of Science in Society: From the Scientific Revolution to the Present*. University of Toronto Press. p. 224. ISBN 978-1-4426-0452-0.  
\item Daniel Brown (2013). *The Poetry of Victorian Scientists: Style, Science, and Nonsense*. Cambridge University Press. p. 6. ISBN 978-1-107-02337-6.  
\item Discovering Water: James Watt, Henry Cavendish, and the Nineteenth Century 'Water Controversy'. Ashgate Publishing, Ltd. 2004. pp. 136–137. ISBN 978-0-7546-3177-4.  
\item Iwan Rhys Morus (2002). Bodies/Machines. Berg. p. 44. ISBN 978-1-85973-695-1.  
\item Kevin C. Knox (6 November 2003). *From Newton to Hawking: A History of Cambridge University's Lucasian Professors of Mathematics*. Cambridge University Press. p. 287. ISBN 978-0-521-66310-6. Retrieved 26 April 2013.  
\item Mario Biagioli (1999). The Science Studies Reader. Routledge, Chapman & Hall, Incorporated. pp. 459–460. ISBN 978-0-415-91868-8.  
Ian Hacking (31 August 1990). The Taming of Chance. Cambridge University Press. pp. 55–60. ISBN 978-0-521-38884-9. Retrieved 1 May 2013.  
\item Report of the Board of Regents. The Institution. 1857. pp. 289–302.  
\item David F. Noble (1979). America by Design: Science, Technology, and the Rise of Corporate Capitalism. Oxford University Press. p. 71. ISBN 978-0-19-502618-4.  
\item Bruce Collier; James MacLachlan (2000). Charles Babbage: And the Engines of Perfection. Oxford University Press. pp. 28–29. ISBN 978-0-19-514287-7.
\item L. T. C. Rolt (1982).Isambard Kingdom Brunel.* Penguin Books. p. 74. ISBN 978-0-14-021195-5.  
\item Babbage 1864, pp. 317–318  
\item Stigler, George J. (1991). "Charles Babbage (1791 + 200 = 1991)." *Journal of Economic Literature*. 29 (3): 1149–1152. JSTOR 2727614.  
\item Lee, John A. N. (1995). International Biographical Dictionary of Computer Pioneers. Taylor & Francis US. p. 60.  
\item "Babbage, Benjamin Herschel." Bright Sparcs Biographical entry. Retrieved 15 May 2008.  
\item Bynum, W. F. "Jones, Thomas Wharton." *Oxford Dictionary of National Biography (online ed.). Oxford University Press. doi:10.1093/ref:odnb/37617. (Subscription or UK public library membership required.)  
\item "Medical Discoveries, Ophthalmoscope." Discoveriesinmedicine.com. Retrieved 29 January 2009.  
\item Franksen, OLE Immanuel; Rushton, RAY (November 1985). "General Notes." *Journal of the Royal Society of Arts. 133 (5352): 880–883. JSTOR 41374062.  
\item Introduction to Cryptography: With Mathematical Foundations in Computer Implementations. CRC Press. 2011. p. 187. ISBN 978-1-4398-1763-6.  
\item Kahn, David L. (1996). The Codebreakers: The Story of Secret Writing. New York: Scribner. ISBN 978-0-684-83130-5.
\item 请参见:
\begin{itemize}
\item J. H. B. Thwaites (1854年8月11日) 《秘密或密码写作》,《艺术学会期刊》,2 (90): 663–664。
\item "C."(查尔斯·巴贝奇)(1854年9月1日),《Thwaites先生的密码》,《艺术学会期刊》,2 (93): 707–708。
\item Thwaites, John H. B.(1854年9月15日)《秘密或密码写作》,《艺术学会期刊》,2 (95): 732–733。
\item "C"(查尔斯·巴贝奇)(1854年10月6日)《Thwaites先生的密码》,《艺术学会期刊》,2 (98): 776–777。
\item 查尔斯·巴贝奇,《哲学家生平片段》(伦敦,英国:朗文出版社,1864年),第496页。
\end{itemize}
\item 霍普尔,乔治·N;巴贝奇,查尔斯;克拉克,海德;韦伯斯特,托马斯;边塞姆,M·S;古德,S·A(1855年)。“《艺术学会期刊》第4卷,第159期”。《艺术学会期刊》,4(159):29–44。JSTOR 41334443。
\item 另见:奥尔·伊曼纽尔·弗兰克森,《巴贝奇先生的秘密:一个密码的故事和APL》(丹麦比尔克雷,斯特兰伯格出版社,1984年;由普伦蒂斯-霍尔公司(新泽西州恩格尔伍德,美国)1985年再版)。
\item 巴贝奇,查尔斯(1857年)。“《破碎玻璃窗原因的相对频率表》”。《力学杂志》,66:82。
\item 巴贝奇,查尔斯(1989年)。马丁·坎贝尔-凯利(编)。《查尔斯·巴贝奇文集》第五卷。伦敦:威廉·皮克林出版社,第137页。ISBN 978-1-85196-005-7。
\item 沃尔福德,科内留斯(1878年)。《保险百科全书》。C.和E. 莱顿,第417页。于2011年2月22日查阅。
\item 坎贝尔-凯利,马丁;巴贝奇,查尔斯(1994年)。“第26章”。《哲学家生平片段》。皮克林与查托出版社,第342页。ISBN 978-1-85196-040-8。
\item 凯伦·蔡斯;迈克尔·莱文森(2009年8月15日)。《亲密的景象:维多利亚家庭的公共生活》。普林斯顿大学出版社,第151页。ISBN 978-1-4008-3112-8。于2013年4月22日查阅。
\item 巴贝奇,查尔斯(1864年),第360页。
\item 汉萨德议会辩论。第三系列,从威廉四世即位开始。27° & 28° 维多利亚,1864年。第176卷。涵盖1864年6月21日至1864年7月29日的时期。议会,托马斯·卡尔森·汉萨德。“街头音乐(大都会)法案”;第4卷,第471页[1]。
\item 斯威德,多龙(2000年)。《齿轮大脑》。伦敦:小布朗公司,第186页。ISBN 978-0-316-64847-9。
\item 格里克,J.(2011年)。《信息:一部历史,一种理论,一场洪流》。伦敦:第四出版社,第104页。
\item B. V. 博登,《比思考更快》,皮特曼出版社(1953年),第8页。
\item 马丁·坎贝尔-凯利(2003年)。《数学表格的历史:从苏美尔到电子表格》。牛津大学出版社,第110页。ISBN 978-0-19-850841-0。
\item 查尔斯·库尔斯顿·吉利斯皮(2009年)。《法国的科学与政治:旧制度的终结》。普林斯顿大学出版社,第485页。ISBN 978-1-4008-2461-8。
\item 《差分机 | 计算机》。《大英百科全书》。于2019年5月13日查阅。
\item 麦康奈尔,安妮塔。“克莱门特,约瑟夫”。《牛津国立传记词典》(在线版)。牛津大学出版社。doi:10.1093/ref:odnb/37291。(需要订阅或英国公共图书馆会员资格)。
\item 罗格尔,丹尼斯(2009年4–6月)。《巴贝奇第一差分机的原型碎片》。IEEE《计算机历史年鉴》,31(2):70–75。doi:10.1109/MAHC.2009.31。S2CID 45564453。
\item “科学技术 | 巴贝奇打印机终于运行”。BBC新闻。2000年4月13日。于2012年4月27日查阅。
\item “概览 – 巴贝奇引擎”。计算机历史博物馆。于2009年1月29日查阅。
\item 谢尔斯,玛吉(2008年5月10日)。“维多利亚时代的‘超级计算机’重生”。BBC新闻。于2008年5月11日查阅。
\item 布鲁斯·科利尔;詹姆斯·麦克拉克兰(2000年)。《查尔斯·巴贝奇与完美的引擎》。牛津大学出版社,第65页。ISBN 978-0-19-514287-7。
\item “巴贝奇引擎”。计算机历史博物馆。于2013年3月6日查阅。
\item 塞鲁齐,保罗(2012年)。《计算:简明历史》。美国:史密森学会,第7–8页。ISBN 978-0-262-51767-6。
\item 格罗斯,本杰明(2015年秋季)。“法国联系”。《蒸馏杂志》,1(3):10–13。于2018年3月22日查阅。
\item 罗哈斯,R.(2024年)。“查尔斯·巴贝奇的计算机程序”。IEEE《计算机历史年鉴》,43(1):6–18。doi:10.1109/MAHC.2020.3045717。
\item 罗哈斯,R.(2024年)。“第一个计算机程序”。《ACM通信》,67(6):78–81。doi:10.1145/3624731。
\item 罗宾·哈默曼,安德鲁·L·拉塞尔(2016年)。《阿达的遗产:从维多利亚时代到数字时代的计算文化》。计算机协会与摩根·克莱普出版社。ISBN 978-1-970001-51-8。
\item 布朗利,艾伦·G(1990年)。“差分机与分析机”(PDF)。在阿斯普雷,威廉(编)。《计算机之前的计算》。艾姆斯:爱荷华州立大学出版社,第59–98页。ISBN 978-0-8138-0047-9。于2022年10月10日从原文存档(PDF),第89页。
\item 斯坦因,多萝西·K(1984年)。“洛夫莱斯夫人的笔记:技术文本与文化背景”。《维多利亚研究》,28(1):33–67,第34页。
\item 科利尔,布鲁斯(1970年)。《可能成为的那些小引擎:查尔斯·巴贝奇的计算机器》(博士论文)。哈佛大学。于2015年12月18日查阅。第3章。
\item 弗尤吉,J,弗朗西斯·J(2003年10–12月)。“洛夫莱斯与巴贝奇及1843年‘笔记’的创造”。《计算历史年鉴》,25(4):16–26。doi:10.1109/MAHC.2003.1253887。见第19页,第25页。
\item “查尔斯·巴贝奇 | 传记与事实”。《大英百科全书》。于2017年12月21日查阅。
\item 《连线》。 “查尔斯·巴贝奇在1840年在都灵留下了一个计算机程序,今天它终于出现了”。cacm.acm.org。于2021年9月24日查阅。
\item 斯特林,布鲁斯。“查尔斯·巴贝奇在1840年在都灵留下了一个计算机程序,今天它终于出现了”。《连线》。ISSN 1059-1028。于2021年9月24日查阅。
\item 本杰明·伍利(1999年)。《科学的新娘》。麦克米伦,第258–260页,第338页。ISBN 978-0-333-72436-1。
\item 迈克尔·林格伦(1990年)。《荣耀与失败:约翰·穆勒、查尔斯·巴贝奇和乔治及爱德华·谢茨的差分机》。麻省理工学院出版社,第98–112页。ISBN 978-0-262-12146-0。
\item I. 格拉特恩-吉尼斯(2003年)。《数学科学历史与哲学伴随百科全书》。约翰·霍普金斯大学出版社,第698页。ISBN 978-0-8018-7396-6。
\item “Plan 28博客”。Blog.plan28.org。于2022年6月10日查阅。
\item 马克夫,约翰(2011年11月7日)。“它开始转动数字齿轮”。《纽约时报》。于2011年11月10日查阅。
\item “‘巴贝奇分析机设计将被数字化’”。BBC新闻。2011年9月21日。于2012年3月19日查阅。
\item “电子时报:微型机器适合太空”。1999年10月11日。于2007年10月13日从原文存档。于2009年1月29日查阅 – 通过Find Articles。
\item “巴贝奇的最后一笑”。《经济学人》。1999年9月9日。
\item “最新的德文新闻”。德文直播。于2014年8月10日从原文存档。于2015年12月5日查阅。
\item “介绍新设计的英国护照”(PDF)。英国政府。2016年11月7日。
\item “杜德马斯顿大厅的景点信息”。VisitBritain。2015年10月8日。于2009年1月29日查阅。
\item 约翰·迈克尔·杜比(2004年2月12日)。《查尔斯·巴贝奇的数学工作》。剑桥大学出版社,第6页。ISBN 978-0-521-52476-6。
\item 瓦莱丽·巴维奇-理查德森。“巴贝奇家族树 2005”。于2007年10月13日从原文存档。于2013年2月9日查阅。  
另见“查尔斯·巴贝奇条目”。ClanBarker。于2013年2月9日查阅。
\item “讣告”。《绅士杂志》。1834年。于2021年10月19日查阅。
\item “亨利·普雷沃斯·巴贝奇 - 巴贝奇引擎”。计算机历史博物馆。于2009年1月29日查阅
\item “亨利·巴贝奇的分析机铣床,1910年”。科学博物馆。2007年1月16日。于2009年1月29日查阅。
\item 霍斯利,维克多(1909年)。“查尔斯·巴贝奇先生的大脑描述,F.R.S”。《皇家学会哲学学报 B》。200(262–273):117–132。Bibcode:1909RSPTB.200..117H。doi:10.1098/rstb.1909.0003。需要订阅。
\item 伊恩·詹姆斯(2010年)。《杰出的工程师:从里凯到香农》。剑桥大学出版社,第50页。ISBN 978-1-139-48625-5。
\item 巴贝奇,内维尔(1991年6月)。“查尔斯·巴贝奇(‘计算机之父’)尸体的尸检报告”。《澳大利亚医学杂志》。154(11):758–759。doi:10.5694/j.1326-5377.1991.tb121318.x。PMID 2046574。S2CID 37539400。
\item 威廉姆斯,迈克尔·R(1998年)。“关于查尔斯·巴贝奇的‘最后言论’”。IEEE《计算历史年鉴》。20(4):10–14。doi:10.1109/85.728225。需要订阅。
\item “约翰·格雷戈里·史密斯,F.R.C.S.(解剖学家)尸检报告”。Science and Society.co.UK。于2009年1月29日查阅。
\item “巴贝奇的大脑”。DanYEY.co.uk。于2009年1月29日查阅。
\item “参观博物馆,展厅,计算机,概述”。科学博物馆。于2010年9月20日从原文存档。于2010年10月25日查阅。
\item Open Plaques上的铭牌 #3061
\item “行星名称:巴贝奇”。行星命名公报。2010年10月18日。于2023年7月7日查阅。
\item “‘关于 CBI’”。cse.umn.edu。明尼苏达大学,科学与工程学院。于2023年7月7日查阅。
\item “‘巴贝奇河瀑布’,育空, 加拿大”。世界瀑布数据库。于2023年7月7日查阅。
\item “‘IEEE-CS,查尔斯·巴贝奇奖’”(PDF)。IPDPS。于2023年7月7日查阅。
\item “‘格雷厄姆·法里什 371-354’”。modelraildatabase.com。于2023年7月7日查阅。
\item “‘巴贝奇岛’,澳大利亚WA 6701”。city-facts.com。于2023年7月7日查阅。
\item “‘巴贝奇大楼:工程与设计的交汇’”。plymouth.ac.uk。于2023年7月7日查阅。
\item 伊林沃思,瓦莱丽(1986年)。“B.001 巴贝奇”。《计算词典》(第二版)。牛津大学出版社。ISBN 978-0-19-853913-1。
\item “《经济学人》,巴贝奇博客”。《经济学人》。于2014年6月7日查阅。
\item 费舍尔,丹(1990年5月6日)。“第87号,巴贝奇感到成长的痛苦”。《达拉斯时报先驱》。
\item 贾斯廷·爱德华兹;阿格尼丝卡·索尔蒂西克·蒙内特(2012年)。《当代文学与流行文化中的哥特:流行哥特》。劳特利奇,第157页。ISBN 978-0-415-80676-3。于2013年4月18日查阅。
\item “电影详情 - 巴贝奇”。英国电影目录。英国文化委员会。2010年。于2021年8月9日查阅。
\item 休斯,特雷弗(2010年12月10日)。“巴贝奇”。YouTube。于2021年11月24日从原文存档。于2021年8月9日查阅。
\item 霍洛维茨,丹尼尔(2008年9月17日)。“巴贝奇”。《天才解码:10部关于科技世界的电影》。Complex Networks。于2021年8月9日查阅。
\item 雷德尼斯,劳伦(2015年6月5日)。“‘洛夫雷斯与巴贝奇的惊险冒险’,悉尼·帕杜亚著”。《纽约时报》。于2022年1月1日从原文存档。于2015年12月2日查阅。
\item 帕杜亚,悉尼(2015年)。《洛夫雷斯与巴贝奇的惊险冒险》。Pantheon图画小说。ISBN 978-0-307-90827-8。
\item 比顿,凯特。“查尔斯与乔治安娜·巴贝奇”,《Hark! A Vagrant》。于2014年12月3日查阅。
\item C. J. D. 罗伯茨,编者。《查尔斯·巴贝奇的天文学讲座》
\end{enumerate}
\subsection{参考文献}  
\begin{itemize}
\item 克雷克,亚历克斯·D·D(2005年2月)。“Faà di Bruno公式的史前时期”。《美国数学月刊》。112(2):119–130。doi:10.2307/30037410。JSTOR 30037410。
\end{itemize}
\subsection{外部链接}
\begin{itemize}
\item 查尔斯·巴贝奇的作品电子书版  
\item 查尔斯·巴贝奇的作品,见《古腾堡计划》  
\item 查尔斯·巴贝奇的相关作品或资料,见互联网档案馆  
\item 查尔斯·巴贝奇的作品,见LibriVox(公共领域有声书)  
\item “与查尔斯·巴贝奇相关的档案材料”。英国国家档案馆。编辑此条目于Wikidata  
\item 《巴贝奇文献》:由科学博物馆图书馆和档案馆收藏的巴贝奇文献,主要涉及巴贝奇的自动计算引擎  
\item 《巴贝奇引擎》:计算机历史博物馆,美国加利福尼亚州山景城。多页巴贝奇及其引擎和同事的详细介绍,包括博物馆正在运行的差分机No.2的复制品视频  
\item 分析引擎博物馆:约翰·沃克(AutoCAD的名人)编制的关于巴贝奇机器的完整技术资料目录。  
\item 查尔斯·巴贝奇:圣安德鲁斯大学数学与统计学系的历史。  
\item 《查尔斯·巴贝奇先生》:来自《泰晤士报》的讣告(1871年)  
\item 《巴贝奇页面》  
\item 查尔斯·巴贝奇,宾夕法尼亚大学在线书目页面  
\item 《巴贝奇差分机》:其工作原理概述  
\item 《关于用符号表示机械运作的方法》,1826年。原版  
\item 查尔斯·巴贝奇研究所:关于“查尔斯·巴贝奇是谁?”的页面,包括传记资料、差分机No.2的描述、巴贝奇的出版物、关于巴贝奇的档案与出版资料、关于巴贝奇与阿达·洛芙莱斯的资料  
\item 《巴贝奇的芭蕾舞》,伊沃·格斯特,1997年《芭蕾舞杂志》  
\item 《巴贝奇的计算机》存档,2018年6月20日于Wayback Machine(1872年)– 来自琳达·霍尔图书馆的完整数字副本  
\item 作者资料在zbMATH数据库中  
\item 乔治·谢茨与爱德华·谢茨于1843年制作的“差分机”  
\item 查尔斯·巴贝奇的肖像,见伦敦国家肖像画廊。编辑此条目于Wikidata
\end{itemize}