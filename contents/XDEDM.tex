% 薛定谔的猫
% license CCBYSA3
% type Wiki

(本文根据 CC-BY-SA 协议转载自原搜狗科学百科对英文维基百科的翻译)

薛定谔的猫是一个思想实验,有时也称为薛定谔的猫悖论,由奥地利物理学家埃尔温·薛定谔(Erwin Schrödinger)在1935年提出;薛定谔试图用它说明将量子力学的哥本哈根诠释应用于日常宏观物体时会产生的问题,该试验场景中展示了一只假想的猫,它能够同时处于活着和死亡的状态,[1][2][3][4][5] 这种状态被称为量子叠加态,由于与一个具有随机性的亚原子事件相关联,这个事件可能发生也可能不发生。

薛定谔的猫的思想实验也经常出现在关于量子力学各种诠释的理论讨论中。薛定谔在描述该思想实验的过程中创造了“纠缠”这一术语。
\begin{figure}[ht]
\centering
\includegraphics[width=8cm]{./figures/f150cfef1150857a.png}
\caption{薛定谔的猫} \label{fig_XDEDM_1}
\end{figure}

\subsection{起源和动机}
\begin{figure}[ht]
\centering
\includegraphics[width=6cm]{./figures/69142a74244e614c.png}
\caption{图为苏黎士胡特大街9号的花园(薛定谔在1921-1926年间曾居住在这里)中一只按真实比例绘制的猫。根据光照条件的不同,猫也呈现活着或死亡的样子。} \label{fig_XDEDM_2}
\end{figure}
1935年,薛定谔提出他的思想实验作为对EPR佯谬论文(EPR佯谬以文章作者爱因斯坦(Einstein)、波多尔斯基(Podolsky)和罗森(Rosen)的名字首字母命名)的讨论。[6] EPR佯谬论文展示了量子叠加态的反直觉性质,量子叠加态指一个量子系统(如一个原子或光子)可作为多个量子态的组合而存在,且每个量子态对应不同的测量值。

根据当时占据主流地位的哥本哈根诠释,量子系统在与外部世界相互作用或被外部世界观察到之前一直处于叠加状态。当这种情况发生时,叠加会塌缩为其中一种确定的状态。EPR实验表明,一个由多个相距很远的粒子组成的系统可以处于这样的叠加状态。薛定谔和爱因斯坦互相通信讨论关于EPR佯谬论文的问题,在信件中,爱因斯坦指出,给定一桶不稳定的火药桶,经过一段时间后,该桶火药理论上会处于爆炸和未爆炸的叠加态。

为进一步说明这一现象,薛定谔描述了一种原则上可以宏观尺度上产生叠加态的方法——使该大尺度系统依赖于位于叠加态的量子粒子。他提出了一个场景,一只猫被锁在一个钢铁小屋中,猫的生死取决于一个具有放射性的原子是否发生衰变并产生辐射。薛定谔指出,根据哥本哈根诠释,这只猫在被观测前处于既活着又死了的状态。薛定谔并不认可会存在一只即死又活的猫,相反,他打算用这个例子来说明现有量子力学观点的的谬误之处。[7]

然而,自薛定谔时代以来,物理学家们提出了诸多其他关于量子力学的数学解释,其中一些理论认为一只处于“活与死”叠加态的猫是相当真实的。[7][7] 意在批判哥本哈根诠释(1935年的主流正统学说)的薛定谔的猫的思想实验,至今仍然是各种现代量子力学诠释理论的试金石。物理学家经常使用每种诠释处理薛定谔的猫的方式来说明和比较每种诠释的特点、优势和劣势。

\subsection{思想实验}
\begin{figure}[ht]
\centering
\includegraphics[width=10cm]{./figures/42dc72f52f290d5a.png}
\caption{薛定谔的猫:一只猫、一瓶毒药和一个放射源被放在一个密封的盒子里。如果内部监视器(如盖革计数器)检测到放射性(即一个原子衰变),烧瓶就会被打碎,释放出杀死猫的毒药。量子力学的哥本哈根诠释表明,经过一段时间后猫会处于活着又死亡的叠加态。然而当你看向盒子里的时候,猫不是活着就是死了,而不可能同时活着或死去。这就提出了一个问题:量子叠加态到底何时结束并塌缩到其中一种可能的状态。} \label{fig_XDEDM_3}
\end{figure}

薛定谔在最初提出思想实验的信中写道:[7][7]
甚至可以构造一个相当荒谬的案例,一只猫被关在一个钢制的小室里,同时小屋中配备以下装置(必须确保仪器远离猫的直接干扰):在盖革计数器里放置极少量的放射性物质,使其在一小时中有一个原子发生衰变的概率为50\%;当发生衰变时,计数管会放电,并通过继电器释放锤子,打碎一小瓶氢氰酸。把整个系统放置一小时,那么如果期间没有发生原子衰变,则猫仍然活着,而一次原子衰变则会触发氰化氢毒杀这只猫。由于放射性物质处于衰变与未衰变的叠加态,故而整个系统的波函数将会包含一个由活着和死亡两种状态等概率的叠加的既活又死猫的状态(原谅这种说法)。

类似这样的典型案例将局限于原子领域的不确定性转变为宏观的不确定性,通过观测可以消除这种不确定。这使我们无法天真地接受采用这种“模糊模型”来表明现实物体的量子性。它本身的确不会包含任何不确定或矛盾的地方,然而图片因摇晃失焦的模糊和本身就是云雾弥漫的照片之间是有本质区别的。

薛定谔著名的思想实验提出了一个问题,“处于叠加态的量子系统何时脱离这种叠加态转而塌缩为其中一个状态?”(更严格地说,从何时起一个量子态不再是多个经典态的非平凡线性组合,而是开始有一个唯一的经典描述?) 如果猫还活着,那么它只会记得它活着的状态。但是,EPR实验的解释与标准的微观量子力学一致地要求宏观物体(如猫和笔记本)并不总是有唯一的经典描述。薛定谔的思想实验阐明了这一明显的悖论。我们的直觉告诉我们,没有观察者可以处于混合状态——然而,上述思想实验却表明猫可以处在生与死的叠加态。猫是否有必要成为观察者,或者猫需要另一个外部观察者使其能够存在于一个单一且定义明确的经典状态中?爱因斯坦觉得每一种选择都很荒谬,他对思想实验突出这些问题的能力印象深刻。在1950年写给薛定谔的一封信中,他写道:

只要一个人是诚实的,他就无法回避关于实体(可处于叠加态)的这种假设,当代物理学家中除了劳厄(Max von Laue)外唯有你注意到这一点的。大多数物理学家根本看不到他们在与现实玩着什么样的危险游戏,他们似乎认为现实是独立于实验的。然而,他们的解释被你的放射性原子+放大器+火药+猫的盒子系统以最优雅的方式驳斥了,盒子系统的波函数同时包含着完好无缺的和炸成碎片的猫。没有人会真正质疑猫的存在与否是独立于观察行为的。[8]

请注意,薛定谔的装置中没有提到火药,它使用盖革计数器作为放大器,使用氢氰酸毒药代替火药。火药是在15年前爱因斯坦给薛定谔的最初建议中提到的,爱因斯坦把它带到了现在的讨论中。

\subsection{对实验的解释}
自薛定谔时代以来,物理学家们提出了其他多种关于量子力学的诠释,这些诠释对薛定谔猫的思想实验中提出的叠加持续多长时间以及它们何时(或是否)塌缩的问题给出了不同的答案

\subsubsection{3.1 哥本哈根诠释}
哥本哈根诠释是一种被广泛接受的量子力学诠释。[9] 按照哥本哈根诠释,当被观察时,一个量子系统将不再处于叠加态,而是塌缩为其中任意一种状态。薛定谔的思想实验清楚的表明,在哥本哈根诠释中,测量或观察的概念并没有得到很好的定义。当采用哥本哈根诠释分析薛定谔的思想实验时会发现,当盒子关闭时,系统处于“衰变的原子核/死猫”和“未衰变的原子核/活猫”叠加态中,并且只有当盒子打开并进行观察时,波函数才会塌缩成这两种状态之一。

然而,与哥本哈根诠释密切相关的主要物理学家之一的尼尔斯·玻尔(Niels Bohr)从来不认为是观测者引起了波函数塌缩,因为他不认为波函数是物理上真实存在的,而只是一种统计工具;因此,薛定谔的猫对他来说并不是什么谜题。在有意识的观察者打开盒子之前,猫就已经处于死了或者活着的状态。[10] 仔细分析一个真实的实验过程会发现,在有意识的观察者对测量结果进行观察之前,测量本身(例如盖革计数器)就会使量子波函数发生塌缩,[11] 尽管关于实验设计的有效性还存在争议。[12] 这种当原子核发射的粒子击中探测器时“观察”就已经发生的观点可以发展成客观塌缩理论。按照这种观点,薛定谔思想实验中是探测器的“无意识观察”使波函数发生塌缩。相比之下,多世界理论从根本上否认曾经发生过塌缩。

\subsubsection{3.2 多世界诠释和一致性历史}
\begin{figure}[ht]
\centering
\includegraphics[width=10cm]{./figures/45484cafac3f5d4a.png}
\caption{根据多世界诠释理解量子力学“薛定谔猫”悖论的示意图。在这个诠释中,每个事件都是一个分支点。猫既活着也死了——不管盒子是否打开——但是“活着”和“死了”的猫在宇宙的不同分支,它们是同样真实的,但不能相互作用。} \label{fig_XDEDM_4}
\end{figure}

1957年,休·埃弗雷特(Hugh Everett)提出了量子力学的多世界诠释,它并不把观察视为一个特殊的过程。在多世界诠释中,猫的生和死的状态在盒子打开后仍然存在,但是彼此之间发生了退相干的。换句话说,当盒子打开时,观察者和猫分裂成两个分支:观察者看着盒中的死猫和观察者看着盒中的活猫。但是由于死态和活态是退相干的,它们之间无法发生有效的信息交流或相互作用。

当打开盒子时,观察者与猫纠缠在一起,于是形成了对应猫的生与死的“观察者状态”;每个观察状态都与猫纠缠或联系在一起,因此“猫状态的观察”和“猫的状态”彼此相互对应。量子退相干确保不同的结果间没有相互作用。同样的,量子退相干机制对于解释历史的一致性也很重要。在多世界诠释中,只有“死猫”或“活猫”的态可以成为一个一致性历史的一部分。退相干通常被认为避免了同时对多个叠加态进行观察。[13][14]

宇宙学家马克斯·泰格马克提出了薛定谔的猫实验的一个变体,称为量子自杀机器。它从猫的角度考察了薛定谔的猫实验,并认为通过使用这种方法,人们可能能够区分哥本哈根诠释和多世界诠释。

\subsubsection{3.3 系综诠释}
系综诠释指出叠加态不过是一个更大的统计系综的子系综。态矢量不适用于单个的猫实验,而仅适用于许多大量类似的猫实验的统计数据。这种诠释的支持者认为这使得薛定谔的猫悖论成为一个无关紧要的平凡问题。

这种诠释抛弃了单一量子物理系统有一个对应的数学描述的观点。

\subsubsection{3.4 关系诠释}
关系诠释认为人类实验者、猫或仪器之间,或生命体和非生命体之间没有本质区别;所有这些都是遵循相同的波函数演化规则的量子系统,都可以被认为是“观察者”。但是关系诠释允许不同的观察者根据他们所掌握的关于系统的信息,对同一系列事件给出不同的描述。[15] 猫可以被认为是实验装置的观察者;同时,实验者可以被认为是盒子系统(猫加上仪器)的一个观察者。在盒子被打开之前,猫,根据其活着或死亡的状态,拥有关于设备状态的信息(原子已经衰变或没有衰变);但是实验者并不掌握这些信息。这样,在同一时刻两位观察者对盒子系统的状态有不同的描述:对猫来说,仪器的波函数“塌缩”了;对实验者来说,盒子系统仍然处于叠加态。直到盒子被打开,两个观察者对所发生的事情都有了相同的信息,两个系统状态才会看起来“塌缩”成相同的确定结果,即猫处于活着的还是死亡的状态。

\subsubsection{3.5 交易诠释}
在交易诠释中,实验设备发射一个逆时间方向行进的超前波,超前波与源发射的顺时间方向行进的波相互作用形成驻波。这些波被认为是物理上真实的,而实验装置是它们的一个“观察者”。在交易诠释中,波函数的塌缩是“非时间性的”,发生在源和设备发出的波之间的整个交易过程中。猫从未处于叠加态。相反,猫在任何特定时间都只处于一种状态,不管人类实验者什么时候看盒子。这样交易诠释就解决了这个量子悖论。

\subsubsection{3.6 量子芝诺效应}
量子芝诺效应指通过高频率观测延缓量子从初始状态到其他状态的演化。

另一方面,存在可以加速量子系统演化的反芝诺效应。例如,如果你频繁地偷看猫盒子里的东西,你可能会延迟或者加速决定性选择的发生。芝诺效应和反芝诺效应都是真实存在并且已在真实的原子系统中观测到。被测量的量子系统必须与周围环境(在本例中是仪器、实验室等等)强耦合以便获得更准确的信息。在薛定谔的猫实验中当没有信息传递到盒子外部时,这种与环境的耦合被认为是一种准测量,一旦猫的健康信息传递到外部世界(通过窥视盒子),准测量就变成了测量。准测量像测量一样会导致量子芝诺效应。[16] 量子芝诺效应告诉我们,即使没有窥视盒子,猫的死亡也会因为环境而被延迟或加速。。

\subsubsection{3.7 客观塌缩理论}
根据客观塌缩理论,当达到客观物理阈值(时间、质量、温度、不可逆性等)时,叠加态就会自发地被破坏(与外部观察无关)。因此,在盒子被打开之前很久,猫应该已经稳定在一个确定的状态。这可以粗略地称为“猫观察自己”,或者“环境观察猫”。

客观塌缩理论需要对标准量子力学进行修改,以允许叠加态被时间演化过程破坏。

\subsection{应用和实验}
\begin{figure}[ht]
\centering
\includegraphics[width=10cm]{./figures/b83d6f8bbff5bc07.png}
\caption{薛定谔猫态量子叠加和退相干环境效应} \label{fig_XDEDM_5}
\end{figure}

薛定谔猫的思想实验是纯粹理论性的,其中涉及的实验装置并非实际建成。然而,很多涉及类似原理的实验已经取得成功,例如一些相对较大(相对量子物理标准而言)系统的叠加态已经实现。[17] 这些实验并没有表明与猫大小一样的物体可以处于重叠态,但是它们提升了存在“猫态”的系统的尺寸上限。在很多实验中,这种状态只能短暂维持,即使冷却到接近绝对零度。
\begin{itemize}
\item 多光子的“猫态”已经实现。[18]
\item 观测到处于叠加态的被捕获的铍离子。[19]
\item 一项涉及超导量子干涉仪(SQUID)的实验薛定谔的思想实验的主题联系在一起:“叠加态并不是说十亿个电子正向流动,十亿个电子反向流动。超导电子总是沿同一方向移动。当超导量子干涉仪中的所有超导电子处于薛定谔猫态时,它们会同时在回路中双向流动。”[20]
\item 一种压电“音叉”已经被制造出来,它可以被置于振动和非振动状态的叠加态。谐振器包含大约10万亿个原子。[21]
\item 一项旨在制备叠加态流感病毒的实验已经被提出。[22]
\item 一项试图通过机电振荡器制造处于“猫态”的细菌的实验被提出。[23]
\end{itemize}
在量子计算中,“猫态”有时指的是GHZ态(Greenberg-Horne-Zeilinger态),其中若干量子比特处于全为$0$和全为$1$两种态的相等叠加态,可表示为:
$$|\psi\rangle = \frac{1}{\sqrt{2}} \left( |00 \ldots 0\rangle + |11 \ldots 1\rangle \right)~.$$

\subsection{扩展内容}
“维格纳的朋友”是薛定谔的实验的一个变体,其包含两个人类观察者:第一个观察者通过观察确定是否看到闪光,然后将他的观察结果传达给第二个观察者。这里的问题是,波函数何时“塌缩”?是在第一个观察者观察实验时“塌缩”,还是只在当第二个观察者被告知第一个观察者的观察结果时“塌缩”?

另一方面,一些著名的物理学家甚至认为,通过一种伪薛定谔猫的效应,天文学家们在1998年对宇宙暗能量的观测可能会“缩短宇宙的预期寿命”,尽管这是一个非常有争议的观点。[24][25]

\subsection{参考文献}
[1]
^Moring, Gary (2001). The Complete Idiot's Guide to Theories of the Universe. Penguin. pp. 192–193. ISBN 1440695725..

[2]
^Gribbin, John (2011). In Search of Schrodinger's Cat: Quantum Physics And Reality. Random House Publishing Group. p. 234. ISBN 0307790444. Archived from the original on 2015-05-17..

[3]
^Greenstein, George; Zajonc, Arthur (2006). The Quantum Challenge: Modern Research on the Foundations of Quantum Mechanics. Jones & Bartlett Learning. p. 186. ISBN 076372470X. Archived from the original on 2015-05-18..

[4]
^Herbert, Nick (2011). Quantum Reality: Beyond the New Physics. Knopf Doubleday Publishing Group. p. 150. ISBN 030780674X. Archived from the original on 2015-05-18..

[5]
^Charap, John M. (2002). Explaining The Universe. Universities Press. p. 99. ISBN 8173714673. Archived from the original on 2015-05-18..

[6]
^Can Quantum-Mechanical Description of Physical Reality Be Considered Complete? Archived 2006-02-08 at the Wayback Machine A. Einstein, B. Podolsky, and N. Rosen, Phys. Rev. 47, 777 (1935).

[7]
^Schrödinger, Erwin (November 1935). "Die gegenwärtige Situation in der Quantenmechanik (The present situation in quantum mechanics)". Naturwissenschaften. 23 (48): 807–812. Bibcode:1935NW.....23..807S. doi:10.1007/BF01491891..

[8]
^Maxwell, Nicholas (1 January 1993). "Induction and Scientific Realism: Einstein versus van Fraassen Part Three: Einstein, Aim-Oriented Empiricism and the Discovery of Special and General Relativity". The British Journal for the Philosophy of Science. 44 (2): 275–305. doi:10.1093/bjps/44.2.275. JSTOR 687649..

[9]
^Wimmel, Hermann (1992). Quantum physics & observed reality: a critical interpretation of quantum mechanics. World Scientific. p. 2. ISBN 978-981-02-1010-6. Archived from the original on 20 May 2013. Retrieved 9 May 2011..

[10]
^Faye, J (2008-01-24). "Copenhagen Interpretation of Quantum Mechanics". Stanford Encyclopedia of Philosophy. The Metaphysics Research Lab Center for the Study of Language and Information, Stanford University. Retrieved 2010-09-19..

[11]
^Carpenter RHS, Anderson AJ (2006). "The death of Schroedinger's cat and of consciousness-based wave-function collapse" (PDF). Annales de la Fondation Louis de Broglie. 31 (1): 45–52. Archived from the original (PDF) on 2006-11-30. Retrieved 2010-09-10..

[12]
^Okón E, Sebastián MA (2016). "How to Back up or Refute Quantum Theories of Consciousness". Mind and Matter. 14 (1): 25–49..

[13]
^Zurek, Wojciech H. (2003). "Decoherence, einselection, and the quantum origins of the classical". Reviews of Modern Physics. 75: 715. arXiv:quant-ph/0105127. Bibcode:2003RvMP...75..715Z. doi:10.1103/revmodphys.75.715..

[14]
^Wojciech H. Zurek, "Decoherence and the transition from quantum to classical", Physics Today, 44, pp. 36–44 (1991).

[15]
^Rovelli, Carlo (1996). "Relational Quantum Mechanics". International Journal of Theoretical Physics. 35 (8): 1637–1678. arXiv:quant-ph/9609002. Bibcode:1996IJTP...35.1637R. doi:10.1007/BF02302261..

[16]
^"How the quantum Zeno effect impacts Schrodinger's cat". phys.org. Archived from the original on 17 June 2017. Retrieved 18 June 2017..

[17]
^"What is the world's biggest Schrodinger cat?". stackexchange.com. Archived from the original on 2012-01-08..

[18]
^"Schrödinger's Cat Now Made Of Light". www.science20.com. 27 August 2014. Archived from the original on 18 March 2012..

[19]
^C. Monroe, et al. A "Schrödinger Cat" Superposition State of an Atom Archived 2012-01-07 at the Wayback Machine.

[20]
^Physics World: Schrödinger's cat comes into view.

[21]
^Scientific American : Macro-Weirdness: "Quantum Microphone" Puts Naked-Eye Object in 2 Places at Once: A new device tests the limits of Schrödinger's cat Archived 2012-03-19 at the Wayback Machine.

[22]
^arXiv, Emerging Technology from the. "How to Create Quantum Superpositions of Living Things"..

[23]
^"Could 'Schrödinger's bacterium' be placed in a quantum superposition?". physicsworld.com. Archived from the original on 2016-07-30..

[24]
^Chown, Marcus (2007-11-22). "Has observing the universe hastened its end?". New Scientist. Archived from the original on 2016-03-10. Retrieved 2007-11-25..

[25]
^Krauss, Lawrence M.; James Dent (April 30, 2008). "Late Time Behavior of False Vacuum Decay: Possible Implications for Cosmology and Metastable Inflating States". Phys. Rev. Lett. US: APS. 100 (17). arXiv:0711.1821. Bibcode:2008PhRvL.100q1301K. doi:10.1103/PhysRevLett.100.171301..