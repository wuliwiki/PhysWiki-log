% 热力学与统计力学导航

\begin{issues}
\issueDraft
\end{issues}

在物理学本科的专业课程中, 往往会先学习热力学再学习统计力学, 后者高级难度更大. 热力学几乎直接开始研究系统的宏观性质如温度,体积,压强等, 而统计力学更强调如何由微观定律如经典力学和量子力学, 用统计的方法导出宏观性质. 统计力学对数学工具以及物理基础知识的要求上比热力学要更高, 但也更能揭示宏观现象的本质.

\subsection{热力学}
热力学最经典的入门内容莫过于理想气体及其状态方程\upref{PVnRT}. 通过该模型, 我们可以定义温度和温标\upref{tmp}. 状态方程只能决定系统达到平衡后的状态, 于是下一步就是学习系统宏观性质随时间缓慢变化的过程, 也就是\textbf{准静态过程}. 在该过程中, 在每一个时刻都可以把系统视为热平衡的. 于是我们可以研究理想气体的一些常见过程: 等压过程\upref{EqPre}, 等体过程\upref{EqVol}, 等温过程\upref{EqTemp}, 绝热过程\upref{Adiab}.

……

\subsection{统计力学}
……
