% Matlab 的程序调试及其他功能
% keys Matlab|调试|调试器|breakpoing
% license Xiao
% type Tutor

\pentry{Matlab 的函数\upref{MatFun}}

\subsection{程序调试}
\begin{figure}[ht]
\centering
\includegraphics[width= 6cm]{./figures/4e4f08080d3ea13a.png}
\caption{在行首设置 Breakpoint}\label{fig_MatOtr_1}
\end{figure}

若要调试程序,可选择一行代码并单击该行前面的横线,这时会出现红色圆点 Breakpoint (\autoref{fig_MatOtr_1}),程序运行到 Breakpoint 会暂停。

此时要查看变量情况,可通过 Workspace 查看各个变量的情况,也可用光标悬停在某个变量上。还可以用 Command Window 改变某些变量的值,或画图等。在这种调试状态下,也可以通过 Edit 菜单中的一些按钮控制接下来程序如何运行(\autoref{fig_MatOtr_2})。
\begin{figure}[ht]
\centering
\includegraphics[width= 5cm]{./figures/054cf884ef614ab7.png}
\caption{Step 菜单}\label{fig_MatOtr_2}
\end{figure}
其中“Continue”(快捷键 F5)是继续运行直到下一个 Breakpoint 或结束。“Step”(F10)是运行到下一行,“Step In”(F11)是进入子程序并暂停,“Step Out”是运行完当前子程序并回到子程序被调用的地方。“Run to Cursor”是运行到光标所在处。

\subsection{warning 和 error 函数}
有时候当我们的程序中出现了某个错误时, 程序会终止并在命令行中返回一个错误信息, 例如我们给 \verb|sin| 函数输入两个变量时, 命令行的 \verb|error| 信息将提示变量个数太多。
\begin{lstlisting}[language=matlabC]
>> sin(1,2)
Error using sin
Too many input arguments.
\end{lstlisting}
另一些情况下当错误不是那么严重时, 我们会得到一个 \verb|warning| 提示, 程序在输出提示后将继续运行。

我们可以在我们自己的脚本或函数中用 \verb|error| 或 \verb|warning| 函数达到同样的效果

\begin{lstlisting}[language=matlab]
function myfun(char)
if char == 'w'
    warning('this is a warning');
elseif char == 'e'
    error('this is an error');
else
    disp('Hello World!');
end
end
\end{lstlisting}

当 \verb|error| 或 \verb|warning| 函数被执行时, 命令行不但会输出对应的字符串, 还会输出它们所在的文件和行号, 如
\begin{lstlisting}[language=matlabC]
>> myfun w
Warning: this is a warning
> In myfun (line 3)
\end{lstlisting}
单击 \verb|myfun| 或 \verb|line 3|, 可在 Editor 中显示对应位置。 若 \verb|myfun| 函数被其他函数调用, 那么调用的位置也会被逐级列出。 使用 \verb|warning| 的效果类似, 但程序不会被终止。

\subsection{分节}
在行首用两个百分号 \verb|%%| 可以对代码进行分节(\autoref{fig_MatOtr_3})。这样做一是可以使代码结构更清晰,二是可以单独选择某一节运行(Edit 菜单中的“Run Section”按钮)。
\begin{figure}[ht]
\centering
\includegraphics[width= 6cm]{./figures/7dc40f7c937060e2.png}
\caption{代码分节}\label{fig_MatOtr_3}
\end{figure}

\subsection{mat 文件}
\begin{itemize}
\item 保存任意变量到文件: \verb|save('file.mat', 'var1', 'var2')|
\item 读取文件中的指定变量: \verb|load('file.mat', 'var1', 'var2')|
\item 读取文件到一个 structure, 然后从中获取变量(可以防止变量名冲突) \verb|data = load('file.mat');|, \verb|data.var1|, \verb|data.var2|。
\end{itemize}


\subsection{工具箱(Toolbox)}
在购买和安装 Matlab 软件时,可以选择各种各样的工具箱,常用的工具箱有曲线拟合(Curve Fitting,从离散的数据点得到一条曲线),图像处理(Image Processing,图像变换,增强,降噪,二值化等),图像获取(Image Acquisition,从相机获取图像),Matlab 编译器(MATLAB Compiler,编译代码,提高运行速度)。注意使用了工具箱功能的代码在没有工具箱的 Matlab 软件上将无法运行。




