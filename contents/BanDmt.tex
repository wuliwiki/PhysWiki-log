% 带对角矩阵(BLAS)
% BLAS|CBLAS|矩阵|数据结构|科学计算|线性代数

\pentry{BLAS 简介\upref{BLAS}}

在 CBLAS 中, 我们可以用一种紧凑的数据结构来表示矩阵, 减少不必要的零元素计算.

要把一个 dense matrix 转换成 column major band matrix, 就把每个对角线都排成一行, 保持每个元素的列标不变. 同理, 要转换成 row major band matrix, 就把每个对角线都排成一列, 保持每个元素的行标不变.

举一个例子, dense matrix 如\autoref{BanDmt_tab1}, column major band matrix 如\autoref{BanDmt_tab2}, row major band matrix 如\autoref{BanDmt_tab3}.

\begin{table}[ht]
\centering
\caption{dense matrix}\label{BanDmt_tab1}
\begin{tabular}{|c|c|c|c|c|c|}
\hline
01  & 04  &    &    &    &   \\
\hline
02  & 05  & 08  &    &    &   \\
\hline
03  & 06  & 09  & 12  &    &   \\
\hline
   & 07  & 10 &  13  & 15  &   \\
\hline
   &    & 11 &  14  & 16  & 17 \\
\hline
\end{tabular}
\end{table}

\begin{table}[ht]
\centering
\caption{column major band matrix}\label{BanDmt_tab2}
\begin{tabular}{|c|c|c|c|c|c|}
\hline
   & 04  & 08  & 12 &  15 &  17 \\
\hline
01  & 05  & 09  & 13 &  16 &    \\
\hline
02  & 06  & 10 & 14 &     &    \\
\hline
03  & 07  & 11 &    &     &    \\
\hline
\end{tabular}
\end{table}

\begin{table}[ht]
\centering
\caption{column major band matrix}\label{BanDmt_tab3}
\begin{tabular}{|c|c|c|c|c|c|}
\hline
   &     &  01  &  04 \\
\hline
   &  02  &  05  &  08 \\
\hline
03  &  06  &  09  &  12 \\
\hline
07  &  10 &  13 &  15 \\
\hline
11 &  14 &  16 &  17 \\
\hline
\end{tabular}
\end{table}
