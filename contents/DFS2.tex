% 树与图的深度优先搜索
% DFS|算法|树与图的深度优先搜索|C++

\begin{issues}
\issueTODO
\end{issues}

本文将介绍树与图的深度优先搜索.

\textbf{树与图的存储:}
存储可以使用邻接表,邻接表的实现可以使用前面学的单链表\upref{List},邻接表就是 $n$ 个单链表,邻接表的所使用的数组需要多开一个 \verb|head| 数组,表示每个单链表的表头.邻接表的插入一般都是\textbf{头插法},即从单链表的表头插入新结点.

\begin{figure}[ht]
\centering
\includegraphics[width=14.25cm]{./figures/DFS2_1.png}
\caption{邻接表插入一个数}} \label{DFS2_fig1}
\end{figure}

可见,对于一张 $n$ 个点 $m$ 条边的图,可以用 $n$ 个单链表构成,$\forall x\in \text{Graph}$ 要想找到 $x$ 的所有出边,可以依据 $x$ 的表头依次访问.

\begin{lstlisting}[language=cpp]
int h[N], e[N], ne[N], idx = 0;

// 在表头是 a 的链表的头结点后面插入一个数 b
void add(int a, int b)
{
    e[idx] = b, ne[idx] = h[a], h[a] = idx ++ ;
}
\end{lstlisting}

\begin{figure}[ht]
\centering
\includegraphics[width=14.25cm]{./figures/DFS2_2.png}
\caption{树} \label{DFS2_fig2}
\end{figure}

\begin{lstlisting}[language=cpp]
const int N = 1e5 + 10, M = N;

int h[N], e[M], ne[M], idx;

void add(int a, int b)  // 添加一条边 a->b
{
    e[ ++ idx] = b, ne[idx] = h[a], h[a] = idx;
}

int main()
{
    int n;  // 结点数
    cin >> n;
    
    memset(h, -1, sizeof h); // 初始化每个单链表的表头
    for (int i = 0; i < n - 1; i ++ )
    {
        int a, b;
        cin >> a >> b;
        add(a, b), add(b, a);
    }
    
    // 打印邻接表
    for (int i = 1; i <= n - 1; i ++ )  // n - 1 条边
    {
        cout << i << ':';
        for (int j = h[i]; j != -1; j = ne[j])
            cout << "->" << e[j];  // j 为下标,e[j] 就是值
        cout << endl;
    }
    
    return 0;
}

/*
输入:
8
1 4
1 2
1 7
4 8
4 5
2 3
3 6

输出:
1:->7->2->4
2:->3->1
3:->6->2
4:->5->8->1
5:->4
6:->3
7:->1

*/
\end{lstlisting}

输出的结果就是每个链表的结点的值.