% 氦原子中的对易算符与能项符号
% license Xiao
% type Tutor

\begin{issues}
\issueDraft
\end{issues}

\pentry{类氢原子的束缚态\nref{nod_HWF}}{nod_1c16}
要思考是否对易, 除了直接计算以外, 也可以思考\autoref{the_Commut_1}  中的其他等效条件, 例如是否存在一组共同本征矢, 又例如 $A$ 是否在 $B$ 的本征子空间闭合, 也就是 $A$ 是否会耦合 $B$ 的不同本征值的本征矢($\mel{b_i}{A}{b_j} \ne 0$)。

氦原子中总哈密顿算
\begin{equation}
H = H_1 + H_2 + V_{12}~,
\qquad H_i = K_i + \frac{L_i^2}{2r_i^2}~.
\end{equation}
$K_i$ 是纯径向算符(只和 $r_i$ 有关), 所有的 $L$ 都是纯角向算符(只和角度有关\upref{SphAM}), 纯径向算符和纯角向算符可对易。 不同 $i$ 的任意算符可对易。

所以 $L_1^2, L_2^2, L^2, L_z$ 两两对易, $L_1^2, L_2^2, L_{1z}, L_{2z}$ 也两两对易, $L_z$ 和 $L_{iz}$ 对易, 但 $L^2$ 和单个 $L_{iz}$ 不对易。

比较复杂的是 $V_{12}$, 既有径向也有角向, 且耦合两个电子(\autoref{eq_HeTDSE_5})
\begin{equation}
[V_{12}, L^2] = [V_{12}, M] = 0~,
\qquad
[V_{12}, L_i^2] \ne 0~,
\qquad
[V_{12}, L_{iz}] \ne 0~.
\end{equation}
从经典力学的角度来看这是成立的。

已经数值验证: $\mel*{l'_1,l'_2,L',M'}{\mathcal Y_{l,l}^{0,0}}{l_1,l_2,L,M} = \delta_{L,L'}\delta_{M,M'}$。 说明 $H$ 只会耦合不同的 $l_1,l_2$ 而不会耦合不同的 $L,M$。 $H$ 的其他部分不会耦合任何不同的分波。 根据\autoref{the_Commut_1}  这说明 $H,L^2,L_z$ 两两对易, 可以在每个 $L,M$ 本征子空间中分别求解能量本征值。 这可以用于束缚态能量求解。

另外容易证明宇称算符 $\Pi$ 和 $H$ 对易, 和 $L^2$、$L_z$ 也对易(用\autoref{eq_GenYlm_2} \autoref{eq_GenYlm_6})。 所以 $H,L^2,L_z,\Pi$ 两两对易。 最后还可以加上粒子交换算符 $P_{12}$, 得到 $H,L^2,L_z,\Pi, P_{12}$ 两两对易。 单自旋态的氦原子的空间波函数满足交换对称, 三重态自旋满足交换反对称。 这样就可以写出氦原子束缚态的能项符号($M=0$)
\begin{equation}
n^{2S+1}L^{o/e}~.
\end{equation}
其中 $L=0,1,2$ 分别对应 $S,P,D$ 等, 单态 $S=0$,三重态 $S=1$, 若所有组成的分波都是 $l_1+l_2$ 为偶, 那么宇称就是 e, 若所有都为奇, 则宇称为 o。

所以若把波函数展开为\autoref{eq_HeTDSE_6}, 在单个 $L,M$ 本征子空间中, 对 $l_1,l_2$ 的求和中还可以指定 $l_1+l_2$ 的奇偶性, 以及二维径向波函数基底的交换对称性。 如果用张量积基底 $\psi(r_1,r_2) = \sum \psi(r_1)\psi(r_2)$, 那么解出来以后既有单态也有三重态。
