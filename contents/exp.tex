% 自然对数(数学分析)
% 自然对数

\pentry{序列的极限\upref{SeqLim}}

\subsection{自然对数的定义}
\begin{definition}{自然对数}
我们定义序列 $\{\qty(1+\frac{1}{n})^n\}$ 的极限为自然对数
\end{definition}
证明 $\{(1+\frac{1}{n})^n\}$ 的敛散性需要用到数列极限的性质.
记 $x_n=(1+\frac{1}{n})^n, n\in \mathbb{N}$,那么(由均值不等式)可以推出
\begin{equation}
  \frac{x_{n+1}}{x_n}=\frac{(n+2)^{n+1}n^{n}}{(n+1)^{2n+1}}>1
\end{equation}
所以 $\{x_n\}$ 是单调递增序列.

  如果能再证明该序列的有界性,就可以由 “\textbf{单调收敛定理}” 证明它是收敛的.

由二项式定理:
\begin{equation}\label{exp_eq1}
\begin{aligned}
  x_n&=1+{n\choose 1}\frac{1}{n}+{n\choose 2}\frac{1}{n^2}+\cdots+{n\choose n}\frac{1}{n^n}\\
  &<1+1+\frac{1}{2!}+\cdots+\frac{1}{n!}\\
  &\leq1+1+\frac{1}{2}+\frac{1}{4}+\cdot\cdot\cdot+\frac{1}{2^{n-1}}\\
  &<3
\end{aligned}
\end{equation}

这样就能推出 $\{x_n\}$ 的有界性.由此我们证明了,序列 $\{x_n=(1+\frac{1}{n})^n\}$ 是单调递增且有上界的,因此它收敛.我们称该序列的极限为\textbf{自然对数} $e$.

  
\subsection{自然对数的性质}
  记 $x_n=(1+\frac{1}{n})^n,y_n=(1+\frac{1}{n})^{n+1}, n\in \mathbb{N}$.我们已经分析过序列 $\{x_n\}$ 的性质,现在来考察 $\{y_n\}$ 的性质.
\begin{equation}
  \begin{aligned}
  \frac{y_n}{y_{n+1}}&=\frac{(n+1)^{2n+3}}{n^{n+1}(n+2)^{n+2}}=\left(\frac{n^2+2n+1}{n^2+2n}\right)^{n+1}\frac{n+1}{n+2}\\
  &=(1+\frac{1}{n^2+2n})^{n+1}\frac{n+1}{n+2}>\left(1+\frac{n+1}{n^2+2n}\right)\frac{n+1}{n+2} \text{(伯努利不等式)}\\
  &=\frac{n^3+4n^2+4n+1}{n^3+4n^2+4n}>1
  \end{aligned}
\end{equation}
  再由 $x_n<y_n$ 可知: $\{y_n\}$ 单调下降有下界,$\{x_n\}$ 单调上升有上界,两个序列极限存在.

又因为 $\lim\limits_{n\rightarrow \infty} x_n/y_n = 1$,所以两个序列极限相等,都是自然对数 $e$:
\begin{equation}
\lim\limits_{n\rightarrow \infty} \left(1+\frac{1}{n}\right)^n=\lim\limits_{n\rightarrow \infty} \left(1+\frac{1}{n}\right)^{n+1}=e
\end{equation}

现在来考察另一个序列 $S_n=1+1+\frac{1}{2!}+\frac{1}{3!}+\cdots+\frac{1}{n!}$.\autoref{exp_eq1} 已经证明了 $x_n<S_n<3$.容易证明 $\{S_n\}$ 单调递增有上界,因此 $\{S_n\}$ 也有极限.

利用二项式定理,通过不懈的努力(利用序列极限的性质),可以证明 $\{S_n\}$ 的极限就等于自然对数.这也暗示了 $e^x$ 的泰勒展开公式就是
\begin{equation}
e^x=1+x+\frac{1}{2!}x^2+\cdots+\frac{1}{n!}x^n +o(x^n)
\end{equation}

\subsection{对数函数}
我们定义:当 $f(x)=e^x$,的反函数为 $f'(x)=\ln(x)$.即 $\ln(e^x)=x$.

对不等式 $x_n<e<y_n$ 两边取 $\ln$ ,我们可以得到不等式:
\begin{equation}
  n\ln\left(1+\frac{1}{n}\right)<1<(n+1)\ln\left(1+\frac{1}{n}\right), \forall n \in \mathbb{N}
\end{equation}

