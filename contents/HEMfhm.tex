% 赫尔曼·冯·亥姆霍兹(综述)
% license CCBYSA3
% type Wiki

本文根据 CC-BY-SA 协议转载翻译自维基百科\href{https://en.wikipedia.org/wiki/Hermann_von_Helmholtz}{相关文章}。

\begin{figure}[ht]
\centering
\includegraphics[width=6cm]{./figures/3999a7bd3b84e395.png}
\caption{} \label{fig_HEMfhm_3}
\end{figure}
赫尔曼·路德维希·费迪南德·冯·亥姆霍兹(Hermann Ludwig Ferdinand von Helmholtz,/ˈhɛlmhoʊlts/;德语:[ˈhɛʁman fɔn ˈhɛlmˌhɔlts];1821年8月31日-1894年9月8日,自1883年起冠以“冯(von)”的贵族头衔)是一位德国物理学家和医生,在多个科学领域作出了重要贡献,尤其以流体动力学稳定性理论而闻名\(^\text{[2]}\)。以他命名的亥姆霍兹协会是德国最大的科研机构联合体\(^\text{[3]}\)。

在生理学和心理学领域,亥姆霍兹以其关于眼睛的数学研究、视觉理论、空间视觉感知的观点、色觉研究、音调感觉与听觉感知理论,以及对感知生理学中经验主义的探讨而著称。在物理学中,他以能量守恒定律、电双层理论、电动力学、化学热力学,以及热力学的力学基础研究而闻名。尽管能量守恒原则的发展也归功于尤利乌斯·冯·迈尔、詹姆斯·焦耳和丹尼尔·伯努利等人,但亥姆霍兹被认为是第一个以最一般形式提出能量守恒原理的人\(^\text{[4]}\)。

作为哲学家,亥姆霍兹以其科学哲学、关于知觉规律与自然规律之间关系的见解、美学科学思想,以及关于科学的文明力量等理念而受到关注。到19世纪末,亥姆霍兹发展出一种广义的康德方法论,包括对知觉空间中可能取向的先验确定,这不仅激发了对康德的新解读\(^\text{[4]}\),也对现代后期的新康德主义哲学运动产生了重要影响\(^\text{[5]}\)。
\subsection{生平}
\subsubsection{早年经历}
亥姆霍兹出生于波茨坦,是当地文理中学校长费迪南德·亥姆霍兹的儿子。父亲曾学习古典语言学和哲学,是出版人兼哲学家伊曼努尔·赫尔曼·费希特(的密友。亥姆霍兹的研究受到约翰·戈特利布·费希特和伊曼努尔·康德哲学思想的影响,他尝试在诸如生理学等经验领域中追溯这些理论的体现。

年轻时,亥姆霍兹对自然科学兴趣浓厚,但父亲希望他学习医学。1842年,亥姆霍兹在柏林的腓特烈-威廉医学外科研究院获得医学博士学位,并在夏里特医院完成为期一年的实习\(^\text{[6]}\)(因为医学专业提供财政资助)。

虽然主要接受的是生理学训练,亥姆霍兹却在许多其他主题上都有著述,从理论物理学到地球年龄的估计,再到太阳系的起源等问题。
\subsubsection{大学任职}
1848年,亥姆霍兹的第一份学术职务是在柏林艺术学院担任解剖学教师\(^\text{[7]}\)。随后,他于1849年在普鲁士的哥尼斯堡大学被任命为生理学副教授。1855年,他接受了波恩大学的全职解剖学与生理学教授职位。然而,他在波恩并不特别满意,三年后调任至巴登的海德堡大学,担任生理学教授。1871年,他接受了最后一个大学职位,在柏林的腓特烈·威廉大学(今柏林洪堡大学)担任物理学教授。
\subsection{研究工作}
\subsubsection{亥姆霍兹}
力学

亥姆霍兹的第一项重要科学成就,是他于1847年撰写的一篇关于能量守恒的论文。这项工作是在其医学研究和哲学背景的语境中完成的。他对能量守恒的研究起初源于对肌肉代谢的探索,试图证明肌肉运动过程中并没有能量的损失,其动机在于说明肌肉的运动不需要任何“生命力”来驱动。这是对当时在德国生理学中占主导地位的自然哲学(Naturphilosophie)和生命力论等投机哲学传统的直接否定。他反对一些生命力论者提出的观点——即“生命力”可以无限地驱动一台机器\(^\text{[4]}\)。

在此前萨迪·卡诺、贝努瓦-保罗·埃米尔·克拉佩龙和詹姆斯·普雷斯科特·焦耳等人的研究基础上,亥姆霍兹提出了一个假设,认为力学、热、光、电和磁都是单一“力”的表现形式——用今天的术语来说,即“能量”。他在其著作《论力的守恒》(Über die Erhaltung der Kraft, 1847)中发表了这一理论\(^\text{[8]}\)。

在19世纪50至60年代,亥姆霍兹与威廉·汤姆森(后来的开尔文勋爵)及威廉·兰金基于前者的出版物,共同推广了“宇宙热寂”这一概念。

在流体动力学方面,亥姆霍兹也作出多项贡献,包括在无粘性流体中提出的“亥姆霍兹涡旋定理”。
\subsubsection{感官生理学}
\begin{figure}[ht]
\centering
\includegraphics[width=6cm]{./figures/287cdf893acb7c66.png}
\caption{亥姆霍兹的多音调汽笛,格拉斯哥亨特博物馆。} \label{fig_HEMfhm_1}
\end{figure}
亥姆霍兹是人类视觉和听觉科学研究的先驱之一。他受到心理物理学的启发,致力于探索可测量的物理刺激与其对应的人类感知之间的关系。例如,通过改变声波的振幅可以使声音听起来更响或更轻,但声音压力振幅的线性增加并不会引起听感上的线性变化。为了使感知到的响度以线性方式变化,声音的物理强度必须以指数方式增加,这一事实如今已广泛应用于电子设备的音量控制之中。亥姆霍兹在实验研究物理能量(物理学)与其感知(心理学)之间关系方面开辟了道路,目标是建立“心理物理定律”。

亥姆霍兹的感官生理学研究为其学生威廉·冯特(Wilhelm Wundt)的工作奠定了基础,冯特被认为是实验心理学的奠基人之一。与亥姆霍兹相比,冯特更明确地将自己的研究描述为经验哲学的一种形式,并将其作为对“心灵”这一独立存在的研究。早年在反对自然哲学时,亥姆霍兹强调唯物主义的重要性,更多地关注“心灵”与身体之间的统一关系\(^\text{[9]}\)。
\begin{figure}[ht]
\centering
\includegraphics[width=6cm]{./figures/801f9de8c71fbaf6.png}
\caption{1848年的亥姆霍兹} \label{fig_HEMfhm_2}
\end{figure}
\subsubsection{眼科光学}
1851年,亥姆霍兹发明了检眼镜,彻底革新了眼科学领域。检眼镜是一种用来检查人眼内部的仪器,这一发明使他一夜成名。那时,亥姆霍兹的研究兴趣日益集中在感官生理学上。他的主要著作《生理光学手册》(德语原名 Handbuch der Physiologischen Optik,英文常译为 Handbook of Physiological Optics 或 Treatise on Physiological Optics,第三卷的英文译本见[此处](https://archive.org/details/physiologicalopt03helmrich)),提出了关于深度知觉、颜色视觉和运动知觉的经验性理论,并在19世纪后半叶成为该领域的基础性参考著作。在1867年出版的第三卷也是最后一卷中,亥姆霍兹阐述了“无意识推理”在知觉中的重要性。该手册首次被翻译成英文是在1924–1925年,由詹姆斯·P·C·索思奥尔代表美国光学学会主编。
他的调节理论(眼睛如何聚焦)在20世纪最后十年之前一直没有受到挑战。

在此后的几十年中,亥姆霍兹不断修订和完善这部手册,多次出版新版。他与持有相反观点的埃瓦尔德·黑林在空间视觉与色觉方面存在激烈争论,这场争论也使得19世纪下半叶的生理学界分裂成两个阵营。
\subsubsection{神经生理学}
1849年,亥姆霍兹在柯尼斯堡工作期间,测量了神经纤维中信号传导的速度。当时大多数人认为神经信号的传播快得无法测量。亥姆霍兹使用了刚解剖出来的青蛙坐骨神经和与之相连的小腿肌肉,并使用检流计作为灵敏的计时装置:他在指针上安装了一面小镜子,用来反射光束到房间对面的刻度尺上,从而极大提高了灵敏度。亥姆霍兹报告称\(^\text{[11][12]}\),神经信号传导速度在 24.6 至 38.4 米/秒之间\(^\text{[10]}\)。
\subsubsection{声学与美学}
\begin{figure}[ht]
\centering
\includegraphics[width=6cm]{./figures/afaa5883003466f4.png}
\caption{冯·亥姆霍兹的最后一张照片,拍摄于他病倒前三天。} \label{fig_HEMfhm_4}
\end{figure}
1863年,亥姆霍兹出版了《音调的感觉》,再次展现了他对感知物理学的浓厚兴趣。这本书对20世纪的音乐学家产生了深远影响。为了识别复杂声音中包含的多个纯正正弦波成分的不同频率或音高,亥姆霍兹发明了亥姆霍兹共鸣器\(^\text{[13]}\)。

亥姆霍兹发现,不同组合的共鸣器可以模拟元音的声音:亚历山大·格拉汉姆·贝尔对此尤其感兴趣,但由于他不会阅读德语,误解了亥姆霍兹的图示,认为亥姆霍兹已通过电线传输多个频率——这将实现电报信号的多路复用。实际上,电能仅用于维持共鸣器的振动。贝尔未能重现他以为亥姆霍兹已完成的实验,但他后来表示,如果当时能读懂德文,他就不会根据“谐波电报”原理发明电话了\(^\text{[14][15][16][17]}\)。

亥姆霍兹于1881年的肖像,由路德维希·克瑙斯绘制。

亚历山大·J·埃利斯翻译的英文版首次出版于1875年(根据1870年第三版德文原作);埃利斯根据1877年第四版德文原作的第二版英译本则于1885年出版;1895年和1912年的第三、第四版英文版为第二版的重印\(^\text{[18]}\)。
\subsubsection{电磁学}
\begin{figure}[ht]
\centering
\includegraphics[width=6cm]{./figures/685163aa0f09d408.png}
\caption{赫尔姆霍兹共鸣器(i)及其仪器设备} \label{fig_HEMfhm_5}
\end{figure}
赫尔姆霍兹在1869年至1871年间研究了电振荡,并于1869年4月30日在海德堡自然历史与医学协会发表题为《论电振荡》的讲座中指出:当线圈与莱顿瓶连接时,所能察觉到的阻尼电振荡持续时间约为 1/50 秒\(^\text{[19]}\)。
\begin{figure}[ht]
\centering
\includegraphics[width=6cm]{./figures/90fb9e0dbecfbad0.png}
\caption{1881年的赫尔姆霍兹,路德维希·瑙斯所绘肖像} \label{fig_HEMfhm_6}
\end{figure}
1871年,赫尔姆霍兹从海德堡迁往柏林,出任物理学教授。他开始对电磁学产生兴趣,后来的赫尔姆霍兹方程便以他的名字命名。尽管他在这一领域并未做出重大贡献,但他的学生海因里希·鲁道夫·赫兹因首次实验证明了电磁辐射而闻名。

奥利弗·赫维赛德曾批评赫尔姆霍兹的电磁理论,认为其允许纵波的存在是不合理的。赫维赛德基于对麦克斯韦方程的研究断言,在真空或均匀介质中不可能存在电磁纵波。然而他并未指出,在边界处或封闭空间中,电磁纵波是可以存在的\(^\text{[20]}\)。
\subsection{哲学}
赫尔姆霍兹在生理学与力学领域的科学研究,使他在科学哲学方面也声名显赫,特别是他关于知觉定律与自然法则之间关系的思想,以及他对欧几里得几何作为唯一可能的几何科学的否定\(^\text{[21]}\)。

他的科学哲学在某种程度上游移于经验主义和先验主义之间【22】。尽管后者常被视为偏重于思辨,他的科学哲学却深受数学物理的影响,用以取代生命力论并阐明能量守恒的一般原理\(^\text{[4]}\)。

他否认欧几里得几何是唯一可能的空间科学,这一点对于理解他对康德空间哲学的重构至关重要。康德的哲学原本主张欧几里得几何是物理空间的独特先验科学,而赫尔姆霍兹则引入了空间先验性的新观念:即在感知空间中确定可能取向的多样体。这些发展激发了对康德哲学的新解读\(^\text{[4]}\),并推动了现代晚期新康德主义哲学运动的兴起。
\subsection{学生与同事}
赫尔姆霍兹在柏林的其他学生与研究伙伴包括:马克斯·普朗克、海因里希·凯泽、欧根·戈尔德施泰因、威廉·维恩、阿图尔·克尼希、亨利·奥古斯都·罗兰、阿尔伯特·迈克耳孙、威廉·冯特、费尔南多·桑福德、阿瑟·戈登·韦伯斯特,以及迈克尔·伊万诺维奇·普平。与他于1869年至1871年在海德堡共事的利奥·柯尼希斯伯格于1902年撰写了他的权威传记。
\subsection{荣誉与遗产}
\begin{figure}[ht]
\centering
\includegraphics[width=6cm]{./figures/b2b43437227c18e6.png}
\caption{赫尔姆霍兹的雕像位于柏林洪堡大学前。} \label{fig_HEMfhm_7}
\end{figure}
\begin{itemize}
\item 1873年,赫尔姆霍兹当选为美国哲学会会员。\(^\text{[23]}\)
\item 1881年,他被选为爱尔兰皇家外科医学院名誉院士。\(^\text{[24]}\)
\item 同年11月10日,他被授予法国荣誉军团勋章指挥官级别(三级军官勋章,第2173号)。
\item 1883年,赫尔姆霍兹教授受皇帝褒奖,被晋升为贵族,自此他和其家族可冠以“冯”字头,称作“冯·赫尔姆霍兹”。这一荣耀虽非封爵或头衔,却具有世袭性质,并赋予其一定社会声望。
\item 1884年,赫尔姆霍兹被苏格兰工程师与造船师学会授予名誉会员称号。\(^\text{[25]}\)
\item 德国最大的科研机构联盟——赫尔姆霍兹协会即以他命名。\(^\text{[3][26]}\)
\item 小行星11573号“Helmholtz”、月球上的赫尔姆霍兹环形山,以及火星上的赫尔姆霍兹撞击坑,均以他命名以示纪念。\(^\text{[27][28][29]}\)
\item 在柏林夏洛滕堡区,有一条街道名为“赫尔姆霍兹街”,以纪念冯·赫尔姆霍兹。\(^\text{[30]}\)
\end{itemize}
\begin{figure}[ht]
\centering
\includegraphics[width=10cm]{./figures/0ed9f707c4057946.png}
\caption{授予赫尔姆霍兹(在首页列出)的法国荣誉军团勋章的法令} \label{fig_HEMfhm_8}
\end{figure}
\subsection{作品}
\begin{itemize}
\item 《论力的守恒》(Über die Erhaltung der Kraft,德文),莱比锡:Wilhelm Engelmann出版社,1889年。
\item 《电磁光学理论讲义》(Vorlesungen über die elektromagnetische Theorie des Lichts,德文),莱比锡:Johann Ambrosius Barth出版社,1897年。
\item 《声学数学原理讲义》(Vorlesungen über die mathematischen Principien der Akustik,德文),莱比锡:Johann Ambrosius Barth出版社,1898年。
\item 《离散质点动力学讲义》(Vorlesungen über die Dynamik discreter Massenpunkte,德文),莱比锡:Johann Ambrosius Barth出版社,1898年。
\item 《连续分布质量的动力学》(Dynamik continuirlich verbreiteter Massen,德文),莱比锡:Johann Ambrosius Barth出版社,1902年。
\item 《热力学理论讲义》(Vorlesungen über die Theorie der Wärme,德文),莱比锡:Johann Ambrosius Barth出版社,1903年。
\item 《理论物理讲义》(Vorlesungen über Theoretische Physik,德文),莱比锡:Johann Ambrosius Barth出版社,1903年。
\end{itemize}
\subsubsection{翻译作品}
\begin{itemize}
\item 《论力的守恒》(On the Conservation of Force,1847年),HathiTrust 提供。
\item 《作为音乐理论生理基础的音调感觉学说》(Lehre von den Tonempfindungen als physiologische Grundlage für die Theorie der Musik,法文),巴黎:Masson出版社,1874年。
\item 赫尔曼·冯·亥姆霍兹(1876年),“显微镜光学能力极限研究”,《每月显微镜杂志》(Monthly Microscopical Journal),第16卷,第15–39页。doi:10.1111/j.1365-2818.1876.tb05606.x
\item 《通俗科学讲座》,纽约:Appleton出版社,1885年。
\item 《论力的守恒》(1895年版),卡尔斯鲁厄1862–1863年冬季系列讲座导论,英文翻译。
\item 《作为音乐理论生理基础的音调感觉》,加州数字图书馆提供下载,基于1877年第四版德文版的第三版英文翻译,由赫尔曼·冯·亥姆霍兹与亚历山大·约翰·埃利斯合著,Longmans, Green出版社,1895年,576页。
\item 《作为音乐理论生理基础的音调感觉》,可通过 Google Books 下载,第四版英文译本,赫尔曼·冯·亥姆霍兹与亚历山大·约翰·埃利斯合著,Longmans, Green出版社,1912年,575页。
\item 《生理光学论》(Treatise on Physiological Optics,1910年),三卷本,美国光学学会于1924–1925年翻译成英文。
\item 《通俗科学讲座》,1885年。
\item 《通俗科学讲座·第二辑》,1908年。
\end{itemize}
\subsection{参见}
\begin{itemize}
\item 亥姆霍兹线圈
\item 柏林人物列表
\item 以赫尔曼·冯·亥姆霍兹命名的事物列表
\item 新康德主义
\item 色彩理论
\end{itemize}
\subsection{参考文献}
\subsubsection{引文}
\begin{enumerate}
\item David Cahan (1993).《赫尔曼·冯·亥姆霍兹与十九世纪科学的基础》,加利福尼亚大学出版社,第198页,ISBN 978-0-520-08334-9。
\item Bobba, Kumar Manoj (2004年1月1日).《稳健流动稳定性:近壁湍流中的理论、计算与实验》(学位论文),Bibcode:2004PhDT.......158B。
\item “这位兼具实践感的博学者”,德国亥姆霍兹联合会官网,检索日期:2024年10月4日。
\item Patton, Lydia. “赫尔曼·冯·亥姆霍兹”,2008年,*斯坦福哲学百科全书*。
\item Heis, Jeremy (2018). “新康德主义”,斯坦福哲学百科全书,检索日期:2024年10月6日。该运动受到多位哲学家的启发,主要包括库诺·费舍尔(Fischer 1860)、赫尔曼·冯·亥姆霍兹(Helmholtz 1867, 1878)、弗里德里希·朗格(Lange 1866)、奥托·李布曼(Liebmann 1865)和爱德华·策勒(Zeller 1862)——这些人于十九世纪中叶提倡回归康德哲学,以对抗思辨形而上学和唯物主义(参见 Beiser 2014b)。
\item R. S. Turner,《在眼中的心灵:视觉与亥姆霍兹-黑林之争》,普林斯顿大学出版社,2014年,第36页。
\item 《爱丁堡皇家学会前院士传记索引 1783–2002》,PDF格式,爱丁堡皇家学会,2006年7月,ISBN 0-902198-84-X;原PDF于2013年1月24日存档,检索日期:2016年10月21日。
\item 英文译本载于《科学文集:从外国科学院和外国期刊中选编的自然哲学论文》(1853年),第114页,由约翰·丁道尔翻译。见 Google Books, HathiTrust。
\item Peter J. Bowler 和 Iwan Rhys Morus (2005).《构建现代科学:一项历史考察》,芝加哥大学出版社,第177页,ISBN 978-0-226-06861-9。
\item Ian Glynn (2010).《科学的优雅》,牛津大学出版社,第147–150页,ISBN 978-0-19-957862-7。
\item 赫尔曼·冯·亥姆霍兹 (1850)。《关于神经刺激传播速度的初步报告》。载于:《解剖学、生理学与科学医学档案》。柏林:Veit & Comp.,第71–73页。柏林马克斯·普朗克科学史研究所(MPIWG)
\item 赫尔曼·冯·亥姆霍兹 (1850)。《关于动物肌肉收缩的时间进程及神经中刺激传播速度的测量》。载于:《解剖学、生理学与科学医学档案》。柏林:Veit & Comp.,第276–364页。MPIWG 柏林
\item 冯·亥姆霍兹,赫尔曼 (1885)。《作为音乐理论生理基础的音调感觉》。亚历山大·J·埃利斯(Alexander J. Ellis)译(第二版英文)。伦敦:朗文·格林公司,第44页。检索日期:2010年10月12日。
\item 《PBS 美国传奇:电话——关于贝尔的更多信息》。PBS。原文于2000年5月11日存档。
\item MacKenzie 2003,第41页。
\item Groundwater 2005,第31页。
\item Shulman 2008,第46–48页。
\item 赫尔曼·L·F·亥姆霍兹医学博士 (1912)。《作为音乐理论生理基础的音调感觉》(第四版)。朗文·格林公司。ISBN 9781419178931。
\item 科尼希斯伯格,利奥 (2018年3月28日)。《赫尔曼·冯·亥姆霍兹》。牛津:克拉伦登出版社。ISBN 978-0-486-21517-4。检索日期:2018年3月28日 – 来源:Google Books。
\item 约翰·D·杰克逊,《经典电动力学》,ISBN 0-471-30932-X。
\item 赫尔曼·冯·亥姆霍兹 (1977)。《论几何公理的起源与意义》。载于:《认识论著作集》,波士顿科学哲学研究丛书,第37卷,第1–38页。doi:10.1007/978-94-010-1115-0\_1。ISBN 978-90-277-0582-2。检索日期:2024年10月13日。
\item Liesbet De Kock (2018)。《亥姆霍兹“差异心理学”的历史化》。载于:《分析哲学史杂志》,第6卷第3期。doi:10.15173/jhap.v6i3.3432。hdl:1854/LU-8552480。S2CID 187618324。检索日期:2022年1月1日。关于赫尔曼·冯·亥姆霍兹在其整体科学哲学及感知理论中,在经验主义与先验主义之间的摇摆,是科学史与科学哲学中备受争议并被充分记录的主题。
\item “美国哲学学会会员记录”。search.amphilsoc.org。检索日期:2021年5月3日。
\item “自1784年以来的爱尔兰皇家外科医学院荣誉院士”。爱尔兰家谱项目,2013年。原文于2018年2月3日存档。检索日期:2016年8月4日。
\item “荣誉会员与院士名单”。苏格兰工程师与造船师协会。
\item “亥姆霍兹协会网站“关于我们”页面中名称历史”。原文于2012年4月14日存档。检索日期:2012年4月30日。
\item “11573号小行星亥姆霍兹(1993 SK3)”。小行星中心。检索日期:2018年2月2日。
\item “月球上的亥姆霍兹撞击坑”。行星命名公报,美国地质调查局天体地质研究项目。
\item “火星上的亥姆霍兹撞击坑”。行星命名公报,美国地质调查局天体地质研究项目。
\item “亥姆霍兹街”。berlin.de,2014年9月21日。检索日期:2018年7月18日。
\end{enumerate}
\subsubsection{资料来源}
\begin{itemize}
\item Cahan, David.《赫尔姆霍兹:科学人生》。芝加哥大学出版社,2018年。ISBN 978-0-226-48114-2。
\item Cohen, Robert 与 Wartofsky, Marx(编、译)。《赫尔姆霍兹:认识论文集》。雷德尔出版社,1977年。
\item Ewald, William B.(编)。《从康德到希尔伯特:数学基础原始文献选辑》,共两卷。牛津大学出版社,1996年。
\item 1876年:《几何公理的起源与意义》,第663–688页。
\item 1878年:《知觉中的事实》,第698–726页。
\item 1887年:《从认识论角度看编号与度量》,第727–752页。
\item Groundwater, Jennifer.《亚历山大·格雷厄姆·贝尔:发明精神》。卡尔加里:海拔出版公司,2005年。ISBN 1-55439-006-0。
\item Jackson, Myles W.《和谐三重奏:19世纪德国的物理学家、音乐家与乐器制造者》。麻省理工学院出版社,2006年。
\item Kahl, Russell(编)。《赫尔姆霍兹文选》,卫斯理大学出版社,1971年。
\item Koenigsberger, Leo.《赫尔姆霍兹传》,Frances A. Welby 译。多佛出版社,1965年。
\item MacKenzie, Catherine.《亚历山大·格雷厄姆·贝尔》。蒙大拿州怀特菲什:凯辛格出版社,2003年。ISBN 978-0-7661-4385-2。检索日期:2009年7月29日。
\item Shulman, Seth.《电话赌局:追踪贝尔的秘密》。纽约:诺顿公司,2008年。ISBN 978-0-393-06206-9。
\end{itemize}
\subsection{延伸阅读}
\begin{itemize}
\item 麦肯德里克,约翰·格雷,《赫尔曼·路德维希·斐迪南·冯·赫尔姆霍兹》,T. Fisher Unwin 出版社,1899年。
\item 科尼希斯伯格,利奥,《赫尔曼·冯·赫尔姆霍兹》,牛津大学克拉伦登出版社,1906年。
\item 大卫·卡恩:《赫尔姆霍兹:一位科学人生》,芝加哥大学出版社,2018年,ISBN 978-0-226-48114-2。
\item 史蒂文·沙平,〈一个“(几乎)万能”的理论家〉,《纽约书评》,第66卷第15期(2019年10月10日),第29–31页。书评对象:David Cahan 所著《赫尔姆霍兹:一位科学人生》(芝加哥大学出版社,2018年,937页,ISBN 978-0-226-48114-2)。
\item 大卫·卡恩(编):《赫尔姆霍兹与十九世纪科学的基础》,加州大学出版社,1994年,ISBN 978-0-520-08334-9。
\item 格雷戈尔·谢曼:《赫尔姆霍兹的机械论:确定性的失落——一项关于从经典到现代自然哲学转型的研究》,多德雷赫特:施普林格出版社,2009年,ISBN 978-1-4020-5629-1。
\item 弗朗茨·维尔纳:《赫尔姆霍兹的海德堡岁月(1858–1871)》(= 海德堡市档案馆特别出版物第8号),附52幅插图,柏林/海德堡:施普林格出版社,1997年。
\item 肯尼斯·L·卡内瓦:《赫尔姆霍兹与能量守恒:创造与接受的语境》,麻省理工学院出版社,剑桥(马萨诸塞州),2021年,ISBN 978-0-262-04573-5。
\end{itemize}
\subsection{外部链接}
\begin{itemize}
\item 《赫尔曼·冯·赫尔姆霍兹》(讣告),英国皇家学会,1894年,《伦敦皇家学会会刊》,伦敦:泰勒与弗朗西斯印刷,第 xvii 页。
\item “赫尔曼·冯·赫尔姆霍兹”,斯坦福哲学百科全书,撰文:莉迪亚·帕顿
\item “赫尔曼·路德维希·斐迪南·冯·赫尔姆霍兹教授博士”,FamilySearch 网站
\item 奥康纳,约翰·J;罗伯逊,埃德蒙·F,“赫尔曼·冯·赫尔姆霍兹”,圣安德鲁斯大学 MacTutor 数学史档案
\item 马克斯·普朗克科学史研究所虚拟实验室中的传记、书目和数字文献资源
\item 国际乐谱图书馆项目(IMSLP)上的赫尔姆霍兹作品免费乐谱(《关于音感的理论》)
\item 赫尔姆霍兹(1867年)《生理光学手册》——来自琳达·霍尔图书馆的数字影印本(2019年4月12日存档)
\item 数学世系项目中的赫尔曼·冯·赫尔姆霍兹资料
\item LibriVox 公共领域有声书平台上的赫尔曼·冯·赫尔姆霍兹作品
\end{itemize}