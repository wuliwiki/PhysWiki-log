% 反函数求导(简明微积分)
% keys 微积分|导数|反函数
% license Xiao
% type Tutor

\pentry{基本初等函数的导数\upref{FunDer}}

若已知 $f(x)$ 的导函数为 $f'(x)$, 则 $f(x)$ 的反函数 $f^{-1}(x)$ 的导函数为
\begin{equation}\label{eq_InvDer_3}
[f^{-1}(x)]' = \frac{1}{f'[f^{-1}(x)]} ~.
\end{equation} 
为了消除上式可能产生的歧义,记 $f(x)$ 的导函数为 $h(x)$,  $f(x)$ 的反函数为 $g(x)$。 上式变为
 \begin{equation}
g'(x) = \frac{1}{h[g(x)]}~.
\end{equation}

\begin{example}{}
在区间 $[0, \infty)$ 上, 令函数 $f(x) = x^2$, 那么反函数为 $f^{-1}(x) = \sqrt{x}$。 已知 $f'(x) = 2x$ (\autoref{eq_FunDer_2}~\upref{FunDer}), 带入\autoref{eq_InvDer_3} 得反函数的导数为
\begin{equation}
[f^{-1}(x)]' = \frac{1}{2\sqrt{x}} ~,
\end{equation}
我们也可以通过直接对 $\sqrt{x} = x^{1/2}$ 求导来验证这一结果的正确性。
\end{example}

\subsubsection{反函可导的条件}
只有函数 $y = f(x)$, 在某个区间 $(x_1, x_2)$ 内连续且单调时。因为如果一个 $y$ 有多个 $x$ 对应,反函数中将会出现一个 $x$ 对应多个 $y$ 的情况。

\subsection{反函数的定义}
令满足上述条件的某函数和反函数分别为 $f(x)$,  $g(x)$, 在有定义的区间内的任何一对满足 $y = f(x)$ 的 $x$,  $y$ 都满足 $g(y) = x$, 则 $g(x)$ 是 $f(x)$ 的反函数。

\subsection{证明}
\begin{figure}[ht]
\centering
\includegraphics[width=6cm]{./figures/79ccffc234742152.pdf}
\caption{在同一点处,$f'=\dv*{y}{x}$, $g'=\dv*{x}{y}$, 互为倒数}\label{fig_InvDer_1}
\end{figure}
根据导数和微分的关系, $y = f(x)$ 在 曲线上的某点 $(x_0, y_0)$, 有
 \begin{equation}\label{eq_InvDer_1}
\dd{y} = f'(x_0) \dd{x}~.
\end{equation}
同一点也满足 $g(y_0) = x_0$, 且
 \begin{equation}\label{eq_InvDer_2}
g'(y_0)\dd{y} = \dd{x}~.
\end{equation}
对比\autoref{eq_InvDer_1} 和\autoref{eq_InvDer_2}, 得
\begin{equation}
g'(y_0) = \dv{x}{y} = \frac{1}{f'(x_0)}~,
\end{equation}
可见\autoref{fig_InvDer_1} 曲线上同一点处 $f'$ 和 $g'$ 互为倒数。 把 ${x_0} = g(y_0)$ 代入上式,得
\begin{equation}
g'(y_0) = \dv{x}{y} = \frac{1}{f'[g(y_0)]}~.
\end{equation} 
上式中, $y_0$ 可以是 $g$ 函数定义域的任意一点,所以
\begin{equation}
g'(y) = \frac{1}{f'[g(y)]}~.
\end{equation} 
或者用习惯上的 $x$ 作为自变量,得
\begin{equation}
g'(x) = \frac{1}{f'[g(x)]}~,
\end{equation}
证毕。
















