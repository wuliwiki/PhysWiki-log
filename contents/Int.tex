% 不定积分(简明微积分)
% 微积分|积分|不定积分

\pentry{基本初等函数的导数\upref{FunDer}}
\subsection{原函数}
正如加法的逆运算是减法、乘法的逆运算是除法一样,求导\upref{Der}有没有他的逆运算呢?一般而言,我们把求导“逆运算”的结果称为原函数(primitive function, antiderivative),正如antiderivative的英文名所暗示的那样(anti-:反,derivative:导函数).

\begin{figure}[ht]
\centering
\includegraphics[width=10cm]{./figures/Int_1.png}
\caption{函数与原函数} \label{Int_fig1}
\end{figure}

\begin{definition}{原函数}
若 $F(x)$ 的导数为 $f(x)$, 那么 $F(x)$ 就是  $f(x)$ 的一个原函数.
\begin{equation}\label{Int_eq2}
F'(x) = f(x)
\end{equation}
\end{definition}

\begin{theorem}{}
事实上,给出一个 $f(x)$, 可以找到许多不同的原函数, 但这些原函数都只相差一个常数. 也就是说, 给 $f(x)$ 的任意一个原函数加上一个常数 $C$, 就可以得到 $f(x)$ 的另一个原函数. $C$ 叫做\textbf{积分常数(constant of integration)}.

或者换句话说,求导是一个“有损”的过程,函数在求导后失去了一个常数项的信息.因此,仅从导函数$f(x)$不能唯一确定被求导的原函数$F(x)$.
\end{theorem}

证明: 由于常函数导数\upref{Der}为 $0$, 给原函数加上常数后\autoref{Int_eq2} 仍然成立
\begin{equation}\label{Int_eq3}
\dv{x} [F(x) + C] = f(x)
\end{equation}
我们可以从几何上来理解该式: 将函数曲线 $y = F(x)$ 整体在 $y$ 方向平移并不影响某个 $x$ 坐标处函数曲线的斜率.

\subsection{不定积分}
初学者可以简单地认为不定积分和原函数是同义词;不过我们使用术语“不定积分”时,通常指的是\textsl{所有的}原函数,即在书写时包括一项积分常数$C$.

\begin{definition}{不定积分}
\begin{equation}\label{Int_eq1}
\int f(x) \dd{x} = F(x) + C
\end{equation}
\end{definition}
注意积分符号 $\int$ 和 $\dd{x}$ 是一个整体算符, 作用在他们中间的函数上\footnote{有时候为了方便也会记为 $\int\dd{x} f(x)$}.

为什么原函数还被称为“不定积分”呢?原函数和积分之间又有什么联系呢?事实上,这两个看似毫无关联的事物可以被牛顿—莱布尼兹公式\upref{NLeib}所联系,将求解定积分转换为求解原函数在积分上下限处的值.

\subsection{不定积分的基本性质}

由于求导是线性运算\upref{DerRul},不定积分也是线性运算.即若干函数的线性组合的积分等于分别对这些函数积分再线性组合.令 $a_n$ 为常数,有
\begin{equation}\label{Int_eq4}
\int [a_1 f_1(x) + a_2 f_2(x)\dots] \dd{x}  = a_1 \int f_1(x) \dd{x} + a_2 \int f_2(x) \dd{x} \dots
\end{equation}

\subsection{不定积分计算方法}
与求导不同,计算不定积分没有特定的步骤,这里介绍几种方法
\begin{enumerate}
\item 最简单直接的方法是把已知的各种常见函数的导数写成积分的形式,例如已知 $\sin x$ 的导数是 $\cos x$, $\cos x$ 的积分就是 $\sin x$ 加任意常数.
\item 换元积分法\upref{IntCV}, 包括第一类换元法和第二类换元法.

\item 分部积分法\upref{IntBP}

\item 查表法.许多高等数学教材(包括本书)都会给出一个积分表.当然,在信息技术发达的今天这种方法几乎已经被计算软件和网站取代.

\item 计算软件和网站.常见的符号计算软件有 Mathematica %未完成:引用
,Maple 等,数学网站有 \href{https://www.wolframalpha.com}{Wolfram Alpha}
,\href{https://www.geogebra.org}{Geogebra}等(建议先把积分技巧练熟再使用这些方法).其中 Wolfram Alpha 对许多积分还会给出详细的计算步骤.
 \end{enumerate}

对于一些常用积分,一般要求能熟记或快速推出.见积分表\upref{ITable} 中的\textbf{常用积分}部分.
