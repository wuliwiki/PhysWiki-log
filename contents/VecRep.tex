% 矢量空间的表示
% 列向量|矢量组|线性组合|坐标|表示

\begin{issues}
\issueOther{本词条需要重新创作和整合,融入章节逻辑体系。}
\end{issues}

\pentry{基(线性代数)\upref{VecSpn}}

由于矢量空间中运算的线性性,可以使用矩阵来表示任何一个矢量空间中的元素和线性变换。对于一个域 $\mathbb{F}$ 上的 $n$ 维线性空间中的矢量,我们惯例上使用一个 $n$ 行 $1$ 列的矩阵来表示,称为\textbf{列向量}。线性变换被表示成一个 $n\times n$ 的矩阵。这些矩阵中的元素都必须取自 $\mathbb{F}$。

需要注意的是,这些表示都依赖于该矢量空间的\textbf{基}的选取。


\subsection{用基向量来表示向量和线性变换}

给定域 $\mathbb{F}$ 上的 $n$ 维线性空间 $V$ 和它的一个基 $\{{e}_i\}_{i=1}^{n}$。由于 $V$ 中的每一个向量都可以唯一地表示成基向量的线性组合,因此我们可以用线性组合的系数来构成一个列向量,作为这个向量在基 $\{{e}_i\}_{i=1}^{n}$ 下的\textbf{坐标}。比如,向量 $a_1 {e}_1+\cdots+a_n {e}_n$ 在这个基下的坐标就是
\begin{equation}
\pmat{a_1\\ \vdots\\ a_n}
\end{equation}
基的选择不同,同一个向量的坐标也就不一样。

在研究线性变换的时候,我们只需要关注线性变换对基向量的变换,就可以据此计算出任意向量的线性变换。如果某一个线性变换 $T$ 把基向量 ${e}_i$ 变换成 $a_{i1} {e}_1+\cdots+a_{in} {e}_n$,那么我们可以在这个基下把 $T$ 表示成一个矩阵:
\begin{equation}
M=\pmat{a_{11},a_{12},\cdots,a_{1n}\\ a_{21},a_{22},\cdots,a_{2n}\\ \vdots\ddots\vdots\\ a_{n1},a_{n2},\cdots,a_{nn}}~.
\end{equation}

这样,如果把 ${v}$ 的坐标是列向量 $\bvec {c}$,那么 $\mat M\bvec{c}$ 就是 $T{v}$ 的坐标。

同样地,线性变换的矩阵表示,也依赖于基的选取。
