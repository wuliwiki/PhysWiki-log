% Clonezilla 笔记
% license Xiao
% type Note

\begin{issues}
\issueDraft
\end{issues}

\subsection{类似备份软件}
\begin{itemize}
\item Bacula
\item BorgBackup 
\item Duplicati
\item Amanda
\item AOMEI (傲梅)
\end{itemize}

\subsection{笔记}
\begin{itemize}
\item 为稳妥起见, 为了恢复后可以自动启动, 每个硬盘还是最好做一次完整的硬盘备份而不仅仅是分区备份。
\item 如果硬盘复制以后无法启动 GRUB (ubuntu 开始时选择操作系统的界面), 用 U 盘中的 ubuntu live 安装 \href{https://help.ubuntu.com/community/Boot-Repair}{Boot Repair} 修复 Grub 即可(可以参考网上的任意相关教程)。
\item 如果只是复制分区,也可以考虑用 Ubuntu live 中的 GParted
\item 保存 clonezilla 镜像的任何文件/文件夹的名字不能包含空格和小括号, 为了保险起见只用数字字母和连线 \verb`-`
\item clonezilla 可以把整个硬盘或者一个分区复制到另一个硬盘或者分区, 或者保存到镜像文件或者从镜像文件恢复
\item clonezilla live 就相当于 ubuntu live, 可以制作一个可以启动的光盘或者 U 盘, 进入以后使用 clonezilla (因为如果正在使用一个系统, 就不能给这个系统所在的分区制作镜像)
\item 其实 ubuntu live 的 disk 软件也有制作分区镜像的功能, 但功能上要简单许多

\item 下载 ubuntu (alternative) 版的 clonezilla 的 iso 文件, 然后用 rufas 制作可启动 U 盘即可
\item 进入以后根据提示一步步操作即可(可以选择中文界面)
\item 默认情况下, 镜像文件的大小只是分区种数据的大小, 而不是分区的大小

\item 如果提示 “磁盘存在不匹配的 GPT 和 MBR 分割表, 如果是 windows 就用 Disk Management, 在 Disk X 右键选 properties, volumn 面板查看。 如果要删除 GPT, 就 clonezilla 的命令行用 \verb|sudo sgdisk -z /dev/sd?| 即可(\verb|sd?| 如 sda, sdb 等)
\item 如果目标硬盘是一个 sata 接口的硬盘, 用 sata 转 usb 线可能会不行, 可能需要直接装到电脑主板上(Yip 的电脑升级 SSD 的时候碰到的)
\item 当把一个装有 windows 系统的小硬盘 clone 到一个大硬盘时, windows 会显示大硬盘的尺寸为小硬盘的尺寸。 这时只需要用 windows 的 Disk Management 软件先把分区缩小 1Mb (或者任意), 它就会发现未分配的空间, 这时再 extend 到最大就可以了。
\item 当恢复分区到一块新的硬盘上时,先把改硬盘进行同样大小的分区或者更大的分区。 可以在 clonezilla 的备份文件夹中查看到备份中的各个分区大小信息。
\end{itemize}

\begin{itemize}
\item GParted 是可以伸缩 ext4 分区的(如果有一个小钥匙图标,就说明被 mount 了, 要先 unmount)
\item clonezilla 无论是文件还原还是分区或硬盘克隆都不支持目标比原来容量更小。 所以要先用 GParted 缩小一下, 接下来如果复制分区的话 GParted 也可以就不需要 clinezilla 了。
\item 若把对分区备份时产生的镜像文件恢复到硬盘(而不是分区)则会出错:\verb`"ocs-live-general" finished with error!`
\end{itemize}
