% 线积分(矢量分析)
% 多元微积分|积分|定积分|线积分|直角坐标系|保守场

% 未完成: 提一下环路积分


\pentry{矢量的内积\upref{Dot},一元矢量函数的积分\upref{IntV}}

\addTODO{标量场的线积分}

我们先来通过一个物理的例子引入\textbf{曲线积分}的概念, 见 “功、功率\upref{Fwork}”。

下面讨论如何在直角坐标系中具体计算线积分。为书写方便,以下省略积分路径 $C_{ab}$。 

将被积曲线的参数方程表示为\footnote{注意这里的 $t$ 不一定代表时间,可以是任意参数,甚至可以是 $x,y,z$ 中的一个。} $x(t),y(t),z(t)$, 则曲线上任意一点都唯一对应一个 $t$ 值。根据微分关系,当 $t$ 增加 $\dd{t}$ 时,曲线上的一小段位移矢量 $\dd{\bvec r} = (\dd{x}, \dd{y}, \dd{z})$ 中
\begin{equation}\label{eq_IntL_1}
\dd{x} = x'(t) \dd{t} \qquad \dd{y} = y'(t) \dd{t} \qquad \dd{z} = z'(t) \dd{t}
\end{equation}
这样,对曲线上任意一点(对应参数 $t$),$\bvec F$ 可表示成 $t$ 的矢量函数 $\bvec F(t) = \bvec F[x(t),y(t),z(t)]$。  $\bvec F$ 的三个分量\footnote{为了书写简洁,这里定义 $x_1\equiv x, x_2\equiv y,x_3\equiv z$。} 则表示为关于 $t$ 的单变量标量函数
\begin{equation}
F_{x_i}(t) = F_{x_i}[x(t),y(t),z(t)] \quad (i = 1,2,3)
\end{equation}

下面将三维空间的线积分转换为三个一元定积分
\begin{equation}\ali{
\int \bvec F(\bvec r) \vdot \dd{\bvec r}  &= \lim_{n \to \infty } \sum_{i = 1}^n \bvec F(\bvec r_i) \vdot \Delta \bvec r_i\\
&= \lim_{n \to \infty } \sum_{i = 1}^n F_x(\bvec r_i)\Delta {x_i} + \lim_{n \to \infty } \sum_{i = 1}^n F_y(\bvec r_i)\Delta{y_i} + \lim_{n \to \infty } \sum_{i = 1}^n F_z(\bvec r_i)\Delta{z_i} \\
&= \int {F_x}(\bvec r) \dd{x}  + \int {F_y}(\bvec r) \dd{y}  + \int {F_z}(\bvec r) \dd{z} 
}\end{equation} 
设积分路径 $C_{ab}$ 的起点对应 $t = a$, 终点对应 $t = b$。 结合\autoref{eq_IntL_1}, 上面每一项积分可以表示为 
\begin{equation}\label{eq_IntL_4}
\int F_{x_i}(\bvec r) \dd{x_i}  = \int_a^b F_{x_i} [\bvec r(t)] x_i'(t) \dd{t} \quad (i=1,2,3)
\end{equation} 
计算这三个关于 $t$ 的定积分再相加,就可以得出线积分结果。

\begin{example}{计算力场对质点的做功}\label{ex_IntL_1}
令力场为 $\bvec F = \alpha r \,\uvec r$, 一质点从原点出发,沿轨迹 $(x-a)^2 + y^2 = a^2$ 的上半部分移动到 $(a,0)$,求力对质点做的功。 若起点终点不变,轨迹改为延 $x$ 轴,结果又如何?

我们先来建立运动轨迹的参数方程。由于运动是一个圆,我们可以使用圆的参数方程。把角度作为参数 $t$, $t\in [0,\pi]$。
\begin{equation}
\begin{cases}
x(t) = a(1-\cos t)\\
y(t) = a \sin t
\end{cases}
\qquad 
\begin{cases}
x'(t) = a \sin t\\
y'(t) = a \cos t
\end{cases}
\end{equation}
把力场在直角坐标系中表示为 $\bvec F(x,y) = \alpha (x\,\uvec x + y\,\uvec y)$, 两个分量分别为 $F_x = \alpha x, F_y = \alpha y$。 由\autoref{eq_IntL_4} $(i=1,2)$, 力场对质点做功等于两个定积分之和
\begin{equation}
W = \int \bvec F \vdot \dd{\bvec r} =\int_0^\pi \alpha a(1-\cos t) \cdot a \sin t \dd{t} + \int_0^\pi \alpha a \sin t \cdot a \cos t \dd{t}
\end{equation}
注意到第一个积分中的第二项恰好是第二个积分的相反数,所以上式变为
\begin{equation}
\int_0^\pi \alpha a^2 \sin t \dd{t} =2 \alpha a^2
\end{equation}

现在来计算延 $x$ 轴的直线轨迹运动的情况。由于轨迹上处处都有 $y=0$, $F_y = 0$,积分只有 $F_x$ 一项。 另外 $x$ 本身就可以作为轨道参数,即 $x(t) = t, y(t) = 0, x\in [0, 2a]$。 代入\autoref{eq_IntL_4} 得做功为
\begin{equation}
W = \int_0^{2a} \alpha x \dd{x} = 2 \alpha a^2
\end{equation}
\end{example}

在上例中, 我们发现对于给定的矢量场, 即使路径不同,当起点和终点相同时, 线积分的结果也相同(虽然我们只计算了两条路径, 但这个结论是正确的)。 具有这样性质的矢量场叫做\textbf{保守场},并总存在一个势能函数。 

\addTODO{环路积分}