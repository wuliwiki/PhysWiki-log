% 实数的表示
% license Xiao
% type Tutor

% \begin{issues}
% \issueOther{不应该需要完备公理,如果需要应该引用\enref{实数的完备公理}{RCompl}}
% \end{issues}

\pentry{完备公理(戴德金分割)\nref{nod_Cmplt}}{nod_f5e1}

我们最熟悉的实数表示方式是十进制小数或者二进制数。 在这里, 我们可以借助实数公理而给予这些表示以严格的逻辑基础。 特别地, 这能够解释一个古老的问题: $0.\dot{9}$ 到底等不等于 $1$?

我们固定一个正整数 $q>1$ 作为进位制的基底。 取 $q=2$ 或 $q=10$ 当然是最熟知的。

我们从如下非常简单的命题开始。 
\begin{lemma}{}
设实数 $x>0$. 则有唯一一个整数 $n\in\mathbb{Z}$ 和唯一一个 $a\in\{1,2,...,q-1\}$, 使得 $aq^n\leq x<(a+1)q^{n}$.
\end{lemma}

实际上, 整个正实数轴被划分成彼此不相交的区间 $[aq^n,(a+1)q^{n})$; 这里 $n$ 跑遍所有整数, $a$ 跑遍 $1,2,...,q-1$. 因此正实数 $x$ 只能属于所有这些区间之中的一个。 

将对应于 $x$ 的这两个整数记为 $n_1$ 和 $a_1$. 则 $x-a_1q^{n_1}$ 是非负实数。 如果它等于零, 那么无需续行, 否则可以对它继续应用上述命题, 而得到 $n_2\in\mathbb{Z}$ 和 $a_2\in\{1,2,...,q-1\}$, 使得 $a_2q^{n_2}\leq x-a_1q^{n_1}<(a_2+1)q^{n_2}$. 由此不等式立刻可得
\[
q^{n_2}\leq a_2q^{n_2}\leq x-a_1q^{n_1}<q^{n_1}~,
\]
从而 $n_2<n_1$. 如果 $x-a_1q^{n_1}-a_2q^{n_2}$ 等于零, 则无需续行, 否则可以再次重复上面的过程。 这样就得到了一个取值于 $\{1,2,...,q-1\}$ 的整数序列 $a_1,a_2,...$, 以及一列递降的整数 $n_1>n_2>...$, 使得
\[ 
0\leq x-a_1q^{n_1}-a_2q^{n_2}-...-a_kq^{n_k}<q^{n_k}~.
\]
递降的整数序列序列 $n_1>n_2>...$ 或者在某处终止 (这意味着上面的操作在某一步因差值为零而停止, 由此 $x$ 在 $q$ 进制下的表达式是有限的), 或者只能递降到负无穷。 因此当 $k\to\infty$ 时, 序列 $a_1q^{n_1}+a_2q^{n_2}+...+a_kq^{n_k}$ 越来越逼近实数 $x$. 于是可以写下如下的形式等式:
\begin{equation}
x=a_1q^{n_1}+a_2q^{n_2}+...+a_kq^{n_k}+... ~
\end{equation}


它的严格含义是, 右端和式的前面有限项组成的序列的极限是 $x$.
