% 复旦大学 2000 量子真题
% license Usr
% type Note

\textbf{声明}:“该内容来源于网络公开资料,不保证真实性,如有侵权请联系管理员”

1.试求出能量为100电子伏的自由电子的德布洛意波的波长。(20分

2.利用测不准关系估计氢原子的基态能量。(20分)

3.粒子系处于下列外场中,指出哪些力学量(动量、能量、角动量、宇称等,或它们的组合)是守恒量。(20分)\\
1)自由粒子(无相互作用,也不受外力)\\
2)无限、均匀柱对称场\\
3)无限、均匀平面场\\
4)中心力场\\
5)均匀交变场\\
6)椭球场\\

4. 设非简谐振子的哈密顿量为(其中$beta$为常数)\\
$\hat{H} = \hat{H}_0 + \hat{H}',$\\
$\hat{H}_0 = -\frac{\hbar^2}{2\mu} \frac{d^2}{dx^2} + \frac{1}{2} \mu \omega^2 x^2$\\
$\hat{H}' = \beta x,$\\

试用微扰论计算其能量(至微扰论二级)及能量本征函数(至微扰论一级)。(20 分)

5. 两电子在宽度为 $L$ 的一维无限深方势阱中,电子间排斥势 $V(|x_1-x_2|)$ 可视为微扰,试求体系第一激发态和第二激发态的能级(至微扰论一级)。(20 分)
