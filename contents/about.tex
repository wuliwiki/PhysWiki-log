% 关于小时百科
% license Xiao
% type Tutor

\subsection{我们想做什么}
\textbf{小时百科不是另一个维基百科。}

看到 “\textbf{小时百科}” 这个标题, 可能许多人会以为这是一个类似维基百科的网站。 但我们并不是在做一个国产的数理领域的维基百科, 虽然由于历史原因我们的项目叫做小时百科, 但目前我们主要做的是\textbf{教学}。 目前大部分内容都更接近\textbf{教材}而不是\textbf{百科}。 这和维基百科有什么区别呢?

维基百科上每个条目基本可以看做是围绕其标题的一个\textbf{综述}, 追求全面、客观、中立、一般性。 它的很多内容对初学者并不友好(\href{https://en.wikipedia.org/wiki/Wikipedia:What_Wikipedia_is_not}{维基百科不是手册、指南、教材})。 例如在维基百科的\href{https://en.wikipedia.org/wiki/Newton's_laws_of_motion}{牛顿运动定律}条目\footnote{我们将只引用维基百科的英文条目, 因为中文条目的质量还有相当大的差距。}中囊括了牛顿力学的发展历史, 牛顿三定律的具体表述, 功和能, 刚体力学, 混沌, 分析力学, 与热力学的关系, 与电磁学、相对论以及量子力学的关系。 全文约 9000 词, 引用文献约 120 个。 这样的结构作为一个百科条目或者一篇综述是符合标准的, 但显然不适合初学者。

作为对比, 如果我们翻开一些优秀的大学物理教材, 里面则可能会先用一章介绍一些简单的矢量微积分, 然后讲解如何画受力分析图, 力的合成与分解, 然后再讲解牛顿三定律, 刚体力学, 最后才会简单介绍分析力学和相对论等, 也通常不会涉及任何量子力学。 也就是说维基百科上的牛顿运动定律涉及到了一本(甚至多本)几百页的教材甚至许多学术论文。

然而, 很多时候即使是我们即使找到这样的教材,也不想把它从头读到尾, 即使在大学的课程中由于时间有限也很少把整本书都覆盖到。 所以如果你的目标是\textbf{有目的的学习某几个知识点}——例如高中生想了解大学物理中如何描述牛顿三定律——而不是系统地按部就班地学习整个学科。 在自学维基百科无果后, 你翻开一本教材通过目录找到对应的章节想要学习。 如无意外, 你会发现想要学的章节看不懂, 因为前面的内容没读。 在没有人指引的情况下, 你最后不得不从第一页开始看。 更糟糕的是, 如果这本书开始就假设你已经学过一些其他的课程(例如微积分和线性代数)那你还需要找来这些教材, 仍然从第一页开始看…… 如果你足够有耐心和毅力, 终于把若干本教材都从头看完弄懂 ——诚然你的基础会非常扎实—— 回过头来你会发现这个过程中学到的大部分内容对你最初的目标(牛顿三定律)来说太过深入, 或者根本用不上(学牛顿三定律并不需要先掌握微分中值定理或行列式), 但你很难从一开始就知道哪些才是你真正需要的。

当然如果你一定要通过维基百科自学, 理论上顺着所有链接和引用, 你最终也可以达到目的, 但你通常并不清楚哪些链接和引用是你当前最需要的。 这就会产生不同页面间海量的依赖关系, 形成一个错综复杂的迷宫。 如果假设每个页面链接到 10 个不同的页面, 那么只需 3 层就会达到惊人的 1000 个页面。 而一些至关重要的预备知识可能会隐藏在更深的链接中。

解决这个问题最好的办法是找一个私人老师说明你想学什么, 学到什么深度, 然后让老师根据这个目标以及你现有的知识背景和愿意投入的时间给你定制一套私人课程。 但显然大部分人并不具有这样优越的条件,小时百科的目标正是做这样一个老师, \textbf{让每个人都能根据自己的知识背景和目标给自己量身定制一套课程}。

具体内容上,我们的近期计划是涵盖理工科专业本科课程中的主要内容, 适用于具有普通高中及以上数学物理基础的读者。

\subsection{我们怎么做}
要达到上面的目标,一些比较人性化的教材在前言中会根据你的时间精力给你一些不同的选择(例如说明时间不够可以只看哪些章节, 甚至会画出一个章节依赖关系的树状图)。 但我们想把这种依赖关系做得更细更友好, 对每个小节都给出完整的依赖关系、 对每个使用的公式定理的来源都进行精确的追溯定位。另外我们希望你在达到学习目标的过程中,\textbf{尽量少学与目标无关的内容}。 为此我们采纳了一些\textbf{核心概念}:
\begin{itemize}
\item \textbf{节点}:把知识划分成一个个最小整体,在知识树中称为\textbf{节点}。 每个节点都有明确的读者画像、\textbf{预备知识}(必须先掌握哪些其他节点)、以及内容范围。 一个页面可以是一个节点, 但如果内容较多不需要通读也可以包含多个节点。
\item \textbf{知识树}:用于代替传统教材的目录,鼓励\textbf{按需学习}优于\textbf{按顺序学习}。 每个节点只有当需要的时候再学。节点间通过\textbf{预备知识}连接。
\item \textbf{自洽}:每个节点中的术语、定义和定理等必须来自本节点或预备知识,正文第一次出现时要有链接,指向的内容尽量具体。
\item \textbf{多版本}:同一个内容做不同版本的页面面向不同读者,即使有部分重复内容。其中包含类似维基百科页面的\textbf{综述版本}。
\item \textbf{封装和分离(可选)}:考虑模块分割时,尽量把 “是什么”、“怎么用”、“为什么” 划分为不同的节点,即使这些节点处于同一个页面中。这是为了进一步避免想学习前者的读者接触到不需要的内容。
\end{itemize}

我们还采用另一些理念,它们可能不会让你更快找到需要的内容,但会让内容对初学者更友好
\begin{itemize}
\item \textbf{多媒体}:大量使用图片、动画、互动演示。
\item \textbf{多例题}:加入大量例题。
\item \textbf{特殊到一般(可选)}:倾向具体的例子、然后倾向特殊范围的定义、最后才给出更一般化的定义。
\end{itemize}

\addTODO{下面讲解中尽量给出更多优秀的具体例子。}

\subsubsection{知识树和节点}
若一个页面包含多个节点,则每个节点默认依赖上一个节点。 完整的知识树见 \href{https://wuli.wiki/tree}{wuli.wiki/tree}, 可以选中任意节点为目标,生成子知识树。

\begin{figure}[ht]
\centering
\includegraphics[width=10cm]{./figures/648204cc09583468.pdf}
\caption{由 “预备知识” 画出的知识树(目标词条为“力场、势能\upref{V}”)}\label{fig_about_1}
\end{figure}

小时百科中每个页面的标题名称必须是全站唯一的, 若同一个内容有不同版本的页面, 我们使用稍微不同的标题或在它后面用括号加以区分。 例如 “角动量(科普)\upref{AngMo}”, “角动量、角动量定理、角动量守恒(单个质点)\upref{AMLaw1}”, “角动量定理、角动量守恒\upref{AMLaw}”, “轨道角动量(量子力学)\upref{QOrbAM}”, “自旋角动量\upref{Spin}” 都分布在不同的部分和章节。

理论上,读者可以直接跳到最感兴的词条,如果 “预备知识” 中列出的词条都已经掌握,就可以开始学习该词条,否则就先掌握 “预备知识” 中的词条。 如果 “预备知识” 出现在词条开始,则必须先掌握,如果出现在正文中,则只有阅读该部分时需要掌握。如果正文中引用了没有出现在“预备知识”中的内容,则读者可自行决定是否阅读。

\subsubsection{多版本}
相对于百科或综述而言, 教材的内容编排是高度主观且取决于作者和目标读者的。

例如 “对称矩阵的本征问题\upref{SymEig}” 和 “厄米矩阵的本征问题\upref{HerEig}” 内容几乎相同,只是后者把对称矩阵推广到了厄米矩阵。 这个例子也体现了特殊到一般的思想。

把同一个话题以不同深度, 严谨度和适用范围等划分成若干个等级的同名词条。 这样读者可以选择螺旋式学习(例如初中,高中,大学物理中所学的话题几乎相同,但程度不同)。 暂定初级词条从科普开始,尽量少使用数学。随着词条级别升高,会使用适用范围更广的定义,更严谨的表述和更抽象的数学等。

\subsubsection{自洽}
\textbf{自洽(self-contained)}。 从宏观来说,一本教材的自洽性指目标读者在学习前是否还需要学习其他教材。 例如大部分本科物理教材对高中生都是不自洽的, 因为它们往往假设读者具有一定的微积分和线性代数基础。从微观来说,一个节点的自洽性体现在严格禁止出没有介绍过的、预备知识以外的术语或定理等。

\subsubsection{封装和分离}
这里的封装借用了面对对象编程的概念。在面对对象编程中,一个对象(object)的使用方法和它的内部实现是严格分离的。使用者只需要了解对象使用方法而不需要了解对象内部的运作机制。在小时百科中,该思想最典型的应用就在介绍一个定理时,把 “有什么用”、 “如何使用” 和 “如何证明” 划分为不同的节点。 这样只想了解 “有什么用”、 “如何使用” 的读者就无需了解一点证明才需要的概念。

许多教材会抱着和读者一起探索的思想,先不直接给出结论,而是在推导过程中一步步得到想要的结论。

为了避免灌输, 可以先搞一个伪证明或者形象的解释,但目的是帮助理解 “是什么、怎么用”。

\subsubsection{特殊到一般}
如二次多项式、实矢量空间的二次型、复矢量空间的二次型

\subsubsection{目录}
我们\textbf{强调知识树而弱化目录}。 

目录只是一个大致的分类,每章尽量按照拓扑排序,也就是每个页面的预备知识必须排在它的前面。

所以在小时百科的\href{http://wuli.wiki/online}{目录}中, 我们并不是按照话题来分类所有页面, 而是创建许多\textbf{部分}。

有的部分可以是一个非常完整和连续的传统教材, 每个页面是一个小节, 适合从头读到尾(当然你也可以根据依赖关系自己决定读哪些)。 而另一些部分则没那么系统, 只是把某个话题下一些相对独立的小教程、博文、讲义、笔记等放到一起(但仍然需要给出具体的依赖关系)。 至于维基百科那样的综述性页面(目前还几乎没有), 我们同样可以根据话题创建\textbf{专门的部分}来收纳它们(目前还没有), 这些综述的正文中又可以进一步链接到其他部分中的页面(例如详细的证明推导)。 小时百科的每个部分(甚至它的每章)都应该有介绍页面来描述它的内容和结构。

综上, 你可以认为小时百科(的目标)是集百科、教材、 博客/笔记、 \href{http://wuli.wiki/apps}{互动演示}、 \href{http://wuli.wiki/forum}{讨论}于一体的, 支持合作编辑的综合性网站\footnote{而尴尬的是我们的网站已经叫做 “百科” 了, 但主要是一些不那么百科的内容。 或许以后会改名。}。

% === 这段应该放到 “创作指导” 中, 这篇文章是给读者看的 ====
% 每个页面应该有对应的审核员(通常是该页面的第一作者)来负责, 其他人想修改或添加内容需要经过他同意, 如果审核者认为要添加的内容超出了本文的范围, 则应该另外创建一个页面。 不应该像百科(综述)一样, 众多不同的人把众多不同的内容塞进同一个页面。

% ========= 回收的内容 ===========
% 如果我们只有一本公认优秀的传统教材,它包含了丰富的导入,动机,定义,定理,讲解,例题,习题。 而且它有一个很明确的目标人群,这些内容的深度和讲解思路,严谨性,都是为这个人群定制的。

% 那么,如何在一字不变的情况下把它做成灵活的模块化结构呢?

% 这样,我们可以把任意一本教材的内容切割并赋予一个树状结构。 就相当于给这本书生成一个知识地图。 书的内容一字不变,但是读者通过地图很容易看出来全书的结构。 一个极端的例子是,整本书都是严格线性的,任意第 $i+1$ 个节点都需要第 $i$ 个节点作为预备知识。 那么所谓的树状图就只有一个分支,这就要求所有读者把整本书从头看到尾或者看到中间的某个节点。 但几乎没有一本教材是这样严格的结构,它通常有许多不同的话题, 例如教材第一节是理论基础 A, 而第二节和第三节分别是话题 B 和话题 C, B 和 C 之间并没有什么联系, 可能一些读者只需要 B 另一些只需要 C。 在没有树状图时,读者并不知道看完 A 就可以直接看 C, 给出树状图以后读者就知道 B 是可以省略的。

\addTODO{整合:属于节点划分问题}
\subsubsection{百科和教材的矛盾}
一个经常遇到的问题是, 有时候一篇文章的范围很难明确界定。 例如 “球谐函数\upref{SphHar}” 页面, 如果它本来是作为某个其他页面的预备知识而创作的一个连贯自洽的小教程, 它不会像维基百科的\href{https://en.wikipedia.org/wiki/Spherical_harmonics}{球谐函数}条目那样包含几乎所有性质以及和其他众多特殊函数的关系, 而只是包含一些在某个场景下比较重要的性质(例如解氢原子的定态波函数\upref{HWF}所需要的那些)以及特定的讲解方式和举例。 但是随着其他需要使用球谐函数作为预备知识的页面的出现, 原作者或其他人可能会倾向于不断往同一个页面补充并引用新的性质, 并调整讲解顺序, 最后导致其越来越接近维基百科——但这样它就变得不那么适合初学/自学了。 可见百科和教材存在不可调和的矛盾。 对于一个话题, 我们不可能写出一个既像教材又像百科的 “部分” 并让所有人都满意。 这就是为什么要创建多个部分。

根据上文提出的思路, 最终理想的情况是, 我们既有维基百科那样关于球谐函数的综述(方便已经学过的人查阅检索, 不包含太多具体的细节和推导,不必照顾初学者), 又有符合不同场景的各种版本分散于其他部分。 我们也\textbf{无需避免相似的内容在不同页面出现重复}, 甚至还可以必要时把已有内容稍作修改创建一个符合不同需求的版本(如果协议兼容)。 这时, 不同的球谐函数页面就会分别被命名为诸如 “球谐函数(综述)”, “球谐函数简介(量子力学)”, “球谐函数表\upref{YlmTab}”, “球谐函数与XXX”, “球谐函数的XX性质” 等等。
