% Lapack 笔记

\begin{issues}
\issueDraft
\end{issues}

\pentry{BLAS 简介\upref{BLAS}}

Intel MKL 提供的 \href{https://www.intel.com/content/www/us/en/developer/tools/oneapi/onemkl-function-finding-advisor.html}{Lapack 函数搜索}.

如果有 Driver 就用尽量用, 其次再选 Computational

问题类型
\begin{itemize}
\item linear equations/system of linear equations 线性方程组
\item nonsymmetric eigenvalue problems 非对称或厄米矩阵本征值问题
\item symmetric eigenvalue problems 对称或厄米矩阵本征值问题
\item generalized symmetric-definite eigenvalue problems 广义对称或厄米正定矩阵本征值问题
\item generalized nonsymmetric eigenvalue problems 广义非对称或厄米正定矩阵本征值问题
\item linear least square (LLS) problems 线性最小二乘法问题
\item generalized LLS problems 广义线性最小二乘法问题
\item singular value decomposition 奇异值分解
\item cosine-sine decomposition 余弦-正弦分解
\end{itemize}

矩阵类型
\begin{itemize}
\item 对称矩阵
\item 厄米矩阵
\item 正交矩阵
\item 酉矩阵
\item 带对角矩阵
\item 三对角矩阵
\end{itemize}

\begin{figure}[ht]
\centering
\includegraphics[width=9cm]{./figures/Lapack_1.png}
\caption{packed storage 把上半或下半三角矩阵存成一行} \label{Lapack_fig1}
\end{figure}

以下是一些分类搜索结果

\subsection{线性方程组}
\subsubsection{Driver - 线性方程组 - 一般矩阵}
\begin{itemize}
\item ?gesv 解方矩阵线性方程组, 多个 RHS
\item ?gesvx gesv 并提供误差
\item ?gesvxx 用额外循环减小 gesv 的误差
\item ?gbsv gesv 的带对角矩阵版本
\item ?gbsvx gbsv 并提供误差
\item ?gbsvxx 用额外循环减小 gbsv 的误差
\item ?gtsv gesv 的三对角矩阵版本
\item ?gtsvx gtsv 并提供误差
\end{itemize}


\subsection{对称本征方程}
\subsubsection{Driver - 实对称矩阵本征方程 - 所有本征值和本征矢(可选)}
\begin{itemize}
\item ?syev 对称矩阵的本征值和本征矢(可选)
\item ?syevd syev 使用 divide and conquer 算法
\item ?spev syev 用 packed storage
\item ?spevd spev 用 packed storage
\item ?sbev syev 用带对角矩阵
\item ?sbevd sbev 用 divide and conquer 算法
\item ?stev syev 使用三对角矩阵
\end{itemize}

\subsection{线性最小二乘法}
\begin{itemize}
\item ?gels Uses QR or LQ factorization to solve a overdetermined or underdetermined linear system with full rank matrix.
\item ?gelsy Computes the minimum-norm solution to a linear least squares problem using a complete orthogonal factorization of A.
\item ?gelss Computes the minimum-norm solution to a linear least squares problem using the singular value decomposition of A.
\item ?gelsd Computes the minimum-norm solution to a linear least squares problem using the singular value decomposition of A and a divide and conquer method.
\end{itemize}
