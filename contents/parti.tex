% 粒子产生
% 粒子|产生|标量场

考虑克莱因-戈登场与一个外部的,经典的源 $j(x)$.考虑场方程
\begin{equation}\label{eq_parti_1}
(\partial^2+m^2)\phi(x) = j(x)~,
\end{equation}
$j(x)$ 是源。 这个方程是由拉式量推出来的。拉式量为
\begin{equation}
\mathcal L = \frac{1}{2} (\partial_\mu\phi)^2 - \frac{1}{2} m^2 \phi^2 + j(x) \phi(x)~,
\end{equation}
这里源 $j(x)$ 只持续一段时间。
\begin{equation}
\phi_{0}(x)=\int \frac{d^{3} p}{(2 \pi)^{3}} \frac{1}{\sqrt{2 E_{\mathbf{p}}}}\left(a_{\mathbf{p}} e^{-i p \cdot x}+a_{\mathbf{p}}^{\dagger} e^{i p \cdot x}\right)~.
\end{equation}
在有源的情况下,\autoref{eq_parti_1} 的解为:
\begin{equation}\label{eq_parti_2}
\begin{aligned}
\phi(x) & =\phi_{0}(x)+i \int d^{4} y D_{R}(x-y) j(y) \\
& =\phi_{0}(x)+i \int d^{4} y \int \frac{d^{3} p}{(2 \pi)^{3}} \frac{1}{2 E_{\mathbf{p}}} \theta\left(x^{0}-y^{0}\right) \\
& \times\left(e^{-i p \cdot(x-y)}-e^{i p \cdot(x-y)}\right) j(y)~.
\end{aligned}
\end{equation}
这时候 $\phi(x)$ 只与 $j$ 的傅立叶变换有关。
\begin{equation}
\tilde j (p) = \int d^4 y e^{ip \cdot y} j(y)~.
\end{equation}
\autoref{eq_parti_2} 整理一下可得
\begin{equation}
\phi(x)=\int \frac{d^{3} p}{(2 \pi)^{3}} \frac{1}{\sqrt{2 E_{\mathbf{p}}}}\left\{\left(a_{\mathbf{p}}+\frac{i}{\sqrt{2 E_{\mathbf{p}}}} \tilde{j}(p)\right) e^{-i p \cdot x}+\text { h.c. }\right\}~.
\end{equation}
哈密顿量为
\begin{equation}
H=\int \frac{d^{3} p}{(2 \pi)^{3}} E_{\mathbf{p}}\left(a_{\mathbf{p}}^{\dagger}-\frac{i}{\sqrt{2 E_{\mathbf{p}}}} \tilde{\jmath}^{*}(p)\right)\left(a_{\mathbf{p}}+\frac{i}{\sqrt{2 E_{\mathbf{p}}}} \tilde{\jmath}(p)\right)~.
\end{equation}
源关闭之后,系统的能量为
\begin{equation}
\langle 0|H| 0\rangle=\int \frac{d^{3} p}{(2 \pi)^{3}} \frac{1}{2}|\tilde{\jmath}(p)|^{2}~.
\end{equation}

