% service、systemd、systemctl 笔记1
% license Xiao
% type Note

\begin{issues}
\issueDraft
\end{issues}

\begin{itemize}
\item 要创建一个 service, 就创建文件 \verb|/etc/systemd/system/名字.service|:
\begin{lstlisting}[language=none,caption=名字.service]
[Unit]
Description=服务的描述

[Service]
ExecStart=/路径/命令 参数1 参数2
# 等 15 秒再运行以上命令,防止系统没有加载完成(如网络设置等)
ExecStartPre=/bin/sleep 15

[Install]
WantedBy=multi-user.target
\end{lstlisting}
\item 注意 \verb|ExecStart| 不支持 \verb|>| 和 \verb`|` 等
\item \verb|sudo systemctl daemon-reload| 使 service 文件生效
\item \verb|sudo systemctl enable 名字.service| 就可以设置开机自启服务
\item \verb|disable| 不开机自启
\item \verb|status| 查看状态
\item \verb|start| 启动
\item \verb|stop| 停止
\item \verb|restart| 停止
\item 也可以在 \verb`ps aux | grep 名字` 查看是否在运行
\end{itemize}

You can start and stop it manually with systemctl start scanner.service and systemctl stop scanner.service and see its status with systemctl status scanner.service.

See Arch Linux help about systemd

\subsubsection{更多参数}
\begin{lstlisting}[language=none,caption=名字.service]
[Unit]
Description=服务的描述
Documentation=文档的url
After=在哪些服务运行后运行(如果它们没运行也没关系)
Before=在哪些服务之前运行(如果它们没运行也没关系)
Wants=在运行之前试图运行的另一个服务(如果失败也没关系)
Requires=在运行之前试图运行的另一个服务(如果失败就不运行)

[Service]
ExecStart=/路径/命令 参数1 参数2
ExecStop=/路径/命令 参数1 参数2
ExecStartPre=/路径/命令 参数1 参数2
Restart=always, on-failure, on-abnormal, on-watchdog, on-abort, no.
Type=simple(默认)|forking|oneshot|dbus, notify and idle.
User=运行服务的用户
Group=运行服务的组
Environment=环境变量
EnvironmentFile=环境变量文件

[Install]
WantedBy=在这个 target 启动该 service,失败也没关系
RequiredBy=失败有关系
\end{lstlisting}


\subsection{supervisor}
\begin{itemize}
\item 参考\href{https://www.digitalocean.com/community/tutorials/how-to-install-and-manage-supervisor-on-ubuntu-and-debian-vps}{这里}。
\end{itemize}
