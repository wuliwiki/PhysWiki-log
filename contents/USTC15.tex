% 中国科技大学 2015 年考研普通物理(B)
% keys 中科大|考研|普通物理


\begin{enumerate}
\item (15分)
一枚质量为 $M$ 的火箭,依靠向正下方喷气在空中保持静止,如果喷出气体的速度为  $\mathrm v$ ,求火箭发动机的功率。\\
\begin{figure}[ht]
\centering
\includegraphics[width=8cm]{./figures/1b3a07ac3d3755b3.png}
\caption{简答题2图示} \label{fig_USTC15_1}
\end{figure}
\item (20分)
如图所示,两个可看作质点的物体,用胶粘在一起,由一段长为 $l=1m$ 的不可伸长的轻绳吊在高度为 $h=3m$ 的天花板上。使物体在水平面内以角速度 $\omega=5rad/s$ 做圆周运动。建立如图所示坐标系,使物体运动的平面为 $XY$ 平面。当物体经过 $Y$ 轴正半轴时,胶突然裂开,其中一个物体脱落。求物体掉在地上的坐标。(重力加速度 $g$ 取 $10m/s^2$)
\item (20分)
一个原长为 $l$、弹性系数为 $k$ 的弹性绳固定在一个光滑平面上的固定点 $O$ ,另一端系一质量为 $m$ 的质点,原先静止在平面上,突然给质点施加一个大小为 $I$ 垂直于绳的冲量,在以后运动中绳被拉伸到最大长度 3$l$ ,求冲量 $I$ 。
\item (15分)
产生动生电动势的非静电力是洛伦兹力,该力推动载流子做功。而另一方面,根据洛伦兹力公式 $F=qv\times B$ ,洛伦兹力对带电粒子不做功。这两个陈述是否矛盾,为什么?
\item (20分)
中性氢原子由原子核和电子云组成,电子云电荷密度分布为 $\rho(r)=-Cqe^{-2r/a}$ ,其中 $q$ 为核电荷, $a$ 为玻尔半径, $C$ 为待定系数。\\
(1)确定 $C$ 的数值;\\
(2)求氢原子核与电子云的相互作用能。
\item (20分)
在半径为 $a$ 的细长螺线管中,均匀磁场的磁感应强度随时间均匀增大,可以表示为 $B=B_0+bt$。均匀导线弯成等腰梯形回路 $ABCDA$ ,上底长为 $a$ ,下底长为2 $a$ ,总电阻为 $R$ ,试求\\
(1)梯形各边上的感生电动势及整个回路的感生电动势;\\
(2) $B$ 、$C$ 两点之间的电势差。\\
\begin{figure}[ht]
\centering
\includegraphics[width=5cm]{./figures/a98426a1dac364fd.png}
\caption{简答题6图示} \label{fig_USTC15_2}
\end{figure}
\item (10分)
若钙原子 $(Z=20)$ 除了一个电子之外的所有电子都被移去。\\
(1)计算该离子基态能量\\
(2)求该离子第一激发态的激发能量及相应跃迁的波长。
\item (10分)
氮原子中两个电子处在 $n_{1}pn_{2}p$ 组态,(1)设 $n_1\neq n_2$ ,有哪些可能的原子态?共有多少可能的状态数目?(2)若 $n_1=n_2$ ,又有哪些可能的原子态?为什么 $^{3}D$ 态不可能存在?
\item (20分) 
钠是 $Z=11$ 的碱金属。\\问:(1)钠基态的电子组态是什么?\\(2)该态的量子数各是多少?光谱项表达式是什么?\\(3)其第一激发态的电子组态是什么?光谱项如何写?\\(4)第一激发态向基态的跃迁,是否是允许跃迁?如果是,写出允许的电偶极跃迁。
\end{enumerate}