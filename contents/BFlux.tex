% 磁通量
% 磁通量|闭合曲面|矢势|环路积分

\begin{issues}
\issueDraft
\issueOther{应该由浅入深,矢势放到后面。添加例题。}
\end{issues}

\pentry{曲面积分 通量\upref{SurInt}}

\footnote{参考 \cite{GriffE} 和 Wikipedia \href{https://en.wikipedia.org/wiki/Magnetic_flux}{相关页面}。}令空间中磁感应强度为 $\bvec B(\bvec r)$ 我们可以通过曲面积分来定义一个通过某曲面的\textbf{磁通量(magnetic flux)}为
\begin{equation}
\Phi  = \int \bvec B(\bvec r) \vdot \dd{\bvec s}~.
\end{equation}
形象来说, 磁通量也可以看作是磁感线通过曲面的条数。 反方向的磁感线共线为负,和正方向磁感线抵消。

磁通量只与曲面的边界有关。 这是因为 $\bvec B$ 的散度恒为零(磁场的高斯定律)
\addTODO{证明}


利用磁场矢势% 未完成:链接
及旋度定理, % 未完成:链接
磁通量变为
\begin{equation} \label{eq_BFlux_2}
\Phi  = \int \curl \bvec A \vdot \dd{\bvec s}  = \oint \bvec A \vdot \dd{\bvec r}~.
\end{equation}
另外, 由于磁场的散度为零% 未完成:链接
, 根据高斯定律, 任何闭合曲面的磁通量都是 0。 用另一种方式来理解: 如果选定一个闭合回路, 以该闭合回路为边界的任何曲面的磁通量都相等。

\subsection{闭合线圈的磁通量}

如何计算一个闭合线圈对自己产生的磁通量呢? 利用磁场矢势公式
\begin{equation}
\bvec A \qty(\bvec r) = \frac{\mu_0 I}{4 \pi} \oint \frac{\dd{\bvec r'}}{\abs{\bvec r - \bvec r'}}~.
\end{equation}
注意在该积分中, $\bvec r$ 视为常量, 积份完后, 积分变量 $\bvec r$ 消失。 现在根据\autoref{eq_BFlux_2} 再次将上式对 $\bvec r$ 进行同一环路积分得到磁通量
\begin{equation}
\Phi  = \frac{\mu_0 I}{4\pi} \oint\oint \frac{\dd{\bvec r'} \dd{\bvec r}}{\abs{\bvec r - \bvec r'}}~.
\end{equation}
