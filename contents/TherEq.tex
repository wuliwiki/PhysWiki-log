% 热平衡、热力学第零定律
% 热平衡|热力学第零定律|温度|容器

\begin{issues}
\issueDraft
\end{issues}

\subsection{热平衡}
热力学研究的对象是一个由大量微观粒子(分子或其他粒子)组成的一个宏观物质系统(例如一个绝热容器中的气体)。经验指出,一个孤立系统(与外界没有物质和能量交换的系统)若放置得足够久,将会达到这样一种状态——系统的各种\textbf{宏观性质}(例如温度、压强、化学势等物理性质和化学性质)在长时间内不发生任何变化。这称为热力学平衡态。

一般来说,热平衡的具体要求有热学平衡、化学平衡和力学平衡\upref{equcri}。简单地来说,就是系统内温度处处相等,没有外场的情况下压强处处相等,化学组成处处相同。这些都是宏观可观测的性质,例如热学平衡意味着没有热流,力学平衡意味着没有粒子宏观流动,而化学平衡意味着没有扩散。对于等温大气模型,虽然不同高度不等压,但由于重力场,气体仍处于力学平衡状态,没有粒子宏观流动,所以也是热平衡系统。

对于非平衡系统,我们可以假定局域平衡,即把系统划分成很多个子系,这些子系包含足够多的粒子,而且又足够小,使得热力学参量(如压强、体积、温度)处处相等。用这种方法,我们也可以非平衡系统作宏观描述。

\subsection{热力学第零定律}
当我们提及\textbf{温度},我们会认为它是度量了一个系统的冷热程度的物理量,或者说这个物理量衡量了系统自发放热的能力(温度越高,那么它的放热能力应当越强。但这些都是基于经验的“直觉”,并非温度的定义。要考虑温度,我们必须思考热平衡系统的“属性”以及不同系统之间的关系。

为了阐释清楚温度概念,人们提出了\textbf{热力学第零定律}:
若物体 $A$ 与物体 $C$ 达到热平衡(将它们接触时没有“热量”传递,表征为宏观性质没有发生变化), 且物体 $B$ 与物体 $C$ 也达到热平衡, 那么 $A$ 和 $B$ 之间同样有热平衡。上面的 $A,B,C$ 可以替换成装有气体的导热容器,定律仍然成立。

这意味着我们可以引入一个物理量(称它为\textbf{“温度”}),\textbf{用同一个“温度”值来标定一切处于热平衡的的物质}。因此热平衡定律指明了比较温度的方法。
\begin{figure}[ht]
\centering
\includegraphics[width=10cm]{./figures/ff0f583180bdf981.png}
\caption{热力学第零定律} \label{fig_TherEq_1}
\end{figure}

过去人们一直将温度概念和热量联系起来;人们将“热”看作一种实体物质,认为温度高的物体有较多的热量。 19世纪40年代,这一错误概念得以澄清,温度与热量这两个概念得以区分。在微观上,温度是处于热平衡系统的微观粒子热运动强弱程度的量度。我们必须对\textbf{温度}概念作出更严格的定义\upref{tmp}。

\addTODO{温度计的概念}
