% 韦恩图
% license Usr
% type Tutor

% 原作者:叶燊Leafshen

\begin{issues}
\issueDraft
\end{issues}

\pentry{集合(高中)\nref{nod_HsSet}}{nod_bcff}

\subsection{韦恩图}
\addTODO{不要出现没有介绍过的术语 “类”}
\textbf{韦恩图(Veen diagram)},是在所谓的集合论(或者类的理论)数学分支中,在不太严格的意义下用以表示集合(或类)的一种草图。它们用于展示在不同的事物群组(集合)之间的数学或逻辑联系,尤其适合用来表示集合(或)类之间的“大致关系”。它也常常被用来帮助推导(或理解推导过程)关于集合运算(或类运算)的一些规律。
\begin{figure}[ht]
\centering
\includegraphics[width=10cm]{./figures/e449e54347ae8e24.png}
\caption{Veen图} \label{fig_SufCnd_1}
\end{figure}
   条件越多,在韦恩图上范围就越小,这似乎很好理解。

   我们将全集用\textbf{黄色}来表示,一个个限制条件瓦解侵蚀,我们用\textbf{粉色}来标记。那么,如图所示:
   \addTODO{粉色标记什么?符合条件还是不符合条件?需要体现粉色重合的情况}
\begin{figure}[ht]
\centering
\includegraphics[width=10cm]{./figures/3c78833bb0aba8e4.png}
\caption{条件越多,在韦恩图上范围就越小} \label{fig_SufCnd_6}
\end{figure}
