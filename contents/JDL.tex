% 角动量(综述)
% license CCBYSA3
% type Wiki

(本文根据 CC-BY-SA 协议转载翻译自维基百科\href{https://en.wikipedia.org/wiki/Angular_momentum}{相关文章})

\begin{figure}[ht]
\centering
\includegraphics[width=6cm]{./figures/0e2f9582f8dddda6.png}
\caption{这个陀螺仪在旋转时由于角动量守恒而保持直立。常用符号:\( L \)   在国际单位制中的基本单位:kg⋅m²⋅s⁻¹   是否守恒:是   由其他量推导:\( L = I\omega = r \times p \)   量纲:\( \mathsf{M L^2 T^{-1}} \)} \label{fig_JDL_1}
\end{figure}
\textbf{角动量}(有时称为动量矩或旋转动量)是线性动量的旋转类比。它是一个重要的物理量,因为它是守恒量——封闭系统的总角动量保持不变。角动量既有方向也有大小,并且两者都守恒。自行车和摩托车、飞盘、膛线子弹以及陀螺仪的有用特性都归因于角动量守恒。角动量守恒也是飓风形成螺旋状以及中子星具有高速旋转率的原因。通常,守恒定律限制了系统可能的运动,但并不能唯一确定其运动方式。

三维角动量在经典力学中表示为伪向量 \( \mathbf{r} \times \mathbf{p} \),即粒子位置向量 \( \mathbf{r} \)(相对于某一原点)与其动量向量的叉积;后者在牛顿力学中为 \( \mathbf{p} = m\mathbf{v} \)。与线性动量不同,角动量取决于原点的选择,因为粒子的位置是从该原点测量的。

角动量是一种广延量;也就是说,任何复合系统的总角动量是其组成部分角动量的总和。对于连续的刚体或流体,系统的总角动量是角动量密度(即单位体积的角动量,当体积趋近于零时)的体积分在整个物体上的积分。

类似于线性动量守恒,如果没有外力作用,线性动量守恒;同样,如果没有外力矩作用,角动量也守恒。力矩可以定义为角动量的变化率,类似于力的作用。任何系统的净外力矩始终等于系统上的总力矩;系统内的所有内力矩之和总是为0(这是牛顿第三运动定律的旋转类比)。因此,对于封闭系统(没有净外力矩),系统的总力矩必须为0,这意味着系统的总角动量是恒定的。

特定相互作用下角动量的变化称为\textbf{角冲量},有时称为“旋转”。角冲量是线性冲量的旋转类比。[3]

\subsection{示例}

对于一个处于轨道中的物体,其角动量 \( L \) 的简单情况表示为:
\[
L = 2\pi Mfr^2~
\]
其中,\( M \) 是轨道物体的质量,\( f \) 是轨道的频率,\( r \) 是轨道的半径。

对于一个绕其轴旋转的均匀刚性球体,其角动量 \( L \) 表示为:
\[
L = \frac{4}{5} \pi Mfr^2~
\]
其中,\( M \) 是球体的质量,\( f \) 是旋转的频率,\( r \) 是球体的半径。

例如,地球相对于太阳的轨道角动量约为 \( 2.66 \times 10^{40} \) kg⋅m²⋅s⁻¹,而地球的自转角动量约为 \( 7.05 \times 10^{33} \) kg⋅m²⋅s⁻¹。

对于一个绕其轴旋转的均匀刚性球体,如果已知其密度而非质量,其角动量 \( L \) 表示为:
\[
L = \frac{16}{15} \pi^2 \rho fr^5~
\]
其中,\( \rho \) 是球体的密度,\( f \) 是旋转频率,\( r \) 是球体的半径。

对于最简单的旋转圆盘,角动量 \( L \) 表示为:[4]
\[
L = \pi Mfr^2~
\]
其中,\( M \) 是圆盘的质量,\( f \) 是旋转频率,\( r \) 是圆盘的半径。

如果圆盘绕其直径旋转(例如抛硬币时),其角动量 \( L \) 表示为:[4]
\[
L = \frac{1}{2} \pi Mfr^2~
\]
\subsection{经典力学中的定义}   
“轨道角动量”重定向至此。关于其他用途,参见\textbf{轨道角动量(消歧义)}。  

“自旋角动量”重定向至此。关于其他用途,参见\textbf{自旋角动量(消歧义)}。

正如角速度一样,物体的角动量有两种特殊类型:\textbf{自旋角动量}是相对于物体质心的角动量,而\textbf{轨道角动量}是相对于选定旋转中心的角动量。例如,地球因为围绕太阳公转而具有轨道角动量,同时由于每天绕极轴自转而具有自旋角动量。总角动量是自旋角动量和轨道角动量的总和。在地球的情况下,主要守恒量是太阳系的总角动量,因为行星与太阳之间会有少量但重要的角动量交换。质点的轨道角动量矢量始终与其轨道角速度矢量 \( \omega \) 平行且成正比,比例常数取决于粒子的质量和其距离原点的距离。刚体的自旋角动量矢量与自旋角速度矢量 \( \Omega \) 成正比,但不总是平行,因此比例常数是一个二阶张量而非标量。
\subsection{二维中的轨道角动量}
角动量是一个矢量量(更准确地说是伪矢量),它表示物体的转动惯量与绕某特定轴的转速(以弧度/秒为单位)的乘积。然而,如果粒子的轨迹位于一个平面内,则可以忽略角动量的矢量性质,将其作为标量(更准确地说是伪标量)处理。[5] 角动量可以看作是线性动量的旋转类比。因此,线性动量 \( p \) 与质量 \( m \) 和线速度 \( v \) 成正比:
\[
p = mv,~
\]

角动量 \( L \) 与转动惯量 \( I \) 和以弧度/秒为单位的角速度 \( \omega \) 成正比。[6]
\[
L = I\omega.~
\]

与只依赖物质数量的质量不同,转动惯量还取决于旋转轴的位置以及物质的分布。与不依赖原点选择的线速度不同,轨道角速度总是相对于固定原点测量。因此,严格来说,\( L \) 应被称为相对于该中心的角动量。[7]

对于单个粒子的圆周运动,可以使用\(I = r^2 m\)和\(\omega = v/r\)来展开角动量为\(L = r^2 m \cdot v/r\)简化为:
\[
L = rmv.~
\]
角动量是旋转半径 \( r \) 和粒子的线动量 \( p = mv \) 的乘积,其中\(v = r\omega\)是线速度(切向速度)。

如果使用与半径向量垂直的运动分量,这种简单的分析也适用于非圆周运动:
\[
L = rmv_{\perp},~
\]
其中\(v_{\perp} = v\sin(\theta)\)是运动的垂直分量。展开为:\(L = rmv\sin(\theta)\),重新排列后为:\(L = r\sin(\theta) mv\),简化后,角动量也可以表示为:
\[
L = r_{\perp} mv,~
\]
其中\(r_{\perp} = r\sin(\theta)\)是力矩臂的长度,即从原点垂直投影到粒子轨迹上的线段。这个定义(力矩臂长度 × 线动量)即为“动量矩”一词的来源。[8]
\subsubsection{来自拉格朗日力学的标量角动量 } 
另一种方法是将角动量定义为机械系统的角坐标 \(\phi\) 的共轭动量(也称为正则动量)。考虑一个质量为 \(m\) 的机械系统,它被约束在半径为 \(r\) 的圆上运动,且没有外部力场的存在。系统的动能为” 
\[ T = \frac{1}{2}mr^2\omega^2 = \frac{1}{2}mr^2{\dot{\phi}}^2. ~\]
势能为:  
\[ U = 0.~ \]
因此,拉格朗日量为:  
\[ \mathcal{L}(\phi, \dot{\phi}) = T - U = \frac{1}{2}mr^2{\dot{\phi}}^2. ~\]
广义动量,即与坐标 \(\phi\) 正则共轭的动量,定义为:  
\[ p_\phi = \frac{\partial \mathcal{L}}{\partial \dot{\phi}} = mr^2{\dot{\phi}} = I\omega = L.~ \]
这里,\(L\) 是角动量,\(I\) 是转动惯量,\(\omega\) 是角速度。
\subsubsection{轨道角动量在三维空间中的描述}
\begin{figure}[ht]
\centering
\includegraphics[width=6cm]{./figures/94214524fc2d2d79.png}
\caption{“在旋转系统中,力(\( F \))、力矩(\( \tau \))、动量(\( p \))和角动量(\( L \))向量之间的关系。\( r \) 是位置向量。”} \label{fig_JDL_2}
\end{figure}
要完全定义三维中的轨道角动量,需要知道位置向量扫过角度的速率、垂直于瞬时角位移平面的方向、相关的质量以及该质量在空间中的分布情况。[9] 通过保留角动量的向量性质,可以保留方程的一般性质,从而能够描述绕旋转中心的任何类型的三维运动——无论是圆周运动、线性运动还是其他形式的运动。用向量表示法,绕原点运动的点质点的轨道角动量可以表示为:
\[ \mathbf {L} = I{\boldsymbol {\omega }}, ~\]
其中:
\begin{itemize}
\item \(I = r^2 m\) 是质点的转动惯量,
\item \({\boldsymbol {\omega }} = \frac {\mathbf {r} \times \mathbf {v} }{r^2}\)是该质点绕原点的轨道角速度,
\item \(\mathbf {r} \)是质点相对于原点的位置向量,\(r = |\mathbf {r}|\)是位置向量的模,
\item \(\mathbf {v} \)是质点相对于原点的线速度,
\item \(m \)是质点的质量。
\end{itemize}
这个表达式可以展开、化简,并通过向量代数的规则重新排列:
\[ \mathbf {L} = \left(r^2 m\right)\left(\frac {\mathbf {r} \times \mathbf {v} }{r^2}\right) = m\left(\mathbf {r} \times \mathbf {v} \right) = \mathbf {r} \times m\mathbf {v} = \mathbf {r} \times \mathbf {p},~ \]
其中,\(\mathbf {p}\)是质点的动量。

这是位置向量 \( \mathbf{r} \) 和粒子的线动量 \( \mathbf{p} = m\mathbf{v} \) 的叉积。根据叉积的定义,角动量向量 \( \mathbf{L} \) 垂直于 \( \mathbf{r} \) 和 \( \mathbf{p} \)。它的方向垂直于角位移的平面,按照右手法则指示的方向——因此角速度从向量的顶端看是逆时针的。相反,角动量向量 \( \mathbf{L} \) 定义了 \( \mathbf{r} \) 和 \( \mathbf{p} \) 所在的平面。

通过定义一个垂直于角位移平面的单位向量 \( \mathbf{\hat{u}} \),可以得到一个标量角速度 \( \omega \),其公式为\(\omega \mathbf{\hat{u}} = {\boldsymbol{\omega}},\)并且\(\omega = \frac{v_{\perp}}{r}, \)
其中 \( v_{\perp} \) 是运动的垂直分量,如上所述。

因此,前一节中的二维标量方程可以赋予方向:
\[
\mathbf{L} = I{\boldsymbol{\omega}} = I\omega \mathbf{\hat{u}} = \left(r^{2}m\right)\omega \mathbf{\hat{u}} = rmv_{\perp} \mathbf{\hat{u}} = r_{\perp}mv \mathbf{\hat{u}},~
\]
对于圆周运动,其中所有运动都垂直于半径 \( r \)。

在球坐标系中,角动量向量表达为:
\[
\mathbf{L} = m\mathbf{r} \times \mathbf{v} = mr^{2} \left( \dot{\theta} \, {\hat{\boldsymbol{\varphi}}} - \dot{\varphi} \sin \theta \, \mathbf{\hat{\boldsymbol{\theta}}} \right).~
\]
\subsection{与线性动量的类比 } 
角动量可以描述为线性动量的旋转类比。与线性动量类似,它涉及质量和位移的元素。不像线性动量,它还涉及位置和形状的元素。  

物理学中的许多问题涉及围绕某个空间点的物质运动,无论是实际围绕它旋转,还是简单地经过它,在这些情况下需要知道移动的物质对该点有什么影响——它是否可以对该点施加能量或在其周围做功?能量,即做功的能力,可以通过让物质运动来储存——这是一种惯性和位移的组合。惯性通过其质量来衡量,位移通过其速度来衡量。它们的乘积:

(惯量) × (位移量) = (惯量·位移量)  
质量 × 速度 = 动量  
\( m \times v = p \)

是物质的动量。[10] 将这个动量参照一个中心点会引入一个复杂性:动量并不是直接作用在该点上的。例如,一个位于车轮外缘的物质颗粒实际上位于一个与车轮半径相同长度的杠杆的末端,其动量使杠杆围绕中心点旋转。这个假想的杠杆被称为力臂。它的效果是按其长度的比例放大动量的作用,这种效果被称为力矩。因此,物质颗粒相对于某个特定点的动量为:

(力臂)×(惯量)×(位移量)=(惯量·位移量的力矩)  
长度 × 质量 × 速度 = 动量矩  
\( r \times m \times v = L \)

角动量,有时也被称为相对于特定中心点的动量矩。方程\( L = rmv \) 将力矩(质量 \( m \) 旋转力臂 \( r \))与线性(直线等效)速度 \( v \) 结合起来。相对于中心点的线速度只是距离 \( r \) 和相对于该点的角速度 \( \omega \) 的乘积:  \( v = r\omega \),这也是一个力矩。因此,角动量包含双重力矩:  \( L = rmr\omega \)。  稍作简化,\( L = r^{2}m\omega \),其中量 \( r^{2}m \) 是粒子的转动惯量,有时也被称为质量的二阶矩。它是旋转惯性的量度。[11]
\begin{figure}[ht]
\centering
\includegraphics[width=6cm]{./figures/889a66f16cec8566.png}
\caption{转动惯量(如图所示),因此角动量,对于每种质量配置和旋转轴的不同,都有所不同。} \label{fig_JDL_3}
\end{figure}
上述平移动量和旋转动量的类比可以用矢量形式表示:[需要引用]
\begin{itemize}
\item \(\mathbf{p} = m\mathbf{v} \) (线性运动中的动量)
\item \( \mathbf{L} = I{\boldsymbol{\omega}} \) (旋转中的角动量)
\end{itemize}

动量的方向与线性运动中的速度方向相关,而角动量的方向与旋转的角速度相关。

由于转动惯量是自旋角动量的关键部分,后者必然包含前者的所有复杂性。转动惯量是通过将质量的基本部分与它们到旋转中心的距离的平方相乘来计算的。因此,总的转动惯量和角动量是关于物质配置及其旋转方向的复杂函数。

对于刚体,例如一个轮子或小行星,旋转的方向只是相对于物体本身物质的旋转轴的位置。这个轴可能穿过质心,也可能不穿过,甚至可能完全位于物体之外。对于同一个物体,角动量可能会对每一个可能的旋转轴取不同的值。当旋转轴通过质心时,角动量达到最小值。

对于围绕某个中心旋转的物体集合,例如太阳系中的所有天体,旋转的方向可能是有些组织化的,就像太阳系一样,大多数天体的旋转轴接近系统的轴。它们的方向也可能完全随机。

简言之,质量越大,距离旋转中心越远(即力臂越长),则转动惯量越大,因此在给定的角速度下,角动量越大。在许多情况下,转动惯量以及角动量可以简化为
\[ I = k^2 m,~ \]
其中 \( k \) 是回转半径,即可以认为整个质量 \( m \) 集中在距离轴的距离。

类似地,对于一个点质量 \( m \),转动惯量定义为
\[ I = r^2 m,~\]
其中 \( r \) 是点质量距离旋转中心的半径。

对于任意多个粒子 \( m_i \),转动惯量为各粒子的总和:
\[ \sum_{i} I_i = \sum_{i} r_i^2 m_i. ~\]
角动量对位置和形状的依赖性反映在其单位与线动量的单位差异上:角动量的单位是 kg⋅m²/s 或 N⋅m⋅s,而线动量的单位是 kg⋅m/s 或 N⋅s。当计算角动量时,使用转动惯量乘以角速度,角速度必须以弧度每秒表示,其中弧度被视为无量纲单位,取值为 1。(在进行量纲分析时,可以使用将弧度视为基本单位的方向分析,但这在国际单位制中并不使用。)角动量的单位可以解释为力矩⋅时间。具有角动量 L N⋅m⋅s 的物体可以通过一个大小为 L N⋅m⋅s 的角冲量将其角速度降为零。

垂直于角动量轴并通过质心的平面有时被称为不变平面,因为如果只考虑系统内部物体之间的相互作用,而不受外部影响,该轴的方向将保持不变。其中一个这样的平面是太阳系的不变平面。
\subsubsection{角动量与力矩}  
牛顿第二运动定律可以通过以下数学公式表达:
\[ F = ma ~\]
即,力 = 质量 × 加速度。对于点粒子,旋转等效方程可以推导如下:
\[ L = I\omega ~\]
这意味着力矩(即角动量的时间导数)为:
\[ \tau = \frac{dI}{dt} \omega + I\frac{d\omega}{dt} ~\]
由于转动惯量是 \( I = mr^2 \),因此:
\[ \frac{dI}{dt} = 2mr\frac{dr}{dt} = 2rp_{\parallel} ~\]
并且:
\[ \frac{dL}{dt} = I\frac{d\omega}{dt} + 2rp_{\parallel} \omega ~\]
简化为:
\[ \tau = I\alpha + 2rp_{\parallel} \omega ~\]
这就是牛顿第二定律的旋转类比。需要注意的是,力矩不一定与角加速度成正比或平行(如预期的那样)。其原因在于粒子的转动惯量可以随时间变化,而对于普通的质量而言,这种情况是不会发生的。