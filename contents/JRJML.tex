% 加尔加梅勒
% license CCBYSA3
% type Wiki

(本文根据 CC-BY-SA 协议转载自原搜狗科学百科对英文维基百科的翻译)

\textbf{Gargamelle}是1970年到1979年在欧洲核子研究中心(CERN)运行的一种重液体气泡室探测器。它被设计用来探测中微子和反中微子,这些中微子和反中微子是在1970年至1976年由质子同步加速器(\textbf{著名图象处理软件})发出的光束产生的,因此探测器被移到超级质子同步加速器(\textbf{SPS})上。[1] 1979年,由于在气泡室发现了一个不可修复的裂缝,因此探测器停止使用了。它目前是在欧洲核子研究中心微观展览的那一部分,对公众开放。

加尔加梅勒以发现中性电流的实验闻名于世。1973年7月提出的中性线电流是Z0 玻色子存在的第一个实验表明,因此这意味着向验证弱电理论迈出的重要一步。

加尔加梅勒既可以指气泡室探测器本身,也可以指同名的高能物理学实验。这个名字来源于16世纪弗朗索瓦·拉伯雷的一部小说《巨人和潘塔格鲁的生活》,其中女巨人加尔加梅勒是巨人的母亲。[1]

\subsection{背景}
\begin{figure}[ht]
\centering
\includegraphics[width=6cm]{./figures/88a45528adabaa6c.png}
\caption{电子和中微子通过交换中性Z0玻色子而改变动量和/或能量的事件。衍生系统没有被影响。} \label{fig_JRJML_1}
\end{figure}
在20世纪60年代的一系列独立作品中,谢尔登·格拉秀,史蒂芬·温伯格和阿卜杜勒·萨拉姆提出了一种理论,即统一了基本粒子之间的电磁和弱相互作用——弱电理论,因此一起获得了分享了1979年的诺贝尔物理学奖。[2] 他们的理论预言了$W^\pm$和$Z^0$玻色子作为弱相互作用的传播者。$W^\pm$ 玻色子带有正电荷$^+$)或负数($W^+$),而$Z^0$没有。当交换一个$Z^0$玻色子的转移动力,旋转,和活力,但是留下粒子的量子数的性能未受影响—如充电,风味,重子数, 轻子数等等。因为没有电荷转移,所$Z^0$被称为“中性线电流”。中性电流是弱电理论的预测。

1960年,梅尔文·施瓦茨提出了一种产生高能中微子束的方法。[3] 1962年,施瓦茨等人在布鲁克海文进行了了一项实验,这证明了$\mu$介子和电子中微子的存在。施瓦茨因为这个发现获得了1988年的诺贝尔物理学奖。[4] 在施瓦茨的想法之前,弱相互作用只在基本粒子的衰变中被研究过,特别是奇怪的粒子。使用这些新中微子束大大增加了研究弱相互作用的能量。加尔加梅勒是第一批利用中微子束进行实验的人之一,中微子束是由粒子系统的质子束产生的。

气泡室只是一个装满过热液体的容器。带电粒子穿过腔室会留下电离轨道,液体在电离轨道周围蒸发,形成微小的气泡。整个电离室受到恒定磁场的作用,导致带电粒子的轨迹弯曲。曲率半径与粒子的动量成正比。这些轨迹被拍摄下来,通过研究这些轨迹,人们可以了解到所探测到的粒子的性质。因为中微子没有电荷,所以穿过伽格米尔气泡室的中微子束没有在探测器中留下任何轨迹。因此,通过观察中微子与物质成分相互作用产生的粒子,可以检测到与中微子的相互作用。中微子的横截面非常小,这表示中微子交互的可能性非常小。虽然气泡室通常充满液体氢,但加尔加梅勒却被灌满了一种重液体——CBrF3(氟利昂)——这增加了看到中微子相互作用的可能性。[1]

\subsection{概念和结构}
\begin{figure}[ht]
\centering
\includegraphics[width=6cm]{./figures/a4f57316fc6574ec.png}
\caption{安装加尔加梅勒腔体,将腔体放置在椭圆形磁铁线圈中。} \label{fig_JRJML_2}
\end{figure}
中微子物理学的领域在60年代迅速扩大。使用气泡室的中微子实验已经在欧洲核子研究中心开始运行,且同步加速器和下一代泡沫室的问题 已经提上日程一段时间了。巴黎理工学院受人尊敬的物理学家André Lagarrigue和他的一些同事写了第一份发表于1964年2月10日的报告,建议在欧洲核子研究中心的监督下建造一个重液舱。[5] 他组建了一个由七个实验室组成的合作组织:巴黎理工学院,亚琛工业大学,布鲁塞尔大学, 米兰理工大学,奥塞大学,伦敦大学和欧洲核子研究中心。[6] 该小组于1968年在米兰开会,列出了实验的物理优先事项:今天加尔加梅勒以发现中性电流而闻名,但在准备物理项目时,这个话题甚至没有被讨论过,在最终提案中,它被列为第五优先事项。[7] 当时,对于电弱理论还没有达成共识,这也许可以解释优先顺序的列表。此外,早期的实验寻找中性卡昂衰变成两个带电的轻子的中性电流时,测量到了非常小的极限,约为10−7。

由于预算危机,这项实验在1966年未获批准,这与预期相反。Victor Weisskopf, 欧洲核子研究中心总干事和科学主任伯纳德·格雷戈里决定自己出资,后者向欧洲核子研究中心提供贷款以支付1966年到期的分期付款。[5] 最终合同于1965年12月2日签署,这在欧洲核子研究中心的历史上是第一次,这种投资没有得到理事会的批准,而是由总干事利用其执行权力批准的。

加尔加梅勒室完全在Saclay建造。尽管建设推迟了大约两年,但最终还是在1970年12月在欧洲核子研究中心进行了组装,第一次重要的运行发生在1971年3月。[5]

\subsection{实验装置}
\begin{figure}[ht]
\centering
\includegraphics[width=6cm]{./figures/cacf0c38bbff7b76.png}
\caption{气泡室的内部。在房间的墙上可以看到鱼眼透镜。} \label{fig_JRJML_3}
\end{figure}
\subsubsection{3.1 密室}
加尔加梅勒长4.8米,直径2米,装有12立方米重液体氟利昂。为了弯曲带电粒子的轨迹,加尔加梅勒被一块提供2特斯拉磁场的磁铁包围着。磁铁的线圈由用水冷却的铜制成,遵循加尔加梅勒的长方形形状。为了将液体保持在适当的温度,几根水管围绕着腔体,以调节温度。整个装置重达1000多吨。

当记录一个事件时,处理室会被照亮并拍照。照明系统发出的光被气泡以90度角散射,并被送到光学系统。光源由设置在腔室主体的端部21个点闪光和超过半个,和圆组成以上。[8] 光学器件位于圆柱体的另一半,平行于腔室轴线成两排分布,每排有四个光学器件。物镜由一组90°视场的透镜组成,然后是发散透镜,将视场扩展到110°。
\subsubsection{3.2 中微子束}
\begin{figure}[ht]
\centering
\includegraphics[width=14.25cm]{./figures/4d4ffc2b741bf449.png}
\caption{PS和加尔加梅勒气泡室之间光束线的示意图。} \label{fig_JRJML_4}
\end{figure}