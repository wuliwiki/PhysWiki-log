% 加尔加梅勒
% license CCBYSA3
% type Wiki

(本文根据 CC-BY-SA 协议转载自原搜狗科学百科对英文维基百科的翻译)

\begin{figure}[ht]
\centering
\includegraphics[width=6cm]{./figures/40960bacfbc83357.png}
\caption{1977年2月,欧洲核子研究中心西侧大厅中的加尔加梅勒气泡室探测器检视图。} \label{fig_JRJML_7}
\end{figure}
\textbf{Gargamelle}是1970年到1979年在欧洲核子研究中心(CERN)运行的一种重液体气泡室探测器。它被设计用来探测中微子和反中微子,这些中微子和反中微子是在1970年至1976年由质子同步加速器(\textbf{著名图象处理软件})发出的光束产生的,因此探测器被移到超级质子同步加速器(\textbf{SPS})上。[1] 1979年,由于在气泡室发现了一个不可修复的裂缝,因此探测器停止使用了。它目前是在欧洲核子研究中心微观展览的那一部分,对公众开放。

加尔加梅勒以发现中性电流的实验闻名于世。1973年7月提出的中性线电流是Z0 玻色子存在的第一个实验表明,因此这意味着向验证弱电理论迈出的重要一步。

加尔加梅勒既可以指气泡室探测器本身,也可以指同名的高能物理学实验。这个名字来源于16世纪弗朗索瓦·拉伯雷的一部小说《巨人和潘塔格鲁的生活》,其中女巨人加尔加梅勒是巨人的母亲。[1]

\subsection{背景}
\begin{figure}[ht]
\centering
\includegraphics[width=6cm]{./figures/88a45528adabaa6c.png}
\caption{电子和中微子通过交换中性Z0玻色子而改变动量和/或能量的事件。衍生系统没有被影响。} \label{fig_JRJML_1}
\end{figure}
在20世纪60年代的一系列独立作品中,谢尔登·格拉秀,史蒂芬·温伯格和阿卜杜勒·萨拉姆提出了一种理论,即统一了基本粒子之间的电磁和弱相互作用——弱电理论,因此一起获得了分享了1979年的诺贝尔物理学奖。[2] 他们的理论预言了$W^\pm$和$Z^0$玻色子作为弱相互作用的传播者。$W^\pm$ 玻色子带有正电荷$^+$)或负数($W^+$),而$Z^0$没有。当交换一个$Z^0$玻色子的转移动力,旋转,和活力,但是留下粒子的量子数的性能未受影响—如充电,风味,重子数, 轻子数等等。因为没有电荷转移,所$Z^0$被称为“中性线电流”。中性电流是弱电理论的预测。

1960年,梅尔文·施瓦茨提出了一种产生高能中微子束的方法。[3] 1962年,施瓦茨等人在布鲁克海文进行了了一项实验,这证明了$\mu$介子和电子中微子的存在。施瓦茨因为这个发现获得了1988年的诺贝尔物理学奖。[4] 在施瓦茨的想法之前,弱相互作用只在基本粒子的衰变中被研究过,特别是奇怪的粒子。使用这些新中微子束大大增加了研究弱相互作用的能量。加尔加梅勒是第一批利用中微子束进行实验的人之一,中微子束是由粒子系统的质子束产生的。

气泡室只是一个装满过热液体的容器。带电粒子穿过腔室会留下电离轨道,液体在电离轨道周围蒸发,形成微小的气泡。整个电离室受到恒定磁场的作用,导致带电粒子的轨迹弯曲。曲率半径与粒子的动量成正比。这些轨迹被拍摄下来,通过研究这些轨迹,人们可以了解到所探测到的粒子的性质。因为中微子没有电荷,所以穿过伽格米尔气泡室的中微子束没有在探测器中留下任何轨迹。因此,通过观察中微子与物质成分相互作用产生的粒子,可以检测到与中微子的相互作用。中微子的横截面非常小,这表示中微子交互的可能性非常小。虽然气泡室通常充满液体氢,但加尔加梅勒却被灌满了一种重液体——CBrF3(氟利昂)——这增加了看到中微子相互作用的可能性。[1]

\subsection{概念和结构}
\begin{figure}[ht]
\centering
\includegraphics[width=6cm]{./figures/a4f57316fc6574ec.png}
\caption{安装加尔加梅勒腔体,将腔体放置在椭圆形磁铁线圈中。} \label{fig_JRJML_2}
\end{figure}
中微子物理学的领域在60年代迅速扩大。使用气泡室的中微子实验已经在欧洲核子研究中心开始运行,且同步加速器和下一代泡沫室的问题 已经提上日程一段时间了。巴黎理工学院受人尊敬的物理学家André Lagarrigue和他的一些同事写了第一份发表于1964年2月10日的报告,建议在欧洲核子研究中心的监督下建造一个重液舱。[5] 他组建了一个由七个实验室组成的合作组织:巴黎理工学院,亚琛工业大学,布鲁塞尔大学, 米兰理工大学,奥塞大学,伦敦大学和欧洲核子研究中心。[6] 该小组于1968年在米兰开会,列出了实验的物理优先事项:今天加尔加梅勒以发现中性电流而闻名,但在准备物理项目时,这个话题甚至没有被讨论过,在最终提案中,它被列为第五优先事项。[7] 当时,对于电弱理论还没有达成共识,这也许可以解释优先顺序的列表。此外,早期的实验寻找中性卡昂衰变成两个带电的轻子的中性电流时,测量到了非常小的极限,约为10−7。

由于预算危机,这项实验在1966年未获批准,这与预期相反。Victor Weisskopf, 欧洲核子研究中心总干事和科学主任伯纳德·格雷戈里决定自己出资,后者向欧洲核子研究中心提供贷款以支付1966年到期的分期付款。[5] 最终合同于1965年12月2日签署,这在欧洲核子研究中心的历史上是第一次,这种投资没有得到理事会的批准,而是由总干事利用其执行权力批准的。

加尔加梅勒室完全在Saclay建造。尽管建设推迟了大约两年,但最终还是在1970年12月在欧洲核子研究中心进行了组装,第一次重要的运行发生在1971年3月。[5]

\subsection{实验装置}
\begin{figure}[ht]
\centering
\includegraphics[width=6cm]{./figures/cacf0c38bbff7b76.png}
\caption{气泡室的内部。在房间的墙上可以看到鱼眼透镜。} \label{fig_JRJML_3}
\end{figure}
\subsubsection{3.1 密室}
加尔加梅勒长4.8米,直径2米,装有12立方米重液体氟利昂。为了弯曲带电粒子的轨迹,加尔加梅勒被一块提供2特斯拉磁场的磁铁包围着。磁铁的线圈由用水冷却的铜制成,遵循加尔加梅勒的长方形形状。为了将液体保持在适当的温度,几根水管围绕着腔体,以调节温度。整个装置重达1000多吨。

当记录一个事件时,处理室会被照亮并拍照。照明系统发出的光被气泡以90度角散射,并被送到光学系统。光源由设置在腔室主体的端部21个点闪光和超过半个,和圆组成以上。[8] 光学器件位于圆柱体的另一半,平行于腔室轴线成两排分布,每排有四个光学器件。物镜由一组90°视场的透镜组成,然后是发散透镜,将视场扩展到110°。
\subsubsection{3.2 中微子束}
\begin{figure}[ht]
\centering
\includegraphics[width=14.25cm]{./figures/4d4ffc2b741bf449.png}
\caption{PS和加尔加梅勒气泡室之间光束线的示意图。} \label{fig_JRJML_4}
\end{figure}
加尔加梅勒是为中微子和反中微子探测而设计的。中微子和反中微子的来源是来自质子同步加速器的能量为26千兆瓦的质子束。质子被磁铁提取,然后被引导通过适当的四极和偶极磁铁阵列,在位置和方向上提供必要的自由度,以调整射束到目标上。目标是一个90 厘米长,直径为5毫米的圆筒铍。[8] 选择目标材料,使得碰撞中产生的强子主要是π介子和kaons,它们都会都衰变为中微子。产生的π介子和kaons具有多种角度和能量,因此它们的衰变产物也将具有巨大的动量扩散。由于中微子没有电荷,它们不能被电场或磁场聚焦。相反,人们通过使用一个由诺贝尔奖得主西蒙·范·德·梅尔发明的磁性喇叭来聚焦二次粒子。喇叭的形状和磁场的强度都 可以被调整,以选择一个最佳聚焦的粒子范围,产生一束聚焦的中微子束,其能量范围被选定为kaons和π介子衰变。通过喇叭使电流方向逆转,可以产生反中微子束。加尔加梅勒在中微子和反中微子束中交替运行。范德梅尔的发明使中微子流量增加了20倍。中微子束的能量在1到10千兆瓦之间。
\begin{figure}[ht]
\centering
\includegraphics[width=6cm]{./figures/cd7ed09492bd4e9a.png}
\caption{从中微子光束线到加尔加梅勒中都可应用范德梅尔的磁性角。} \label{fig_JRJML_5}
\end{figure}
聚焦后,$\pi$介子和kaons被引导通过一个70米长的隧道,使许它们衰变。没有衰变的π介子和kaons击中隧道末端的屏蔽后并被吸收。衰变时,$\pi$介子和kaons通常衰过程是 $\pi \to \mu + v$和 ,$K\to \mu + v$这意味着中微子的流量与μ介子的流量成正比。由于$mu$介子没有被强子吸收,带电μ介子的通量在长屏蔽中被电磁减速过程所阻止。中微子通量是通过相应的μ介子通量,用六个平面的硅金探测器在屏蔽层的不同深度测量的。[8]

在1971-1976年期间,在强度方面获得了很大的改进,首先是安装了一个新的质子同步加速器助推器—其次是仔细研究了光束光学。

\subsection{结果和发现}
\begin{figure}[ht]
\centering
\includegraphics[width=6cm]{./figures/d456978af0275a06.png}
\caption{这一事件显示了在加尔加梅勒气泡室中产生的真实轨迹,它首次证实了轻子中性电流相互作用。中微子与一个电子相互作用,电子的轨迹是水平的,在没有产生μ介子的情况下以中微子的形式出现。} \label{fig_JRJML_6}
\end{figure}
加尔加梅勒的第一个主要任务是寻找μ介子中微子和反中微子在核子上硬散射的证据 。1972年3月,强子中性电流的第一个迹象变得明显。于是优先权就改变了[9] 然后决定采取双管齐下的方式来寻找当前中立的候选人。一行个方法是以搜索 腊的货币单位i件——涉及与 液体中子 相互作用的事件例如$v_\mu+e^-\to v_\mu+e^-$或者$v_\mu+e^-\to v_\mu+e^-$。另一个方法是搜索强子集成电路事件——涉及从强子散射的中微子,例如$v + p \to v + p$, $v + n \to v + p + \pi^-$或者$p \to v + n + \pi^+$,加上带有许多强子的事件。轻子事件的横截面很小,但背景也相对较小。强子事件有更大的背景,最广泛的原因是中微子在反应室与周围的物质中相互作用时产生的中子。没有电荷的中子不会在气泡室中被探测到,对它们相互作用的探测将模拟中性电流事件。为了减少中子背景,强子事件的能量必须大于1 GeV。

1972年12月在亚琛发现了第一个轻子事件的例子。到1973年3月,已发现166个强子事件,102个中微子事件和64个反中微子事件。[9] 然而,中子背景的问题悬而未决,无法解释强子事件。这个问题通过研究带电电流事件得到了解决,这些带电电流事件也具有满足强子事件选择的相关中子相互作用。[10] 通过这种方式,可以监控中子背景通量。1973年7月19日,加尔加梅勒合作组织在欧洲核子研究中心的一次研讨会上提出了中性电流的发现。

加尔加梅勒的合作发现了两种轻子中性电流——涉及中微子与电子相互作用的事件——以及强子中性电流——中微子从核子中散射的事件。这一发现非常重要,因为它支持了弱电理论,而弱电理论现在是标准模型的支柱。弱电理论的最终实验证明是在1983年,当时UA1和UA2协作发现了W± Z0 玻色子。

最初加尔加梅勒的首要任务是测量中微子和反中微子的横截面和结构功能。这样做的原因是为了测试核子的夸克模型。首先,中微子和反中微子的截面被证明与能量呈性关系,这是人们对核子中点状成分散射的预期。中微子和反中微子结构函数的结合使得核子中夸克的净数目得以确定,这与3的结果非常一致。此外,将中微子结果与美国坦福直线加速器中心 SLAC)的结果进行了比较,人们发现夸克有分数电荷,并通过实验证明了这些电荷的值:+2⁄3 e,-1⁄3 e.这个结果发表于1975年,为夸克的存在提供了重要证据。[11]

\subsection{参考文献}
[1]
^"Gargamelle". CERN. Retrieved 12 August 2017...

[2]
^"The Nobel Prize in Physics 1979". Nobelprize.org. 15 October 1979. Retrieved 28 July 2017..

[3]
^Schwartz, M. (15 March 1960). "Feasibility of Using High-Energy Neutrinos to Study the Weak Interactions". Physical Review Letters. 4 (6): 306–307. Bibcode:1960PhRvL...4..306S. doi:10.1103/PhysRevLett.4.306..

[4]
^"Nobel Prize in Physics 1988: Press Release". Nobelprize.org. Retrieved 16 August 2017..

[5]
^Pestre, Dominique (1996). Gargamelle and BEBC. How Europe's Last Two Giant Bubble Chambers were Chosen. Amsterdam: North-Holland. pp. 39–97..

[6]
^Haidt, Dieter (2015). "The Discovery of Weak Neutral Currents". In Schopper, Herwig; Di Lella, Luigi. 60 Years of CERN Experiments and Discoveries. Singapore: World Scientific. pp. 165–185. Retrieved 12 August 2017..

[7]
^"Proposal for a Neutrino Experiment in Gargamelle". 16 March 1970. CERN-TCC-70-12. Retrieved 12 August 2017..

[8]
^Musset, P.; Vialle, J.P. (1978). "Neutrino Physics with Gargamelle". In Jacob, M. Gauge Theories and Neutrino Physics. Amsterdam: North-Holland Publishing. pp. 295–425..

[9]
^Cundy, Donald; Christine, Sutton. "Gargamelle: the tale of a giant discovery". CERN Courier. CERN. Retrieved 15 August 2017..

[10]
^Cundy, Donald (1 July 1974). Neutrino Physics. 17th International Conference on High-energy Physics. London: CERN. pp. 131–148..

[11]
^Deden, H.; et al. (27 January 1975). "Experimental Study of Structure Functions and Sum Rules in Charge-Changing Interactions of Neutrinos and Antineutrinos on Nucleons" (PDF). Nuclear Physics B. 85 (2): 269–288. Bibcode:1975NuPhB..85..269D. doi:10.1016/0550-3213(75)90008-5. Retrieved 18 August 2017..