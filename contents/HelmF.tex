% 亥姆霍兹自由能
% 自由能

\pentry{热力学第一定律\upref{Th1Law},热力学第二定律\upref{Td2Law}}

亥姆霍兹自由能,简称自由能,常用 $A$ 或 $F$ 表示(本文将用 $F$ 表示)。自由能是一个热力学态函数,\textbf{对于一个恒温的封闭热力学系统,系统对外界做的功总是小于(不可逆过程)或等于(可逆过程)自由能的减少量}。我们称它为\textbf{最大功原理}。

由热力学第一定律\upref{Th1Law},对于系统的任意一个过程,都有 $T\dd S\ge \dd Q=\dd U+\dd W$。在等温过程中,$T\dd S=\dd (TS)$,所以 $\dd W\le - \dd (U-TS)$。对照最大功原理,我们可以有如下定义。

我们将 $F$ 定义为 $U-TS$,$U$ 为系统的内能,$S$ 为熵。由热力学第一定律(\upref{Th1Law}\autoref{Th1Law_eq2})可得
\begin{equation}
\dd F=-S\dd T-p\dd V
\end{equation}

假如考虑表面张力\upref{sftens},则液体的自由能可以写作
\begin{equation}
\dd F=-S\dd T-p \dd V+\sigma \dd S
\end{equation}
\subsection{自由能判据}
由前面的分析可知,对于\textbf{恒温恒容系统},当外界对系统不做功时,系统总是趋向自由能减小的状态。所以我们可以得出稳定平衡状态的判据。

等温等容系统处在稳定平衡状态的必要和充分条件为:
\begin{equation}
\Delta F>0
\end{equation}
这里的 $\Delta F$ 指的是在等温等容条件下系统可能发生的任何虚变动引起的自由能改变量。

如果对 $F$ 作泰勒展开,准确到二级,则有 $\Delta F=\dd F+\frac{1}{2}\dd[2]{F}$。此时平衡条件为 $\dd F=0$,稳定性条件为 $\dd[2]{F}>0$。这一点与熵判据\upref{equcri}的原理一样。
