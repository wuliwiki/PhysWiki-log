% 暗物质在其他星系尺度存在的证据
% license Usr
% type Tutor


还有许多其他支持暗物质存在的证据,这些证据基于星系尺度的观测。追踪动力学测量可以应用于其他系统,特别是矮星系。这些系统中有大量的暗物质,质光比通常在100或更高。

此外,可以使用矮星系本身作为追踪器来确定螺旋星系的旋转曲线。这种方法证实了使用其他方法获得的结果。

在20世纪70年代,早期的数值模拟及其分析解释表明,需要一个暗物质球状晕来确保螺旋星系中盘的稳定性。在没有它的情况下,模拟的盘在几个旋转周期内就会瓦解,变成恒星的混乱分布。然而,后来更精细的研究表明,暗物质晕对盘有负面效果,使其变成条带并最终减慢条带的旋转,这与许多旋转盘的观测存在相矛盾。看来在这个方面不能简单地得出关于暗物质角色的结论。通过星系-星系透镜效应也可以证明星系周围存在大质量的暗物质晕。这是由于前景星系的引力透镜效应引起的背景星系图像的扭曲,类似于太阳如何引力偏转背景恒星的光。光线轨迹的偏差是$\delta \theta = 4GM_{lens}/b$,其中$b$是冲击参数,$M_{lens}$是透镜的质量,即前景星系。几何光学意味着,如果源和透镜在视线上重合,质量为$M_{lens}$的透镜在距离$d_{lens}$处的引力会聚焦成一个“爱因斯坦环”(或弧),角度大小为$\theta_E = \sqrt{(4GM_{lens}(1/d_{lens} - 1/d_{source}))}$,如果源和透镜不在一条直线上,则会形成多重图像。这种现象被称为强透镜效应,尽管偏差通常很小,$\delta\theta \ll 1$,并且只有当透镜非常巨大,源足够接近它时才能观察到。更常见的是弱透镜效应,其中透镜不够强大,无法形成多重可观察图像或弧线,但确实会导致背景星系图像的扭曲。通过测量平均属性,可以推断出典型透镜星系中的物质量,发现它大于可见质量。这些观测还可以确定暗物质晕的典型密度分布,发现它有些扁平。