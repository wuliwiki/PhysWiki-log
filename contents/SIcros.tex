% 单电子跃迁截面(一阶微扰)
% keys 电磁波|光电离
% license Xiao
% type Tutor

\pentry{量子散射的微分截面\nref{nod_ParWav}, 跃迁概率(一阶微扰)\nref{nod_HionCr}}{nod_2663}
本文使用\enref{原子单位制}{AU}。

\subsection{束缚态之间的跃迁截面}
对于频率为 $\omega$ 的\enref{平面电磁波}{VcPlWv}, 截面可以想象成与电磁波传播方向垂直放置的一块面积为 $\sigma(\omega)$ 的面元, 使得原子从电磁波中吸收的平均功率恰好等于电磁波经过该面元的功率。

所以自然地,可以用以下思路定义\textbf{束缚态之间的跃迁截面}: 想象有许多相同的原子,在两个特定束缚态之间的跃迁需要的能量为 $\omega_{ij}$,所以平均每个原子在某个跃迁中吸收的能量为
\begin{equation}
E = \omega_{ij}P_{j\to i}~,
\end{equation}
$P_{j\to i}$ 是电子从束缚态 $\ket{j}$ 跃迁到另一个束缚态 $\ket{i}$ 的概率。

另一方面, 电磁波在单位面积上单位频率的能量为\upref{WpEng} $s(\omega)$, 那么定义每个原子的截面定义为
\begin{equation}
E = \sigma_{j\to i} s(\omega_{ij})~,
\end{equation}
事实上, 这里的 $\sigma_{j\to i}$ 并不只是截面, 而是截面乘以一个吸收谱线的宽度 $\Delta \omega$, 否则量纲对不上。 但为了方便一般还是简称为截面。

从\autoref{eq_HionCr_5}  来看, $P_{j\to i}$ 同样和 $s(\omega_{ij})$ 成正比,所以从两式中消去后截面 $\sigma_{j\to i}$ 和入射的波包无关:
\begin{equation}
\sigma_{j\to i} = \frac{4\pi^2 q^2}{c m^2 \omega_{ij}} \abs{\uvec e \vdot\mel{i}{\bvec p}{j}}^2~.
\end{equation}
这和 \cite{Bransden} 的式 4.46 的定义一样, 除了偶极子近似。

\subsection{束缚态到连续态的跃迁截面}
对于束缚态到连续态的跃迁, 单位频率下原子吸收的能量为
\begin{equation}
\dv{E}{\omega} = \sigma(\omega) s(\omega)~,
\end{equation}
如果再对立体角微分得某频率的微分截面 % \addTODO{链接}
\begin{equation}\label{eq_SIcros_1}
\pdv{E}{\omega}{\Omega} = \pdv{\sigma}{\Omega} s(\omega)~,
\end{equation}
使用速度规范, 当波包经过以后, $\bvec A = 0$。 令初始束缚态 $\ket{j}$ 的能量为 $-I_0$, 把 $\omega$ 看作 $k$ 的函数
\begin{equation}
\omega(k) = \frac{k^2}{2m} + I_0~,
\end{equation}
\autoref{eq_SIcros_1} 积分得总能量
\begin{equation}
E = \iint \pdv{\sigma}{\Omega} s(\omega) \dd{\omega}\dd{\Omega} = \iint \pdv{\sigma}{\Omega} s(\omega(k)) \frac{k}{m}\dd{k}\dd{\Omega}~.
\end{equation}
另一方面, 每个原子从 $\ket{j}$ 电离平均吸收的能量为
\begin{equation}
E = \iint \omega(k) P_{j\to {\bvec k}} \cdot k^2 \dd{k}\dd{\Omega}~.
\end{equation}
对比以上两式可以把微分截面表示为 $\bvec k$ 的函数, 也可以看成是单位矢量 $\uvec k$ 和 $\omega$ 的函数
\begin{equation}
\pdv{\sigma}{\Omega} = \frac{km \omega}{s(\omega)} P_{j\to \bvec k}~,
\end{equation}

长度规范下有(\autoref{eq_HionCr_6})
\begin{equation}\label{eq_SIcros_2}
\pdv{\sigma}{\Omega} = \frac{4\pi^2 m\omega k q^2}{c} \abs{\mel{\bvec k}{\uvec e \vdot\bvec r}{j}}^2~.
\end{equation}
速度规范下有(\autoref{eq_HionCr_5})
\begin{equation}\label{eq_SIcros_8}
\pdv{\sigma}{\Omega} = \frac{4\pi^2 k q^2}{c m \omega} \abs{\mel{\bvec k}{\uvec e \vdot \bvec p}{j}}^2~.
\end{equation}
这与 \cite{Merzbacher} 的式 19.86 完全相同。

\begin{lstlisting}[language=matlab, caption=hydrogen\_ioniz\_dif\_cross\_section\_z.m]
% hydrogen single ionization differential cross section
% use <Coulomb plane wave|z|n,l,m> as transition matrix
% eq_SIcros_2
function [ret, mel] = hydrogen_ioniz_dif_cross_section_z(Z,n,l,m,k,r_max)
% fine structure constant = 1/(speed of light) in a.u.
alpha = 0.007297352569278;
mel = zeros(size(k));
parfor i = 1:numel(k)
    mel(i) = hydrogen_trans_dipole_plane(Z, k(i), 0, 0, n, l, m, r_max);
end
ret = (4*pi^2*(k.^2/2+0.5*Z^2/n^2).*k)*alpha .* abs(mel).^2;
end
\end{lstlisting}

\addTODO{单位能量的总截面是多少?}
% \begin{equation}
% \sigma = \int \pdv{\sigma}{\Omega}\dd{\Omega}~.
% \end{equation}
