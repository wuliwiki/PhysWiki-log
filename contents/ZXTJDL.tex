% 自旋统计定理(综述)
% license CCBYSA3
% type Wiki

本文根据 CC-BY-SA 协议转载翻译自维基百科\href{https://en.wikipedia.org/wiki/Spin\%E2\%80\%93statistics_theorem}{相关文章}。

自旋-统计定理证明了粒子的内禀自旋(不源于轨道运动的角动量)与该类粒子集合的量子统计性质之间的关系是量子力学数学的必然结果。  

在以**约化普朗克常数** \( \hbar \) 为单位的描述下,**所有在三维空间中运动的粒子** 具有以下特性:  
- **整数自旋(integer spin)** 的粒子服从 **玻色-爱因斯坦统计(Bose–Einstein statistics)**;  
- **半整数自旋(half-integer spin)** 的粒子服从 **费米-狄拉克统计(Fermi–Dirac statistics)** \(^\text{[1][2]}\)。