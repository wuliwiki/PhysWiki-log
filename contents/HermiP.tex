% Hermite 多项式

\begin{issues}
\issueDraft
\end{issues}

\footnote{本文参考 Wikipedia \href{https://en.wikipedia.org/wiki/Hermite_polynomials}{相关页面}.}Hermite 多项式,中文也译作埃尔米特多项式.

\begin{figure}[ht]
\centering
\includegraphics[width=10cm]{./figures/HermiP_1.pdf}
\caption{Hermite 多项式(来自 Wikipedia)} \label{HermiP_fig1}
\end{figure}

\begin{equation}
H_n(x) \equiv (- 1)^n \E^{x^2} \dv[n]{x} \qty(\E^{-x^2})
\end{equation}
前 6 阶 Hermite 多项式分别为
\begin{equation}
\begin{array}{l}
H_0(x) = 1\\
H_1(x) = 2x\\
H_2(x) = 4x^2 - 2
\end{array}
\qquad
\begin{array}{l}
H_3(x) = 8x^3 - 12x\\
H_4(x) = 16x^4 - 48x^2 + 12\\
H_5(x) = 32x^5 - 160x^3 + 120x
\end{array}
\end{equation}

\subsection{性质}
\textbf{正交性}
\begin{equation}% Mathematica 已验证
\int_{-\infty}^\infty H_m(x) H_n(x) \E^{-x^2} \dd{x} = \sqrt{\pi}2^n n! \delta_{m,n}
\end{equation}
其中 $\E^{-x^2}$ 被称为权函数.
