% 可逆过程
% license CCBYSA3
% type Wiki

(本文根据 CC-BY-SA 协议转载自原搜狗科学百科对英文维基百科的翻译)

在热力学中,可逆过程是一种反应方向可以通过其周围环境引起系统某些性质的微小变化而“逆转”的过程。 在整个可逆过程中,系统与其周围环境处于热力学平衡。逆转了的过程既不会改变系统,也不会改变周围的环境。因为可逆过程需要无限长的时间才能完成,所以完全可逆的过程是不存在的。然而,如果发生变化的系统响应速度比所发生的变化快得多,那么它与可逆性之间的偏差可以忽略不计。在一个可逆循环中,即一个循环可逆过程中,如果一个半循环之后是另一个半循环,系统及其周围环境将返回到它们的原始状态。

热力学过程会以两种方式之一进行:可逆或不可逆。可逆性意味着反应在准平衡状态下持续进行。在理想的热力学可逆过程中,由系统做的或在系统上做的功所产生的能量将会最大化,来自热量的能量将为零。然而,热量不能完全转化为功,并且总是在一定程度上损失(到周围环境)。(只有在循环的情况下才是这样。在理想过程的情况下,热量可以完全转化为功,例如,理想气体在活塞-气缸装置中的等温膨胀。)最大化的功和最小化的热的现象可以在压力-体积坐标上可视化,如代表所做的功的平衡曲线下方的面积所示。为了使功最大化,必须精确地遵循平衡曲线。

另一方面,不可逆过程是偏离曲线的结果,因此减少了做的总功;不可逆过程可以描述为偏离平衡的热力学过程。不可逆性被定义为一个过程的可逆功和实际功之差。如用压力和体积来描述,即它发生在系统的压力(或体积)的变化过于剧烈而短暂,以至于体积(或压力)没有时间达到平衡的时候。不可逆的一个经典例子是使一定体积的气体释放到真空中。通过释放样本上的压力,进而使其占据较大的空间,系统和周围环境在膨胀过程中不平衡,而且几乎不做功。然而,随着热量流入环境,相应的能量散失,因此需要作出大量的功,才能逆转这一过程(将气体压缩回其原始体积和温度)。[1]

可逆过程的另一个定义是一种在发生后可以逆转,并且当它逆转时,会使系统及其周围环境恢复到初始状态的过程。用热力学术语来说,过程的“发生”是指从一种状态到另一种状态的转变。

\subsection{不可逆性}
在不可逆的过程中,会发生有限的变化;因此,系统在整个过程中并不处于平衡状态。在不可逆循环的同一个点,系统将处于相同的状态,但是在每个循环之后,环境会发生永久性的变化。[2] 不可逆性是一个过程中可逆功和过程实际功之差,如下式所示:$I = W_{rev} -W_{a}$
\begin{figure}[ht]
\centering
\includegraphics[width=12cm]{./figures/3ff3e4cf4e7c4ae1.png}
\caption{可逆绝热过程:左边的状态可以从右边的状态达到,反之亦然,不与环境交换热量。} \label{fig_KNGC_1}
\end{figure}

\subsection{边界和状态}
可逆过程以这样一种方式改变系统的状态,即系统及其周围环境总熵的净变化为零。可逆过程定义了热机在热力学和工程学中能达到的效率的界限:可逆过程是指没有热量从系统中作为“废物”损失的过程,因此机器处于其可能达到的最高效(见卡诺循环)。

在某些情况下,区分可逆过程和准静态过程很重要。可逆过程总是准静态的,但反过来并不总是正确的。[2] 例如,在活塞和汽缸之间存在摩擦的汽缸中气体的无限小的压缩是一个准静态的,但不可逆的过程。[2] 尽管系统从其平衡状态仅被改变了极小的量,但是热量由于摩擦已经不可逆地损失,并且不能通过简单地沿相反方向极少量地移动活塞来回收。

\subsection{工程古语}
从历史上看,特斯拉原理一词被用来描述(在其他一些事情中)尼古拉·特斯拉发明的某些可逆过程。[3] 然而,这个短语已不再是常规用法。该原理陈述为,有些系统可以以互补的方式逆转和运作。它是在特斯拉研究交流电时开发的,交流电的大小和方向是周期性变化的。在特斯拉涡轮的演示中,旋转的圆盘和固定在轴上的机械由发动机操作。如果涡轮机的运行被逆转,圆盘就会作为一个泵工作。[4]

\subsection{参考文献}
[1]
^Lower, S. (2003) Entropy Rules! What is Entropy? Entropy.

[2]
^Sears, F.W. and Salinger, G.L. (1986), Thermodynamics, Kinetic Theory, and Statistical Thermodynamics, 3rd edition (Addison-Wesley.).

[3]
^Electrical Experimenter, January 1919. p. 615. [1].

[4]
^"Tesla's New Monarch of Machines". New York Herald Tribune, Oct. 15, 1911. (Available online. Tesla Engine Builders Association. [2]).