% 海因茨·霍普夫(综述)
% license CCBYSA3
% type Wiki

本文根据 CC-BY-SA 协议转载翻译自维基百科 \href{https://en.wikipedia.org/wiki/Heinz_Hopf}{相关文章}。

\begin{figure}[ht]
\centering
\includegraphics[width=6cm]{./figures/20a0d14143ce70b3.png}
\caption{1954年的霍普} \label{fig_HYCH_1}
\end{figure}
海因茨·霍普(Heinz Hopf,1894年11月19日-1971年6月3日)是德国数学家,研究领域包括动力系统、拓扑学和几何学。
\subsection{早期生活与教育}
霍普出生于德国帝国的格雷布申(现波兰弗罗茨瓦夫的格拉比辛),父亲是威廉·霍普,母亲是伊丽莎白(原姓基尔赫纳)。他的父亲出生为犹太人,霍普出生一年后父亲皈依了新教;母亲来自一个新教家庭。\(^\text{[3][4]}\)

霍普于1901年至1904年就读于卡尔·米特尔豪斯高年级男子学校,随后进入布雷斯劳的科尼格·威廉中学。他从小便显示出数学天赋。1913年,他进入上西里西亚的弗里德里希·威廉大学,聆听了恩斯特·施泰尼茨、阿道夫·克内塞尔、马克斯·德恩、厄尔哈德·施密特和鲁道夫·斯图尔姆的讲座。第一次世界大战爆发后,霍普积极参军,曾两次受伤,并于1918年获得铁十字勋章(一级)。

战争结束后,霍普继续在海德堡(1919/20冬季和1920年夏季)\(^\text{[5]}\)和柏林(从1920/21年冬季开始)继续他的数学教育。他在路德维希·比伯巴赫的指导下学习,并于1925年获得博士学位。
\subsection{生涯}
在他的博士论文《流形的拓扑与度量之间的联系》(德文原题:Über Zusammenhänge zwischen Topologie und Metrik von Mannigfaltigkeiten)中,霍普证明了:任何单连通的、完全的、具有常截面曲率的黎曼三维流形,在全局上等距于欧几里得空间、球面或双曲空间。他还研究了超曲面上向量场零点的指标,并将其总和与曲率联系起来。大约六个月后,他给出了一个新的证明,说明流形上向量场零点指标的总和与所选向量场无关,并等于该流形的欧拉示性数。这个定理现在被称为庞加莱–霍普定理(Poincaré–Hopf 定理)。

霍普在获得博士学位后的那一年曾在哥廷根大学工作,当时大卫·希尔伯特、理查德·柯朗特、卡尔·龙格和埃米·诺特都在该校任教。在那里,他结识了帕维尔·亚历山德罗夫,并与他建立了终身友谊。

1926 年,霍普回到柏林,在那里讲授组合拓扑课程。1927/28 学年,他与亚历山德罗夫一同获得洛克菲勒基金会奖学金,在普林斯顿大学度过。那时,所罗门·勒夫谢茨、奥斯瓦尔德·维布伦以及 J. W. 亚历山大都在普林斯顿任教。也正是在这一时期,霍普发现了从
$S^3 \to S^2$ 的映射的霍普不变量(Hopf invariant),并证明霍普纤维化(Hopf fibration)具有不变量 1。

1928 年夏天,霍普回到柏林,在柯朗特的建议下与帕维尔·亚历山德罗夫合作编写一本拓扑学著作,计划分三卷出版,但最终只完成了一卷,于 1935 年出版。

1929 年,他拒绝了普林斯顿大学的职位邀请。1931 年,霍普接替赫尔曼·外尔在苏黎世联邦理工学院(ETH)的职位。1940 年他再次收到普林斯顿的邀请,但再次拒绝。然而,两年后,由于纳粹没收了他的财产,他不得不申请瑞士国籍——他父亲的基督教皈依并未说服德国当局将他们视为“雅利安人”。

1946/47 年和 1955/56 年,霍普访问美国,在普林斯顿大学逗留,并在纽约大学和斯坦福大学讲学。他曾于 1955 年至 1958 年担任国际数学联盟(IMU)主席。\(^\text{[6]}\)
\subsection{个人生活}
1928 年 10 月,霍普与安雅·冯·米克维茨(Anja von Mickwitz,1891–1967)结婚。

荣誉与奖项
他曾获得普林斯顿大学、弗赖堡大学、曼彻斯特大学、巴黎大学、布鲁塞尔自由大学和洛桑大学的名誉博士学位。1949 年,他被选为海德堡科学院的通讯院士。1957 年当选为美国国家科学院院士,1961 年成为美国艺术与科学院院士,\(^\text{[7]}\)并于 1963 年当选为美国哲学学会会员。\(^\text{[8]}\)他曾于 1932 年在苏黎世国际数学家大会(ICM)上担任特邀报告人,并于 1950 年在美国马萨诸塞州剑桥举办的 ICM 上担任大会报告人。\(^\text{[9]}\)

为纪念霍普,苏黎世联邦理工学院设立了“海因茨·霍普奖”,以表彰在纯数学领域做出杰出科学工作的学者。
\subsection{另见}
\begin{itemize}
\item 余霍普夫群
\item 余同伦群
\item EHP谱序列
\item 霍普夫群
\item 霍普夫对象
\item 量子群(
\end{itemize}
\subsection{代表性出版物}
\begin{itemize}
\item Alexandroff P., Hopf H.《拓扑学》第1卷,1935年,B出版社
\item Hopf, Heinz (1964),《海因茨·霍普文选》,为庆祝其70岁生日由苏黎世联邦理工学院出版,柏林、纽约:施普林格出版社,MR 0170777
\item Hopf, Heinz (2001),《论文集 / Collected papers / Gesammelte Abhandlungen》,柏林、纽约:施普林格出版社,ISBN 978-3-540-57138-4,MR 1851430
\end{itemize}
\subsection{参考资料}
\begin{enumerate}
\item [数学世系项目中的 Heinz Hopf](https://www.mathgenealogy.org/id.php?id=5562)
\item I.M. James 编 (1999年8月24日). 《拓扑学史》. Elsevier. 第991页. ISBN 978-0-08-053407-7.
\item “Heinz Hopf”. 圣安德鲁斯大学.
\item “Hopf, Heinz” (PDF). RobertNowlan.com. 原始PDF于2011年1月6日存档.
\item “Heinz Hopf”. 数学史海德堡资料库 (Historia Mathematica Heidelbergensis).
\item “国际数学联盟(IMU):1952–2014年执行委员会名单”. [mathunion.org](https://www.mathunion.org). 原始链接于2015年1月8日存档. 访问于2017年3月20日.
\item “Heinz Hopf”. 美国艺术与科学院. 访问于2022年11月14日.
\item “APS会员历史”. [search.amphilsoc.org](https://search.amphilsoc.org). 访问于2022年11月14日.
\item Hopf, H. (1950). “拓扑学中的n维球面与射影空间” (PDF). 载于:《1950年国际数学家大会论文集》,美国马萨诸塞州剑桥,1950年8月30日至9月6日,第1卷,第193–202页. 原始PDF于2013年12月28日存档. 访问于2017年12月5日.
\end{enumerate}
\subsection{进一步阅读}
\begin{itemize}
\item Bagni, Giorgio T. “Heinz Hopf”. 国际数学教育委员会 (ICMI) 历史网页.
\item Hilton, Peter J. (1972), “Heinz Hopf”, 《伦敦数学会会刊》, 4 (2): 202–217, doi:10.1112/blms/4.2.202.
\end{itemize}
\subsection{外部链接}
\begin{itemize}
\item O'Connor, John J.;Robertson, Edmund F.,《海因茨·霍普(Heinz Hopf)》,圣安德鲁斯大学 MacTutor 数学史档案馆
\item 《闭超曲面的曲率积分》,D.H. Delphenich 翻译
\item 《n维流形中的向量场》,D.H. Delphenich 翻译
\end{itemize}