% 磁标势
% 磁标势|电流|磁偶极子

\subsection{磁标势推导}
\begin{equation}
\oint_L \bvec H \cdot \dd{l} = \int_S \bvec J \cdot \dd{S}~.
\end{equation}
当 $S$ 内无电流亦环路 $L$ 内无电流时
\begin{equation}
\oint_ L\bvec H \cdot \dd \bvec l = 0~.
\end{equation}
在该区域,有 $\bvec J=0$,则在该区域内,磁场方程为 
\begin{equation}\label{eq_elecdy_3}
\curl \bvec H =0~,
\end{equation}
\begin{equation}\label{eq_elecdy_2}
\div \bvec B=0~,
\end{equation}
\begin{equation}\label{eq_elecdy_1}
\bvec B=\mu_0(\bvec H+\bvec M)=f(\bvec H)~.
\end{equation}
\autoref{eq_elecdy_1} 的写法为函数形式因为在如铁磁性物质中,线性关系 $\bvec B=\mu_0 \bvec H$ 不成立。
而 $\bvec H$ 与 $\bvec B$ 的关系可以由磁滞回线确定。
把\autoref{eq_elecdy_1} 代入\autoref{eq_elecdy_2} 得
\begin{equation}\label{eq_elecdy_4}
\div \bvec H= - \div \bvec M~.
\end{equation}
把分子电流看作由一对假想的磁荷组成的磁偶极子,则和电场中的 $\div \bvec P=-\rho _p$ 对应。
\begin{equation}\label{eq_elecdy_5}
\rho_m=-\mu_0 \div \bvec M~.
\end{equation}
因而 在 $\bvec J=0$ 区域内, 由\autoref{eq_elecdy_3} \autoref{eq_elecdy_4} \autoref{eq_elecdy_5} 开始微分方程可以写为
\begin{equation}
\curl \bvec H=0 ~,
\end{equation}
\begin{equation}
\div \bvec H= \dfrac{\rho_m}{\mu_0}~.
\end{equation}
对比与静电场微分方程
\begin{equation}
\div \bvec E=\dfrac{\rho_f+\rho_p}{\epsilon}~,
\end{equation}
\begin{equation}
\curl \bvec E=0~.
\end{equation}
故而可以引入磁场标势
\begin{equation}
\bvec H=-\div \varphi_m~.
\end{equation}
