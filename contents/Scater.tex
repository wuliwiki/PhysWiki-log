% 散射
% keys 散射|经典力学|有心力

\begin{issues}
\issueDraft
\end{issues}

\pentry{中心力场问题\upref{CenFrc}}

% 散射不就是要画一张图吗? 说明微分散射截面, 以及彩虹\dots\dots
% 散射, 定义微分截面 % d sigma/d Omega 真的就是面积元比对应的立体角元

\begin{figure}[ht]
\centering
\includegraphics[width=9.5cm]{./figures/71268d38e0a7bf6f.pdf}
\caption{微分截面的定义} \label{fig_Scater_1}
\end{figure}

每个入射距离 $b$ 对应一个出射角度 $\theta$, 所以 $\dd\Omega = 2\pi \sin\theta \dd{\theta}$, $\dd{\sigma} = 2\pi b \dd{b}~,$
\begin{equation}
\frac{\dd{\sigma}}{\dd{\Omega}} = \frac{b}{\sin\theta} \abs{\dv{b}{\theta}}~.
\end{equation}

若散射是轴对称的, $\bvec j$ 是粒子流密度, \autoref{fig_Scater_1} 中入射环的面积为 $\dd{\sigma}$, 出射环的面积为 $r^2\dd{\Omega}$。 所以从前者中穿过的单位时间粒子数必定等于从后者穿过的
\begin{equation}
\abs{\bvec j_{in}}\dd{\sigma} = \bvec j_{out} \vdot \uvec r\ \ r^2\dd{\Omega}~,
\end{equation}
所以用粒子流密度表示微分截面就是
\begin{equation}
\dv{\sigma}{\Omega} = \lim_{r\to\infty} \frac{(\bvec j_{out} \vdot \uvec r) r^2}{\abs{\bvec j_{in}}}~.
\end{equation}
