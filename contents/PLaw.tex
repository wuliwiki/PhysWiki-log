% 动量定理、动量守恒
% 动量守恒|动量定理|合外力|质点系
\pentry{质点系的动量\upref{SysMom}}

以下我们将根据牛顿定律推导出系统的\textbf{动量定理}, 即系统总动量的变化率等于合外力
\begin{equation}\label{eq_PLaw_1}
\dot {\bvec p} = \bvec F^{out}~.
\end{equation}
由此可以得出系统所受合外力为零时系统总动量不随时间变化, 即\textbf{动量守恒(conservation of momentum)}。

由\autoref{eq_SysMom_2}~\upref{SysMom}, 还可以把\autoref{eq_PLaw_1} 改写为类似牛顿第二定律的形式, 一些教材中被称为\textbf{质点系的牛顿第二定律}
\begin{equation}\label{eq_PLaw_2}
M \bvec a_c = \bvec F^{out}~,
\end{equation}
其中 $\bvec a_c$ 是质点系质心的加速度。 系统合外力为零时, 质心加速度为零, 即静止或者做匀速运动。

\subsection{推导}
在牛顿力学中任何系统都可以看做质点系,质点系中第 $i$ 个质点可能受到系统内力 $\bvec F_i^{in}$ 或系统外力 $\bvec F_i^{out}$。 由单个质点的动量定理\upref{PLaw1},
\begin{equation}
\dv{t} \bvec p_i = \bvec F_i^{in} + \bvec F_i^{out}~.
\end{equation}
总动量的变化率为
\begin{equation}
\dv{\bvec P}{t} = \sum_i \dv{t} \bvec p_i  = \sum_i \bvec F_i^{in}  + \sum_i \bvec F_i^{out}~.
\end{equation}
由“质点系\upref{PSys}” 中的结论, 上式右边第一项求和是系统合内力, 恒为零。 于是我们得到系统的动量定理
\begin{equation}
\dv{\bvec P}{t} = \sum_i \bvec F_i^{out}~,
\end{equation}
可见当和外力(即等式右边)为零时, 动量 $\bvec P$ 不随时间变化, 也就是\textbf{动量守恒}。
\addTODO{导弹爆炸, 人船模型}

\begin{example}{静止原子核的转变}
% 图未完成
一个原来静止的原子核,经放射性衰变,放出一个动量为 $9.22\e{-16}{\rm g\cdot cm/s}$ 的电子,同时该核在垂直方向上又放出一个动量为 $5.33\e{-16}{\rm g\cdot cm/s}$ 的中微子。问蜕变后原子核的动量的大小和方向。

解:由于这个静止的原子核在蜕变的全过程中没有受到其他外力,所以对该原子核构成的系统,总动量守恒。即有
\begin{equation}
\bvec p_{\rm B}+\bvec p_{\rm e}+\bvec p_{\rm \nu}=0~,
\end{equation}
即有
\begin{equation}
p_{\rm B}=|\bvec p_{\rm B}|=|-\bvec p_{\rm e}-\bvec p_{\rm \nu}|=\sqrt{p_{\rm e}^{2}+p_{\rm \nu}^{2}}=10.65\e{-16}{\rm g\cdot cm/s}~,
\end{equation}
\begin{equation}
\theta=\arctan\frac{5.33}{9.22}=30^\circ~.
\end{equation}
\end{example}
