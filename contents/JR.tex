% 詹姆斯·焦耳(综述)
% license CCBYSA3
% type Wiki

本文根据 CC-BY-SA 协议转载翻译自维基百科\href{https://en.wikipedia.org/wiki/James_Prescott_Joule}{相关文章}。

詹姆斯·普雷斯科特·焦耳 (James Prescott Joule) FRS FRSE (/dʒuːl/;[1][2][a] 1818年12月24日—1889年10月11日) 是一位英国物理学家、数学家和酿酒师,出生于兰开夏郡的索尔福德。焦耳研究了热的本质,并发现了热与机械功之间的关系。这一发现促成了能量守恒定律的提出,进而导致了热力学第一定律的发展。国际单位制中能量的导出单位“焦耳”(joule)即以他的名字命名。

他与开尔文勋爵(Lord Kelvin)合作,开发了一个绝对热力学温标,后被称为开尔文温标(Kelvin scale)。焦耳还观察到了磁致伸缩现象,并发现了电阻中的电流与其散热之间的关系,即焦耳第一定律(Joule's first law)。他关于能量转化的实验成果首次发表于1843年。

\subsection{早年生活} 
詹姆斯·焦耳出生于1818年,其父本杰明·焦耳(Benjamin Joule,1784–1858)是一位富有的酿酒师,母亲是爱丽丝·普雷斯科特(Alice Prescott)。他出生在索尔福德的新贝利街(New Bailey Street)。[3] 焦耳年轻时接受了著名科学家约翰·道尔顿(John Dalton)的指导,并深受化学家威廉·亨利(William Henry)以及曼彻斯特工程师彼得·尤尔特(Peter Ewart)和伊顿·霍奇金森(Eaton Hodgkinson)的影响。他对电学充满了兴趣,他和兄弟曾通过相互施加电击以及给家里的仆人施加电击进行实验。[4]  

成年后,焦耳接管了家族的酿酒业务,科学对他来说只是一个严肃的爱好。大约在1840年,他开始研究用新发明的电动机取代酿酒厂的蒸汽机的可行性。他关于这一主题的首批科学论文发表于威廉·斯特金(William Sturgeon)的《电学年鉴》(*Annals of Electricity*)。焦耳还是伦敦电学会(London Electrical Society)的成员,该学会由斯特金等人创立。[需要引用]  

部分出于商人对经济效益量化的需求,部分出于科学上的好奇心,焦耳开始研究哪种原动机效率更高。1841年,他发现了“焦耳第一定律”:“\textbf{任何伏打电流的正确作用所产生的热量,与该电流强度的平方乘以其所经历的导电阻力成正比。}”[5] 他进一步认识到,在蒸汽机中燃烧一磅煤比在电池中消耗一磅昂贵的锌更为经济。焦耳用一个通用标准来衡量不同方法的输出——将一磅重的物体提升一英尺的能力,即“英尺-磅”标准。[需要引用]  

然而,焦耳的兴趣逐渐从单纯的财务问题转向探讨从某一给定能源中可以提取多少功,这使他开始思考能量的可转化性。1843年,他发表了实验结果,显示他在1841年量化的加热效应是由于导体内的热量生成,而非热量从设备的其他部分转移而来。这一发现直接挑战了热质理论,该理论认为热量既不能被创造也不能被销毁。自1783年安托万·拉瓦锡(Antoine Lavoisier)提出以来,热质理论一直在热学领域占据主导地位。拉瓦锡的声望以及萨迪·卡诺(Sadi Carnot)自1824年以来在热机理论中的成功实践,确保了热质理论的影响力。然而,年轻的焦耳既不属于学术界,也不属于工程专业,他的研究因此面临重重困难。热质理论的支持者通常会指出珀尔帖–塞贝克效应的对称性,以此声称热量和电流在一定程度上是可以通过可逆过程相互转化的。[需要引用]  

\subsection{热的机械当量}  
通过对电动机的进一步实验和测量,焦耳估算出热的机械当量为每卡路里4.1868焦耳,即将一克水的温度升高一开尔文所需的功。[b] 他在1843年8月于科克举行的英国科学促进会化学分会会议上宣布了这一结果,却遭遇了一片沉默。[7]  

焦耳并未气馁,他开始寻求一种纯机械的方式来证明功可以转化为热。他通过将水强制流过一个带孔的圆筒,测量液体因粘性加热而产生的轻微升温。他得到了机械当量为每英热单位770英尺磅力(4,140 J/Cal)。焦耳认为,电学方法和纯机械方法所获得的数值在至少两位有效数字上相符,是功可以转化为热的有力证据。  

“\textbf{无论机械力如何消耗,总会产生一个完全等价的热量。}”

——J.P. 焦耳,1843年8月  

焦耳尝试了第三种方法,他测量了压缩气体时产生的热量与所做功之间的关系。他得到的机械当量是每英热单位798英尺磅力(4,290 J/Cal)。在许多方面,这个实验为焦耳的批评者提供了最容易攻击的目标,但焦耳通过巧妙的实验消除了预期的反对意见。1844年6月20日,焦耳在皇家学会宣读了他的论文,[8][9] 但皇家学会拒绝发表这篇论文,最终他不得不于1845年在《哲学杂志》(*Philosophical Magazine*)上发表。[10]  

在论文中,焦耳明确拒绝了卡诺(Carnot)和埃米尔·克拉佩龙(Émile Clapeyron)的热质理论,这种拒绝部分源于神学驱动:[需要引用]  

“\textbf{我认为,这一理论……与被公认为哲学原则的观点相悖,因为它得出了一种结论,即动生力(vis viva)可能由于装置的不当配置而被销毁。例如,克拉佩龙先生得出结论,‘火炉的温度比锅炉的温度高出1000°C到2000°C,在热量从火炉传到锅炉的过程中,存在巨大的动生力损失。’我相信,只有造物主才拥有摧毁的能力,因此我断言……任何理论如果在实施时要求力的消失,必然是错误的。}”

在这里,焦耳采用了“动生力”(即能量)的语言,这可能是因为霍奇金森(Hodgkinson)在1844年4月为文学与哲学学会(*Literary and Philosophical Society*)宣读了尤尔特(Ewart)的《论动能的测量》(*On the measure of moving force*)的一篇评论。[需要引用]  

\textbf{1845年6月,焦耳在英国科学促进会剑桥会议上宣读了他的论文《论热的机械当量》(On the Mechanical Equivalent of Heat)。[11] }

在这篇论文中,他描述了他最著名的实验之一:利用重物下落产生的机械功,带动放置在绝热水桶中的桨轮旋转,从而使水的温度升高。他在实验中估算出的机械当量为每英热单位819英尺磅力(4,404 J/Cal)。  

1845年9月,他向《哲学杂志》(*Philosophical Magazine*)投稿,发表了一封描述其实验的信件。[12]  

1850年,焦耳发表了一个更精确的测量结果:每英热单位772.692英尺磅力(4,150 J/Cal),这一数值更接近20世纪的估算值。[13]
\begin{figure}[ht]
\centering
\includegraphics[width=6cm]{./figures/8c1e402dc0d2c7d6.png}
\caption{焦耳的热量实验装置,1845} \label{fig_JR_1}
\end{figure}
\subsection{接受与优先权}
\begin{figure}[ht]
\centering
\includegraphics[width=8cm]{./figures/5775cb63f8605901.png}
\caption{焦耳用于测量热的机械当量的装置} \label{fig_JR_2}
\end{figure}
焦耳工作的初期抗拒很大程度上源于其对极其精确测量的依赖。他声称能够测量到1/200华氏度(约3毫开尔文)的温差。这种精度在当时的实验物理学中确实罕见,但他的质疑者可能忽视了他在酿酒技术方面的经验以及对相关实践技术的掌握。[14] 同时,他得到了科学仪器制造商约翰·本杰明·丹瑟(John Benjamin Dancer)的有力支持。焦耳的实验与鲁道夫·克劳修斯(Rudolf Clausius)的理论研究相辅相成,而克劳修斯被一些人认为是能量概念的共同发明者。[需要引用]  

焦耳提出了一种热的动理论(他认为热是一种旋转动能,而不是平移动能),这需要一个概念上的飞跃:如果热是分子的运动形式,为什么分子的运动不会逐渐消失?焦耳的观点要求人们相信分子的碰撞是完全弹性的。值得注意的是,原子和分子的存在在接下来的50年内并未被广泛接受。[需要引用]  

尽管如今可能难以理解热质理论的吸引力,但在当时它似乎有一些明显的优势。卡诺(Carnot)关于热机的成功理论也是基于热质假设的,直到后来开尔文勋爵(Lord Kelvin)证明,即使不假设热质流体,卡诺的数学理论同样有效。[需要引用]  

然而,在德国,赫尔曼·亥姆霍兹(Hermann Helmholtz)注意到了焦耳的研究以及朱利叶斯·罗伯特·冯·迈尔(Julius Robert von Mayer)1842年发表的类似研究。尽管两人的研究自发表以来都被忽视,但亥姆霍兹在1847年的著作中明确提出了能量守恒定律,并对两人都给予了认可。[需要引用]

1847年,焦耳在牛津英国科学促进会的另一场演讲吸引了乔治·加布里埃尔·斯托克斯(George Gabriel Stokes)、迈克尔·法拉第(Michael Faraday)以及年少才俊且特立独行的威廉·汤姆森(William Thomson,后来的开尔文勋爵)的注意。汤姆森刚刚被任命为格拉斯哥大学自然哲学教授。斯托克斯倾向于支持焦耳,而法拉第对其印象深刻,但仍存有疑虑。汤姆森对焦耳的研究表现出兴趣,但持怀疑态度。[需要引用]  

同年稍晚,汤姆森和焦耳意外地在霞慕尼(Chamonix)相遇。焦耳于8月18日与阿米莉亚·格莱姆斯(Amelia Grimes)结婚,随后前往蜜月旅行。然而,尽管新婚热情高涨,焦耳和汤姆森还是计划在几天后进行实验,测量萨朗什瀑布(Cascade de Sallanches)顶部与底部的温差,尽管这一实验后来被证明不可行。[需要引用]  

尽管汤姆森认为焦耳的实验结果需要理论解释,但他仍然热情地为卡诺–克拉佩龙(Carnot–Clapeyron)学派辩护。在他1848年的绝对温度论文中,汤姆森写道:“将热(或热质)转化为机械效应可能是不可能的,当然也是未被发现的。”[15][16] 然而,论文的脚注表明他对热质理论的首次怀疑,提到了焦耳“非常卓越的发现”。令人惊讶的是,汤姆森并没有将论文副本寄给焦耳,但当焦耳最终读到这篇论文时,他在10月6日写信给汤姆森,声称自己的研究已经证明了热可以转化为功,并计划进行进一步的实验。汤姆森于27日回信,透露他也在计划自己的实验,并希望能够调和他们的观点。尽管汤姆森没有进行新的实验,但在接下来的两年里,他对卡诺的理论日益不满,并逐渐相信焦耳的理论。在他1851年的论文中,汤姆森选择妥协,并宣布:“热的动力原理的整个理论基于两个命题,分别归功于焦耳以及卡诺和克劳修斯。”[需要引用]  

当焦耳读到这篇论文后,他立即写信给汤姆森,表达了自己的意见和疑问。由此,两人开始了一段富有成效但主要通过书信往来的合作。焦耳负责进行实验,而汤姆森负责分析结果并提出进一步实验的建议。这段合作从1852年持续到1856年,其间的发现包括\textbf{焦耳-汤姆森效应(Joule–Thomson effect)}。他们发表的研究成果在推动焦耳工作和动理论获得广泛认可方面发挥了重要作用。[需要引用]
\subsection{动理论}
\begin{figure}[ht]
\centering
\includegraphics[width=6cm]{./figures/1b95fcbf9f355491.png}
\caption{詹姆斯·普雷斯科特·焦耳} \label{fig_JR_3}
\end{figure}
运动学是研究运动的科学。焦耳是道尔顿(John Dalton)的学生,因此毫不意外,他坚定地相信原子理论,尽管在他那个时代,许多科学家对此仍持怀疑态度。他也是少数接受约翰·赫拉帕斯(John Herapath)关于气体动理论的被忽视研究的人之一。此外,他深受彼得·尤尔特(Peter Ewart)1813年论文《论动能的测量》(*On the measure of moving force*)的深远影响。[需要引用]  

焦耳认识到他的发现与热的动理论之间的关系。他的实验室笔记表明,他认为热是一种旋转运动形式,而非平移运动。[需要引用]  

焦耳无法抗拒将自己的观点追溯到弗朗西斯·培根(Francis Bacon)、艾萨克·牛顿爵士(Sir Isaac Newton)、约翰·洛克(John Locke)、本杰明·汤普森(Benjamin Thompson,亦称朗福德伯爵)和汉弗里·戴维爵士(Sir Humphry Davy)。尽管这种联系在一定程度上是合理的,但焦耳进一步通过朗福德的出版物估算出了热的机械当量的数值——1,034英尺磅。一些现代作家批评这一方法,认为朗福德的实验并不代表系统的定量测量。焦耳在一份个人笔记中争辩称,迈尔(Mayer)的测量并不比朗福德的更精确,也许他希望证明迈尔并未先于自己开展相关工作。[需要引用]  

焦耳还被认为在1869年写给曼彻斯特文学与哲学学会的一封信中解释了日落时的绿色闪光现象。实际上,他仅仅提到(并附有一幅草图)最后一瞥的颜色呈蓝绿色,并未尝试解释这一现象的成因。[17]

\subsection{已发表作品}

\begin{itemize}
\item 《关于金属导电体和电解过程中电池中产生的热量》(*"On the Heat evolved by Metallic Conductors of Electricity, and in the Cells of a Battery during Electrolysis"*)。《哲学杂志》(*Philosophical Magazine*),第19卷(124期):260页,1841年。doi:10.1080/14786444108650416。  
\item 《关于磁电现象的热效应及热的机械价值》(*"On the Calorific Effects of Magneto-Electricity, and on the Mechanical Value of Heat"*)。《哲学杂志》,第3辑,第23卷(154期):435–443页,1843年。doi:10.1080/14786444308644766。  
\item 《关于空气膨胀与压缩引起的温度变化》(*"On the Changes of Temperature Produced by the Rarefaction and Condensation of Air"*)。《伦敦皇家学会会刊》(*Proceedings of the Royal Society of London*),第5卷:517–518页,1844年。doi:10.1098/rspl.1843.0031。  
\item 《关于空气膨胀与压缩引起的温度变化》(*"On the Changes of Temperature Produced by the Rarefaction and Condensation of Air"*)。《哲学杂志》,第3辑,第26卷(174期):369–383页,1845年。doi:10.1080/14786444508645153。  
\item 《关于热的机械当量》(*"On the Mechanical Equivalent of Heat"*)。《英国科学促进会通讯和摘要》(*Notices and Abstracts of Communications to the British Association for the Advancement of Science*),第15卷,1845年6月,于剑桥英国科学促进会宣读。  
\item 《关于热与普通机械动力形式之间等效关系的存在》(*"On the Existence of an Equivalent Relation between Heat and the ordinary Forms of Mechanical Power"*)。《哲学杂志》,第3辑,第27卷(179期):205–207页,1845年。doi:10.1080/14786444508645256。  
\item 《关于热的机械当量》(*"On the Mechanical Equivalent of Heat"*)。《伦敦皇家学会哲学会刊》(*Philosophical Transactions of the Royal Society of London*),第140卷:61–82页,1850年。doi:10.1098/rstl.1850.0004。  
\item 《詹姆斯·普雷斯科特·焦耳的科学论文》(*The Scientific Papers of James Prescott Joule*),伦敦:物理学会,1884年。OL 239730M。  
\item 《联合科学论文》(*Joint Scientific Papers*),伦敦:Taylor and Francis出版社,1887年。  
\end{itemize}
\begin{figure}[ht]
\centering
\includegraphics[width=6cm]{./figures/987938820c21224e.png}
\caption{《科学论文集》第一卷和第二卷} \label{fig_JR_4}
\end{figure}
\begin{figure}[ht]
\centering
\includegraphics[width=6cm]{./figures/326a72bf88e45c98.png}
\caption{《科学论文集》第一卷的标题页} \label{fig_JR_5}
\end{figure}
\begin{figure}[ht]
\centering
\includegraphics[width=6cm]{./figures/23bd04a29da1bd49.png}
\caption{《科学论文集》第一卷的序言} \label{fig_JR_6}
\end{figure}
\begin{figure}[ht]
\centering
\includegraphics[width=6cm]{./figures/5489cdf3ec0b5fad.png}
\caption{《科学论文集》第一卷中的插图} \label{fig_JR_7}
\end{figure}
\subsection{荣誉}
\begin{figure}[ht]
\centering
\includegraphics[width=6cm]{./figures/c568f0d069cfa5c4.png}
\caption{曼彻斯特市政厅内的焦耳雕像} \label{fig_JR_8}
\end{figure}
焦耳在索尔(Sale)的家中去世[18],葬于当地的布鲁克兰公墓(Brooklands Cemetery)。他的墓碑上刻着数字“772.55”,这是他1878年测得的热的机械当量。在该测量中,他发现,在海平面上,将一磅水的温度从60°F升高到61°F,需要消耗772.55英尺磅的功。墓碑上还刻有《约翰福音》的一句经文:“趁着白日,我们必须做那差我来者的工;黑夜将到,就没有人能做工了。”[19] 此外,他去世所在城镇的韦瑟斯彭酒吧(Wetherspoon's Pub)以他的名字命名为“J. P. Joule”。  

\textbf{焦耳的诸多荣誉和奖项包括:}  

\begin{itemize}
\item 皇家学会会士(1850年)  
\item 皇家奖章(1852年),表彰其发表于1850年《哲学会刊》(*Philosophical Transactions*)的关于热的机械当量的论文  
\item 科普利奖章(1870年),表彰其关于热力学理论的实验研究  
\item 曼彻斯特文学与哲学学会会长(1860年)  
\item 英国科学促进会会长(1872年、1887年)  
\item 苏格兰工程师与造船师学会名誉会员(1857年)[20]  
\end{itemize}

\textbf{荣誉学位} 
\begin{itemize}
\item 法学博士(LL.D.),都柏林三一学院(1857年)  
\item 民法博士(DCL),牛津大学(1860年)  
\item 法学博士(LL.D.),爱丁堡大学(1871年)  
\end{itemize}

\begin{itemize}
\item 1878年,焦耳因其科学贡献获得英国民事名单养老金,每年200英镑。  
\item \textbf{皇家艺术学会的阿尔伯特奖章(1880年):} 
“表彰其通过极为艰辛的研究,确立了热、电和机械功之间的真实关系,为工程师在将科学应用于工业领域时提供了可靠的指导。” 
\end{itemize} 

在威斯敏斯特教堂的北唱诗班席有焦耳的纪念碑[21],但他并未埋葬在那里,尽管某些传记中提到过这样的说法。在曼彻斯特市政厅内,由阿尔弗雷德·吉尔伯特(Alfred Gilbert)制作的焦耳雕像矗立于约翰·道尔顿(John Dalton)雕像的对面。
\begin{figure}[ht]
\centering
\includegraphics[width=6cm]{./figures/d298537aa94a9f01.png}
\caption{索尔布鲁克兰公墓中的焦耳墓碑} \label{fig_JR_9}
\end{figure}
\subsection{家庭}
焦耳于1847年与阿米莉亚·格莱姆斯(Amelia Grimes)结婚。她于1854年去世,距他们结婚仅七年。他们育有三个孩子:长子本杰明·阿瑟·焦耳(Benjamin Arthur Joule,1850–1922)、女儿爱丽丝·阿米莉亚(Alice Amelia,1852–1899),以及次子乔(Joe,1854年出生,但三周后夭折)。
\subsection{另见}  
\begin{itemize}
\item 潜热(Latent heat)  
\item 显热(Sensible heat)  
\item 内能(Internal energy)
\end{itemize}  
\subsection{参考文献}  
\subsubsection{脚注} 
1. 《牛津英语词典》(OED):“尽管一些姓氏为 Joule 的人将其发音为 /dʒaʊl/,另一些则发音为 /dʒəʊl/(OED 格式为 /dʒoʊl/),但几乎可以肯定,J.P. 焦耳(以及至少他的部分亲属)使用的是 /dʒuːl/。”\\  
2. 焦耳的单位定义为 1 英尺磅力/英热单位(1 ft lbf/Btu),相当于 5.3803×10⁻³ J/cal。因此,焦耳的估计值为 4.15 J/cal,与20世纪初接受的值4.1860 J/cal相比非常接近。[6]  
\subsubsection{引用文献}
1. Murray 1901,第606页。\\
2. Allen 1943,第354页。\\ 
3. 《传记索引》2006年版。\\  
4. “本月物理历史:1840年12月:焦耳关于将机械功转化为热量的摘要”。\\  
5. Joule 1841,第260页。\\
6. Zemansky 1968,第86页。\\
7. Joule 1843,第263、347和435页。\\
8. Joule 1844。\\
9. Joule 1884,第171页。\\
10. Joule 1845,第369–383页。\\
11. Joule 1845b,第31页。\\ 
12. Joule 1845c,第205–207页。\\ 
13. Joule 1850,第61–82页。\\
14. Sibum 1995。\\
15. Thomson 1848。\\
16. Thomson 1882,第100–106页。\\
17. Joule 1884,第606页。\\
18. GRO 死亡登记:1889年12月8a 121 Altrincham – James Prescott Joule。\\
19. 《约翰福音》9:4。\\
20. Cameron, Stuart D.(无日期)。“名誉会员和研究员”。《苏格兰工程师与造船师学会》。检索于2019年9月17日。\\
21. Hall 1966,第62页。\\
\subsubsection{来源}
\begin{itemize}
\item Allen, H. S.(1943年)。“James Prescott Joule and the Unit of Energy”。《Nature》。**152**(3856):354。 [Bibcode:1943Natur.152..354A](https://doi.org/10.1038/152354a0)。ISSN 0028-0836。S2CID 4182911。
\item 《爱丁堡皇家学会1783–2002年院士传记索引》。爱丁堡皇家学会。2006年7月。ISBN 978-0-902-19884-7。存档于[原始文档](https://doi.org/10.1016/0039-3681(94)00036-9) (PDF格式)。2013年1月24日检索。
\item Cardwell, Donald S. L.(1991年)。《James Joule: A Biography》。曼彻斯特大学出版社。ISBN 978-0-7190-3479-4。
\item Hall, Alfred Rupert(1966年)。《The Abbey Scientists》。R. & R. Nicholson。
\item Murray, James Augustus Henry(1901年)。《A new English dictionary on historical principles; founded mainly on the materials collected by the Philological society》。牛津:Clarendon。
\item Sibum, H. O.(1995年)。“Reworking the mechanical value of heat: instruments of precision and gestures of accuracy in early Victorian England”。《Studies in History and Philosophy of Science》。**26**(1):73–106。[Bibcode:1995SHPSA..26...73S](https://doi.org/10.1016/0039-3681(94)00036-9)。
\item Thomson, William(1848年)。“On an Absolute Thermometric Scale founded on Carnot's Theory of the Motive Power of Heat, and calculated from Regnault's Observations”。《Philosophical Journal》。
\item Thomson, William(1882年)。《Mathematical and Physical Papers》。剑桥大学出版社。
\item Zemansky, Mark W(1968年)。《Heat and thermodynamics: an intermediate textbook》(第5版)。纽约:McGraw-Hill。OCLC 902055813。
\end{itemize}
\subsubsection{进一步阅读}
\begin{itemize}
\item Bottomley, J. T. (1882)。 《James Prescott Joule》。 *Nature*。**26** (678): 617–620。 [Bibcode:1882Natur..26..617B](https://doi.org/10.1038/026617a0)。 [doi:10.1038/026617a0](https://doi.org/10.1038/026617a0)。
\item Forrester, J. (1975)。 《Chemistry and the Conservation of Energy: The Work of James Prescott Joule》。*Studies in the History and Philosophy of Science*。**6**(4): 273–313。 [Bibcode:1975SHPSA...6..273F](https://ui.adsabs.harvard.edu/abs/1975SHPSA...6..273F/abstract)。 [doi:10.1016/0039-3681(75)90025-4](https://doi.org/10.1016/0039-3681(75)90025-4)。
\item Fox, R., 《James Prescott Joule, 1818–1889》,收录于 North, J. (1969)。 *Mid-nineteenth-century scientists*。Elsevier。第72–103页。ISBN 0-7190-3479-5。
\item Glazebrook, Richard Tetley (1892)。《Joule, James Prescott》,收录于 Lee, Sidney (编辑)。 *Dictionary of National Biography*。第30卷。伦敦:Smith, Elder & Co。
\item Reynolds, Osbourne (1892)。《Memoir of James Prescott Joule》。第6卷。曼彻斯特,英格兰:曼彻斯特文学与哲学学会。2014年3月5日检索。
\item Smith, C. (1998)。*The Science of Energy: A Cultural History of Energy Physics in Victorian Britain*。伦敦:Heinemann。ISBN 0-485-11431-3。
\item Smith, Crosbie (2011年1月6日)。 《Joule, James Prescott》。*Oxford Dictionary of National Biography*(在线版)。牛津大学出版社。[doi:10.1093/ref:odnb/15139](https://doi.org/10.1093/ref:odnb/15139)。(需要订阅或英国公共图书馆会员)。
\item Smith, C.; Wise, M.N. (1989)。 *Energy and Empire: A Biographical Study of Lord Kelvin*。剑桥大学出版社。ISBN 0-521-26173-2。
\item Steffens, H.J. (1979)。 *James Prescott Joule and the Concept of Energy*。Watson。ISBN 0-88202-170-2。
\item Walker, James (1950)。 *Physics 4th Edition*。Pearson。ISBN 978-0-321-54163-5。
\item 《Obituary : Dr. Joule》。 *Electrical Engineer*(10月18日)。伦敦:Biggs & Co: 311–312。1889。 "James Joule obituary."
\end{itemize}
\subsection{外部链接}
\begin{itemize}
\item \textbf{伦敦国家肖像画廊的詹姆斯·普雷斯科特·焦耳肖像}(在维基数据上编辑)  
\item \textbf{《詹姆斯·普雷斯科特·焦耳的科学论文》(1884)}——由焦耳注释  
\item \textbf{《詹姆斯·普雷斯科特·焦耳的联合科学论文》(1887)}——由焦耳注释  
\item \textbf{经典论文(1845年和1847年)}——在ChemTeam网站上,《关于机械热当量》和《关于热与普通机械功形式之间等效关系的存在》 
\item \textbf{伦敦科学博物馆中的焦耳水摩擦实验装置 } 
\item \textbf{《关于热和弹性流体结构的一些评论》(1851年)}——焦耳对气体分子速度的估算  
\item \textbf{曼彻斯特大学图书馆中的焦耳手稿}  
\item \textbf{曼彻斯特大学关于焦耳的材料}——包括焦耳住宅和墓地的照片  
\item \textbf{索尔福德大学的焦耳物理实验室}
\end{itemize}