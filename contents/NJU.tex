% 南京理工大学 2004 年 研究生入学考试试题 普通物理(B)
% license Usr
% type Note

\textbf{声明}:“该内容来源于网络公开资料,不保证真实性,如有侵权请联系管理员”

\subsection{填空题(每空2分,总共32分)}

1. 已知质点作半径为 \( R \) 的匀加速率圆周运动,其角位置 \( \theta = \theta_0 + \omega_0 t + bt^2/2 \),其中 \( \theta_0 \), \( \omega_0 \), \( b \) 均为常数,则质点在 \( t \) 时刻的速率为 $\underline{\hspace{2cm}}$,\( t \) 时刻的法向加速度大小为 $\underline{\hspace{2cm}}$,\( t \) 时刻的切向加速度大小为$\underline{\hspace{2cm}}$。

2. 已知一质点作简谐振动,其振动方程为 \( y = 0.1 \cos (100\pi t + \pi/3) \) (米),则 \( t = 1 \) 秒时质点的振动速度大小为$\underline{\hspace{2cm}}$ 米/秒,振动加速度大小为 $\underline{\hspace{2cm}}$ 米/秒\(^2\)。

3. 已知一沿 $ +x $ 方向传播的简谐波的波动方程为$y = 0.05 \cos (100\pi t - 20x + \pi/3)$ (米)。则该简谐波的传播速度为$\underline{\hspace{2cm}}$,波长为$\underline{\hspace{2cm}}$ 在波线上同一时刻相位差为 $\pi/3$ 的两点间的距离为$\underline{\hspace{2cm}}$

4.一半径为$R$的金属球,带电$Q$,则球中心0点的电场强度为$\underline{\hspace{2cm}}$,其电势为球内任一点的电势为$\underline{\hspace{2cm}}$。

5.一半径为$R$的圆环状载流回路,由$N$匝线圈组成,设一匝线中通过电流$I$,则圆环中圆环铀线上距圆环中心 $x$心处的磁感应强度大小为$\underline{\hspace{2cm}}$,方向为$\underline{\hspace{2cm}}$,处的磁感应强度大小为$\underline{\hspace{2cm}}$。

6.写出磁场中高斯定理的数学表达式$\underline{\hspace{2cm}}$,它说明磁场足$\underline{\hspace{2cm}}$。
\subsection{填空题(每空2分,总共28分)}
1.如图,曲线$I$表示 27℃的氧气分子的Maxwell速率分布,则图示中 $v_1=\underline{\hspace{2cm}}$,Ⅱ也表示氧气分子某一温度下的Maxwell速率分布,且$v_2=600m/s$,则曲线Ⅱ对应的氧气的理想气体温标 $T_2=\underline{\hspace{2cm}}$
\begin{figure}[ht]
\centering
\includegraphics[width=6cm]{./figures/db4de85e346e1990.png}
\caption{} \label{fig_NJU_1}
\end{figure}

2.一理想气体经历一次卡诺循环对外做功$1000J$,卡诺循环高温热源温度$T_1=500K$,低混热源温度$T_2=300K$,则在一次循环中,内能增量为,在高温热源处吸热$Q=\underline{\hspace{2cm}}$在低温热源处放热 $Q_2=\underline{\hspace{2cm}}$。

3.一长度为$L$平饭洪璃片,一端相连,另一端有一个直径为$d$的细丝夹住,利用波长为$\lambda$的单色平行光垂直照射,则平板波璃片上的干涉条纹间距为$\underline{\hspace{2cm}}$,若把细丝另一端移动,则条纹向$\underline{\hspace{2cm}}$移动,条纹间距变$\underline{\hspace{2cm}}$。

4.爱因斯坦狭义相对论的两条基本假说是$\underline{\hspace{2cm}}$,绝对黑体的单色吸收比是$\underline{\hspace{2cm}}$,测不准关系表达式是$\underline{\hspace{2cm}}$。

5.一屯子以 $0.6c$ 的速度运动,则该电子的运动质量为$\underline{\hspace{2cm}}$,该电子的动能等于$\underline{\hspace{2cm}}$,若该电子从$0.6c$的速度被加速到$0.8c$,外力需做功$\underline{\hspace{2cm}}$。
\subsection{三、10分}
一劲度系数为$k$的轻弹簧水平放置,一端固定,另一端连接一质量为$m$的物体,$m$ 与地面间的滑动摩拣系数为$\mu$;在弹簧为原长时,对静止物体施一沿$x$正方向的恒力F($F>f,f$,为摩擦力的大小),试求弹簧的最大伸长量。
\begin{figure}[ht]
\centering
\includegraphics[width=6cm]{./figures/18447cebcc698c98.png}
\caption{} \label{fig_NJU_2}
\end{figure}
\subsection{四、10分}
一质量为 $m$、半径为$R$的定滑轮(可看作均匀薄圆盘),可绕垂直于纸面的水平光滑轴0无摩擦地转动,轮緣绕一细轻绳,绳下端挂一质量为$m_2$的物体,物体从静止开始下降,设绳与滑轮之间不打滑,求任一时刻$t$物体下降的速度。
\begin{figure}[ht]
\centering
\includegraphics[width=6cm]{./figures/89921bea7a855144.png}
\caption{} \label{fig_NJU_3}
\end{figure}
\subsection{五、12分}
画出卡诺正循环示意图,做简要说明,并证明理想气体卡诺正循环热机效串$\eta=1-T_2/T_1$,$T_1$为高温热源温度,$T_2$为低温热源温度。
\subsection{六、12分}
一半径为$R$、均匀带电$q$的圆环的中心为0点,求(1)圆环轴线上距0点$x$距离的$P$点与0点之间的电势差,(2)$P$点的电场强度。
\subsection{七、10分}
一内外半径分别为 $R_1,R_2$的无限长圆筒,载有电流为$I$,求圆筒内外的磁感应强度分布。
\subsection{八、12分}
用波长为 $632.8nm$ 的单色光垂直照射一每厘米 1250 条刻痕的光栅,刻痕宽度与狭缝宽度相等,求(1)最多能看到第几级?(2)总共能看到多少条谱线。
\subsection{九、12分}
钠黄光$\lambda=589.3nm$ 照射某光电池时,为截止所有光电子到达阳极,需加 $0.3V$ 的反向电压。如果用$\lambda=400nm$的光照射这个光电池,则需多大反向电压才能截止所有光电子到达阳极。
\subsection{十、12分}
求经 20万伏的电压加速的电子的动能、运动速度大小、德布罗意物质波长。

附物理常数: \\\\
电子质量  $m_e = 9.11 \times 10^{-31} \\, \text{千克}$, \\\\
电子电量  $e = 1.6 \times 10^{-19} \\, \text{库仑}$, \\\\
普朗克常数  $h = 6.63 \times 10^{-34} \\, \text{焦耳·秒}$, \\\\
真空中光速  $c = 3 \times 10^8 \\, \text{米/秒}$.
