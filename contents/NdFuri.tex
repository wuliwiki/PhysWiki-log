% 多元函数的傅里叶级数
% keys 傅里叶级数|多元函数
% license Xiao
% type Tutor

\pentry{傅里叶级数\upref{FSExp}, 重积分\upref{IntN}}{nod_9bd1}

我们知道有限长区间中一个性质足够良好(满足迪利克雷条件)函数可以展开为三角函数的线性组合, 同理, 一个性质足够良好的 $N$ 元函数 $f(x_1, \dots, x_N)$ 也可以展开为 $N$ 个三角函数乘积的线性组合
\begin{equation}
f(x_1, \dots, x_N) = \sum_{n_1,\dots, n_N} C_{n_1,\dots, n_N} \exp(\I \frac{n_1\pi}{l_1} x_1) \dots \exp(\I \frac{n_N \pi}{l_N} x_N)~.
\end{equation}
其中 $x_i$ 的区间长度为 $l_i$, 每个指标求和时取负无穷到正无穷的所有整数。 系数 $C_{n_1,\dots, n_N}$ 可以由 $N$ 重积分得到
\begin{equation}
\begin{aligned}
C_{i_1,\dots, i_N} &= \frac{1}{2^Nl_1\cdots l_N}\int_{-l_N}^{l_N}\cdots\int_{-l_1}^{l_1} \exp(-\I \frac{n_1 \pi}{l_1} x_1) \dots \exp(-\I \frac{n_N \pi}{l_N} x_N)\\
 & \times f(x_1, \dots, x_N) \dd{x_1}\dots\dd{x_N}~.
\end{aligned}
\end{equation}

为了方便讨论, 我们取 $N = 2$, 更高元的情况类比可得。 区间 $x\in [0, a], y\in [0, b]$ 内的二元函数 $f(x, y)$ 的傅里叶展开为
\begin{equation}
f(x, y) = \sum_{m = -\infty}^\infty \sum_{n = -\infty}^\infty C_{m, n} \exp(\I \frac{m \pi}{a} x) \exp(\I \frac{n \pi}{b} y)~.
\end{equation}
系数由二重积分计算
\begin{equation}
C_{m, n} = \frac{1}{4ab}\int_{-b}^b\int_{-a}^a \exp(-\I \frac{m \pi}{a} x) \exp(-\I \frac{n \pi}{b} y) f(x, y) \dd{x} \dd{y}~.
\end{equation}

可以证明一组正交归一的函数基底为
\begin{equation}
b_{m,n}(x, y) = \frac{1}{\sqrt{2a}}\exp(-\I \frac{m \pi}{a} x) \frac{1}{\sqrt{2b}}\exp(-\I \frac{n \pi}{b} y)~.
\end{equation}
定义讨论的二元函数空间的内积\upref{InerPd} 为
\begin{equation}
\braket{f_{m',n'}}{g_{m,n}} = \int_0^a\int_0^b f^*_{m',n'}(x, y) g_{m,n}(x, y) \dd{x}\dd{y}~.
\end{equation}
那么由\autoref{eq_FSExp_4}~\upref{FSExp} 易证基底满足正交归一化条件
\begin{equation}
\braket{b_{m',n'}}{b_{m,n}} = \delta_{m,m'} \delta_{n,n'}~.
\end{equation}

\addTODO{举例}
