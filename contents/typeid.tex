% C++ 的 typeid (笔记)
% license Xiao
% type Note

\begin{issues}
\issueDraft
\end{issues}

\subsection{基础}
\begin{itemize}
\item \href{https://en.cppreference.com/w/cpp/language/typeid}{参考}
\item 使用 \verb|typeid| 关键字必须使用 \verb|<typeinfo>| 头文件
\item 用法: \verb|const std::type_info &tid = typeid(变量名或类型)|
\item 用 \verb|tid.name()| 会返回一个表示该类型的字符串, 取决于编译器, 且不一定是 human readable 的(例如 g++)。
\item \verb|tid.hash_code()| 返回类型的 hash, 类型是 \verb|size_t|
\item 所以检查变量类型最好的办法还是在 \verb|gdb|\upref{gdbNt} 用 \verb|p 变量| 显示类型和值, \verb|ptype 变量| 或 \verb|pt 变量| 显示类型。 检查长度用 \verb|p sizeof(变量)|。 \verb|typeid| 一般只用于检查两个类型是否相同或者一个变量是否是某个类型。
\item \verb|tid1 == tid2| 可以判断是否是同一类型。 这里加上 \verb|const| 或者 \verb|&| (或者两者)视为同一类型。 但加上 \verb|*| (指针)不是同一类型。
\end{itemize}

\subsection{继承相关}
\pentry{C++ 的 Class 笔记\upref{CppCla}}{nod_d8d1}
\begin{itemize}
\item 如果类型 \verb|B| 继承 \verb|A|, 定义 \verb|B b; A *pa = &b;|, 那么 \verb|typeid(*pa) == typeid(b)| 结果是什么呢? 这取决于 \verb|A,B| 的具体定义。
\item 用 \verb|g++ -std=11| 测试, 如果 \verb|A| 中没有 \verb|virtual| 函数(即不存在多态), 那么返回 \verb|false|。
\item 如果 \verb|A| 不是抽象类但有 \verb|virtual|(存在多态)则返回 \verb|true|。
\item 如果 \verb|A| 是一个抽象类(即存在 pure virtual function)则会产生编译错误, 因为 \verb|typeid| 无法对抽象类使用。
\item 如果 \verb|A| 不是抽象类但有 \verb|virtual|(存在多态)则返回 \verb|true|。
\item 所以(我猜) \verb|typeof()| 一般来说和是静态编译的, 但如果使用了多态, 那么将变为一个动态操作。
\end{itemize}
