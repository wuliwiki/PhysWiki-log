% Python 基础
% keys 基础
% license Xiao
% type Tutor

\begin{issues}
\issueDraft
\end{issues}

\subsection{标识符}

\begin{enumerate}
\item \textbf{标识符(identifier)}是指变量、函数、类、模块或其他对象的名称。 标识符的命名规范有:
\begin{itemize}
\item 开头不可以是数字
\item 必须由数字,字母,下划线组成
\item 区分大小写
\item 不能使用内置关键词(比如:\verb`print,int,float` 等等)
\end{itemize}
\item 在Python中,有一些常见的命名规则和约定,这些规则有助于编写可读性强且一致的代码。以下是一些常见的Python命名规则:
\begin{itemize}
\item
1.变量名和函数名:\\
2.使用小写字母。\\
3.使用下划线来分隔单词,这被称为蛇形命名法。\verb`my_test`)。\\
4.选择具有描述性的名称,以便代码的含义清晰可见。(my_name)

\item 常量名:\\
1.使用大写字母。\\
2.如果需要,可以使用下划线来分隔单词。\\
3.常量通常在模块级别定义。(\verb`PI, MAX_VALUE`)\\
\item 类名:\\
1.使用驼峰命名法。(\verb`CamelCase`)\\
2.每个单词的首字母都大写,不使用下划线。\\
3.例如:\verb`MyClass, PersonInfo, HttpRequest`.
\end{itemize}
\end{enumerate}