% 暗物质的速度分布
% license Usr
% type Tutor


然后,我们讨论暗物质速度分布 $f(x, v)$,我们将主要关注银河系。这种处理可以很容易地扩展到其他星系,尤其是矮星系,通过适当替换参数。最初,我们将忽略暗物质速度分布对位置 $x$ 的依赖(这在本节末尾放宽)。此外,银河系参考系中的暗物质速度分布通常假定为各向同性,因此 $f(v)$ 是仅 $v = |v|$ 的函数,我们经常谈论暗物质速度分布。

%**图 2.5:**来自样本数值模拟的暗物质速度分布。顶部(改编自 Kuhlen 等人(2010)[64]):在银河系参考系中的分布(左)和在地球参考系中的分布(右),如从 GHALO N-体模拟获得。实线红线是平均分布,浅(深)绿色阴影区域给出了 68\% 的散射(包络)。点线是最佳拟合的麦克斯韦-玻尔兹曼分布。左侧(改编自 Bozorgnia 和 Bertone(2017)[64]):在银河系参考系中的分布,来自样本流体动力学模拟。虚线黑线对应于具有 220 km/s 峰值速度的麦克斯韦-玻尔兹曼分布。

Eddington 公式将球面密度 $\rho (r)$ 与球面速度分布相关联。这被用来展示高斯分布对应于等温球体 $\rho \propto 1/r^2$。现在可以反转这个过程,找到更现实剖面的的速度分布。例如,幂律 $\rho \propto 1/r^\gamma$(适用于小 $r$)对应于 $f(v) \propto v^{(\gamma-6)/(2-\gamma)}$。这提供了一个有趣的近似,尽管 Eddington 公式背后的假设并不满足:分布不是严格随时间独立的,并且重子物质形成非球形盘。

这种讨论表明,暗物质速度分布不能严格遵循高斯麦克斯韦-玻尔兹曼分布。特别是,获得速度大于银河系逃逸速度 $v_{esc}$ 的暗物质粒子倾向于蒸发掉。为了解释这一点,银河系参考系中的暗物质速度分布$ f(v)$,通常假定为在有限逃逸速度处尖锐截止的各向同性麦克斯韦-玻尔兹曼分布
\begin{equation}
f(v) = N e^{(-v^2/v^2_0)} \Theta (v_{esc} - v)~.
\end{equation}
规范化常数固定,使得$ \int d^3v f(v) = 1$,在 $v_{esc} \gg v_0$ 的极限下,$N = \pi^{(-3/2)}v^{(-3)}_0$(这里 $d^3v = 4\pi v^2 dv$)。参数 $v_{esc}$ 和 $v_0$ 估计如下:

- 应用到等温球体的位力定理 $\langle K\rangle = -1/2\langle V \rangle$ 意味着 $2\sigma^2 = \langle v^2\rangle  = v^2_{circ}$,其中 $v_{circ}$ 是局部圆周速度,测量值为 $v_{circ} \sim (220 \pm 10) km/s$,来自恒星的运动。由于$ \langle v^2\rangle = 3v^2_0/2$ 这表明 $v_0 = \sqrt{2/3} v_{circ}$。然而,将暗物质分布近似为等温球体并不完全现实。特别是,它意味着逃逸速度无限大,因为封闭质量$ M(r)$ 在$ r \rightarrow\infty$ 时发散。

- 观测到的恒星运动表明,从银河系本地逃逸速度是
\begin{equation}
v_{esc} \sim (544 \pm 35) km/s~.
\end{equation}
测量到的古老恒星的速度,预计与暗物质的速度分布相似,表明
\begin{equation}
220 km/s < v_0 < 270 km/s~.
\end{equation}
N-体模拟表明了一个更复杂的图景。

- N-体模拟似乎表明了一个分布,具有凸起和凹陷,与简单的麦克斯韦-玻尔兹曼分布显著不同。此外,分布特征在$ v < v_{esc} $处有一个更平滑的截止,并且可以参数化为(Lisanti 等人(2011),Kuhlen 等人(2010)[64])
\begin{equation}
f(v) = N_k \bigg[  exp \bigg(\frac{v^2_{esc} - v^2}{kv^2_0}\bigg)-1\bigg]^k \Theta(v_{esc} - v)~.
\end{equation} 
其中 $1.5 < k < 3.5$。麦克斯韦-玻尔兹曼分布在 $k \rightarrow 0$ 的极限下得到。这些速度分布在图 2.4a 中绘制。

- 更近期的包括重子的流体动力学模拟似乎表明了一个更接近标准麦克斯韦-玻尔兹曼的速度分布,尽管 v0 比方程(2.25)中的值大几十 km/s。不同模拟之间的差异,以及同一模拟的不同实现之间的差异非常大,因此似乎不可能更精确地确定 v0。流体动力学模拟似乎也发现了一个比暗物质-仅模拟更各向同性的速度分布。

暗物质的平均速度随着银河系中的位置 r 变化 [68],见图 2.6。数值模拟建议一个简单的幂律缩放
v^3_0(r) ∝ r^χρ(r), (2.27)
其中指数 χ ≈ 1.9 − 2.1 在暗物质-仅模拟中,χ ≈ 1.60 − 1.67 在包括重子的模拟中。实际上,如 Bertschinger(1985)[68] 的开创性工作所预测的,基于球对称坍缩的这种关系,χ ≈ 1.875。位置变化的 v0 的大小取决于假定的 χ 值和暗物质密度剖面 ρ(r),见方程(2.27)。实际上,然而,v0 并没有太大变化,至少对于 r 在太阳位置 r_⊙ ≃ 8.3 kpc 内的范围内。对于从暗物质-仅模拟中得到的 χ 值,暗物质粒子在银河系中心平均较慢,在太阳系位置之外的外围较快。另一方面,对于包括重子的模拟得出的 χ 值,发现相反的趋势。关系(2.27)在 r 上保持 2.5 个数量级,从大约 r ≈ 0.1 kpc(N-体模拟的分辨率限制)到至少 r ≈ 20 kpc。

逃逸速度 vesc 显然也是位置依赖的。如果只考虑暗物质,逃逸速度可以很容易地从其引力势中预测出来,该势由半径 r 内封闭的暗物质质量决定。然而,在银河系的内部,重子构成了物质内容的一个重要部分,这使得确定 vesc(r) 更为复杂。有时使用以下经验参数化(见 Cirelli 和 Cline(2010)[68]):v^2_(esc)(r) = 2v^2_0(r) ∫ 2.29 + ln(10 kpc/r) . (2.28)
使用方程(2.27)对 v0(r),可以发现逃逸速度通常在银河系内部最高,正如预期的那样,尽管详细行为取决于 χ 和 ρ 的值。

额外的可能复杂情况,如暗物质流,将在 2.4.5 节中讨论。值得注意的是,尽管暗物质速度分布相当不确定,但大部分暗物质具有 v ∼ 10^(-3) 是稳健的,对于 

