% 电场的高斯定律证明
% keys 库仑定律|散度定理|电场|通量|球面|闭合曲面
% license Xiao
% type Tutor

\pentry{电场的高斯定律\upref{EGauss}, 球坐标系中的梯度散度\upref{SphNab}}

以下我们用使用库仑定律和散度定理\upref{Divgnc} 严谨地证明电场的高斯定律\upref{EGauss}。

我们先看一个位于坐标原点, 电荷为 $q$ 的点电荷产生的电场(\autoref{eq_Efield_4}~\upref{Efield})
\begin{equation}\label{eq_EGausP_1}
\bvec E(\bvec r) = \frac{q}{4\pi\epsilon_0}\frac{\uvec r}{r^2}~.
\end{equation}
要计算某点的散度, 最方便的做法是使用球坐标的散度算子(\autoref{eq_SphNab_2}~\upref{SphNab})得
\begin{equation}
\div \frac{\uvec r}{r^2} = \frac{1}{r^2} \pdv{r} \qty(r^2 \frac{1}{r^2}) = 0~.
\end{equation}
注意由于\autoref{eq_EGausP_1} 在原点处无定义, 也不存在偏导数, 该结论不适用于 $\bvec r = \bvec 0$。

由于散度算符是线性% 未完成
的, 即使空间中有许多点电荷, 第 $i$ 个产生的电场为 $\bvec E_i(\bvec r)$, 空间中任何点(除了点电荷的位置)的电场散度都为零。
\begin{equation}\label{eq_EGausP_3}
\div \qty(\sum_i \bvec E_i) = \sum_i \div \bvec E_i = 0~.
\end{equation}
我们还是假设只有一个点电荷位 $q$ 于坐标原点, 现在我们以原点为球心做一个半径为任意 $R > 0$ 的球面 $\mathcal S$, 并计算电场在球面上的通量为
\begin{equation}\label{eq_EGausP_2}
\Phi = \oint_{\mathcal S} \bvec E \vdot \dd{\bvec s} = \frac{q}{4\pi\epsilon_0 R^2} \cdot 4\pi R^2 = \frac{q}{\epsilon_0}~.
\end{equation}
然而根据散度定理(\autoref{eq_Divgnc_1}~\upref{Divgnc}), 如果 $\div \bvec E$ 在球内处处为零, 应该有 $\Phi = 0$ 才对。 所以问题应该出在原点, 由于散度定理要求矢量场在闭合曲面内部处处可偏导, % 链接未完成
 所以\autoref{eq_EGausP_2} 的结果并不严格适用于散度定理\footnote{但有办法可以弥补, 见 “点电荷电场的散度”}。 % 链接未完成

\subsubsection{绕开奇点}
我们可以以另一种方式避开 $\bvec r = 0$ 处的奇点使用散度定理。 结合\autoref{eq_EGausP_3} 和\autoref{eq_EGausP_2}, 可以用散度定理证明: 如果在球面 $\mathcal S$ 外部再任取一个闭合曲面 $\mathcal S'$ (正方向也向外), 那么电场在 $\mathcal S'$ 上的通量应该也是相同的:

我们把这两个曲面共同看成是它们之间的球壳形体积 $\bar{\mathcal V}$ 的内外表面(但 $\mathcal S$ 的正方向需要改变)。 而 $\bar{\mathcal V}$ 中散度处处有定义且等于零, 所以由散度定理得内外表面的电场总通量也为零
\begin{equation}
\Phi = -\oint_{\mathcal S} \bvec E(\bvec r) \vdot \dd{\bvec s} + \oint_{\mathcal S'} \bvec E(\bvec r)\vdot \dd{\bvec s} = \int_{\bar{\mathcal V}} \div \bvec E(\bvec r) \dd{V} = 0~,
\end{equation}
移项可得两个面积分相等。 证毕。

所以, 对于任意两个包含同一个点电荷 $q$ 的曲面(正方向都向外), 电场在它们上的通量都等于 $q/\epsilon_0$, 而 $q$ 也不一定需要在原点。 再根据电场的叠加原理和点乘的分配律(\autoref{eq_Dot_3}~\upref{Dot}), 如果曲面内有多个点电荷 $q_1, \dots, q_N$, 那么它们在曲面上的通量等于每个点电荷产生的通量之和, 即
\begin{equation}
\oint_{\mathcal S} \bvec E(\bvec r) \vdot \dd{\bvec s} = \frac{1}{\epsilon_0} \sum_i q_i~,
\end{equation}
这就证明了高斯定律的积分形式。 若曲面内的电荷是连续分布的, 我们只需要把求和变为对 $\mathcal S$ 内部的电荷密度分布 $\rho(\bvec r)$ 的体积分即可
\begin{equation}
\oint_{\mathcal S} \bvec E(\bvec r) \vdot \dd{\bvec s} = \frac{1}{\epsilon_0} \int \rho(\bvec r) \dd{V}~.
\end{equation}
注意如果电荷密度 $\rho(\bvec r)$ 处处为有限值, 那么电场的散度也将处处有定义。 对比散度定理可得空间中任意一点 $\bvec r$ 处电场的散度与电荷密度成正比。
\begin{equation}
\div \bvec E = \frac{\rho}{\epsilon_0}~,
\end{equation}
这就证明了电场高斯定律的微分形式。
