% 常见物理量
% 密度|流密度|通量|速度|加速度

\begin{issues}
\issueDraft
\end{issues}

% 未完成: 说明单位xx 的 xx 其实说的是极限

\pentry{速度、加速度\upref{VnA}, 通量\upref{SurInt}}

我们已知速度和加速度都由极限定义, 事实上, 许多其他物理量同样通过极限来定义。

\subsection{密度}
我们常把\textbf{质量密度} 简称为\textbf{密度}, 记为 $\rho$, 定义为
\begin{equation}
\rho = \lim_{\Delta V \to 0} \frac{\Delta m}{\Delta V}~,
\end{equation}
其中 $\Delta m$ 是体积元 $\Delta V$ 内的质量。 当我们讨论的是宏观物体的密度时, 这个极限的理解是, $\Delta V$ 远小于宏观尺度, 却远大于微观尺度。 因为微观粒子(质子, 电子等) 的体积非常小, 它们之间有很大的空间, 如果按照数学定义取无限小, 很大概率会得到密度为零。

类似地我们可以定义其他物理量的密度, 如\textbf{电荷密度}(将上式的质量 $m$ 换成净电荷 $q$), 粒子数密度(将 $m$ 换成粒子数 $N$)等。

\subsection{流密度}
(未完成)

% 在通量

% 该式中的 $\bvec F$ 有时被称为\textbf{流密度}。 以上面的水流场为例, 若把一定时间内流过曲面的水看做通量, 那么“单位体积中水的质量” 就是水的\textbf{质量密度}, 而流密度就是密度乘以速度矢量。


% 质量流密度
% 电流密度
