% 一维方势阱中的光电离模拟(Matlab)

\begin{issues}
\issueDraft
\end{issues}

\pentry{Crank-Nicolson 算法解一维含时薛定谔方程(Matlab)\upref{CraNic}}

\addTODO{代码未完成}
\begin{lstlisting}[language=matlab]
% Crank-Nicolson 法解一维薛定谔方程
% 等间距网格,稀疏矩阵

function TDSE_cn1d
% ==== 参数设置 ======
xmin = -80; xmax = 80; Nx = 1000; % x 网格
tmin = 0; tmax = 20; Nt = 400; % 时间网格
Nplot = 10; % 画图步数
ax = [xmin, xmax, -0.5, 0.5];
% 初始波函数
123234
% 方势垒 [-L,L], 高 V0
L = 4; V0 = 4;
E0 = 1; a = 0.2; w0 = 1; tc = 5; % 电场波包参数
Ef = @(t) E0*exp(-a*(t-tc).^2).*sin(w0*t); % 电场函数
V = @(x,t) V_fun(x,t,Ef,L,V0,q); % 势能函数
% ===================
close all;
psi_gs = @(x) 1/(2*pi*sig_x^2)^0.25/sqrt(1 + 1i*t0/(2*m*sig_x^2))...
      *exp(-(x-x0-p0*t0/m).^2/(2*sig_x)^2/(1 + 1i*t0/(2*m*sig_x^2)))...
      .*exp(1i*p0*(x-p0*t0/(2*m)));
x = linspace(xmin, xmax, Nx); dx = (xmax-xmin)/(Nx-1);
t = linspace(tmin, tmax, Nt); dt = (tmax-tmin)/(Nt-1);
psi = psi_gs(x).';
% 准备稀疏矩阵 [对角线, 上对角线, 下对角线]
ind1 = [1:Nx, 1:Nx-1, 2:Nx]; % 行标
ind2 = [1:Nx, 2:Nx, 1:Nx-1]; % 列标
% 动能矩阵非零元
T = -1/(2*m)*[-2*ones(1,Nx), ones(1,2*Nx-2)]/dx^2;
A = (1i*dt/4)*T; % 对角元稍后更新
figure; plot(x, real(psi)); hold on;
plot(x, imag(psi));
for it = 1:Nt
    % 更新对角元
    A(1:Nx) = 0.5 + (1i*dt/4)*(T(1:Nx) + V(x, t(it)+dt/2));
    Asp = sparse(ind1, ind2, A, Nx, Nx);
    tmp = Asp \ psi;
    % err = norm(Asp*tmp - psi);
    psi = tmp - psi;
    if mod(it, Nplot) == 0
        clf;
        plot(x, real(psi)); hold on;
        plot(x, imag(psi));
        plot([xmin, -L, -L, L, L, xmax], [0, 0, 0.4, 0.4, 0, 0], 'k');
        axis(ax);
        pause(0.2);
    end
end
end

% 势能函数:方势垒 + 电场
function ret = V_fun(x,t,L,V0,q,Ef)
ret = zeros(size(x));
ret(x > -L & x < L) = V0;
% 电场势能
ret = ret - (q*Ef(t))*x;
end
\end{lstlisting}
