% 中国科技大学 2016 年考研普通物理
% keys 中国科技大学|考研|普通物理



\subsection{简答/选择题(每题15分,共30分)}
\begin{enumerate}
\item 设玻尔兹曼常数为 $k$ ,一理想双原子分子气体,处于温度 $T$ 时,其分子平均能量是多少?
\item 以下哪些电场可以存在于没有电荷的局部空间内? $A$ 是常数, $i$ , $j$ , $k$ 分别是直角坐标系 $x$ , $y$ , $z$ 方向的单位矢量。请选择所有合适的答案。\\
A. $A$(2xy\textbf{i}-xz\textbf{k}) B. $A$(-xy\textbf{j}+xz\textbf{k}) C. $A$(xz\textbf{i}+xz\textbf{j}) D. $A$xyz(\textbf{i}+\textbf{j})
\end{enumerate}
\subsection{计算题(每题20分,共120分)}
\begin{enumerate}
\item 如图1所示,半径为 $R_1$ 的导体球外有同心的导体球壳,壳的内外半径分别为 $R_2$ 和 $R_3$ 。已知球壳带的电量为 $Q$ ,内球和无穷远处电势为0,求内球的电荷量和球壳的电势。\\
\begin{figure}[ht]
\centering
\includegraphics[width=5cm]{./figures/9e960b0eed690a7b.png}
\caption{计算题1图示} \label{fig_USTC16_1}
\end{figure}
\item 半径为 $R$ 的球面上均匀分布电荷 $q$ ,该球面以角速度 $\omega$ 绕它的直径旋转。求这个系统的磁矩。
\item 一水桶中装有足够多的水。让该水桶以一定的角速度 $\omega$ 绕其对称轴在水平面内稳定旋转,试定量计算水面的形状。重力加速度为 $g$ ,结论用以转轴为 $z$ 轴的柱坐标系表达。
\item 你双手拿着一面积足够大的平板迎着一喷射的水柱,水柱的流量为 $0.1m^{3}/s$ ,水流的速度为 $5m/s$ 。\\
(1)求平板静止时你给平板的力;\\
(2)如果你拿着平板以 $1m/s$ 的速度迎着喷射的水柱移动,该力为多大?假设水的密度为 $1000kg/m^{3}$ .
\item 一摩尔范德瓦尔斯状态方程的气体,如果它的内能由式 $u=cT-a/V$ ( $V$ 为摩尔体积, $a$ 为状态方程常数之一, $c$ 为常数)给出,计算气体的摩尔比热容 $c_v$ 和 $c_p$ .
\item 设某理想气体的绝热指数 $\gamma =c_p/c_v$ 为温度 $T$ 的函数。\\
(1)证明在准静态绝热过程中,气体的T和V满足函数关系 $F(T)V=C$ ,式中 $C$ 为常数,函数 $F(T)$ 的表达式为:\\
 $\ln F(T)=\int \frac{dT}{(\gamma-1)T}$ .\\
(2)利用(1)的结果,证明该气体的可逆卡诺循环的效率仍为:\\
$\eta =1-\frac{T_2}{T_1}$
\end{enumerate}