% 陪集和同余
% keys 子群|陪集|左陪集|右陪集|同余|阶|拉格朗日定理
% license Xiao
% type Tutor

\pentry{子群\upref{Group1}}

\begin{definition}{左陪集}

给定一个群 $G$ 和它的一个子群 $H$。从 $G$ 中任意挑一个元素 $x$ 出来,用它来\textbf{左乘} $H$ 中的每一个元素,得到一个集合 $\{x, xh_1, xh_2, \cdots\}$,这里的 $h_n$ 要取遍每一个在 $H$ 中的元素。这个集合,也可以写为 $\{xh|h\in H\}$,被叫做\textbf{子群 $H$ 关于元素 $x$ 的左陪集(left coset)},记作 $xH$。 

\end{definition}

类似地还可以定义右陪集,不过二者选其一作为代表来研究就可以了,我们习惯上主要研究左陪集。

注意这个表示方法:$xH=\{xh|h\in H\}$。这是一种非常简洁的表达方法,可以理解成 $x$ 逐个左乘 $H$ 中的元素得到的集合,也可以理解成是 $x$ 和 $H$ 之间的一个运算,结果是另一个集合。类似地,我们也可以用群里任意两个子集(不一定要求是子群)$A$ 和 $B$ 来生成一个新的子集 $AB=\{ab|a\in A, b\in B\}$,就是用 $A$ 中的每一个元素去左乘 $B$ 中的每一个元素,得到的所有结果的集合。当然,这种表示方法也可以推广到一切“元素之间可以进行运算的集合”:集合 $A$ 和 $B$ 之间的运算,就是 $A$ 和 $B$ 所有可能的元素运算的结果构成的集合。比如说,考虑到整数集合上可以进行小学所学的乘法和加法运算,我们也可以把 $n\mathbb{Z}$ 的表示方法理解为用 $n$ 去乘全体整数所得到的集合,这个集合在加法下构成群\footnote{在乘法下一般不构成群,除非 $n$ 是素数,在这种情况下 $n\mathbb{Z}$ 配合乘法和加法就构成了最常见的一种有限域。}。

\begin{theorem}{}
如果 $h\in H$,而 $H$ 是一个\textbf{群},那么 $hH=H$。
\end{theorem}

\textbf{证明}:

思路是这样的:首先证明,$h$ 去乘 $H$ 中的任何元素,结果还在 $H$ 里;接着证明,$H$ 中的任何元素,也总在 $hH$ 里。具体证明过程如下:

由\textbf{封闭性},对于任意 $h_0\in H$,总有 $hh_0\in H$,因此 $hH\subseteq H$。

任取 $h_0\in H$。由\textbf{逆元存在性},$h$ 有逆元 $h^{-1}\in H$,因此又由\textbf{封闭性}知 $h^{-1}h_0\in H$。于是有 $h_0=hh^{-1}h_0\in hH$,即 $H\subseteq hH$。

综上,$hH=H$。

\textbf{证毕}。

对于一般的左陪集 $xH$,如果用 $xH$ 中任意元素 $xh$(即 $h$ 是 $H$ 中的任意元素) 来左乘 $H$,得到的 $xhH$ 仍然是同一个左陪集:$xH=xhH$(因为 $H=hH$。)。 所以我们可以用左陪集中的\textbf{任何一个元素} $x'$ 来作为代表,把这个左陪集写成 $x'H$。

特别地,考察这个形式的集合\footnote{其中省略号表示一直列举下去, 但排除与前面重复的元素。 所以这个集合可能是有限的。}:$\{e, h, hh, hhh, hhhh, \cdots\}$。 那么同样地由于封闭性和运算唯一性可知,这个集合还是群 $H$ 的一个子集。特别地,$H$ 的群运算限制在这个集合上能构成一个循环群(\autoref{ex_Group_2}~\upref{Group})。只要我们把 $n$ 个 $h$ 相乘的结果记为 $h^n$,$n$ 个 $h^{-1}$ 相乘的结果记为 $h^{-n}$,那么如此生成的循环群就可以用指数的加法运算来处理了。由单个元素可以生成循环群这一概念,我们引入以下定义:

\begin{definition}{元素的阶}\label{def_coset_1}
给定一个群 $G$ 和一个 $g\in G$,如果存在一个正整数 $n$ 使得 $g^n=e$,那么我们称 $g$ 是一个有限阶的元素。特别地,所有满足条件的 $n$ 中最小的那一个,被称为 $g$ 的\textbf{阶(order)}或者\textbf{指数},记为 $\opn{ord}g$。特别地,如果任何整数 $n$ 都不能使 $g^n=e$,那么记 $\opn{ord}g=\infty$;规定 $\opn{ord}e=0$。
\end{definition}

\begin{definition}{子群的指数}\label{def_coset_2}
给定群$G$及其子群$H$,则$H$在$G$中的\textbf{指数(index)}是其在$G$中左陪集或右陪集的数量,记为$\abs{G:H}$,$[G:H]$或$(G:H)$。
\end{definition}

\begin{example}{$n\mathbb{Z}$ 的左陪集}\label{ex_coset_2}
整数加群的群运算是通常的加法,所以我们可以把元素 $k$ 所在的左陪集记为 $k+n\mathbb{Z}$。比如说,当 $n$ 为 $6$,$k$ 为 $1$ 时,$1+6\mathbb{Z}=\{\cdots, -11, -5, 1, 7, 13, \cdots\}$;当 $k$ 为 $8$ 时,$8+6\mathbb{Z}=\{\cdots, -10, -4, 2, 8, 14, \cdots\}=2+6\mathbb{Z}$。

$6\mathbb{Z}$ 一共有 $6$ 个不同的左陪集;类似地,$n\mathbb{Z}$ 一共有 $n$ 个不同的左陪集。
\end{example}
注意左陪集不一定是子群, 例如 $1 + 6\mathbb Z$ 中没有单位元。

\begin{example}{$3\mathbb{Z}$ 的子群和左陪集}\label{ex_coset_3}
$3\mathbb{Z}$ 虽然是 $\mathbb{Z}$ 的子群,但是它也可以从中再分离出子群来。由于 $3\mathbb{Z}$ 的集合是由全体 $3$ 的倍数构成的,因此全体 $6$ 的倍数构成的集合 $6\mathbb{Z}$ 是 $3\mathbb{Z}$ 的真子集,我们已经知道它构成群了。也就是说,$6\mathbb{Z}$ 既是 $\mathbb{Z}$ 的子群,又是 $3\mathbb{Z}$ 的子群,甚至还是 $2\mathbb{Z}$ 的子群。

从\autoref{ex_coset_2} 中我们知道,$6\mathbb{Z}$ 作为 $\mathbb{Z}$ 的子群,有 $6$ 个左陪集;但作为 $3\mathbb{Z}$ 的左陪集的时候只有 $6/3=2$ 个左陪集。

下图可以形象地表示 $\mathbb{Z}$,$3\mathbb{Z}$ 以及 $6\mathbb{Z}$ 之间的关系,一般的群和子群的关系也可以类似理解。

\begin{figure}[ht]
\centering
\includegraphics[width=5cm]{./figures/ab2952e7de275f05.png}
\caption{子群和陪集的示意图。整张图表示 $\mathbb{Z}$ 本身;中间的两个绿色区域表示 $3\mathbb{Z}$,左边的两个紫色区域表示 $1+3\mathbb{Z}$ 而右边的两个紫色区域表示 $2+3\mathbb{Z}$;中间上面的绿色区域表示 $6\mathbb{Z}$,而下面的绿色区域表示 $3+6\mathbb{Z}$,剩下的四个紫色区域分别表示 $6\mathbb{Z}$ 的另外四个左陪集。} \label{fig_coset_1}
\end{figure}

在这个图中可以清楚地看到,$3\mathbb{Z}=6\mathbb{Z}\cup(3+6\mathbb{Z})$。这很容易理解,因为 $3$ 的倍数无非两种情况,$6$ 的倍数或者 $6$ 的倍数再加 $3$。

\end{example}

左陪集的意义是将群划分成互不相交的子集,这是一个等价类划分,也就是说,“$x$ 和 $y$ 属于同一个左陪集”是一个等价关系\upref{Relat}。于是我们有了如下定理: 

\begin{theorem}{}\label{the_coset_1}

左陪集划分是一个等价类划分。

\end{theorem}

\textbf{证明}:

给定一个群 $G$,两个元素 $x, y\in G$,再给定它的一个子集 $H$,那么“$y$ 在左陪集 $xH$ 中”的等价表述,可以是“$y\in xH$”;在这个属于关系两边同时左乘一个 $x^{-1}$,还能得到更常用的等价表述:$x^{-1}y\in H$\footnote{如果你不理解为什么可以像解方程一样两边同时乘以一个元素,请再琢磨琢磨上文中 $xH$ 的定义是什么。}。

我们用最后这个等价表述来检查,“在同一个左陪集中”这一关系,是否是等价关系。
\begin{itemize}
\item 自反性:对于任意的 $x\in G$,由于 $e\in H$,故显然有 $x=xe\in xH$。因此 $x$ 和自己在同一个左陪集中。
\item 对称性:如果 $x^{-1}y\in H$,由于 $H$ 是个群,故 $y^{-1}x=(x^{-1}y)^{-1}\in H$。因此 $x$ 也在左陪集 $yH$ 中。对称性还说明,如果 $y$ 在 $xH$ 中,那么 $x$ 也在 $yH$ 中,因此这两个表述可以合二为一为“$x$,$y$ 在同一个左陪集中”。
\item 传递性:如果 $x^{-1}y\in H$,$y^{-1}z\in H$,那么由于 $H$ 的封闭性,$(x^{-1}y)(y^{-1}x)\in H$,拆开括号后得到 $x^{-1}z\in H$。 因此 $z$ 也在 $xH$ 中。
\end{itemize}

到此,我们证明了“在同一个左陪集中”是一个等价关系。由这个等价关系划分的左陪集,是一种等价划分,左陪集彼此互不相交。

\textbf{证毕}

\begin{definition}{同余}
对于群 $G$ 和其子群 $H$,如果 $x, y\in G$ 满足 $x, y$ 在同一个左陪集中,那么我们称 $x, y$ \textbf{模 $H$ 同余}\footnote{对比整数\upref{intger}中的同余概念。两个“同余”有什么联系吗?}。
\end{definition}

知道了左陪集是等价类,我们很容易发现群的一个优美的结构。在集合论中,子集的基数\upref{Set}可以是任意的,只要它小于等于原集合的基数就可以;但是子群的阶却必须能够整除原群的阶。这就是群论的\textbf{拉格朗日定理(Lagrange's Theorem)}:

\begin{theorem}{拉格朗日定理}\label{the_coset_2}

给定群 $G$,如果 $H$ 是 $G$ 的子群,那么 $|H|$ 可以整除 $|G|$。

\end{theorem}

\textbf{证明}:

假设 $x\in G$。根据 $xH$ 的定义,$|xH|\leq|H|$。 又由群运算的唯一性,这个不等式应该取等号:$|xH|=|H|$。所以每个左陪集的基数都一样大。

左陪集彼此不相交,每个元素 $x$ 都属于左陪集 $xH$,因此 $|G|$ 等于各个左陪集的基数之和,也就是 $|H|$ 的倍数。

\textbf{证毕}

拉格朗日定理还说明,任何有限群 $G$ 的元素,其指数都是 $|G|$ 的因子。
