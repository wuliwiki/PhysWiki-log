% Logical Volume Management(LVM)笔记
% license Xiao
% type Note

\begin{issues}
\issueDraft
\end{issues}

\pentry{Linux 分区和文件系统操作笔记\nref{nod_fdisk}}{nod_ec58}

\begin{itemize}
\item LVM 可以用于把若干个物理硬盘合并为一整个逻辑硬盘, 并在上面划分若干逻辑分区, 这些分区可以随时伸缩。
\item 据说 overhead 很小, 基本不影响读写速度。
\item 如果重启, mount 的就没有了, 用 \verb`sudo vgscan` 可以找到 \textbf{Volume Group(VG)}, 然后重新 mount 如百科服务器的 \verb`mount /dev/LVM/data /mnt/drive`
\item 其他命令包括 \verb|vgs|, \verb|lvs|, \verb|lvdisplay|, 它们都是指向 \verb|/sbin/lvm| 的软链, 但是 \verb|lvm| 会根据 \verb|arg[0]| 的不同来区分用的是哪个命令。
\end{itemize}

\begin{itemize}
\item 一个\href{https://www.howtoforge.com/linux_lvm}{教程}
\item \verb|apt-get install lvm2 dmsetup mdadm reiserfsprogs xfsprogs|
\item \verb|fdisk -l|
\item Now we prepare our new partitions for LVM: \verb|pvcreate /dev/sdb1 /dev/sdc1 /dev/sdd1 /dev/sde1|
\item 如果要删除, 用 \verb|pvremove /dev/sdb1 /dev/sdc1 /dev/sdd1 /dev/sde1|
\item \verb|pvdisplay| 可以显示信息
\item \verb|vgextend VG名 /dev/硬盘| 可以对 volume group 进行拓展
\item \verb|lvextend -l +100%FREE /dev/centos/root|
\end{itemize}
