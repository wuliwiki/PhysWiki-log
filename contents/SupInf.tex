% 上确界与下确界
% 确界|limit

\begin{issues}
\issueDraft
\issueTODO
\end{issues}

\addTODO{应整合到“六大实数完备性公理”的后续词条中.}
\addTODO{可以移动到序论中}

\pentry{完备公理\upref{Cmplt}} 

\subsection{上下确界}

有了实数的完备性,我们就可以得到一个很重要的概念,确界.为了介绍确界,我们首先要熟悉“界”的概念.

\begin{definition}{界}
设 $S$ 是实数集合 $\mathbb{R}$ 的非空子集.

如果存在实数 $a$,使得对于任意的 $x\in S$,都有 $a\geq x$,那么称 $a$ 是 $S$ 的一个\textbf{上界};如果不存在这样的 $a$,则称 $S$ 的上界是 $+\infty$.

如果存在实数 $b$,使得使得对于任意的 $x\in S$,都有 $b\leq x$,那么称 $b$ 是 $S$ 的一个\textbf{下界};如果不存在这样的 $b$,则称 $S$ 的下界是 $-\infty$.
\end{definition}

\begin{example}{}
对于实数集上的区间 $[a, b]$,$b$ 是它的一个上界,$b+1$ 也是它的一个上界.

对于“全体正偶数”的集合,它的下界可以是 $2$,也可以是 $0$、$-110$、$-507$ 等,但它的上界只有 $+\infty$.
\end{example}

实数子集的界通常不是唯一的.比上界大的实数都是上界,比下界小的实数也都是下界.只有当上界是 $+\infty$ 或者下界是 $-\infty$,该上界或下界才是唯一的.

界无法保证唯一性,因此很难用来刻画集合本身的性质.比如说,区间 $[0, 1]$ 和区间 $[0, 2]$ 是不同的集合,但实数 $2$ 都是它们的上界,于是光描述某些上界是完全无法体现这两个集合的区别的.但是我们也容易想到,一个集合的所有上界中,有一个上界是可以唯一确定的,这就是我们接下来要定义的\textbf{上确界}.

\begin{definition}{确界}
设 $S$ 是实数集合 $\mathbb{R}$ 的非空子集.

如果存在实数 $a$,使得对于任意的 $x\in S$,都有 $a\geq x$,且对于任意实数 $y<a$,$y$ 都不是 $S$ 的上界\footnote{即存在 $x\in S$,使得 $x>y$.},那么称 $a$ 是 $S$ 的一个\textbf{上确界(supremum)},记为 $\opn{sup} S=a$.

如果存在实数 $b$,使得对于任意的 $x\in S$,都有 $b\leq x$,且对于任意实数 $y>a$,$y$ 都不是 $S$ 的下界,那么称 $b$ 是 $S$ 的一个\textbf{下确界(infimum)},记为 $\opn{inf} S=b$.



\end{definition}


对于很多集合来说,上确界就是其中最大的元素,下确界就是其中最小的元素,比如区间 $[a, b]$ 的上下确界就分别是 $b$ 和 $a$.那么我们为什么不用集合的最大最小值来讨论,而是非要定义个确界呢?这是因为不是所有集合都有最大最小元素的,比如区间 $(a, b)$,它的上下确界依然是 $b$ 和 $a$,但它却没有最大最小值.

上面这段分析暗含了一个问题:如果不是所有集合都有最大最小值,那能否保证确界的存在呢?比如说,$S$ 的下确界是 $S$ 的界所构成的集合中的最大值,那我们能不能保证这个界的集合一定有最大值呢?

答案是肯定的,我们称之为\textbf{确界原理}.

\begin{definition}{确界原理}
如果非空的有界实数子集 $S$ 有上界,那么 $S$ 必存在上确界.

由此可推论,如果 $S$ 有下界,那么 $S$ 必存在下确界.
\end{definition}

注意,我们将确界原理写成了一条定义,因为它是刻画实数完备性的公理之一,而不是可供证明的定理.这些公理是完全等价的,任何一条都能推出其它所有,故任选其一作为公理来定义实数的完备性即可.完备性公理的整合描述请参见\textbf{实数的完备公理}\upref{RCompl}.

\subsection{存在唯一性}
有如下定理:

\begin{theorem}{上下确界的存在唯一性}
设 $\mathfrak{R}$ 是一个实数模型, $S\subset\mathfrak{R}$ 是其非空子集, 存在一个上界. 那么 $S$ 有\textbf{唯一确定的上确界}. 对于下确界的存在唯一性同理.
\end{theorem}

不论在哪个实数模型中, 唯一性都可以由序公理立即得到.

如果以戴德金分割\upref{ReNum}为基础构造实数模型, 那么对于给定的实数集 $S$, 它的上下确界都是可以明确地构造出来的. 实际上, 对于 $x\in S$, 如果写 $L_x$ 为其戴德金分割的下类, $R_x$ 为其戴德金分割的上类 (记得它们都是 $\mathbb{Q}$ 的子集), 那么交集
$$
R_s=\bigcap_{x\in S}R_x
$$
在加上最小元素 (如果有一个有理数能成为它的最小元素的话) 之后也仍然是一个上类, 而并集
$$
L_i=\bigcup_{x\in S}L_x
$$
在除去最大元素 (如果它本身有最大元素的话) 之后也仍然是一个下类.

\begin{exercise}{}
试证明这个论断; 只需要验证如上构造的 $R_s$ 和 $L_i$ 的确符合分割上类/下类的定义即可.
\end{exercise}

不难发现, 由分割上类 $R_s$ 确定的实数 $s$ 正符合上确界的定义, 而由分割下类 $L_i$ 确定的实数 $i$ 正符合下确界的定义.

\begin{exercise}{}
试证明这个论断. 提示: 例如, 设 $t<s$, 那么有一有理数 $q$ 严格介于 $t$ 和 $s$ 之间. 
\end{exercise}