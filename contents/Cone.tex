% 圆锥曲线的极坐标方程(高中)
% keys 极坐标系|圆锥曲线|离心率
% license Xiao
% type Tutor

\begin{issues}
\issueDraft
\end{issues}

\pentry{极坐标系\nref{nod_Polar},圆锥曲线的统一定义\nref{nod_HsCsFD}}{nod_f498}

\subsection{圆锥曲线的极坐标方程}

\begin{figure}[ht]
\centering
\includegraphics[width=8cm]{./figures/f04d0ae1f160e70c.pdf}
\caption{由离心率定义圆锥曲线}\label{fig_Cone_1}
\end{figure}

如\autoref{fig_Cone_1} 所示,以 $O$ 点为原点,使极轴垂直于准线。则有$d = p + r \cos \theta$,代入离心率定义得
\begin{equation}\label{eq_Cone_2}
e=\frac{r}{d}=\frac{r}{p + r \cos \theta }~.
\end{equation}
整理后可以得到圆锥曲线的极坐标方程。
\begin{definition}{圆锥曲线的极坐标方程}
选取一个焦点作为原点,对称轴作为极轴,可以得到圆锥曲线的极坐标方程为:

\begin{equation}\label{eq_Cone_3}
r(\theta) = \frac{ep}{1 - e\cos \theta },\qquad\theta\in\{\theta|r>0\}~.
\end{equation}

其中 $e$ 是\textbf{离心率(eccentricity)}, $p$ 是\textbf{焦准距(focal parameter)}。 极角 $\theta$ 的取值范围是所有使 $r>0$ 的值\footnote{下文为了方便表述, 把 $\theta$ 的取值范围限制在一个圆周内, 即 $(-\pi,\pi]$ 或 $[0, 2\pi)$。},具体来说:
\begin{itemize}
\item 当 $0 < e < 1$ 时,  $\theta$ 可以取任意实数;
\item 当 $e = 1$ 时, $\theta$ 可以取任意不等于 $2\pi n$ 的实数($n$ 取任意整数);
\item 当 $e > 1$ 时, 要求 $\theta_0< \theta + 2\pi n < 2\pi-\theta_0$, 其中 $n$ 取任意整数,其中$\displaystyle\theta_0 = \arccos\frac{1}{e}$。
\end{itemize}  
\end{definition}

注意 $p$ 和 $e$ 分别控制圆锥曲线的大小和形状。由于抛物线的 $e = 1$ 不变, 所以所有抛物线的形状都相同。

若定义过焦点且平行于准线的直线被圆锥曲线截出的线段长度为圆锥曲线的\textbf{通径},记为 $2l$,则$l$称为\textbf{半通径}。那么有 $r(\pi /2) = l$。 代入\autoref{eq_Cone_3} 得 $l = ep$。 所以\autoref{eq_Cone_3} 又可以写为

\begin{equation}\label{eq_Cone_5}
r(\theta)  = \frac{l}{1 - e\cos \theta }~.
\end{equation}

在一些文献中, 也把\autoref{eq_Cone_5} 中的负号写为正号, 此需要把\autoref{fig_Cone_2} 中的曲线旋转 $180^\circ$, 因为 $-\cos\theta = \cos(\theta - \pi)$。 上述 $\theta$ 的取值范围也需要加上 $\pi$。


\begin{figure}[ht]
\centering
\includegraphics[width=11cm]{./figures/52670f52be70ae3b.pdf}
\caption{不同离心率 $e$ 的圆锥曲线($p = 1$)} \label{fig_Cone_2}
\end{figure}



\subsection{椭圆}

这相当于把一个单位圆(方程 $x^2 + y^2 = 1$)在 $x$ 轴和 $y$ 轴分别拉长了 $a$ 倍和 $b$ 倍。 我们这里用焦点和准线的定义来推导出上式, 以证明它们等价。 我们不妨先以一个焦点为原点定义直角坐标系, 且令 $x$ 轴指向另一个焦点, 则有
\begin{equation}
r = \sqrt{x^2 + y^2}~, \qquad \cos\theta = \frac{x}{\sqrt{x^2 + y^2}}~.
\end{equation}

其中 $a$ 为\textbf{半长轴}, $b$ 为\textbf{半短轴}。这就是椭圆的第二种定义, 即把单位圆沿两个垂直方向分别均匀拉长 $a$ 和 $b$。 所以也可以

代入椭圆的极坐标方程\autoref{eq_Cone_5}  得
\begin{equation}
\sqrt{x^2 + y^2} = p + ex~.
\end{equation}
两边平方并整理得
\begin{equation}\label{eq_Cone_1}
(1 - e^2) \qty( x - \frac{ep}{1 - e^2} )^2 + y^2 = \frac{p^2}{1 - e^2}~.
\end{equation}
由此可见,如果我们把椭圆左移焦点到椭圆中心的距离 $ep/(1 - e^2)$,椭圆将具有标准形式,称之为焦距,即
\begin{equation}
c = \frac{ep}{1 - e^2}~.
\end{equation}
\autoref{eq_Cone_1}  和对比系数得
\begin{equation}
a = \frac{p}{1 - e^2}~, \qquad b = \frac{p}{\sqrt {1 - e^2} }~.
\end{equation}
以上两式可以将椭圆的极坐标方程转为直角坐标方程。 另外易证
\begin{equation}
a^2 = b^2 + c^2~.
\end{equation}
若要从直角坐标方程变回极坐标方程, 将逆转得
\begin{equation}\label{eq_Cone_4}
e = \frac{c}{a}~,\qquad
p = \frac{b^2}{a}~.
\end{equation}

\subsection{双曲线的两支}
\begin{figure}[ht]
\centering
\includegraphics[width=6cm]{./figures/3ca758fbfc998b7f.pdf}
\caption{双曲线分为互不相连的左右两支} \label{fig_Cone_3}
\end{figure}
根据双曲线的其他定义\upref{Hypb3}, 对同一个 $e>1$, 双曲线事实上是两条曲线, 每条曲线称为一支。 \autoref{fig_Cone_2} 中仅画出了离焦点较近的一支。 上文已经提到 $\theta_0< \theta < 2\pi-\theta_0$。

事实上\autoref{eq_Cone_5} 也可以表示双曲线的另一支, 只需要取 $-\theta_0< \theta < \theta_0$, 此时 $r$ 恒为负值。 若我们在极坐标中定义 $(-r, \theta)$ 和 $(r, \theta + \pi)$ 表示同一点, 就可以画出另一支。 或者说, 把\autoref{eq_Cone_5} 中的  $r,\theta$ 分别替换为 $-r$ 和 $\theta+\pi$ 就得到了这支双曲线的正常极坐标方程($r > 0$) 和极角范围
\begin{equation}\label{eq_Cone_6}
r(\theta) = -\frac{l}{1 + e\cos\theta} \qquad (\pi - \theta_0<\theta < \pi + \theta_0)~.
\end{equation}

