% 序列的极限

\pentry{极限\upref{Lim}, 实数集的拓扑\upref{ReTop}}

\subsection{基本定义解析}

序列的极限是分析数学中最基本的定义. 词条 数列的极限(简明微积分)\upref{Lim0} 和 极限\upref{Lim} 已经给出了一些序列极限的例子, 它的形式定义以及背后的直观解释. 为完整起见, 这里再重复一次序列极限的定义:

\begin{definition}{数列的极限}
考虑数列$\{a_n\}$.若存在一个实数$A$,使得对于\textbf{任意}给定的\textbf{正实数} $\varepsilon > 0$(无论它有多么小),总存在正整数 $N_\epsilon$, 使得对于所有编号 $n>N_\varepsilon$ ,都有 $\abs{a_n - A} < \varepsilon$成立,那么数列 $a_n$ 的极限就是 $A$.

将“数列$\{a_n\}$的极限是$A$”表示为$\lim\limits_{n\to\infty}a_n=A$.
\end{definition}

正如之前两个词条所解释的, 等式$\lim\limits_{n\to\infty}a_n=A$所表达的含义是"序列$a_n$随着$n$的增大将可以任意地接近$A$". 或者说, 对于序列$\{a_n\}$进行极限运算, 就是要找到"序列$a_n$越来越接近的那个数". 这种运算显然跟实数的四则运算不一样.

有极限的序列常常称为收敛 (convergent) 的. 如果没有极限, 则序列称为发散 (divergent) 的.

\begin{example}{求基本极限}
证明$$\lim\limits_{n\to\infty}\frac{1}{2^n}=0.$$

在给出严格证明之前, 首先来看看序列$\{2^{-n}\}$到底能够多么接近零. 直观上, 我们知道它衰减的速度非常快, 例如第四项$2^{-4}=0.0625$, 而第八项已经是$2^{-8}=0.00390625$. 相比之下, 倒数序列$\{1/n\}$的第四项只是$1/4=0.25$, 第八项只是$1/8=0.125$. 因此, 即便不借助对数运算, 也可以说明序列$\{2^{-n}\}$会逐渐接近于零.

转向严格证明. 首先注意到初等的不等式$2^n>n$对于任何整数$n\geq1$都成立; 这可以使用数学归纳法得到. 因此, 给定一个误差$\varepsilon>0$之后, 要使得$2^{-n}$同零的误差不大于$\varepsilon$, 只需要$1/n$同零的误差不大于$\varepsilon$就够了, 而为了达到这一点, 只要$n>1/\varepsilon$就够了. 因此, 只要取脚码
$$
N_\varepsilon=\left[\frac{1}{\varepsilon}\right]+1,
$$
即可保证当$n>N_\varepsilon$时有$2^{-n}<\varepsilon$.
\end{example}

当然, 直观上容易看出, 序列$\{2^{-n}\}$衰减得比倒数序列$\{1/n\}$要快多了. 上面的证明当然远远不是最精确的. 为了刻画一个有极限的序列$\{a_n\}$收敛的速度, 可以考虑如下问题: 给定了一个误差$\varepsilon>0$之后, 为了使得$|a_n-A|<\varepsilon$能够一直成立, 脚码$n$至少得是多大? 与此相关的概念正是无穷小的阶. 

\subsection{基本性质}
序列的极限运算有如下基本性质:

\begin{theorem}{极限的基本性质}
\begin{itemize}
\item 序列的极限若存在, 则必定是唯一的.
\item 极限运算保持序关系: 如果$\lim\limits_{n\to\infty}a_n=A$, $\lim\limits_{n\to\infty}b_n=B$, 而且从某个$n$开始有$a_n\geq b_n$, 那么必然有$A\geq B$.
\item 设$\lim\limits_{n\to\infty}a_n=A$, $\lim\limits_{n\to\infty}b_n=B$, 则序列$\{a_n\pm b_n\}$和$\{a_n b_n\}$都有极限, 且$\lim\limits_{n\to\infty}a_n\pm b_n=A\pm B$, $\lim\limits_{n\to\infty}a_nb_n=AB$.
\item 设$\lim\limits_{n\to\infty}a_n=A$, $\lim\limits_{n\to\infty}b_n=B\neq0$, 则
$$
\lim\limits_{n\to\infty}\frac{a_n}{b_n}=\frac{A}{B}.
$$
\end{itemize}
\end{theorem}

以序列之和的极限为例进行说明. 