% 国际单位制词头
% keys 国际单位|数量级|词头|千米
% license Xiao
% type Tutor

\pentry{国际单位制\nref{nod_Consts}}{nod_fa2a}

\footnote{参考 Wikipedia \href{https://en.wikipedia.org/wiki/Metric_prefix}{相关页面}。}在使用国际单位时, 我们会在一些单位符号或全称前面加上一个表示数量级的前缀, 正式的名字叫做\textbf{国际单位制词头(metric prefix)}以方便书写。 如长度单位 $\Si{m}$ (米) 可以加不同的词头拓展成更小的单位 $\Si{cm}$(厘米, centimeter), $\Si{mm}$(毫米, millimeter), $\Si{km}$ (千米, kilometer)。

国际单位制共有 20 个词头, 下面我们来看每个词头的全称和代表的数量级(\autoref{tab_UniPre_1})。

\begin{table}[ht]
\centering
\caption{词头列表}\label{tab_UniPre_1}
\begin{tabular}{|c|c|c|c|c|c|}
\hline
全称 & 词头 & 数量级 & 全称 & 词头 & 数量级 \\
\hline
分(deci) & d & $10^{-1}$ & 十(deca) & da & $10^1$ \\
\hline
厘(centi) & c & $10^{-2}$ & 百(hecto) & h & $10^2$ \\
\hline
毫(milli) & m & $10^{-3}$ & 千(kilo) & k & $10^3$ \\
\hline
微(micro) & $\mu$ & $10^{-6}$ & 兆(mega) & M & $10^6$ \\
\hline
纳(nano) & n & $10^{-9}$ & 吉(giga) & G & $10^9$ \\
\hline
皮(pico) & p & $10^{-12}$ & 太(tera) & T & $10^{12}$ \\
\hline
飞(femto) & f & $10^{-15}$ & 拍(peta) & P & $10^{15}$ \\
\hline
阿(atto) & a & $10^{-18}$ & 艾(exa) & E & $10^{18}$ \\
\hline
仄(zepto) & z & $10^{-21}$ & 泽(zetta) & Z & $10^{21}$ \\
\hline
幺(yocto) & y & $10^{-24}$ & 尧(yotta) & Y & $10^{24}$ \\
\hline
\end{tabular}
\end{table}
