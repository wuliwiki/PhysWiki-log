% 应力、应变场的唯一性与叠加原理
% license Xiao
% type Tutor

\begin{issues}
\issueDraft
\end{issues}

\pentry{应力\upref{STRESS},位移与应变\upref{Strain}}

\subsection{唯一性}
\begin{figure}[ht]
\centering
\includegraphics[width=7cm]{./figures/53939ee695242d56.pdf}
\caption{应力与应变场的唯一性} \label{fig_SSPRN_1}
\end{figure}
\footnote{本文参考了陆明万与冯西桥的《弹性力学》}应力与应变场有一个非常好的性质:如同静电场\upref{empoi}问题,只要边界条件(即材料所受的外力,或材料所受的约束)与材料种类确定,那么材料内部应力、应变的分布也就唯一确定。换言之,只要一个解满足所有的边界条件,那么他就是唯一的、正确的解。这个性质允许我们先\textsl{不择手段}地解出应力、应变场,再用唯一性原理说明这个解是正确的。

\subsection{线性;叠加原理}
\begin{figure}[ht]
\centering
\includegraphics[width=12cm]{./figures/54c9efddd89a4f0f.pdf}
\caption{应力与应变场的可叠加性} \label{fig_SSPRN_2}
\end{figure}
应力与应变场的另一个非常好的性质是,应力与应变场满足叠加原理。也就是说,总的应力场等于各外力单独引起的应力场之和。我们在说明本构关系\upref{ELACST}时其实已经运用了叠加原理。%注意,单独一组外力(例如“第一组边界条件”)引起的应力场不一定满足总的边界条件(例如“最终的边界条件”)。

\subsection{圣维南原理}
%我也不大会这个...
