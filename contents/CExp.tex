% 指数函数(复数)
% keys 复数|指数函数|三角恒等式|欧拉公式|导数
% license Xiao
% type Tutor

\begin{issues}
\issueTODO
\end{issues}

%未完成
%(计划内容: 定义,复平面上的特征,是解析函数,反函数,物理上最广的应用)。

\pentry{指数函数, 复数\upref{CplxNo}, 基本初等函数的导数\upref{FunDer}}

\footnote{参考 Wikipedia \href{https://en.wikipedia.org/wiki/Euler's_formula}{相关页面}。}我们来把指数函数从实数自变量拓展到复数自变量的情况。 后者有多种等效的定义方式, 我们首先来看一个较简单的定义

\begin{definition}{复指数函数}
令 $z = x + \I y$ 为任意复数\upref{CplxNo}, 那么复数域中的指数函数的一种定义就是
\begin{equation}\label{eq_CExp_1}
w = \E^z = \E^{x + \I y} = \E^x(\cos y + \I\sin y)~,
\end{equation}
\end{definition}

\begin{figure}[ht]
\centering
\includegraphics[width=5.7cm]{./figures/8e240d79f6337164.pdf}~.
\caption{复数域中的指数函数} \label{fig_CExp_1}
\end{figure}

在复平面上表示这个函数,则指数的实部 $x$ 控制函数值 $w$ 的模长, 虚部 $y$ 控制 $w$ 的幅角, 如\autoref{fig_CExp_1}
 \begin{equation}
\abs{w} = \E^x \qquad \arg(w) = y~,
\end{equation}
当指数为纯虚数(令 $x = 0$, $y = \I\theta$, $\theta$ 为实数)时,\autoref{eq_CExp_1} 变为著名的\textbf{欧拉公式(Euler's formula)}
\begin{equation}\label{eq_CExp_2}
\E^{\I \theta} = \cos \theta + \I\sin \theta~,
\end{equation}
虽然这里的 $\theta$ 是实数(最常见的情况),但根据复数域三角函数的定义\upref{CTrig}, 对于任何复数 $z$,同样满足欧拉公式
\begin{equation}
\E^{\I z} = \cos z + \I\sin z~.
\end{equation}
将“三角函数(复数)\upref{CTrig}”中的\autoref{eq_CTrig_1} 和\autoref{eq_CTrig_2} 代入即可证明。

根据\autoref{eq_CExp_1} 的定义结合两角和公式(\autoref{eq_TriEqv_1}~\upref{TriEqv}), 容易证明(留做习题) $\E^z$ 同样满足恒等式
\begin{equation}
\E^{z_1 + z_2} = \E^{z_1}\E^{z_2}~.
\end{equation}

\subsection{和实函数的 “兼容性”}
根据定义容易看到, 在实数轴 $x$ 上, \autoref{eq_CExp_1} 中 $y = 0$, 复指数函数就还原成熟悉的实数指数函数了。 所以二者使用同一个符号不会带来歧义, 复指数函数是实指数函数在复平面上的拓展。

但乍看之下, 拓展有许多方式, 为什么一定是\autoref{eq_CExp_1} 呢? 因为在复变函数中, 我们一般研究的是解析函数。 若要求 $\E^x$ 拓展到复数域后是解析函数\autoref{eq_CExp_1} 是唯一的定义。 详见 “柯西—黎曼条件\upref{CauRie}”。

\subsection{导数}
虽然我们只学了实函数的导数\upref{FunDer}, 还没有系统地学习复变函数求导, 但我们可以根据\autoref{eq_CExp_2} 求出一个常见的导数公式。 令 $a, x$ 为实数, $a$ 为常数,
\begin{equation}\ali{
\dv{x} \E^{\I ax} &= -a\sin(ax) + \I a\cos(ax)\\
&= \I a[\cos(ax) + \I \sin(ax)]\\
&= \I a \E^{\I ax}~,
}\end{equation}
进一步拓展, 令复常数 $c = a + \I b$ 得
\begin{equation}
\dv{x}\E^{c x} = \dv{x} \qty(\E^{ax}\E^{\I bx}) = (a + \I b)\E^{(a+\I b)x} = c\E^{cx}~.
\end{equation}
可见 $\E^z$ 的求导与实数域的 $\E^x$ 类似(\autoref{eq_FunDer_1}~\upref{FunDer})。

这只是对实数 $x$ 求导的特殊情况, 一般来说, 复变函数是可以对复数求导的, 详见 “复变函数的导数、柯西—黎曼条件\upref{CauRie}”。

\subsection{欧拉公式的导出}

欧拉公式最初是将泰勒级数进行推广得到的。

首先,我们有如下三个展开:
\begin{equation}\label{eq_CExp_3}
\sin x = x - \frac{1}{3!} x^3 + \frac{1}{5!} x^5 - \frac{1}{7!} x^7 \ldots
\qquad (x \in \mathbb R)~,
\end{equation}
\begin{equation}\label{eq_CExp_5}
\cos x = 1 - \frac{1}{2!} x^2 + \frac{1}{4!} x^4 -\frac{1}{6!} x^6 \ldots
\qquad (x \in \mathbb R)~,
\end{equation}
\begin{equation}
\E^x =1 + x + \frac{1}{2!} x^2 + \frac{1}{3!} x^3  \ldots
\qquad (x \in \mathbb R)~.
\end{equation}


由于我们已经知道复数的加减乘除运算,欧拉便尝试直接把虚数 $\I$ 插入\autoref{eq_CExp_4}  中,得到
\begin{equation}\label{eq_CExp_4}
\E^{\I x} =1 + \I x - \frac{1}{2!} x^2 - \frac{1}{3!} \I x^3  \ldots
\qquad (x \in \mathbb R)~.
\end{equation}

把\autoref{eq_CExp_4} 中的实部和虚部分开,再和\autoref{eq_CExp_3} 、\autoref{eq_CExp_5} 对比,即得欧拉公式\autoref{eq_CExp_1} 。

\subsection{复指数函数的级数定义}
同理, 用级数法也可以直接推导出\autoref{eq_CExp_1}。 事实上, 这种方法叫做 “解析拓延”, 就是在保证函数解析的前提下给函数扩大定义域, 例如从实轴拓展到复平面。

使用复数的加法和乘法, 可以用级数来定义复指数函数
\begin{definition}{复指数函数}
令复指数函数为
\begin{equation}
\E^z = \sum_{n=0}^\infty \frac{z^n}{n!}~.
\end{equation}
\end{definition}
\addTODO{证明等效}
