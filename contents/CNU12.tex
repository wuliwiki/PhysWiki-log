% 首都师范大学 2012 年 考研 量子力学
% license Usr
% type Note

\textbf{声明}:“该内容来源于网络公开资料,不保证真实性,如有侵权请联系管理员”

一、能量为 $E$的粒子沿$x$正方向入射,遇到势垒$U(x)=\delta(x)$。(共30分,每小题15分)
\begin{enumerate}
\item 波函数标准化条件中的连续性变成什么性质了?
\item 透射率是多少?
\end{enumerate}

二、某势阱中有两个粒子分别处于两个不同的单粒子本征态,空间波函数表示为$\varphi_1$,和$\varphi_2$,忽略粒子间相互作用。(共30分,每小题10分)
\begin{enumerate}
\item 两个粒子都是电子,体系有几个相互独立的总波函数?
\item 两个粒子都是自旋为0,写出总波函数的空间部分。
\item 一个是电子,一个是自旋为1的粒子,整个体系的总自旋最小量子数多大?
\end{enumerate}

三、宽为$a$的一维无限深势阱中的粒子处于归一化状态:
$$\psi(x) = \sqrt{\frac{1}{2a}} \sin  \frac{\pi x}{a} \left(1 - 2 \sqrt{3} \cos  \frac{\pi x}{a} \right)~$$
能量可观测值及其几率是多少?(30分)

四、对于单电子自旋(共30分每小题10分)
\begin{enumerate}
\item 将自旋算符$S_y$,对角化。(10分)
\item 写出$S_x$ 在$S_y$,表象下的表示。(10分)
\item 写出$S_y$的两个本征态在$S_y$表象下的表示(10分)
\end{enumerate}

五、证明题(共30分,每小题15分,任选两题,如果都做,按得分最高的两道题计分)
\begin{enumerate}
\item 导出定态薛定谔方程。
\item 导出微扰一级修正的能量表达式。
\item 证明平均值公式$\overline{F} = \int \psi^* \hat{F} \psi d\tau$。
\end{enumerate}
