% 可分扩张
% 可分多项式|完全域|完美域|形式微商|微商|导数|形式导数|separable polynomial|重根|自同构|形式微分|Frobenius同态


\pentry{分裂域\upref{SpltFd}}

%找到两篇深入讨论可分扩张的材料,可以作为本文的辅助:https://kconrad.math.uconn.edu/blurbs/galoistheory/separable1.pdf和https://kconrad.math.uconn.edu/blurbs/galoistheory/separable2.pdf

\begin{figure}[ht]
\centering
\includegraphics[width=12cm]{./figures/SprbEx_1.pdf}
\caption{本节内容示意图.} \label{SprbEx_fig1}
\end{figure}


由\autoref{SpltFd_the1}~\upref{SpltFd}可见,不可约多项式在其分裂域中有无重根,决定了该分裂域的自同构数量.一个域的全体自同构配合映射的复合,能构成一个群,域的许多性质都蕴含在这个群的结构中.群的元素数量自然是其重要性质之一.

综上所述,研究多项式的重根是非常重要的课题.

本节首先要提出形式微商的概念,这是一个用于判断多项式有无重根的有力工具.接着,就多项式有无重根问题,我们引入“可分”的概念,得到一系列如可分元素、可分多项式、可分扩张、完美域等重要概念,这一过程中会体现出形式微商简洁有力的作用.注意,本节的“完备域”或者说“完全域”,和数学分析中“所有柯西列都收敛”的“完备”不是一个概念;不过为了避免区分负担,我们尽量使用“完美域”这一术语.

本节中如无特别声明或定义,多项式都是指\textbf{首一多项式},即最高次项系数为$1$.这是合理的简化:我们讨论的是域上的多项式,因此任何多项式总可以乘以最高次项系数的乘法逆元来得到首一多项式.

另外,由于本词条篇幅限制,关于可分扩张的更多讨论会在\textbf{本原元素与单代数扩张}\upref{PrmtEl}和\textbf{可分元素的单扩张是可分扩张}\upref{SprbE2}词条中完成.

\subsection{形式微商与重根}

为了研究重根,我们借用微积分的知识,引入\textbf{形式微商}的概念

\begin{definition}{形式微商}\label{SprbEx_def1}

设$\mathbb{F}$是一个域,$f\in\mathbb{F}[x]$.若$f$表达为
\begin{equation}
f(x) = \sum_{i=0}^n a_ix^i
\end{equation}
其中各$a_i\in\mathbb{F}$,那么定义\textbf{形式微商}算子$\opn{D}:\mathbb{F}[x]\to\mathbb{F}[x]$为:
\begin{equation}
\opn{D}f = \sum_{i=0}^{n-1} (i+1)a_{i+1}x^i = \sum_{i=1}^{n} ia_{i}x^{i+1}
\end{equation}

形式微商也可称为“形式求导”、“形式微分”等.

\end{definition}

\autoref{SprbEx_def1} 形式微商就是直接套用微积分中的求导操作,只不过这里没有求导的概念,而是就进行多项式变换,其形式就是求导或者求导的推广,因此才叫\textbf{形式}微商.

\begin{example}{}
考虑域$\mathbb{Z}_3$上的多项式$f(x)=x^5+2x^2-x-2$,则
\begin{equation}
\opn{D}f(x) = 2x^4+x-1
\end{equation}

显然,这和真正的求导不同.一方面我们没有在域$\mathbb{Z}_3$上定义极限的概念,另一方面实数域上$f$的导函数应该是$5x^4+4x-1$.
\end{example}

\begin{example}{一个很抽象的例子}

考虑$\mathbb{R}$上的函数集合$S=\{\text{非零函数}\}\cup\{f\mid f(x)\equiv 0\}$,则$S$配上函数之间逐点相加和逐点相乘的运算,构成一个域,记为$\mathbb{S}$.

取$f, g\in\mathbb{S}$,构成多项式$F(y)=(f+g)y^2+(fg)y$.则
\begin{equation}
\opn{D}F(y) = (2f+2g)y+fg
\end{equation}

\end{example}

注意,$\opn{D}f(x)$这一表示应理解为“多项式$\opn{D}f$和表示其自变量的抽象符号$x$”,而不是“对多项式$f(x)$进行$\opn{D}$操作”.换言之,$\opn{D}f$是整体,故应是$\opn{D}f$和$(x)$,而不是$\opn{D}$和$f(x)$.

容易验证,形式微商有以下性质:

\begin{theorem}{形式微商的性质}\label{SprbEx_the1}

给定域$\mathbb{F}$,$\opn{D}$是$\mathbb{F}[x]$上的形式微商算子.则:

1. 任取$a\in\mathbb{F}$,则作为多项式,$\opn{D}a=0$.

2. 当$\opn{ch}\mathbb{F}=0$\footnote{概念见\autoref{field_def2}~\upref{field}. }时,对于$f\in\mathbb{F}[x]$有:$\opn{D}f=0 \iff \opn{deg}f=0$.

3. 对于$f, g\in\mathbb{F}[x]$,$a, b\in\mathbb{F}$,有$\opn{D}(af+bg)=a\opn{D}f+b\opn{D}g$.

4. 对于$f, g\in\mathbb{F}[x]$,有$\opn{D}(fg) = (\opn{D}f)g+f\opn{D}g$.

\end{theorem}

用\autoref{SprbEx_the1} 中的性质可知,如果$a$是$f\in\mathbb{F}[x]$在$\mathbb{F}$上的重根,那么存在$g\in\mathbb{F}[x]$使得$f(x)=(x-a)^2g(x)$.这样一来,$\opn{D}f(x)=(x-a)\qty[2g(x)+(x-1)\opn{D}g(x)]$.也就是说,$a$还是$\opn{D}f$的根.

\autoref{SprbEx_the1} 中这些性质都是由导数的性质自然启发而得的.但要注意的是,函数的导数和多项式的形式微商在概念上有所重叠,却不是互相包含的.函数的导数可以用来处理非多项式的函数,而多项式的形式微商又可以处理非实数域的多项式,所以二者都有对方所不能处理的领域,勿随意混为一谈.\autoref{SprbEx_the1} 中的性质能成立,也是需要额外验证的,而不能说因为导数有这些性质所以形式微商就一定有.

下面这个定理说的就是导数所不具备的性质:

\begin{theorem}{}\label{SprbEx_the2}
设$\mathbb{K}$是$f\in\mathbb{F}[x]$的分裂域,$a\in\mathbb{K}$是$f$的一个$k$重根.

1. 如果$\opn{ch}\mathbb{F}\nmid k$,那么$a$是$\opn{D}f$的$k-1$重根;

2. 如果$\opn{ch}\mathbb{F}\mid k$,那么$a$是$\opn{D}f$的\textbf{至少}$k$重根.
\end{theorem}

\textbf{证明}:

由题设可知,存在$g(x)\in\mathbb{K}[x]$使得$f(x) = (x-a)^k g(x)$,其中$g(a)\neq 0$.

1. 

\begin{equation}
\opn{D}f(x)=(x-a)^{k-1}\qty[kg(x)+(x-a)\opn{D}g(x)]
\end{equation}

显然,$[kg(x)+(x-a)\opn{D}g(x)]\mid_{x=a}\neq 0$.因此,$a$是$\opn{D}f$的$k-1$重根.


2. 

当$\opn{ch}\mathbb{F}\mid k$,则$k=0$.因此

\begin{equation}
\begin{aligned}
\opn{D}f(x)&=(x-a)^{k-1}\qty[kg(x)+(x-a)\opn{D}g(x)]\\
&=(x-a)^k\opn{D}g(x)
\end{aligned}
\end{equation}

故$a$至少是其$k$重根.如果$\opn{D}g(a)=0$,重数还会更高.

\textbf{证毕}.



\begin{corollary}{}\label{SprbEx_cor1}
设$\mathbb{K}$是次数大于$1$的多项式$f\in\mathbb{F}[x]$的分裂域,则$f$在$\mathbb{K}$中无重根的充分必要条件是$(f, \opn{D}f)=1$\footnote{即最大公因子为$1$.}.
\end{corollary}

\textbf{证明}:

必要性:

由于$f$在$\mathbb{K}$中无重根,而域的特征不可能是$1$,因此适用\autoref{SprbEx_the2} 中第一个情况.

因为$\mathbb{K}$是$f\in\mathbb{F}[x]$的分裂域,且$f$无重根,故在$\mathbb{K}$中有
\begin{equation}
f(x) = a_0(x-a_1)(x-a_2)\cdots(x-a_n)
\end{equation}
其中各$a_i\in\mathbb{K}$且各不相同.

则
\begin{equation}
\begin{aligned}
\opn{D}f(x) &= a_0[(x-a_2)(x-a_3)\cdots(x-a_n)\\
&+(x-a_1)(x-a_3)\cdots(x-a_n))\\
&\vdots\\
&+(x-a_1)(x-a_2)\cdots(x-a_{n-1}))
]
\end{aligned}
\end{equation}

$f$的因子必是若干个不同的$(x-a_i)$的积,但各$(x-a_i)$都不是$\opn{D}f$的因子.

充分性:

反设有重根,那至少是个二重根.于是$f$可以写成:

\begin{equation}
f(x) = a_0(x-a_1)(x-a_2)\cdots(x-a_n)
\end{equation}
其中各$a_i\in\mathbb{K}$,且$a_1=a_2$.

那么$(x-a_1)=(x-a_2)$就是$f$和$\opn{D}f$的公因子.

\textbf{证毕}.



\begin{corollary}{}\label{SprbEx_cor2}
设$\mathbb{K}$是$f\in\mathbb{F}[x]$的分裂域,且$f$是$\mathbb{F}[x]$上次数大于$1$的\textbf{不可约}多项式,那么$f$在$\mathbb{K}$中无重根的充分必要条件是$\opn{D}f \neq 0$.
\end{corollary}

\textbf{证明}:

注意\autoref{SprbEx_cor1} 和\autoref{SprbEx_cor2} 的题设差异在于,后者多了“在$\mathbb{F}[x]$上不可约”这一要求.我们只需要证明加上这一要求时,$(f, \opn{D}f)=1 \iff \opn{D}f\neq 0$即可.

$\Rightarrow$:

如果$\opn{D}f=0$,则$(f, \opn{D}f)=f\neq 1$.

$\Leftarrow$:

设$g(x)=(f(x), \opn{D}f(x))\in\mathbb{F}[x]$,于是$g\mid f$.由于$f$不可约,故要么$g=1$,要么$g\propto f$\footnote{这里$\propto$是“正比”符号,意为$g$与$f$只相差$a$倍,其中$a\in\mathbb{F}$.但是考虑到本节开头的声明,无特别定义,则$g$也是首一多项式,这里实际上也可以写成$g=f$.}.

但$g\mid \opn{D}f$导致$\opn{deg}g\leq\opn{deg}\opn{D}f<\opn{deg}f$,因此必有$g(x)=1$.



\textbf{证毕}.

对于次数大于$1$的多项式$f$,其形式微商可以为$0$.比如,$\mathbb{Z}_3$上的多项式$x^3+1$就是这样.不过$x^3+1$在$\mathbb{Z}_3[x]$上是可约的.

下面我们给出一个例子,说明存在次数大于$1$的不可约多项式之形式微商为$0$,作为\autoref{SprbEx_cor2} 的一个小验证.





\begin{example}{}\label{SprbEx_ex1}

考虑这样一个域$\mathbb{F}$,其定义为:$\mathbb{F}$是全体以$t$为不定元(或称抽象符号)、以$\mathbb{Z}_2$为系数域的有理式\footnote{即要么是$0$,要么是两个非零多项式的比,如$(x^4+3)/(x-2)$.}构成的集合,其加法和乘法运算都继承$\mathbb{Z}_2$的运算.

换言之,$\mathbb{F}$中的元素都形如$\frac{f(t)}{g(t)}$,其中$f(t), g(t)\in\mathbb{Z}_2[t]$且$\opn{deg}g\geq 0$.

显然,$t$也是$\mathbb{F}$中的一个元素,也就是我们说的系数、数字等.而$\opn{ch}\mathbb{F}=2$,因为它的素域是$\mathbb{Z}_2$.

考虑$\mathbb{F}[x]$中的多项式$p(x) = x^2-t$,它的分裂域要引入新元素$\sqrt{t}$.在分裂域中,注意到
\begin{equation}
\begin{aligned}
(x-\sqrt{t})^2&=x^2+t-2\sqrt{t}\\
&=x^2+t\\
&=x^2-t
\end{aligned}
\end{equation}

因此$p(x)$在其分裂域中有二重根$\sqrt{t}$.

$p(x)$显然是不可约的.这是因为如果可约,那么它应该有两个一次多项式因子,也就是说它的根都在$\mathbb{F}$中,而根据$\mathbb{F}$的定义,这显然是不成立的.

综上,我们得到的就是这样一个例子:$p(x)$是$\mathbb{F}[x]$上次数大于$1$的不可约多项式,且$\opn{D}p=0$,符合\autoref{SprbEx_cor2} 的论断,即它有重根.

\end{example}

一般来说,\autoref{SprbEx_ex1} 中的$\mathbb{Z}_2$替换为任意素域$\mathbb{Z}_p$、$x^2-t$也替换为$x^p-t$后,结论依然成立.感兴趣的读者可自行验证这一点.其中,证明$x^p-t$不可约,可以使用整环上的\textbf{爱森斯坦判别式}\upref{EsstCr}.

\begin{corollary}{}\label{SprbEx_cor3}

若域$\mathbb{F}$的特征为$0$,则其上任意次数大于$1$的不可约多项式在其分裂域(或大一些,$\mathbb{F}$的代数闭包)上没有重根.

\end{corollary}

证明只需要说其形式微商不可能为$0$即可,很简单,在此不赘述.




\subsection{可分多项式与可分扩张}



\subsubsection{可分元与可分扩张}



\begin{definition}{可分多项式}\label{SprbEx_def2}

设$\mathbb{F}$是域.若多项式$f(x)\in\mathbb{F}[x]$在其分裂域中无重根,则称$f$为$\mathbb{F}$上\textbf{可分(separable)}多项式\footnote{这里采用GTM 242 \textsl{Abstract Algebra}的定义.有些材料中(如孟道骥《抽象代数I-代数学基础》)会把无重根的\textbf{不可约}多项式定义为可分的,而一般的多项式则只要求其不可约因子是可分的,即定义为可分多项式.我们通常研究的是不可约多项式,在这一点上所有“可分”定义都是一致的,所以一般多项式的可分定义之分歧无伤大雅.}.


\end{definition}

简单来说,可分多项式就是在系数域的代数闭包里没有重根的多项式.由\autoref{SprbEx_cor3} 可知,只有在域特征为某个素数的情况下才有可能出现不可分多项式,比如\autoref{SprbEx_ex1} 那样的.


\begin{definition}{可分元}
如果域$\mathbb{F}$的代数元$a$在$\mathbb{F}$上的最小多项式是可分的,那么称这个元素是\textbf{可分}元.
\end{definition}

\begin{definition}{可分扩张}
如果域扩张$\mathbb{K}/\mathbb{F}$中,任意$a\in\mathbb{K}$都是$\mathbb{F}$的可分元,那么称该扩张是\textbf{可分}扩张.
\end{definition}




要验证一个扩张是否可分,逐一去验证它的所有元素是否可分,实在是过于麻烦.幸运的是,下面的\autoref{SprbEx_the6} 说明,只要扩张的生成元是可分的,那么这个扩张就可分;不幸的是,这个证明并不简短,并且现有的知识很难理解,因此为了防止词条臃肿,这里只叙述该命题,并且在证明之前都不会用它来证明任何东西.证明另见\textbf{可分元素的单扩张是可分扩张}\upref{SprbE2}词条.

\begin{theorem}{}\label{SprbEx_the6}
域$\mathbb{F}$上的不可约可分多项式$f(x)$的分裂域$\mathbb{K}$是$\mathbb{F}$的可分扩张.
\end{theorem}



不可分的不可约多项式,虽然有重根,但是却有一个极为良好的性质:


\begin{theorem}{}\label{SprbEx_the3}
设域$\mathbb{F}$的特征为素数$p$,$f(x)\in\mathbb{F}[x]$是其上的\textbf{不可分不可约}多项式,$\mathbb{K}$是$f(x)\in\mathbb{F}[x]$的分裂域.

则$f(x)$在$\mathbb{K}$中每个根的重数是相同的.
\end{theorem}

\textbf{证明}:

考虑$\mathbb{F}[x]$上的多项式
\begin{equation}
f(x) = \sum_{i=0}^n a_ix^i
\end{equation}


由于其不可分、不可约,故由\autoref{SprbEx_def2} 和\autoref{SprbEx_cor2} 知,$a_i\neq 0 \Rightarrow p\mid i$.也就是说我们可以把$f$中次数不是$p$整数倍的单项式挖掉.

换言之,存在$g_0(x)\in\mathbb{F}[x]$,使得$f(x)=g_0(x^p)$.

由于“$g_0(x)$可约” $\Rightarrow$ “$g_0(x^p)$可约” $\iff$ “$f(x)$可约”,可知“$f(x)$不可约” $\Rightarrow$ “$g_0(x)$不可约”. 总之,$g_0$必须是$\mathbb{F}$上的不可约多项式.

如果$g_0(x)$也不可分,那么也可以进行相同的挖去操作,得到$g_1(x)$,其中$g_0(x) = g_1(x^p)$,使得$f(x) = g_0(x^p) = g_1((x^p)^p) = g_1(x^{p^2})$.由于多项式中只有有限个单项式,因此这一方法可以在有限步内得到一个可分的不可约多项式.为了方便,就设$h(x)$是最终得到的\textbf{可分的}不可约多项式.

于是,存在$k\in\mathbb{Z}$,使得$f(x) = h(x^{p^k})$.

由于$h(x)$是可分的,故
\begin{equation}
h(x) = \prod_{j=1}^{r}(x-a_j)
\end{equation}
其中各$a_j\in\mathbb{K}$互不相同.

于是
\begin{equation}
f(x) = h(x^{p^k}) = \prod_{j=1}^{r}(x^{p^k}-a_j)
\end{equation}

现在设$b_j$是$x^{p^k}-a_j$的一个根,即$b_j^{p^k}=a_j$.那么由二项式展开以及$p=0$:
\begin{equation}
(x-b_j)^{p^k} = x^{p^k}-a_j
\end{equation}

因此
\begin{equation}
f(x) = \prod_{j=1}^{r} (x-b_j)^{p^k}
\end{equation}

也就是说,$f$有$r$个根$b_j$,每个的重数都是$p^k$.显然,$p^k\cdot r = \opn{deg}f = n$.

\textbf{证毕}.







\subsubsection{完美域}


\begin{definition}{完美域}\label{SprbEx_def3}

若域$\mathbb{F}[x]$上的每个不可约多项式都是可分多项式,则称$\mathbb{F}$是一个\textbf{完备域(perfect domain)},或\textbf{完全域},或\textbf{完美域}.


\end{definition}


或许“完美域”的术语更合适一些,以免和分析意义下的(柯西)完备域/完全域(complete field)相混淆.

\begin{theorem}{}\label{SprbEx_the5}
\autoref{SprbEx_def3} 一个等价表述为:所有代数扩张都是可分扩张,则为完美域.
\end{theorem}




\textbf{证明}:

按照可分元和可分扩张的定义,“所有代数扩张都是可分扩张”等价于“所有代数元的最小多项式都是可分多项式”.但是,任意一个不可约多项式都可以是一个代数元的最小多项式,因此该命题又等价于“所有不可约多项式都是可分多项式”.

\textbf{证毕}.









\subsubsection{特征不为$0$时的完美域}


由\autoref{SprbEx_cor3} 可知,特征为$0$的域都是完美域,因此我们接下来只讨论特征为素数$p$的情况.首先,我们给出两个相似的简单引理,对之后的定理证明有帮助:


\begin{lemma}{}\label{SprbEx_lem2}
设域$\mathbb{F}$的特征为$p$,则对于任意$a_i\in\mathbb{F}$和任意正整数$k$,只要$p\mid k$,就有

\begin{equation}
(\sum a_i)^k = \sum a_i^k
\end{equation}

\end{lemma}

由于证明思路是完全一样的,因此省略\autoref{SprbEx_lem2} 的证明,读者可参考下面\autoref{SprbEx_lem1} 的证明.


\begin{lemma}{}\label{SprbEx_lem1}
设域$\mathbb{F}$的特征为$p$,则对于任意多项式
\begin{equation}
f(x) = \sum_{i=0}^n a_ix^i
\end{equation}
和任意正整数$k$,只要$p\mid k$,就有
\begin{equation}
\qty(f(x))^k = \qty(\sum_{i=0}^n a_ix^i)^k = \sum_{i=0}^n (a_ix^i)^k
\end{equation}
\end{lemma}

\textbf{证明}:

由组合的知识,$(f(x))^k$展开式中(单纯展开,还未进行同类项合并)形如$(a_ix^i)^k$的项只出现一次,但其它项出现的次数都是$k$的倍数.由于$p\mid k$,故在$\mathbb{F}$上$k=0$,故只有形如$(a_ix^i)^k$的项被保留.

\textbf{证毕}.


这两个定理揭示了有限域上极为重要而简洁的性质.我们简单看看一个例子,利用起这个性质:




\begin{example}{}\label{SprbEx_ex2}
考虑$\mathbb{Z}_p$的单超越扩域$\mathbb{Z}_p(\pi)$.则$x^p-\pi$是$\mathbb{Z}_p(\pi)$上的不可约多项式\footnote{考虑到$\mathbb{Z}_p(\pi)$是整环,且$\pi$是其素元素,应用\textbf{爱森斯坦判别式}\upref{EsstCr}即可得证.},但$\opn{D}(x^p-\pi)=0$,从而是一个不可分的不可约多项式.因此,$\mathbb{Z}_p(\pi)$不是完美域.

\end{example}



在域$\mathbb{F}$上,任取$a\in\mathbb{F}$,如果存在$b\in\mathbb{F}$使得$a=b^p$,那么据\autoref{SprbEx_lem2} ,$x^p-a$一定是有重根的:$x^p-a=(x-b)^p$\footnote{和\autoref{SprbEx_ex2} 比起来,这里爱森斯坦判别式无效了,因为任何元素都不是素元素了.理由很简单,$a=b^p$,所以$a\mid b\cdot b^{p-1}$.于是,要么$a$是\textbf{单位}(\autoref{FctTre_def3}~\upref{FctTre}),这样$a\mid 1$;要么$a$不是素元素.总之$x^p-a$没法符合爱森斯坦判别式中素元素或素理想的要求.}.所以为了构造完美域,我们必须要求所有这类多项式都是可约的,这样才不违反“不存在不可约不可分多项式”的规定.

因此,如果对于任意$a\in\mathbb{F}$,总存在$b\in\mathbb{F}$使得$a=b^p$,那么至少能保证形如$x^p-a$的多项式不会违反完美域的要求.但这个条件是不是能保证所有不可约多项式都是无重根的呢?这是一个很有力的猜想,而下面的\autoref{SprbEx_the4} 说明它确实是成立的.





\begin{theorem}{特征不为零时完美域的等价定义}\label{SprbEx_the4}
设域$\mathbb{F}$的特征为$p$,则$\mathbb{F}$是完美域的充要条件为:$\forall a\in\mathbb{F}$,存在$b\in\mathbb{F}$使得$a=b^p$.

等价表述为:定义映射$\sigma:\mathbb{F}\to\mathbb{F}$为$\sigma(a)=a^p$,则$\mathbb{F}$是完美域的充要条件为$\sigma(\mathbb{F})=\mathbb{F}$.
\end{theorem}

\textbf{证明}:

充分性:

设$\forall a\in\mathbb{F}$,存在$b\in\mathbb{F}$使得$a=b^p$,但反设存在\textbf{不可分}的不可约多项式$f(x)\in\mathbb{F}[x]$.

由\autoref{SprbEx_cor2} ,存在\textbf{可分}的不可约多项式$h(x)\in\mathbb{F}[x]$和正整数$k$,使得$f(x)=h(x^{p^k})$.

设
\begin{equation}
h(x) = \sum_{i=0}^r a_ix^i
\end{equation}
则存在$b_i\in\mathbb{F}$使得$a_i=b_i^{p^k}$.

\begin{equation}\label{SprbEx_eq1}
\begin{aligned}
f(x)&=h(x^{p^k}) \\
&=\sum_{i=0}^r a_i(x^{p^k})^i\\
&=\sum_{i=0}^r b_i^{p^k}(x^{i})^{p^k}\\
&=\sum_{i=0}^r \qty(b_ix^i)^{p^k}\\
&=\qty(\sum_{i=0}^r b_ix^i)^{p^k}
\end{aligned}
\end{equation}

\autoref{SprbEx_eq1} 的最后一个等号,应用了\autoref{SprbEx_lem1} .这么一来,$f(x)$就可约了,和“不可约”的假设矛盾.故反设不成立,故不存在不可分的不可约多项式,也就是说$\mathbb{F}$是完美域.

必要性:

设存在$a\in\mathbb{F}$,使得对于任意$b\in\mathbb{F}$,都有$b^p\neq a$.

考虑多项式$f(x) = x^p-a$.设$a_0$是$f(x)$在其分裂域上的一个根,则由于$p$是一个素数,知$\{a_0^i\mid i\in\mathbb{Z}\}$配合乘法构成群$\mathbb{Z}_p$.因此易得:各$a_0^i$都是$f(x)$的根,也是其全部的根.

于是$f(x)=(x-a_0)^p$.如果它有非平凡因子,那么因子必形如$(x-a_0)^k$,其中$k\in(0, p)$,而$a_0^k\not\in\mathbb{F}$,所以这个因子不可能在$\mathbb{F}[x]$中.因此,$f$在$\mathbb{F}[x]$上是不可约的.

显然,$f$有重根\footnote{你也可以计算$\opn{D}f$看看是否符合\autoref{SprbEx_cor2} .},因此不可分.这个不可分不可约多项式的存在意味着$\mathbb{F}$不完全,从而得证必要性.

\textbf{证毕}.




特征为$p$的域上,域自同构$\sigma(a)=a^p$称为\textbf{Frobenius同态}.


\autoref{SprbEx_the1} 真是一个深刻的结论,使得我们可以仅从域中元素的性质判断其多项式的性质.









\begin{corollary}{}\label{SprbEx_cor4}
有限域都是完美域.
\end{corollary}

\textbf{证明}:

设有限域$\mathbb{F}$的特征为$p$.我们只需要证明$\mathbb{F}^p=\{a^p\mid a\in\mathbb{F}\}=\mathbb{F}$即可.

考虑映射$\sigma:\mathbb{F}\to\mathbb{F}$,定义为$\sigma(a)=a^p$.

由于$a^p-b^p=(a-b)^p$\footnote{当$p$为奇数的时候,$-b^p=(-b)^p$;当$p=2$的时候,$-b=b$,所以仍然有$-b^p=(-b)^p$.},可知$\sigma(a)=\sigma(b)\iff a=b$.因此,这是一个\textbf{双射}.

因此$\mathbb{F}^p=\{a^p\mid a\in\mathbb{F}\}=\mathbb{F}$.

\textbf{证毕}.

另外,由于$\sigma(ab)=\sigma(a)\sigma(b)$和$\sigma(a+b)=(a+b)^p=a^p+b^p=\sigma(a)+\sigma(b)$,因此\autoref{SprbEx_cor4} 证明中的$\sigma$还是一个\textbf{域同构}.


\begin{corollary}{}\label{SprbEx_cor5}
设$\mathbb{F}$是完美域,$\mathbb{K}$是其代数扩张,则$\mathbb{K}$也是完美域.
\end{corollary}


\textbf{证明}:

% 由于扩域不改变域的特征以及特征为$0$的域都是完美域,我们只需要讨论特征为素数$p$的情况.又因为我们只需要证明,如果$x_0$是$\mathbb{F}$的代数元,那必存在$y_0\in\mathbb{F}(x_0)$使得$y_0^p=x_0$,故只需要考虑单代数扩张的情况.

% 设$x_0$在$\mathbb{F}$上的最小多项式为$f(x)=\sum_{i=0}^n a_ix^i$.

% 由于$\mathbb{F}$是完美域,所以可以取$b_i\in\mathbb{F}$使得$b_i^p=a_i$.定义$h(x) = \sum_{i=0}^n b_ix^i$.由于$\opn{ch}\mathbb{F}=p$,据\autoref{SprbEx_lem1} 知
% \begin{equation}
% \begin{aligned}
% \qty(h(x))^p&=(\sum_{i=0}^n b_ix^i)^p\\
% &=\sum_{i=0}^n b_i^p x^{ip}\\
% &=\sum_{i=0}^n a_i x^{ip}\\
% &=f(x^p)
% \end{aligned}
% \end{equation}

% \textbf{如果}$g(x)=\sum_{i=0}^nb_ix^i$在$\mathbb{F}$上是可约的,即存在$c(x)=\sum_{i=0}^rc_ix^i$和$d(x)=\sum_{i=0}^{n-r}d_ix^i$使得$h(x)=c(x)d(x)$,那么据\autoref{SprbEx_lem2} ,知
% \begin{equation}
% f(x) = (\sum_{i=0}^rc_i^px^i)(\sum_{i=0}^rd_i^px^i)
% \end{equation}
% 即$f$也可约.但由于$f$是最小多项式,故矛盾,故$g$在$\mathbb{F}$上\textbf{不可约}.







% 考虑映射$\sigma:\mathbb{F}\to\mathbb{F}$,定义为$\sigma(a)=a^p$.由\autoref{SprbEx_the4} ,
% \begin{equation}
% \sigma(\mathbb{F})=\mathbb{F}
% \end{equation}
% 易验证,$\sigma$是域同构.

% 设$x_0$是$\mathbb{F}$的一个代数元,其最小多项式为$f(x)$.

% 将$\sigma$开拓为$\mathbb{F}[x_0]\to\mathbb{F}[x_0^p]$上的环同构,定义为$\sigma(\sum_{i=0}^n a_ix_0^i)=\sum_{i=0}^n a_i^p(x_0^p)^{i}$.


由\autoref{FldExp_cor2}~\upref{FldExp},$\mathbb{K}$的代数扩张都是$\mathbb{F}$的代数扩张,从而是$\mathbb{F}$的可分扩张,从而是$\mathbb{K}$的可分扩张.由可分扩张的等价定义\autoref{SprbEx_the5} 即得证.


\textbf{证毕}.


\begin{example}{}\label{SprbEx_ex3}
考虑$\mathbb{Z}_5$的单代数扩张$\mathbb{Z}_5(\sqrt{2})=\{0, 1, 2, 3, 4, \sqrt{2}, \sqrt{3}\}$.

则这是一个完美域,其中对于任意$a\in\mathbb{Z}_5(\sqrt{2})$,都有$a^5=a$.
\end{example}


最后要提一点:一个域的全体可分元素构成的集合,仍然是一个域,称为其\textbf{可分闭包},见\autoref{SprbE2_cor4}~\upref{SprbE2}后的讨论.







































