% 逻辑门、布尔运算
% license Usr
% type Tutor

\begin{issues}
\issueDraft
\end{issues}

\footnote{参考 Wikipedia \href{https://en.wikipedia.org/wiki/Boolean_algebra}{相关页面}。}大家有没有想过,电脑或手机的 CPU 是如何进行运算的呢?我们知道 CPU 使用的主要是\textbf{数字电路(digital circuit)},当前几乎所有数字电路中所有的数据都是通过 $0$ 和 $1$ 来表示的,例如当一根导线处于高电平就代表 $1$ 否则代表 $0$。 这就是为什么我们常说计算机是二进制的,所有数据信息都需要转换为二进制编码才可以被计算机处理。 数字电路区别于模拟电路,例如耳机线或无线广播的电磁波就通常是模拟信号,根据信号的振幅和频率等大小表示(模拟)声波震动的大小,它的值可以是连续变化的。

举一个数字电路的简单例子:如果 8 根导线表示一个 8 位\textbf{二进制整数},而另外 8 根表示另一个数,那么如何通设计一个数字电路可以在通电的一瞬间立即用额外的 8 根导线输出它们相加、减、乘、除的结果呢? 答案就是使用\textbf{逻辑门}。 除此之外,逻辑门还可以实现数据的储存等功能,电脑中的固态硬盘通常就是大量使用下文的 NAND 门实现的。

\begin{itemize}
\item 数学上来说,一个\textbf{布尔变量(boolean variable)}的值只能是 $0$ 或 $1$,通常也叫做\textbf{假(false)}和\textbf{真(true)}。例如下面的 $A,B,C$ 就是布尔变量。
\item \textbf{与、或、非(AND, OR, NOT)}是三种最常见的逻辑门。
\item 与门:两个输入都是 $1$ 时输出 $1$,否则输出 $0$。
\item 或门:两个输入都是 $0$ 时输出 $0$,否则输出 $1$。
\item 非门:只有一个输入,输入 $0$ 时输出 $1$, 否则输出 $0$。
\item 与、或、非运算可以用符号表示为 $A + B$, $A\cdot B$, $\overline A$
\item 另一种常用的符号是 $A\lor B$(或), $A\land B$(与), $\neg A$(非)。 这类似于\enref{并集、交集、补集}{HsSet}的符号。
\item CPU 里面几乎所有功能都是逻辑门实现的。 例如做整数的加减乘除运算的电路。
\item (私货)强烈推荐游戏:\href{https://turingcomplete.game/}{图灵完备}——从逻辑门开始搭 CPU,实现汇编码编程。
\item 类比数学公式,如果有一个比较复杂的逻辑运算表达式,如何把它变形和化简,而不影响计算结果呢?这就需要以下定理的帮助。 这就好比简化一条若干未知数的四则运算的表达式,需要把四则运算的规则熟练使用方可做到。
\item 我们先直接列出这些公式,下文再证明。 事实上最简单的证明方法就是对 $A,B,C$ 每一种取值的组合都证明等式两边相等即可。 但下文的证明会使用更巧妙的办法。
\item \textbf{结合律(associative laws)} $A\cdot(B\cdot C) = (A\cdot B)\cdot C$; $A+(B+C)=(A+B)+C$
\item \textbf{交换律(commutative laws)} $A+B=B+A$;$A\cdot B=B\cdot A$
\item \textbf{有界律(identity laws)} $A+0=A$;$A\cdot 1=A$;$A\cdot 0 = 0$;$A+1=1$(这就是为什么要用加号和乘号,很多性质都很相似)
\item \textbf{幂等律(idempotent laws)} $A+A=A$;$A\cdot A=A$
\item \textbf{互补律(complement laws)} $A+\overline A = 1$; $A\cdot \overline A = 0$
\item \textbf{分配律(distributive laws)} $A\cdot(B+C)=A\cdot B+A\cdot C$; $A+(B\cdot C) = (A+B)\cdot(A+C)$
\item \textbf{吸收律(absorption laws)} $A\cdot(A+B) = A$; $A+A\cdot B = A$ (分配律的特殊情况)
\item \textbf{德摩根律(De Morgan's laws)}又称\textbf{对偶律} $\overline{A+B} = \overline A \cdot \overline B$; $\overline{A \cdot B} = \overline A + \overline B$
\item 德摩根律可拓展到多个变量如 $\overline{A+B+C} = \overline A \cdot \overline B \cdot \overline C$; $\overline{A \cdot B\cdot C} = \overline A + \overline B + \overline C$
\item \textbf{两边取 NOT} 等式依然成立。
\item 德摩根定律两边取 NOT,就有 $A+B = \overline{\overline A \cdot \overline B}$; $A \cdot B = \overline{\overline A + \overline B}$。 这说明\textbf{与或非中,与和或有一个是多余的}。
\item \textbf{与或非中,非门是必须的}(无法用 AND 和 OR 构成)。
\item \textbf{共识定理(consensus theorem)} $A + \overline A \cdot B = A + B$
\item \textbf{与非门 (NAND,NOT AND)} $\overline{A\cdot B}$ 可以构成任意门(例如两个输入相连就是 NOT 门)
\item \textbf{或非门(NOR,NOT OR)} $\overline{A+B}=\overline A\cdot\overline B$ 检查是否输入都为零,也可以构成任意门(例如两个输入相连就是 NOT 门)。
\item \textbf{异或门(XOR)} 当两个输入不同时输出 $1$,否则输出 $0$。 $A\cdot\overline B+\overline A\cdot B$。
\item \textbf{同或门(XNOR)} 当两个输入相同时输出 $1$,即异或门取非。
\item \textbf{从真值表构建逻辑电路}:若干个输入变量经过一系列逻辑运算得到一个输出变量。 如果已知每种输入组合对应的输出结果,那么可以用 NOT 把每种输入的组合都整理为全 1 并做 AND, 再把所有 AND 的结果做 OR 即可(例如上面的 XOR 表达式就是这么写出来的)。但复杂情况下,该方法的初始表达式可能较为复杂,化简也麻烦,所以设计电路还是需要一些技巧和试错。
\item 例如要检查 $A,B,C$ 中是否只有一个 1,就写出 $A\cdot\overline B\cdot\overline C + \overline A\cdot B\cdot\overline C + \overline A\cdot\overline B\cdot C$ 再根据喜好进行某种变形或化简。
\item NOR,XOR,AND 可以分别检测两个输入之和是否为 $0$,$1$,$2$,可以用于加法电路或计数电路。
\end{itemize}

\begin{exercise}{}
\begin{itemize}
\item 如何创建 2 或 3 个变量的计数电路(也就是加法电路)?(结果分别是 4 或 5 bit)
\item 如何创建两个 $N$ 位二进制数的加法电路?(每个数的每 bit 是一个变量)
\item 如何创建 $2^N$ 个变量的计数电路?(结果表示为二进制数,每个 bit 一个变量)
\end{itemize}
\end{exercise}
