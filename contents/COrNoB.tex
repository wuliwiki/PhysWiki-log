% 连续正交归一基底与傅里叶变换
% keys 正交归一|delta 函数|内积
% license Xiao
% type Tutor

\pentry{傅里叶变换(指数)\nref{nod_FTExp},狄拉克 delta 函数\nref{nod_Delta}, 内积\nref{nod_InerPd}}{nod_3bf9}

\subsection{离散的函数基底}
本文使用\enref{狄拉克符号}{braket}。在 “\enref{傅里叶级数(三角)}{FSTri}” 中,我们介绍了正交归一函数基底的概念, 即把满足一定条件的一元函数的集合看作一个\enref{矢量空间}{LSpace},两个函数(矢量)的\enref{内积}{InerPd}定义为
\begin{equation}
\braket{f}{g} = \int_{-\infty}^{+\infty} f(x)^* g(x) \dd{x}~.
\end{equation}
其中 $*$ 表示复共轭, 如果空间中的函数都是实函数则可忽略。

该空间中的一组正交归一基底用\enref{狄拉克符号}{braket}表示为 $\ket{x_i}$ ($i = 1, 2,\dots$), 基底的个数可以是有限个或无限个, 空间的维数就是基底的个数。

基底满足正交归一条件(\autoref{eq_OrNrB_3})
\begin{equation}\label{eq_COrNoB_2}
\braket{x_i}{x_j} = \delta_{i,j}~.
\end{equation}
若这组正交归一基底是完备的, 那么如果一个函数可以分解为它们的线性组合:
\begin{equation}\label{eq_COrNoB_6}
\ket{f} = \sum_j c_j\ket{x_j}~.
\end{equation}
两边左乘 $\bra{x_i}$, 则有
\begin{equation}\label{eq_COrNoB_10}
\braket{x_i}{f} = \sum_j c_j\braket{x_i}{x_j} = \sum_j c_j \delta_{i,j} = c_i~.
\end{equation}
即
\begin{equation}\label{eq_COrNoB_5}
c_i = \braket{x_i}{f}~,
\end{equation}
\autoref{eq_COrNoB_10} 的过程相当于用正交归一性把 $\ket{x_i}$ 项从\autoref{eq_COrNoB_6} 的求和中筛选了出来。 我们得到几何矢量中一个熟悉的结论: 一个矢量关于一组正交归一基底的坐标等于它在每个基底上的投影。

\subsection{连续的函数基底}

我们接下来用类似的方法来理解傅里叶变换(\autoref{eq_FTExp_6})
\begin{equation}\label{eq_COrNoB_4}
g(k) = \frac{1}{\sqrt{2\pi }} \int_{-\infty }^{+\infty } f(x)\E^{-\I kx} \dd{x}~,
\end{equation}
\begin{equation}\label{eq_COrNoB_3}
f(x) = \frac{1}{\sqrt{2\pi }} \int_{-\infty }^{+\infty } g(k)\E^{\I kx} \dd{k}~.
\end{equation}
我们令所有可以做傅里叶变换的函数构成的空间为 $X$, 从傅里叶变换的公式, 我们猜想该空间的正交归一 “基底” 为
\begin{equation}\label{eq_COrNoB_1}
\ket{k} = \frac{1}{\sqrt{2\pi}} \E^{\I kx} \qquad (k \in \mathbb R)~.
\end{equation}
严格来说, $X$ 空间的函数必须要满足 $\braket{x}{x}$ 为有限值, 而\autoref{eq_COrNoB_1} 中的函数显然不满足这点, 所以它们并不属于 $X$ 空间, 而是一个包含 $X$ 的更大的空间, 所以这个 “基底” 只是一个形象的说法, 需要加上引号。

显然, \autoref{eq_COrNoB_1} 中的任意两个 “基底” 的内积都不收敛, 而且 $k$ 的取值是\textbf{连续}的, 所以我们不可能用\autoref{eq_COrNoB_2} 表示它们的正交归一关系。 但通过狄拉克 delta 函数的\autoref{eq_Delta_8} 
\begin{equation}
\int_{-\infty}^{+\infty} \E^{\I kx}\dd{x} = 2\pi \delta(k)~.
\end{equation}
可以得到一个和\autoref{eq_COrNoB_2} 类似的关系
\begin{equation}\label{eq_COrNoB_11}
\begin{aligned}
\braket{k'}{k} &= \int_{-\infty}^{+\infty} \frac{\E^{-\I k'x}}{\sqrt{2\pi}} \frac{\E^{\I kx}}{\sqrt{2\pi}}\dd{x}\\
&= \frac{1}{2\pi}\int_{-\infty}^{+\infty} \E^{\I (k'-k)x}\dd{x}
= \delta(k' - k)~,
\end{aligned}
\end{equation}
即
\begin{equation}\label{eq_COrNoB_8}
\braket{k'}{k} = \delta(k' - k)~.
\end{equation}
这可以看作是\textbf{连续基底的正交归一条件}。

现在, 把\autoref{eq_COrNoB_4}  和\autoref{eq_COrNoB_3} 用狄拉克符号表示为
\begin{equation}\label{eq_COrNoB_9}
g(k) = \braket{k}{f}~,
\end{equation}
\begin{equation}\label{eq_COrNoB_7}
f(x) = \int_{-\infty}^{+\infty} g(k') \ket{k'} \dd{k'}~.
\end{equation}
它们可以分别看作是把\autoref{eq_COrNoB_5} 和 \autoref{eq_COrNoB_6} 拓展到连续基底的情况。 根据定义, 任何能做傅里叶(反)变换的 $f(x)$ 必定能展开成\autoref{eq_COrNoB_7} 的形式。 再来证明\autoref{eq_COrNoB_9}, 过程和\autoref{eq_COrNoB_10} 类似: \autoref{eq_COrNoB_7} 两边左乘 $\bra{k}$, 使用 $\delta$ 函数的性质\autoref{eq_Delta_7}  把积分中 $\ket{k}$ 基底的系数 “筛选” 出来
\begin{equation}
\begin{aligned}
\braket{k}{f} &= \int_{-\infty}^{+\infty} g(k) \braket{k}{k'} \dd{k}\\
&= \int_{-\infty}^{+\infty} g(k) \delta(k-k') \dd{k'}
= g(k)~,
\end{aligned}
\end{equation}
证毕。

注意该证明并不仅限于傅里叶变换一种情况, 任何连续的基底 $\ket{k}$ 只要满足正交归一条件\autoref{eq_COrNoB_8}, 且可以展开某函数 $f(x)$, 就都能使\autoref{eq_COrNoB_9} 成立。
