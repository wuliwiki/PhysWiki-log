% 高尔顿板
% license Xiao
% type Tutor

\begin{issues}
\issueDraft
\end{issues}

\pentry{二项分布\nref{nod_BiDist}, 中心极限定理\nref{nod_CLT}}{nod_a8a6}

\begin{figure}[ht]
\centering
\includegraphics[width=5cm]{./figures/13261dc68acc42d5.png}
\caption{高尔顿板, 图片来自 Wikipedia} \label{fig_Galton_1}
\end{figure}

共有 $N+1$ 个槽, 小球落在第 $k$ ($k = 0,1,\dots,N$)个槽中的概率符合二项分布 $f(k,n,p)$
\begin{equation}
P(k) = C_N^k p^k (1-p)^{(n-k)}~,
\end{equation}
其中 $p$ 是每一层中小球向右滚动的概率。 一般来说 $p = 1/2$。

二项分布的平均值为 $Np$, 方差为 $Np(1-p)$。 根据\enref{中心极限定理}{CLT},当 $N\to\infty$ 时, 分布曲线是一个高斯分布。
