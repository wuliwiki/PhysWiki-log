% 电磁力和引力
% keys 电磁力|引力|作用量
% license Usr
% type Tutor

\pentry{光与物质粒子的统一(相对论点粒子的作用量)\nref{nod_RAct},相对论补全\nref{nod_Comple}}{nod_e590}
\cite{AZee}本节将以一种“统一”的角度给出电磁力和引力的作用量。约定:重复指标代表求和,$\mu,\nu,\rho,\ldots$ 指标取值遍及所有分量 $0,1,2,3$,$i,j,k,\ldots$ 范围除去时间分量 $0$。

\subsection{从自由粒子到势阱中的粒子}
在光与物质粒子的统一(相对论点粒子的作用量)\upref{RAct}一节,我们得到了自由粒子的作用量,其具有下面的形式
\begin{equation}\label{eq_EleGra_1}
S=-m\int\sqrt{-\eta_{\mu\nu}\dd x^\mu\dd x^\nu}=-m\int\sqrt{\dd t^2-\dd{\bvec x}^2}~.
\end{equation}
现在考虑粒子处于势 $V(x)$ 中。尽管非相对论情形(Newton力学)时处于势为 $V(x)$ 的粒子作用量为
\begin{equation}
S_{NR}=\int\dd t\qty(\frac{1}{2}m\qty(\dv{\bvec x}{t})^2-V(x)),~
\end{equation}
但是我们并不能理所当然的将 $V(x)$ 加入\autoref{eq_EleGra_1} 中得到相对论情形处于势阱 $V(x)$ 中的粒子。换言之,我们不知道如何将 $V(x)$ 放入\autoref{eq_EleGra_1} 中。

尽管如此,然而可以肯定的是,将 $V(x)$ 放入\autoref{eq_EleGra_1} 中,只有两种可能:a.根号外面,b.根号里面。因此,我们有如下的两种选择:
\begin{enumerate}
\item E:
\begin{equation}
S=-\int{m\sqrt{-\eta_{\mu\nu}\dd x^\mu\dd x^\nu}+V(x)\dd t}.~
\end{equation}
\item G:
\begin{equation}\label{eq_EleGra_4}
S=-m\int\sqrt{\qty(1+\frac{2V}{m})\dd t^2-\dd{\bvec x}^2}~.
\end{equation}
\end{enumerate}
选项G来源于非相对论极限:首先,$\abs{\dd{\bvec x}}\ll \dd t$,所以
\begin{equation}\label{eq_EleGra_2}
S\approx-m\int \qty{\sqrt{\qty(1+\frac{2V}{m})}\dd t-\frac{\dd{\bvec{x}^2}}{2\sqrt{1+\frac{2V}{m}}\dd t}}~.
\end{equation}
其次,令 $V\ll m$(即 $V\ll mc^2$),则
\begin{equation}
\sqrt{1+\frac{2V}{m}}\approx1+\frac{V}{m}.~
\end{equation}
由于在\autoref{eq_EleGra_2} 中,第二项已经远小于第一项了,因此第二项不需要再保留到 $\frac{V}{m}$ 的修正项,因此
\begin{equation}\label{eq_EleGra_11}
\begin{aligned}
S\approx&-m\int \qty{(1+\frac{V}{m})\dd t-\frac{\dd{\bvec x}^2}{2\dd t}}\\
=&\int \dd t\qty{\frac{1}{2}m\qty(\dv{\bvec x}{t})^2-V-m}~.
\end{aligned}
\end{equation}
上式表明选项 $G$ 的作用量在适宜的极限下取Newton作用量的形式,除了多出一个常数 $-m$,而这一项我们已经知道代表着什么(见\autoref{sub_RAct_1})。

然而,无论是选项 E 还是 G,添加在作用量中的项都不是Lorentz不变的。因此,为了保持理论的Lorentz不变性(见相对论补全\upref{Comple}开头的说明),我们必须要进行修正。

\subsection{相对论补全}
无论是 E 还是 G ,若我们认为 $V$ 是通过外界固定和施加的,那么作用量无论如何都不能是Lorentz不变的。因此,为了保持Lorentz不变性,必须改变 $V(x)$ 的形式。改变的关键便是相对论补全\upref{Comple}。
\subsubsection{E的改进}

首先看 E,注意 $V(x)$ 是和 $\dd t$ 结合的,因此可将 $V(x)$ 视为Lorentz矢量场 $A_\mu(x)$ 的时间分量 $A_0(x)$,而 $V(x)\dd t$ 仅是 $A_\mu(x)\dd x^\mu=A_0(x)\dd t+A_i(x)\dd x^i$ 的第一项。因此,我们只需引入一个矢量场 $A_\mu(x)$,从而得到如下作用量
\begin{equation}\label{eq_EleGra_3}
S=\int\qty{-m\sqrt{-\eta_{\mu\nu}\dd x^\mu\dd x^\nu}+A_\mu(x)\dd x^\mu}~.
\end{equation}
当我们对 $x^\mu$ 进行Lorentz变换时,也必须对 $A_\mu$ 进行。两个Lorentz矢量的缩并,显然是一个Lorentz标量,因此,\autoref{eq_EleGra_3} 的作用量是Lorentz不变的。

\subsubsection{G的改进}
对 G 改进的关键是,平等的对待 $\dd t$ 和 $\dd{\bvec x}$。若 $\dd t^2$ 被某些函数乘,那么 $\dd{\bvec x}$ 也应被某些函数乘。记 $\qty(1+\frac{2V}{m})$ 为 $g$,因此得到形如 $g(x)\dd t^2-\tilde g(x)\dd{\bvec x}^2$ 的式子。而在进行Lorentz变换时,$\dd t^2$ 将变换为 $\dd t^2,\dd x^i\dd x^j$ 和 $\dd t\dd x^i$ 的线性组合。这似乎预示着根号下必须出现 $\dd t\dd x^i$ 的项。

注意到 $\dd t^2-\dd{\bvec x}^2$ 是通过 $\dd x^\mu\dd x^\nu$ 和Minkowski度规 $\eta_{\mu\nu}$ 缩并出现的,因此我们应当将 $\qty(1+\frac{2V}{m})$ 补全为一个Lorentz张量 $g_{\mu\nu}(x)$ 的分量。换句话说,我们应该将 $\eta_{\mu\nu}\dd x^\mu\dd x^\nu$ 推广为随时空变化的矩阵场 $g_{\mu\nu}\dd x^\mu\dd x^\nu$。因此得到G的下面的改进:
\begin{equation}\label{eq_EleGra_12}
S=-m\int\sqrt{-g_{\mu\nu}\dd x^\mu\dd x^\nu}.~
\end{equation}
而\autoref{eq_EleGra_4} 仅仅是上式的特殊情形。

\subsection{电磁学的出现}

现在看看\autoref{eq_EleGra_3} 对应的运动方程是什么样的。首先利用本征时间(\autoref{def_GeoSpa_1})将其参数化:
\begin{equation}\label{eq_EleGra_5}
S=-m\int\dd \tau\sqrt{-\eta_{\mu\nu}\dv{x^\mu}{\tau}\dv{x^\nu}{\tau}}+\int\dd \tau A_\mu(x(\tau))\dv{x^\mu}{\tau}~.
\end{equation}
注意第二项是在粒子的时空位置 $x^\mu(\tau)$ 处计算的。换句话说,场 $A_\mu(x)$ 遍及时空,但粒子仅仅在特定的位置取样。

对\autoref{eq_EleGra_5} 变分,得到
\begin{equation}\label{eq_EleGra_6}
\begin{aligned}
&\delta(-m\int\dd \tau\sqrt{-\eta_{\mu\nu}\dv{x^\mu}{\tau}\dv{x^\nu}{\tau}})=m\int\dd \tau\eta_{\mu\nu}\dv{x^\mu}{\tau}\dv{\delta x^\nu}{\tau}\\
&=-m\int\dd \tau\eta_{\mu\nu}\dv[2]{x^\mu}{\tau}\delta x^\nu,\\
&\delta\int\dd \tau A_\mu(x)\dv{x^\mu}{\tau}=\int\dd \tau \qty{A_\mu(x(\tau))\dv{\delta x^\mu}{\tau}+[\partial_\nu A_\mu(x(\tau))\delta x^\nu]\dv{ x^\mu}{\tau}}\\
&=\int\dd \tau \qty{-\partial_\nu A_\mu(x)+\partial_\mu A_\nu(x)}\dv{x^\nu}{\tau}\delta x^\mu\\
\end{aligned}~.
\end{equation}

因此,反对称的张量场
\begin{equation}\label{eq_EleGra_7}
F_{\mu\nu}\equiv\partial_\mu A_\nu(x)-\partial_\nu A_\mu(x).~
\end{equation}
出现了。联立\autoref{eq_EleGra_6} 和\autoref{eq_EleGra_7} 得
\begin{equation}
\delta S=\int\dd\tau \qty(-m\eta_{\mu\nu}\dv[2]{x^\mu}{\tau}+F_{\nu\mu}\dv{x^\mu}{\tau})\delta x^\nu~.
\end{equation}
从而运动方程为
\begin{equation}
m\dv[2]{x^\rho}{\tau}={F^\rho}_{\mu}\dv{x^\mu}{\tau}~.
\end{equation}

利用四动量的定义\upref{Comple} $p^\mu:=\dv{x^\mu}{\tau}$,上式写为
\begin{equation}
\dv{p^\rho}{\tau}={F^\rho}_{\mu}\dv{x^\mu}{\tau}~.
\end{equation}
定义
\begin{equation}
E^i:=F^{0i},\quad B_i:=\frac{1}{2}\epsilon_{ijk}F^{jk}.~
\end{equation}
则 $\epsilon^{imn}B_i=\frac{1}{2}(\delta^m_j\delta^{n}_k-\delta^m_k\delta^{n}_j)F^{jk}=F^{mn}$。于是
 \begin{equation}
 \begin{aligned}
  \dv{p^i}{\tau}=&{F^i}_{0}\dv{x^0}{\tau}+F^{ij}\dv{x^j}{\tau}\\
  =&{E^i}\dv{x^0}{\tau}+\epsilon^{kij}B_k\dv{x^j}{\tau}\\
  \dv{p^0}{\tau}=&F^{0}_i\dv{x^i}{\tau}.
 \end{aligned}~
 \end{equation}

 注意 $p^\mu=(E,\bvec p)$,并改用参数 $t$ ,则上式写为
 \begin{equation}
 \dv{\bvec p}{t}=\bvec E+\bvec v\times \bvec B,\quad \dv{E}{\tau}=\bvec E\cdot \bvec v,~
 \end{equation}
 因此,我们得到了电磁学中粒子在磁场 $\bvec B$ 中运动的Lorentz力定律,和粒子在电场 $\bvec E$ 中如何获得能量的定律。

\subsubsection{电荷的概念}
注意作用量中的 $x^\mu$ 代表的是粒子的时空坐标,为了不和时空本身混肴,我们最好用 $X^\mu$ 标记。因此,当有多个粒子时,一般的作用量则写为
\begin{equation}\label{eq_EleGra_8}
S=-\sum_a m_a\int\dd \tau\sqrt{-\eta_{\mu\nu}\dv{X_a^\mu}{\tau_a}\dv{X_a^\nu}{\tau_a}}+\sum_a e_a\int\dd \tau A_\mu(X_a(\tau_a))\dv{X_a^\mu}{\tau_a}~.
\end{equation}
这里,每一粒子通过一个不同的强度 $e_a$ 与场 $A_\mu$ “耦合”,而当只有一个粒子时,$e$ 可以归到 $A_\mu$ 里面。我们称 $e_a$ 为\textbf{电荷}(charge)。对应的运动方程则为
\begin{equation}
\dv{p_a^\rho}{\tau_a}=e_a{F^\rho}_{\mu}(X_a(\tau_a))\dv{X_a^\mu}{\tau_a}~.
\end{equation}
\autoref{eq_EleGra_8} 无参数化的形式为
\begin{equation}
S=\int\sum_a \qty{-m_a\sqrt{-\eta_{\mu\nu}\dd X_a^\mu\dd X_a^\nu}+e_a A_\mu(X_a)\dd X_a^\mu}~.
\end{equation}

\subsection{引力的出现}
对于选项 G,我们获得了
\begin{equation}\label{eq_EleGra_10}
S=-m\int\sqrt{-g_{\mu\nu}\dd x^\mu\dd x^\nu}.~
\end{equation}
其中,$g_{00}=-\qty(1+\frac{2V}{m}),g_{0i}=g_{i0}=0,g_{ij}=\delta_{ij}$ 是其特例。

让我们处理这一特例
\begin{equation}\label{eq_EleGra_9}
S=-m\int\sqrt{\qty(1+\frac{2V}{m})\dd t^2-\dd {\bvec x}^2}.~
\end{equation}
考虑Newton引力下,质量为 $m$ 的粒子在质量为 $M$ 的物体下的受到的势 $V=-\frac{GMm}{r}$。我们可以去掉 $2V/m$ 中的 $m$(这当然需要代表引力质量的 $-GMm/r$ 中的 $m$ 和代表惯性质量的 $ma$ 中的 $m$ 是一样的),得到
\begin{equation}
S=-m\int\sqrt{\qty(1-\frac{2GM}{r})\dd t^2-\dd {\bvec x}^2}.~
\end{equation}

假设粒子静止在势中,即 $\dd {\bvec x}=0$,那么
\begin{equation}
S=-m\int\sqrt{\qty(1-\frac{2GM}{r})\dd t^2}\approx-m\int\dd t\qty(1-\frac{GM}{r}).~
\end{equation}
由于是静止粒子,因此成立 $\dd\tau=\qty(1-\frac{GM}{r})\dd t$,或 $\dd t=\dd \tau/\qty(1-GM/r)>\dd\tau$。这就是说,处于引力场中的粒子的时间走慢了。

引力影响了时间的流动!

\subsubsection{引力和弯曲时空}
考虑一堆不同质量的粒子,则代替 \autoref{eq_EleGra_9} 的是
\begin{equation}
S=-\sum_a m_a\int\sqrt{\qty(1+\frac{2V(x_a)}{m_a})\dd t_a^2-\dd {\bvec x_a}^2}.~
\end{equation}
若 $V(x_a)$ 不正比于 $m_a$,那么粒子经历的时间 $\dd \tau_a=\sqrt{\qty(1+\frac{2V(x_a)}{m_a})\dd t_a^2-\dd {\bvec x_a}^2}$ 将依赖于粒子的质量。即不同质量的粒子经历不同的时间流逝,除非 $V(x_a)$ 正比于 $m_a$。

现在,对一般的情形,即\autoref{eq_EleGra_10} 在多粒子情形将写为
\begin{equation}
S=-\sum_a m_a\int\sqrt{-g_{\mu\nu}(x_a)\dd x_a^\mu\dd x_a^\nu}.~
\end{equation}
其中,$g_{\mu\nu}(x_a)$ 依赖于粒子 $a$ 的性质,例如质量。注意上式就像是弯曲空间中的不同曲线长度的表达式,因此,粒子在引力场中,就等价于在弯曲时空一样。

注意,在数学上,平坦定义为曲率张量处处为0的,这意味着度规 $g_{\mu\nu}$ 处处一样。因此,若度规依赖于时空点 $x$,则时空便是非平坦的,即弯曲时空。

从上面可看出,Newton 引力是弯曲时空的一种特例(见\autoref{eq_EleGra_11} 和\autoref{eq_EleGra_12} ),因为 $g$ 依赖于时空点 $x_a$。因此,引力可视为弯曲时空的表现。











