% 列向量
% license Xiao
% type Wiki

\begin{issues}
\issueDraft
\issueMissDepend
\end{issues}
% TODO:改名为 列向量与行向量

\subsection{列向量}

几何向量的坐标让我们可以以一个全新的视角看待向量这个概念,我们可以把数组 $(a_1, \cdots, a_n)$ 称为一个向量$\bvec{a}$;由于我们常常会把它竖着记为
\begin{equation}
\pmat{a_1 \\ \vdots \\ a_n}~,
\end{equation}
这种向量被称为\textbf{列向量},$n$ 被称为 $a$ 的\textbf{维度},$a_i$ 被称为 $a$ 的第 $i$ 坐标\footnote{数学中没有规定一定要从 $1$ 开始计数,也可以从 $0$ 开始。}。对于列向量来说,存在一组特别的基底 $\{\bvec{e}_i\}_{i = 1}^n$,称为\textbf{标准基底},其中 $\bvec{e}_i$ 是第 $i$ 坐标为 $1$,其他坐标为 $0$ 的列向量,因此任何一个列向量都可以写成
\begin{equation}
\bvec{a} = a_1 \bvec{e}_1 + \cdots + a_n \bvec{e}_n~
\end{equation}
的形式。

第 $i$ 坐标 $a_i$ 的取值可以和几何向量一样取实数 $\mathbb{R}$,也可以取一些其他的数,比如复数 $\mathbb{C}$。方便起见,我之后只考虑实向量,$n$ 维向量就是 $n$ 维空间的 $\mathbb{R}^n$ 上的一点。

\addTODO{检查是否定义过 $n$ 维空间的 $\mathbb{R}^n$}
% Giacomo:要不要把《数学基础》提到《微积分》之前?

\addTODO{数集的定义和链接}

实数取值的 $2$ 维(或者 $3$ 维)列向量,等价于选取了坐标系的几何向量——由标准基底的存在,列向量并不是几何向量的推广。几何向量和列向量都是更一般的向量的特殊情况。

\subsection{行向量}

如果把向量“横过来”,我们就得到了\textbf{行向量},
\begin{equation}
\pmat{a_1 & \cdots & a_n}~,
\end{equation}
(注意,一般 $(a_1, \cdots, a_n)$ 表示的是列向量,区别在于逗号)。
% Giacomo:我一般用方括号表示矩阵,也是为了作出区别。

行向量和列向量的定义“本身”没有任何区别,对于某个外星人而言完全可以把这两个符号反过来;真正重要的是行向量和列向量之间的运算:考虑一个 $n$ 维行向量 $\bvec{a}$ 和 列向量 $\bvec{b}$,我们定义 $\bvec{a}$ \textbf{乘} $\bvec{b}$ 为
\begin{equation}
\bvec{a} \bvec{b} = \pmat{a_1 & \cdots & a_n} \pmat{b_1 \\ \cdots \\ b_n} = \sum_{i = 1}^n a_i b_i~,
\end{equation}
是一个数。

注:$l$是left的首字母,意味着从左边乘。

% 到矩阵再讲
% $\bvec{b}$ \textbf{乘} $\bvec{a}$ 为
% \begin{equation}
% \bvec{b} \bvec{a} = \pmat{b_1 \\ \cdots \\ b_n} \pmat{a_1 & \cdots & a_n} = \sum_{i = 1}^n a_i b_i~,
% \end{equation}

从这个角度来说,行向量是“列向量的函数”:
\begin{equation}
\begin{aligned}
l_{\bvec{a}}: \mathbb{R}^n &\to \mathbb{R} \\
\bvec{b} &\mapsto \bvec{a} \bvec{b} = \sum_{i = 1}^n a_i b_i~,
\end{aligned}
\end{equation}

\addTODO{再检查是否定义过映射“$\to$”,如果没有就把《数学基础》提到《微积分》之前,或者新开一个part?}

不过,正如我们之前提过的——“行向量和列向量的定义‘本身’没有任何区别”,因此反过来看列向量也是“行向量的函数”:
\begin{equation}
r_{\bvec{b}}(\bvec{a}) = \bvec{a} \bvec{b} = \sum_{i = 1}^n a_i b_i~,
\end{equation}
注:$r$是right的首字母,意味着从右边乘。

\subsection{转置}

我们可以通过转置把行/列向量相互转化,考虑列向量 $\bvec{a} = \pmat{a_1 \\ \vdots \\ a_n}$,它的转置记为 $\bvec{a}^T$(或者$\bvec{a}^\top$、${}^t \bvec{a}$)是行向量
\begin{equation}
\pmat{a_1 & \cdots & a_n}~.
\end{equation}

\addTODO{添加到Conven.tex}

% Giacomo:TODO:把用转置重写几何向量(的坐标)的内积。