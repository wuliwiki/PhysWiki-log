% 简谐振子(高中)
% keys 弹簧振子|simple harmonic oscillation|单摆
% license Usr
% type Tutor





要进一步了解简谐振子,请参考\enref{简谐振子(经典力学)}{SHO}。


\subsection{概念}


若一个物体受力所产生的加速度总是指向一个位置,且该加速度的大小正比于物体到这个位置的距离,我们就说这个物体结合它所受的力,构成了一个\textbf{简谐振子},加速度指向的位置被称为该简谐振子的\textbf{平衡位置}。

叫这个名字,首先是因为简谐振子会反复在平衡位置附近做周期性运动,而不会无限远离平衡位置;其次是因为这种周期性运动是最简单的形式,从而被冠以“简单而和谐”之名。我们首先看一些简谐振子的例子,再学着分析它们的运动。




\begin{example}{弹簧振子}

将轻质弹簧的一端固定,另一端连接在一个质量为$m$的物体上,物体可以沿着一个方向运动,弹簧也在这个方向上。此时弹簧和物体构成了一个简谐振子,我们可以称之为“弹簧振子”。

当弹簧处于原长时,物体不受弹簧的作用力,此时物体所在位置就是该简谐振子的平衡位置。当物体到平衡位置的距离为$x$时,物体受到大小为$kx$的拉力,方向指向平衡位置,其中$k$是弹簧的劲度系数。于是,物体的加速度总是指向平衡位置,且其大小正比于到平衡位置的距离:$a=\frac{k}{m}x$。

\end{example}


\begin{example}{单摆}

\pentry{导数的计算(高中)\nref{nod_HsDerB}}{nod_3d19}

将长为$L$的轻质杆的一端固定,另一端连接在一个质量为$m$的物体上,让系统在匀强引力场作用下运动。该系统被称为一个“单摆”,它\textbf{不是}简谐振子,但可以近似认为是一个简谐振子。


\begin{figure}[ht]
\centering
\includegraphics[width=10cm]{./figures/8adf458a581996c7.pdf}
\caption{单摆示意图。单摆的一端固定在$\color{Red}O\color{Black}$点,另一端拴着质量为$\color{RoyalBlue}m\color{Black}$的物体,物体可在屏幕平面内运动。由于摆长恒为$L$,物体的运动轨迹是以$\color{Red}O\color{Black}$为圆心的一个圆。重力的方向竖直向下,故单摆处于平衡位置时,杆沿着图中虚线。当单摆偏离平衡位置,杆与平衡位置夹角为$\theta$时,物体所受合力$\color{Orange}\vec{F}$垂直于杆指向下方,大小为$mg\sin\theta$;由几何规律也可知,物体到平衡位置的\textbf{竖直}距离为$L-L\cos\theta$。} \label{fig_SHOhs_1}
\end{figure}


显然,当杆沿着重力方向(竖直方向)下垂的时候,单摆处于平衡位置。设在平衡位置时物体的速度大小为$v_0$,角速度为$\omega_0$;当杆偏离竖直方向的角度为$\theta$时,物体速度为$v_\theta$,角速度为$\omega_\theta$;此时物体所受合力$\color{Orange}\vec{F}$垂直于杆指向下方,大小为$mg\sin\theta$,因此物体加速度为$\frac{\dd v_\theta}{\dd t}=g\sin\theta$,角加速度为$\frac{\dd \omega_\theta}{\dd t}=\frac{1}{L}\frac{\dd v_\theta}{\dd t}=\frac{g}{L}\sin\theta$。

当$\theta$很小时,$\sin\theta\approx\theta$,故此时可以认为,参数$\theta$的“加速度”等于$\frac{g}{L}\theta$,和弹簧振子中“$x$的加速度等于$\frac{k}{m}x$”同理。因此,以$\theta$为参数来描述单摆的运动,在角度比较小的情况下,可以认为单摆是一个简谐振子。

\end{example}


\begin{example}{二次曲线上的滚动}

设水平面上有一光滑曲面,其上各点的高度$h$和水平位置$x$的关系为$h=x^2$。空间中存在竖直向下、垂直于水平面的匀强引力场$\vec{g}$。一小球在曲面上无摩擦地滚动。

\begin{figure}[ht]
\centering
\includegraphics[width=10cm]{./figures/53df2b0c3a2b2796.pdf}
\caption{二次曲线简谐振动} \label{fig_SHOhs_2}
\end{figure}

小球在曲面上一点处时,其加速度相当于小球在一个无摩擦的、与曲线在该点处相切的斜面上一样。由此可算得小球的加速度为
\begin{equation}
\frac{\dd v}{\dd t} = -g\frac{x^2}{\sqrt{x^2+x^4}} = -gx\frac{1}{\sqrt{1+x^2}}~. 
\end{equation}
其水平分量为
\begin{equation}
\frac{\dd v_\parallel}{\dd t} = \frac{\dd v}{\dd t}\cdot\frac{x}{\sqrt{x^2+x^4}} = -\frac{g}{1+x^2}x~. 
\end{equation}
当$x$很小的时候,近似有$\frac{g}{1+x^2}=g$,从而小球的水平加速度近似为$gx$,正比于小球的水平位移并总是指向平衡位置,因此小球和曲面的组合也近似构成一个简谐振子。

\end{example}
%重新设计一个%设计了一个曲线,使得小球在上面的运动真的是简谐振动,但此曲线长得太丑了


你看,简谐运动的形式多种多样,不一定非要是弹簧振子的形式,甚至坐标也不一定非要是我们熟悉的一维位移。无论形式如何,简谐振子的运动总是遵循如下方程:
\begin{equation}\label{eq_SHOhs_1}
\frac{\dd}{\dd t}\qty(\frac{\dd}{\dd t}s(t)) = -cs(t) ~, 
\end{equation}
其中$c$是一个任意的正数,$s$是用来表示简谐振子运动状态的参量(它是时间$t$的函数),$\frac{\dd}{\dd t}$表示“关于时间求导”。对于弹簧振子,参量$s$就用物块的位移$x$,于是$\frac{\dd}{\dd t}x$就是物块的速度,而$\frac{\dd}{\dd t}\qty(\frac{\dd}{\dd t}x)$就是物块的加速度。对于单摆,尽管此时用的参量是偏离角度$\theta$,物理意义和位移$x$不同,但数学结构完全一致,因此$\theta$随时间的变化也跟简谐振子中$x$随时间的变化相似。

接下来,我们就要研究如何从\autoref{eq_SHOhs_1} 中求出$s(t)$是怎样的函数。



\subsection{简谐运动}

\autoref{eq_SHOhs_1} 可以理解为,函数$s(t)$关于$t$求两次导后,所得结果是自己的负数倍。什么样的函数具有这样的性质呢?正余弦函数:
\begin{equation}\label{eq_SHOhs_2}
\left\{
\begin{aligned}
    \frac{\dd}{\dd t}\qty(\frac{\dd}{\dd t}\cos(\omega t+\varphi_0)) ={}& -\omega^2\cos(\omega t+\varphi_0), \\
    \frac{\dd}{\dd t}\qty(\frac{\dd}{\dd t}\sin(\omega t+\varphi_0)) ={}& -\omega^2\sin(\omega t+\varphi_0)~. 
\end{aligned}
\right. 
\end{equation}
当然,\autoref{eq_SHOhs_2} 其实有些累赘,因为只要选择适当的$\varphi_0$,就可以用$\cos$函数来表示$\sin$函数了:$\cos(\omega t+\varphi_0-\frac{\pi}{2})=\sin(\omega t+\varphi_0)$;反之亦然。接下来,我们统一用$\cos$函数。

比较\autoref{eq_SHOhs_1} 和\autoref{eq_SHOhs_2} ,可见满足\autoref{eq_SHOhs_1} 的函数$s(t)$应为
\begin{equation}
s(t) = A\cos(\sqrt{c}t+\varphi_0)~,  
\end{equation}
其中$A$和$\varphi_0$是待定常数,由初值条件决定。比如说,如果已经知道一个弹簧振子在某一时刻的速度和位置,就能算出相应的系数$A$和$\varphi_0$,从而确定弹簧的整个运动轨迹。显然,$A$是整个周期运动中振子偏离平衡位置的最远“距离”,衡量了振动的幅度,因此被称为\textbf{振幅};$\varphi_0$表示了$t=0$的时候$\cos$函数处于其周期的哪个阶段,即相位,因此我们将$\varphi_0$称为\textbf{初相位}。

我们用一个例子来学习如何确定振幅和初始相位。



\begin{example}{弹簧振子的运动}

设弹簧振子由一端固定的弹簧和连接在弹簧另一端的物块构成,弹簧劲度系数为$k$,物块质量为$m$。以物块的平衡位置为原点,弹簧伸长的方向为正方向,建立一维坐标系。

设$x$为物块的位移,则其运动满足
\begin{equation}
m\frac{\dd}{\dd t}\qty(\frac{\dd}{\dd t}x) = -kx~, 
\end{equation}
于是物块的位移关于时间的函数应该是
\begin{equation}\label{eq_SHOhs_3}
x(t) = A\cos\qty(\sqrt{\frac{k}{m}}t+\varphi_0)~, 
\end{equation}
其速度就是
\begin{equation}\label{eq_SHOhs_4}
\frac{\dd}{\dd t}x(t) = -A\sqrt{\frac{k}{m}}\sin\qty(\sqrt{\frac{k}{m}}t+\varphi_0)~. 
\end{equation}

设在$t=0$时刻,物块在$x_0$处,其速度为$0$。将这三个量代入\autoref{eq_SHOhs_3} 和\autoref{eq_SHOhs_4} ,可得
\begin{equation}
\left\{
\begin{aligned}
x_0 ={}& A\cos\qty(\sqrt{\frac{k}{m}}\times0+\varphi_0), \\
0 ={}& -A\sqrt{\frac{k}{m}}\sin\qty(\sqrt{\frac{k}{m}}\times 0+\varphi_0)~. 
\end{aligned}
\right. 
\end{equation}
由此可以解得
\begin{equation}
\left\{
\begin{aligned}
\varphi_0 ={}& 0, \\
A ={}& x_0~. 
\end{aligned}
\right. 
\end{equation}

这非常符合直觉,因为$t=0$时刻物块速度为$0$,意味着此时物块已经达到了最远距离,振幅就应该是$x_0$。

\end{example}


























