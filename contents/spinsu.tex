% 自旋求和
% 自旋|费米子|求和

计算费曼图的时候,我们经常需要对费米子的极化态进行求和.我们可以进行这样的求和
\begin{align}\nonumber
\sum_{s = 1,2}u^s(p)\bar u^s(p ) & = \sum_s \begin{pmatrix}
\sqrt{p\cdot \sigma}\xi^s \\
\sqrt{p\cdot\bar\sigma} \xi^s
\end{pmatrix}\begin{pmatrix}
\xi^{s\dagger}\sqrt{p\cdot\bar\sigma} & \xi^{s\dagger}\sqrt{p\cdot \sigma}
\end{pmatrix}\\\nonumber
& = \begin{pmatrix}
\sqrt{p\cdot\sigma}\sqrt{p\cdot \bar \sigma} & \sqrt{p\cdot \sigma}\sqrt{p\cdot\sigma} \\
\sqrt{p\cdot\bar\sigma} \sqrt{p\cdot \bar\sigma} & \sqrt{p\cdot\bar\sigma} \sqrt{p\cdot\sigma}
\end{pmatrix} \\
& = \begin{pmatrix}
m & p\cdot \sigma \\
p\cdot \bar \sigma & m
\end{pmatrix}~.
\end{align}
第二行的化简过程中我们用到了这样的归一化条件
\begin{equation}
\sum_{s=1,2}\xi^s\xi^{s\dagger} = 1 = \begin{pmatrix}
1 & 0 \\
0 & 1
\end{pmatrix}~.
\end{equation}
我们推出了自旋求和公式
\begin{equation}
\sum_s u^s(p)\bar u^s(p) = \gamma \cdot p + m ~.
\end{equation}
同理
\begin{equation}
\sum_s v^s(p)\bar v^s(p) = \gamma\cdot p - m ~.
\end{equation}
$\gamma \cdot p$是一个经常要用到的物理量.我们可以用$p\!\!\!/\equiv \gamma^\mu p_\mu$来代替.
