% 理想气体状态方程
% keys 理想气体|压强|体积|温度|碰撞
% license Xiao
% type Tutor

\pentry{准静态过程\nref{nod_Quasta},分子撞击对容器壁的压强\nref{nod_MolPre}}{nod_5565}

\footnote{参考 Wikipedia \href{https://en.wikipedia.org/wiki/Ideal_gas}{相关页面}。}\textbf{理想气体(ideal gas)}是热力学中一个理想化的模型, 可以类比\enref{自由落体}{ConstA}, \enref{简谐振子等}{SHO}。 该模型描述静止的密闭容器中一定量气体的压强, 体积和温度之间的关系。 该模型假设:
\begin{enumerate}
\item 理想气体由大量\footnote{“大量” 具体指阿伏伽德罗常数数量级, 即 $~10^{23}$。}运动的微观粒子组成。每个粒子都是质量为 $m$ 的质点\footnote{注意,在标准状态下气体分子间平均距离和气体半径之比约为 $30$,所以可以近似地看作质点。},它的行为服从牛顿运动定律。

\item 粒子间无相互作用(这意味着分子势能是刚球势)\footnote{对于非常稀薄的气体,可以忽略分子间相互作用来做近似处理}。粒子与粒子、粒子与容器之间发生碰撞,所有碰撞都是弹性碰撞。

\item 组成理想气体的粒子的运动是\textbf{完全无序的、各向同性的}\footnote{对于实际的气体分子,每时每刻都会与其他气体分子发生充分多次的碰撞,而且从碰撞结果看充满了随机性。所以它几乎是完全无序、各向同性的。}。完全无序体系无宏观运动。这也要求理想气体不能处在外场中\footnote{例如处于重力场中的大气,其压强随高度的变化而变化。尽管如此,如果我们取其中一个小体积元,还是可以看成是理想气体}。
\end{enumerate}

理想气体状态方程最开始由实验得到, 这是因为在日常环境下, 大部分气体与理想气体符合较好。 其最常见的形式为
\begin{equation}\label{eq_PVnRT_1}
PV = nRT~.
\end{equation}
其中 $P$ 是气体的压强(处处相等), $V$ 是容器的体积, $n$ 是气体分子的摩尔数, $R$ 是\textbf{气体常数(gas constant)}, $T$ 是热力学温度, 单位为开尔文(见下文)。

一摩尔包含的粒子个数由\textbf{阿伏伽德罗常数(Avogadro constant)}精确定义为(\enref{见物理学常数}{Consts},下同)
\begin{equation}
N_A = 6.02214076\e{23}~.
\end{equation}
为了定义气体常数, 我们先来介绍热力学中一个重要的常数——\textbf{玻尔兹曼常数(Boltzmann constant)}, 它被精确定义为
\begin{equation}
k_B = 1.380649\e{-23} \Si{J/K}~.
\end{equation}
理想气体常数被定义为
\begin{equation}\label{eq_PVnRT_2}
R = k_B N_A = 8.31446261815324 \,\Si{J/K}~.
\end{equation}
令容器中总分子数为 $N = n N_A$, 所以状态方程(\autoref{eq_PVnRT_1}) 也可以记为
\begin{equation}\label{eq_PVnRT_4}
PV = N k_B T~.
\end{equation}

从微观角度, \enref{热力学温度}{tmp} 可以由理想气体分子平均动能定义\footnote{注意这里的动能是平动动能, 我们暂时不讨论气体分子的转动。}
\begin{equation}\label{eq_PVnRT_3}
\bar E_k = \frac32 k_B T~,
\end{equation}
所以气体分子总动能为
\begin{equation}\label{eq_PVnRT_5}
E_k = \frac32 Nk_B T = \frac{3}{2}nRT~.
\end{equation}

\begin{example}{标准状况下气体分子密度}
我们可以由理想气体状态方程得出化学中一个常用的常数: 标准状况下($273.15 \Si{K}$、$101 \Si{kPa}$) 每摩尔气体的体积约为 $22.4L$。 将温度和压强代入\autoref{eq_PVnRT_1} 得
\begin{equation}
\frac{V}{n} = \frac{RT}{P} = \frac{8.314 \times 273.15}{1.01\e5}\Si{m^3/mol} = 22.48 \Si{L/mol}~,
\end{equation}
注意理想气体的假设使结果有微小误差。
\end{example}

\subsection{由经典力学推导}

我们在 “\enref{分子撞击对容器壁的压强}{MolPre}” 中已经详细推导了压强和分子速度的关系, 以下重复的部分将简略带过。 注意分子密度趋近于 0 的假设导致分子之间没有作用力, 每个分子的运动都是可以看作是独立的。

假设长方体容器的 $x, y, z$ 三个方向的边长分别为 $a, b, c$, 则体积为  $V = abc$。 考虑一个初始延任意方向运动的分子, 与容器壁发生完全弹性碰撞, 它在 $x$ 方向的周期(即运动一个来回所需的时间)为 $2a/v_x$, 每个周期带给 $x = a$ 容器壁的冲量为 $2m v_x$, 该容器壁面积为 $bc$, 所以受到该粒子的平均压强 $P$ 为冲量除以周期除以面积 $mv_x^2/V$, 即
\begin{equation}
P_i V = mv_{x,i}^2~,
\end{equation}
注意这里我们用角标 $i$ 来表示第 $i$ 个分子。 如果有 $N$ 个分子, 质量都为 $m$, 那么
\begin{equation}
P V = \sum P_i V = 2 N \qty(\frac12 m \overline {v_x^2}) = 2 N \bar E_{kx}~,
\end{equation}
注意 $\overline {v_x^2}$ 和 $\bar v_x^2$ 是不同的。 前者是先对每个分子的速度取平方再平均。 而后者是先计算速度的平均值在平方。 等式右边等于 $2N$ 乘以所有分子在 $x$ 方向的平均动能。 由于我们假设分子运动各向同性(即不会出现某些方向的分子运动较快), 所以平均的总动能等于单方向平均动能的 3 倍(注意这里我们假设是 3 维空间, 如果是 $N_d$ 维空间, 就是 $N_d$ 倍)
\begin{equation}
\bar E_k = \frac{1}{2} m (\overline {v_x^2} + \overline {v_y^2} + \overline {v_z^2}) = \frac{3}{2} m \overline {v_i^2}~.
\end{equation}
所以有(参考\autoref{eq_MolPre_4}~\upref{MolPre})
\begin{equation}\label{eq_PVnRT_6}
P V = \frac23 N \bar E_k~,
\end{equation}
这说明压强和体积的乘积等于分子\textbf{平动动能}的 $2/3$(注意不包括转动,振动等动能)。 用\autoref{eq_PVnRT_3} 消去 $E_k$, 就得到了\autoref{eq_PVnRT_4} 形式的理想气体状态方程。

\subsection{多原子分子理想气体}\label{sub_PVnRT_1}
对于多原子分子理想气体,假设理想气体的每个分子含有 $K$ 个原子,而每个原子都可以抽象为三维空间内的一个质点。故每个 $K$ 原子分子的自由度是 $3K$,从而 $N$ 个气体分子所含的总自由度数为 $3NK$。

对于每个分子的 $3K$ 个自由度又可以细分为描述质心运动的平动自由度 $\mathcal{l}_t$、体现分子空间位形取向的转动自由度 $\mathcal{l}_r$ 和体现分子内原子间距离变化的振动自由度 $\mathcal{l}_v$。在三维空间中,平动自由度为 $\mathcal{l}_t = 3$。转动自由度及振动自由度则由分子的几何形状决定。

首先定义\textbf{线性分子}为分子结构位的所有原子位于一条直线上的化学分子。对于 $K > 2$ 且分子为\textbf{非线性分子}的情况,确定一个气体分子的空间取向需要 $3$ 个欧拉角,因此 $\mathcal{l}_r = 3$,进而 $\mathcal{l}_v = 3K-\mathcal{l}_t-\mathcal{l}_r = 3K-6$;对于 $K=2$ 或 $K>2$ 且分子为\textbf{线性分子}的情况,转动自由度数减小为 $\mathcal{l}_r = 2$,从而类似的可以得到振动自由度 $\mathcal{l}_v = 3K-5$。

多原子气体分子的状态方程可以由其自由能求导得出,仍满足 $PV=Nk_BT$。