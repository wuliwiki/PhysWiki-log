% 南京大学 2014 年考研普通物理
% 南大|南京大学|普物|普通物理
\subsection{力学}
1. 一质量为 $m_{1}$,长为 $2l$ 的均匀细杆,在其一端有一个很小的光滑圆孔。开始时,细杆在光滑水平面上以速度 $v$ 平动。某时,将一光滑小钉突然穿过小孔固定在平面上,求此后杆做定轴转动的角速度及杆对钉的反作用力。
\begin{figure}[ht]
\centering
\includegraphics[width=8.5cm]{./figures/d092f136a94fe335.pdf}
\caption{力学第一题图} \label{fig_NJU14_1}
\end{figure}
2. 一石块静止地悬挂在弹簧秤上,称得重量为 $W$。如果把石块挂在弹簧秤上(此时弹簧末伸长), 让它自行下坠, 则弹簧称上读数最大可达 $W_{1}$。 求 $W$ 与 $W_{1}$ 关系。
\begin{figure}[ht]
\centering
\includegraphics[width=12cm]{./figures/a4e15d1a3b8cb745.pdf}
\caption{力学第二题图} \label{fig_NJU14_2}
\end{figure}
\subsection{热学}
1. 根据麦克斯韦分布律求分子平动动能的最可几值。

2. 用一理想热泵从温度为 $T_{0}$ 的河水中吸热给某一建筑物保暖。设热泵的功率为 $W$。该建筑物单位时间内向外散射的热量为 $\alpha\left(T-T_{0}\right)$,试问\\
(1) 该建筑物室内的平衡温度 $T_{1}$ 是多少?\\
(2) 若把热泵更换为同功率的加热器直接对建筑物加热, 其平衡温度为 $T_{2}$, 又是多少?\\
(3) 上述两种方式中何者更为经济?\\
\subsection{电磁学}
1. 电荷 $Q$ 均匀分布在半径为 $R$ 的半球壳上,求球心处的电场和电势。
\begin{figure}[ht]
\centering
\includegraphics[width=6cm]{./figures/9a0e0297a3499c96.pdf}
\caption{电磁学第一题图} \label{fig_NJU14_4}
\end{figure}
2. 长度为 $2 \pi a$ 电阻为 $r$ 的均匀细导线,首尾相接形成一个半径为 $a$ 的圆,现将一个内电阻 为 $R$ 的伏特表用导线 (无 $R$) 连在圆的两点上。设两点之间的弧度为 $\theta$,一个均匀变化的匀强磁场垂直于圆所在的平面,$\opn dB/\opn dt =k$,试问伏特表在不同的接法下,读数分别为多少?
\begin{figure}[ht]
\centering
\includegraphics[width=8cm]{./figures/d370cc3380b1a91e.pdf}
\caption{电磁学第二题图} \label{fig_NJU14_3}
\end{figure}
\subsection{光学}
1. 白光垂直照射到空气中一厚度为 $380 \mathrm{~nm}$ 的肥㚖膜上,设肥㚖膜的折射率为 $1.33$,问该膜的正面呈现什么颜色?反面?

2. 在某单缝衍射实验中,光源发出的光含有两种波长 $\lambda_{1}$ 和 $\lambda_{2}$,垂直入射于单缝上,假如波 长 $\lambda_{1}$ 的光的第一衍射极小与波长 $\lambda_{2}$ 的光的第二衍射极小重合,试问\\
(1) 这两种波长之间有何关系?\\
(2) 在这两种波长的光形成的衍射图样中,是否还有其他极小相重合?