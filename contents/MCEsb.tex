% 巨正则系综法
% keys 巨正则系综|统计力学|能级

\begin{issues}
\issueDraft
\end{issues}

巨正则系综的方法与正则系综方法的区别时,热力学系统可以与一个粒子源接触,从而粒子数可以是不固定的。

假设一个热力学系统与温度为 $T$ 的大热源和化学势为 $\mu$ 的大粒子源接触,系统可以和热源交换热量,可以与粒子源交换粒子,因此它的能量和粒子数是不守恒的。那么系统的一个能量为 $E$,粒子数为 $N$ 的微观状态出现的概率正比于
\begin{equation}
P(E,N)\propto \exp(\mu N-E)~.
\end{equation}
对所有可能的系统微观状态的物理量结果求平均, 就得到了物理量的测量结果
\begin{equation}
\langle O\rangle=\sum_{N}\sum_i P(E_i,N)\exp(\mu N-E_i)~.
\end{equation}


\subsection{巨配分函数}
巨配分函数 $\Xi=\sum_{N,i} \exp((\mu N-E_i)/kT)$。由于 $\sum_{N,i} P(E_i,N)=1$,再利用 $P(E,N)\propto \exp(\mu N-E)$,所以可以得到 $P(E_i,N)=\exp(\mu N-E_i)/\Xi$。
\begin{equation}
P(E_i,N)=\frac{\E^{(\mu N - E_i)/(kT)}}{\Xi}~,
\end{equation}
这保证了所有状态的概率之和为一,满足归一化关系。

利用巨配分函数可以得到许多有用的结果。例如
\begin{equation}
\begin{aligned}
&\langle E-\mu N\rangle = \sum_{N,i} (E_i-\mu N) \frac{\exp((\mu N - E_i)/kT)}{\Xi}=-\frac{1}{\Xi}\frac{\partial }{\partial \beta}\Xi=-\frac{\partial}{\partial\beta} \ln \Xi~.\\
&\langle N\rangle =kT \frac{1}{\Xi} \frac{\partial}{\partial \mu}\Xi=kT\frac{\partial}{\partial \mu}\ln \Xi
\end{aligned}
\end{equation}
\subsection{“能级导向”}
\begin{equation}\ali{
\Xi & = \sum_{N=1}^\infty  \sum_{\{n_i\}}^*  \exp(N\mu  - \sum_{i=1}^\infty n_i \varepsilon_i)\beta
= \sum_{N=1}^\infty  \sum_{\{n_i\}}^* z^N \prod_{i=0}^\infty \qty(\E^{-\varepsilon_i\beta})^{n_i}\\
&= \sum_{N=1}^\infty \sum_{\{n_i\}}^* \prod_{i=0}^\infty \qty(z \E^{-\varepsilon_i \beta})^{n_i}
= \sum_{n_1}^* \sum_{n_2}^* \dots \prod_{i=0}^\infty \qty(z \E^{-\varepsilon_i \beta})^{n_i}\\
&= \sum_{n_1}^* \qty(z \E^{-\varepsilon_i\beta})^{n_1} \sum_{n_2}^* \qty(z \E^{ -\varepsilon_i\beta})^{n_2}\dots
= \prod_i^\infty \sum_{n_i}^* \qty(z \E^{-\varepsilon_i \beta })^{n_i}~.
}\end{equation}

\subsection{系统的热力学性质}
由最大概率项假设,
\begin{equation}
1 = \frac{\Omega \E^{(\mu N - E)/(kT)}}{\Xi}
= \frac{\E^{S/k} \E^{(\mu N - E)/(kT)}}{\Xi}~,
\end{equation}
\begin{equation}
\E^{S/k} \E^{(\mu N - E)/(kT)} = \Xi~,
\end{equation}
\begin{equation} 
E - ST - \mu N =  - kT\ln \Xi~.
\end{equation}
令 $\Phi  =  - kT\ln \Xi $ 叫做\textbf{巨势}
\begin{equation}
\Phi  = E - ST - \mu N~,
\end{equation}
\begin{equation}
\Phi  = E - ST - \mu N = F - G = E - ST - (E - ST + PV) =  - PV~.
\end{equation}
考虑到 $\dd{E} = T\dd{S} - P\dd{V} + \mu \dd{N}$
\begin{equation}
\dd{\Phi} = -P\dd{V} - S\dd{T} - N\dd{\mu}~,
\end{equation}
所以
\begin{equation}
S = -\qty(\pdv{\Phi}{T})_{V,\mu } ~,\qquad 
N = - \qty(\pdv{\Phi}{\mu})_{V,T} ~,\qquad
P = - \qty(\pdv{\Phi}{V})_{T,\mu}~.
\end{equation}
另外有一个求能级分布的公式
\begin{equation}
\ev{n_i} = \frac{1}{\Xi} \sum_{N=1}^\infty \sum_{\{n_i\}}^* n_i \exp(N\mu - \sum_{i=1}^\infty n_i \varepsilon_i)\beta = -\frac{1}{\beta \Xi } \pdv{\Xi}{\varepsilon_i} =  - kT \pdv{\varepsilon_i} \ln \Xi  = \pdv{\Phi}{\varepsilon_i}~.
\end{equation}