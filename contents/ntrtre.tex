% 数论三角和与高斯和
% keys 三角和
% license Usr
% type Tutor

\pentry{同余与剩余类\nref{nod_modres},指数函数(复数)\nref{nod_CExp}}{nod_521a}

\subsection{数论三角和}
由于 $e^{\pi \I} = -1$,利用 $e^{2\pi \I} = 1$ 可以构造出诸多三角和。在数论中我们记
\begin{equation}
e(t) := e^{2\pi \I t} ~,
\end{equation}
则对于 $t$ 的有理值,当 $x \equiv y \pmod m$ 时,将有
\begin{equation}
e\left(\frac{x}{m}\right) = e\left(\frac{y}{m}\right) ~.
\end{equation}
这是数论三角和的重要性质。

\subsection{Gauss 和}
\begin{definition}{Gauss 数论三角和}
定义 $S(m, n)$ 为\textbf{高斯(数论三角)和},\footnote{在数论中,求和取遍剩余系时,一般都用字母 $h$ 而非其他字母。}
\begin{equation}
S(m, n) := \sum_{h = 0}^{n-1} e\left(\frac{h^2m}{n}\right) ~.
\end{equation}

\end{definition}
Gauss 和在二次剩余中应用较多。

考虑对于任意 $r$,
\begin{equation}
e\left(\frac{(h + rn)^2m}{n}\right) = e\left(\frac{h^2m}{n}\right) ~,
\end{equation}
故我们可以不将 $h$ 限定在从 $0$ 到 $(n-1)$ 取遍,而只要 $h$ 取遍一个完全剩余系即可。此时用记号 $h(n)$ 表示取遍一个 $n$ 的完全剩余系。
\begin{corollary}{}
\begin{equation}
S(m, n) := \sum_{h(n)} e\left(\frac{h^2 m}{n}\right) ~.
\end{equation}

\end{corollary}

\pentry{线性同余\nref{nod_linmod}}{nod_d40a}
下面引出由线性同余中的\autoref{the_linmod_3}~\upref{linmod}的一个推论。
\begin{corollary}{}
若 $(n, n') = 1$,则
\begin{equation}
S(m, nn') = S(mn', n) \times S(mn, n') ~.
\end{equation}

\end{corollary}
\textbf{证明}:考虑 $h$ 取遍 $n$ 的一个完全剩余系而 $h'$ 取遍 $n'$ 的一个完全剩余系,则由\autoref{the_linmod_3}~\upref{linmod}将立刻得到 $h_0 = hn' +h'n$ 取遍 $nn'$ 的完全剩余系。而
\begin{equation}
mh_0^2 = m(h'n + hn')^2 \equiv mh^2n'^2 + mh'^2n^2 \pmod{nn'} ~,
\end{equation}
故
\begin{equation}
\begin{aligned}
S(mn', n) \times S(mn, n') &= \left[ \sum_{h(n)} e\left(\frac{h^2mb'}{n}\right) \right] \times \left[\sum_{h'(n')} e\left(\frac{h'^2mn}{n'}\right)\right] \\
&= \sum_{h(n), h'(n')} \left[ e\left(\frac{h^2 mn'}{n} + \frac{h'^2 mn}{n'}\right)\right] \\
&= \sum_{h(n), h'(n')} \left[ e\left( \frac{m(h^2 n'^2 + h'^2 n^2)}{nn'} \right)\right]\\
&= \sum_{h_0(nn')} e\left( \frac{m h_0^2}{nn'} \right) = S(m, nn')~.
\end{aligned}
\end{equation}
证毕!

\begin{theorem}{Gauss 和的等价定义}
利用 Legendre 符号,对素数 $n$ 的 Gauss 和 $S(1, n)$ 给出了一个等价定义:
\begin{equation}
S(1, n) = \sum_{h=0}^{n-1} \left(\frac hp\right) e\left(\frac{h}{n}\right) ~.
\end{equation}

\end{theorem}