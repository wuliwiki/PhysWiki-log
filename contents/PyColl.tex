% Python 的数据结构与算法笔记
% license Xiao
% type Note

\begin{issues}
\issueDraft
\end{issues}

\subsection{list}
\begin{itemize}
\item \verb`list3 = list1 + list2` 可以拼接到一个新的 list。
\item \verb`list1 += list2` 或者 \verb`list1.extend(list2)` 可以拼接两个 list 到 \verb`list1`。 但如果用 \verb`list1.appnd(list2)` 就会把 \verb`list2` 作为一整个元素添加。
\end{itemize}

\subsection{set}
\begin{itemize}
\item \verb`s = {1,2,3}` 创建 set, 注意 \verb`{}` 是空字典, \verb`set()` 创建空 set。
\item 插入元素 \verb`s.add(一个元素)`, \verb`s.update(一些元素如list/tuple/set/dict等)`, 如果元素已经存在会被忽略。
\item \verb`元素 in s` 可以判断是否存在一个元素
\end{itemize}

\subsection{deque}
\begin{itemize}
\item double ended queue (实现: linked list ?)
\item \href{https://docs.python.org/3/library/collections.html}{官方文档}
\end{itemize}

\subsection{heapq}
\begin{itemize}
\item priority queue
\item \href{https://docs.python.org/3/library/heapq.html}{官方文档}
\end{itemize}

\subsection{树}
\begin{itemize}
\item \href{https://stackoverflow.com/questions/2442014/tree-libraries-in-python}{一些树算法库}
\end{itemize}


\subsection{图}
\begin{itemize}
\item \href{https://wiki.python.org/moin/PythonGraphLibraries}{一些图算法库}。
\end{itemize}

\subsection{算法题常用}
\begin{itemize}
\item 基础: basic data structures and oop, methods for strings and arrays, custom sorting, enumerate, zip functions
\item 进阶 map, reduce, filter, lambda expressions
\item 模块 collections itertools
\end{itemize}
