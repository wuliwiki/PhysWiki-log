%开普勒第三定律的证明
%20min
方法一:
\pentry{开普勒第一定律证明; 开普勒第二定律证明}%未完成链接
在开普勒第一定律证明%未完成链接
中,得出行星轨道的极坐标方程为
\begin{equation}\label{Keple3_eq1}
  r = \frac{p}{{1 - e \cdot \cos \theta }}
\end{equation}
其中 $p = {{{h^2}}}/{{GM}}$,  $h$ 是行星的角动量%未完成链接
与质量之比, $G$ 是引力常数, $M$ 是中心天体的质量. $e$ 是圆锥曲线的离心率, 当 $0 \le e < 1$ 时, 轨道是椭圆, 取等号时, 轨道是圆的. \\
椭圆的半长轴为
\begin{equation}\label{Keple3_eq2}
  a = \frac{1}{2}\left( {r\left( 0 \right) + r\left( \pi  \right)} \right) = \frac{1}{{1 - {e^2}}}\frac{{{h^2}}}{{GM}}
\end{equation}
椭圆的半短轴为
\begin{equation}\label{Keple3_eq3}
  b = a\sqrt {1 - {e^2}}  = \frac{1}{{\sqrt {1 - {e^2}} }}\frac{{{h^2}}}{{GM}}
\end{equation}
椭圆的面积为
\begin{equation}\label{Keple3_eq4}
  S = \pi ab = \frac{\pi }{{\sqrt {{{\left( {1 - {e^2}} \right)}^3}} }}\frac{{{h^4}}}{{{{\left( {GM} \right)}^2}}}
\end{equation}
由开普勒第二定律证明%未完成链接
中的结论, 单位时间扫过的面积为
\begin{equation}\label{Keple3_eq5}
  \frac{{dS}}{{dt}} = \frac{L}{{2m}} = \frac{h}{2}
\end{equation}
其中 $L$ 是角动量, $m$ 是行星的质量.\\
所以行星的周期为
\begin{equation}\label{Keple3_eq6}
  T = {S \mathord{\left/
 {\vphantom {S {\frac{{dS}}{{dt}} = }}} \right.
 \kern-\nulldelimiterspace} {\frac{{dS}}{{dt}} = }}\frac{{2\pi }}{{\sqrt {{{\left( {1 - {e^2}} \right)}^3}} }}\frac{{{h^3}}}{{{{\left( {GM} \right)}^2}}}
\end{equation}
所以由\autoref{Keple3_eq2} 与\autoref{Keple3_eq6} 得 ${{{T^2}}}/{{{a^3}}} = {{4{\pi ^2}}}/{{GM}}$ 是一个常数, 但注意这个常数与中心天体的质量有关, 所以不同中心天体的比例常数不同. \\
这个比例系数不用记忆, 需要用时, 只需用高中的知识计算天体圆周运动时的特例即可. 在注意圆作为特殊的椭圆, 其半长轴就是半径.\\
