% 第三方的
% license Usr
% type Test

这是一个关于概率的问题,涉及到基本概率和条件概率的计算。首先我们来定义一下事件:  事件A:甲抽到有奖的奖券。 事件B:乙抽到有奖的奖券。 接下来,我们根据题目描述来计算各个概率: (1)甲抽到有奖的奖券的概率。因为总共有10张奖券,其中有3张是有奖的,所以: \begin{equation}
 P(A) = \frac{3}{10}  ~
\end{equation}(2)乙抽到有奖的奖券的概率。这里需要考虑两种情况: - 如果甲没有中奖,乙抽到有奖的奖券的概率是
\begin{equation}
 \frac{3}{9}~
\end{equation}
 - 如果甲中奖了,乙抽到有奖的奖券的概率是 \begin{equation}
 \frac{2}{9}~
 \end{equation}因为此时盒子里剩下8张奖券,其中2张是有奖的。 由于甲中奖和不中奖的概率各为 \begin{equation}
 ( \frac{3}{10} )~,(\frac{7}{10})~
 \end{equation}所以乙抽到有奖的奖券的总概率 \begin{equation}
 
 \end{equation}  为: \begin{equation}
  P(B) = P(A) \cdot P(B|A) + P(\overline{A}) \cdot P(B|\overline{A})~
 \end{equation}\begin{equation}
 = \frac{3}{10} \cdot \frac{2}{9} + \frac{7}{10} \cdot \frac{3}{9} = \frac{6}{90} + \frac{21}{90} = \frac{27}{90} = \frac{3}{10} ~
 \end{equation}(3)甲和乙都抽到有奖的奖券的概率。这可以通过甲抽到有奖的奖券,然后乙接着抽到有奖的奖券的概率来计算: \begin{equation}
  P(AB) = P(A) \cdot P(B|A) ~
 \end{equation}
 \begin{equation}
 = \frac{3}{10} \cdot \frac{2}{9} = \frac{6}{90} = \frac{1}{15} ~
 \end{equation}
 (4)在乙抽到有奖的条件下,甲抽到有奖的概率,这是一个条件概率。根据条件概率的定义,我们有: \begin{equation}
  P(A|B) = \frac{P(AB)}{P(B)}  = \frac{\frac{1}{15}}{\frac{3}{10}}= \frac{1}{15} \cdot \frac{10}{3} = \frac{2}{9}  ~
 \end{equation}
 以上就是这个问题的详细解答。

