% 恩斯特·策梅洛(综述)
% license CCBYSA3
% type Wiki

本文根据 CC-BY-SA 协议转载翻译自维基百科\href{https://en.wikipedia.org/wiki/Ernst_Zermelo}{相关文章}。

\begin{figure}[ht]
\centering
\includegraphics[width=6cm]{./figures/e3826f3f611894f0.png}
\caption{} \label{fig_Zerme_1}
\end{figure}
恩斯特·弗里德里希·费迪南德·策梅洛(Ernst Friedrich Ferdinand Zermelo,发音:/zɜːrˈmɛloʊ/;德语:[tsɛɐ̯ˈmeːlo];1871年7月27日-1953年5月21日)是德国的逻辑学家和数学家,他的工作对数学基础具有重要影响。他因在发展策梅洛-弗伦克尔公理化集合论以及证明良序定理方面的贡献而闻名。此外,他在1929年关于国际象棋棋手排名的研究,首次描述了一种对偶比较模型,这一方法在多个应用领域中继续产生深远的影响。
\subsection{生活}
恩斯特·策梅洛于1889年毕业于柏林的路易森斯特第高等学校(现为海因里希·施利曼中学)。随后,他在柏林大学、哈雷大学和弗赖堡大学学习数学、物理和哲学。他于1894年在柏林大学完成了博士学位,论文题目为变分法(Untersuchungen zur Variationsrechnung)。策梅洛继续留在柏林大学,成为普朗克的助手,并在其指导下开始研究流体动力学。1897年,策梅洛前往哥廷根大学,这时的哥廷根大学是世界领先的数学研究中心,他于1899年完成了博士后资格论文。

1910年,策梅洛离开哥廷根,受聘为苏黎世大学数学系的教授,直到1916年辞职。他于1926年被授予弗赖堡大学的名誉教授职位,但因不满阿道夫·希特勒的政权,於1935年辞去该职位。在第二次世界大战结束后,应策梅洛的请求,他被重新恢复了弗赖堡的名誉教授职务。
\subsection{集合论研究}
1900年,在巴黎举行的国际数学家大会上,大卫·希尔伯特向数学界提出了著名的希尔伯特问题,这是一个列出23个未解的基础性问题的清单,数学家们应在接下来的一个世纪内进行攻克。第一个问题是集合论中的问题,即由康托尔在1878年提出的连续统假设,在阐述这个问题时,希尔伯特还提到了需要证明良序定理。
\begin{figure}[ht]
\centering
\includegraphics[width=6cm]{./figures/4e5a90def210d0d7.png}
\caption{恩斯特·策梅洛(1953年,弗赖堡) } \label{fig_Zerme_2}
\end{figure}
在希尔伯特的影响下,策梅洛开始研究集合论问题,并于1902年发表了关于超限基数相加的第一篇论文。到那时,他已经发现了所谓的拉塞尔悖论。1904年,他成功地迈出了希尔伯特所建议的关于连续统假设的第一步,证明了良序定理(即每个集合都可以良序)。这一结果使策梅洛声名鹊起,并于1905年被任命为哥廷根大学的教授。他基于幂集公理和选择公理的良序定理证明并未被所有数学家接受,主要因为选择公理是非构造性数学的典型范例。1908年,策梅洛成功地提出了改进版的证明,利用了德德金德对集合“链”的概念,这一版本得到了更广泛的接受;主要是因为在同一年,他还提出了集合论的公理化方案。

策梅洛于1905年开始对集合论进行公理化;尽管他未能证明其公理系统的一致性,但他于1908年仍然发表了他的结果。有关这篇论文的概要以及原始公理和编号,请参见策梅洛集合论的相关文章。

1922年,亚伯拉罕·弗伦克尔和托尔尔夫·斯科勒姆独立地改进了策梅洛的公理系统。由此产生的系统,现在被称为策梅洛-弗伦克尔公理(ZF),是目前最常用的集合论公理化系统。
\subsection{策梅洛的导航问题}
\begin{figure}[ht]
\centering
\includegraphics[width=6cm]{./figures/8ca600b730f3a4e5.png}
\caption{恩斯特·策梅洛的墓碑,位于弗赖堡(Freiburg im Breisgau)君特塔尔区(Günterstal)的君特塔尔公墓(Friedhof Günterstal)。} \label{fig_Zerme_3}
\end{figure}
策梅洛导航问题提出于1931年,是一个经典的最优控制问题。该问题涉及一艘船在水域中航行,从起点\(O\)出发,前往目的地\(D\)。船具有一定的最大速度,我们希望推导出最佳的控制策略,以便在最短的时间内到达\(D\)。

在不考虑外部因素如水流和风的情况下,最优控制策略是船始终朝向\(D\)航行。此时,船的航迹就是从\(O\)到\(D\)的线段,这显然是最优的。但如果考虑到水流和风的影响,当施加在船上的合力不为零时,忽略水流和风的控制策略将无法得到最优路径。
\subsection{出版物}
\begin{itemize}
\item 策梅洛,恩斯特(2013),埃宾豪斯,海因茨-迪特尔;弗雷泽,克雷格·G.;金守,明宏(编辑),《恩斯特·策梅洛—全集,第I卷:集合论,杂文》,海德堡科学院数学与自然科学学会文集,第21卷,柏林:Springer-Verlag,doi:10.1007/978-3-540-79384-7,ISBN 978-3-540-79383-0,MR 2640544  
\item 策梅洛,恩斯特(2013),埃宾豪斯,海因茨-迪特尔;金守,明宏(编辑),《恩斯特·策梅洛—全集,第II卷:变分法,应用数学与物理》,海德堡科学院数学与自然科学学会文集,第23卷,柏林:Springer-Verlag,doi:10.1007/978-3-540-70856-8,ISBN 978-3-540-70855-1,MR 3137671

\item 让·范·海耶诺特(1967)。《从弗雷格到哥德尔:数学逻辑源书,1879–1931》,哈佛大学出版社。  
\item 1904年。《证明每个集合都可以良序》,139-141。  
\item 1908年。《良序可能性的一个新证明》,183-198。  
\item 1908年。《集合论基础研究 I》,199-215。  
\item 1913年。《集合论在国际象棋理论中的应用》,收录于拉斯穆森,E.(编辑),2001年,《游戏与信息读本》,Wiley-Blackwell:79-82。  
\item 1930年。《关于边界数与集合的领域:集合论基础研究的新探讨》,收录于埃德尔,威廉·B.(编辑),1996年,《从康德到希尔伯特:数学基础的源书》,2卷,牛津大学出版社:1219-1233。
\end{itemize}
他人作品:
\begin{itemize}
\item 《策梅洛的选择公理及其起源、发展与影响》,格雷戈里·H·摩尔,1982年,作为《数学与物理科学历史研究》系列第8卷,Springer Verlag,纽约。
\end{itemize}
\subsection{另见}
\begin{itemize}
\item 选择公理
\item 无穷公理
\item 大小限制公理
\item 并集公理
\item 博尔兹曼大脑
\item 选择函数
\item 累积层次
\item 对偶比较
\item 冯·诺依曼宇宙
- 14990 策梅洛,小行星
\end{itemize}
\subsection{参考文献}
\begin{itemize}
\item 格拉特-吉内斯,艾佛(2000)。《数学根源的探索 1870–1940》,普林斯顿大学出版社。
\item 埃宾豪斯,海因茨-迪特尔(2007)。《恩斯特·策梅洛:走近他的生活与工作》,Springer。ISBN 978-3-642-08050-0。
\item 金守,明宏(2004)。"Zermelo and set theory",《符号逻辑学通报》,10(4):487-553。doi:10.2178/bsl/1102083759。MR 2136635。S2CID 231795240。
\item 施瓦尔贝,乌尔里希;沃克,保罗(2001)。"Zermelo and the Early History of Game Theory"(PDF)。《博弈与经济行为》,34(1):123-137。doi:10.1006/game.2000.0794。原文存档于2017年4月1日(PDF)。
\item 范达伦,迪尔克;埃宾豪斯,海因茨-迪特尔(2000年6月)。"Zermelo and the Skolem Paradox",《符号逻辑学通报》,6(2):145-161。CiteSeerX 10.1.1.137.3354。doi:10.2307/421203。hdl:1874/27769。JSTOR 421203。S2CID 8530810。
\end{itemize}
\subsubsection{引用文献}
\begin{itemize}
\item 策梅洛,恩斯特(1929)。"Die Berechnung der Turnier-Ergebnisse als ein Maximumproblem der Wahrscheinlichkeitsrechnung",《数学杂志》(德语),29(1):436-460。doi:10.1007/BF01180541。S2CID 122877703。
\item 卡普兰斯基,欧文(2020)。《集合论与度量空间》,普罗维登斯:美国数学学会,第36-37页。ISBN 978-1-4704-6384-7。
\end{itemize}
\subsection{外部链接}
\begin{itemize}
\item 在互联网档案馆中关于恩斯特·策梅洛的作品
\item 欧'康纳,约翰·J.;罗伯逊,埃德蒙·F.,"恩斯特·策梅洛",圣安德鲁斯大学数学历史档案(MacTutor History of Mathematics Archive)
\item 策梅洛导航
\end{itemize}