% 整环
% keys 整环|零因子|整除性
% license Xiao
% type Tutor

\begin{issues}
\issueOther{不应需要环同态作为预备}
\end{issues}


\pentry{环同态\upref{RingHm}}



我们常见的整数环属于一类性质极为良好的环,我们称这个分类为“无零因子交换幺环”,或者“整环”。注意区分“整数环”和“整环”这两个相似的术语,前者特指整数构成的环,后者则指一类环。

\begin{definition}{零因子}\label{def_Domain_1}
在环 $R$ 中,如果有两个\textbf{非零}的元素 $a, b$ 使得 $ab=0$,那么我们称 $a$ 是一个\textbf{左零因子},而 $b$ 是一个\textbf{右零因子}。如果某元素即使左零因子又是右零因子,那么我们称它是一个\textbf{零因子(zero divisor)}。
\end{definition}

简单来说,零因子就是“相乘得到零的非零元素”,其名称的含义就是“零的因子”。整数环里不存在零因子,但是这个概念也不难理解:考虑环 $\mathbb{Z}_{12}$,在这个环里,$3\not=0$,$4\not=0$,但是 $3\times 4=12=0$。

\begin{definition}{整环}
对于环 $R$,如果它的乘法交换并且没有零因子,那么我们称这个环是一个\textbf{整环(domain)}。
\end{definition}

整环的概念可以记为“无零因子交换幺环”,用以指代它的三个关键特点:“无零因子”、“交换”和“有乘法单位元”。最后一个特点在本章的语境下显得冗余,因为我们限定环都是含有乘法单位元的;强调幺环只是为了避免使用其它术语体系时可能的混淆。

\begin{example}{整环的例子}
\begin{itemize}
\item 整数环
\item 多项式环
\item 高斯整数环 $\mathbb{Z}[\I]$
\end{itemize}

\end{example}

\subsection{常用概念}

接下来是一系列非常有用的概念,也是后续的进阶词条的基础。

\begin{definition}{因子和整除性}
在整环 $R$ 中,如果对于 $a, b\in R$,存在 $r\in R$ 使得 $b=ar$,那么我们称 $a$ 整除 $b$,记为 $a|b$,同时称 $a$ 是 $b$ 的一个\textbf{因子(factor)}或\textbf{除数(divisor)}。
\end{definition}

\begin{definition}{单位}
对于整环 $R$,如果 $u\in R$ 有\textbf{乘法逆元}$u^{-1}\in R$,那么称 $u$ 是 $R$ 的一个\textbf{单位(unit)}。
\end{definition}

\begin{definition}{真因子}
在整环 $R$ 中,如果对于 $a, b\in R$,存在 $r\in R$ 使得 $b=ar$,并且 $r$ 不是一个单位,那么称 $a$ 是 $b$ 的一个\textbf{真因子(proper factor)}。
\end{definition}

真因子的特点是单向性,如果 $a$ 是 $b$ 的一个真因子,那么绝不可能出现 $b|a$,因为真因子的定义中要求 $b=ar$ 中的 $r$ 是乘法不可逆的。从因子分解的角度,我们可以定义如下等价关系:如果两个元素 $a, b\in R$ 之间只差一个单位因子,即存在单位 $u$ 使得 $a=bu$,那么我们可以把 $a$ 和 $b$ 等价起来。检查一下,如此定义的关系满足等价关系的三个公理,因此是一个等价关系。作为一个例子,整数环中的单位只有两个,$1$ 和 $-1$,因此这个等价关系应用到整数环中就是把所有 $n$ 和其对应的 $-n$ 等价起来。在讨论因子分解时,我们常常使用这个等价划分。

接下来介绍的两个概念,素元素和不可约元素,都是整数中素数概念的推广,只不过选用了素数的不同特点。

\begin{definition}{素元素}
对于整环中的元素 $p\in R$,如果它满足“如果任何 $a, b\in R$ 使得 $p|ab$,则必有 $p|a$ 或者 $p|a$”,则称它为 $R$ 的一个\textbf{素元素(prime element)}。
\end{definition}

\begin{definition}{不可约元素}
对于整环中不是幺元的元素 $p\in R$,如果它在 $R$ 中没有真因子,那么称它为一个\textbf{不可约元素(irreducible element)}。
\end{definition}

对于一般的整环,素元素必然是不可约元素:

\begin{theorem}{素元素必是不可约元素}
给定整环 $R$ 和其素元素 $p$,则 $p$ 必然是不可约元素。
\end{theorem}

\textbf{证明}:

反设 $p$ 可约,则存在 $p$ 的两个真因子 $a, b$ 使得 $p=ab$,从而 $p|ab$。由素元素的定义必须有 $p|a$ 或 $p|b$,而这和“真因子”的单向性矛盾。故设定不成立,即 $p$ 不可约。

\textbf{证毕}。



