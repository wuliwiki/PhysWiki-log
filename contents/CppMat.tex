% C++ 矩阵类的实现
% keys C++|数组|容器|堆栈|矩阵|类|列主序|行主序
% license Xiao
% type Tutor

\pentry{C++ 中类的定义和继承}{nod_df0c}

\addTODO{本文代码还没验证过}

容器指的就是像 \verb`std::vector` 这样的可以储存多个元素的对象, 下面介绍的矢量,矩阵,高维数组等都是容器。

C/C++ 的数组是储存于内存的 stack (栈)中的, 而 stack 一般只有几个 Mb, 一旦数组太大就会产生 stack overflow 错误。 第二是 stack 中的数据的大小都只能在编译时确定(即 constant expression, 例如 literal 或者宏定义), 所以我们不能在运行是确定数组的长度(例如从文件中读取, 或通过运行时的计算得到)。
\begin{lstlisting}[language=cpp]
// 用 C/C++ array 的嵌套定义矩阵
#define N1 5
#define N2 5
double a[N1][N2]; // 定义并在栈中分配内存
// 全部元素赋值为 0
for (long i = 0; i < N1; ++i)
    for (long j = 0; j < N2; ++j)
        a[i][j] = 0;
\end{lstlisting}

所以在一般来说容器都会 allocate (分配)到内存的 heap (堆)中。 在 C++ 中, 在 heap 中分配内存用 \verb`new`, 释放则用 \verb`delete`。

\verb`std::vector` 可以说是使用最广的矢量容器, 理论上我们可以用 \verb`vector` 的 \verb`vector` 定义一个 heap 中的矩阵。
\begin{lstlisting}[language=cpp]
// 用 std::vector 的嵌套定义矩阵
vector<vector<double>> a;
// 分配内存
long N1 = 10, N2 = 10;
a.resize(N1);
for (long i = 0; i < N1; ++i)
    a[i].resize(N2);
// 全部元素赋值为 0
for (long i = 0; i < N1; ++i)
    for (long j = 0; j < N2; ++j)
        a[i][j] = 0;
\end{lstlisting}
注意由于用了多次 \verb`new`, 数据在内存中不一定是连续的, 甚至不一定按 \verb`i` 的顺序排列。 这样不是很好, 为了性能和语法上的考虑, 一般需要给每个容器分配一块连续的内存(例如我们会想要用单个连续索引来获得矩阵元)。

最后一种方法就是小时百科的 \enref{SLISC 库}{SliMat}中的底层实现方法, 即在堆中分配一段连续的内存, 并把双索引转换乘单索引来获取矩阵元。
\begin{lstlisting}[language=cpp]
Long N1 = 10, N2 = 10;
double *a;
new double a[N1*N2];
// 全部元素赋值为 0
for (long i = 0; i < N1; ++i)
    for (long j = 0; j < N2; ++j)
        a[N2*i + j] = 0; // 行主序
        // a[i + N1*j] = 0; // 列主序
// 或者直接使用单索引赋值更快
for (long i = 0; i < N1*N2; ++i)
    a[i] = 1;
delete a[];
\end{lstlisting}
前面两个例子中双索引都是\enref{行主序}{MatSto}的, 即第二个索引增加 1, 内存地址也增加 1。 而这个例子中我们既可以使用行主序也可以使用列主序(第一个索引增加 1, 内存地址增加 1), 还可以使用单索引。 可见这种方式要灵活得多。

注意这只是一个底层的原理, 我们需要把这个功能封装到一个矩阵类 \verb`Mat` 中, 在使用的时候可以达到例如以下效果
\begin{lstlisting}[language=cpp]
MatDoub a(10, 10); // constructor 会执行 new
// 全部元素赋值为 0
for (long i = 0; i < a.n1(); ++i) // n1() n2() 用于获取行数列数
    for (long j = 0; j < a.n2(); ++j)
        a(i, j) = 0;
// 使用单索引
for (long i = 0; i < a.size(); ++i) // size() 用于获取总长度
    a[i] = 1;

// 改变尺寸, 先用 delete 再用 new 重新分配, 数据丢失
a.resize(20, 20);
\end{lstlisting}
为了区分行主序和列主序, 我们可以设计两个类似的矩阵类 \verb`CmatDoub` 是列主序矩阵, \verb`MatDoub` 是行主序矩阵。

我们来看一个简化版的 \verb`CmatDoub` 定义, 不考虑使用 \verb`const CmatDoub`。
\begin{lstlisting}[language=cpp]
class CmatDoub
{
protected:
    int m_N1, m_N2;
    double *p;
public:
    CmatDoub(): m_N1(0), m_N2(0) {}
    CmatDoub(int N1, int N2): m_N1(N1), m_N2(N2) { new double m_p[N1*N2]; }
    Doub& operator()(int i, int j) { return p[m_N2*i + j]}
    int n1() { return m_N1; }
    int n2() { return m_N2; }
    int size() { return m_N1 * m_N2; }
    void resize(int N1, int N2)
    {
        m_N1 = N1; m_N2 = N2;
        if (N1 * N2 != m_N1 * m_N2) {
            delete p[]; new p [N1*N2];
        }
    }
    ~CmatDoub() { delete p[]; }
};
\end{lstlisting}
用这个矩阵类就可以实现上例中的使用方式。
