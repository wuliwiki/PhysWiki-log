% 受阻落体
% 自由落体|阻力|微分方程|分离变量

\pentry{匀加速运动\upref{ConstA}}

在自由落体的基础上, 若假设质点受到的空气阻力的大小与其速度成正比, 比例系数为 $\alpha$, 那么根据牛顿第二定律\upref{New3} 可以列出动力学方程 (假设向下为正方向)
\begin{equation}
ma = F = mg - \alpha v
\end{equation}
考虑到加速度是速度的导数, 上式变为
\begin{equation}\label{RFall_eq2}
\dv{v}{t} = g - \frac{\alpha}{m} v
\end{equation}
这是速度关于时间的函数 $v(t)$ 与其一阶导数 $\dot v(t)$ 的关系式, 即微分方程\upref{ODE}. 与自由落体问题不同的是, 这个方程的右边含有未知函数 $v(t)$, 所以不可能直接将等式两边积分解得 $v(t)$. 我们可以根据微分与导数的关系, 将上式两边同乘 $\dd{t}$ 并整理得
\begin{equation}\label{RFall_eq3}
\frac{1}{g - \alpha v/m} \dd{v} = \dd{t}
\end{equation}
这样我们就得到了 $v$ 和 $t$ 的微分\upref{Diff}关系, 即每当 $t$ 增加一个微小量时, 如何求 $v$ 对应增加的微小量. 注意等式左边仅含 $v$, 右边仅含 $t$, 所以这一步叫做\textbf{分离变量}, 我们称\autoref{RFall_eq2} 为\textbf{可分离变量}的微分方程. 假设 $v$ 和 $t$ 之间的关系可以表示为
\begin{equation}\label{RFall_eq4}
F(v) = G(t)
\end{equation}
那么对等式两边微分即可得到\autoref{RFall_eq3} 的形式. 令 $f(v)$ 和 $g(t)$ 分别为 $F(v)$ 和 $G(t)$ 的导函数, 有
\begin{equation}\label{RFall_eq5}
f(v) \dd{v} = g(t) \dd{t}
\end{equation}
对比\autoref{RFall_eq3} 可得 $f(v) = 1/(g - \alpha v/m)$ 和 $g(t) = 1$, 把二者做不定积分\upref{Int} 得原函数. 首先显然 $G(t) = t + C_1$. 对 $f(v)$ 积分可用“积分表\upref{ITable}” 中的\autoref{ITable_eq1} 和 \autoref{ITable_eq10} 得
\begin{equation}
F(v) = -\frac{m}{\alpha} \ln\abs{g - \frac{\alpha}{m} v} + C_2 = -\frac{m}{\alpha} \ln(g - \frac{\alpha}{m} v) + C_2
\end{equation}
 上式中绝对值符号可去掉是因为在\autoref{RFall_eq2} 中根据物理情景可知 $\dv*{v}{t}$ 始终大于零. 把两原函数代回\autoref{RFall_eq4} (这时可以把 $C_1$ 和 $C_2$ 合并为一个待定常数 $C$), 整理可得
\begin{equation}
v = \frac{m}{\alpha} \qty( g - \E^{-\alpha C/m} \E^{-\alpha t/m} )
\end{equation}
这就是微分方程\autoref{RFall_eq2} 的通解, 可代入原微分方程以验证是否成立. 以后我们把以上这种由\autoref{RFall_eq5} 形式求\autoref{RFall_eq4} 形式的步骤简称为“对方程两边积分”. 由于方程阶数为 1, 通解仅含有一个待定常数. 为了确定这个待定常数, 我们用题目给出的\textbf{初值条件}, 即 $t = 0$ 时 $v = 0$, 代入通解可解得 $C$, 再把 $C$ 代回通解得满足初始条件的\textbf{特解}
\begin{equation}
v(t) = \frac{mg}{\alpha} \qty( 1- \E^{-\alpha t/m} )
\end{equation}
从该式可以看出, 当 $t = 0$ 时, 质点速度为 0, 符合初始条件, 而当 $t\to +\infty$ 时, $ v(t) \to mg/\alpha$. 可见质点的速度会无限趋近一个最大值, 而这个最大值恰好可以使阻力 $\alpha v$ 等于重力 $mg$. 利用这一条件, 即使不解微分方程, 也可以很快算出质点的末速度.

