% 核裂变
% license CCBYSA3
% type Wiki

(本文根据 CC-BY-SA 协议转载自原搜狗科学百科对英文维基百科的翻译)

\begin{figure}[ht]
\centering
\includegraphics[width=6cm]{./figures/7e648b4500b5acbc.png}
\caption{诱发裂变反应。铀-235吸收一个中子后会转化成处于激发态的铀-236,激发能来自中子的动能加上原子核与中子的结合能。铀-236会分裂成高速运动的轻元素(裂变产物),并释放少量自由中子。同时会产生一个或多个“瞬发γ射线”(图中未标出)。} \label{fig_HLB_12}
\end{figure}

在核物理和核化学中,\textbf{核裂变}是指核反应或放射性衰变过程中原子核分裂成更小、更轻的核的现象。裂变过程通常产生自由中子和γ 光子,同时释放出大量的能量。

重元素核裂变于1938年12月17日由德国人奥托·哈恩和他的助手弗里茨·施特拉斯曼发现,1939年1月莉泽·迈特纳和她的侄子奥托·弗里施给出了理论解释。弗里希将这一过程比喻为生物细胞的分裂。重核素的裂变是一个放热反应,会以电磁辐射和裂变碎片的动能形式释放大量能量。为了使裂变过程释放能量,裂变产物元素的总结合能应该大于起始元素的结合能。

裂变是一种核嬗变,因为裂变产物与初始原子属于不同的元素。裂变产生的两个核的质量通常比较接近,对于一般的可裂变同位素,其裂变产物的质量比约为3比2。[1][2]大多数裂变是二元裂变(产生两个带电碎片),但偶尔(每1000次事件发生2到4次)会发生三元裂变,产生三个带正电荷的碎片。三元裂变过程中最小的裂变碎片大小可位于质子到氩核之间。

除了已经被人类开发利用的中子诱导裂变之外,还存在另一种形式的裂变,称为自发放射性衰变(不需要中子诱导),该现象易发生在具有较高质量数的同位素中。自发裂变于1940年由弗廖罗夫、彼得扎克和库尔恰托夫[3]在莫斯科发现,当时他们决定通过实验验证尼尔斯·玻尔作出的一个预测,即没有中子轰击时,铀几乎不发生裂变,然而实验结论正相反。[3]

产物组成的不可预测性(产物可能的种类很多且无规律性)将裂变与量子隧穿过程区分开来,如质子发射、α衰变和团簇衰变等量子隧穿过程每次的产物都是相同的。核裂变是核电站及核武器的能量来源。作为核燃料的物质在被裂变中子撞击时会发生裂变,而在它们裂变过程中又会发射中子。这使得自持的核链式反应成为可能,这种核链式反应可以在核反应堆中以受控的速率释放能量,或者在核武器中以非常快速、不受控制的速率释放能量。

核燃料中包含的自由能是同等质量的化学燃料(如汽油)的数百万倍,这使得核裂变成为一种非常高效的能源。然而,核裂变的产物的放射性通常比作为裂变燃料的重元素高得多,并且其半衰期相当长,导致了核废料的问题。对核废料积聚和核武器的潜在破坏力的担忧影响了人们和平利用裂变作为能源的愿望。

\subsection{ 物理概述}
\subsubsection{1.1 机制}

\textbf{放射性衰变}

核裂变可以在没有中子轰击的情况下发生,这是一种放射性衰变。这种类型的裂变(称为自发裂变)除了少数重同位素之外很罕见。
\begin{figure}[ht]
\centering
\includegraphics[width=8cm]{./figures/b2a24667fa77a86b.png}
\caption{一种中子诱导核裂变事件的视觉演示,其中速度较慢的中子被铀-235原子的原子核吸收,原子核分裂成两种快速运动的较轻元素(裂变产物)和额外的中子。释放的大部分能量以裂变产物和中子的动能的形式存在。} \label{fig_HLB_1}
\end{figure}

\textbf{核反应}

在工程核设施中,基本上所有的核裂变都是以“核反应”形式发生的,其产生于轰击过程中亚原子的碰撞。在核反应中,亚原子粒子与原子核发生碰撞并使其发生变化。核反应是由轰击机制驱动的,与自发放射性衰变过程不同,后者具有相对稳定的指数衰减规律和特征半衰期。

人们目前已经发现许多种核反应。核裂变与其他类型的核反应有很大不同,因为它可以通过核裂变链式反应被放大并控制。在核裂变链式反应中,每个裂变事件释放的自由中子可以触发更多的裂变过程,这反过来又释放更多的中子并导致更多的裂变。

能够维持裂变链式反应的元素同位素被称为核燃料,我们称该种核素是可裂变的。最常见的核燃料是235U(铀的同位素,原子量为235,常用于核反应堆中)和239Pu(钚的同位素,原子量为239)。这些燃料的裂变产物的原子质量呈成双峰分布,峰值分布在95u和135u附近。大多数核燃料的自发裂变非常缓慢,而是主要通过α-β衰变链,其过程可长达千年至数万年甚至更久。在核反应堆或核武器中,绝大多数裂变事件是由中子轰击引起的,这些中子本身是由先前的裂变事件产生的。

裂变燃料中的核裂变是易裂变核素俘获中子时产生的核激发能的结果,这些能量来自于中子和原子核之间相互吸引的核力,该能量使原子核变形为双瓣状的“液滴”,原子核的“两叶”均带正电荷,当两叶之间的距离超过核力能够维持其不分离的范围时,两个裂变随便就完成了分离,进而被相互排斥的电荷进一步分开距离越来越远,因此这是一个不可逆过程。可裂变同位素(如铀-238)中也会发生类似的过程,但这些同位素需要由快中子(如热核武器中核聚变产生的中子)提供额外的能量才可以发生裂变。

根据原子核的液滴模型,核裂变产物应具有相同的原子量。通过更复杂的核壳层模型可以从机理上解释为何通常一种裂变产物比另一种稍小。玛丽亚·格佩特·梅耶提出了一种基于核壳层模型的裂变理论。

最常见的裂变过程是二元裂变,如上文所述,两个裂变产物的原子量通常分布在在95±15和135±15 u 区间。二元裂变发生的概率最大,而在核反应堆中,每1000次裂变事件中,还会发生2到4次三元裂变,三元裂变的过程产生三个带正电荷的碎片(加上中子),其中最小的裂变碎片的质量范围可以在质子(原子质量Z=1)至氩(原子质量Z=18)之间。最常见的小碎片由90\%的氦-4核组成,其能量高于α衰变产生的α粒子(即所谓“长程α粒子”,能量约16MeV),加上氦-6和氚核。三元裂变不太常见,但最终仍会在核反应堆的燃料棒中产生大量氦-4和气体氚。[4]
\begin{figure}[ht]
\centering
\includegraphics[width=8cm]{./figures/6ba36d8f2d7b9973.png}
\caption{U-235 、Pu-239 (当前核电反应堆中两种典型)和 U-233 (用于钍增殖循环)热中子诱导裂变产物的质量分布。} \label{fig_HLB_2}
\end{figure}

\subsubsection{1.2 能量学}
\textbf{能量输入}

在最常见的过程二元裂变过程中(产生两个带正电荷的裂变产物+中子),重核的裂变需要大约7-8MeV来克服将原子核保持为球形或近似球形的核力,并使其变形为双瓣形(类似“花生”),接着两瓣在正电荷的排斥力下继续彼此分离,一旦裂变碎片被推到临界距离外,短程强相互作用核力就不能再将它们保持在一起,它们的分离过程从碎片之间的(远程)电磁排斥开始,最终以两个高能裂变碎片相互远离结束。

裂变输入能量中约6MeV是由一个中子通过强力与重元素原子核结合的过程提供;然而,在许多可裂变核素中,这一数量的能量不足以产生裂变。例如,对于能量小于1MeV的中子,铀-238的裂变反应截面接近零。如果没有任何其他机制提供额外的能量,原子核不会裂变,而只会吸收中子,就像铀-238可以吸收慢中子的甚至部分快中子而变成铀-239。引发裂变所需的剩余能量可以由另外两种机制提供:其中一种是提高入射中子的动能,中子动能超过1MeV(称为快中子)时,随着中子能量增加可裂变重核素发生裂变的可能性随之增加。这种高能中子能够直接使铀-238发生裂变。然而,这一过程在核反应堆中不可能大量发生,因为任何类型裂变产生的裂变中子中只有很小一部分具有足够的能量使铀-238有效裂变(裂变中子的最概然能量为2 MeV,但能量中值仅为0.75 MeV,这意味着其中一半的中子小于这一能量)。[5]

在锕系元素中,那些具有奇中子数的核素(例如具有143个中子的铀-235)与具有偶中子数的相同元素的同位素(例如具有146个中子的铀-238)相比,俘获一个额外中子释放的结合能要多1到2 MeV。这些额外的结合能来自中子配对效应,泡利不相容原理允许俘获中子占据与原子核中最后一个中子相同的核轨道,从而使两者形成一对。在这样的核素中,裂变对中子动能没有要求,因为所有必要的能量都可以通过吸收中子的结合能提供,无论是慢中子还是快中子(前者用于慢中子核反应堆,后者用于快中子反应堆和武器)。如上所述,可裂变元素中有一部分可以利用它们自己的裂变中子产生裂变(从而可以使用相对少量的物质即可发生核裂变链式反应),这些元素被称为“易裂变元素”。易裂变同位素的例子有铀-235和钚-239。
\begin{figure}[ht]
\centering
\includegraphics[width=6cm]{./figures/6eaac8000a9e515f.png}
\caption{液滴模型解释二元裂变过程。能量输入使原子核变形为“雪茄”形,然后是“花生”形,接着由于两部分之间的距离超过了短程核力吸引范围而分裂,然后被电荷斥力推开。液滴模型预言两个裂变碎片具有相同大小。而如通常实验观察到的,核壳层模型理论允许它们大小不同。} \label{fig_HLB_3}
\end{figure}

\textbf{能量输出}

典型的裂变事件释放大约两亿电子伏特(200 MeV),相当于大约2万亿开尔文。无论是可裂变元素还是易裂变元素,发生裂变的同位素种类对对释放的能量只有很小的影响。这可以通过结合能曲线(下图)看出,注意到锕系核素的平均结合能约为每个核子7.6 MeV,从曲线左侧可以看出裂变产物的结合能趋于每个核子8.5 MeV左右。因此,在锕系元素质量范围内的任何同位素的裂变事件中,起始元素的每个核子大约释放0.9 MeV能量。铀-235被慢中子诱发裂变释放的能量几乎与铀-238被快中子诱发裂变释放的能量相同。这种能量释放规律对钍和各种次锕系元素也适用。[6]

相比之下,大多数化学物质氧化反应(如煤或TNT的燃烧)单个反应最多释放几个eV 的能量。所以,核燃料的单位质量可用能量比普通化学燃料大至少一千万倍。核裂变的能量以裂变产物和碎片的动能及电磁辐射(γ射线)的形式释放。在核反应堆中,这些能量通过热粒子、γ射线与反应堆工作物质的原子碰撞而转化为热能,工作物质通常为水,部分为重水或者熔盐.
\begin{figure}[ht]
\centering
\includegraphics[width=6cm]{./figures/65e0bb7b5fd8acc1.png}
\caption{库仑爆炸动画,演示正电荷原子核簇的情况,类似于裂变碎片簇。色调与原子核电荷成正比。在这个时间尺度上的电子(较小的点)只能通过频闪观测到,色调水平与其动能成正比。} \label{fig_HLB_4}
\end{figure}
当一个铀核分裂成两个子核碎片时,铀核质量的0.1\%[7]转化为裂变能量,约200MeV。对于铀-235(总平均裂变能量202.79 MeV[8]),通常约为169 MeV表现为子核的动能,由于库仑排斥作用,子核分离的速度约为光速的3\%。此外,每次裂变平均发射2.5个中子,每个中子的平均动能约为2 MeV(总计4.8 MeV)。[9]裂变反应也释放出约7MeV的瞬发伽马射线,也就是说核裂变爆炸或临界事故中约3.5\%的能量以伽马射线形式释放,不到2.5\%的能量以快中子动能形式释放(两种辐射站总能量的约6\%),其余能量作为裂变碎片的动能(碎片的能量几乎瞬间就通过与周围物质的撞击转化为热能)。[10][11]在原子弹中,这种热量可能会使炸弹核心的温度升高到1亿开尔文,并通过产生次级软X射线将部分能量转换为电离辐射,而在核反应堆中,裂变碎片动能主要产生低温热,本身很少或部产生电离辐射。

所谓的中子弹(增强辐射武器)以电离辐射(特别是中子)的形式释放更大比例的能量,但这些都是依靠核聚变阶段产生额外辐射的热核装置,在纯核裂变武器中,辐射能量占总能量的比例总是在6\%附近。

每次核裂变反应瞬时释放能量约181 MeV,为全时段释放总能量的89\%。剩余约11\%的能量以裂变产物的具有各种半衰期的β衰变以及与这些β衰变相关联的延迟γ射线形式释放。例如,在铀-235中,这些延迟能量包括约6.5MeV的β射线,8.8MeV的反中微子(与β射线同时产生),以及约6.3MeV的处于激发态的β衰变产物的延迟伽马射线中(平均每次裂变约产生10次伽马射线发射)。因此,裂变总能量的大约6.5\%在事件发生后的某个时间作为非即时或延迟电离辐射释放,延迟电离能量几乎被伽马射线和β射线平分。

在已经运行一段时间的反应堆中,放射性裂变产物将积累到稳态浓度,使得它们的衰变率等于它们的形成率,从而它们对反应堆热量(通过β衰变)的贡献比例与这些辐射能量占裂变能量的比例相同。在这些条件下,延迟电离辐射(来自放射性裂变产物的延迟γ和β辐射)对反应堆稳态热功率的贡献约为6.5\%。当反应堆突然关闭(急停)时,剩余的热量输出就来源于此。因此,一旦反应堆关闭时,反应堆衰变热输出的初始值为全稳态裂变功率的6.5\%。然而,在几小时内,由于这些同位素的衰变,衰变功率会变小很多。

剩余的延迟能量(8.8MeV/202.5MeV=总裂变能量的4.3\%)以反中微子形式释放,反中微子不被认为是“电离辐射”,原因是反中微子不会被反应堆捕获而产生热量,而是以接近光速直接穿过所有材料(包括地球)逃逸到行星际空间(吸收的量极小)。中微子辐射通常不被归类为电离辐射,因为它几乎完全不被吸收,因此不会产生电离效应(尽管有极小概率发生中微子电离事件)。而几乎所有其余的辐射(6.5\%延迟的β和γ辐射)最终都在反应堆堆芯或其屏蔽层中转化为热量。

一些涉及中子吸收或最终产生能量的过程非常重要,如当中子被铀-238原子捕获并产生钚-239时,中子动能不会立即产生热量,但是如果钚-239后续发生裂变,则会释放这些能量。另一方面,从裂变子产物发射的半衰期最高达几分钟的缓发中子对反应堆控制非常重要,因为当核反应在缓发临界区进行时,要依赖这些中子进行超临界链式反应(其中每个裂变循环产生的中子多于吸收的中子),因此这些中子会确定一个核反应规模翻倍的特征时间。如果没有它们的存在,核链式反应将是瞬发临界的,其反应规模的增加比人类干预所能控制的要快。在这种情况下,第一个实验性原子反应堆在操作人员能够手动关闭之前,就可能已经跑到了危险和混乱的“瞬发临界反应”状态(为此,设计者恩利克·费密引入了由电磁驱动的的辐射计数触发控制棒,它可以自动落入芝加哥一号堆的中心)。如果这些缓发中子被俘获而不产生裂变,它们最终也会产生热量。[12]

\subsubsection{1.3 裂变产物和结合能}
在裂变中,倾向于产生质子数为偶数的碎片,这被称为裂变碎片电荷分布的奇偶效应。然而,裂变碎片\textbf{质量数}分布没有奇偶效应。这一结果归因于核子对断裂。

在核裂变事件中,原子核可以分裂成较轻原子核的不同组合,但最常见的不是分裂成质量大约相等的两个质量数约为$120\mathbf{u}$的原子核,而是裂变为质量稍有区别的两个部分(取决于裂变核素和过程),其中一个子核的质量约为90至$100\mathbf{u}$, 另一个约为130到$140\mathbf{u}$。[13]不相等的裂变在能量上更有利,因为这允许一个产物的质量数更接近能量最低的60 $\mathbf{u}$(约为裂变元素质量的四分之一),而另一个原子核的质量为135 $\mathbf{u}$仍然很接近最紧密结合原子核的质量范围(换句话说,原子质量120处左侧的结合能曲线比右侧稍陡)。

\subsubsection{1.4 活化能和结合能曲线的来源}
重元素的核裂变能够产生可利用的能量是因为原子序数和原子量接近Ni-62和铁-56附近的中等质量原子核具相较更重的核素具有更大的平均结合能(每个核子结合能的平均值),所以当重的原子核被分裂时能量被释放出来。裂变产物的质量和($\mathbf{Mp}$)的质量小于初始裂变燃料核的质量($\mathbf{M}$)。根据质能等效性公式 $\mathbf{E=mc^2}$,超过部分的质量 $\mathbf{\delta m = M - M_p}$ 以光子(伽马射线)和裂变碎片的动能的形式转化为能量释放。
\begin{figure}[ht]
\centering
\includegraphics[width=8cm]{./figures/91b38fc2d7dd6b20.png}
\caption{“结合能曲线”:普通元素素每个核子的结合能图。} \label{fig_HLB_5}
\end{figure}
平均结合能随原子序数的变化源于组成原子核基本粒子(质子和中子)间的两种基本相互作用力,即原子核核子间相互吸引作用的核力以及质子间的静电排斥力。核力只在相对较短的范围内起作用(几个核子直径),其遵循指数衰减的汤川势这使得它在较大距离时可以忽略。静电排斥的范围更大,因为它以平方反比定律衰减,当原子核大于约12 核子直径时总静电斥足以克服核力使它们自发地不稳定。出于同样的原因,较大的原子核(直径超过大约八个核子)相比较小的原子核每个核子的结合更不紧密;将一个大原子核分裂成两个或更多的中等大小的原子核会释放能量。

也因为强相互作用核力的作用范围很短,较大的稳定原子核必须比最轻的元素具有更多比例的中子,最轻的元素在中子与质子比例为1:1时最稳定。而质子数超过20的原子核不可能是稳定的,除非它们的中子数超过质子数。额外的中子增加了强相互作用核力(作用于所有核子之间),而没有增加质子-质子排斥,从而使重元素更稳定。裂变产物的中子和质子的平均比例与它们的母核大约相同,因此容易发生β衰变(β衰变将中子变成质子),因为与类似质量的稳定同位素相比,它们的中子比例太多。

裂变产物核产生β衰变的趋势是核反应堆产生的高放废物的根本原因。裂变产物往往是β发射体,能够发射高速电子,因为裂变产物原子中的中子转化为质子过程中要保持电荷守恒。

\subsubsection{1.5 链式反应}
\begin{figure}[ht]
\centering
\includegraphics[width=6cm]{./figures/697673c2e47ba806.png}
\caption{核裂变链式反应的示意图。1. 一个铀-235 原子吸收一个中子,并裂变成两个新原子(裂变碎片),释放三个新中子和一些结合能。2. 其中一个中子被铀-238 原子吸收,不会继续反应。另一个中子逃逸了而没有与任何东西碰撞,也没有继续反应。一个中子与铀-235原子碰撞,然后铀-235原子裂变并释放两个中子和一些结合能。3. 这两个中子都与铀-235原子碰撞,每个原子裂变并释放一至三个中子,使反应可以持续进行。} \label{fig_HLB_6}
\end{figure}
几种重元素,如铀、钍和钚,能够发生自发裂变(以放射性衰变形式)和诱发裂变(以核反应形式)。能够被自由中子诱发裂变的核素称为可裂变的;能够被速度较慢的热中子诱发裂变的核素也被称为易裂变的。一些特别易裂变并容易获得的核素(特别是铀-233、铀-235和钚-239)被称为核燃料,因为它们可以维持链式反应,并且产量足够多。

所有可裂变和易裂变核素都会发生少量的自发裂变反应,向燃料样品中释放少量的自由中子。这些中子会迅速从燃料中逃逸出来,成为自由中子,平均寿命大约15分钟,随后衰变为质子和β粒子,然而几乎所有中子都会在这之前被附近的其他原子核吸收(新产生的裂变中子以大约光速的7\%移动,即使慢化中子速度也达到8倍声速)。一些中子会撞击核燃料原子并引发进一步的裂变,释放更多的中子。如果有足够多的核燃料组件,或者逃逸中子被充分容纳,则新产生中子的数量超过了从组件逃逸出的中子,可持续的核裂变链式反应就产生了。

支持可持续核链式反应的组件称为临界装置,如果组件几乎完全由核燃料制成,则称为临界质量。“临界”一词指决定燃料自由中子的数量的微分方程的尖点:燃料小于临界质量时,则中子的数量主要由放射性衰变决定,大于临界质量时,则中子数量由链式反应决定。实际应用中核燃料的临界质量很大程度上取决于其几何形状和周围的材料。

并非所有可裂变同位素都能维持链式反应。例如,铀-238是铀含量最高的同位素,它是可裂变的,但不是易裂变的:当受到1MeV以上的高能中子撞击时诱发裂变。然而,铀-238裂变产生的中子中只有极少量有足够的能量引发再次裂变,所以这种同位素不可能发生链式反应。而用慢中子轰击铀-238可以使其吸收中子变为铀-239并通过β发射衰变为镎-239再通过相同的过程衰变为钚-239;这就是增殖反应堆中的生产钚-239的过程。因为钚-239也是易裂变的核燃料,因此在产生足够的钚-239后,原位钚的产生也有助于维持其他类型反应堆中的中子链式反应。据估计,在燃料的整个循环周期中,一个标准“非增殖”反应堆产生的功率高达一半是由钚-239裂变产生的。

即使没有链式反应,可裂变的非易裂变核素也可以用作裂变能量源。快中子轰击铀-238可诱导裂变,只要外部中子源存在,就会释放能量。这在所有反应堆中都有重要的影响,在这些反应堆中,来自易裂变核素的快中子会导致附近的铀-238发生裂变,这意味着核燃料中的少量铀-238都会被“燃烧”,特别是在使用高能中子的快中子增殖反应堆中。同样的裂变效应被用来增加热核武器的威力,方法是给核武器加装铀-238套层使其与核聚变释放的中子反应。但是如果使用具有慢化作用的材料会降低次级中子速度从而减少核裂变链式反应的爆炸威力。

\subsubsection{1.6 裂变反应堆}
\begin{figure}[ht]
\centering
\includegraphics[width=8cm]{./figures/a932264247bfb744.png}
\caption{德国 菲利普斯堡(巴登-符腾堡州) 核电厂的冷却塔。} \label{fig_HLB_7}
\end{figure}
临界裂变反应堆是最常见的核反应堆。在临界裂变反应堆,燃料原子裂变产生的中子被用来诱导更多的裂变,以维持可控的能量释放。产生工程化但非自持的裂变反应的装置是亚临界裂变反应堆。这种装置使用放射性衰变或粒子加速器来触发裂变。

临界裂变反应堆主要用于三个目的,工程上通常要进行不同的权衡,以利用裂变链式反应产生的热量或中子:
\begin{itemize}
\item 动力反应堆旨在为提供热量以供发电,可以作为发电站或核潜艇等电力系统的核岛部分。
\item 研究反应堆旨在为科学、医学、工程或其他研究目的产生中子和/或活化放射源。
\item 增殖反应堆旨在利用更丰富的同位素大量生产核燃料。如典型的快中子增殖反应器能够利用自然界含量丰富的铀-238(不是核燃料)生产钚-239(一种核燃料)从。热增殖反应堆曾试验使用钍-232来生产易裂变同位素铀-233(钍燃料循环),该技术正在被研究和开发。
\end{itemize}
虽然原则上所有裂变反应堆都可以满足上述三种需求,但在实践中,这些任务会导致工程目标的冲突,大多数反应堆在设计建造时只考虑了上述任务之一。(有几个早期的反例,如汉福德反应堆,现已退役)。动力反应堆通常将裂变产物的动能转化为热量,用于加热工作流体并驱动热机产生机械能或电能。工作流体通常是水并采用蒸汽透平,但有些设计使用其他物质气体氦。研究堆产生用途广泛的中子,裂变热则未被利用。增殖反应堆是一种特殊形式的研究反应堆,但需要注意的是,被辐照的样品通常是燃料本身,即铀-238和铀-235的混合物。

\subsubsection{1.7 裂变炸弹}
\begin{figure}[ht]
\centering
\includegraphics[width=6cm]{./figures/1f15cfdd92ea3e22.png}
\caption{1945年8月9日,日本长崎投下原子弹升起的蘑菇云,在爆炸点上方超过18公里高。据估计有39000人被原子弹炸死,[11]其中23145-28113人为日本工厂工人,2000人为韩国奴隶劳工,150人为日本战斗人员。[12][13][14]} \label{fig_HLB_8}
\end{figure}
裂变炸弹(注意与聚变炸弹区分)是核武器的一种,也称为原子弹,是一种设计用于在释放的能量导致反应堆爆炸(和链式反应停止)之前,尽快释放尽可能多的能量的核反应装置。发展核武器是早起进行核裂变研究的背后动机,在第二次世界大战(1939年9月1日-1945年9月2日)期间美国通过曼哈顿计划进行了大部分关于裂变链式反应的早期科学工作,最终导致了战争期间发生的三起涉及裂变炸弹的事件。第一颗代号为“小玩意”的裂变炸弹在1945年7月16日在美国新墨西哥州沙漠中进行的三位一体核试中被引爆。另外两枚代号为“小男孩"和"胖子”,分别在1945年8月6日和9日的战斗中被投放到日本城市广岛和长崎。

第一颗裂变炸弹的爆炸性就达到相同质量的化学炸药的数千倍。例如,小男孩总共重约4吨(其中60kg是核燃料)、11英尺(3.4米)长,其爆炸当量相当于大约一万五千吨三硝基甲苯(TNT),毁灭了广岛市的大部分。现代核武器(包括热核聚变弹以及一级或多级裂变弹)的能量是第一批纯裂变原子弹的数百倍,因此现代的单枚核导弹弹头重量不到小男孩的1/8(例如W88),爆炸当量为47.5万吨TNT,可破坏城市面积比后者大十倍。

虽然核武器中裂变链式反应的基本物理原理与可控核反应堆相似,但这两种类型的装置必须设计得非常不同。核弹被设计成一次释放所有能量,而反应堆被设计成持续产生稳定的能量。虽然反应堆过热由可能并曾经导致堆芯熔化和蒸汽爆炸,但由于其燃料的铀浓度相对更低使得核反应堆不可能产生与核武器相同破坏力的爆炸。同样,从核弹中获取有用的能量也很困难,尽管已有一个在研发的火箭推进系统“猎户座计划”试图通过在一个充分防护和屏蔽的航天器后面引爆裂变炸弹来进行推进。

核武器的战略重要性是核裂变技术具有政治敏感性的主要原因。从工程角度来看,可行的裂变炸弹设计相对简单,对于很多国家来说并非难事。然而,获得裂变核材料以实现其设计方案是相对困难的,这是除了拥有生产裂变材料的特别工艺的部分现代工业化国家之外,其他所有国家都没有核武器的关键。

\subsection{历史}
\subsubsection{2.1 核裂变的发现}
核裂变于1938年在凯泽·威廉化学学会(今天是德国柏林自由大学的一部分)的大楼里被发现,此前该学会在放射性科学以及阐述原子成分的核物理学方面进行了近50年的工作。

在1911年,欧内斯特·卢瑟福提出了一个原子模型,包括一个非常小、致密且带正电荷(由于质子的缘故)的原子核以及围绕其运行的带负电的电子组成。[14] 尼尔斯·玻尔在1913年考虑电子的量子行为对其进行了改进。昂利·贝可勒耳、玛丽·居里、皮埃尔·居里和卢瑟福的工作进一步揭示,虽然原子核是紧密结合的,但可以发生不同形式的放射性衰变从而转化为其他元素。(例如α衰变:发射α粒子——由两个质子和两个中子结合在一起的一个与氦核相同的粒子。)

在核嬗变方面已经做了一些研究工作。1917年,卢瑟福利用α粒子轰击氮实现了氮向氧的转化:$^{14}N + \alpha \to ^{17}O + p$。 这是第一次在试验室观察到核反应,即一个衰变产生的粒子被用来转化另一个原子核的反应。最终,在1932年,卢瑟福的同事欧内斯特·瓦耳顿和约翰·考克饶夫实现了完全人工的核反应和核嬗变,他们使用加速的质子撞击锂-7,将其分裂成两个$\alpha$粒子。这一壮举被称为“分裂原子”试验,并为他们赢得1951年的诺贝尔物理学奖,获奖原因是“人工加速的原子粒子对原子核的转化”,虽然这并不是下文所说的在重元素中发现的现代核裂变反应。[15]

同时为了理解恒星的能量来源,关于核聚变的可能性开始被研究。第一个人工聚变反应是由马克·奥利芬特在1932年完成的,他使用两个加速的氘核(由一个质子与一个中子组成)产生一个氦-3核。[16]

英国物理学家詹姆斯·查德威克在1932年发现中子后,[17] 恩利克·费密和他在罗马的同事在1934年研究了用中子轰击铀的结果。[18]费米的结论是,他的实验创造了带有93和94个质子的新元素,并将其命名为 “ausonium”和“hesperium” 。然而,并非所有人都认可费米对其结果的分析,尽管他因为“展示了中子辐照产生的新放射性元素的存在,以及相关的关于慢中子引发核反应的发现”而赢得1938年诺贝尔物理学奖。

德国化学家伊达·诺达克在1934年出版的著作中特别提出,与其说是创造了一种新的、更重的元素93,不如说“原子核分裂成几个大碎片是可信的”[19][20]然而,诺达克的结论当时并没有被采纳。
\begin{figure}[ht]
\centering
\includegraphics[width=8cm]{./figures/3af37715ba994fd0.png}
\caption{奥托·哈恩和弗里茨·施特拉斯曼在1938年发现核裂变的实验装置} \label{fig_HLB_9}
\end{figure}
费米的结果发表后,奥托·哈恩、莉泽·迈特纳和弗里茨·施特拉斯曼开始在柏林进行类似的实验。1938年3月“德奥合并”事件发生,奥地利被占领并并入纳粹德国,奥地利犹太人迈特纳于失去了的公民身份,但她于1938年7月逃到瑞典,并开始通过邮件与柏林的哈恩通信。巧合的是,她的侄子奥托·弗里希也是难民并且当时正在瑞典,迈特纳收到哈恩12月19日的来信,信中描述了他的化学试验证明了用中子轰击铀的一些产物是钡。哈恩认为原子发生了分裂但他还不知道其物理基础是什么。钡的原子质量比铀小40\%,以前已知的放射性衰变理论无法解释原子核质量如此大的差异。弗里希对此持怀疑态度,但迈特纳相信哈恩作为化学家的能力。玛丽·居里多年来一直在从镭中分离钡,这种技术是非常成熟的。根据弗里希的说法:

这是个错误吗?不,莉泽·迈特纳说;哈恩是一个很好的化学家。但是钡是如何由铀形成的呢?没有比质子或氦核(α粒子)更大的碎片被从原子核中分离过,并且几乎没有足够的能量从原子核中剥离如此大的一块质量。铀核也不可能被劈开。原子核不像易碎的固体,可以被劈开或打碎;乔治·盖莫夫很早就提出原子核更像是一个液滴,且玻尔也给出了很好的论据。也许一个液滴可以以更平缓地方式将自己分成两个更小的液滴,首先变长,然后收缩,最后被扯断而不是被打碎?我们知道有强作用力可以阻止这一过程,就像普通液滴的表面张力倾向于阻止它分裂成两个更小的液滴一样。但是原子核有一个重要区别:它们是带电的,电荷斥力可以抵消表面张力的作用。

我们发现,铀原子核的电荷大到集合可以完全克服表面张力,因此铀核可能确实像一个非常不稳定的水滴,随时可以在最轻微的扰动下分裂,比如一个中子的撞击。但还有一个问题,分离的两个水滴由于电荷斥力相互高速分开从而获得非常大的能量,总计约200MeV,这些能量从何而来?...莉泽·迈特纳...计算出,铀核分裂形成的两个原子核的质量比原来的铀核轻大约五分之一质子的质量。根据爱因斯坦的质能公式$\mathbf{E=mc^2}$,消失的五分之一质子质量正相当于200MeV。这就是能量的来源,一切都说通了。

简而言之,迈特纳和弗里希正确地解释了哈恩的结果,认为铀核分裂成两半。弗里希建议将这一过程命名为“核裂变”,类似于活细胞分裂成两个细胞的过程,后来被称为二元裂变。正如“链式反应”这个术语借用于化学,“裂变”这个术语是从生物学中借用的。

1938年12月22日,哈恩和斯特拉斯曼把手稿寄给了《自然科学》,称他们在用中子轰击铀后发现了元素钡。[21]同时,他们向瑞典的迈特纳传达了这些结果。她和弗里希正确地将结果解释为核裂变。[22]弗里希在1939年1月13日通过实验证实了这一点 。[23][24]为了证明用中子轰击铀产生的钡是核裂变的产物,哈恩于1944年“因发现重核的裂变”而被授予诺贝尔化学奖(唯一获奖者)。(这个奖项实际上是在1945年颁发给哈恩的,因为“诺贝尔化学委员会认为今年的提名没有一个符合阿尔弗雷德·诺贝尔遗嘱中概述的标准。”在这种情况下,诺贝尔基金会的规定允许将当年的奖金保留到下一年。)[25]
\begin{figure}[ht]
\centering
\includegraphics[width=8cm]{./figures/1c4929e8c90bc411.png}
\caption{纪念奥托·哈恩发现核裂变的德国邮票(1979年)} \label{fig_HLB_10}
\end{figure}
新发现的消息迅速传播开来,这被认为是一种全新的物理效应,具有巨大的科学以及潜在的实用潜力。迈特纳和弗里希对哈恩和斯特拉斯曼发现的解释同尼尔斯·玻尔一起跨越了大西洋,他将在普林斯顿大学演讲。拉比和威利斯·兰姆,两位在普林斯顿工作的哥伦比亚大学物理学家,听到了这个消息并把它带回了哥伦比亚大学。拉比说他告诉了恩利克·费密;而费米说是兰姆告诉他的。玻尔随后从普林斯顿到哥伦比亚去见费米。玻尔在办公室里没有找到费米,他去了回旋加速器区域,发现了赫伯特·安德森。玻尔抓住他的肩膀说:“年轻人,让我给你解释一些新的令人兴奋的物理现象。”[26]哥伦比亚大学的一些科学家很清楚,他们应该尝试检测中子轰击铀核裂变释放的能量。1939年1月25日,哥伦比亚大学的一个小组在美国进行了第一次核裂变实验,[27]这是在普平物理试验室的地下室完成的;小组成员有赫伯特·安德森、尤金·布斯、约翰·邓宁、恩利克·费密、诺里斯·格拉索和弗朗西斯·斯洛克。实验包括将氧化铀放入电离室,用中子照射,并测量由此释放的能量。结果证实裂变正在发生,并有强烈证据暗示发生裂变的是铀-235 。第二天,在乔治·华盛顿大学和华盛顿卡内基研究所的联合主持下,第五届华盛顿理论物理会议在华盛顿哥伦比亚特区召开。在那里,关于核裂变的消息传播得更远,这促成了更多的实验验证。[28]

在此期间,当时居住在美国的匈牙利物理学家里奥·西拉德意识到,中子驱动的重原子裂变可以用来产生核链式反应。这种使用中子的反应是他在1933年首次提出的想法,当时他读到了卢瑟福关于利用他的团队在1932年通过质子分裂锂的实验来发电的观点的一些轻蔑的评论。然而,西拉德未能通过富中子的轻原子实现中子驱动的链式反应。理论上,如果在中子驱动的链式反应中产生的次级中子数量大于1,那么每个这样的反应可以触发多个额外的反应,产生指数级增加的反应数量。因此,铀的裂变有可能产生大量用于民用或军用目的的能量(即发电或原子弹)。

在第二次世界大战前夕,西拉德敦促费米(在纽约)和弗雷德里克·约里奥·居里(在巴黎)避免发表连锁反应的报告,以免纳粹政府觉察到制造裂变武器的可能性。费米犹豫了一下后同意了。但是约里奥·居里没有同意,1939年4月他在巴黎的团队,包括汉斯·冯·哈尔班和卢·科瓦尔斯基,在《自然》上发表论文指出铀-235每次核裂变发射的中子数为3.5个。[29](他们后来将这一数值修正为每次裂变发射2.6个中子。)西拉德和沃尔特·津恩同时进行的工作证实了这些结果。研究结果表明了建造核反应堆(西拉德和费米最初称之为“中子反应堆”)甚至核弹的可能性。然而,裂变和链式反应系统仍有许多未知之处。

\subsubsection{2.2 裂变链式反应的发现}
\begin{figure}[ht]
\centering
\includegraphics[width=6cm]{./figures/a3e20e2418848ba4.png}
\caption{第一座人工反应堆——芝加哥一号堆的图纸,。} \label{fig_HLB_11}
\end{figure}
“链式反应”在当时是一个众所周知的化学现象,在1933年,西拉德就已经预见到可以使用中子诱发类似的核物理过程,尽管当时西拉德不知道可以用什么材料启动这一过程。西拉德认为中子是这种情况的理想选择,因为它们不含静电荷。

随着铀裂变产生裂变中子的消息,西拉德立即意识到了使用铀进行核链式反应的可能性。夏天,费米和西拉德提出了用一个核反应堆来实现这一过程的想法。该堆将使用天然铀作为燃料。费米更早就表明,如果中子是低能的(所谓的“慢”或“热”中子),它们被原子俘获的效率要高得多,因为由于量子效应,它将使原子对中子反应截面大得多。因此,为了慢化裂变铀核释放的次级中子,费米和西拉德提出了石墨“慢化剂”,快速、高能的次级中子将与石墨“慢化剂”碰撞,有效地降低它们的速度。有了足够的铀和足够纯的石墨,他们的“堆”理论上可以维持慢中子链式反应。这将产生热量以及放射性裂变产物。

1939年8月,西拉德和匈牙利难民物理学家泰勒和维格纳认为德国人可能会利用裂变链式反应,并试图引起美国政府对这一问题的注意。为此,他们说服了德国犹太难民阿尔伯特·爱因斯坦以他的名义给总统富兰克林·罗斯福写了一封信。爱因斯坦和西拉德的信中提出了用船运送铀炸弹的可能性,这将摧毁“整个港口和周围的大部分农村”。总统于1939年10月11日——第二次世界大战在欧洲爆发后不久,美国加入战争前两年——收到了这封信。罗斯福下令授权一个科学委员会监督关于铀的研究工作进展,并拨出一小笔钱用于反应堆研究。

在英国,詹姆斯·查德威克基于鲁道夫·佩尔斯的一篇论文,提出了一种利用天然铀的原子弹,临界状态所需的质量为30–40 吨。在美国,罗伯特·奥本海默认为边长10cm的氘化铀立方体(约含11 千克铀)可能会发生“自爆”。在这种设计中,仍然认为需要使用减速剂来使核弹发生裂变(实际上如果易裂变核素被分离提纯就无需慢化剂)。12月,维尔纳·海森堡向德国国防部提交了一份关于铀炸弹可能性的报告。这些模型中的大多数仍然假设炸弹将由慢中子反应提供能量——类似于堆芯熔毁的反应堆。

在英国伯明翰,弗里希与德国犹太难民佩尔斯合作。他们设想使用纯净的铀-235,它的反应截面刚刚确定,比铀-238或天然铀(含99.3\%的铀-238)大得多。假设铀-235对快中子发生裂变的反应界面与慢中子相同,他们得到了采用纯铀-235型炸弹的临界质量只有6千克而不是吨量级,由此产生的爆炸将是巨大的。(实际结果是15kg,尽管实际核弹中数倍于该量的铀(如小男孩))。1940年2月,他们提交了弗里希-佩尔斯备忘录。讽刺的是,他们当时仍然被官方认为是“敌国移民”。格伦·西博格、约瑟夫·肯尼迪、阿瑟 · 瓦尔以及意大利犹太难民埃米利奥·塞格雷此后不久在中子轰击产生铀-238产生的铀-239的衰变产物中发现了钚-239,并确定它是同铀-235一样,是一种易裂变材料。

从技术上讲,分离铀-235是一项极难的挑战,因为铀-235和铀-238在化学性质相同,它们的质量仅相差三个中子。然而,如果能够分离出足够数量的铀-235,就可以产生快中子裂变链式反应。这将是极具爆炸性的、真正的“原子弹”。钚-239可以在核反应堆中生产的发现指向了快中子裂变炸弹的另一种方法。这两种方法都非常新颖,还没有被很好地理解,并且对于它们是否能在短时间内开发成功还存在相当大的怀疑。

1941年6月28日,美国科学研究与开发办公室在美国成立,以调动科学资源并将研究成果应用于国防。9月,费米组装了他的第一个“核反应堆”,试图在铀中产生慢中子诱导的链式反应,但由于缺乏适当的材料,或可用的适当材料不足,实验未能达到临界状态。

人们发现用天然铀燃料生产裂变链式反应绝非易事。早期的核反应堆不使用浓缩铀,因此,它们需要使用大量高度纯化的石墨作为中子慢化剂。在核反应堆中使用普通水(而不是重水)则需要浓缩的核燃料 ——将稀有的铀-235从更常见的铀-238中分离出来。反应器还需要包含化学纯度极高的的中子慢化剂材料,例如氘(重水)、氦、铍或碳(通常为石墨,碳的高纯度是必需的,因为许多化学杂质,例如天然硼的硼-10组分是非常强的中子吸收剂,因此会毒害链式反应并过早结束它。)

为了实现核能发电和武器生产,必须解决工业规模生产此类材料的问题。直到1940年,美国生产的铀金属总量不超过几克,甚至纯度也令人怀疑;金属铍不超过几公斤;和不超过几千克的浓缩氧化氘(重水)。最后,慢化剂所需的纯度的碳从未能批量生产。

弗兰克·斯佩丁用铝热剂或“艾姆斯”过程解决了大量生产高纯铀的问题。艾姆斯实验室成立于1942年,旨在大规模生产后续研究所必需的天然(未浓缩)铀金属。1942年12月2日,芝加哥一号堆核链式反应成功临界,同其他生产核弹原料钚的反应度相同,其采用非浓缩的天然铀作为燃料,这要感谢西拉德发现即使是天然铀“堆”,也可以使用非常纯的石墨作为慢化剂。在战时的德国,未能意识到非常纯石墨能够作为漫画集导致反应堆设计依赖重水,而盟军对重水生产地挪威的袭击使该设计无法实施。这些困难——以及其他许多困难——阻止了纳粹在战争期间建造一个能够达到临界状态的核反应堆,尽管他们从未像美国那样在核研究方面投入那么多精力,而是把重点放在其他技术上。

\subsubsection{2.3 曼哈顿计划及其他}
在美国,制造原子武器的全面努力始于1942年底。1943年,这项工作被美国陆军工程兵团接管,被称为曼哈顿工业区,即通常所说的曼哈顿计划,具有最高机密,由莱斯利·理查德·格罗夫斯中将领导。该项目的几十个厂区中包括:华盛顿州的汉福德区,拥有第一个工业规模核反应堆;橡树岭 (田纳西州)主要关注的是铀浓缩;洛斯阿拉莫斯(新墨西哥周)是原子弹开发和设计研究的科学中心。其他地点,尤其是伯克利辐射实验室和芝加哥大学冶金实验室,也发挥了重要的贡献作用。该项目的总体科学方向由物理学家罗伯特·奥本海默负责。

1945年7月,被称为“三位一体“的第一个原子爆炸装置在新墨西哥州沙漠被引爆。它的燃料是汉福德制造的钚。1945年8月,又有两个原子弹——采用铀-235的“小男孩”和钚的“胖子”被用来轰炸日本广岛和长崎。

在第二次世界大战后的几年里,许多国家参与了以研究核反应堆和核武器为目的而进行得关于核裂变的进一步研究。英国于1956年启动了第一座商业核电站。截至2013年,共计31个国家有437座反应堆。

\subsubsection{2.4 地球上的自然裂变链式反应堆}
自然界中的临界反应堆并不常见。在加蓬奥克洛的三个矿床中,发现了16个地点(所谓的奥克洛化石反应堆),在这些地点发现约20亿年前发生了自持核裂变。直到1972年才为人所知(但保罗·黑田东彦在1956年就提出相关的预测[30]),当法国物理学家弗朗西斯·佩林发现了奥克洛化石反应堆时,人们意识到大自然已经击败了人类。由正常水作慢化剂的大规模天然铀裂变链式反应在很久以前就已经发生了,不过现在不可能发生了。这个古老的过程能够使用普通的水作为慢化剂是因为20亿年前,天然铀中含有更多的寿命较短的铀-235(约3\%),而现在的天然铀中只有0.7\%的铀-235,必须浓缩到3\%才能用于轻水反应堆。

\subsection{参考文献}
[1]
^M. G. Arora & M. Singh (1994). Nuclear Chemistry. Anmol Publications. p. 202. ISBN 81-261-1763-X..

[2]
^Gopal B. Saha (1 November 2010). Fundamentals of Nuclear Pharmacy. Springer. pp. 11–. ISBN 978-1-4419-5860-0..

[3]
^Петржак, Константин (1989). "Как было открыто спонтанное деление" [How spontaneous fission was discovered]. In Черникова, Вера. Краткий Миг Торжества — О том, как делаются научные открытия [Brief Moment of Triumph — About making scientific discoveries] (in Russian). Наука. pp. 108–112. ISBN 5-02-007779-8.CS1 maint: Unrecognized language (link).

[4]
^S.Vermote等人(2008)“243-Cm(n th,f)和244-Cm(SF)三元粒子发射的比较研究”核裂变的动力学方面:第六届国际会议录。J.Kliman,M. G. Itkis,S. Gmuca(编辑。)中。World Scientific出版有限公司。新加坡有限公司。ISBN 9812837523。.

[5]
^J.伯恩(2011)中子、原子核和物质纽约州米尼奥拉市Dover Publications,第259页,ISBN 978-0-486-48238-5。.

[6]
^Marion Brünglinghaus. "Nuclear fission". European Nuclear Society. Retrieved 2013-01-04..

[7]
^汉斯·贝特(1950年4月),《氢弹》,,原子科学家公报,第99页。.

[8]
^V, Kopeikin; L, Mikaelyan and; V, Sinev (2004). "Reactor as a Source of Antineutrinos: Thermal Fission Energy". Physics of Atomic Nuclei. 67 (10): 1892. arXiv:hep-ph/0410100. Bibcode:2004PAN....67.1892K. doi:10.1134/1.1811196..


[9]
^这些裂变中子具有宽的能谱,范围从0到14兆电子伏,平均值为2兆电子伏,模式(统计)为0.75兆电子伏。参见伯恩,作品引用。.

[10]
^波顿研究所的核事件及其后果...“大约82\%两个大的裂变碎片的动能释放出来。这些碎片非常巨大和高度带电的粒子,容易与物质相互作用。它们将能量迅速转移到周围的武器材料上,这些材料会迅速变热”.

[11]
^"Nuclear Engineering Overview The various energies emitted per fission event pg 4. "167 MeV" is emitted by means of the repulsive electrostatic energy between the 2 daughter nuclei, which takes the form of the "kinetic energy" of the fission products, this kinetic energy results in both later blast and thermal effects. "5 MeV" is released in prompt or initial gamma radiation, "5 MeV" in prompt neutron radiation (99.36\% of total), "7 MeV" in delayed neutron energy (0.64\%) and "13 MeV" in beta decay and gamma decay(residual radiation)" (PDF). Technical University Vienna. Archived from the original (PDF) on May 15, 2018..

[12]
^"Nuclear Fission and Fusion, and Nuclear Interactions". National Physical Laboratory. Retrieved 2013-01-04..

[13]
^L. Bonneau; P. Quentin (2005). "Microscopic calculations of potential energy surfaces: Fission and fusion properties" (PDF). AIP Conference Proceedings. 798: 77. doi:10.1063/1.2137231. Archived from the original on September 29, 2006. Retrieved 2008-07-28.CS1 maint: Unfit url (link).

[14]
^E. Rutherford (1911). "The scattering of α and β particles by matter and the structure of the atom" (PDF). Philosophical Magazine. 21 (4): 669–688. Bibcode:2012PMag...92..379R. doi:10.1080/14786435.2011.617037..

[15]
^"Cockcroft and Walton split lithium with high energy protons April 1932". Outreach.phy.cam.ac.uk. 1932-04-14. Archived from the original on 2012-09-02. Retrieved 2013-01-04..

[16]
^"Sir Mark Oliphant (1901–2000)" (PDF). University of Adelaide. Archived from the original (PDF) on 5 October 2013. Retrieved 5 October 2013..

[17]
^查德威克在2006年宣布了他的初步发现:J. Chadwick (1932). "Possible Existence of a Neutron" (PDF). Nature. 129 (3252): 312. Bibcode:1932Natur.129Q.312C. doi:10.1038/129312a0.随后,他在以下内容中更详细地传达了他的发现:Chadwick, J. (1932). "The existence of a neutron". Proceedings of the Royal Society A. 136 (830): 692–708. Bibcode:1932RSPSA.136..692C. doi:10.1098/rspa.1932.0112.;和Chadwick, J. (1933). "The Bakerian Lecture: The neutron". Proceedings of the Royal Society A. 142 (846): 1–25. Bibcode:1933RSPSA.142....1C. doi:10.1098/rspa.1933.0152..

[18]
^E.Fermi,E. Amaldi,O. D'Agostino,F. Rasetti和e . segrè(1934)" radiattivitàprovo cata da bondamento di neutroni III,"La Ricerca Scientifica,vol . 5,不。 1页 452–453。.

[19]
^Ida Noddack (1934). "Über das Element 93". Zeitschrift für Angewandte Chemie. 47 (37): 653. doi:10.1002/ange.19340473707..

[20]
^塔克,艾达·伊娃。Astr.ua.edu。检索于2010年12月24日。.

[21]
^O. Hahn & F. Strassmann (1939). "Über den Nachweis und das Verhalten der bei der Bestrahlung des Urans mittels Neutronen entstehenden Erdalkalimetalle ("On the detection and characteristics of the alkaline earth metals formed by irradiation of uranium with neutrons")". Naturwissenschaften. 27 (1): 11–15. Bibcode:1939NW.....27...11H. doi:10.1007/BF01488241.。作者被确定在柏林凯泽威廉化学研究所。1938年12月22日收到。.

[22]
^L. Meitner & O. R. Frisch (1939). "Disintegration of Uranium by Neutrons: a New Type of Nuclear Reaction". Nature. 143 (3615): 239. Bibcode:1939Natur.143..239M. doi:10.1038/143239a0.。该文件日期为1939年1月16日。迈特纳被确定在斯德哥尔摩科学院物理研究所。弗里希被认定在哥本哈根大学理论物理研究所。.

[23]
^O. R. Frisch (1939). "Physical Evidence for the Division of Heavy Nuclei under Neutron Bombardment". Nature. 143 (3616): 276. Bibcode:1939Natur.143..276F. doi:10.1038/143276a0. Archived from the original on January 23, 2009..

[24]
^"Physical Evidence for the Division of Heavy Nuclei under Neutron Bombardment". 17 January 1939. Archived from the original on 2008-01-08.给编辑的这封信的实验是在13日进行的 1939年1月;看见理查德·罗德斯(1986年)原子弹的制造西蒙和舒斯特。第263和268页,ISBN 0-671-44133-7。.

[25]
^"The Nobel Prize in Chemistry 1944". Nobelprize.org. Retrieved 2008-10-06..

[26]
^理查德·罗德斯。(1986年)原子弹的制造西蒙和舒斯特,第268页,ISBN 0-671-44133-7。.

[27]
^H. L. Anderson; E. T. Booth; J. R. Dunning; E. Fermi; G. N. Glasoe & F. G. Slack (1939). "The Fission of Uranium". Physical Review. 55 (5): 511. Bibcode:1939PhRv...55..511A. doi:10.1103/PhysRev.55.511.2..

[28]
^理查德·罗德斯(1986)。原子弹的制造西蒙和舒斯特,第267-270页,ISBN 0-671-44133-7。.

[29]
^H. Von Halban; F. Joliot & L. Kowarski (1939). "Number of Neutrons Liberated in the Nuclear Fission of Uranium". Nature. 143 (3625): 680. Bibcode:1939Natur.143..680V. doi:10.1038/143680a0..

[30]
^P. K. Kuroda (1956). "On the Nuclear Physical Stability of the Uranium Minerals" (PDF). The Journal of Chemical Physics. 25 (4): 781. Bibcode:1956JChPh..25..781K. doi:10.1063/1.1743058..