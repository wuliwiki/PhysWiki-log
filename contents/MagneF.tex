% 磁场
% keys 磁感应强度|磁场|磁感线
% license Xiao
% type Tutor

\pentry{电场\nref{nod_Efield},矢量叉乘\nref{nod_Cross}}{nod_3a38}
\subsection{磁场的定义}
在学习静电学的过程中,我们已经知道,空间中某静止的点电荷 $q$ 所受到的电场力可以写为 $\bvec F=q\bvec E$,而 $E$ 是当前这一点的电场强度\footnote{在经典电磁学中,这个电场不包含来自试探电荷 $q$ 本身的部分,否则会出现发散困难。}。然而对于一个运动的电荷,人们发现除了 $q\bvec E$ 以外,还存在一个力的分量,它总是垂直于运动方向而且与速度和电荷量成正比。这意味着空间中还存在某个\enref{矢量场}{Vfield},我们把它记为 $\bvec B(\bvec r)$,那么完整的电磁力公式中应当还有 $\bvec F'=q\bvec v\times \bvec B$ 这一项。这个矢量场被称为\textbf{磁场}。在小时百科中\textbf{磁场(magnetic field)}指的是\textbf{磁感应强度(magnetic inductance)}, 一般记为 $\bvec B$。一般地,我们可以用广义洛伦兹力\autoref{eq_Lorenz_2} 来定义: 空间中某点的磁场使得运动时经过该点的点电荷所受的电磁力为
\begin{equation}\label{eq_MagneF_1}
\bvec F = q(\bvec E + \bvec v \cross \bvec B)~.
\end{equation}
其中 $q$ 是电荷量, $\bvec E$ 是该点的电场, $\bvec v$ 是速度, $\cross$ 是\enref{叉乘}{Cross}。

由于历史原因, “磁场强度” 这个而名字已经被占用% \addTODO{链接}
, 所以 $\bvec B$ 只好叫做磁感应强度。 在比较新的教材中, 磁场一般指磁感应强度。 磁场也可以使用\enref{安培力}{FAmp}来定义, 但安培力在微观本质上也是洛伦兹力。 


磁感应强度单位为\textbf{特斯拉(Tesla)},也就是 $\Si{kg\cdot C^{-1}s^{-1}}$。注意它乘上速度的量纲和电荷的量纲以后,$\Si{kg\cdot ms^{-2}}$ 刚好是力的量纲,这与洛伦兹力公式\autoref{eq_MagneF_1} 是相符的。
\subsection{磁感线}
与电场线一样, 我们可以在空间中画出许多有方向的\textbf{磁感线}(有时候也称为磁力线)。磁感线有以下这些性质:
\begin{enumerate}
\item 磁感线上任意一点的方向等于该点处磁场的方向。
\item 磁感线穿过单位横截面(该横截面与磁感线垂直,也就是说与磁场垂直)的条数与截面处的磁场大小近似成正比。
\item 任意两条磁感线都不相交。磁感线无源也无汇,它要么自己形成一条环路,要么起始于无穷远并延伸至无穷远。
\end{enumerate}
磁感线通常用于磁场的可视化和帮助理解。由于我们可以绘制的磁感线是有限的,它并不能精确地定量描述空间中的各点的磁场,但是它给出了一个非常直观的物理图景。磁感线的以上几条性质也是富含深意的。注意到,多条电场线可以交汇于一点(比如点电荷),而磁感线是无源且无汇的,这实际上暗示着磁场的\enref{高斯定律}{MagGau}应当比电场更加简洁。


\subsection{静磁学}
\pentry{电流密度\nref{nod_Idens}}{nod_238f}
在静电学中,电场能够由电荷的密度分布来确定,那么磁场又应当如何确定呢?

对于随时间变化的电荷与电流分布,其激发的电场与磁场往往是复杂而难以计算的。所以我们希望从最简单的情形入手,研究“静态”的问题。例如,空间中静态的点电荷分布会导致空间中\textbf{只有电场,没有磁场},而这些电荷可以由库仑定律直接地计算。

现在考虑这样的情形,假如我们用 $\rho(\bvec r,t),J(\bvec r,t)$ 描述空间中的\textbf{电荷密度分布}与\textbf{电流密度分布},那么\textbf{静磁学}所指的就是这样的情形:
\begin{equation}
\pdv{\rho}{t}=0,\pdv{\bvec J}{t}=0~.
\end{equation}
\textbf{不同于静态的点电荷分布},这里我们采取的是用电荷电流密度分布所描述的一种经典电磁学模型。尽管电荷密度分布是不随时间变化的,但\textbf{电荷仍然可以在空间中运动}。一个最简单而直接的例子就是沿着环形回路的恒定的\enref{电流}{I},如果我们忽略掉微观的电子的运动,而是从宏观的视角去测量空间中每一点的电荷、电流密度,那么只要电流环路不变,$\rho,\bvec J$ 就是不随时间发生变化的。

在这样的静磁学情形下,磁场就诞生了。磁场的方向可以由\enref{右手定则}{RHRul}判断,将大拇指对准电流的方向,则磁感线是围绕着电流的环路,它们的方向是顺着右手四指的方向的。磁场的定量计算由\enref{安培环路定律(静电学)}{AmpLaw}和\enref{毕奥—萨伐尔定律}{BioSav}给出。
