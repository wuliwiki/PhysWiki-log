% 电偶极子 2
% 电偶极子|电磁学|静电|电荷

\pentry{电偶极子\upref{eleDpl}}

可以拓展到多个电荷的情况或者连续分布的情况
\begin{equation}
\bvec p = \sum_i \bvec r_i q_i
\end{equation}
\begin{equation}
\bvec p = \int \bvec r \rho(\bvec r) \dd{V}
\end{equation}

注意只有被求和或者积分的所有电荷之和为零, 偶极子 $\bvec p$ 才不随参考系改变
\begin{equation}
\sum_i (\bvec r_i + \bvec d) q_i = \sum_i \bvec r_i q_i + \bvec d \sum_i q_i
\end{equation}
若电荷之和不为零, 我们可以定义一个和质心性质类似的中心
\begin{equation}
\bvec r_0 = \frac{\sum_i \bvec r_i q_i}{\sum_i q_i}
\end{equation}
可以证明这个位置和参考系无关.
\begin{equation}
\frac{\sum_i (\bvec r_i + \bvec d) q_i}{\sum_i q_i} = \bvec r_0 + \bvec d
\end{equation}
如果以 $\bvec r_0$ 为原点, 偶极子为零.

那多级展开到底应该关于哪一点进行呢? 笔者认为最好的选择是(想像一个巨大的正电荷左右分别有两个等大反号的小电荷, 中心当然应该是在大电荷上)
\begin{equation}
\bvec r_0 = \frac{\sum_i \bvec r_i \abs{q_i}}{\sum_i \abs{q_i}}
\end{equation}
这个位置同样与坐标系选取无关.

\subsection{匀强电场中电偶极子的势能}
\begin{equation}\label{eleDP2_eq1}
E = -\bvec p \vdot \bvec E
\end{equation}
\addTODO{推导}
