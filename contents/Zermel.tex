% 策梅洛-弗兰克尔集合论(综述)
% license CCBYSA3
% type Wiki

本文根据 CC-BY-SA 协议转载翻译自维基百科\href{https://en.wikipedia.org/wiki/Zermelo\%E2\%80\%93Fraenkel_set_theory}{相关文章}。

在集合论中,泽梅洛-弗兰克尔集合论,以数学家恩斯特·泽梅洛和亚伯拉罕·弗兰克尔命名,是一个公理化系统,旨在20世纪初提出,以构建一个没有类似罗素悖论之类悖论的集合理论。如今,泽梅洛-弗兰克尔集合论,包含历史上具有争议的选择公理(AC),是标准的公理化集合论形式,因此也是数学的最常见基础。包含选择公理的泽梅洛-弗兰克尔集合论简称为ZFC,其中C代表“选择”\(^\text{[1]}\),而ZF指的是没有包含选择公理的泽梅洛-弗兰克尔集合论的公理。

非正式地说\(^\text{[2]}\),泽梅洛-弗兰克尔集合论旨在形式化一个单一的基本概念,即遗传的良基集合,使得所有在讨论宇宙中的实体都是这样的集合。因此,泽梅洛-弗兰克尔集合论的公理仅涉及纯集合,并防止其模型中包含尿元素(不是集合本身的元素)。此外,适当类(由其成员共享的属性定义的数学对象集合,这些集合过大无法作为集合处理)只能间接处理。具体来说,泽梅洛-弗兰克尔集合论不允许存在一个普遍集合(包含所有集合的集合),也不允许无限制的理解,从而避免了罗素悖论。冯·诺依曼-伯奈斯-哥德尔集合论(NBG)是泽梅洛-弗兰克尔集合论的一个常用保守扩展,它确实允许对适当类进行显式处理。

泽梅洛-弗兰克尔集合论的公理有许多等价的表述。大多数公理陈述了从其他集合定义的特定集合的存在。例如,配对公理指出,给定任何两个集合\(a\)和\(b\),存在一个新的集合\(\{a,b\}\),其中恰好包含\(a\) 和 \(b\)。其他公理描述了集合成员关系的属性。这些公理的目标是,若将每个公理解释为关于冯·诺依曼宇宙中所有集合的陈述(也称为累积层次结构),那么每个公理都应该成立。

泽梅洛-弗兰克尔集合论的元数学已经得到了广泛的研究。该领域的标志性成果确立了选择公理与其他泽梅洛-弗兰克尔公理的逻辑独立性,以及连续统假设与ZFC的逻辑独立性。像ZFC这样的理论的一致性不能在该理论内部证明,正如哥德尔的第二不完备性定理所示。
\subsection{历史}
现代集合论的研究由乔治·康托尔和理查德·德德金德在1870年代发起。然而,天真的集合论中的悖论(如罗素悖论)的发现,促使人们寻求一种更加严谨的集合论形式,以避免这些悖论。

1908年,恩斯特·泽梅洛提出了第一个公理化集合论——泽梅洛集合论。然而,正如亚伯拉罕·弗兰克尔在1921年给泽梅洛的信中首次指出的,这一理论无法证明某些集合和基数的存在,而这些集合和基数在当时大多数集合论学者中被视为理所当然,特别是基数阿列夫-欧米伽(\(\aleph_\omega\))以及集合 \(\{Z_0, \mathcal{P}(Z_0), \mathcal{P}(\mathcal{P}(Z_0)), \mathcal{P}(\mathcal{P}(\mathcal{P}(Z_0))), \dots\}\),其中 \(Z_0\) 是任何无限集合,\(\mathcal{P}\) 是幂集运算。\(^\text{[3]}\)此外,泽梅洛的一个公理使用了一个概念,即“确定”属性,其操作性含义并不明确。1922年,弗兰克尔和托拉尔夫·斯科勒姆独立提出将“确定”属性操作化为能够以一阶逻辑中能够表达的良构公式来表示,其中原子公式仅限于集合成员关系和相等关系。他们还独立提出用替换公理代替规范公理公理模式。将此公理模式以及正则性公理(由约翰·冯·诺依曼首次提出)添加到泽梅洛集合论中,得到的理论被称为ZF。将选择公理(AC)或等价的陈述添加到ZF中,得到的理论就是ZFC。
\subsection{形式语言}  
形式上,ZFC 是一种一排序理论,使用一阶逻辑。等号符号可以被视为一个原始的逻辑符号,或者作为具有完全相同元素的高层缩写。前者的方法最为常见。其符号系统有一个谓词符号,通常表示为\(\in\),这是一个二元关系符号,表示集合成员关系。例如,公式\(a \in b\)表示\(a\)是集合\(b\)的元素(也可以读作\(a\)是\(b\)的成员)。

有不同的方法来表述形式语言。一些作者可能会选择不同的连接词或量词。例如,逻辑连接词 NAND 本身就可以编码其他连接词,这种特性称为功能完备性。本节旨在在简洁性和直观性之间找到平衡。

语言的字母表由以下部分组成:
\begin{itemize}
\item 用于表示集合的可数无限量的变量
\item 逻辑连接词:\(\lnot\),\(\land\),\(\lor\)
\item 量词符号:\(\forall\),\(\exists\)
\item 等号符号:\(=\)
\item 集合成员符号:\(\in\)
\item 括号:()
\end{itemize}
使用这个字母表,形成良构公式(wff)的递归规则如下:
\begin{itemize}
\item 让 \(x\) 和 \(y\) 是任何变量的元变量。以下是构建原子公式(最简单的wff)的两种方式:\\
\(x = y\)\\
\(x \in y\)\\
\end{itemize}
\begin{itemize}
\item -让 \(\phi\) 和 \(\psi\) 是任何wff的元变量,\(x\) 是任何变量的元变量。以下是有效的wff构造:\\
\(\lnot \phi\)\\
\((\phi \land \psi)\)\\
\((\phi \lor \psi)\)\\
\(\forall x \phi\)\\
\(\exists x \phi\)
\end{itemize}
一个良构公式可以看作是一个语法树。叶节点始终是原子公式。节点\(\land\)和\(\lor\)各有两个子节点,而节点\(\lnot\)、\(\forall x\)和\(\exists x\)各有一个子节点。尽管有可数无限多的wff,但每个wff都有有限数量的节点。
\subsection{公理}  
ZFC 公理有许多等价的表述。\(^\text{[5]}\)以下是来自 Kunen (1980) 的特定公理集。下列公理按顺序表达,采用了一阶逻辑与高层缩写的混合方式。

公理 1-8 构成 ZF,而公理 9 将 ZF 转变为 ZFC。根据 Kunen (1980),我们用等价的良序定理代替选择公理作为公理 9。

所有 ZFC 的表述都意味着至少存在一个集合。Kunen 包括了一个直接断言集合存在的公理,尽管他指出这样做仅仅是“为了强调”\(^\text{[6]}\)。在这里省略这一公理可以通过两种方式进行辩解。首先,在通常用来形式化 ZFC 的一阶逻辑标准语义中,讨论领域必须是非空的。因此,存在某物是一个一阶逻辑的逻辑定理——通常表述为某物与它自己相同,\(\exists x (x = x)\)。因此,每个一阶理论都证明某物存在。然而,如上所述,由于 ZFC 的预期语义中只有集合,因此在 ZFC 的语境下,这一逻辑定理的解释是某个集合存在。因此,无需额外的公理来断言集合的存在。其次,即便 ZFC 被形式化为所谓的自由逻辑,其中仅凭逻辑本身不能证明某物存在,然而无限公理断言存在一个无限集合。这意味着某个集合存在,因此,依然不需要包括一个单独的公理来断言集合的存在。
\subsubsection{外延公理}  
如果两个集合具有相同的元素,则它们是相等的(即是同一个集合)。
\[
\forall x \forall y \left[ \forall z \left( z \in x \Leftrightarrow z \in y \right) \Rightarrow x = y \right]~
\]
这个公理的逆命题可以从等式的代换性质推导出来。ZFC 是在一阶逻辑中构造的。一些一阶逻辑的表述包括身份(即等号);而其他表述则不包括。如果构建集合论的一阶逻辑中不包含等号“=”,则\( x = y \)可以定义为以下公式的缩写:\(^\text{[7]}\)
\[
\forall z \left[ z \in x \Leftrightarrow z \in y \right] \land \forall w \left[ x \in w \Leftrightarrow y \in w \right]~
\]
在这种情况下,外延公理可以重新表述为:
\[
\forall x \forall y \left[ \forall z \left( z \in x \Leftrightarrow z \in y \right) \Rightarrow \forall w \left( x \in w \Leftrightarrow y \in w \right) \right]~
\]
这表示,如果\( x \)和\( y \)具有相同的元素,则它们属于相同的集合。\(^\text{[8]}\)
\subsubsection{正则性公理(也叫做基础公理)}
每一个非空集合 \( x \) 包含一个成员 \( y \),使得 \( x \) 和 \( y \) 是不相交的集合。
\[
\forall x \left[ \left( \exists a \left( a \in x \right) \right) \Rightarrow \exists y \left( y \in x \land \lnot \exists z \left( z \in y \land z \in x \right) \right) \right]~
\]
或用现代符号表示为:\(\forall x \left( x \neq\emptyset \Rightarrow\exists y \left( y \in x \land y \cap x = \emptyset \right) \right)\)

这条公理(以及配对公理和并集公理)意味着,例如,没有集合是它自己的元素,并且每个集合都有一个序数等级。
\subsubsection{规范公理模式(或分离公理,或限制性理解公理)}   
子集通常使用集合构造符号表示。例如,偶整数可以作为整数集\( \mathbb{Z} \)的一个子集构造,该子集满足同余模条件\( x \equiv 0 \pmod{2} \):
\[
\{ x \in \mathbb{Z} : x \equiv 0 \pmod{2} \}.~
\]
一般来说,集合\( z \)的一个子集,满足公式\( \varphi(x) \)(其中\( x \)是自由变量),可以写为:
\[
\{ x \in z : \varphi(x) \}.~
\]
规范公理模式声明,这种子集总是存在的(它是一个公理模式,因为对于每个\( \varphi \),都有一个公理)。正式地,令\( \varphi \)为 ZFC 语言中的任意公式,所有自由变量为 \( x, z, w_1, \ldots, w_n \)(其中\( y \) 在 \( \varphi \)中不是自由变量)。那么:
\[
\forall z \forall w_1 \forall w_2 \ldots \forall w_n \exists y \forall x \left[ x \in y \Leftrightarrow \left( (x \in z) \land \varphi(x, w_1, w_2, \ldots, w_n, z) \right) \right].~
\]
注意,规范公理模式只能构造子集,不允许构造更一般形式的实体:
\[
\{ x : \varphi(x) \}.~
\]
这个限制是为了避免罗素悖论(让 \( y = \{x : x \notin x\} \),那么\( y \in y \Leftrightarrow y \notin y \)),以及伴随天真集合论中无限制理解的变体(因为在这个限制下,\( y \)仅指代那些不属于自己的集合,并且尚未证明\( y \in z \),尽管 \( y \subseteq z \)成立,因此\( y \)处于一个独立的位置,从而无法引用或理解自己;因此,从某种意义上说,这个公理模式表示为了基于公式\( \varphi(x) \)构造一个\( y \),我们需要事先限制\( y \)所涉及的集合 \( z \),使得\( y \)在其中是外部的,因此\( y \)无法引用自己;换句话说,集合不应引用自身)。

在ZF的某些其他公理化中,这个公理是多余的,因为它可以从替换公理模式和空集合公理中推导出来。

另一方面,规范公理模式可以用来证明空集合的存在,表示为\( \varnothing \),一旦至少知道一个集合存在。这样做的一种方法是使用一个没有任何集合拥有的属性\( \varphi \)。例如,如果\( w \)是任何现有的集合,则空集合可以构造为:
\[
\varnothing = \{ u \in w \mid (u \in u) \land \lnot (u \in u) \}.~
\]
因此,空集合公理由这里呈现的九个公理推导出来。外延公理意味着空集合是唯一的(不依赖于 \( w \))。通常会进行定义扩展,将符号\( \varnothing \)添加到ZFC的语言中。
\subsubsection{配对公理}  
如果\( x \)和\( y \)是集合,则存在一个集合,它包含\( x \)和\( y \)作为元素。例如,如果\( x = \{1, 2\} \)且\( y = \{2, 3\} \),那么 \( z \)将是\(\{\{1, 2\}, \{2, 3\}\}\)。
\[
\forall x \forall y \exists z \left( (x \in z) \land (y \in z) \right)~
\]
必须使用规范公理模式将其简化为一个仅包含这两个元素的集合。配对公理是 Z 的一部分,但在 ZF 中是多余的,因为它可以从替换公理模式中推导出来,前提是我们有一个至少包含两个元素的集合。至少有两个元素的集合的存在可以通过无限公理或通过应用两次幂集公理到任何集合并结合规范公理模式来确保。
\subsubsection{并集公理}  
集合元素的并集存在。例如,集合\(\{\{1, 2\}, \{2, 3\}\}\)的并集是\(\{1, 2, 3\}\)。

并集公理声明,对于任何集合的集合\(\mathcal{F}\),存在一个集合\(A\),它包含每个元素,该元素是\(\mathcal{F}\)中某个成员的成员:
\[
\forall \mathcal{F} \, \exists A \, \forall Y \, \forall x \left[ (x \in Y \land Y \in \mathcal{F}) \Rightarrow x \in A \right]~
\]
尽管这个公式并没有直接断言\(\cup \mathcal{F}\)的存在,但可以通过上面使用规范公理模式从\(A\)构造出集合\(\cup \mathcal{F}\):
\[
\cup \mathcal{F} = \{ x \in A : \exists Y \, (x \in Y \land Y \in \mathcal{F}) \}~
\]
\subsubsection{替换公理模式} 
替换公理模式断言,集合通过任何可定义的函数映射后的像也会落在一个集合内。

正式地,令\( \varphi \)为 ZFC 语言中的任意公式,其自由变量为\( x, y, A, w_1, \dotsc, w_n \),因此特别地\( B \)在\( \varphi \)中不是自由变量。则:
\[
\forall A \forall w_1 \forall w_2 \ldots \forall w_n \left[ \forall x \left( x \in A \Rightarrow \exists ! y \, \varphi \right) \Rightarrow \exists B \, \forall x \left( x \in A \Rightarrow \exists y \left( y \in B \land \varphi \right) \right) \right]~
\]
(独特存在量词\( \exists ! \)表示存在恰好一个元素,使其满足给定的命题。)

换句话说,如果关系\( \varphi \)表示一个可定义的函数\( f \),\( A \)表示其定义域,并且对于每个\( x \in A \),\( f(x) \)是一个集合,那么\( f \)的值域是某个集合\( B \)的子集。这里表述的形式,其中\( B \)可能大于严格必要的大小,有时被称为集合公理模式。
\subsubsection{无限公理}
令\( S(w) \)简写为 \( w \cup \{w\} \),其中\( w \)是某个集合。(我们可以通过应用配对公理,令 \( x = y = w \),从而得到集合\( z = \{w\} \),证明\( \{w\} \)是一个有效的集合。)然后,存在一个集合\( X \),使得空集合\( \varnothing \)(根据公理定义)是\( X \)的一个成员,并且,凡是集合\( y \)是\( X \) 的成员,则\( S(y) \)也必定是\( X \)的成员。
\[
\exists X \left[ \exists e \left( \forall z \, \neg (z \in e) \land e \in X \right) \land \forall y \left( y \in X \Rightarrow S(y) \in X \right) \right].~
\]
或者用现代符号表示为:
\[
\exists X \left[ \varnothing \in X \land \forall y \left( y \in X \Rightarrow S(y) \in X \right) \right].~
\]
更通俗地说,存在一个集合\( X \),其包含无限多个成员。然而,必须证明这些成员都是不同的,因为如果两个元素相同,则序列将在有限的集合循环中回绕。正则性公理防止了这种情况的发生。满足无限公理的最小集合\( X \)是冯·诺依曼序数\( \omega \),它也可以被视为自然数集合\( \mathbb{N} \)。
\subsubsection{幂集公理} 
根据定义,集合\( z \)是集合\( x \)的子集,当且仅当集合\( z \)的每个元素也是集合\( x \)的元素:
\[
(z \subseteq x) \Leftrightarrow (\forall q (q \in z \Rightarrow q \in x)).~
\]
幂集公理声明,对于任何集合\( x \),存在一个集合\( y \),它包含集合\( x \)的每一个子集:
\[
\forall x \exists y \forall z (z \subseteq x \Rightarrow z \in y).~
\]
然后使用规范公理模式来定义幂集\( \mathcal{P}(x) \),作为包含集合\( x \)子集的集合\( y \)的子集:
\[
\mathcal{P}(x) = \{ z \in y : z \subseteq x \}.~
\]
公理 1-8 定义了 ZF。这些公理的替代表述通常会遇到,其中一些在 Jech (2003) 中列出。一些 ZF 的公理化包含一个公理,断言空集合存在。配对、公理、公理、替换公理和幂集公理通常以这样的形式表达:集合 \( x \) 的成员是那些公理断言 \( x \) 必须包含的集合。

接下来增加以下公理,以将 ZF 转化为 ZFC:
\subsubsection{良序公理(选择公理)} 
最后一个公理,通常称为选择公理,这里作为良序的一种属性呈现,如 Kunen (1980) 所述。对于任何集合\( X \),存在一个二元关系\( R \),它良序\( X \)。这意味着\( R \)是集合\( X \)上的线性顺序,使得\( X \)的每个非空子集在顺序\( R \)下都有一个最小元素。
\[
\forall X \exists R (R \, \text{well-orders} \, X).~
\]
给定公理 1–8,许多陈述可以证明等价于公理 9。其中最常见的一种如下所示。令\( X \)是一个其成员都为非空的集合。那么,存在一个从\( X \)到\( X \)成员的并集的函数\( f \),称为“选择函数”,使得对于所有\( Y \in X \),都有\( f(Y) \in Y \)。选择公理的第三种等价版本是佐恩引理。

由于当\( X \)是有限集合时选择函数的存在可以很容易地从公理 1–8 推导出来,因此选择公理(AC)只对某些无限集合才有意义。AC 被描述为非构造性的,因为它断言存在一个选择函数,但没有说明如何“构造”这个选择函数。
\subsection{通过累积层次结构的动机}  
ZFC 公理的一个动机是由约翰·冯·诺依曼引入的集合的累积层次结构。\(^\text{[10]}\)在这一观点中,集合论的宇宙是分阶段建立的,每个阶段对应一个序数。在阶段 0 时,还没有集合。在随后的每个阶段,如果一个集合的所有元素已经在前面的阶段中被加入,则该集合会被添加到宇宙中。因此,空集合在阶段 1 中被添加,包含空集合的集合在阶段 2 中被添加。\(^\text{[11]}\)通过这种方式获得的所有集合的集合,被称为\( V \)。可以通过为每个集合指定其被加入\( V \)的第一个阶段,来将\( V \)中的集合排列成一个层次结构。

可以证明,一个集合属于\( V \)当且仅当该集合是纯粹的并且良基的。如果序数类具有适当的反射属性,则\( V \)满足 ZFC 的所有公理。例如,假设集合\( x \)在阶段\( \alpha \)被添加,这意味着\( x \)的每个元素都在阶段\( \alpha \)之前的某个阶段被添加。那么,\( x \)的每个子集也会在(或之前)阶段\( \alpha \)被添加,因为 \( x \) 的任何子集的所有元素也都是在阶段\( \alpha \) 之前添加的。这意味着,任何通过分离公理能够构造的\( x \)的子集都会在(或之前)阶段\( \alpha \)被添加,而\( x \)的幂集将在\( \alpha \)之后的下一个阶段被添加。\(^\text{[12]}\)

将集合宇宙视为累积层次结构的图像是 ZFC 以及相关公理集合理论(如冯·诺依曼–伯奈斯–哥德尔集合论(通常称为 NBG)和摩尔斯–凯利集合论)的特点。累积层次结构与其他集合论(如新基础理论)不兼容。

可以改变\( V \)的定义,使得在每个阶段,不是将所有先前阶段联合的子集添加到当前阶段,而只有那些在某种意义上是可定义的子集才被添加。这样会得到一个更“狭窄”的层次结构,进而得到构造宇宙\( L \),它也满足 ZFC 的所有公理,包括选择公理。是否\( V = L \)是 ZFC 公理的独立问题。尽管\( L \)的结构比\( V \)更规律且行为更好,但很少有数学家认为\( V = L \)应该作为额外的“构造性公理”添加到 ZFC 中。
\subsection{元数学}  
\subsubsection{虚拟类}  
适当类(由其成员共享属性定义的数学对象集合,这些集合过于庞大,无法作为集合处理)在 ZF(因此也在 ZFC)中只能间接处理。在 ZF 和 ZFC 中,一个替代适当类的方法是虚拟类符号构造,它由奎因在 1969 年引入,其中整个构造\( y \in \{ x \mid Fx \} \)被简单地定义为\( Fy \)\(^\text{[13]}\)。这种表示法为类提供了一个简单的符号,它们可以包含集合,但不一定是集合,同时不需要承诺类的本体论(因为该符号可以在语法上转换为仅使用集合的形式)。奎因的方法基于伯奈斯和弗兰克尔(1958)早期的方法。虚拟类也用于莱维(Levy, 2002)、竹内和扎林(Takeuti & Zaring, 1982),以及 ZFC 的 Metamath 实现中。
\subsubsection{有限公理化}   
替换公理模式和分离公理模式各自包含无限多的实例。蒙塔古(Montague, 1961)在其 1957 年的博士论文中首先证明了一个结果:如果 ZFC 是一致的,那么不可能仅用有限多的公理来公理化 ZFC。另一方面,冯·诺依曼–伯奈斯–哥德尔集合论(NBG)可以被有限公理化。NBG 的本体论包括适当类和集合;集合是任何可以作为另一个类的成员的类。NBG 和 ZFC 在某种意义上是等价的集合论,即任何不涉及类且可以在一个理论中证明的定理,也可以在另一个理论中证明。
\subsubsection{一致性} 
哥德尔的第二不完备性定理表明,一个递归公理化的系统,如果能够解释罗宾逊算术,则只有在该系统不一致的情况下,才能证明它自身的一致性。此外,罗宾逊算术可以在一般集合论中进行解释,这只是 ZFC 的一个小片段。因此,ZFC 的一致性不能在 ZFC 本身内证明(除非它实际上是不一致的)。因此,在将 ZFC 与普通数学等同的程度上,ZFC 的一致性无法在普通数学中证明。ZFC 的一致性确实可以从一个弱不可达基数的存在中得出,如果 ZFC 是一致的,这一基数在 ZFC 中是无法证明的。然而,人们认为 ZFC 不太可能包含一个未被发现的矛盾;普遍认为,如果 ZFC 不一致,这一事实现在应该已经被发现。可以确定的是,ZFC 对天真集合论的经典悖论具有免疫力:罗素悖论、布拉利-福尔蒂悖论和康托尔悖论。

阿比安和拉马基亚(1978)研究了 ZFC 的一个子理论,该子理论包含外延公理、并集公理、幂集公理、替换公理和选择公理。通过使用模型,他们证明了该子理论的一致性,并证明了外延公理、替换公理和幂集公理中的每一条都是与该子理论的其他四个公理独立的。如果这个子理论增添了无限公理,那么并集公理、选择公理和无限公理中的每一条都是与其余五个公理独立的。因为存在非良基模型,这些模型满足 ZFC 的每个公理,除了正则性公理,因此该公理与其他 ZFC 公理是独立的。

如果 ZFC 是一致的,它无法证明范畴理论所需的不可达基数的存在。如果 ZF 增加了塔尔斯基公理,那么这种性质的巨大集合是可能的。\(^\text{[14]}\)假设该公理将无限公理、幂集公理和选择公理(上述 7-9)转化为定理。
\subsubsection{独立性} 
许多重要的命题是 ZFC 独立的。独立性通常通过强制法证明,方法是展示每个可数的传递模型(有时通过增加大基数公理)都可以扩展以满足相关命题。然后,展示一个不同的扩展满足该命题的否定。通过强制法的独立性证明自动证明了与算术命题、其他具体命题和大基数公理的独立性。一些与 ZFC 独立的命题可以在特定的内模型中证明成立,例如在构造宇宙中。然而,某些关于构造集合的命题与假定的大基数公理不一致。

强制法证明以下命题与 ZFC 独立:
\begin{itemize}
\item 构造性公理(\( V = L \))(这也不是 ZFC 公理)
\item 连续统假设
\item 钻石原理
\item 马丁公理(这不是 ZFC 公理)
\item 苏斯林假设
\end{itemize}
备注:
\begin{itemize}
\item \( V = L \)的一致性可以通过内模型证明,但不能通过强制法证明:每个 ZF 模型都可以被修剪以成为\( \text{ZFC} + V = L \)的模型。
\item 钻石原理意味着连续统假设和苏斯林假设的否定。
\item 马丁公理加上连续统假设的否定意味着苏斯林假设。
\item 构造宇宙满足广义的连续统假设、钻石原理、马丁公理和库雷帕假设。
\item 库雷帕假设的失败与强不可达基数的存在是等一致的。
\end{itemize}
强制法方法的变体也可以用来证明选择公理的一致性和不可证明性,即选择公理与 ZF 的独立性。选择公理的一致性可以通过证明内模型\( L \)满足选择公理来(相对容易)验证。(因此,ZFC 的每个模型都包含一个 ZF 的子模型,从而\( \text{Con(ZF)} \)推出\( \text{Con(ZFC)} \)。)由于强制法保留了选择公理,我们不能直接从一个满足选择公理的模型中产生一个与选择公理矛盾的模型。然而,我们可以使用强制法创建一个包含适当子模型的模型,即一个满足 ZF 但不满足选择公理的子模型。

另一种证明独立性结果的方法,与强制法无关,基于哥德尔的第二不完备性定理。这种方法利用被检验独立性的命题,来证明 ZFC 的集合模型的存在,在这种情况下,\( \text{Con(ZFC)} \)为真。由于 ZFC 满足哥德尔第二定理的条件,ZFC 的一致性不能在 ZFC 内部证明(前提是 ZFC 实际上一致)。因此,任何允许这种证明的命题都不能在 ZFC 中被证明。这种方法可以证明大基数的存在不能在 ZFC 中证明,但不能证明在假设这些基数存在的前提下,ZFC 是无矛盾的。
\subsubsection{提议的补充}  
统一集合论学者的项目,旨在通过额外的公理来解决连续统假设或其他元数学模糊性,有时被称为“哥德尔的计划”\(^\text{[15]}\)。数学家们目前就哪些公理是最可信的或“自明的”、哪些公理在不同领域中最有用以及有用性与可信性之间应该如何权衡展开辩论;一些“多重宇宙”集合论学者认为,有用性应当是选择通常采用哪些公理的唯一终极标准。一种思潮依赖于扩展集合的“迭代”概念,通过采用强制公理来构建一个具有有趣和复杂但合理可处理结构的集合论宇宙;另一种思潮则主张构建一个更整洁、较少杂乱的宇宙,可能专注于一个“核心”内模型。\(^\text{[16]}\)
\subsection{批评}  
ZFC 因过于强大和过于弱小而受到批评,还因为它未能捕捉到如适当类和普遍集合等对象。

许多数学定理可以在比 ZFC 弱得多的系统中证明,例如皮亚诺算术和二阶算术(如逆数学程序所探讨的)。桑德斯·麦克莱恩(Saunders Mac Lane)和所罗门·费费尔曼(Solomon Feferman)都曾提出这一点。部分“主流数学”(即与公理化集合论不直接相关的数学)超出了皮亚诺算术和二阶算术的范围,但仍然,所有这些数学都可以在比 ZFC 更弱的理论 ZC(带选择的泽梅洛集合论)中进行。ZFC 的许多力量,包括正则性公理和替换公理模式,主要是为了促进集合论本身的研究。

另一方面,在公理化集合论中,ZFC 相对较弱。与新基础理论不同,ZFC 不承认普遍集合的存在。因此,在 ZFC 下的集合宇宙不是在集合代数的基本运算下封闭的。与冯·诺依曼–伯奈斯–哥德尔集合论(NBG)和摩尔斯–凯利集合论(MK)不同,ZFC 不承认适当类的存在。ZFC 的另一个比较弱点是,ZFC 中包含的选择公理比 NBG 和 MK 中包含的全局选择公理要弱。

有许多与 ZFC 独立的数学命题。这些命题包括连续统假设、怀特黑德问题和正常摩尔空间猜想。通过在 ZFC 中添加马丁公理或大基数公理等公理,可以证明其中一些猜想。其他一些则可以在 ZF+AD 中决定,其中 AD 是确定性公理,这是一种与选择公理不兼容的强假设。大基数公理的一个吸引力在于,它们使得 ZF+AD 中的许多结果可以在 ZFC 加上某些大基数公理的情况下确立。Mizar 系统和 metamath 采用了塔尔斯基–格罗滕迪克集合论,这是 ZFC 的一个扩展,使得涉及格罗滕迪克宇宙(在范畴论和代数几何中遇到的)证明可以被形式化。
\subsection{另见}  
\begin{itemize}
\item 数学基础  
\item 内部模型  
\item 大基数公理
\end{itemize}  
相关的公理化集合论:
\begin{itemize}
\item 摩尔斯–凯利集合论  
\item 冯·诺依曼–伯奈斯–哥德尔集合论  
\item 塔尔斯基–格罗滕迪克集合论  
\item 构造性集合论  
\item 内部集合论
\end{itemize}
\subsection{注释}
\begin{enumerate}
\item 切谢尔斯基(Ciesielski)1997年,第4页:“策梅洛-弗兰克尔公理(缩写为ZFC,其中C代表选择公理(Axiom of Choice))”
\item 库宁(Kunen)2007年,第10页
\item 埃宾豪斯(Ebbinghaus)2007年,第136页
\item 哈尔拜森(Halbeisen)2011年,第62–63页
\item 弗兰克尔、巴-希勒尔与莱维(Fraenkel, Bar-Hillel & Lévy)1973年
\item 库宁(Kunen)1980年,第10页
\item 哈彻(Hatcher)1982年,第138页,定义1
\item 弗兰克尔、巴-希勒尔与莱维(Fraenkel, Bar-Hillel & Lévy)1973年
\item 舍恩菲尔德(Shoenfield)2001年,第239页
\item 舍恩菲尔德(Shoenfield)1977年,第2节
\item 欣曼(Hinman)2005年,第467页
\item 关于“V满足ZFC”的完整论证,参见舍恩菲尔德(1977年)。
\item 林克(Link)2014年
\item 塔斯基(Tarski)1939年
\item 费弗曼(Feferman)1996年
\item 沃尔乔弗(Wolchover)2013年
\end{enumerate}