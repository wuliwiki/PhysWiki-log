% 天津大学 2015 年考研量子力学
% 考研|天津大学|量子力学|2015

\subsection{30分}
$\varPsi (x)=A[\frac{1}{\sqrt{3}}\varPsi_{201}(x)-\sqrt{\frac{2}{3}}\varPsi_{311}(x)]$,求归一化函数和角动量平均值.
\subsection{30分}
\begin{enumerate}
\item $[L^{2},L_{x}]$ 的对易关系,定义 $\hat{L}_{+}=\hat{L}_{x}+i\hat{L}_{y}$,$\hat{L}_{-}=\hat{L}_{x}-i\hat{L}_{y}$,求 $[\hat{L}_{+},\hat{L}_{z}]$;$[\hat{L}_{-},\hat{L}_{z}]$;$[\hat{L}_{+},\hat{L}_{-}]$ 作用在波函数 $Y_{lm}$ 的结果;(10分)
\item 简述光的波粒二象性的过程;(5分)
\item 已知 $e^{i\rho_{y}\partial}=A+i\rho_{y}B$,求 $A$,$B$ 与 $\partial$ 的关系.($\partial$ 为常数)(15分)
\end{enumerate}
\subsection{30分}
$\varPsi (x,0)=A[\varPsi_{0}(x)+x\varPsi_{1}(x)]$
\begin{enumerate}
\item 求 $t$ 时刻的波函数;
\item 坐标的平均值;
\item 能量平均值并解释其结果.
\end{enumerate}
\subsection{30分}
一质量为 $M$,带电荷量为 $q$ 的粒子在 $(0,a)$ 的无限深势阱中运动,受到弱均匀外电场 $\epsilon$ 的作用,求波函数至一级修正,能量至二级修正.
\subsection{30分}
与电子一样,中子的自旋也是 $\frac{1}{2}$ 并且具有磁矩 $\vec{\mu}=g\vec{S}$.
\begin{enumerate}
\item 若两个中子之间有相互作用能量是 $g\vec{S}_{1}\vec{S}_{2}$,求体系的能级和波函数,讨论能级简并度.
\item 若受到磁场 $B$,则能量和简并度有何改变.
\end{enumerate}