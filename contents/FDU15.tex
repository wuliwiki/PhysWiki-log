% 复旦大学 2015 年考研普通物理
% keys 复旦|考研|普通物理
% license Copy
% type Tutor

1. 图示系统处于同一铅垂平面内, 细长直杆 $A B$ 的长度为 $r$, 以匀角速度 $\omega$ 绕轴 $A$ 作顺时针转动, 通过其 $B$ 端与半径为 $R=2 r$ 的扇形板 $D G H$ 的圆弧边缘 $(D$ 为其圆心)接触, 从而带动扇形板绕轴 $O$ 转动, 且 $O D=\sqrt{3} R / 3$, 试求图示瞬时 ( $A B$ 处于水平位置, $D G$ 处于铅垂位置) 扇形板的角速度和角加速度。(25 分)

\begin{figure}[ht]
\centering
\includegraphics[width=7cm]{./figures/48419d71983a5ddf.png}
\caption{解答题 1 图示} \label{fig_FDU15_1}
\end{figure}

2. 真空中, 有一平行板电容器, 两块极板均为半径为 $a$ 的圆板, 将它连接到一个交变电源上, 使极板上的电荷按规律 $Q=Q_{0} \sin \omega t$ 随时间 $t$ 变化 (式中 $Q_{0}$ 和 $\omega$ 均为常量 $)$ 。在略去边缘效应的条件下,试求两极板间任一点的磁场强度 $\bvec{H}$。(15 分)

3. 如图所示, 细圆环管在相连部件带动下沿水平直线轨道纯滚动, 管内有一壁虎,相对于环管爬行,壁虎可被视为一点, 在图中以小球 $B$ 代替。图示瞬间, 壁虎与环管的中心处于同一水平线上, 壁虎相对环管的速率为 $u$, 相对速度的方向朝下,相对速度大小的改变率等于 $0$, 环管中心 $O$ 点的速度向右, 速度大小也为 $u$, 加速度为 $0$。环管的中心半径等于 $R$ 。求在此瞬时:
(15 分)

\begin{figure}[ht]
\centering
\includegraphics[width=5cm]{./figures/f24115e6ffb5ea41.png}
\caption{解答题 3 图示} \label{fig_FDU15_2}
\end{figure}

(1)壁虎相对地面的速度大小;

(2)壁虎相对地面的加速度大小;

(3)壁虎在相对地面的运动轨迹上所处位置点的曲率半径的大小。

4. 用钠光 $(\lambda=589.3 \mathrm{~nm})$ 观察迈克尔逊干涉条纹。由于观察屏的大小限制, 屏上只能看到有限个亮环。设开始时屏上有 $6$ 个亮环, 中心是亮的。移动平面镜 $R$ 后, 看到中心冒出了 $8$ 个亮环, 此时观察屏上共有 $10$ 个亮环, 中心也是亮的。求:(15 分)

(1) $M_{1}$ 移动的距离:

(2) $M_{1}$ 移动时, $M_{1}$ 和平面镜 $M_{2}$ 的虚像 $M_{2}^{\prime}$ 之间的等效空气膜是变薄, 还是变厚?

(3) 移动 $M_{1}$ 后, 等效空气膜的厚度。

5. 一平行板电容器的电容为 $100 p F$, 极板的面积为 $100 \mathrm{~cm}^{2}$, 极板间充满相对介电常数为 $5.4$ 的云母电介质,真空介电常数为 $8.85 \times 10^{-12}(F / m)$, 试求极板上电势差为 $50 V$ 时,(20 分)

(1)云母中的电场强度 $E$;

(2)电容器极板上的自由电荷;

(3)云母介质面上的极化面电荷。

6. 质量为 $m$ 、半径为 $r$ 的均质球置于水平桌面上, 受一水平方向的冲击力作用, 力的作用线通过球心, 力的冲量为 $J$, 试求:(20 分)

(1)求冲击刚结束时球心的速度 $v_{0}$;

(2)由于受到䍒面摩擦力的作用, 球最终将在桌面上作纯滚动, 求这时球的角速度 $\omega$;

(3) 求球与桌面间摩擦所消耗的能量。(已知均质球对过球心转轴的转动惯量为 $\left.\frac{2}{5} m r^{2}\right)$

7. 如图所示为一个方解石直角棱镜(左端)和一个普通光学玻璃棱镜(右端) 组成的双折射分束棱镜, 两棱镜的锐角均为 $45^{\circ}$, 光胶而成, 方解石棱镜的光轴方向如图所示。已知方解石 $n_{o}=1.6584, n_{e}=1.4864$, 玻璃棱镜折射率 $n=1.54$ 。(20 分)
\begin{figure}[ht]
\centering
\includegraphics[width=7cm]{./figures/d4afdcd4d5b22664.png}
\caption{解答题 7 图示} \label{fig_FDU15_3}
\end{figure}
(1) 分别画出自然光沿垂直于 $A B$ 面和沿垂直于 $B C$ 面的方向入射时, 从棱镜出射的光束方向。说明出射光的偏振态, 标出各光束的振动方向。

(2) 对于自然光垂直于 $A B$ 面入射的情形, 计算两束出射光之间的夹角。

8. 试证明:刚性绝热容器的放气过程中容器内理想气体的状态参数服从 $T p^{-\frac{k-1}{k}}=$ 常数

9. 一个焦距为 $60 \mathrm{~mm}$, 相对孔径为 $1: 2$ 的投影物镜,将物平面放大 $6$ 倍投影在 屏幕上, 如果屏幕上允许的几何弥散斑直径为 $0.3 \mathrm{~mm}$, 求基准物面前后的几何景深。
(20 分)

10. 相同形状、相同体积的导体和绝缘体均匀电介质,分别放入静电场中, 在 达到静电平衔以后。试回答
(20 分)
(1)两种情况下,电场变化的主要不同之处:
(2) 当电介质的介电常数 $\epsilon$ 趂向无限大时, 两种情况下电场变化的结果会相同吗?为什么?

11. 一个无限长的直圆柱导体, 内有一无限长的直圆柱空洞, 空洞的轴线与圆 柱体的轴线平行但不重合, 相距为 $d$. 如果有电流密度为 $\bvec{j}$ 的均匀电流沿导体的轴线方向流动, 计算空洞内的磁感应强度。(20 分)

12. 有一线度非常小的磁偶极子, 它具有指向 $x$ 方向的固有磁矩 $\bvec m$, 并正以速率 $v$ 沿 $x$ 轴运动。另有一半径为 $a$, 电阻为 $R$ 的细导线圆环, 固定在垂直于 $x$ 轴的平面上,圆心位于距磁偶极子距离为 $r$ 的 $x$ 轴上。试求此时磁偶极子受到的阻力。假设 $v \ll c, a \ll r$, 导线环的自感和磁偶极子磁矩的变化可忽略。(20 分)
