% 常系数线性齐次微分方程
% keys 复数|complex number|微分方程|高阶微分方程|ordinary differential equation|解法|特征方程|欧拉方程
% license Xiao
% type Tutor

\pentry{指数函数(复数)\nref{nod_CExp},线性微分方程的一般理论\nref{nod_ODEb1}}{nod_13e3}

本节讨论的是齐次方程的解法,是常系数线性方程中最简单的情况。非齐次方程的几种解法请参见\enref{拉普拉斯变换与常系数线性微分方程}{ODELap}文章。

\addTODO{引用所有常系数非齐次方程解法的文章。}

\subsection{实数轴上的复值函数}

首先介绍一个很有用的概念,复值函数。

复值函数和复变函数是不一样的。复变函数中的“变”指变量,因此复变函数是复数域到复数域的映射;但复值函数的自变量还是实数,它只是实数域到复数域的映射而已。

由于任何复数都可以写成 $a+\I b$ 的形式,其中 $a, b\in\mathbb{R}$,因此一个复值函数可以看成是两个实值函数的组合。

设 $f(x), g(x)$ 是某给定区间上的连续实值函数,那么复值函数 $z(x)=f(x)+\I g(x)$ 的极限被定义为 $\lim\limits_{x\to x_0}z(x)=\lim\limits_{x\to x_0}f(x)+\I\lim\limits_{x\to x_0}g(x)$。进一步,$z(x)$ 的导函数是 $z'(x)=f'(x)+\I g'(x)$。

复数的指数由欧拉公式定义:对于实数 $a, b$,有
\begin{equation}
\E^{a+\I b}=\E^a(\cos b+\I\sin b)~.
\end{equation}

如果 $K$ 是一个常数复数,那么 $\E^{Kt}$ 就是一个关于实变量 $t$ 的复值函数。这个函数继承了实值函数 $\E^{at}$ 的很多优良性质。

\begin{theorem}{}
设 $K, K_i$ 是复常数,$z(t), z_i(t)$ 是区间 $[a, b]$ 上可导的复值函数,那么我们有:
\begin{itemize}
\item $\frac{\dd }{\dd t}(K_1z_1(t)+K_2z_2(t))=K_1z_1'(t)+K_2z_2'(t)$,
\item $\frac{\dd}{\dd t}(z_1(t)z_2(t))=z'_1(t)z_2(t)+z_1(t)z'_2(t)$,
\item $\E^{(K_1+K_2)t}=\E^{K_1t}\E^{K_2t}$,
\item $\frac{\dd }{\dd t}\E^{Kt}=K\E^{Kt}$.
\end{itemize}
\end{theorem}

另外,如果记 $\overline{K}$ 是复数 $K$ 的共轭\footnote{即如果存在实数 $a, b$ 使 $K=a+\I b$,那么 $\overline{K}=a-\I b$。},那么我们还有
\begin{equation}
\E^{\overline{K}}=\overline{\E^K}~.
\end{equation}

复值函数也可以作常系数常微分方程的解。考虑到常微分方程的全体解构成一个线性空间,复值函数可以看成是把这个解空间拓展为一个复数域上的线性空间。如果 $\varphi(x)$、$\phi(x)$ 是某个常微分方程的实值解,那么 $\I\varphi(x)$、$\varphi(x)+\I \phi(x)$ 也都是同一个方程的解。


\subsection{常系数齐次线性微分方程}



由字面意义,可得定义:
\begin{definition}{}\label{def_ODEb2_1}
形如
\begin{equation}\label{eq_ODEb2_1}
\frac{\mathrm{d}^n}{\dd t^n}x(t)+a_1\frac{\mathrm{d}^{n-1}}{\dd t^{n-1}}x(t)+\cdots+a_{n-1}\frac{\mathrm{d}}{\dd t}x(t)+a_nx(t)=0~
\end{equation}
的方程,称为\textbf{常系数齐次线性微分方程}。其中各 $a_i$ 都是常数。
\end{definition}

接下来我们分情况,详细讨论\autoref{def_ODEb2_1} 中方程的解法。

\subsubsection{无重根的特征方程}


求导运算中最方便的函数是什么?是 $\E^{Kt}$,因为其结果总是正比于原来的函数。假设存在复数 $\lambda$,使得 $x(t)=\E^{\lambda t}$ 是\autoref{eq_ODEb2_1} 的解,那么我们得到
\begin{equation}\label{eq_ODEb2_2}
\E^{\lambda t}\qty(\lambda^n+a_1\lambda^{n-1}+\cdots+a_{n-1}\lambda+a_n)=0~.
\end{equation}

由于 $\E^{\lambda t}$ 处处不为 $0$,\autoref{eq_ODEb2_2} 就可以化为
\begin{equation}\label{eq_ODEb2_3}
\lambda^n+a_1\lambda^{n-1}+\cdots+a_{n-1}\lambda+a_n=0~.
\end{equation}

\autoref{eq_ODEb2_3} 就被称为\autoref{eq_ODEb2_1} 的\textbf{特征方程(characteristic equation)}或\textbf{辅助方程(auxiliary equation)}。

如果从\autoref{eq_ODEb2_3} 中解出 $\lambda$\footnote{$\lambda$ 可以是复数,这就是为什么我们要先介绍复值函数的概念。},那么 $x(t)=\E^{\lambda t}$ 就是\autoref{eq_ODEb2_1} 的一个解。根据预备知识中的\autoref{the_ODEb1_1},如果\autoref{eq_ODEb2_3} 有 $n$ 个互不相同的解 $\lambda_i$,那么 $\{\E^{\lambda_i t}\}$ 就构成了\autoref{eq_ODEb2_1} 的一组基解。这样,我们就把解\textbf{常系数齐次线性微分方程}化为解\textbf{代数方程}了。


事实上,由于我们常把高阶求导写成低阶求导的乘积形式,比如 $\frac{\mathrm{d}^3}{\dd t^3}=\frac{\dd }{\dd t}\cdot\frac{\dd }{\dd t}\cdot\frac{\dd }{\dd t}$,,因此\autoref{eq_ODEb2_3} 形式上相当于将 $\frac{\dd }{\dd t}$ 替换为 $\lambda$ 并消去 $x(t)$ 后的结果。之所以我们可以将 $\lambda$ 视为一个数字去解,也是因为 $\E^{\lambda t}$ 在微分算子作用下的特殊性。


\begin{example}{}
考虑方程
\begin{equation}
\frac{\mathrm{d}^2}{\dd t^2}x(t)-\frac{\dd }{\dd t}x(t)-2x(t)=0~.
\end{equation}
其特征方程为
\begin{equation}
(\lambda-2)(\lambda+1)=0~,
\end{equation}

因此其通解为

\begin{equation}
x(t)=A\E^{2t}+B\E^{-t}~.
\end{equation}
其中 $A, B$ 是常数。




\end{example}






\subsubsection{有重根的情况}



有的时候,特征方程存在重根,导致无法解出 $n$ 个不同的 $\lambda$,这时候该怎么办呢?我们先观察以下例子。

\begin{example}{}\label{ex_ODEb2_1}
考虑方程
\begin{equation}\label{eq_ODEb2_4}
\frac{\mathrm{d}^2}{\dd t^2}x(t)-2\frac{\dd }{\dd t}x(t)+x(t)=0~.
\end{equation}

其特征方程为
\begin{equation}
(\lambda-1)^2=0~,
\end{equation}
只能解出一个二重根 $\lambda=1$。按照之前的讨论,我们只能给出一个基解 $\E^{t}$,还有一个解怎么办呢?

尝试把 $t\E^{t}$ 代进\autoref{eq_ODEb2_4} 试试,我们发现这也是一个解。这样,我们就有了两个线性无关解,也就解出了\autoref{eq_ODEb2_4}。




\end{example}




为什么\autoref{ex_ODEb2_1} 中给出的 $t\E^{t}$ 也构成一个解呢?

我们首先考虑 $k$ 重根里最简单的情况,即这个根为 $\lambda_0=0$。这意味着特征方程里有一个因子 $\lambda^k$,进而意味着\autoref{eq_ODEb2_1} 里有一个因子 $\frac{\mathrm{d}^k}{\dd t^k}=\qty(\frac{\dd}{\dd t})^k$,故原方程的形式应为
\begin{equation}
\qty(\frac{\mathrm{d}^n}{\dd t^n}+a_1\frac{\mathrm{d}^{n-1}}{\dd t^{n-1}}+\cdots+a_{n-k}\frac{\mathrm{d}^k}{\dd t^k})x(t)=0~.
\end{equation}
或更直观地,
\begin{equation}
\begin{aligned}
&\qty(\frac{\dd}{\dd t})^k\qty(\frac{\mathrm{d}^{n-k}}{\dd t^{n-k}}+a_1\frac{\mathrm{d}^{n-k-1}}{\dd t^{n-k-1}}+\cdots+a_{n-k})x(t)\\=&0\\
=&\qty(\frac{\mathrm{d}^{n-k}}{\dd t^{n-k}}+a_1\frac{\mathrm{d}^{n-k-1}}{\dd t^{n-k-1}}+\cdots+a_{n-k})\qty(\frac{\dd}{\dd t})^kx(t)~,
\end{aligned}
\end{equation}
相当于我们有两个方程
\begin{equation}
\leftgroup{
    &\qty(\frac{\dd}{\dd t})^kx(t)=0\\
    &\qty(\frac{\mathrm{d}^{n-k}}{\dd t^{n-k}}+a_1\frac{\mathrm{d}^{n-k-1}}{\dd t^{n-k-1}}+\cdots+a_{n-k})x(t)=0~.
}
\end{equation}

事实上,如果将特征方程拆分成 $(\lambda-\lambda_1)^{k_1}(\lambda-\lambda_2)^{k_2}\cdots$ 的形式,原方程实际上就是在解每一个 $\qty(\frac{\dd}{\dd t}-\lambda_i)^{k_i}x(t)=0$。因此,我们只需要考虑形如 $\qty(\frac{\dd}{\dd t}-\lambda_0)^kx(t)=0$ 的方程即可。

在最简单情形下,我们要考虑的方程是
\begin{equation}
\qty(\frac{\dd}{\dd t})^kx(t)=0~,
\end{equation}
易验证它的一个基本解组是 $\{1, t, t^2, \cdots, t^{k-1}\}$。这是因为形如 $t^m$ 的多项式函数,能够在 $\qty(\frac{\dd}{\dd t})^k$ 算子作用下变成 $0$,其中 $k>m$。

那么,对于一般的情况,即
\begin{equation}\label{eq_ODEb2_5}
\qty(\frac{\dd}{\dd t}-\lambda_0)^kx(t)=0~,
\end{equation}
我们要考虑的就是什么样的 $x(t)$ 能被 $\qty(\frac{\dd}{\dd t}-\lambda_0)^k$ 变成 $0$。

答案自然是形如 $t^m\E^{\lambda_0 t}$ 的函数。

首先观察到
\begin{equation}
\qty(\frac{\dd}{\dd t}-\lambda_0)t^m\E^{\lambda_0 t}=mx^{m-1}\E^{\lambda_0 t}~,
\end{equation}
因此每个 $\qty(\frac{\dd}{\dd t}-\lambda_0)$ 算子就像是在单独给 $t^m$ 降阶。

于是
\begin{equation}
\qty(\frac{\dd}{\dd t}-\lambda_0)^mt^m\E^{\lambda_0 t}=m!\E^{\lambda_0 t}~,
\end{equation}
再作用一次,就得到
\begin{equation}
\qty(\frac{\dd}{\dd t}-\lambda_0)^{m+1}t^m\E^{\lambda_0 t}=0~.
\end{equation}
因此,对于任意 $m<k$,$t^m\E^{\lambda_0 t}$ 都是\autoref{eq_ODEb2_5} 的解。

这样,我们就得到齐次线性微分方程的完整通解了:

\begin{theorem}{齐次线性微分方程的通解}
将齐次线性微分方程\autoref{eq_ODEb2_1} 分解为如下形式:
\begin{equation}\label{eq_ODEb2_6}
\qty(\frac{\dd}{\dd t}-\lambda_1)^{k_1}\qty(\frac{\dd}{\dd t}-\lambda_2)^{k_2}\cdots\qty(\frac{\dd}{\dd t}-\lambda_r)^{k_r}x(t)=0~.
\end{equation}
换句话说,就是\autoref{eq_ODEb2_1} 的特征方程有根 $\lambda_1, \lambda_2, \cdots, \lambda_r$,且 $\lambda_i$ 的重数为 $k_i$。

那么\autoref{eq_ODEb2_1} 或者说\autoref{eq_ODEb2_6} 的基本解组为
\begin{equation}
\{
    \E^{\lambda_i t}, t\E^{\lambda_i t}, \cdots, t^{k_i-1}\E^{\lambda_i t}
\}~,
\end{equation}
其中 $i\in\{1, 2, 3, \cdots, r\}$。



\end{theorem}









