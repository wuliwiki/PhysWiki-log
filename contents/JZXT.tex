% 巨正则系综(综述)
% license CCBYSA3
% type Wiki

本文根据 CC-BY-SA 协议转载翻译自维基百科\href{https://en.wikipedia.org/wiki/Grand_canonical_ensemble}{相关文章}。

在统计力学中,巨正则系综(grand canonical ensemble,亦称宏观正则系综)是一种用于描述与热库处于热力学平衡(热平衡与化学平衡)的粒子力学系统可能状态的统计系综\(^\text{[1]}\)。该系综中的系统具有开放性特征——系统可与热库交换能量和粒子,因此系统的不同微观态可能具有不同的总能量与总粒子数。系统的体积、形状及其他外部坐标在所有可能微观态中保持恒定。

巨正则系综的热力学变量为化学势(符号:\(\mu\))与绝对温度(符号:\(T\))。该系综的特性还受体积(符号:\(V\))等力学变量的影响,这些变量决定了系统内部状态的性质。因此,该系综常被称为\(\mu VT\)系综,因为这三个物理量均为该系综的固定参数。
\subsection{基础概念}
在统计力学中,巨正则系综的基础概念可表述如下:该系综对每个微观态赋予的概率\(P\)由以下指数形式给出:
\[
P = e^{(\Omega + \mu N - E)/(kT)},~
\]
式中,\(N\)表示微观态中的粒子数,\(E\)为微观态的总能量,\(k\)为玻尔兹曼常数。  

物理量Ω称为巨势,对特定系综而言为常数。当选择不同的\(\mu,V,T\)参数时,概率分布和\(\Omega\)值将发生改变。巨势Ω具有双重作用:其一作为概率分布的归一化因子(所有微观态的概率之和必须为1);其二可通过函数\(\Omega(\mu,V,T)\)直接计算许多重要的系综平均值。  

当系统允许多种粒子数变化时,概率表达式推广为:  
\[
P = e^{(\Omega + \mu_1N_1 + \mu_2N_2 + \ldots + \mu_sN_s - E)/(kT)},~
\]  
其中,\(\mu_1\)表示第一种粒子的化学势,\(N_1\)为对应微观态中该种粒子数,依此类推(\(s\)为不同粒子种类总数)。需注意粒子数的精确定义(参见下文关于粒子数守恒的注释)。  

部分学者将巨正则系综的分布称为广义玻尔兹曼分布\(^\text{[2]}\)。

巨正则系综特别适用于描述以下系统:系统的几何形状固定,但因与热库接触(例如导体中的电子对应电接地,空腔中的光子对应吸光表面),其能量和粒子数可发生显著涨落。该系综为严格推导无相互作用量子粒子系统的费米-狄拉克统计或玻色-爱因斯坦统计提供了自然框架(参见下文示例)。 

\textbf{公式注记} 

该概念的另一表述形式将概率写为:\(P = \frac{1}{\mathcal{Z}} e^{(\mu N - E)/(kT)},\)  
其中通过巨配分函数\(\mathcal{Z} = e^{-\Omega/(kT)}\)替代了巨势Ω。本文中以巨势为基础的公式,均可通过简单的数学变换,利用巨配分函数重新表述。  
\subsection{适用性}  
巨正则系综是描述与一个热库保持热平衡和化学平衡的孤立系统可能状态的系综(推导过程类似于热浴推导正常正则系综的方法,可以在Reif\(^\text{[3]}\)中找到)。巨正则系综适用于任何巨小的系统,无论是小型还是巨型;只需要假设与其接触的热库要巨得多(即取宏观极限)。

系统被孤立的条件是必要的,以确保其具有明确的热力学量和演化。\(^\text{[1]}\)然而,在实际应用中,通常希望使用巨正则系综来描述与热库直接接触的系统,因为正是这种接触保证了系统的平衡。在这些情况下,使用巨正则系综通常是通过以下方式之一来合理化的:1)假设接触是微弱的,或者 2)将一部分热库的连接纳入到分析中的系统中,以便正确地模拟连接对感兴趣区域的影响。另一种方法是使用理论方法来建模连接的影响,从而得到一个开放的统计系综。

另一个出现巨正则系综的情况是考虑一个巨型的热力学系统(即“与自身处于平衡状态的系统”)。即使系统的精确条件实际上不允许能量或粒子数的变化,巨正则系综仍然可以用来简化某些热力学性质的计算。原因在于,一旦系统非常巨,各种热力学系综(微正则系综、正则系综)在某些方面会变得与巨正则系综等效。\(^\text{[注 1]}\)当然,对于小系统,即使在均值上,不同的系综也不再是等效的。因此,当应用于粒子数固定的小系统(如原子核)时,巨正则系综可能会非常不准确。\(^\text{[4]}\)
\subsection{性质}  
\begin{itemize}
\item 唯一性:对于给定温度和化学势的系统,大正则系综是唯一确定的,并且不依赖于任意选择,如坐标系的选择(经典力学)或基底的选择(量子力学)。\(^\text{[1]}\) 大正则系综是唯一一个在常温、常体积和常化学势下重现基本热力学关系的系综。\(^\text{[5]}\)  
\item 统计平衡(稳态):尽管底层系统在不断运动,大正则系综本身并不会随时间演化。实际上,系综仅仅是系统守恒量(能量和粒子数)的函数。\(^\text{[1]}\)  
\item 与其他系统的热平衡和化学平衡:两个系统,各自由大正则系综描述,且具有相等的温度和化学势,若接触并达到热平衡和化学平衡[note 2],它们将保持不变,且最终合并的系统将由同样的温度和化学势的大正则系综描述。\(^\text{[1]}\) 
\item 最大熵:对于给定的机械参数(固定\(V\)),大正则系综中对对数概率的平均值 \( -\langle \log P \rangle \)(也称为“熵”)是对于任何具有相同 \( \langle E \rangle \)、 \( \langle N_1 \rangle \) 等的系综(即概率分布\(P\))所能达到的最大值。\(^\text{[1]}\) 
\item 最小大势能:对于给定的机械参数(固定\(V\))以及给定的\(T,\mu_1,\cdots,\mu_s\)的值,大正则系综的平均值\({\displaystyle \left\langle E + kT\log P - \mu_{1}N_{1} - \ldots \mu_{s}N_{s} \right\rangle}\)是任何系综中可能的最小值。\(^\text{[1]}\)
\end{itemize}
\subsection{大势能、系综平均和精确微分}
函数\(\Omega(\mu_1,\dots,\mu_s,V,T)\)的偏导数给出了重要的巨正则系综平均量:\(^\text{[1][6]}\)
\begin{itemize}
\item 粒子数的平均值  
\[\langle N_1 \rangle = -\frac{\partial \Omega}{\partial \mu_1}, \quad \dots \quad \langle N_s \rangle = -\frac{\partial \Omega}{\partial \mu_s}~\],  
\item 平均压强  
\[\langle p \rangle = -\frac{\partial \Omega}{\partial V},~\]
\item 吉布斯熵  
\[S = -k \langle \log P \rangle = -\frac{\partial \Omega}{\partial T},~\]
\item 以及平均能量  
\[\langle E \rangle = \Omega + \langle N_1 \rangle \mu_1 + \dots + \langle N_s \rangle \mu_s + S T.~\]
\end{itemize}
精确微分:从上述表达式可以看出,函数 Ω 具有精确的微分
\[
d\Omega = -S dT - \langle N_1 \rangle d\mu_1 - \dots - \langle N_s \rangle d\mu_s - \langle p \rangle dV.~
\]
热力学第一定律:将上述关系式代入 Ω 的精确微分,得到类似热力学第一定律的方程,唯一不同的是某些量带有平均符号:\(^\text{[1]}\)
\[
d\langle E \rangle = T dS + \mu_1 d\langle N_1 \rangle + \dots + \mu_s d\langle N_s \rangle - \langle p \rangle dV.~
\]
热力学波动:能量和粒子数的方差为\(^\text{[7][8]}\)
\[
\langle E^2 \rangle - \langle E \rangle^2 = k T^2 \frac{\partial \langle E \rangle}{\partial T} + k T \mu_1 \frac{\partial \langle E \rangle}{\partial \mu_1} + k T \mu_2 \frac{\partial \langle E \rangle}{\partial \mu_2} + \dots ,~
\]
\[
\langle N_1^2 \rangle - \langle N_1 \rangle^2 = k T \frac{\partial \langle N_1 \rangle}{\partial \mu_1}.~
\]
波动中的关联:粒子数和能量的协方差为\(^\text{[1]}\)
\[
\langle N_1 N_2 \rangle - \langle N_1 \rangle \langle N_2 \rangle = k T \frac{\partial \langle N_2 \rangle}{\partial \mu_1} = k T \frac{\partial \langle N_1 \rangle}{\partial \mu_2}.~
\]
\[
\langle N_1 E \rangle - \langle N_1 \rangle \langle E \rangle = k T \frac{\partial \langle E \rangle}{\partial \mu_1}.~
\]
\subsection{例子系综}
巨正则系综的有用性在下面的例子中得到了说明。在每个例子中,巨正则势能是基于以下关系计算的:
\[
\Omega = -kT \ln \left( \sum_{\text{microstates}} e^{(\mu N - E)/kT}\right)~
\]
这个关系是为了使微观状态的概率总和为1所必需的。
\subsubsection{非相互作用粒子的统计}

\textbf{玻色子和费米子(量子})

在量子系统中,如果有许多非相互作用的粒子,则热力学计算是简单的。\(^\text{[9]}\)由于粒子是非相互作用的,可以计算一系列单粒子驻留态,每个驻留态代表一个可以被包含进系统总量子态的可分离部分。为了避免将这些“状态”与总的多体态混淆,我们暂时称这些单粒子驻留态为轨道,并规定每个可能的粒子内部属性(如自旋或极化)都算作一个单独的轨道。每个轨道可以被粒子(或多个粒子)占据,或者保持为空。

由于粒子是非相互作用的,我们可以从以下角度来看待:每个轨道形成一个独立的热力学系统。因此,每个轨道都是一个独立的巨正则系综,它如此简单,以至于其统计可以立即在此推导出来。我们聚焦于仅一个标记为\(i\)的轨道,对于该轨道中\(N_i\)个粒子的一个微观状态,其总能量为\(N_i\epsilon_i\),其中\(\epsilon_i\)是该轨道的特征能级。该轨道的巨正则势能由以下两种形式之一给出,具体取决于该轨道是玻色子轨道还是费米子轨道:

\begin{itemize}
\item 对于费米子,泡利排斥原理只允许轨道有两个微观状态(占据0或1),因此得到一个二项级数:
\[
\Omega_i = -kT \ln \left( \sum_{N_i=0}^{1} e^{(N_i \mu - N_i \epsilon_i)/kT} \right) = -kT \ln \left( 1 + e^{(\mu - \epsilon_i)/kT} \right)~
\]
\item 对于玻色子,\(N_i\)可以是任何非负整数,并且由于粒子不可区分,每个\(N_i\)的值都算作一个微观状态,从而得到一个几何级数:
\[
\Omega_i = -kT \ln \left( \sum_{N_i=0}^{\infty} e^{(N_i \mu - N_i \epsilon_i)/kT} \right) = +kT \ln \left( 1 - e^{(\mu - \epsilon_i)/kT} \right)~
\]
\end{itemize}
在每种情况下,值\(\langle N_i \rangle = -\frac{\partial \Omega_i}{\partial \mu}\)给出了轨道上粒子的热力学平均数:费米-狄拉克分布适用于费米子,玻色-爱因斯坦分布适用于玻色子。再次考虑整个系统,通过将所有轨道的\(\Omega_i\)相加,得到总的巨正则势能。

\textbf{不可区分的经典粒子}

在经典力学中,同样可以考虑不可区分的粒子(事实上,不可区分性是以一致的方式定义化学势的前提条件;所有给定种类的粒子必须是可以互换的\(^\text{[1]}\))。我们可以将具有大致均匀能量\(\epsilon_i\)的单粒子相空间区域视为一个标记为 \(i\)的“轨道”。

由于这个轨道实际上包含许多(无限多个)不同的状态,因此会出现两个复杂问题。简要来说:
\begin{itemize}
\item 由于多粒子相空间中包含了\(N_i!\)个相同的实际状态副本(由粒子的不同精确状态排列组合形成),因此需要一个\(1/N_i!\)的重计数修正。
\item 轨道的宽度是任意选择的,因此还需要一个与\(N_i\)无关的比例因子。
\end{itemize}
由于\(1/N_i!\)的重计数修正,求和现在呈现出指数级幂级数的形式,
\[
\Omega_i \propto -kT \ln \left( \sum_{N_i=0}^{\infty} \frac{1}{N_i!} e^{(N_i \mu - N_i \epsilon_i)/kT} \right)
\propto -kT \ln \left( e^{e^{(\mu - \epsilon_i)/kT}} \right)
\propto -kT e^{\frac{\mu - \epsilon_i}{kT}},~
\]
值\(\langle N_i \rangle \propto -\frac{\partial \Omega_i}{\partial \mu}\)对应于麦克斯韦-玻尔兹曼统计。
\subsubsection{孤立原子的电离}
巨正则系综可以用来预测一个原子更倾向于处于中性状态还是电离状态。一个原子能够存在于与中性状态相比,电子数更多或更少的电离状态。如下面所示,电离状态可能在热力学上更为优选,具体取决于环境。考虑一个简化模型,其中原子可以处于中性状态或两个电离状态之一(详细计算还包括激发态和状态的简并因子\(^\text{[10][11]}\)):
