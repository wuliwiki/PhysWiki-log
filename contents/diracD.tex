% 狄拉克场的传播子
% keys 传播子|两点编时格林函数|推迟传播子
% license Xiao
% type Tutor
\pentry{狄拉克场的量子化\upref{Diracq}}{nod_fe5f}

我们先计算出电子从 $y$ 到 $x$ 传播的振幅与正电子从 $x$ 到 $y$ 传播的振幅,计算过程中利用了旋量自旋求和\upref{spinsu}的结果:
\begin{equation}
\begin{aligned}
\langle 0 | \psi_a(x) \bar\psi_b(y)|0\rangle &= \langle 0|\int \frac{\dd[3] \bvec p}{(2\pi)^3} \frac{1}{2E_{\bvec p}}\sum_s u^s_a(p)\bar u^s_b(p) e^{-ip(x-y)} |0\rangle\\
&=(i\not \partial_x+m)_{ab}\int \frac{\dd[3] \bvec p}{(2\pi)^3}\frac{1}{2E_{\bvec p}}e^{-ip(x-y)}~,\\
\langle 0 | \bar\psi_b(y) \psi_a(x)|0\rangle &=-(i\not\partial_x +m)_{ab}\int \frac{\dd[3] \bvec p}{(2\pi)^3}\frac{1}{2E_{\bvec p}}e^{-ip(y-x)}~.
\end{aligned}
\end{equation}
由于积分测度 $\int \frac{\dd[3]\bvec  p}{(2\pi)^3}\frac{1}{2E_{\bvec p}}$ 的洛伦兹不变性,对于类空间隔的两个点 $x,y$,上面两个振幅相加为 $0$,即 $\psi_\alpha(x)$ 和 $\bar\psi_\beta(y)$ 的反对易子为 $0$。这是 Dirac 场情形下因果性的体现。我们可以定义 Dirac 场的推迟传播子
\begin{equation}
\begin{aligned}
S^{ab}_R(x-y)&=\theta(x^0-y^0) \langle 0| \{\psi_a(x),\bar\psi_b(y) \} | 0\rangle = (i\not \partial_x+m)_{ab} D_R(x-y)\\
&=(i\not\partial_x+m)_{ab}\int \frac{\dd[4] p}{(2\pi)^4}\left(\frac{i}{p^2-m^2}\right)e^{-ip(x-y)}\\
&=\int \frac{\dd[4] p}{(2\pi)^4}\frac{i(\not p+m)_{ab}}{p^2-m^2}e^{-ip(x-y)}~.
\end{aligned}
\end{equation}
利用等式 $\not \partial \not \partial=\gamma^\mu\partial_\mu\gamma^\nu\partial_\nu=\partial^2$,可以得出,$S_R(x-y)$ 是 Dirac 算符的推迟格林函数,即
\[
(i\not \partial_x-m)S_R(x-y) = i\delta^{(4)}(x-y) \cdot \mathbb{1}_{4\times 4}~.
\]
我们后面将要用到的仍然是 Feynman 传播子,也被称为编时传播子
\begin{equation}
\begin{aligned}
S_F(x-y)&=\int \frac{\dd[4] p}{(2\pi)^4}\frac{i(\not p+m)}{p^2-m^2+i\epsilon}e^{-ip(x-y)}\\
\notag&=\langle 0|T\psi(x)\bar\psi(y)|0\rangle \\
\notag&=\theta(x^0-y^0) \langle 0|\psi(x)\bar\psi(y) |0\rangle -\theta(y^0-x^0)\langle 0|\bar\psi(y)\psi(x)|0\rangle~.
\end{aligned}
\end{equation}
类似地,也可以证明 $S_F(x-y)$ 是 Dirac 算符的格林函数
\[
(i\not \partial_x-m)S_F(x-y) = i\delta^{(4)}(x-y) \cdot \mathbb{1}_{4\times 4}~.
\]
