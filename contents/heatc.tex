% 热传导定律、扩散方程与输运过程
% keys 热传导|输运过程|传递过程|粘滞定律|菲克定律
% license Xiao
% type Tutor

\pentry{热传导定律\nref{nod_Heatco}}{nod_a6ef}
%建议和上述条目合并

\subsection{傅里叶传导定律}
定义热流密度 $\bvec h$:单位时间单位面积的热流量。傅里叶热传导定律说的是

\begin{equation}
\bvec h = -\kappa \bvec \nabla T~,
\end{equation}

其中 $\kappa$ 为系统的热导率。这意味着热量会沿温度梯度的方向传导。

如果简化系统模型,设各层温度不均匀,在同一个高度上各处温度相等。那么傅里叶热传导定律可以简化为
\begin{equation}
H=\frac{\Delta Q}{\Delta t}=-\kappa \frac{\dd T}{\dd z}S~.
\end{equation}

再考虑一般的情况:\textbf{三维热传导方程}。由于单位体积内需要吸收 $c\rho \Delta T$ 的热量才能升高 $\Delta T$ 的温度,所以可以得到积分关系式
\begin{equation}\label{eq_heatc_1}
\int_V c\rho \frac{\partial T}{\partial t} \dd V = \int_V \frac{\partial U}{\partial t} \dd V= \oint_S \kappa \bvec \nabla T \cdot \bvec \dd S~.
\end{equation}

由此可以得到微分表达式 $c\rho \frac{\partial T}{\partial t} = \kappa \nabla^2 T$,即
\begin{equation}\label{eq_heatc_2}
\frac{\partial T}{\partial t}=k\nabla^2 T~,
\end{equation}
其中 $k=\kappa/c\rho$ 称作热扩散率。

注意上面讨论的是\textbf{没有热源、热扩散系数处处相等}的情况。如果有热源,则方程右端要加上 $Q$。如果热扩散系数并不处处相等,例如两个不同的介质的交界处,往往要对此设定边界条件——例如一维杆两端与空气接触的热对流现象、开水与空气接触时逐渐散热的现象等;在这类边界上如果温度差不大,有\textbf{牛顿冷却定律}成立:热流密度 $\bvec h$ 与温度的梯度成正比,但这个比例系数将与各种复杂的因素有关。

\begin{example}{平衡温度分布}
一维杆 $0\le x\le L$,左端右端与恒温热源接触,左端温度恒为 $T_1$,右端温度为 $T_2$,中间部分绝热。求其平衡温度分布。

平衡态热流处处为 $0$,所以 $\nabla^2 T=0$,即 $\partial^2 T/\partial x^2=0$,所以 $T$ 关于 $x$ 的函数是一次函数。
\begin{equation}
T(x)=T_1+\frac{T_2-T_1}{L}x~.
\end{equation}

\begin{figure}[ht]
\centering
\includegraphics[width=12cm]{./figures/c1825b863a5ccc5f.pdf}
\caption{平衡温度分布} \label{fig_heatc_2}
\end{figure}
\end{example}

\begin{example}{恒源扩散}
一维杆 $0\le x\le +\infty$,初始温度(与无穷远处温度)为$T_0$,左端与恒温热源$T_s$接触,求杆各处温度随时间的变化。

解微分方程,得
\begin{equation}
T(x,t)=T_0+(T_s-T_0)\left(1-erf\left(\frac{x}{2\sqrt{kt}}\right)\right)~,
\end{equation}
其中$erf(x)=\frac{2}{\sqrt{\pi}}\int^x_0 e^{-t^2} \dd t$被称为高斯误差函数(很遗憾,这个积分没有初等形式的表达式)

\begin{figure}[ht]
\centering
\includegraphics[width=12cm]{./figures/161f7e1e10ea8dac.pdf}
\caption{恒源扩散,\href{https://wuli.wiki/apps/diffus.html}{一个可视化的动图}} \label{fig_heatc_1}
\end{figure}
\end{example}

\subsection{传递过程}
当系统处于非平衡态时,会自发地向平衡态过度,从而产生\textbf{动量、能量、质量}等宏观的流动,这些过程统称为\textbf{耗散过程}。传递过程(也叫输运过程)在微观上就是耗散过程。例如当热学平衡条件不满足时,有温度梯度,从而有热传导方程(能量的传递);力学平衡条件不满足时,有粘滞现象(动量的传递),从而有牛顿粘滞定律;化学平衡条件不满足时,有扩散现象(质量的传递),从而有菲克(Fick)扩散定律。我们先给出这三个定律的表达式:
\begin{align}
h=-\kappa \frac{\dd T}{\dd z}~,\\
J_p=-\eta \frac{\dd u}{\dd z}~,\\
J_M=-D\frac{\dd \rho}{\dd z}~.
\end{align}

其中菲克定律的 $J_M$ 为质量流密度,该方程能很好地解释气体扩散现象,但对于液体并不成立,原因是液体分子不像气体分子那样自由运动。但该方程也能描述液体中粒子的浓度差带来的扩散现象,例如墨汁在水中的扩散,葡萄糖分子在水中的扩散等。

其中牛顿粘滞方程的 $J_p$ 为动量流密度,一些参考文献上将粘滞定律写作 $\tau = -\mu \frac{\dd u}{\dd z}$,其中 $\tau$ 为剪应力,该方程则表达了剪应力与流体速度场的梯度成正比。这些方程从直观上是容易想象的,虽然宏观上代表不同的现象,但其方程形式却是相同的。在文章“\enref{气体输运过程”}{gasTra}中,我们将发现这三种输运过程在\textbf{微观上的机制}本质是相同的。

\begin{example}{菲克第二定律:扩散方程}
类似于温度的 \autoref{eq_heatc_1},物质扩散过程中我们有局域物质守恒。局域物质守恒意味着一定时间内离开或进入某个区域的物质总量,等于该时间内这个区域内物质的变化量:
\begin{equation}
\pdv{\rho}{t} + \div \bvec J = 0~,
\end{equation}
向守恒方程带入上述的菲克扩散定律(也称菲克第一定律) $\bvec J_M=-D \grad \rho$ ($J_M=-D \dv{\rho}{z}$的一般形式),就可以得到菲克第二定律,即我们熟知的扩散方程。 形式上这与 \autoref{eq_heatc_2} 类似:
\begin{equation}
\pdv{\rho}{t} - D \nabla^2 \rho = 0~,
\end{equation}
在一维情况下,
\begin{equation}
\pdv{\rho}{t} - D \pdv[2]{\rho}{z} = 0~.
\end{equation}
\end{example}

