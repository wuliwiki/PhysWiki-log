% 佐恩引理(综述)
% license CCBYSA3
% type Wiki

本文根据 CC-BY-SA 协议转载翻译自维基百科\href{https://en.wikipedia.org/wiki/Zorn\%27s_lemma}{相关文章}。

\begin{figure}[ht]
\centering
\includegraphics[width=6cm]{./figures/d0b4921eea445e97.png}
\caption{佐恩引理可以用来证明每个连通图都有一个生成树。所有是树的子图构成一个由包含关系排序的集合,而链的并集是一个上界。佐恩引理指出,必须存在一个极大树,而这个极大树就是生成树,因为图是连通的。\(^\text{[1]}\)对于有限图,例如这里所示的图,实际上不需要使用佐恩引理。} \label{fig_ZornS_1}
\end{figure}
佐恩引理,也称为库拉托夫斯基–佐恩引理,是集合论中的一个命题。它表明,对于一个包含每个链(即每个全序子集)上界的偏序集,该集合必定包含至少一个极大元素。

该引理由卡济米日·库拉托夫斯基于1922年证明(假设选择公理),并由马克斯·佐恩于1935年独立证明。\(^\text{[2]}\)它出现在若干关键性定理的证明中,例如泛函分析中的哈恩–巴拿赫定理、每个向量空间都有一个基的定理,\(^\text{[3]}\) 拓扑学中的泰赫诺夫定理(即每个紧致空间的乘积是紧致的),以及抽象代数中的定理(在带有单位的环中,每个真理想都包含在一个极大理想中,且每个域都有代数闭包)\(^\text{[4]}\)。

佐恩引理与良序定理等价,并且与选择公理等价,意味着在ZF(不含选择公理的泽梅洛–弗兰克尔集合论)中,任何一个引理都足以证明另外两个。\(^\text{[5]}\)佐恩引理的早期表述是豪斯多夫最大原理,它表明给定偏序集的每个全序子集都包含在该偏序集的一个极大全序子集中。\(^\text{[6]}\)
\subsection{动机}  
为了证明一个数学对象的存在,这个对象可以被视为某个偏序集中某种方式下的极大元素,可以尝试通过假设不存在极大元素,并利用超限归纳法和该情况的假设来得到矛盾,从而证明该对象的存在。佐恩引理整理了一个情境需要满足的条件,以便使这种证明方法有效,并使数学家们无需每次都手动重复超限归纳法的论证,而只需检查佐恩引理的条件。

如果你正在分阶段构建一个数学对象,发现(i)即使经过无限多个阶段,你仍然没有完成,且(ii)似乎没有任何东西能阻止你继续构建,那么佐恩引理可能能够帮助你。

— 威廉·蒂莫西·高尔斯,《如何使用佐恩引理》\(^\text{[7]}\)
\subsection{引理的陈述}  
前提概念:
\begin{itemize}
\item 一个集合\( P \)配备一个二元关系\( \leq \),如果该关系是自反的(即对于每个\( x \),有\( x \leq x \))、反对称的(如果\( x \leq y \)且\( y \leq x \)都成立,则\( x = y \))、传递的(即如果\( x \leq y \) 且 \( y \leq z \)成立,则\( x \leq z \)),那么我们称\( P \) 是通过\( \leq \) 部分有序的。给定\( P \)中的两个元素 \( x \)和\( y \),如果\( x \leq y \),则称\( y \)大于或等于\( x \)。词语“部分”表示并非每一对部分有序集中的元素都需要在该顺序关系下可比较,也就是说,在一个带有顺序关系\( \leq \)的部分有序集\( P \)中,可能存在元素\( x \)和\( y \),使得既不\( x \leq y \) 也不\( y \leq x \)。一个有序集,如果其中每一对元素都可以比较,则称为全序的。
\item 部分有序集\( P \)的每个子集\( S \)可以通过将从\( P \)继承的顺序关系限制到\( S \)上,自己被看作是部分有序的。如果一个部分有序集\( P \)的子集\( S \) 在继承的顺序下是全序的,则称\( S \)是一个链(在\( P \) 中)。
\item 如果部分有序集\( P \)中有一个元素\( m \),使得没有其他元素大于 \( m \),即没有\( P \)中的元素\( s \)满足\( s \neq m \)且\( m \leq s \),则称\( m \)是最大元素(相对于\( \leq \))。根据顺序关系的不同,部分有序集可能有任意数量的最大元素。然而,完全有序集最多只能有一个最大元素。
\item 给定部分有序集\( P \) 的一个子集\( S \),如果\( P \)中的元素\( u \)大于或等于\( S \) 中的每个元素,则称\( u \)是 \( S \)的上界。在这里,\( S \)不要求是链,并且\( u \)必须与\( S \)中的每个元素可比较,但不需要是\( S \)的元素。
\end{itemize}
佐恩引理可以表述为:

\textbf{佐恩引理}—\(^\text{[8][9]}\)设\( P \)是一个部分有序集,满足以下两个条件:
\begin{enumerate}
\item \( P \) 非空;
\item \( P \) 中的每个链都有一个上界。
\end{enumerate}
则,\( P \)至少有一个最大元素。

事实上,条件(1)是多余的,因为条件(2)特别指出空链在\( P \)中有上界,这意味着\( P \)是非空的。然而,在实际操作中,人们通常会先检查条件(1),然后仅对非空链验证条件(2),因为空链的情况已经由条件(1)处理。

在布尔巴基的术语中,一个部分有序集称为归纳的,如果每个链在该集合中都有上界(特别地,该集合因此是非空的)\(^\text{[10]}\)。因此,引理可以表述为:

\textbf{佐恩引理}—\(^\text{[11]}\)每个归纳集都有一个最大元素。

对于某些应用,以下变体可能是有用的。

\textbf{推论}—\(^\text{[12]}\)设\( P \)是一个部分有序集,其中每个链都有上界,并且\( a \)是 \( P \) 中的一个元素。则在\( P \)中存在一个最大元素\( b \),使得\( b \geq a \)。

事实上,设\( Q = \{x \in P \mid x \geq a\} \),它继承自\( P \)的部分顺序关系。那么,对于\( Q \) 中的一个链,\( P \)中的上界也在\( Q \) 中,因此\( Q \)满足佐恩引理的假设,且\( Q \)中的最大元素也是\( P \)中的最大元素。
\subsection{示例应用}  
\subsubsection{每个向量空间都有一个基} 
佐恩引理可以用来证明每个向量空间\( V \)都有一个基。\(^\text{[13]}\)

如果\( V = \{0\} \),则空集是\( V \)的基。现在,假设\( V \neq \{0\} \)。设\( P \)是由\( V \)中所有线性无关子集构成的集合。由于\( V \)不是零向量空间,存在一个非零元素\( v \in V \),因此\( P \)包含线性无关子集\( \{v\} \)。此外,\( P \)按集合包含关系部分有序(见包含顺序)。寻找 \( V \)的一个极大线性无关子集等同于在\( P \)中寻找一个极大元素。

为了应用佐恩引理,取\( P \)中的一个链\( T \)(即 \( T \)是\( P \)的一个全序子集)。如果\( T \)是空集,那么\( \{v\} \)就是\( P \)中\( T \)的一个上界。假设\( T \) 是非空的。我们需要证明\( T \)有上界,即存在一个包含\( T \)中所有元素的线性无关子集\( B \) 。

取\( B \)为\( T \)中所有集合的并集。我们希望证明\( B \) 是\( P \)中\( T \)的一个上界。为了做到这一点,足够证明\( B \)是\( V \)中的一个线性无关子集。

假设相反,\( B \)不是线性无关的。那么存在向量\( v_1, v_2, \dots, v_k \in B \)和标量\( a_1, a_2, \dots, a_k \),它们并非全为零,满足
\[
a_1 \mathbf{v}_1 + a_2 \mathbf{v}_2 + \cdots + a_k \mathbf{v}_k = \mathbf{0}.~
\]
由于\( B \)是\( T \)中所有集合的并集,存在一些集合\( S_1, S_2, \dots, S_k \in T \),使得对于每个\( i = 1, 2, \dots, k \),有 \( v_i \in S_i \)。由于\( T \)是全序的,\( S_1, S_2, \dots, S_k \)中必须有一个集合包含其他所有集合,因此必定存在某个集合\( S_i \)包含\( v_1, v_2, \dots, v_k \)中的所有元素。这意味着\( S_i \)中有一个线性依赖的向量集,然而这与\( S_i \)线性无关的假设相矛盾(因为 \( S_i \)是\( P \)的成员)。

佐恩引理的假设已被验证,因此在\( P \)中存在一个极大元素,也就是说,存在一个 \( V \)的极大线性无关子集\( B \)。

最后,我们证明\( B \)确实是\( V \)的基。足够证明\( B \)是\( V \)的一个生成集。假设为了矛盾,\( B \)不是生成集。那么存在某个\( v \in V \),它不在 \( B \)的生成子空间内。这意味着\( B \cup \{v\} \)是 \( V \)中一个线性无关的子集,且它比\( B \) 大,这与\( B \) 的极大性矛盾。因此,\( B \)是\( V \)的生成集,从而\( B \)是\( V \)的基。
\subsubsection{每个非平凡的有单位环包含一个极大理想}
佐恩引理可以用来证明每个非平凡的有单位环\( R \)包含一个极大理想。

设\( P \)为\( R \)中所有真理想的集合(即\( R \)中的所有理想,除了\( R \) 本身)。由于\( R \)是非平凡的,集合\( P \)包含平凡理想\( \{0\} \)。此外,\( P \)按集合包含关系部分有序。在\( R \)中寻找一个极大理想等同于在 \( P \)中寻找一个极大元素。

为了应用佐恩引理,取\( P \)中的一个链\( T \)。如果\( T \)是空集,那么平凡理想 \( \{0\} \) 就是 \( P \)中\( T \) 的一个上界。假设\( T \)是非空的。我们需要证明\( T \)有上界,即存在一个理想\( I \subseteq R \),它包含\( T \)中所有元素,但仍小于\( R \)(否则它就不是一个真理想,因此不在\( P \)中)。

取\( I \)为\( T \)中所有理想的并集。我们希望证明\( I \)是\( P \)中\( T \)的一个上界。我们首先证明\( I \)是\( R \)的一个理想。为了使\( I \)成为理想,必须满足三个条件:
\begin{enumerate}
\item \( I \) 是 \( R \) 的一个非空子集;
\item 对于所有 \( x, y \in I \),\( x + y \in I \);
\item 对于所有 \( r \in R \) 和所有 \( x \in I \),\( rx \in I \)。
\end{enumerate}
\textbf{#1 -\( I \)是\( R \)的一个非空子集。}

因为\( T \)至少包含一个元素,而且该元素至少包含\( 0 \),所以并集\( I \)至少包含\( 0 \)并且非空。\( T \)中的每个元素都是\( R \)的子集,因此并集\( I \)仅包含\( R \)中的元素。

\textbf{#2 -对于每个\( x, y \in I \),和\( x + y \in I \)。}

假设\( x \)和\( y \)是\( I \)的元素。那么存在两个理想\( J, K \in T \),使得\( x \in J \)且\( y \in K \)。由于 \( T \)是全序的,我们知道\( J \subseteq K \)或\( K \subseteq J \)。不失一般性,假设第一种情况成立。由于\( x \)和 \( y \) 都是理想\( K \)的成员,因此它们的和\( x + y \)也是\( K \)的成员,这证明了\( x + y \in I \)。

\textbf{#3 -对于每个\( r \in R \)和每个\( x \in I \),乘积\( rx \in I \)。}

假设\( x \)是\( I \)的元素。那么存在一个理想\( J \in T \),使得\( x \in J \)。如果\( r \in R \),那么\( rx \)是\( J \)的元素,因此也是\( I \)的元素。所以,\( I \)是\( R \)的一个理想。

现在,我们证明\( I \)是一个真理想。一个理想当且仅当它包含 1 时才等于\( R \)。(显然,如果它是 \( R \),那么它包含 1;另一方面,如果它包含 1,并且\( r \)是\( R \)中的任意元素,那么\( r1 = r \)是理想的元素,因此该理想等于\( R \)。)因此,如果\( I \)等于\( R \),那么它将包含 1,这意味着\( T \)中的一个成员将包含 1,因此它将等于\( R \)——但是\( R \)被明确地排除在\( P \)之外。

佐恩引理的假设已经验证,因此在\( P \)中存在一个极大元素,也就是说,在\( R \)中存在一个极大理想。
\subsection{证明概述} 
以下是佐恩引理的证明概要,假设选择公理成立。假设该引理是错误的。则存在一个部分有序集(即 poset\( P \),使得每个全序子集都有上界,并且对于\( P \)中的每个元素,都存在一个更大的元素。对于每个全序子集\( T \),我们可以定义一个更大的元素\( b(T) \),因为\( T \)有上界,而该上界有一个更大的元素。

为了实际定义函数\( b \),我们需要使用选择公理(明确地说:令\(B(T) = \{ b \in P : \forall t \in T, b \geq t \}\),即\( T \)的上界集合。选择公理提供了函数\( b \),使得\(b: b(T) \in B(T)\)成立)。

使用函数\( b \),我们将定义\( a_0 < a_1 < a_2 < a_3 < \dots < a_\omega < a_{\omega+1} < \dots \) 这些元素在\( P \)中。这个不可数的序列非常长:索引不仅仅是自然数,而是所有的序数。实际上,这个序列对于集合 \( P \)来说太长了;序数太多了(是一个适当类),比任何集合中的元素还要多(换句话说,给定任何序数集合,总存在一个更大的序数),因此集合\( P \)很快就会耗尽,然后我们就会遇到所需的矛盾。

这些\( a_i \)是通过超限递归定义的:我们在\( P \)中任意选取\( a_0 \)(这是可能的,因为 \( P \)包含空集的上界,因此\( P \)不是空的),对于任何其他的序数\( w \),我们设定\( a_w = b(\{a_v : v < w\}) \)。因为\( a_v \)是全序的,所以这是一个有根的定义。

上述证明可以不显式地引用序数,通过将初始段\( \{a_v : v < w\} \)作为\( P \)的子集来表述。这些集合可以容易地表征为良序链\( S \subseteq P \),其中每个\( x \in S \)满足\( x = b(\{y \in S : y < x\}) \)。通过注意到我们总是可以找到“下一个”初始段,无论是通过取所有这样的\( S \)的并集(对应于极限序数情况),还是通过将\( b(S) \)添加到“最后”一个\( S \)(对应于后继序数情况),我们可以得到矛盾。\(^\text{[14]}\)

这个证明实际上表明,佐恩引理的一个稍强版本是成立的:

\textbf{引理}——如果\( P \)是一个部分有序集,其中每个良序子集都有上界,并且如果\( x \)是\( P \)中的任何元素,那么\( P \)中存在一个大于或等于\( x \)的极大元素。也就是说,存在一个与\( x \)可比较的极大元素。

或者,可以使用相同的证明来证明豪斯多夫最大原理。这是例如在 Halmos 的《天真集合论》或下文的证明中给出的证明。

最后,布尔巴基–维特定理也可以用来提供一个证明。
\subsection{证明}  
证明的基本思路是将证明简化为证明佐恩引理的以下弱形式:

\textbf{引理}——设\( F \)是一个由某个固定集合的子集构成的集合,且\( F \)满足以下属性:
\begin{enumerate}
\item \( F \)非空;
\item \( F \)中每个全序子集的并集仍在\( F \)中,其中顺序关系是根据集合包含关系来定义的;
\item 对于 \( F \)中的每个集合\( S \),每个\( S \)的子集都在\( F \)中。
\end{enumerate}
那么,\( F \)在集合包含关系下具有一个极大元素。

(注意,严格来说,(1) 是多余的,因为 (2) 已经暗示空集属于\( F \)。)请注意,上述内容是佐恩引理的一个弱形式,因为佐恩引理特别指出,满足上述 (1) 和 (2) 的任何子集集合都有一个极大元素((3) 不需要)。关键是,反过来,佐恩引理可以从这个弱形式推导出来。\(^\text{[15]}\)确实,设\( F \) 是 \( P \)中所有链的集合。然后它满足上述所有属性(它非空,因为空子集是一个链)。因此,根据上述弱形式,我们找到\( F \)中的一个极大元素\( C \),即\( P \)中的一个极大链。根据佐恩引理的假设,\( C \)在\( P \)中有一个上界\( x \)。然后,这个 \( x \) 是一个极大元素,因为如果\( y \geq x \),则 \( \widetilde{C} = C \cup \{y\} \) 比 \( C \)更大或相等,因此\( \widetilde{C} = C \)。因此,\( y = x \)。

弱形式的证明可以参见豪斯多夫最大原理的证明部分。事实上,极大链的存在正是豪斯多夫最大原理的断言。

同样的证明还表明了佐恩引理的以下等价变体:\(^\text{[16]}\)

\textbf{引理}——设 \( P \) 是一个部分有序集,其中每个链在 \( P \) 中都有最小上界。那么,\( P \) 中存在一个极大元素。

事实上,显然,佐恩引理蕴含上述引理。反过来,上述引理蕴含前述的佐恩引理的弱形式,因为并集给出了最小上界。
\subsection{佐恩引理蕴含选择公理} 
佐恩引理蕴含选择公理的证明展示了佐恩引理的一个典型应用。\(^\text{[17]}\)(该证明的结构与哈恩–巴拿赫定理的证明完全相同。)

给定一个非空集合的集合\( X \)和其并集\( U := \bigcup X \)(由于并集公理,\( U \)是存在的),我们希望证明存在一个函数
\[
f : X \to U~
\]
使得对于每个\( S \in X \),都有\( f(S) \in S \)。为此,考虑集合
\[
P = \{ f : X' \to U \mid X' \subset X, f(S) \in S \}~
\]
它按扩展部分有序;即,如果且仅如果\( f \leq g \),则\( f \) 是 \( g \)的限制。如果\( f_i : X_i \to U \)是\( P \) 中的一个链,则我们可以通过设置\( f(x) = f_i(x) \)来在并集\( X' = \bigcup_i X_i \)上定义函数\( f \),当\( x \in X_i \) 时。这是良定义的,因为如果\( i < j \),则\( f_i \)是 \( f_j \)的限制。函数\( f \)也是\( P \)的一个元素,并且是所有\( f_i \) 的一个公共扩展。因此,我们已经证明了\( P \)中的每个链都有一个上界。于是,根据佐恩引理,存在一个极大元素\( f \in P \),它在某个\( X' \subset X \)上定义。我们希望证明\( X' = X \)。假设不是这样;则存在一个集合\( S \in X - X' \)。由于\( S \)非空,它包含一个元素 \( s \)。我们可以通过设置\( g|_{X'} = f \) 和\( g(S) = s \)来扩展\( f \)为一个函数\( g \)。(注意,这一步不需要选择公理。)函数\( g \)属于\( P \),并且\( f < g \),这与\( f \)的极大性矛盾。◻

实质上相同的证明也表明,佐恩引理蕴含良序定理:令\( P \)为给定集合\( X \)的所有良序子集的集合,然后证明\( P \)中的一个极大元素是\( X \)。\(^\text{[18]}\)
\subsection{历史}  
豪斯多夫最大原理是与佐恩引理相似的早期表述。

卡济米日·库拉托夫斯基在1922年证明了一个接近现代表述的引理版本\(^\text{[19]}\)(它适用于按包含关系排序并对良序链的并集封闭的集合)。本质上相同的表述(通过使用任意链而不仅仅是良序链来弱化)是马克斯·佐恩在1935年独立给出的\(^\text{[20]}\),他将其提出为一个新的集合论公理,用以替代良序定理,并展示了它在代数中的一些应用,还承诺在另一篇论文中证明其与选择公理的等价性,但该论文从未发表。

“佐恩引理”这个名称似乎源于约翰·图基,他在1940年出版的《拓扑学中的收敛与一致性》一书中使用了这个名称。布尔巴基的《集合论》1939年版中提到一个类似的最大原理称为“le théorème de Zorn”\(^\text{[21]}\)。在波兰和俄罗斯,“库拉托夫斯基–佐恩引理”这个名称更为流行。
\subsection{佐恩引理的等价形式}  
佐恩引理在 ZF 中与以下三个主要结果等价:
\begin{enumerate}
\item 豪斯多夫最大原理  
\item 选择公理  
\item 良序定理
\end{enumerate}
一则著名的笑话暗指这一等价性(可能违背人类直觉),据说是杰瑞·博纳(Jerry Bona)所说:“选择公理显然是对的,良序原理显然是错的,而谁能说得清楚佐恩引理呢?”\(^\text{[22]}\)

佐恩引理还与一阶逻辑的强完备性定理等价。\(^\text{[23]}\)

此外,佐恩引理(或其等价形式之一)还蕴含了其他数学领域中的一些重要结果。例如:
\begin{enumerate}
\item 巴拿赫扩展定理,它用于证明泛函分析中最基本的结果之一——哈恩–巴拿赫定理
\item 每个向量空间都有一个基,这是线性代数中的一个结果(它与佐恩引理等价\(^\text{[24]}\))。特别地,实数作为有理数上的向量空间,具有一个哈梅尔基。
\item 每个交换单位环都有一个极大理想,这是环论中的一个结果,称为克鲁尔定理,佐恩引理与之等价\(^\text{[25]}\)
\item 拓扑学中的泰赫诺夫定理(它与佐恩引理也等价\(^\text{[26]}\))
\item 每个真滤子都包含在超滤子中,这一结果导出了第一阶逻辑的完备性定理\(^\text{[27]}\)
\end{enumerate}
从这个意义上说,佐恩引理是一个强大的工具,适用于数学的许多领域。
\subsubsection{在选择公理弱化下的类比}  
佐恩引理的一个弱化形式可以从 ZF + DC(即替换选择公理为依赖选择公理的泽梅洛–弗兰克尔集合论)中证明。可以通过观察没有极大元素的集合等价于该集合的顺序关系是完整的,从而表达佐恩引理,这将使我们能够应用依赖选择公理来构造一个可数链。因此,任何仅包含有限链的部分有序集必须具有一个极大元素。\(^\text{[28]}\))

更一般地,通过将依赖选择公理强化到更高的序数,我们可以将上一段的陈述推广到更高的基数。\(^\text{[28]}\))在允许任意大的序数的极限情况下,我们恢复了在前一部分中使用选择公理证明的完整佐恩引理。
\subsection{在流行文化中}  
1970年的电影《佐恩引理》以该引理命名。

该引理在《辛普森一家》的《巴特的新朋友》一集中被提到。[29]
\subsection{另见}  
\begin{itemize}
\item 反链 – 不可比较元素的子集  
\item 链完备部分有序集 – 每个链都有最小上界的部分有序集  
\item Szpilrajn扩展定理– 关于顺序关系的数学结果  
\item 塔尔斯基有限性 – 包含有限数量元素的数学集合  
\item Teichmüller–Tukey引理(有时称为Tukey引理)  
\item 布尔巴基–维特定理 – 一个没有选择公理的固定点定理,可以与选择公理结合使用来证明佐恩引理
\end{itemize}
\subsection{注释}
\begin{enumerate}
\item 塞尔,让-皮埃尔(2003),《树》(Trees),Springer数学专著,Springer出版社,第23页。
\item 摩尔(2013),第168页。
\item 威兰斯基,阿尔伯特(1964)。《泛函分析》(Functional Analysis)。纽约:布莱索德尔出版社(Blaisdell),第16–17页。
\item 耶赫(2008),第2章第2节“选择公理在数学中的一些应用”。
\item 耶赫(2008),第9页。
\item 摩尔(2013),第168页。
\item 威廉·蒂莫西·高尔斯(2008年8月12日)。《如何使用佐恩引理》("How to use Zorn's lemma")。
\item 哈尔莫斯(1960),第16节。
\item 朗,塞尔日(2002)。《代数》(Algebra)。数学研究生教材(Graduate Texts in Mathematics)第211卷(修订第三版)。施普林格出版社(Springer-Verlag)。第880页。ISBN 978-0-387-95385-4。
杜迈特,戴维·S.;富特,理查德·M.(1998)。《抽象代数》(Abstract Algebra)(第二版)。普伦蒂斯·霍尔出版社(Prentice Hall)。第875页。ISBN 978-0-13-569302-5。
伯格曼,乔治·M.(2015)。《泛代数与通用构造导论》(An Invitation to General Algebra and Universal Constructions)。大学教材系列(Universitext)(第二版)。施普林格出版社(Springer-Verlag)。第162页。ISBN 978-3-319-11477-4。
\item 布尔巴基(1970),第III章第2节第4小节,定义3。
\item 布尔巴基(1970),第III章第2节第4小节,定理2。
\item 布尔巴基(1970),第III章第2节第4小节,推论1。
\item 斯密茨,蒂姆。《每个向量空间都有基的一个证明》(A Proof that every Vector Space has a Basis)(PDF)。检索于2022年8月14日。
\item 莱文,乔纳森·W.(1991)。《佐恩引理的一个简单证明》("A simple proof of Zorn's lemma")。《美国数学月刊》(The American Mathematical Monthly)98(4):353–354。doi:10.1080/00029890.1991.12000768。
\item 哈尔莫斯(1960),第16节。注:原文献中通过指出存在一个保序嵌入
\( s: P \hookrightarrow \mathfrak{P}(P) \),
并利用“传递性”从佐恩引理的弱形式推导出 \( s(P) \) 的极大元存在性(等价于 \( P \) 的极大元存在性)。原文中“传递性”意义不明确,此处提供另一种替代推理。
\item 哈尔莫斯(1960),第16节,习题。
\item 哈尔莫斯 1960,§16. 练习。
\item 哈尔莫斯 1960,§17. 练习。
\item 库拉托夫斯基,卡西米尔(1922)。《消除数学推理中超限数的一种方法》(法文)(PDF)。《数学基础》(法语)。3:76–108。doi:10.4064/fm-3-1-76-108。检索于2013年4月24日。
\item 佐恩,马克斯(1935)。《关于超限代数方法的评论》。《美国数学学会通报》。41卷10期,第667–670页。doi:10.1090/S0002-9904-1935-06166-X。
\item 坎贝尔 1978,第82页。
\item 克兰茨,史蒂文·G(2002)。《选择公理》。载于《计算机科学逻辑与证明技术手册》(第121–126页)。Springer出版社。doi:10.1007/978-1-4612-0115-1_9。ISBN 978-1-4612-6619-8。
\item 贝尔,J·L·与斯洛姆森,A·B·(1969)。《模型与超积》(Models and Ultraproducts)。北荷兰出版公司。第5章定理4.3,第103页。
\item 布拉斯,安德烈亚斯(1984)。《基的存在性蕴涵选择公理》。载于《公理集合论》(Axiomatic Set Theory),《当代数学》(Contemporary Mathematics)第31卷,第31–33页。doi:10.1090/conm/031/763890。ISBN 9780821850268。
\item 霍奇斯,W·(1979)。《克鲁尔定理蕴涵佐恩引理》。《伦敦数学会杂志》(Journal of the London Mathematical Society)s2-19卷第2期,第285–287页。doi:10.1112/jlms/s2-19.2.285。
\item 凯利,约翰·L(1950)。《吉洪诺夫乘积定理蕴涵选择公理》。《数学基础》(Fundamenta Mathematicae)37卷,第75–76页。doi:10.4064/fm-37-1-75-76。
\item 贝尔,J·L·与斯洛姆森,A·B·(1969)。《模型与超积》(Models and Ultraproducts)。北荷兰出版公司1。
\item 沃尔克,埃利奥特·S·(1983)。《依赖选择公理与佐恩引理的若干形式》。《加拿大数学通报》(Canadian Mathematical Bulletin)26卷3期,第365–367页。doi:10.4153/CMB-1983-062-5。
\item 《佐恩引理 | 辛普森一家与数学秘密》(Zorn's Lemma | The Simpsons and their Mathematical Secrets)。
\end{enumerate}
\subsection{参考文献}  
\begin{itemize}
\item Bourbaki, N (1970). *Théorie des Ensembles*. Hermann.  
\item Campbell, Paul J. (1978年2月). "The Origin of 'Zorn's Lemma'"。*Historia Mathematica*,5 (1):77–89。doi:10.1016/0315-0860(78)90136-2。  
\item Halmos, Paul (1960). *Naive Set Theory*。普林斯顿,新泽西州:D. Van Nostrand Company。  
\item Ciesielski, Krzysztof (1997). *Set Theory for the Working Mathematician*。剑桥大学出版社。ISBN 978-0-521-59465-3。  
\item Jech, Thomas (2008) [1973]. *The Axiom of Choice*。纽约矿冶出版公司:Dover Publications。ISBN 978-0-486-46624-8。  
\item Moore, Gregory H. (2013) [1982]. *Zermelo's axiom of choice: Its origins, development & influence*。Dover Publications。ISBN 978-0-486-48841-7。
\end{itemize}  
\subsection{进一步阅读} 
\begin{itemize}
\item The Zorn Identity 在 n-category cafe。 
\end{itemize} 
\subsection{外部链接}  
\begin{itemize}
\item "Zorn lemma",Encyclopedia of Mathematics,EMS Press,2001 [1994]  
\item Zorn's Lemma 在 ProvenMath,包含选择公理与佐恩引理等价的正式证明,细节完整。  
\item Zorn's Lemma 在 Metamath,另一种正式证明。(适用于最近浏览器的 Unicode 版本。)
\end{itemize}