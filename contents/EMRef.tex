% 电磁场的参考系变换
% 参考系变换|洛伦兹变换|电磁场

\begin{issues}
\issueDraft
\end{issues}

\pentry{相对论速度变换\upref{RelVel}, 麦克斯韦方程组\upref{MWEq}, 洛伦兹力\upref{Lorenz}}

到目前位置我们只在同一个参考系中分析电磁学问题, 我们下面以两个例题来分析在不同参考系之间电磁场该如何变换. 我们将会发现, 在讨论电磁场的参考系变换时必须考虑狭义相对论效应才能不发生矛盾, 即麦克斯韦方程组\upref{MWEq}天然与洛伦兹变换\upref{SRLrtz}而不是伽利略变换兼容. 爱因斯坦创造狭义相对论时所发的论文 《论动体的电动力学》 讨论的就是这类问题.

从以下推导可知, 若电磁场 $\bvec E, \bvec B$ 都垂直于相对速度 $\bvec u$,% 参考新概念电磁学
\begin{equation}\label{EMRef_eq1}
\bvec E' = \gamma_u (\bvec E + \bvec u \cross \bvec B)
\end{equation}
\begin{equation}\label{EMRef_eq2}
\bvec B' = \gamma_u (\bvec B - \bvec u \cross \bvec E/c^2)
\end{equation}
其中
\begin{equation}
\gamma_u = \frac{1}{\sqrt{1 - u^2/c^2}}
\end{equation}
若电磁场平行于相对速度 $\bvec u$, 则它们不改变. 对任意方向的电磁场, 需要把它们分解为平行和垂直方向并分别计算.

\subsection{电场的变换}
在 $S$ 参考系中有一个电荷密度为 $\lambda$ 的无限长直导线于 $x$ 轴重合并静止. 距离导线 $r_0$ 处有一个电荷为 $q$ 的粒子沿 $\uvec x$ 方向以速度 $v$ 运动. 另外一个参考系 $S'$ 相对 $S$ 沿 $\uvec x$ 方向运动, 速度为 $u$. 求 $S'$ 系中的电磁场, 以及粒子在两参考系中的受力.
\begin{figure}[ht]
\centering
\includegraphics[width=6cm]{./figures/EMRef_1.pdf}
\caption{电场的变换} \label{EMRef_fig1}
\end{figure}

\subsubsection{错误的分析}
对一个学了电磁学但不懂相对论的人而言, 在 $S$ 中, 粒子只受电场力. 而在 $S'$ 中, 粒子不仅受同样的电场力, 而且 $v \ne u$ 时导线中的电流产生磁场还会给粒子一个额外的洛伦兹力. 这就产生了矛盾. 该分析的错误在于使用了牛顿的时空观, 没有考虑相对论效应.

\subsubsection{正确的分析}
上面 $S$ 系的分析是正确的,空间中只有电场没有磁场. 在 $S'$ 系中的确会出现磁场, 但由于导线出现了相对论的 “尺缩短” 效应(链接未完成), 电荷密度变大, 电场也比原来要大. 另外由于粒子在两参考系中的速度不同, 所以受力也存在一定变换关系而不要求相等(链接未完成)下文会证明粒子在两参考系中作任意运动的受力符合这种关系.

\subsubsection{具体计算}
在 $S$ 中,
\addTODO{链接高斯定理和环路定理例题}
\begin{equation}
E_{z,0} = \frac{1}{2\pi\epsilon_0 r_0} \lambda
\qquad
B_{y,0} = 0
\end{equation}
在 $S'$ 中, 线电荷密度变为 $\lambda' = \gamma_u \lambda$, 电流为 $I' = \gamma_u \lambda u$, 电磁场分别为
\begin{equation}
E'_z = \frac{1}{2\pi\epsilon_0 r_0} \gamma_u \lambda
\qquad
B'_y = -\frac{\mu_0}{2\pi r_0}\gamma_u \lambda u
\end{equation}

以上是直接从电荷计算出的结果. 一种更一般的角度是从 “电磁场在不同参考系中转换” 的角度来看待问题, 而不理会 $S$ 中的电磁场由什么样的电荷电流分布产生. 即 $S$ 系的电场经过某种线性变换后, 得到 $S'$ 系中的电场和磁场. 我们姑且假设这种变换必须是线性变换(链接未完成), 即 $S$ 系中电磁场的 6 个分量线性变换到 $S'$ 中电磁场的 6 个分量. 可以看出垂直 $\bvec u$ 的电场的变换公式为
\begin{equation}\label{EMRef_eq5}
\bvec E' = \gamma_u \bvec E
\end{equation}
\begin{equation}\label{EMRef_eq6}
\bvec B' = -\gamma_u \frac{\bvec u}{c^2} \cross \bvec E
\end{equation}
也就是说, 若参考系垂直磁场方向移动, 则新参考系中电场会变强, 并且会出现一个磁场.

\addTODO{粒子的受力分析}

\subsection{磁场的变换}
现在来讨论磁场的变换, 为了使 $S$ 系中只存在磁场, 我们需要把问题改的稍微复杂一些:

在 $S$ 参考系中有一个延 $x$ 轴的无限长直导线. 我们假设导线中的所有正电荷的线密度为 $\lambda$, 以速度 $v_0\uvec x$ 运动, 导线中的所有负电荷的线密度为 $-\lambda$,  以速度 $-v_0\uvec x$ 运动. 距离导线 $r_0$ 处有一个电荷为 $q$ 的粒子沿 $\uvec x$ 方向以速度 $v$ 运动. 另一个参考系 $S'$ 相对 $S$ 沿 $\uvec x$ 方向以速度 $u$ 运动. 求这两个参考系中的电磁场, 以及粒子所受的电磁力是否相符?
\addTODO{图}

在 $S$ 系中, 导线的电流为
\begin{equation}
I_0 = 2 \gamma_0 \lambda v_0 = \frac{2 \lambda v_0}{\sqrt{1 - v_0^2/c^2}}
\end{equation}
假设粒子在 $z$ 轴正半轴, 距离导线 $r_0$, 粒子处电磁场大小为
\begin{equation}
E_z = 0 \qquad B_y = -\frac{\mu_0}{2\pi} \frac{I_0}{r_0}
\end{equation}

在 $S'$ 系中, 导线中正负电荷的速度为
\begin{equation}
v'_{0, \pm} = \frac{\pm v_0-u}{1 \mp uv_0/c^2}
\end{equation}
电荷线密度分别变为
\begin{equation}
\lambda'_\pm = \frac{\lambda}{\sqrt{1 - v'^2_{0,\pm}/c^2}}
\end{equation}
粒子和导线间的距离仍然是 $r_0$, 粒子处的电场大小为
\begin{equation}
E'_z = \frac{\lambda'_+ + \lambda'_-}{2 \pi \epsilon_{0} r_0}
\end{equation}
磁场为(右手定则\upref{RHRul}决定正方向)
\begin{equation}
B'_z = -\frac{\mu_0}{2\pi} \frac{\lambda'_+ v'_{0,+} + \lambda'_- v'_{0,-}}{r_0}
\end{equation}
经过一系列化简, 可得变换公式
\addTODO{引用一个公式: $\gamma_{u+v} = \gamma_u\gamma_v(1 + uv/c^2)$}
\begin{equation}
E'_z = \gamma_u u B_y
\qquad
B'_y = \gamma_u B_y
\end{equation}
若把粒子放在 $+y$ 轴, 同理可得 $E_y, B_z$ 的变换, 总结后可以写处垂直方向电磁场的矢量形式
\begin{equation}\label{EMRef_eq3}
\bvec E' = \gamma_u \bvec u \cross \bvec B
\end{equation}
\begin{equation}\label{EMRef_eq4}
\bvec B' = \gamma_u \bvec B
\end{equation}
也就是说, 若参考系垂直磁场方向移动, 则新参考系中磁场会变强, 并且会出现一个电场.

合并\autoref{EMRef_eq5} \autoref{EMRef_eq6} 以及\autoref{EMRef_eq3} \autoref{EMRef_eq4}, 就得到了\autoref{EMRef_eq1} \autoref{EMRef_eq2}.

\addTODO{粒子的受力分析}
