% 霍尔效应
% 霍尔效应|电势差|Hall|磁场|霍尔电势差

1879 年霍尔(E. C. Hall)首先观察到,把一载流导体薄片放在磁场中时,如果磁场方向垂直于薄片平面,则在薄片的上、下两侧面会出现微弱的电势差.这一现象称为霍尔效应(Hall effect).此电势差称为\textbf{霍尔电势差}.实验测定,霍尔电势差的大小与电流$I$及磁感应强度$B$成正比,而与薄片沿$\mathbf B$方向的厚度$d$成反比.它们的关系可写成:
\begin{equation}
U=V_{1}-V_{2}=R_{\mathrm{H}} \frac{I B}{d}
\end{equation}