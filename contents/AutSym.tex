% 有限对称群的性质
% license Xiao
% type Tutor


\pentry{置换群、对称群\nref{nod_Perm},群的同态与同构\nref{nod_Group2}}{nod_ef14}
\begin{definition}{}
阶数为1或2的置换,称为\textbf{对合变换(involution)。}
\end{definition}
\begin{theorem}{}
任意有限置换群都可以表示为两个对合变换的复合。
\end{theorem}
\textbf{证明:}
我们首先证明,循环置换可以分解为两个对合变换的复合。

一个n元循环置换可以看作n边形上的旋转,而我们知道,二维空间上的旋转可以分解为两个反射,只要保证两次反射轴的夹角是旋转角度的$1/2$,对合变换就是这种反射变换的置换表示。以正五边形为例,该过程如下所示:
\begin{figure}[ht]
\centering
\includegraphics[width=14cm]{./figures/f32c9320160af59c.png}
\caption{} \label{fig_AutSym_2}
\end{figure}
该循环的分解写作$(12345)=[(23)(14)][(13)(45)]$,显然,$[(13)(45)]$与$[(23)(14)]$就是图中所示的“反射变换”。

由于\textbf{$n$元置换群总可以拆分成不相交的循环乘积},而每个循环都可以拆成对合之积,则这些对合可以重新组合成两组。如设某置换群可拆分成三个不相交循环之积,用$\sigma$表示对合变换,且下标首字母不同代表不同循环,则有:
\begin{equation}
\begin{aligned}
f=f_1f_2f_3&=[\sigma_{11}\sigma_{12}][\sigma_{21}\sigma_{22}][\sigma_{31}\sigma_{32}]\\
&=[\sigma_{11}\sigma_{21}\sigma_{31}][\sigma_{12}\sigma_{22}\sigma_{32}]\\
&=\sigma'_1\sigma'_2~.
\end{aligned}
\end{equation}
得证。

从将循环元素的分解过程中,我们可以看到,对合变换是不相交的对换乘积,\textbf{称这些不相交的对换为一个对合变换的组分。}
\subsubsection{有限置换群的自同构群}
在本节,我们先阐明置换群的内自同构与群本身的关系,再阐明自同构与群的关系。
\begin{theorem}{}\label{the_AutSym_1}
当$n>2$时,$\opn{Inn}S_n\cong S_n$
\end{theorem}

\textbf{证明:}

由\autoref{the_Group2_2} 得,$\opn{Inn}S_n\cong S_n/C(S_n)$,因此我们只需要证明$n>2$时,有限置换群的中心是单位元即可。

设任意$\sigma_n\in C(S_n)-e$,那么$\sigma_n$有三种可能,下面证明,每一种情况总能找到与之不交换的群元素。

\begin{itemize}
\item 若$\sigma_n$是对换,比如$(a\,b)$,则$(b\,c)(a\,q)$必然与之不交换。
\item 若$\sigma_n$是$k$循环,比如$(a\,b\,c\,d)$,则$(a\,b)$便与之不交换。
\item 若$\sigma_n$是不相交的循环乘积,则在两个循环里各取一元素,组成的对换与之不交换。
\end{itemize}
因此,$\sigma_n$是空集,$C(S_n)=e$,证明成立。

自同构是比内自同构更广些的概念,只要求双射同态。思及前文,我们已经证明任意置换都可以拆分成两个不相关的对合之复合。令对合变换表示为$\sigma$,易证自同构将其映射为对合,因为$f(\sigma^2)=f(e)=f^2(\sigma)$,且能把共轭类映射为共轭类,$f(\tau\sigma\tau^{-1})=f(\tau)f(\sigma)f^{-1}(\tau)$。下面我们来探讨,对合变换彼此共轭需要的条件。


\begin{lemma}{}
对合变换共轭,当且仅当二者\textbf{组分一致}。
\end{lemma}
将对合变换$\sigma$拆成一系列互不相交的组分乘积,然后用$\tau$来做共轭变换,有几种可能结果:
\begin{itemize}
\item $\tau$与$\sigma$无相交元素,此时$\tau\sigma\tau^{-1}=\sigma$,因此组分一致。
\item $\tau$与$\sigma$其中一个组分\textbf{只有一个相交元素},则$\tau$的其他组分和$\sigma$交换得$e$,相交部分相乘后与$\sigma$其他组分无相交,因此组分也一致\footnote{譬如$(2\,3)(3\,4)(2\,3)=(2\,4)$}。
\item $\tau$包含$\sigma$的一个组分,则乘积结果为该组分,因此组分不变。
\item $\tau$的一个组分是在$\sigma$的两个组分中各取一个元素。如$(1\,4)[(1\,2)(3\,4)](1\,4)=(2\,4)(1\,3)$。
\item $\tau$是以上情况的任意组合,由于组分彼此互不相交,因此可交换位置到与$\sigma$中的关联组分相邻,最后得到组分不变的结果。
\end{itemize}

这条引理告诉我们,可以利用组分数量来对对合变换进行共轭类划分。

因此我们可以设集合$I_k^n$是$S_n$中由$k$个组分复合而成的的对合变换。考虑到一个组分是一个对换,$k\in[0,n/2]$。

因为自同构是同态映射,且把共轭类映射到共轭类,因此$f(I_k^n)\subset I_{k'}^n,k'\in [0,n/2]$。又因为自同构是双射,所以若 $f(I_a^n)=f(I_b^n)$,则$|I_a^n|=|I_b^n|$。所以为了聊自同构映射后的结果,我们需要研究对合集合的基数。

由对合集合的定义可知,$I_a^n$是从$n$个元素中选取$2a$个元素进行有序排列,且该排列里不同对换的顺序,以及一个对换里的元素顺序与对合变换的结果无关。因此,若设$G_k=|I_k^n|$,则

\begin{equation}
G_k=\frac{A_{n}^{2k}}{2^k\times k!}=\frac{n!}{2^k(n-k)!k!}~.
\end{equation}

\textbf{若$G_a=G_b$,唯一解得$n=6$},此解为$|I^6_3|=|I^6_1|$。解的唯一性说明除却$n=6$,$f(I_k^n)=I_k^n$。

\begin{corollary}{}
当$n>2$且$n\neq 6$时,$\opn{Aut}S_n\cong S_n$。
\end{corollary}
该推论比\autoref{the_AutSym_1} 更强一些。由于$\opn{Inn}S_n\subset \opn{Aut} S_n$且$\opn{Inn}S_n\cong S_n$,因此我们只需要证明$|\opn{Aut}S_n|=|S_n|$即可。

因为置换总可以拆成不相交的循环,因此我们先看自同构对循环的作用。由于循环是对合变换的乘积,而在$n\neq 0$的时候,$f(I_k^n)=I_k^n$,因此自同构把对换映射为对换,且这种对换之间的映射是双射。比如对于任意$f$,若设$f(1\,2)=f(a\,b)=(c\,d)$,则
\begin{equation}
f((1\,2)(a\,b))=f(1\,2)f(a\,b)=I~,
\end{equation}
违反了自同构保元素阶数不变的性质。

对于3阶循环,我们有
\begin{equation}
f(1\,2\,3)=f((1\,3)(1\,2))=f(1\,3)f(1\,2)=(a\,b)(c\,d)~,
\end{equation}
因阶数之不变,$b=c,a\neq d$或$b\neq c,a=d$。

同理可知,对于$k$阶循环,自同构相当于将其映射到$k$个对换上,可由$k$个元素标记。现在取$S_n$中的$n$阶循环,令
\begin{equation}
f(1\,2\,...n)=f(1\,n)f(1\,n-1)...f(1\,2)=(i_1\,i_n)(i_1\,i_{n-1})...(i_1\,i_2)~,
\end{equation}
那么对于任意$a,b\neq 1$必有$f(a\,b)=(i_a\,i_b)$。因为$f(a\,b)$与$f(1\,a),f(1\,b)$都分别只包含一个相同元素。倘若包含的是$i_1$,即$f(a\,b)=(i_1\,i_k)$,则$f(a\,b)=f(1\,k)$,违背$f$的双射性。所以对于$S_n$,任意$f$实际上指定了一个双射$a\rightarrow i_a$,$i_a$是从1至$n$中指定一个元素,所以$|\opn{Aut}S_n|=|S_n|$,得证。
