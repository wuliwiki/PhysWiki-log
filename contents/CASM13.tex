% 中国科学院 2012 年考研数学(甲)
% keys 中科院|2023|数学
% license Xiao
% type Tutor



\subsection{选择题}(本题满分50分,每小题5分。请从题目所列的选项中选择一个正确项填充空格。每题的四个备选项中只有一个是正确的,不选、错选或多选均不得分。请将你的选择标清题号写在考场发的答题纸上,直接填写在试题上无效。)

1.函数$f(x)$的导数$f'(x)$在$(-\infty,\infty)$上是连续函数,$a>0$,则函数$F(x)=\leftgroup{
&a,&&f(x)\geqslant a\\ 
&f(x), &&-a<f(x)<a \\
&-a,&&f(x) \leqslant -a }$

一定是()\\
(A)有界可微函数$\quad$(B)有界连续函数 \\
(C)连续可微函数$\quad$ (D)以上结论都不正确

2.$\displaystyle \lim_{\substack{x \to 0}}(\frac{1}{n^2+2n+1}+\frac{2}{n^2+2n+2}+\dots+\frac{n}{n^2+2n+n})=$()\\
(A)1 $\qquad$ (A) $ \infty \qquad$  (A)$\frac{1}{2}$ $\qquad$   (D)0

3.函数$f(x)=(x+2cosx)^2$在区间$[0,\frac{\pi}{2}]$上的最大值是()\\
(A)$\frac{\pi ^2}{36}+\frac{\sqrt{3}\pi}{3}+1 \qquad$     (B)$\frac{\pi ^2}{36}+\frac{\sqrt{3}\pi}{3}+2 \qquad$  \\
(C) $\frac{\pi ^2}{36}+\frac{\sqrt{3}\pi}{3}+3\qquad$       (D)$\frac{\pi^2}{4} \qquad$  

4.设$f(x)=x(x+1)\dots(x+20)$,下面四个结论正确的是()\\
(A)$f'(-1)>0,f'(-2)>0\quad$ (B)$f'(-1)>0,f'(-2)<0\quad$ \\
(C)$f'(-1)<0,f'(-2)<0\quad$ (D)$f'(-1)<0,f'(-2)>0$

5.已知$\displaystyle g(x)\int_{0}^{2} f(x)\dd{x}=10$,则$\displaystyle \int_{0}^{2} f(x)\dd{x} \int_{0}^{2} g(x)\dd{x}=$()\\
(A)20 $\quad$ (B)10 $\quad$ (C)5 $\quad$ (D)不能确定

6.$ \displaystyle \lim_{\substack{x \to 0,y \to 0}} \frac{xy}{\sqrt[3]{x^4+y^{12}}}$=()\\
(A)0 $\quad$ (B)$\frac{1}{\sqrt{2}} \quad$ (C)$ \frac{1}{\sqrt[3]{2}}\quad$ (D)不存在  

7.$f(u)$在区间$(-2,2)$内可导,且$f'(u)>0,f(0)=0$,$L$为单位圆周$x^2+y^2=1$被$y=x$和$y$轴正半轴所夹的弧段,则关于弧长的曲线积分$c_1=\int_{L}f(2xy)\dd{s}$和$c_2=\int_{L}f(2x^2-1)\dd{s}$满足()\\
(A)$c_1>0,c_2>0\quad$ (B)$c_1>0,c_2<0$\\
 (C)$c_1<0,c_2>0\quad$ (D)$c_1<0,c_2<0$

8.设二阶线性齐次常系数微分方程$y''+ay'+by=0$的任一解$y(x)$满足$x \to +\infty$时$y \to 0$,则实数$a,b$满足()\\
(A)$a>0,b>0\quad$ (B)$a>0,b<0\quad$ (C)$a<0,b>0\quad$ (D)$a<0,b<0$

9.幂级数$\displaystyle \sum_{n=1}^{\infty} \frac{(x+1)^n}{\sqrt{n}}$的收敛域是()\\
(A)$[-2,0)\quad$ (B)$(-2,0)\quad$ (C)$(-2,0]\quad$ (D)$[-2,0]$

10.过点$(0,0,1)$且与直线$\leftgroup{&x=t+1\\ &y=-t-4\\&z=2t}$及$\frac{x-1}{-1}=\frac{y}{2}=\frac{z}{-1}$都平行的平面方程为()\\
(A)$5x+2y-z+1=0\quad$ (B)$5x-y-3z+3=0$\\
(C)$3x+y-z+1=0\quad$ (D)$-3x-y+z+1=0$


\subsection{应用题}
1.计算$\displaystyle \lim_{x \to 0+}x^{(x^x-1)}$。


2.求微分方程$y''=y'(y-3)$满足初始条件$y(0)=1,y'(0)=-\frac{5}{2}$的解。


3.求函数$f(x)=\pi ^2-x^2$在区间$[-\pi,\pi)$上的傅里叶级数。

4.求曲面积分$\int \int_{S}xy\dd{y}\dd{z}+z^2\dd{x}\dd{y}$,其中$S$为由$z=\sqrt{x^2+y^2}(0\leqslant z \leqslant 1)$的上侧(法向量与$z$轴正向夹角为锐角的一侧)及$z=1$的下侧围成的有向曲面。

5.假设函数$f(x)$满足$f(1)=1$且对于$x\geqslant 1$,$f'(x)=\frac{1}{x^2+f^2(x)}$证明$\displaystyle \lim_{x \to \infty }f(x)$存在,且不大于$1+\frac{\pi}{4}$。

6.设两个连续函数$f,g$满足:当$x \in [0,1]$,$f(x)+g(x)\neq0$。证明存在唯一的数$0\leqslant a \leqslant 1$使得$\int_{a}^{1}|f(x)|\dd{x}=\int_{0}^{a}g^2(x)\dd{x}$。

7.证明$\displaystyle \lim_{x \to +\infty}\frac{1}{x}\int_x^{2x}|cost|\dd{t}=\frac{2}{\pi}$。

8.设$f(x)=\frac{1}{1+x^2}-xe^x\int_0^1 f(x)\dd{x}$,求$f(x)$和$f'(x)$。

9.函数$f(x)$在$[a,b]$上连续,在$(a,b)$内可导。证明存在$\zeta,\eta \in (a,b)$,使得$f'(\eta)=(b^2+ab+a^2+2)\frac{f'(\zeta)}{3 \zeta ^2 +2}$。

10.函数$f(x)$在$[0,2]$上二阶可导,且对任意$x \in [0,2]$,有$|f(x)| \leqslant  1$和$|f''(x)| \leqslant 1$,证明,对任意$x \in [0,2],|f'(x)|\leqslant 2$成立。

