% 误差函数
% keys 高斯积分|误差函数|级数展开
% license Xiao
% type Tutor

\pentry{高斯积分\nref{nod_GsInt}}{nod_4667}

\begin{figure}[ht]
\centering
\includegraphics[width=10cm]{./figures/e453b1d372658295.pdf}
\caption{误差函数} \label{fig_Erf_1}
\end{figure}

\footnote{参考 Wikipedia \href{https://en.wikipedia.org/wiki/Error_function}{相关页面}。}\textbf{误差函数(error function)}的定义为
\begin{equation}\label{eq_Erf_1}
\erf(x) = \frac{1}{\sqrt{\pi}} \int_{-x}^x \E^{-t^2} \dd{t}
= \frac{2}{\sqrt{\pi}} \int_{0}^x \E^{-t^2} \dd{t}~,
\end{equation}
这是一个奇函数(\autoref{fig_Erf_1})。 根据\autoref{eq_GsInt_1}~\upref{GsInt}有
\begin{equation}
\erf(\pm\infty) = \pm 1~.
\end{equation}
由\autoref{eq_NLeib_10}~\upref{NLeib}, 误差函数的导数为
\begin{equation} % 已验证 Matlab 的 erf()
\erf'(x) = \frac{2}{\sqrt{\pi}} \E^{-x^2}~.
\end{equation}
即 $\E^{-x^2}$ 的不定积分(原函数)为
\begin{equation}\label{eq_Erf_5}
\int \E^{-x^2} \dd{x} = \frac{\sqrt{\pi}}{2} \erf(x) + C~.
\end{equation}

误差函数的反函数称为\textbf{反误差函数(inverse error function)}, 记为 $\erf^{-1}$, 定义域为 $(-1,1)$。

\subsection{级数展开}
\pentry{泰勒级数\nref{nod_Taylor}}{nod_8a28}
由指数函数的级数展%链接未完成
开得
\begin{equation}
\E^{-x^2} = \sum_{n=0}^\infty \frac{1}{n!} (-x^2)^n = \sum_n \frac{(-1)^n}{n!} x^{2n}~.
\end{equation}
对各项做不定积分代入\autoref{eq_Erf_5} 可得误差函数的级数展开为
\begin{equation}
\erf(x) = \frac{2}{\sqrt{\pi}} \sum_{n=0}^\infty \frac{(-1)^n}{(2n+1)n!} x^{2n+1}
= \frac{2}{\sqrt{\pi}}\qty(x - \frac{x^3}{3} + \frac{x^5}{10} - \frac{x^7}{42} + \frac{x^9}{216} \dots)~.
\end{equation}
由该式可以把误差函数拓展到复数域。

\begin{example}{}\label{ex_Erf_1}
令 $a$ 为实数且 $a > 0$, 计算不定积分
\begin{equation}
\int \exp(-ax^2 + bx) \dd{x}~.
\end{equation}

解: 我们可以先将指数部分凑平方得
\begin{equation}
-ax^2 + bx = -t^2 + \frac{b^2}{4a}~,
\end{equation}
其中
\begin{equation}
t = \sqrt{a} x - \frac{b}{2\sqrt{a}} \qquad~,
\end{equation}
使用上式进行换元积分得
\begin{equation}
\int \exp(-ax^2 + bx) \dd{x} = \frac{1}{\sqrt{a}} \E^{{b^2}/(4a)} \int \E^{-t^2} \dd{t}
= \frac12 \sqrt{\frac{\pi}{a}} \E^{{b^2}/(4a)} \erf\qty(\sqrt{a} x - \frac{b}{2\sqrt{a}})~.
\end{equation}
\end{example}

\begin{example}{}\label{ex_Erf_2}
计算无穷积分(即 $e^{-ax^2}$ 的傅里叶变换\upref{FTExp})
\begin{equation}
g(k) = \int_{-\infty}^{+\infty} \E^{-a x^2} \E^{-\I kx} \dd{x}~.
\end{equation}
令 $b = -\I k$, 使用\autoref{ex_Erf_1} 的结论有
\begin{equation}\label{eq_Erf_2}
g(k) = \frac12 \sqrt{\frac{\pi}{a}} \E^{-{k^2}/(4a)} \eval{\erf\qty(\sqrt{a} x + \frac{\I k}{2\sqrt{a}})}_{-\infty}^{+\infty} = \sqrt{\frac{\pi}{a}} \E^{-{k^2}/(4a)}~.
\end{equation}
\end{example}
