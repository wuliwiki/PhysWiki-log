% 南京大学 2019 年考研普通物理
% 普通物理|普物|南大
\subsection{力学}
1. 两小球质量均为 $m$, 使两球位于同一竖直线上,球 1 在球 2 上方 $h$ 处。在自由释放球1的同时,以初速度 $v_{0}$ 将球 2 竖直上抛。小球所受空气阻力与速度成正比,比例系数为 $k$,设空气阻力很小,求两球能在运动过程中相遇的条件。

2. 质量为 $m$ 长为 $l$ 的杆 $\mathrm{AB}$ 开始时 $\mathrm{A}$ 端用轴固定,并竖直放置。质量为 $m / 2$ 的小球以 $v_{0}$ 撞击 $\mathrm{B}$ 端并镶嵌其中,求该杆转过 $90^{\circ}$ 到水平面时杆对轴施加的压力。
\begin{figure}[ht]
\centering
\includegraphics[width=6cm]{./figures/NJU19_1.pdf}
\caption{力学第二题图} \label{NJU19_fig1}
\end{figure}
\subsection{热学}
1. 一摩尔理想气体初始条件为 $p=p_{0}$,$V=V_{0}$,等温膨胀使其体积变为原来的两倍,再沿直线回到原来的状态,求制冷系数。
\begin{figure}[ht]
\centering
\includegraphics[width=6cm]{./figures/NJU19_2.pdf}
\caption{热力学第一题} \label{NJU19_fig2}
\end{figure}
2. 质量为 $m$ 的均匀杆的两端分别与温度为 $T_{0}$,$2T_{0}$ 的恒温热源接触, 设等压比热为 $c_{p}$,现撤去两个热源,求达到平衡温度后杆的熵变。
\subsection{电磁学}
1. 有一带电为 $q$ 、长为 $l$ 的带电导线。\\
(1)求杆延长线上距离杆中心 $x$ 处电场;\\
(2)求杆中垂线上距离杆中心 $x$ 处电场。
\subsection{光学}
1. 衍射光栅没毫米刻线为300条。用两种可见光混合而成的光做衍射,衍射角为 $24^{\circ}$ 的位置上出现重叠波。\\
(1)求两种可见光波长;\\
(2)求还可以见到重叠波的角度。\\
(已知可见光波长为 $390-750 \mathrm{~nm}$,$\sin 24^{\circ}=0.407$)
