% 函数的性质(高中)
% keys 函数|变换|平移|旋转|伸缩|单调|对称|奇偶性|初等函数|周期
% license Xiao
% type Tutor

\pentry{函数(高中)\nref{nod_functi}}{nod_6f70}

在前面的学习中,我们已经接触了函数的概念,以及如何通过复合运算来将多个函数结合起来。现在,我们将深入探讨一个函数的各种性质,比如它的单调性、奇偶性和周期性。这些性质是理解函数行为和性质的关键,它们能够帮助我们更好地分析函数在不同情况下的表现。

然后,我们会大概介绍一下高中阶段需要掌握的几种主要函数类型。这些函数在数学和实际应用中扮演着重要角色,它们构成了函数世界的一大部分。这里会把它们作为两部分来整体介绍,后面的学习中会具体介绍每一个函数的细节。

\subsection{零点}\label{sub_HsFunC_3}

函数的零点是指使函数值为零的自变量 $x$,从几何角度来看,函数 $f(x)$ 的零点就是其图像与 $x$ 轴的交点。一般情况下\footnote{确实有一些函数的零点是连续的,例如 $f(x) = 0$ 以及由此复合得到的函数。},零点是孤立的,而零点之外的点通常连续构成一个区间。在这些区间中,函数往往表现出某种特性(例如取正值或负值),通常,函数在零点的两侧会发生符号变化\footnote{从正值变为负值,或从负值变为正值}。而零点则作为这些区间的端点,成为函数符号变化的分界线。因此,判断某个点是否属于某个区间时,通常需要将其与零点进行比较。事实上,函数的零点之所以具有特殊意义,正是因为它常常对应某种边界条件,标志着状态的转变或某种变化的界限,比如物体的平衡点以及工程问题中的最优设计或操作条件等。

\begin{definition}{零点}
对于函数 $f(x)$,若 $f(x_0) = 0$ ,则称$x_0$为 $f(x)$ 的\textbf{零点(zero point)},即:
\begin{equation}
x_0 \in \{ x \mid f(x) = 0 \}~.
\end{equation}
\end{definition}

函数的零点与方程的解有密切联系,零点沟通了函数与方程。任何方程总可以通过移项转换成形如 $f(x) = 0$ 的形式\footnote{高中阶段只涉及一元函数,因此此处指的是一元方程},而这个表达式所描述的,就是函数 $f(x)$ 的零点,零点集就是方程的解集。这一点在初中学习时,相信你就已经感受过了。

两个函数的交点也可以用零点的形式来表示。假如要求解两个函数 $f(x)$ 和 $g(x)$ 的交点,其实就是寻找满足 $f(x) = g(x)$ 的点。通过设 $F(x) = f(x) - g(x)$,可以将问题转化为求 $F(x)$ 的零点,这样 $F(x)$ 的零点对应的就是 $f(x)$ 和 $g(x)$ 的交点。特别地,当函数 $f(x)$ 取某个固定值 $a$ 时,这个问题可以看作是 $g(x) = a$ 的特殊情况。此时我们设 $F(x) = f(x) - a$。根据函数的\aref{平移性质}{sub_FunTra_1},这相当于将函数 $f(x)$ 向下平移 $a$ 个单位(如果 $a$ 是负值,就向上平移 $|a|$ 个单位)\footnote{这里如果看不懂可以先跳过,看完\enref{函数的变换}{FunTra}再回过头来理解。}。因此,函数取某个值的点,其实就是平移后的零点。自然,$f(x)$ 的零点就是 $g(x) = 0$ 的特殊情况了。

如果一个零点在方程中出现多次,称为重根。例如,对于 $f(x) = (x - 1)^2$,$x = 1$ 是零点,但它是一个重数为 2 的零点。重数的概念在多项式函数的分析中很重要,零点的重数还与函数在该点的图像行为有关,比如多项式曲线在某些重的重根处“接触”而不是“穿过” $x$ 轴。这是很关键的一点,在不等式部分会着重介绍。

另外,在后面会接触到的导数中,零点也有特殊的含义。一般而言,一阶导数的零点称为\textbf{驻点},可能是函数的\aref{极值点}{sub_HsFunC_2}。二阶导数的零点往往对应于函数的\textbf{拐点},也就是曲率改变的点。此处只是提及,具体内容会在导数部分详细讲解。

\subsubsection{零点存在定理}

尽管,在高中阶段涉及到的零点通常可以通过代入某些特殊值来求解,但有些时候,并不是需要求解某个具体的零点值,这时只要能够证明在某个区间上存在零点就可以了。于是需要使用\textbf{零点存在定理(Existence Theorem of Zero Points)}。

\begin{theorem}{零点存在定理}
若函数 $f(x)$ 在$[a,b]$上连续\footnote{这里的连续是一个高中没有接触过的概念,此时只能感性地理解为在画函数图像的时候,笔尖不会离开纸面。具体的探索会在大学阶段进行。},且满足 $f(a)$ 和 $f(b)$异号,即$f(a)f(b)<0$ ,则在 $(a, b)$ 上至少存在一个零点,即:
\begin{equation}
\exists x_0\in(a,b),f(x_0)=0~.
\end{equation}
\end{theorem}

通常,在使用零点存在定理时,首先需要找到一个区间的两个端点,然后通过证明函数在这些端点上的值符号相反(一个为正,一个为负),从而推断该区间内存在至少一个零点。接着,假设这个零点为 $x_0$,然后进一步研究 $x_0$ 的一些性质。

零点存在定理其实是一个更广泛定理——“\aref{介值定理}{sub_conff_4}”在取值为0时的特例。介值定理本质上反映了\enref{实数完备性}{RCompl}这一深层性质。因此,零点存在定理是一个非常基础的定理,但正因其基础性,证明过程相对复杂,在高中阶段不要求掌握其证明。在大学阶段,还会学习与零点存在定理相关的一组定理,被称为\enref{中值定理}{MeanTh}。这些定理构成了非常强大的数学工具,不仅能够判断零点或某些特殊取值的存在性,甚至还提供了高精度的估计方法。

\subsubsection{二分法}

基于零点存在定理,有一种在某个区间内逐步逼近函数零点的数值求解方法称为“二分法”。它的核心思路就是,始终保持区间端点函数值异号,这样就可以保证在区间上始终有零点,然后逐步缩小区间的范围,来保证精度。

\begin{theorem}{“二分法”算法}
初始条件:

给定函数 $f(x)$ 和一个函数连续的区间 $[a, b]$,且满足$f(a) \cdot f(b) < 0$,给出两个精度,一个是最小区间长度$d$,另一个是与零值接近的一个容许范围$\varepsilon$。

迭代过程:

\begin{enumerate}
\item 计算区间中点 $m = \frac{a + b}{2}$。
\item 计算 $f(m)$。
\item 判断 $f(m)$ 的符号:
\begin{itemize}
\item 如果 $f(m) = 0$,则 $m$ 就是零点。
\item 如果 $f(a) \cdot f(m) < 0$,则零点位于区间 $[a, m]$,将 $b$ 更新为 $m$。
\item 如果 $f(b) \cdot f(m) < 0$,则零点位于区间 $[m, b]$,将 $a$ 更新为 $m$。
\end{itemize}
\end{enumerate}

停止条件:

$|b - a|<d$ 或 $|f(m)|<\varepsilon$ 。此时,$m$ 可以作为零点的近似解。
\end{theorem}

迭代过程就是不断循环,直到符合停止条件为止。二分法的名称也来自于每次都要计算区间中点的行为。

二分法的收敛速度较为稳定,但相对较慢。当然,除此之外还有“牛顿法”等一系列方法,它们是使用计算机求解方程的重要工具,它们构成了一个名为\enref{数值分析}{NordEq}的学科。如果将求解零点的行为,看作搜索的话,二分法也是一种搜索方法。这两部分在高中并不涉及,提及只为扩展视野。

\subsection{变化率}\label{sub_HsFunC_4}

初中阶段就接触过斜率,斜率就像道路的坡度,坡度越大,路越陡。

\begin{definition}{斜率}
直线$y=f(x)$沿 $x$ 轴方向每前进单位长度时,$y$ 轴方向的变化量,表示直线倾斜程度,称为\textbf{斜率(Slope)},一般记作$k$,即:
\begin{equation}
k = \frac{f(x_2)-f(x_1)}{x_2-x_1}~.
\end{equation}
\end{definition}

事实上斜率本身也是直线与$x$正方向夹角的正切值,即$k=\tan\theta$。由于两点确定一条直线,在函数上任取不同的两点,就可以得到一条直线,这条直线的斜率称为平均变化率,表示函数在某个区间内的整体变化速度。

\begin{definition}{平均变化率}\label{def_HsFunC_3}
函数$f(x)$在某区间$[a, b],(a\neq b)$上的端点连线的斜率,称为函数 $f(x)$ 在区间 $[a, b]$ 上的\textbf{平均变化率(Average Rate of Change)},即:
\begin{equation}
\frac{\Delta y}{\Delta x}=\frac{f(b) - f(a)}{b - a}~.
\end{equation}
\end{definition}

平均变化率顾名思义,就像计算某段时间内的平均速度,反映了区间内整体的变化趋势。既然有整体趋势,就像瞬时速度,自然就有瞬时趋势。当两个点越来越靠近时,近到几乎可以认为是同一个点了,这时,连线就成为了在这个局部的切线。这时的平均变化率可以称为\textbf{瞬时变化率(instantaneous rate of change)},而它的另一个名字就是\textbf{导数(Derivative)}。\enref{导数}{HsDerv}是函数的性质之一,它是函数研究的一个非常重要的主题。

\subsection{单调性}\label{sub_HsFunC_1}

\textbf{单调性(Monotonicity)}粗略地描述了函数在某一区间内的函数值变化趋势。想象你在爬一座山,随着你向上走,如果海拔高度不断增加。这就是一个单调递增的过程。到达山顶后,开始下山时,如果海拔一直下降,这时可以认为是单调递减的过程。换句话说,函数的单调性揭示了函数在某段范围内是上升、下降,还是保持不变。

\begin{definition}{单调性}\label{def_HsFunC_1}
设$f(x)$是定义在$D$上的函数,若在$I\subseteq D$上,对$\forall x_1,x_2\in I,x_1< x_2$,函数均满足:
\begin{equation}
f(x_1)<f(x_2)~.
\end{equation}
则称函数$f(x)$在区间$I$上\textbf{单调递增(monotonically increasing})\footnote{此处采取高中教材上的定义。其实,满足这个条件时称作\textbf{严格单调递增 (Strictly increasing)},而单调递增则是指函数满足$f(x_1)\leq f(x_2)$时。递减也相同。这个概念会在大学阶段区分,目前给出作为提醒。},或函数$f(x)$在区间$I$上是\textbf{增函数};若满足
\begin{equation}
f(x_1)>f(x_2)~.
\end{equation}
则称函数$f(x)$在区间$I$上\textbf{单调递减(monotonically decreasing}),或函数$f(x)$在区间$I$上是\textbf{减函数}。

若函数在定义域上单调,则称为\textbf{单调函数 (monotonic functions)}。
\end{definition}

通常会习惯性地将单调性表达为:$y$随着$x$的增加而增加(或减少),但这种表述方式有引入因果性的嫌疑,即$x$先变化,$y$后变化。但这是一个历史问题,当前数学领域并不认为二者的变化有因果关系,因此尽管这种表达似乎更容易理解,但要知道他本身是存在问题的。

函数在某个区间上递增的充要条件是这个区间上任意两点的连线斜率为正\footnote{这可以通过移项,再相除得到。},同理,递减则为负。而这正是他们的平均变化率,因此:

\begin{theorem}{判断单调性的充要条件}
设$f(x)$是定义在$D$上的函数,若在$I\subseteq D$上,对$\forall x_1,x_2\in I,x_1\neq x_2$,记$\frac{\Delta y}{\Delta x}$为$x_1,x_2$的平均变化率,若
\begin{equation}
\frac{\Delta y}{\Delta x}>0~.
\end{equation}
恒成立,则函数$f(x)$在区间$I$上单调递增;若
\begin{equation}
\frac{\Delta y}{\Delta x}<0~.
\end{equation}
恒成立,则函数$f(x)$在区间$I$上单调递减。
\end{theorem}

注意,$\displaystyle f(x)=\frac{1}{x}$并非在定义域上递减,而是在两个区间上分别是递减。显然,如果从$y$轴左右分别选取一点连线,则平均变化率是正的。

根据导数与平均变化率的关系,可知还可以用导数来确定单调性。这一部分会在\enref{导数的性质}{HsDerC}中介绍。单调性可以辅助确定某函数在区间内的最大值或最小值,即在单调闭区间上,函数的最值在区间端点处取得。

\subsection{最值点与极值点}\label{sub_HsFunC_2}

就像山峰是起起伏伏的,函数也不总是只有一个“峰”,极值点就是用来描述各个“山峰和山谷”的。如果函数的单调性在某个点处由之前的递增转为之后的递减,那么这个点就称为\textbf{极大值点(Local Maximum Point)}。反之,如果函数的单调性在某个点处由之前的递减转为之后的递增,那么这个点就称为\textbf{极小值点(Local Minimum Point)}。二者合称为\textbf{极值点(Extremum Point )}。

就像地球上的珠穆朗玛峰或马里亚纳海沟,最值点描述的就是在整个定义域上最大、小的点。

\begin{definition}{最值}
对定义在$D$上的函数$f(x)$,若$x_0$满足:
\begin{equation}
\forall x\in D,f(x_0)\geq f(x)~.
\end{equation}
则称$x_0$为$f(x)$的\textbf{最大值点(Point of Maximum)},$f(x_0)$为$f(x)$的\textbf{最大值(Maximum Value)};
\begin{equation}
\forall x\in D,f(x_0)\leq f(x)~.
\end{equation}
则称$x_0$为$f(x)$的\textbf{最小值点(Point of Minimum)},$f(x_0)$为$f(x)$的\textbf{最小值(Minimum Value)}。
二者合称\textbf{最值(Extreme Value)}。
\end{definition}

最值有两种可能性,它或者是某个极值点,或者是定义域的区间端点(如果定义域是闭区间的话),实际求解时,需要把各个点对应的函数值列出来进行比较。一个函数不一定有最值,也不一定有极值。例如一个定义在开区间上的单调递增的函数,因为单调,不存在极值点,又因为无法取到端点的函数值,从而也不存在最值。

\subsection{对称性与奇偶性}

从几何角度来看,函数的\textbf{对称性(symmetry)}通常分为两种:轴对称性和中心对称性。这两个概念在初中阶段就有基础介绍。回忆一下,初中在学习这两个性质时分别举了$y=x^2$和$\displaystyle y={1\over x}$的例子。轴对称性意味着函数的图像可以沿某条垂直线“折叠”,两侧的部分会完全重合;中心对称性意味着如果将函数的图像围绕对称中心旋转 $180^\circ$,图像将保持不变。值得注意的是,无论是轴对称还是中心对称,讨论的都是函数自身的性质,它们揭示了函数在定义域上的对称行为。

下面的定义其实可以从上面介绍的两个特性的本质上推知,因此这里不给出推理过程,可以自己试试。

\begin{definition}{对称性}
如果函数$f(x)$对定义域$D$上的任意$x$均满足:
\begin{equation}
f(a-x)=f(a+x)~.
\end{equation}
则称,函数$f(x)$关于直线$x=a$\textbf{轴对称(Axial Symmetry)};
\begin{equation}
f(a-x)+f(a+x)=2b~.
\end{equation}
则称,函数$f(x)$关于点$(a,b)$\textbf{中心对称(Central Symmetry )}。
\end{definition}

定义要求函数的定义域关于$a$对称,也就是定义域满足$\forall x\in D,2a-x\in D$,否则会造成运算时函数未定义的情况,这也符合直觉。

满足对称性的函数,在对称区间上的单调性有规律:轴对称时单调性相反,中心对称时单调性相同。

有两个比较特殊的对称性称为\textbf{奇偶性(Parity)},其中偶函数是指关于$y$轴对称的函数,奇函数是指关于原点中心对称的函数,也就是上述两个定义中,$a=b=0$的特例。

\begin{definition}{奇偶性}
设函数$f(x)$定义在$D$上,且$D$是关于$0$对称的。若对任意的$x\in D$,有:
\begin{itemize}
\item $f(-x)=f(x)$,则称$f(x)$是\textbf{偶函数(Even Function)}。
\item $f(x)+f(-x)=0$,则称$f(x)$是\textbf{奇函数(Odd Function)}。
\end{itemize}
\end{definition}

一般为了同时检验函数的奇偶性,会直接求解$f(-x)$,若满足$f(-x)=f(x)$则是偶函数,若满足$f(-x)=-f(x)$则是奇函数。

注意,上面的轴对称性只给出了与$x$轴垂直的变化,因为如果不满足这个条件,则原本的函数可能不符合函数定义。这与函数\aref{旋转}{sub_FunTra_3}的情况类似。另外注意这里要区分轴对称性和反函数的区别。“轴对称性”指的是一个函数自身的性质,而“反函数”是两个函数之间的一种关系,满足反函数的两个函数关于$y=x$轴对称。

根据上面的规律,奇函数在$y$轴两侧单调性相同,偶函数在$y$轴两侧单调性相反,同时也可通过知道函数在对称位置上的取值得到对称前的函数值。奇偶性另一个应用就是在计算对称区间内的积分时,可以减少工作量。偶函数在对称区间内的积分等于一侧积分的二倍。奇函数则恒为0。这部分现在高中不涉及,此处作为扩展。

\subsection{周期性}

在生活中,很多事情都是有规律的,比如每天日出日落、四季轮回。数学中用周期性来描述这种“规律”。周期性的函数图象好比一首不断循环的旋律,它遵循着固定的步调,过一段时间就会“回到原点”,再继续以同样的方式变化。

\begin{definition}{周期}
设函数$f(x)$定义在$D$上,若存在非$0$常数$T$使得
\begin{equation}
\forall x\in D,f(x+T)=f(x)~.
\end{equation}
则称,$T$是函数的一个\textbf{周期(period)}。满足条件的最小正数称为\textbf{最小正周期(the least positive period 或 the fundamental period)}。
\end{definition}

周期函数具有一些特性:
\begin{itemize}
\item 周期函数不一定有最小正周期,比如函数部分介绍的\aref{狄利克雷函数}{eq_functi_1},由于任意有理数都是他的周期,而不存在最小的正有理数。又如常函数$f(x)=c,c\in\mathbb{R}$,任意实数都是它的周期,从而同样不存在最小正周期。
\item 如果$T$是函数的一个周期,那么他的整数倍,如$2T$、$3T$、$-T$等都是函数的一个周期。因此,一般会将全部的周期写为最小正周期的整数倍形式,即$kT,k\in \mathbb{Z}$,这里$T$是最小正周期。
\item 周期函数的图像会在每个周期$T$内重复。无论在$x$轴上平移多少个周期,函数的形状和取值都会保持不变。
\item 两个周期分别为$T_1,T_2$的周期函数,只有满足它们的周期之比$\displaystyle k={T_1\over T_2}$为有理数,即$\displaystyle k={p\over q},p,q\in\mathbb{N}^*$时,他们的和才是周期函数,周期为$T=pT_2=qT_1$。周期相同可以认为是二者周期之比为$1$的特殊情况,此时和仍然是周期函数且周期为原周期。
\item 周期函数在定义域内一定不是单调的,因此周期函数也没有反函数。
\end{itemize}

高中阶段涉及的周期函数主要是两类:一类是抽象函数,也就是不给出表达式,然后利用周期性的特性来等量替换;另一类是三角函数,这将在\enref{三角函数}{HsTrFu}的部分详细讲解。

\subsection{特殊的函数}

在高中阶段会涉及到的两种特殊的函数包括初等函数和分段函数。

\subsubsection{初等函数}

高中研究的函数都是初等函数。\textbf{初等函数(Elementary Functions)}指的是由基本初等函数经过基本运算(加减乘除)以及复合形成的函数。

基本初等函数:
\begin{itemize}
\item 常值函数
\item 幂函数
\item 指数函数
\item 对数函数
\item 三角函数
\item *反三角函数\footnote{这里为了完整,提及了反三角函数,但在高中阶段完全不涉及。}
\end{itemize}

初等函数之所以被称为初等函数就是因为它的性质很好。性质好一般指:具有确定的解析表达式,并且在它们的定义域上是连续和可导的甚至无穷阶可导,容易进行求导、积分等操作。这些性质简单说来就是连续且光滑。对于高中而言不用了解太细,只要知道,高中学习的内容在这些函数上进行操作时,基本都是适用的,不用证明它是否适用。而证明他们是否可用、为何可用,则是大学阶段很重要的一步内容。

\subsubsection{分段函数}

分段函数是指在定义域的不同子集上采用不同规则的函数,比如:
$$f(x) =
\begin{cases}
x^2,\qquad&x \geq 0 \\
-x^2,\qquad&x < 0
\end{cases}~.$$

它在描述不连续现象、特定区间内的行为时非常有效。当然除了强行将两个函数拼成一个函数之外,还有一些常见的分段函数。下面会给出定义,并稍作探讨。

\subsubsection{绝对值函数}

\textbf{绝对值函数(Absolute Value Function)}在初中就已经非常熟悉了。它用于度量一个数与零的距离,将所有负数映射为相反数,而正数和零保持不变。

\begin{definition}{绝对值函数}\label{def_HsFunC_2}
\begin{equation}
|x| =\sqrt{x^2}=
\begin{cases}
x,\qquad&x > 0 \\
0,\qquad&x=0\\
-x,\qquad&x < 0
\end{cases}~.
\end{equation}
\end{definition}
\addTODO{绝对值函数图像}

它的性质为:
\begin{itemize}
\item 连续,$(-\infty,0]$递减,$[0,+\infty)$递增、偶函数
\item 图像由$y=x$和$y=-x$在上半平面的部分构成。
\item 在$0$处是一个“尖点”,不可导。
\end{itemize}

\subsubsection{取整函数}

\textbf{取整函数(Floor Function)},也称为\textbf{下取整函数},是定义在$\mathbb{R}$上的一个函数,它将给定实数向下舍入为不大于该数的最大整数,即

\begin{definition}{取整函数}
\begin{equation}
[x] = n,\qquad n \leq x<n+1~.
\end{equation}
\end{definition}

也就是取整操作是将小数部分舍去,并向下取整。

\addTODO{取整函数图像}

它的性质为:
\begin{itemize}
\item 非连续、非单调、非奇偶($[0.5]=0,[-0.5]=-1$)
\item 图像是一个阶梯状的函数,随着 $x$ 的增加,每次当 $x$ 跨越一个整数时,函数值就会“跳跃”到下一个整数。
\end{itemize}

\subsubsection{狄利克雷函数}

这个函数在介绍函数时就已经见过,$D(x)$是定义在$\mathbb{R}$上的一个函数,它在自变量取有理数时为1,否则为0,即:
\begin{definition}{狄利克雷函数}
\begin{equation}
D(x)=\begin{cases}
1,\qquad& x\in\mathbb{Q} \\
0,\qquad& x\notin\mathbb{Q}  \\
\end{cases} ~.
\end{equation}
\end{definition}

他是很多性质的特例:
\begin{itemize}
\item 非连续、非单调
\item 偶函数
\item 没有最小正周期的周期函数
\item 无法画出图像
\end{itemize}

最后,除却本文提到的性质,还有一些性质是高中不会涉及到的,此处给出供感兴趣的同学了解:
\begin{itemize}
\item \enref{极限}{FunLim}(Limits)
\item \enref{连续性}{contin}(Continuity),一致连续
\item \enref{有界性}{conff}(Boundedness)
\item \enref{凹凸性}{conff}(Convexity/Concavity)
\item \enref{渐近线}{Asmpto}(Asymptotes):水平渐近线(Horizontal Asymptote)、垂直渐近线(Vertical Asymptote)、斜渐近线(Oblique Asymptote)
\end{itemize}