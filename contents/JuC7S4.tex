% Julia 第 7 章小结
% keys 第7章 小结

本文授权转载自郝林的 《Julia 编程基础》。 原文链接:\href{https://github.com/hyper0x/JuliaBasics/blob/master/book/ch07.md}{第 7 章 参数化类型}。


\subsubsection{7.4 小结}

我们在这一章主要讨论了参数化类型。参数化是 Julia 类型系统中的一个非常重要且强大的特性。它允许一个类型自身拥有参数,以使其可以代表一个完整的类型族群。在很多时候,参数化类型就相当于一种对数据结构的泛化定义,因此我们也可以称之为泛型(即泛化类型)。我们之前讲过的抽象类型、原语类型和复合类型都可以被参数化。其中,参数化复合类型最常用,但抽象类型的参数化意义更大。

除了这些核心的概念和编程方式之外,我们还讲述了与参数化有关的更多知识。这包括类型参数的值域、类型的类型,以及值化的表示法。其中,值化的表示法尤为重要。因为它可以使参数化类型的具体化变得非常的灵活,而且还可以让我们对参数化类型的认识更加深刻。

我们还用一定的篇幅介绍了 Julia 中的元组。元组是一种比较简单的参数化类型,同时它还是一种非常常用的容器。我们介绍了三种元组,即:普通元组、有名元组和可变参数元组。

在认真阅读这一章之后,我相信你会对 Julia 的参数化类型有一个正确且比较深入的认知。容器是广泛应用类型参数化的典型案例。我们在后面还会讲解更多、更复杂的容器。不过别担心,一旦你熟悉了元组,那么理解其他容器就会容易很多。