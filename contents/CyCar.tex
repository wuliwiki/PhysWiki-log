% 柱坐标与直角坐标的转换
% 柱坐标|直角坐标|转换


\pentry{柱坐标系\upref{Cylin}}
当我们讨论柱坐标和直角坐标的转换时, 通常令两个原点和 $z$ 轴重合, $\theta = 0$, $z = 0$ 为 $x$ 轴的正方向, $\theta = \pi/2$, $z = 0$ 为 $y$ 轴的正方向。 这时两种坐标之间的变换关系为。
\begin{equation}\label{eq_CyCar_1}
\begin{cases}
x = r\cos \theta \\
y = r\sin \theta  \\
z = z 
\end{cases}
\end{equation}

\begin{equation}\label{eq_CyCar_2}
\begin{cases}
r = \sqrt{x^2+y^2} \\
\theta = \Arctan(y,x)  \\
z = z 
\end{cases}
\end{equation}
其中 $\opn{Arctan}$ 是四象限反正切函数\upref{Arctan}, 也记为 $\opn{atan2}$。 注意根据\autoref{eq_CyCar_1} , 同一个直角坐标可以对应不同的极坐标, 例如将 $\theta$ 增加 $2\pi$ 的整数倍, 直角坐标不变。 但根据\autoref{eq_CyCar_2} , 我们可以找到两种坐标间的一一对应\upref{Set}关系。


\subsection{矢量变换}
两组基底之间的变换关系为
\begin{equation}\label{eq_CyCar_3}
\begin{cases}
\uvec r = R_{11}\uvec x + R_{12}\uvec y + R_{13}\uvec z\\
\uvec \theta = R_{21}\uvec x + R_{22}\uvec y + R_{23}\uvec z\\
\uvec z = R_{31}\uvec x + R_{32}\uvec y + R_{33}\uvec z
\end{cases}
\end{equation}
\begin{equation}\label{eq_CyCar_4}
\begin{cases}
\uvec x = R_{11} \uvec r + R_{21} \uvec \theta  + R_{31} \uvec z \\
\uvec y = R_{12} \uvec r + R_{22} \uvec \theta  + R_{32} \uvec z \\
\uvec z = R_{13} \uvec r + R_{23} \uvec \theta  + R_{33} \uvec z
\end{cases}
\end{equation}
其中 $\mat R$ 是关于角度 $\theta$ 的三维旋转矩阵\upref{Rot3D}
\begin{equation}
\mat R = \pmat{
    \cos\theta & \sin\theta & 0\\
    -\sin\theta & \cos\theta & 0\\
    0 & 0 & 1
}
\end{equation}

若任意矢量 $\bvec v$ 在直角坐标系和球坐标系中分别表示为
\begin{equation}\label{eq_CyCar_6}
\bvec v = v_x \uvec x + v_y \uvec y + v_z \uvec z
\end{equation}
\begin{equation}\label{eq_CyCar_5}
\bvec v = v_r \uvec r + v_\theta \uvec \theta + v_z \uvec z
\end{equation}
则坐标变换关系可以用矩阵乘法表示
\begin{equation}\label{eq_CyCar_8}
\pmat{v_r \\ v_\theta \\ v_z}
= \mat R \pmat{v_x \\ v_y \\ v_z}
\end{equation}
\begin{equation}\label{eq_CyCar_7}
\pmat{v_x \\ v_y \\ v_z}
= \mat R\Tr \pmat{v_r \\ v_\theta \\ v_z}
\end{equation}
\subsection{推导}
空间中一点 $P$ 的位矢在 $xy$ 平面的分量为 $r$。 其可以进而分解成 $x$ 分量 和 $y$ 分量  $x = r\cos \theta$,  $y = r\sin \theta$, 而点 $P$ 垂直分量即为 $z$ ,这就得到了\autoref{eq_CyCar_1} 。有了\autoref{eq_CyCar_1} 中的三条关系,就可以很容易解出\autoref{eq_CyCar_2} 中的三条关系。

现在推导变换关系\autoref{eq_CyCar_3} 。由于 $\uvec r,\uvec \theta ,\uvec z $ 都是关于 $(r, \theta, z)$ 的函数,所以在考察一点 $(r, \theta, z)$ 时, $\uvec r$ 的柱坐标是 $(1, \theta, 0)$,  根据\autoref{eq_CyCar_1} 变换到直角坐标为
\begin{equation}
(\cos \theta,\,\sin \theta,\,0)
\end{equation}
写成矢量的形式,就是
 \begin{equation}
\uvec r = \cos \theta \,\uvec x + \sin \theta \,\uvec y 
\end{equation}
至于\autoref{eq_CyCar_3} 的第二条式子,在同一个柱坐标 $(r,\theta ,z)$ 处, $\uvec \theta $ 的柱坐标为 $(1, \theta + \pi /2, 0)$, 根据\autoref{eq_CyCar_1} 变换到直角坐标再化简就得到直角坐标和对应的矢量形式为
\begin{equation}
(-\sin \theta ,\,\cos \theta , \,0)
\end{equation}
\begin{equation}
\uvec \theta  = -\sin \theta  \,\uvec x + \cos \theta \,\uvec y
\end{equation}
对于 $\uvec z$ ,容易看出
\begin{equation}
\uvec z = \uvec z
\end{equation}
将基底变换\autoref{eq_CyCar_3}  和\autoref{eq_CyCar_4} 分别代入\autoref{eq_CyCar_5}  和\autoref{eq_CyCar_6} 得坐标变换\autoref{eq_CyCar_7} 和\autoref{eq_CyCar_8} ,详见 “三维旋转矩阵\upref{Rot3D}”。
\subsection{两方向的夹角}
\pentry{内积\upref{Dot}}
若已知柱坐标系中两个方向分别为 $(\sqrt{1-z_1^2}, \theta_1, z_1)$ 和 $(\sqrt{1-z_2^2}, \theta_2, z_2)$, 如何求它们之间的夹角 $\alpha$ 呢? 我们可以先计算两个单位矢量的直角坐标, 然后对它们进行内积即可得到两矢量夹角的余弦值。 由\autoref{eq_CyCar_1}, 两矢量的直角坐标分别为
\begin{equation}
(\sqrt{1-z_1^2}\cos\theta_1,\ \sqrt{1-z_1^2}\sin\theta_1,\ z_1)
\qquad
(\sqrt{1-z_2^2}\cos\theta_2,\ \sqrt{1-z_2^2}\sin\theta_2,\ z_2)
\end{equation}
利用三角恒等式(\autoref{eq_TriEqv_2}~\upref{TriEqv}), 得
\begin{equation}\ali{
\cos\alpha &= \sqrt{1-z_1^2}\sqrt{1-z_2^2}\cos\theta_1\cos\theta_2 +  \sqrt{1-z_1^2}\sqrt{1-z_2^2}\sin\theta_1\sin\theta_2 + z_1z_2\\
&=\sqrt{(1-z_1^2)(1-z_2^2)}\cos\theta_1\cos\theta_2 +  \sqrt{(1-z_1^2)(1-z_2^2)}\sin\theta_1\sin\theta_2 + z_1z_2
}\end{equation}
