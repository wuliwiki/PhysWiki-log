% Julia 第 6 章小结
% 第6章 小结

本文授权转载自郝林的 《Julia 编程基础》。 原文链接:\href{https://github.com/hyper0x/JuliaBasics/blob/master/book/ch06.md}{第 6 章 字符和字符串}。


\subsubsection{6.5 小结}

我们在本章主要讲解了字符和字符串。这两者都可以表示处于 Unicode 代码空间中的字符。但不同的是,前者只能表示一个字符,而后者可以表示多个字符。

我们首先简要地介绍了 ASCII 编码和 Unicode 编码标准,并提及了后者中的一种编码格式:UTF-8。Julia 通常采用 UTF-8 编码格式把字符转换为由若干个字节承载的二进制数。

然后,我们讲述了 Julia 中的字符值。这包括它的表示与操作和它的类型与转换方法。多个字符可以组成一个字符串。所以我们紧接着又讲了字符串值的表示以及在其类型之上的设定。这些设定是我们操作字符串值的基础。我们可以对字符串值做的操作有,获取长度、索引、截取、拼接、插值,以及搜索和比较。

除了常规的字符串值,我们还可以利用简单的前缀编写非常规的字符串值,以表示某类特殊值。比如,原始字符串、任意精度的整数和浮点数、版本号、正则表达式,以及只读的字节数组。在某些场景下,这些特殊值是非常有用的。

字符和字符串是我们在 Julia 编程过程中非常常用的两类值。它们的表示方式颇多,且操作方法多样。我们往往需要根据具体情况对它们加以合理的运用。最后再强调一下,字符值和字符串值都是不可变的!
