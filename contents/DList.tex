% 双链表
% 双链表|数据结构|链表|C++

本文对应了上一篇文章,这里讲解\textbf{双链表}。

双链表在竞赛中用的不多,通常是因为需要优化某些问题而使用双链表。

\begin{figure}[ht]
\centering
\includegraphics[width=14.25cm]{./figures/DList_1.png}
\caption{双链表示意图} \label{DList_fig1}
\end{figure}

双链表上的每个结点都有 $4$ 个值,$\mathtt{Lnext}$ 指针表示它左边的结点的下标,$\mathtt{Rnext}$ 指针表示它右边的结点的下标,其他的数组和变量和单链表\upref{List}的存储代表一个意思。

用 C++ 的指针和结构体来实现双链表:
\begin{lstlisting}[language=cpp]
struct Double_List
{
    int value;
    Double_List *prev, *next;  // 左指针和右指针
};
\end{lstlisting}

这里来详细的讲解一下如何使用数组来模拟双链表。

数组模拟双链表需要这么几个数组和变量:
\begin{lstlisting}[language=cpp]
// 为了简化代码,我们用 e 表示 value, l 表示 Lnext, r 表示 Rnext
int e[N], l[N], r[N], idx;
\end{lstlisting}

双链表的基本操作有:\begin{enumerate}
\item 在第 $k$ 个插入的数左侧插入一个数;
\item 在第 $k$ 个插入的数右侧插入一个数;
\item 将第 $k$ 个插入的数删除
\end{enumerate}

插入 ($\text{insert}$) 操作:

\begin{figure}[ht]
\centering
\includegraphics[width=14.25cm]{./figures/DList_2.png}
\caption{在第 $k$ 个插入的点的右边插入一个数} \label{DList_fig2}
\end{figure}

第 $1$ 步:先让要插入的点的右指针指向第 $k$ 个插入的点右边的点

第 $2$ 步:让要插入的点的左指针指向第 $k$ 个插入的点

第 $3$ 步:让第 $k$ 个插入的点的右边的点的左指针指向要插入的点

第 $4$ 步:让第 $k$ 个插入的点的右指针指向要插入的点


\begin{lstlisting}[language=cpp]

// 在第 k 个插入的数的右边插入一个数
void insert(int k, int x)
{
    e[idx] = x;         // 赋值
    r[idx] = r[k];      // 第 1 步
    l[idx] = k;         // 第 2 步
    l[r[k]] = idx;      // 第 3 步
    // r[k] 是第 k 个插入的点的右边的点,
    // l[r[k] 就是第 k 个插入的点的右边的点的左指针
    r[k] = idx ++ ;     // 第 4 步
}


// 熟练掌握了之后就可以写成一行了。
e[idx] = x, r[idx] = r[k], l[idx] = k, l[r[k]] = idx, r[k] = idx ++ ;

\end{lstlisting}

\textbf{注意:}

第 $1$ 步和第 $2$ 步的操作的位置可互换,但是第 $3$ 步和第 $4$ 步的位置不可互换。
因为如果先操作第 $4$ 步的话,\verb|r[k]| 就是修改之后的值了,如果接下来再调用 \verb|l[r[k]]| 就不对了。

如果想要在第 $k$ 个插入的数的左边插入一个数怎么办呢?

最直接的办法就是照着刚才的逻辑对应着再写一遍,当然还有更见简便的方法。

就是直接调用 \begin{lstlisting}[language=cpp]
insert(l[k + 1], x);
// 注意 k 要加 1,因为 idx 从 2 开始,所以下标也从 2 开始
\end{lstlisting}
意思就是在第 $k$ 个插入的数的左边插入一个数。

删除($\text{remove}$)操作:
\begin{figure}[ht]
\centering
\includegraphics[width=14.25cm]{./figures/DList_3.png}
\caption{删除第 $k$ 个插入的点} \label{DList_fig3}
\end{figure}

双链表的删除操作和单链表的删除类似

\begin{enumerate}
\item 让第 $k$ 个插入的点的左边的结点的右指针指向第 $k$ 个插入的点的右边的结点
\item 让第 $k$ 个插入的点的右边的结点的左指针指向第 $k$ 个插入的点的左边的结点
\end{enumerate}

对应到代码上:

\begin{lstlisting}[language=cpp]

// 删除第 k 个插入的点
void remove(int k)
{
    r[l[k]] = r[k], l[r[k]] = l[k];
}

// l[k] 为第 k 个插入的点的左边的结点,r[l[k]] 为第 k 个插入的点的左边的结点的右指针
// r[k] 为第 k 个插入的点的右边的结点,l[r[k]] 为第 k 个插入的点的右边的结点的左指针
\end{lstlisting}

这样,双链表的基本操作就讲完了。