% 极限存在的判据、柯西序列

\pentry{序列的极限\upref{SeqLim}}

在之前的例子中, 我们都是在已经猜出序列极限的情况下来证明极限等式的。 但是, 对于比较复杂的序列, 又该如何判定它是否有极限?

第一个判据如下:

\begin{theorem}{}
单调有界的实数序列必然有极限。
\end{theorem}
\textbf{证明。} 不妨设 $\{a_n\}$ 是单调递增的序列, 而且有上界。 按照确界原理\upref{SupInf}, 数集 $\{a_n:n\in\mathbb{N}\}$ 有唯一的上确界 $A$, 也就是说成立如下两件事: 第一, $a_n\leq A$ 对于任何 $n$ 都成立; 第二, 任给 $\varepsilon>0$, 数 $A-\varepsilon$ 都不是数集 $\{a_n:n\in\mathbb{N}\}$ 的上界。 于是, 存在一个脚码 $N_\varepsilon$ 使得 $A-\varepsilon<a_{N_\varepsilon}\leq A$. 根据单调性, 这表示对于 $n>N_\varepsilon$ 总有
\begin{equation}
A-\varepsilon<a_n\leq A~,
\end{equation}




即 $|a_n-A|<\varepsilon$. 于是 $A$ 是序列 $\{a_n\}$ 的极限。 \textbf{证毕。}

\begin{exercise}{哪里用到了完备性?}
单调有界的有理数序列的极限不一定是有理数, 例如 $\sqrt{2}$ 的不足近似值序列
$$
1.4,\,1.41,\,1.414,...~
$$
这是因为有理数集不完备。 在上面的证明中, 哪里用到了实数集的完备性?
\end{exercise}

不过, 显然也有很多有极限的序列不是单调的, 例如震荡着收敛到零的序列 $\{(-1)^n/n\}$. 对于这种不单调的序列, 该怎么判断它是否有极限呢?

我们从一些直观推演开始。 如果序列 $\{a_n\}$ 有极限 $A$, 那么 $a_n$ 在脚码 $n$ 充分大时可以以任意小的误差逼近 $A$. 如果 $m,n$ 是两个充分大的脚码, 那么差值 $|a_m-a_n|$ 就可以用累进误差来控制:
$$
|a_m-a_n|\leq |a_m-A|+|a_n-A|~.
$$
于是差值 $|a_m-a_n|$ 最终总可以任意接近零。 这就启发我们提出如下的

\begin{theorem}{柯西判据}
序列 $\{a_n\}$ 有极限, 当且仅当任给 $\varepsilon>0$, 都存在正整数 $N_\varepsilon$, 使得当 $m,n>N_\varepsilon$ 时有 $|a_m-a_n|<\varepsilon$. 满足这个性质的序列称为基本列 (fundamental sequence) 或者柯西序列 (Cauchy sequence).
\end{theorem}

\textbf{证明。} 如果序列 $\{a_n\}$ 有极限 $A$, 那么在给定 $\varepsilon>0$ 后, 存在正整数 $N_\varepsilon$, 使得当 $n>N_\varepsilon$ 时有 $|a_n-A|<\varepsilon/2$. 从而当 $m,n>N_\varepsilon$ 时有
$$
|a_m-a_n|\leq|a_m-A|+|a_n-A|<\varepsilon~.
$$


反过来, 设序列 $\{a_n\}$ 满足柯西判据的条件。 于是它当然是有界的。 我们考虑如下的两个序列:
$$
x_n=\sup\{a_{n+1},a_{n+2},...\}~,
\quad
y_n=\inf\{a_{n+1},a_{n+2},...\}~.
$$
则 $\{x_n\}$ 是单调递减的, $\{y_n\}$ 是单调递增的 (设想一个越来越小的集合, 它的上界当然会越变越小, 而下界则会越变越大). 按照第一个判据, 序列 $\{x_n\}$ 有极限 $x$. 显然有 $y_n\leq x\leq x_n$, 同时
$$
y_n=\inf_{k\geq n}a_k\leq a_n\leq \sup_{k\geq n}a_k=x_n~.
$$
于是便有 $|a_n-x|\leq x_n-y_n$. 

另一方面, 由于 $\{a_n\}$ 是基本列, 故任给 $\varepsilon>0$, 都有整数 $N$ 使得 $k\geq N$ 时有 $|a_N-a_k|<\varepsilon/3$. 于是当 $n\geq N$ 时,
$$
a_N-\frac{\varepsilon}{3}\leq \inf_{k\geq n}a_k=y_n
\leq x_n=\sup_{k\geq n}a_k\leq a_N+\frac{\varepsilon}{3}~.
$$
这表示 $x_n-y_n\leq 3\varepsilon/3<\varepsilon$. 从而当 $n\geq N$ 时有 $|a_n-x|<\varepsilon$. 这表示 $\lim_{n\to\infty}a_n=x$.\textbf{证毕。}

\begin{exercise}{}
\begin{enumerate}
\item 在上述证明中, 序列 $y_n$ 也有极限 $y$. 上面的推理对于 $y$ 能不能成立? 由此, 能否证明实际上 $x=y$?
\item 写出柯西收敛准则的否命题形式。 也就是说, 怎样的序列不是柯西序序列, 怎样的序列不是发散的?
\end{enumerate}
\end{exercise}

柯西判据对于判别级数是否收敛是很基本的。
