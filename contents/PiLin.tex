% 皮卡-林德勒夫定理
% license Xiao
% type Tutor

\pentry{李普希茨条件\nref{nod_LipCon}}{nod_b471}
\textbf{皮卡-林德勒夫定理(Picard-Lindelöf theorem)}是分析数学中的一个基本定理, 又称为\textbf{柯西-李普希茨定理(Cauchy-Lipschitz theorem)}. 它断言: 常微分方程(组)的初值问题只需要满足一些非常宽泛的条件, 就是唯一可解的。 

由于许多经典物理问题都可以化归为常微分方程组, 所以皮卡-林德勒夫定理可以用来说明这些物理问题的决定论 (deterministic) 特性: \textbf{给定了系统的初始状态之后, 系统的演化就唯一确定了。}

对于不满足皮卡-林德勒夫定理条件的常微分方程组, 尚有皮亚诺存在定理。 后者无法保证解的唯一性。

\subsection{定理的表述与辨析}
\begin{theorem}{皮卡-林德勒夫定理}
设 $I\subset\mathbb{R}$ 是开区间, $X$ 是\enref{巴拿赫空间}{banach}, $U\subset X$ 是开集。 设有映射 $f:U\times I\to X$, 对于 $X$ 变量满足局部\enref{李普希茨条件}{LipCon}, 即对于任意 $x_0\in U$, $t_0\in I$, 都存在 $x_0$ 的小邻域 $\bar B_X(x_0,R)\subset U$ 和 $t_0$ 的小邻域 $[t_0-r,t_0+r]\subset I$, 以及一个正数 $L>0$, 使得对于任何 $x_1,x_2\in \bar B_X(x_0,R)$ 和 $t\in[t_0-r,t_0+r]$, 都有
$$
|f(x_1,t)-f(x_2,t)|_X\leq L|x_1-x_2|_X~.
$$

则对于任何 $t_0\in I$, $x_0\in U$, 都存在一个正数 $T>0$, 使得常微分方程的初值问题
$$
\frac{\dd}{\dd t}u(t)=f(u(t),t)~,\quad u(t_0)=x_0~
$$
在区间 $[t_0-T,t_0+T]\cap I$ 上有唯一解。
\end{theorem}

虽然定理的精确表述有点繁琐, 但它背后的意思很简单: \textbf{对于常微分方程}
\begin{equation}\label{eq_PiLin_1}
\frac{\dd}{\dd t}u(t)=f(u(t),t)~.
\end{equation}
\textbf{只要右边的函数 $f$ 满足李普希茨条件, 也就是下面 \autoref{eq_PiLin_2},那么它的初值问题就唯一可解。} 
\begin{equation}\label{eq_PiLin_2}
\exists \text{constant}\ L > 0, \forall (x, y_1), (x, y_2) \in D(\bar{D}), \left|f(x, y_1)-f(x, y_2)\right| \le L \left|y_1 - y_2\right| ~.
\end{equation}
其中,$f(x, y)$ 在区域 $D$(闭区域 $\bar{D}$)上有定义。满足 \autoref{eq_PiLin_2} 就称 $f(x, y)$ 在区域 $D$(闭区域 $\bar{D}$)上关于 $y$ 满足李普希茨条件。

一般来说, 实际问题中的 $f$ 都是连续可微的, 这其实比李普希茨条件还要强。

"唯一可解"的意思实际上比定理表述得还要更多。 在定理的表述中, 初值问题 \autoref{eq_PiLin_1} 只是局部唯一可解的。 但是, 如果 $t_0+T\in I$, 那么还可以将 $u(t_0+T)$ 作为点 $t_0+T$ 处的新初值, 从而将解沿着时间轴继续延拓下去。 按照这个思路, 就可以得到唯一的\enref{极大解}{MaxSol}.

在实际应用中, 空间 $X$ 一般都是实数空间 $\mathbb{R}^n$. 这时候 \autoref{eq_PiLin_1} 就是有 $n$ 个未知函数的常微分方程组。 对于形如
$$
y^{(n)}(t)=F(t,y(t),y'(t),...,y^{(n-1)}(t))~
$$
的 $n$ 阶方程, 只要命
$$
u(t)=\left(\begin{array}{c}
y(t)\\
y'(t)\\
...\\
y^{(n-1)}(t)
\end{array}
\right)~,\quad
f(u(t),t)=\left(\begin{array}{c}
u_2(t)\\
u_3(t)\\
...\\
F\left(t,u_1(t),u_2(t),...,u_n(t)\right)
\end{array}
\right)~,
$$
就得到了有 $n$ 个未知函数的常微分方程组。 这表示: \textbf{对于 $n$ 阶常微分方程, 如果要确定它的一个特解, 一般来说需要给定它的直到 $n-1$ 阶导数在某点处的值。}

\subsection{证明}
对于给定的 $t_0\in I$ 和 $x_0\in U$, 就取定理表述中的邻域 $\bar B_X(x_0,R)\subset U$ 和 $[t_0-r,t_0+r]\subset I$. 映射 $f$ 在 $\bar B_X(x_0,R)\times[t_0-r,t_0+r]$ 上是有界的, 不妨设它的上界为 $M$. 对于 $T\leq r$, 命 $J_T=[t_0-T,t_0+T]$. 设 $\mathfrak{X}_T$ 是从 $J_T$ 出发的到 $\bar B_X(x_0,R)$ 的连续映射的集合, 赋予度量
$$
\|u_1(t)-u_2(t)\|:=\sup_{t\in J_T}|u_1(t)-u_2(t)|_X~.
$$
则 $\mathfrak{X}_T$ 是一个完备度量空间。 考虑它上面的非线性算子
$$
(\Phi u)(t):=x_0+\int_{t_0}^tf(u(s),s)\dd s~.
$$ 
这个算子将 $\mathfrak{X}_T$ 的映射变换为从 $J_T$ 到 $X$ 的映射。 则初值问题 \autoref{eq_PiLin_1} 等价于不动点型方程 $u=\Phi u$. 这提示我们可以使用\enref{压缩映像原理}{ConMap}.

欲使得 $\Phi$ 将 $\mathfrak{X}_T$ 映射为自身, 只需要 $T\leq R/M$; 实际上, 这时候
$$
\|\Phi u-x_0\|
\leq\sup_{t\in J_T}\left|\int_{t_0}^tf(u(s),s)\dd s\right|_X
\leq TM\leq R~,
$$
在此基础上, 欲使得 $\Phi$ 成为压缩映射, 只需要 $T<1/2L$; 实际上, 这时候对于 $u_1,u_2\in\mathfrak{X}_T$ 有
$$
\begin{aligned}
\|\Phi u_1-\Phi u_2\|
&\leq\sup_{t\in J_T}\left|\int_{t_0}^t[f(u_1(s),s)-f(u_2(s),s)]\dd s\right|_X \\
&\leq TL\|u_1-u_2\|\\
&<\frac{1}{2}\|u_1-u_2\|~.
\end{aligned}
$$
所以, 取 $T=\min\left(1/2L,R/M\right)$, 即可保证 $\Phi$ 是 $\mathfrak{X}_T$ 到自己的压缩映像, 从而有唯一不动点。 这不动点正是初值问题 \autoref{eq_PiLin_1} 的唯一解。 \textbf{证毕。}

由于用到了压缩映像原理来证明, 便可以由此得到收敛到解的近似解序列。 因此, 皮卡-林德勒夫定理实际上给出了一个近似求解常微分方程(组)的算法: 对于初值问题 \autoref{eq_PiLin_1} , 近似解可以由迭代序列
$$
u_{n+1}(t)=x_0+\int_{t_0}^tf(u_n(s),s)\dd s~
$$
给出, 而且这个序列收敛到真解的速度是指数式的。 这个算法叫做\textbf{逐次迭代法}. 由于极大解的存在, 这个迭代算法实际上并不受 $t\in [t_0-T,t_0+T]$ 的限制, 而可以取为极大存在区间中的任意一点。

\subsection{应用}
显然, 皮卡-林德勒夫定理保证了这样一个事实: 

\textbf{对于一个给定的常微分方程(组)的初值问题, 只要它满足皮卡-林德勒夫定理的条件(这是很宽泛的), 那么不论用什么办法求得它的解都一定是问题的唯一解。}

这就为许多初等的推理提供了逻辑上的保证。 例如, 对于常系数\enref{二阶线性方程}{Ode2}
$$
u''+au'+bu=0~,
$$
我们过去"猜测"它的解应该是形如 $e^{rt}$ 的函数的线性叠加, 其中 $r$ 满足特征方程 $r^2+ar+b=0$. 现在有了皮卡-林德勒夫定理, 我们便可以保证解一定是这种形式。 

以上推理在微分几何学中有重要的用处。 它保证了流形上光滑向量场的流一定是局部存在且唯一的。

进一步地, 正如开头所说, 皮卡-林德勒夫定理保证了许多常见的经典力学系统的决定论特性。 例如, 对于\enref{哈密顿系统}{HamCan}, 只要哈密顿函数在相空间的区域上是连续可微的, 那么给定了广义坐标和广义动量的初始值之后, 系统就一定存在唯一的演化。 如果回到牛顿力学的语言, 这表示: \textbf{在一个经典力学系统中, 只要力场是连续可微的, 那么给定了质点系的初始位置和初始速度, 系统的演化就唯一决定了。}

注意, 经典力学系统的决定论特性与所谓的"混沌特性", 例如解对微扰的敏感性或者遍历性都不冲突; "混沌特性"描述的是系统的长期行为, 在足够短的时间尺度之下, 系统仍然可以是决定论式的。

最后, 有必要给出一个皮卡-林德勒夫定理不成立的例子来说明以上逻辑的局限。 这再次显示出现代物理学的一个基本思想: \textbf{任何物理系统的数学模型都有其适用范围。}
\begin{example}{非唯一的解}
考虑一个很简单的初值问题
$$
u'(t)=\sqrt{|u(t)|}~,\quad u(0)=0~.
$$
显然 $u(t)\equiv0$ 是一个解, 但分离变量后积分可以看出 $u(t)=t^2/4$ 也是问题的解。 之所以会出现这种非唯一性, 是因为函数 $\sqrt{|x|}$ 在 $x=0$ 的任何邻域内都不满足李普希茨条件。 然而, 如果把 $\sqrt{|x|}$ 视为一维空间中某流体的流场, 这就表示流体微团的行为在 $x=0$ 这个奇点附近是不明确的 (或者说, 出现了某种类似"湍流"的特性). 我们可以对此作出一个更物理的解释: 之所以出现这样的问题, 无非是因为在 $x=0$ (即"湍流"出现的点) 附近不能再用流体速度场去描述流体的运动, 在这里必须引入更精确的模型。
\end{example}
