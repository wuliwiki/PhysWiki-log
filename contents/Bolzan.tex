% 伯恩哈德·波尔查诺(综述)
% license CCBYSA3
% type Wiki

本文根据 CC-BY-SA 协议转载翻译自维基百科\href{https://en.wikipedia.org/wiki/Bernard_Bolzano}{相关文章}。

\begin{figure}[ht]
\centering
\includegraphics[width=6cm]{./figures/f44771c523b92573.png}
\caption{} \label{fig_Bolzan_1}
\end{figure}
伯纳德·博尔查诺(UK: /bɒlˈtsɑːnoʊ/, US: /boʊltˈsɑː-, boʊlˈzɑː-/;德语:[bɔlˈtsaːno];意大利语:[bolˈtsaːno];原名伯纳尔杜斯·普拉西杜斯·约翰·内波穆克·博尔查诺;1781年10月5日–1848年12月18日)是捷克数学家、逻辑学家、哲学家、神学家和天主教神父,具有意大利血统,以其自由主义观点而著称。

博尔查诺使用德语写作,这是他的母语。[6] 大部分他的工作是在他去世后才获得广泛关注。
\subsection{家庭}  
博尔查诺是两位虔诚天主教徒的儿子。他的父亲,伯纳德·庞培乌斯·博尔查诺,是一位意大利人,曾移居布拉格,在那里娶了来自布拉格讲德语的毛雷尔家族的玛丽亚·凯瑟莉亚·毛雷尔。他们的十二个孩子中只有两个活到成年。
\subsection{职业生涯}  
博尔查诺十岁时进入布拉格的皮亚尔修道士中学,期间从1791年到1796年就读于该校。[7]

博尔查诺于1796年进入布拉格大学,学习数学、哲学和物理学。从1800年起,他开始学习神学,并于1804年成为一名天主教神父。1805年,他被任命为布拉格大学宗教哲学新设立的讲座教授。[5] 他不仅在宗教领域,甚至在哲学领域也是一位受欢迎的讲师,并于1818年被选为哲学系系主任。

博尔查诺因他关于军事主义的社会浪费以及战争不必要性的教义,疏远了许多教职员工和教会领袖。他主张全面改革教育、社会和经济制度,旨在将国家的利益引导向和平,而非国家之间的武装冲突。他的政治信仰,虽然他经常与他人分享,最终被奥地利当局视为过于自由。在1819年12月24日,他因拒绝撤回自己的信仰而被解除教授职务,并被流放到乡村,之后将精力投入到他关于社会、宗教、哲学和数学的著作中。

尽管被禁止在主流期刊上发表文章作为流放的条件,博尔查诺依然继续发展他的思想,并通过个人或在东欧不太知名的期刊上发表他的作品。1842年,他回到布拉格,并于1848年去世。
\subsection{数学工作}  
博尔查诺对数学做出了若干原创性的贡献。他的总体哲学立场是,与当时流行的许多数学观念相反,最好不要将直观的概念,如时间和运动,融入数学之中。[8] 为此,他是最早开始将严谨性引入数学分析的数学家之一,他的三部主要数学著作《Beyträge zu einer begründeteren Darstellung der Mathematik》(1810年)、《Der binomische Lehrsatz》(1816年)和《Rein analytischer Beweis》(1817年)。这些著作展示了“……一种发展分析的新方法”,其最终目标直到五十年后才实现,当时这些著作引起了卡尔·魏尔施特拉斯的关注。[9]

在数学分析的基础方面,博尔扎诺为引入完全严谨的ε–δ极限定义做出了贡献。博尔查诺是第一个认识到实数具有下确界性质的数学家。[10] 像他同时代的许多人一样,他对戈特弗里德·莱布尼茨的无穷小概念持怀疑态度——这一概念曾是微积分的早期假定基础。博尔扎诺的极限概念与现代的相似:极限不是无穷小量之间的关系,而是应当通过依赖变量如何在自变量接近某个确定的数值时接近另一个确定的数值来表述。

博尔查诺还给出了代数基本定理的第一个纯分析证明,这一定理最初是高斯从几何角度证明的。他还给出了中值定理(也称为博尔扎诺定理)的第一个纯分析证明。今天,他最为人们所记得的是博尔查诺–魏尔施特拉斯定理,这一定理由卡尔·魏尔施特拉斯独立发展并在博尔扎诺的证明发表多年后才公布,最初被称为魏尔施特拉斯定理,直到博尔扎诺的早期工作被重新发现后,才改为博尔扎诺–魏尔施特拉斯定理。[11]
\subsection{哲学著作}  
博尔查诺(Bolzano)死后出版的《无限的悖论》(*Paradoxien des Unendlichen*,1851年)受到许多后来的杰出逻辑学家的高度赞赏,包括查尔斯·桑德斯·皮尔士(Charles Sanders Peirce)、乔治·坎托尔(Georg Cantor)和理查德·德德金德(Richard Dedekind)。然而,博尔查诺最为人称道的作品是他的《科学学说》(*Wissenschaftslehre*,1837年),这是一本四卷本的著作,涵盖了现代意义上的科学哲学、逻辑学、认识论和科学教育学。他在这部作品中发展出的逻辑理论被认为是开创性的。其他著作包括四卷本的《宗教学教程》(*Lehrbuch der Religionswissenschaft*)和形而上学著作《永生论》(*Athanasia*),后者为灵魂不朽提供了辩护。博尔查诺还在数学领域做出了有价值的工作,直到奥托·斯托尔茨(Otto Stolz)在1881年重新发现并重新出版了他许多失传的期刊文章,这些工作才得到了广泛的认可。
\subsubsection{《科学学说》(Wissenschaftslehre)}  
在他1837年的《科学学说》中,博尔查诺试图为所有科学提供逻辑基础,建立在诸如部分关系、抽象对象、属性、句型、思想和命题本身、总和与集合、集合体、物质、依附、主观思想、判断和句子发生等抽象概念之上。这些尝试是他在数学哲学领域早期思想的延伸,例如他1810年的《贡献》(*Beiträge*),在其中他强调了逻辑后果之间的客观关系与我们主观上对这些关系的认识之间的区别。对博尔查诺而言,仅仅确认自然或数学真理是不够的,科学(无论是纯粹科学还是应用科学)的适当职能是基于可能对我们直觉显而易见或不显而易见的基本真理来寻求合理性。

在数学分析的基础方面,博尔扎诺为引入完全严谨的ε–δ极限定义做出了贡献。博尔查诺是第一个认识到实数具有下确界性质的数学家。[10] 像他同时代的许多人一样,他对戈特弗里德·莱布尼茨的无穷小概念持怀疑态度——这一概念曾是微积分的早期假定基础。博尔扎诺的极限概念与现代的相似:极限不是无穷小量之间的关系,而是应当通过依赖变量如何在自变量接近某个确定的数值时接近另一个确定的数值来表述。

博尔查诺还给出了代数基本定理的第一个纯分析证明,这一定理最初是高斯从几何角度证明的。他还给出了中值定理(也称为博尔扎诺定理)的第一个纯分析证明。今天,他最为人们所记得的是博尔查诺–魏尔施特拉斯定理,这一定理由卡尔·魏尔施特拉斯独立发展并在博尔扎诺的证明发表多年后才公布,最初被称为魏尔施特拉斯定理,直到博尔扎诺的早期工作被重新发现后,才改为博尔扎诺–魏尔施特拉斯定理。[11]
\subsection{哲学著作}  
博尔查诺(Bolzano)死后出版的《无限的悖论》(Paradoxien des Unendlichen,1851年)受到许多后来的杰出逻辑学家的高度赞赏,包括查尔斯·桑德斯·皮尔士(Charles Sanders Peirce)、乔治·坎托尔(Georg Cantor)和理查德·德德金德(Richard Dedekind)。然而,博尔查诺最为人称道的作品是他的《科学学说》(Wissenschaftslehre,1837年),这是一本四卷本的著作,涵盖了现代意义上的科学哲学、逻辑学、认识论和科学教育学。他在这部作品中发展出的逻辑理论被认为是开创性的。其他著作包括四卷本的《宗教学教程》(Lehrbuch der Religionswissenschaft)和形而上学著作《永生论》(Athanasia),后者为灵魂不朽提供了辩护。博尔查诺还在数学领域做出了有价值的工作,直到奥托·斯托尔茨(Otto Stolz)在1881年重新发现并重新出版了他许多失传的期刊文章,这些工作才得到了广泛的认可。
\subsubsection{《科学学说》(*Wissenschaftslehre*)}  
在他1837年的《科学学说》中,博尔查诺试图为所有科学提供逻辑基础,建立在诸如部分关系、抽象对象、属性、句型、思想和命题本身、总和与集合、集合体、物质、依附、主观思想、判断和句子发生等抽象概念之上。这些尝试是他在数学哲学领域早期思想的延伸,例如他1810年的《贡献》(Beiträge),在其中他强调了逻辑后果之间的客观关系与我们主观上对这些关系的认识之间的区别。对博尔查诺而言,仅仅确认自然或数学真理是不够的,科学(无论是纯粹科学还是应用科学)的适当职能是基于可能对我们直觉显而易见或不显而易见的基本真理来寻求合理性。

\textbf{《科学学说导论》} 
 
博尔查诺在他的著作中首先解释了他所指的“科学学说”以及我们的知识、真理与科学之间的关系。他指出,人类知识是由所有真理(或真命题)构成的,这些是真正被人类知道或曾经知道的。然而,这只是所有存在的真理中的一小部分,尽管对一个人来说仍然太庞大而无法完全理解。因此,我们的知识被划分为更易于接触的部分。博尔查诺称这种真理的集合为科学(Wissenschaft)。需要注意的是,科学中的所有真命题并不需要被人类所知道;因此,这也是我们能够在科学中做出发现的原因。

为了更好地理解和掌握科学的真理,人类创造了教科书(Lehrbuch),这些教科书当然只包含人类已知的科学真命题。但是,如何知道我们该如何划分知识,即哪些真理应该归为一类?博尔查诺解释说,我们最终会通过某种反思来知道这一点,而如何划分我们知识为不同科学的规则将本身成为一门科学。这门告诉我们哪些真理应该归为一类并应在教科书中加以解释的科学,就是《科学学说》(*Wissenschaftslehre*)。

\textbf{形而上学} 
在《科学学说》中,博尔查诺主要关注三个领域:\\

(1) 语言领域,由单词和句子构成。\\  
(2) 思维领域,由主观思想和判断构成。\\  
(3) 逻辑领域,由客观思想(或思想本身)和命题本身构成。

博尔查诺将《科学学说》中的大部分篇幅用来解释这些领域及其相互关系。

他的系统中有两个重要的区分。首先,是部分与整体的区分。例如,单词是句子的部分,主观思想是判断的部分,客观思想是命题本身的部分。其次,所有对象都可以分为两类:一类是存在的对象,意味着它们在因果上有联系,并且位于时间和/或空间中;另一类是不存在的对象。博尔查诺的独创性观点是,逻辑领域是由后一类对象所构成的。

\textbf{命题本身(Satz an Sich)} 

\textbf{Satz an Sich} 是博尔查诺《科学学说》中的一个基本概念。它在第19节一开始就被引入。博尔查诺首先引入了“命题”(无论是口头的、书面的、思维的,还是命题本身)和“思想”(无论是口头的、书面的、思维的,还是思想本身)的概念。“草是绿色的”是一个命题(*Satz*):在这组词汇的结合中,某事被陈述或断言。然而,“草”只是一个思想(Vorstellung)。它代表着某种事物,但并没有断言任何事情。博尔查诺的命题概念相当广泛:“矩形是圆的”也是一个命题——尽管它由于自我矛盾而是假的——因为它是由可以理解的部分以可理解的方式构成的。

博尔查诺并没有给出命题本身(*\textbf{Satz an Sich})的完整定义,但他提供了足够的信息来帮助我们理解他的意思。命题本身(\textbf{Satz an Sich})有以下特点:(i)没有存在(即:它没有时间或地点的位置);(ii)它要么为真,要么为假,无论是否有任何人知道或认为它为真或为假;(iii)它是思维存在体所“把握”的。因此,书面句子(例如“苏格拉底有智慧”)把握了命题本身,即命题[苏格拉底有智慧]。书面句子确实存在(它在某一时刻位于某一地点,比如现在就在你的计算机屏幕上),并表达了命题本身,而命题本身则位于“本身的领域”(即an sich)。  
(博尔查诺对*an sich*一词的使用与康德的用法有很大不同;关于康德对该词的使用,参见an sich。)

每个命题本身都是由本身的思想构成的(为了简便起见,我们将“命题”指代为“命题本身”,将“思想”指代为客观思想或本身的思想)。思想通过否定方式定义为命题的组成部分,而这些部分本身并不是命题。一个命题至少由三个思想组成,即:一个主语思想,一个谓语思想和联结词(即“有”或“有的”其他形式)。(虽然也有包含命题的命题,但我们现在不考虑这些情况。)

博尔查诺识别了几种类型的思想。有简单思想,它没有部分(例如,博尔查诺用[某物]作为例子),但也有复杂思想,它由其他思想组成(博尔查诺举了[无]的例子,它由[不]和[某物]两个思想构成)。复杂思想可以有相同的内容(即相同的部分),却不一定是相同的——因为它们的组成部分连接方式不同。[一支黑色笔配蓝色墨水]的思想与[一支蓝色笔配黑色墨水]的思想不同,尽管这两种思想的部分是相同的。

\textbf{思想与对象}  

理解思想不需要具有对象这一点非常重要。博尔查诺使用“对象”一词来表示被思想所表征的事物。具有对象的思想代表该对象。但没有对象的思想则代表什么也没有。(不要被术语搞混:没有对象的思想就是没有表征的思想。)

为了进一步说明,博尔查诺使用了一个例子。思想[圆形的正方形]没有对象,因为应当被表征的对象是自我矛盾的。另一个例子是思想[无],它显然没有对象。然而,命题[圆形正方形的思想具有复杂性]的主语思想是[圆形正方形的思想]。这个主语思想确实有一个对象,即思想[圆形正方形]。但是,这个思想本身没有对象。

除了没有对象的思想外,还有一些思想只有一个对象,例如思想[登月的第一个人]只代表一个对象。博尔查诺称这些思想为“单一思想”。显然,也有思想具有多个对象(例如[阿姆斯特丹的市民]),甚至具有无限多个对象(例如[质数])。

\textbf{感觉与简单思想}

博尔查诺有一个复杂的理论来解释我们是如何感知事物的。他通过“直觉”这一术语来解释感觉,德语中称为Anschauung。直觉是一个简单的思想,它只有一个对象(Einzelvorstellung),除此之外,它也是独特的(博尔查诺需要这个来解释感觉)。直觉(Anschauungen)是客观思想,它们属于“本身的”领域,这意味着它们没有存在。正如所说,博尔查诺对直觉的论证是通过对感觉的解释来进行的。

当你感知一个真实存在的对象时,比如一朵玫瑰,发生的情况是:玫瑰的不同方面,如它的气味和颜色,导致你产生一种变化。这种变化意味着,在感知玫瑰之前和之后,你的心智状态是不同的。因此,感觉实际上是你心智状态的变化。这个变化与对象和思想有什么关系呢?博尔查诺解释说,这种变化在你的心智中本质上是一个简单的思想(Vorstellung),比如“这个气味”(这朵特定玫瑰的气味)。这个思想代表着它的对象——变化。除了是简单的,这种变化还必须是独特的。这是因为,字面上讲,你不能两次体验完全相同的感觉,也不能两个同时闻到同一朵玫瑰的人,拥有完全相同的嗅觉体验(尽管它们会非常相似)。因此,每一次感觉都会引起一个独特的(新的)简单思想,其对象是某种特定的变化。

现在,这个思想在你的心智中是一个主观思想,意味着它在特定的时间属于你。它有存在。但是,这个主观思想必须与一个客观思想相对应,或者说,它的内容是一个客观思想。这就是博尔查诺引入直觉(Anschauungen)的地方;它们是简单、独特且客观的思想,对应于我们由感觉引起的变化的主观思想。因此,每一个单独的可能的感觉都有一个相应的客观思想。从图示的角度来看,整个过程是这样的:每当你闻到一朵玫瑰时,它的气味会在你身上引起一个变化。这个变化是你对特定气味的主观思想的对象。这个主观思想对应于直觉或Anschauung

\textbf{逻辑学}。

根据博尔查诺的观点,所有命题都是由三个(简单或复杂)元素组成的:主语、谓语和联结词。博尔查诺更喜欢使用“有”而非传统的联结词“是”。原因在于,“有”不同于“是”,它能够将一个具体的术语(例如“苏格拉底”)与一个抽象的术语(例如“秃头”)联系起来。博尔查诺认为,“苏格拉底有秃头”比“苏格拉底是秃头”更为合适,因为后者形式不够基本:“秃头”本身是由“某物”、“那个”、“有”和“秃头”这些元素构成的。博尔查诺还将存在命题简化为这种形式:“苏格拉底存在”仅仅变成了“苏格拉底有存在(Dasein)”。

在博尔查诺的逻辑理论中,一个重要的概念是“变式”概念:各种逻辑关系是通过命题在其非逻辑部分被替换为其他部分时,真值变化的不同来定义的。例如,逻辑分析命题是指所有非逻辑部分可以在不改变真值的情况下被替换的命题。如果两个命题在它们的某个组成部分x上是“兼容的”(verträglich),那么就至少有一个术语可以插入,使得两个命题都为真。命题Q如果可以从命题P中“推导”(ableitbar)出来,且相对于它们某些非逻辑部分,任何替换这些部分并使得P为真也使Q为真。那么,如果一个命题可以从另一个命题推导出来,且对于其所有非逻辑部分均成立,那么就说它是“逻辑上可推导”的。除了推导关系,博尔查诺还有一种更严格的“基础”关系(Abfolge)。这是一种不对称的关系,适用于真实命题之间,当其中一个命题不仅从另一个命题中可推导出来,而且还由另一个命题进行解释时。

\textbf{真理}

博尔查诺区分了“真”和“真理”在日常用法中的五种含义,所有这些含义他都认为是无问题的。这些含义按正当性排列如下:

I. \textbf{抽象客观含义}:真理表示一种属性,这种属性可以应用于一个命题,主要是一个“命题本身”,即基于这个属性,命题表达的内容在现实中与表达的内容相符合。反义词:虚假、虚伪、谎言。

II. \textbf{具体客观含义}:(a)真理表示一个具有抽象客观含义的真理属性的命题。反义词:(a)虚伪。

III. \textbf{主观含义}:(a)真理表示一个正确的判断。反义词:(a)错误。

IV. \textbf{集体含义}:真理表示一组或一系列真实的命题或判断(例如,圣经中的真理)。

V. \textbf{不正当含义}:真实表示某个对象在现实中正如某种称谓所表明的那样。例如“真正的上帝”。反义词:虚假、不真实、幻觉。

博尔查诺的主要关注点是具体客观含义:具体客观的真理或真理本身。所有真理本身都是一种命题本身。它们不存在,即它们不像思维和言语的命题那样具有时空定位。然而,某些命题具有成为真理本身的属性。成为一个思维命题并不是真理本身概念的一部分,尽管考虑到上帝的全知,所有真理本身也都是思维真理。概念“真理本身”和“思维真理”是可以互换的,因为它们适用于相同的对象,但它们并不完全相同。

博尔查诺给出的正确的(抽象客观)真理定义是:一个命题为真,如果它表达的内容适用于其对象。因此,正确的(具体客观)真理的定义应当是:一个真理是一个命题,它表达的内容适用于其对象。这个定义适用于真理本身,而不是思维或已知的真理,因为这个定义中的所有概念都不从属于某种心理或已知的概念。

博尔查诺在《科学学说》 (§§31–32) 中证明了三件事:

至少存在一个真理本身(具体客观含义):\\
1. 1.没有真实的命题(假设)\\
2. 1.是一个命题(显然)\\
3. 1.是真实的(假设)并且是虚假的(因为1.)\\
4. 1.是自相矛盾的(因为3.)\\
5. 1.是虚假的(因为4.)\\
6. 至少存在一个真实的命题(因为1.和5.)。\\
B. 存在不止一个真理本身:\\
7. 只有一个真理本身,即 \(A\) 是 \(B\)(假设)\\
8. \(A\)是\(B\)是一个真理本身(因为7.)\\
9. 除了\(A\)是\(B \)之外没有其他真理本身(因为7.)\\
10. 9. 是一个真实的命题/一个真理本身(因为7.)\\
11. 存在两个真理本身(因为8.和10.)\\
12. 存在不止一个真理本身(因为11.)。\\
C. 存在无限多的真理本身:\\
13. 只有\(n\)个真理本身,即\(A\)是\(B\)....  \(Y\)是\(Z\)(假设)\\
14. \(A\)是\(B\)...... \(Y\)是\(Z\)是\(n\)个真理本身(因为13.)\\
15. 除了\(A\)是\(B\)...... \(Y\)是\(Z\)之外没有其他真理本身(因为13.)\\
16. 15. 是一个真实的命题/一个真理本身(因为13.)\\
17. 存在 \(n+1\) 个真理本身(因为14.和16.)\\
18. 步骤1到5可以对\(n+1\)重复,这将导致\(n+2\)个真理,以此类推,直到无限(因为\(n\)是一个变量)\\
19. 存在无限多的真理本身(因为18.)。

\textbf{判断与认识}
  
已知的真理由真理本身和判断(Bolzano, Wissenschaftslehre §26)组成。判断是陈述一个真实命题的思想。在判断过程中(至少当判断的内容是一个真实命题时),对象的观念与特征的观念以某种方式结合(§23)。在真实判断中,对象的观念与特征的观念之间的关系是一个实际/存在的关系(§28)。

每个判断都以命题为其内容,命题可以是真或假。每个判断都是存在的,但不是“为自己”(für sich)。与命题本身相比,判断依赖于主观的心理活动。然而,并不是每种心理活动都必须是判断;记住,所有判断都有命题作为内容,因此所有判断必须是对或错。单纯的呈现或思想是心理活动的例子,它们不必以陈述的形式存在,因此不是判断(§34)。

以真实命题为内容的判断可以称为认识(§36)。认识同样依赖于主体,因此与真理本身相对,认识是允许有程度差异的;一个命题可以更为或更少为人所知,但它不能更加或更少为真实。每一个认识必然包含一个判断,但并不是每个判断必然是认识,因为也有一些判断不是正确的。Bolzano 坚持认为没有所谓的虚假认识,只有虚假判断(§34)。
\subsection{哲学遗产}
Bolzano 周围形成了一个朋友和学生的圈子,他们传播了他的思想(即所谓的“Bolzano 圈”),但他思想对哲学的影响最初似乎注定是微小的。[3]

Alois Höfler(1853–1922),曾是 Franz Brentano 和 Alexius Meinong 的学生,后来成为维也纳大学的教育学教授,创造了“维也纳学派与奥地利 Bolzano 传统之间的失落环节。”[17] 然而,Bolzano 的工作通过 Edmund Husserl[18] 和 Kazimierz Twardowski[19],两位 Brentano 的学生,被重新发现。通过他们,Bolzano 成为现象学和分析哲学的塑造性影响。
\subsection{著作}
\begin{itemize}
\item Bolzano: Gesamtausgabe(Bolzano:全集),由 Eduard Winter、Jan Berg [sv]、Friedrich Kambartel 和 Bob van Rootselaar 主编,斯图加特:Fromman-Holzboog,1969 年起出版。(目前已出版 103 卷,28 卷正在准备中)。[20]  
\item Wissenschaftslehre,4 卷,第二版修订版由 W. Schultz 编,莱比锡 I–II 1929,III 1980,IV 1931;批判版由 Jan Berg 主编:Bolzano 的 Gesamtausgabe,第 11–14 卷(1985–2000)。  
\item Bernard Bolzano's Grundlegung der Logik. Ausgewählte Paragraphen aus der Wissenschaftslehre,第 1 和 2 卷,附有补充文本总结、引言和索引,编辑 F. Kambartel,汉堡,1963,1978²。  
\item Bolzano, Bernard(1810),Beyträge zu einer begründeteren Darstellung der Mathematik. Erste Lieferung(《数学更为严谨展示的贡献》;Ewald 1996,第 174–224 页,以及《伯恩哈德·博尔扎诺的数学著作》,2004,第 83–137 页)。  
\item Bolzano, Bernard(1817),Rein analytischer Beweis des Lehrsatzes, dass zwischen je zwey Werthen, die ein entgegengesetztes Resultat gewähren, wenigstens eine reele Wurzel der Gleichung liege(《纯粹的解析证明:在任何两值之间,如果它们给出的结果符号相反,那么至少存在一个实根》;Ewald 1996,第 225–248 页)。  
\item Franz Prihonsky(1850),Der Neue Anti-Kant, Bautzen(对《纯粹理性批判》的评估,由博尔扎诺的朋友 F. Prihonsky 去世后出版)。  
Bolzano, Bernard(1851),Paradoxien des Unendlichen,C.H. Reclam(《无穷的悖论》;Ewald 1996,第 249–292 页(摘录))。  
\end{itemize}

博尔扎诺的大部分著作都以手稿形式保存,因此流传较少,且对该学科的发展影响较小。
\subsubsection{翻译与编选}
\begin{itemize}
\item Theory of Science(《科学理论》),由 Rolf George 编辑并翻译,加利福尼亚大学出版社,伯克利与洛杉矶,1972 年。  
\item Theory of Science(《科学理论》),由 Jan Berg 编辑,附有引言,Burnham Terrell 从德语翻译,D. Reidel 出版公司,德尔夫特与波士顿,1973 年。  
\item Theory of Science(《科学理论》),首个完整的英文翻译版,分四卷,由 Rolf George 和 Paul Rusnock 翻译,牛津大学出版社,纽约,2014 年。  
\item The Mathematical Works of Bernard Bolzano**(《伯恩哈德·博尔扎诺的数学著作》),由 Steve Russ 翻译与编辑,牛津大学出版社,纽约,2004 年(2006 年再版)。  
\item On the Mathematical Method and Correspondence with Exner(《数学方法与与 Exner 的通信》),由 Rolf George 和 Paul Rusnock 翻译,Rodopi 出版社,阿姆斯特丹,2004 年。  
\item Selected Writings on Ethics and Politics(《伦理学与政治学选集》),由 Rolf George 和 Paul Rusnock 翻译,Rodopi 出版社,阿姆斯特丹,2007 年。  
\item Franz Prihonsky, The New Anti-Kant(《弗朗茨·普里霍恩斯基:新的反康德》),由 Sandra Lapointe 和 Clinton Tolley 编辑,帕尔格雷夫·麦克米兰出版社,纽约,2014 年。  
\item Russ, S. B.(1980)。"A translation of Bolzano's paper on the intermediate value theorem"(《博尔扎诺关于中值定理的论文翻译》)。*Historia Mathematica*,7(2):156–185。doi:10.1016/0315-0860(80)90036-1。(《纯粹解析证明:在任何两值之间,如果它们给出的结果符号相反,那么至少存在一个实根的定理》翻译,1817年布拉格出版)
\end{itemize}
\subsection{另见}  
\begin{itemize}
\item 罗马天主教科学家神职人员列表
\end{itemize}
\subsection{注释}  
\begin{enumerate}
\item Routledge 哲学百科全书(1998):“Ryle, Gilbert (1900-76)”。  
\item Sandra Lapointe,《Bolzano的逻辑实在论》,载于:Penelope Rush(编),《逻辑的形而上学》,剑桥大学出版社,2014年,第189–208页。  
\item Morscher, Edgar。“伯纳德·博尔扎诺”。载于Zalta, Edward N.(编)。《斯坦福哲学百科全书》。  
\item Paul Rusnock, Jan Sebestík, 《伯纳德·博尔扎诺:他的生平与著作》,牛津大学出版社,2019年,第33页。  
\item Chisholm, Hugh, 编(1911)。“博尔扎诺,伯纳德”。《大英百科全书》(第11版)。剑桥大学出版社。  
\item O'Hear, Anthony(1999),《康德以来的德国哲学》,皇家哲学研究所补充系列,皇家哲学研究所伦敦,第44卷,剑桥大学出版社,第110页,ISBN 9780521667821,他的母语是德语。  
\item Morscher, Edgar(2018),“伯纳德·博尔扎诺”,载于Zalta, Edward N.(编),《斯坦福哲学百科全书》(2018年冬季版),斯坦福大学形而上学研究实验室,检索于2024年4月6日。  
\item Boyer 1959,第268–269页。  
\item O'Connor & Robertson 2005。  
\item Raman-Sundström, Manya(2015年8–9月)。“紧致性的教育历史”,《美国数学月刊》,122(7):619–635。arXiv:1006.4131。doi:10.4169/amer.math.monthly.122.7.619。JSTOR 10.4169/amer.math.monthly.122.7.619。S2CID 119936587。  
\item Boyer & Merzbach 1991,第561页。  
\item Bolzano,《数学方法》,第2节。  
\item Bolzano,《数学方法》,第3节。  
\item Bolzano,《数学方法》,第4节。  
\item Bolzano,《科学学》,第72节。  
\item Stefan Roski,《博尔扎诺的基础观念》,法兰克福,Klostermann出版社,2017年。  
\item Fisette, Denis(2014)。《奥地利哲学及其机构:关于维也纳大学哲学学会(1888–1938)的评论》。《心智、价值与形而上学》(PDF)。Springler,第1、11页。ISBN 978-3-319-04199-5。OCLC 5680356536。存档(PDF)于2018年7月20日。  
\item Wolfgang Huemer,“胡塞尔对心理主义的批判及其与布伦塔诺学派的关系”,载于:Arkadiusz Chrudzimski 和 Wolfgang Huemer(编),《现象学与分析:中欧哲学论文集》,Walter de Gruyter出版社,2004年,第205页。  
\item Maria van der Schaar,《Kazimierz Twardowski: 哲学语法》,Brill出版社,2015年,第53页;Peter M. Simons,《从博尔扎诺到塔尔斯基的中欧哲学与逻辑:选集》,Springer出版社,2013年,第15页。  
\item frommann-holzboog.de
\end{enumerate}
\subsection{参考文献 }  
\begin{itemize}
\item Boyer, Carl B.(1959),《微积分的历史及其概念发展》,纽约:Dover出版公司,MR 0124178。  
\item Boyer, Carl B.; Merzbach, Uta C.(1991),《数学的历史》,纽约:John Wiley & Sons出版社,ISBN 978-0-471-54397-8。  
\item Ewald, William B.(编)(1996),《从康德到希尔伯特:数学基础的源泉》,2卷,牛津大学出版社。  
\item Künne, Wolfgang(1998),“博尔扎诺,伯纳德”,《劳特利奇哲学百科全书》,第1卷,伦敦:劳特利奇出版社,第823-827页。  
\item O'Connor, John J.; Robertson, Edmund F.(2005),“博尔扎诺”,《MacTutor数学历史档案》。
\end{itemize}
\subsection{进一步阅读} 
\begin{itemize}
\item Edgar Morscher [德](1972),《从博尔扎诺到梅农:逻辑实在论的历史》,见:Rudolf Haller(编),《超越存在与非存在:梅农研究的贡献》,学术出版社,第69-102页。  
\item Kamila Veverková,《伯纳德·博尔扎诺:对其思想及其圈子的全新评估》,翻译:安吉洛·肖恩·富兰克林,莱克星顿书籍,2022年。
\end{itemize} 
\subsection{外部链接} 
\begin{itemize}
\item  Morscher, Edgar. 《伯纳德·博尔扎诺》。见:Zalta, Edward N.(编),《斯坦福哲学百科全书》。  
\item  Šebestík, Jan [捷克语]。《博尔扎诺的逻辑》。见:Zalta, Edward N.(编),《斯坦福哲学百科全书》。  
\item  Sandra Lapointe 在《互联网哲学百科全书》中的《博尔扎诺的数学知识哲学》条目。  
\item 《伯纳德·博尔扎诺的哲学:逻辑与本体论》。  
\item 《伯纳德·博尔扎诺:英语翻译书目》。  
\item 《博尔扎诺哲学著作的注释书目》。  
\item 《博尔扎诺实践哲学的注释书目》(宗教、美学、政治)。  
\item 伯纳德·博尔扎诺在数学家谱项目中的资料。  
\item 关于伯纳德·博尔扎诺的著作或研究在《互联网档案馆》上的资料。  
\item 博尔扎诺文献收藏:数字化博尔扎诺的著作。  
\item 《科学学》第一卷在Google Books中的电子版。  
\item 《科学学》第二卷在Google Books中的电子版。  
\item 《科学学》第三卷至第四卷在Google Books中的电子版。  
\item 《科学学》第一卷在Archive.org中的电子版(第162至243页缺失)。  
\item 《科学学》第二卷在Archive.org中的电子版。  
\item 《科学学》第四卷在Archive.org中的电子版。  
\item 《科学学》第三卷在Gallica中的电子版。  
\item 《科学学》第四卷在Gallica中的电子版。
\end{itemize} 
