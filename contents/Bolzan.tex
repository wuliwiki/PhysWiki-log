% 伯恩哈德·波尔查诺(综述)
% license CCBYSA3
% type Wiki

本文根据 CC-BY-SA 协议转载翻译自维基百科\href{https://en.wikipedia.org/wiki/Bernard_Bolzano}{相关文章}。

\begin{figure}[ht]
\centering
\includegraphics[width=6cm]{./figures/f44771c523b92573.png}
\caption{} \label{fig_Bolzan_1}
\end{figure}
伯纳德·博尔查诺(UK: /bɒlˈtsɑːnoʊ/, US: /boʊltˈsɑː-, boʊlˈzɑː-/;德语:[bɔlˈtsaːno];意大利语:[bolˈtsaːno];原名伯纳尔杜斯·普拉西杜斯·约翰·内波穆克·博尔查诺;1781年10月5日–1848年12月18日)是捷克数学家、逻辑学家、哲学家、神学家和天主教神父,具有意大利血统,以其自由主义观点而著称。

博尔查诺使用德语写作,这是他的母语。[6] 大部分他的工作是在他去世后才获得广泛关注。
\subsection{家庭}  
博尔查诺是两位虔诚天主教徒的儿子。他的父亲,伯纳德·庞培乌斯·博尔查诺,是一位意大利人,曾移居布拉格,在那里娶了来自布拉格讲德语的毛雷尔家族的玛丽亚·凯瑟莉亚·毛雷尔。他们的十二个孩子中只有两个活到成年。
\subsection{职业生涯}  
博尔查诺十岁时进入布拉格的皮亚尔修道士中学,期间从1791年到1796年就读于该校。[7]

博尔查诺于1796年进入布拉格大学,学习数学、哲学和物理学。从1800年起,他开始学习神学,并于1804年成为一名天主教神父。1805年,他被任命为布拉格大学宗教哲学新设立的讲座教授。[5] 他不仅在宗教领域,甚至在哲学领域也是一位受欢迎的讲师,并于1818年被选为哲学系系主任。

博尔查诺因他关于军事主义的社会浪费以及战争不必要性的教义,疏远了许多教职员工和教会领袖。他主张全面改革教育、社会和经济制度,旨在将国家的利益引导向和平,而非国家之间的武装冲突。他的政治信仰,虽然他经常与他人分享,最终被奥地利当局视为过于自由。在1819年12月24日,他因拒绝撤回自己的信仰而被解除教授职务,并被流放到乡村,之后将精力投入到他关于社会、宗教、哲学和数学的著作中。

尽管被禁止在主流期刊上发表文章作为流放的条件,博尔查诺依然继续发展他的思想,并通过个人或在东欧不太知名的期刊上发表他的作品。1842年,他回到布拉格,并于1848年去世。
\subsection{数学工作}  
博尔查诺对数学做出了若干原创性的贡献。他的总体哲学立场是,与当时流行的许多数学观念相反,最好不要将直观的概念,如时间和运动,融入数学之中。[8] 为此,他是最早开始将严谨性引入数学分析的数学家之一,他的三部主要数学著作《Beyträge zu einer begründeteren Darstellung der Mathematik》(1810年)、《Der binomische Lehrsatz》(1816年)和《Rein analytischer Beweis》(1817年)。这些著作展示了“……一种发展分析的新方法”,其最终目标直到五十年后才实现,当时这些著作引起了卡尔·魏尔施特拉斯的关注。[9]

在数学分析的基础方面,博尔扎诺为引入完全严谨的ε–δ极限定义做出了贡献。博尔查诺是第一个认识到实数具有下确界性质的数学家。[10] 像他同时代的许多人一样,他对戈特弗里德·莱布尼茨的无穷小概念持怀疑态度——这一概念曾是微积分的早期假定基础。博尔扎诺的极限概念与现代的相似:极限不是无穷小量之间的关系,而是应当通过依赖变量如何在自变量接近某个确定的数值时接近另一个确定的数值来表述。

博尔查诺还给出了代数基本定理的第一个纯分析证明,这一定理最初是高斯从几何角度证明的。他还给出了中值定理(也称为博尔扎诺定理)的第一个纯分析证明。今天,他最为人们所记得的是博尔查诺–魏尔施特拉斯定理,这一定理由卡尔·魏尔施特拉斯独立发展并在博尔扎诺的证明发表多年后才公布,最初被称为魏尔施特拉斯定理,直到博尔扎诺的早期工作被重新发现后,才改为博尔扎诺–魏尔施特拉斯定理。[11]
\subsection{哲学著作}  
博尔查诺(Bolzano)死后出版的《无限的悖论》(*Paradoxien des Unendlichen*,1851年)受到许多后来的杰出逻辑学家的高度赞赏,包括查尔斯·桑德斯·皮尔士(Charles Sanders Peirce)、乔治·坎托尔(Georg Cantor)和理查德·德德金德(Richard Dedekind)。然而,博尔察诺最为人称道的作品是他的《科学学说》(*Wissenschaftslehre*,1837年),这是一本四卷本的著作,涵盖了现代意义上的科学哲学、逻辑学、认识论和科学教育学。他在这部作品中发展出的逻辑理论被认为是开创性的。其他著作包括四卷本的《宗教学教程》(*Lehrbuch der Religionswissenschaft*)和形而上学著作《永生论》(*Athanasia*),后者为灵魂不朽提供了辩护。博尔察诺还在数学领域做出了有价值的工作,直到奥托·斯托尔茨(Otto Stolz)在1881年重新发现并重新出版了他许多失传的期刊文章,这些工作才得到了广泛的认可。
\subsubsection{《科学学说》(Wissenschaftslehre)}  
在他1837年的《科学学说》中,博尔察诺试图为所有科学提供逻辑基础,建立在诸如部分关系、抽象对象、属性、句型、思想和命题本身、总和与集合、集合体、物质、依附、主观思想、判断和句子发生等抽象概念之上。这些尝试是他在数学哲学领域早期思想的延伸,例如他1810年的《贡献》(*Beiträge*),在其中他强调了逻辑后果之间的客观关系与我们主观上对这些关系的认识之间的区别。对博尔察诺而言,仅仅确认自然或数学真理是不够的,科学(无论是纯粹科学还是应用科学)的适当职能是基于可能对我们直觉显而易见或不显而易见的基本真理来寻求合理性。
\begin{itemize}
\item 
\end{itemize}