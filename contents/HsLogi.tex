% 命题与推理(高中)
% keys 逻辑|高中|命题|推理
% license Xiao
% type Tutor


\begin{issues}
\issueDraft
\end{issues}

\pentry{集合\nref{nod_HsSet}}{nod_fc7f}

\textbf{逻辑}是研究如何进行正确推理和论证的学科。逻辑能帮助你对事物进行理解和判断。在数学中,逻辑用于证明定理,确保论证的严谨性和准确性。\textbf{数理逻辑}是逻辑学的一个分支,它应用数学的方法研究逻辑。

在高中阶段接触和学习一部分逻辑内容是非常必要的。一方面,尽管高中数学教材中在逐渐减少逻辑的内容体量,但仍无法完全删除逻辑部分,足见其基础性。同时,作为数学根基的内容,学习逻辑对于理解高中的知识内容(不仅是数学学科,对于理科内容甚至是文科内容)具有相当的助益。另外,对于大部分本科生,这一部分内容在往往也会作为学生已知的部分略讲或跳过。

请注意,本文会涉及到大量的新名词,你或许曾经在生活或学习中使用过它,但明晰它的内容,会让你未来再见到它时,有耳目一新的感觉。建议在学习的过程中,对比进行理解和记忆。

\subsection{命题}

每个领域都有自己的研究对象,就像研究集合时,研究的对象是集合和元素。逻辑的研究对象是命题。

一条可供判断的陈述句被称为\textbf{命题}(proposition)。命题使用\textbf{真值}(truth value)来表示判断的结果。真值用\textbf{布尔值}表示。布尔值有两个,分别为“\textbf{真}”和“\textbf{假}”,分别记作:“${\rm True}$”、“$1$”、“$T$”和“${\rm False}$”、“$0$”、“$F$”。

根据命题的判断结果区分,若被判断为真,则称为\textbf{真命题},判断为假,则称为\textbf{假命题}。一个命题要么为真,要么为假。如果一条陈述句既真又假,或者无法判断真假,则不能称为命题。这种情况出现时,一般是陈述句中的某些概念的定义不清晰。例如:“这个班级中长得好看的人占比$30\%$”就不称为命题,因为“好看的人”是一个定义不清晰的概念。

不能被分解成更简单部分的命题,称为\textbf{原子命题}(atomic proposition,或\textbf{基本命题},basic proposition)。例如:“自然数2是一个偶数”、“猫在椅子上”等是原子命题,而“2是偶数而且3不是偶数”、“如果猫在椅子上,那么猫不在窝里”就不是原子命题。

如果一个命题包含变量,则称为\textbf{开放命题}(open proposition)。开放命题的真值取决于变量的取值,例如,“自然数$x$是一个偶数”是一个开放命题,也是一个简单命题。如果$x=2$,则为真命题;如果$x=3$,则为假命题。注意,有时限定条件会成为开放命题的变量,比如“总人口最多的城市是重庆市”则隐含了空间和时间的限定条件,如果把条件限定在“2021年中国”则是真命题,如果把条件限定在“2022年全世界”则是假命题。

\subsection{*与命题相似、相关的词汇}

下面将会介绍一些经常使用的,与命题相似或相关的词汇。他们在数学学习甚至是其他学科的学习中经常出现,但使用时却极易混淆。后面的内容较长,这里先简单概括一下:

\begin{itemize}
\item 概念是从实体到思维的抽象结果,是在脑海中的认识和理解,定义是对概念的语言描述,可以用来判别一个事物是否属于某个概念,或区分不同概念。定义描述概念时,或者明确其属性,或者确定其范围。
\item 公理是不需要证明默认是真命题的命题,悖论是一种既真又假的陈述。这两者对数学领域的变迁举足轻重。
\item 由部分已知信息提出的命题,就是猜想。如果猜想得证,并且影响深远,一般会称为定理。证明定理之前使用的一些有用的命题会称为引理,证明定理之后推出的针对具体情况的命题称为推论。
\end{itemize}

这部分内容与考试内容无直接的关系,如果不想详细了解,也可以直接\aref{跳过}{sub_HsLogi_1}。

\subsubsection{概念、定义}

\textbf{概念}(concept)是通过抽象从一群事物中提取出来的,用于反映这些事物共同特性的思维单位。\textbf{定义}(definition)是为了规范某个概念的范围而用语言进行的简要、完整的陈述。通常,描述概念的定义时,会给出概念的内涵或外延:
\begin{itemize}
\item \textbf{内涵}指概念所有特性和关系,反映所指事物的本质属性或特有属性,描述了概念的“质”、“属性”。
\item \textbf{外延}指所有包括在这个概念中的事物,反映概念的适用范围或具体对象,描述了概念的“量”、“范围”。
\end{itemize}

\begin{example}{用“内涵”和“外延”分别描述平面上的“三角形”}

内涵:三条线段组成的闭合的多边形,称为三角形。

外延:锐角三角形、直角三角形和钝角三角形统称三角形。

\end{example}

一般说来,一个概念的内涵越丰富,包含的特征越多,其外延就越小;反之,一个概念的内涵越简单,包含的特征越少,其外延就越大。例如,“菱形”的内涵比“平行四边形”更“多”,但其外延比“平行四边形”要“小”,也就是说“菱形”的限定条件比“平行四边形”更多,但存在一些“平行四边形”不是“菱形”。也可以借助维恩图来理解内涵与外延的关系,如果用“黄色”表示全集,即所有的个体,用“粉色”来表示各种性质,那么随着性质的增多,也即条件或要求的增多,个体的数量就越来越少,即“黄色”在维恩图上的范围越来越小。
\begin{figure}[ht]
\centering
\includegraphics[width=5cm]{./figures/3c78833bb0aba8e4.png}
\caption{性质越多,个体越少} \label{fig_SufCnd_6}
\end{figure}

\subsubsection{公理、悖论}

在某个理论系统中,被视为无需证明的真命题,称为\textbf{公理}(axiom)。作为系统的基本假设,它们是逻辑推理和演绎系统的基础,用于推导其他真命题。初中时学习过的欧几里得几何的五条公理就是最早的公理系统。

对于一个理论系统而言,如何选取公理是最重要的事情。公理是整个理论大厦的基石,决定了理论的框架和结构,直接影响了整个理论系统的结论是否可信。一组精心选取的公理不仅能确保理论的严密性和一致性,还能提高其应用的广泛性和实用性。数学领域的进步,往往是由人们对某条公理的质疑和挑战开始,并由数学家对公理进行调整或重建而引发的。在选取公理时,一般有几项原则:

\begin{itemize}
\item 公理必须是自洽的,即公理之间不能相互矛盾。
\item 公理应尽可能简洁明了,减少使用概念的数量。
\item 公理应具有普遍性和广泛适用性,涵盖尽可能多的情况。
\item 理想情况下,每一个公理都应该是独立的,即不能通过其他公理推导出来。
\end{itemize}

当看到公理时,不必将其奉为神明,根据前面的原则去审视他们,或许就能够发现一个不一样的体系。

\begin{example}{非欧几何}
人们一直怀疑“欧几里得几何”中的一条公理——“过直线外一点有且只有一条直线与此直线平行”不是独立的。结果研究之后发现,尽管它是独立的,但并不具有普遍性,于是通过修改它创建了“非欧几何”:

\begin{itemize}
\item 双曲几何(hyperbolic geometry)修改后的公理为:“过直线外一点有无数条直线与此直线平行”。
\item 椭圆几何(elliptic geometry)修改后的公理为:“过直线外一点不存在直线与此直线平行”(或者说,所有直线最终都会相交)。
\end{itemize}

这两者都在现实中有广泛的应用:双曲几何在广义相对论中用于描述大质量天体附近的时空曲率,而椭圆几何则在全球定位系统(GPS)和其他卫星导航系统中用来处理地球的形状和测量地面距离。
\end{example}

\textbf{悖论}(paradox)并非命题,而是一条在逻辑推理中出现了表面看上去互相矛盾结论的陈述句,即:既没有办法判定其为真,又没有办法判定其为假的陈述。悖论的出现通常是因为一些隐含假设或定义问题导致的,它蕴涵着深刻的思想,往往揭示了理论体系中的潜在问题和局限性。一般解决悖论的过程,会带来公理、概念上的明晰和变化。

\begin{example}{罗素悖论}
罗素悖论:一个剃胡子的人,他的原则是只给“不给自己剃胡子的人”剃胡子,那么他该不该给自己剃胡子?

思考:
\begin{itemize}
\item 若这个人给自己剃胡子,那么它不属于“不给自己剃胡子的人”,因此他不应该给自己剃胡子。
\item 若这个人不给自己剃胡子,那么它属于“不给自己剃胡子的人”,因此他应该给自己剃胡子。
\end{itemize}
因此,不论他是否给自己剃胡子,都会造成与自己的原则相冲突。这个悖论揭示了集合论中的基本矛盾,从而推动了数学基础的重新构建。
\end{example}

\subsubsection{定理、猜想、引理、推论}

在数学领域,对进一步研究和应用有深远影响的一些真命题,称为\textbf{定理}(theorem)。定理常常用于构建更多的理论和解决复杂的问题。定理通常会有一个正式的名字,包含条件和结论两部分,并且以明确的数学语言表述。理解和掌握定理及其证明过程,对于数学和学习和研究至关重要。

\textbf{猜想}(conjecture)是基于部分已知信息或观察提出的一种尚未得到真假性结论的命题。猜想在数学中具有重要作用,因为它们往往为进一步的研究和理论发展提供了方向。很多定理在确认成为定理之前,都曾是猜想。

注意,这里所说的“尚未得到真假性结论”的命题并不与命题的定义中“可供判断”相冲突。命题的要求是“可供”判断,至于怎么样才能判断命题的真假则不包含在内。比如,“每一个不小于6的偶数都是两个素数的和”,这句话本身是可供判断的,比如只要有人给出一个反例就说明这句话是假的,又或者如果有人证明出来就说明他是真的。但是在当下,这句话我们尚不知道它的真伪。而刚才给出的例子就是著名的“哥德巴赫猜想”。

有时,证明一个定理的过程太长了,如果证明过程中有一些与证明过程中其他内容关系不大,且可以重复使用的结论,一般会提前作为一个部分先行证明,并在定理的证明过程中作为一个独立的结论使用,这个部分就称为\textbf{引理}(lemma)。而一个比较重要的定理,往往会根据不同情况得到一些更具体的、便于使用的命题,这些命题称为\textbf{推论}(corollary)。


\subsection{推理方法}\label{sub_HsLogi_1}

下面将从演绎推理、归纳推理和溯因推理出发去介绍推理方法,上述三种方法是推理的主要形式,其余的推理类型本质上是对他们形式和风格上的混搭。%如何引用:https://factmyth.com/the-different-types-of-reasoning-methods-explained-and-compared/

\subsubsection{演绎推理}

从一般性原则推导出具体结论的推理方法,称作\textbf{演绎推理}(deductive reasoning)。演绎推理的特点便是,只要保证推理模式和前提正确,那么结论必然为真。在数学领域,最常做的便是建立公理和概念,之后进行演绎推理。最有名的演绎推理便是苏格拉底的“三段论”(syllogism)。

\begin{example}{证明“苏格拉底会死”}
\begin{itemize}
\item 大前提:所有人都会死。
\item 小前提:苏格拉底是人。
\item 结论:苏格拉底会死。
\end{itemize}
\end{example}

演绎推理通常非常可靠,一般得出错误结论的场合,都是没有正确审视前提

\subsubsection{归纳推理}

从具体的、个别的观察出发,得出一般性的结论的推理方法,称为\textbf{归纳推理}(inductive reasoning)。在生活、学习、研究中,我们总是能观察到很多现象,有些现象是偶然出现的,也有些现象是循环重复的,通过观察、猜想、总结,就能得到很多的结论。

但是,就像哪怕寒假的时候,你都在家,也不是永远在家。由于,大多数能够被观察到的现象都并不是现象的全体,只是总体的一部分,也就是说,归纳推理是从基于有限的观测得出的,因此即使所有前提为真,结论也不一定为真,得到的结论并不必然,而是具有一定可能性的。就像,如果有人通过寒假观察了两周就归纳出了一个“你永远会在家”的结论,显然是不合理的。

很多人对归纳推理嗤之以鼻,觉得它不可靠,但是在探索未知领域,尤其是没有现成体系或前人经验的领域时,归纳推理非常有用。它可以从现象中从无到有,归纳出新的猜想方向。尽管归纳得出的结论受到观察样本数量的限制,可能存在以偏概全的风险,但相比于完全未知的准确结论,一个模糊但可获得的结论或者准确但未必正确的方向,对实际情况更有帮助。因此,尽管归纳推理并不总是严格,但只要在使用时保持警惕,它依然非常有用。

\subsubsection{溯因推理}

\subsubsection{还原推理}
还原推理(reductive reasoning)又称为\textbf{归谬法}或\textbf{反证法}

\subsubsection{*类比推理}


\subsubsection{对于处理题目的启发}

做数学题的过程,也就是进行演绎推理的过程。保证自己的基础知识牢固,对题目中的信息理解正确都是在保证演绎推理的前提不出错。对条件应用的范围准确、每一步的严丝合缝都是在保证演绎的过程准确。这样就能得到正确的结果。而在处理陌生领域、方向不甚清晰的问题时,先举出一些例子,然后通过归纳来猜想出大概的方向,往往是找到思路的关键。
