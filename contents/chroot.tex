% chroot 笔记

\begin{issues}
\issueDraft
\end{issues}

参考\href{https://www.howtogeek.com/441534/how-to-use-the-chroot-command-on-linux/}{这篇文章}.
以及\href{https://devops.stackexchange.com/questions/2826/difference-between-chroot-and-docker}{docker 和 chroot 的区别}(见 Docker 笔记\upref{Docker}).

\begin{itemize}
\item \verb|sudo chroot 要作为根目录的目录 切换根目录后shell程序所在的路径| 运行以后, chroot 会把第一个 arg 变为根目录, 并在其中开启第二个 arg 的 shell
\item 注意在 Ubuntu 20.04 中使用会出现 bug, 例如退格键变成空格, vim 排版混乱等. 暂时还没空解决.
\end{itemize}

一个便捷使用 chroot 的脚本(修改自\href{https://gist.github.com/xmonader/5d1fc6134f1f65acd0d10f71453adb27}{这里})

\begin{lstlisting}[language=bash]
# 根目录
chr=test_root
# 要拷贝的命令
bins="bash touch ls rm mkdir rmdir pwd cp mv\
          cat vim tree find ln which sh ldd chmod"

# 拷贝可执行文件以及依赖库
copy_chroot () {
	if [ $# != 2 ] ; then
		echo "usage $0 PATH_TO_BINARY TARGET_FOLDER"
		exit 1
	fi

	PATH_TO_BINARY=`which $1`
	TARGET_FOLDER="$2"

	# if we cannot find the the binary we have to abort
	if [ ! -f "$PATH_TO_BINARY" ] ; then
		echo "The file '$PATH_TO_BINARY' was not found. Aborting!"
		exit 1
	fi

	# copy the binary to the target folder
	# create directories if required
	echo "---> copy binary itself"
	cp --parents -v "$PATH_TO_BINARY" "$TARGET_FOLDER"

	# copy the required shared libs to the target folder
	# create directories if required
	echo "---> copy libraries"
	for lib in `ldd "$PATH_TO_BINARY" | cut -d'>' -f2 \
            | awk '{print $1}'` ; do
	if [ -f "$lib" ] ; then
			cp -v --parents "$lib" "$TARGET_FOLDER"
	fi  
	done

	# I'm on a 64bit system at home. the following
    # code will be not required on a 32bit system.
	# however, I've not tested that yet
	# create lib64 - if required and link the content from lib to it
	if [ ! -d "$TARGET_FOLDER/lib64" ] ; then
		mkdir -v "$TARGET_FOLDER/lib64"
	fi
}

mkdir "$chr"

for bin in $bins; do
	echo; echo "====== $bin ======";
	copy_chroot $bin "$chr"
done

# 进入新环境
chroot $chr `which bash`
\end{lstlisting}
