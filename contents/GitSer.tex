% Git 服务器搭建

\begin{issues}
\issueDraft
\end{issues}

\pentry{Git 简介\upref{Git}}

\begin{itemize}
\item 首先创建一个 bare repo (bare repo 的拓展名习惯用 .git)
\verb|git clone --bare .../project .../project.git|

\item 最简单的方法(但是任何操作都需要密码)
将 \verb|project.git| 上传到服务器的任何一个目录 \verb|<dir>|
现在, 随时随地都可以 clone 了.
\verb`git clone user@111.222.333.444:<dir>/project.git`
这个命令本质上是在使用 ssh 通信, 如果 ssh 能用这个就能用.
前提是 user 在 \verb|project.git| 文件夹有读写权限.

\item 但是像 ssh 一样, 每次使用都要输入 ssh 密码. 如果不想输密码, 就使用公钥密钥登陆, 具体见 ssh 笔记. 如果成功, 连 ssh 的时候不需要密码, 那么 git 也不需要密码了.

\item 如果是一个本地已有的 repo 想要更换服务器, 那么同样在服务器创建 bare repo, 然后在本地 repo 的根目录 \verb`git remote add 名称 user@111.222.333.444:<dir>/project.git` 即可, push 的时候可以 `git push 名称 master` 即可. 也可以把 `名称` 设置为 push 的默认对象(具体见 git 文档).

\item 如果要指定端口, 使用 \verb`:` 和 \verb`ssh://` 即可(如果不用 \verb`ssh://` 可能会提示密码错误), 如 \verb`git clone ssh://root@localhost:6500/root/github/PhysWikiScan`
\end{itemize}
