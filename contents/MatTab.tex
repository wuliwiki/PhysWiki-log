% Matlab 的 Table 类型

\begin{issues}
\issueDraft
\end{issues}

\begin{itemize}
\item 参考 \href{https://www.mathworks.com/help/matlab/tables.html}{table 文档}.
\item \verb|t = readtable('xxx.csv')| 可以读取包含文字和数据的表格, 输出一个 \verb|table| 类型. 该类型类似于 \verb|cell| 矩阵, 但每一列都有名称, 叫做一个 \verb|variable|. 每一列的元素有相同的类型.
\item \verb|writetable('xxx.csv', t)| 把数据写入表格文件.
\item \verb|t(i,j)| 获取矩阵元, 但得到的类型仍是 \verb|table| 矩阵切割等操作和普通矩阵一样, 也得到 \verb|table| 类型.
\item 除了指标外, 也可以用列标题来指定列: \verb|t(i, '列标题')| 或者 \verb|t{i, '列标题'}|
\item \verb|t{i,j}| 可以获取具体类型的矩阵元. 如果单元格是文字, 那该元素就是 \verb|cell|, 如果是数值类型, 就是 \verb|double|.
\item \verb|data2array(t(1:2,3:4))|  可以把表格的一部分(必须所有元素类型相同)转换为矩阵. 如果都是数值, 则变为 \verb|double| 矩阵, 都是 \verb|cell| 则变为 \verb|cell| 矩阵.
\end{itemize}
