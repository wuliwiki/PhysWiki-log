% 南京理工大学 2007 年 研究生入学考试试题 普通物理(B)
% license Usr
% type Note

\textbf{声明}:“该内容来源于网络公开资料,不保证真实性,如有侵权请联系管理员”

\subsection{一、填空题 I(30 分,每空 2 分)}
1. 一质点运动方程为$\vec{r}=4t^2i+(2t^3+3)\vec{j}$ ,则质点运动的轨道方程为___________,质点从 $t=0$ 到 $t=1s$ 的位移为________,质点在 $t=1s$ 时速度为__________,$t=1s$ 时加速度为__________。

2. 一轻质弹簧振子,弹簧的倔强系数 $k=25(N/m)$,初始动能为 $0.02J$,初始势能为 $0.06J$,则其振幅 $A=$_______,当位移 $x=$________时,动能与势能相等。

3. 一频率为 $500Hz$ 的平面谐波,其波速 $u=350m/s$,则其波长为_________$m$;在同一波线上,周相差为 $\pi/3$ 的两点的间距为___________$m$。

4. 若入射波方程为$y_1=A\cos(\omega t+\frac{2\pi x}{\lambda})$ ,在 $x=0$ 处反射,若反射端为固定端,则反射波方程为 $y_2=$_____________(假设振幅不变),合成波方程为____________,波节点的位置为_______________。

5. $1mol$ 双原子刚性分子的理想气体,在一等压过程中对外作功 $100J$,则该等压
过程中理想气体内能变化为___________,吸收热量为______________。

6. 均匀带电的半径为 $R$ 的金属球,带电 $Q$,在距球心为 $a(a<R)$的一点 P 处
的电场强度大小 $E$=_________,电势大小 $U=$___________。
\subsection{二、填空题 II(28 分,每空 2 分)}
1. 波长为 $500nm$ 的单色光垂直照射一缝宽为 $0.25mm$ 的单缝,衍射图像中,中
央明纹极大两边第四暗纹极小距离为 $4mm$,则焦距为___________,其中央明
纹极大宽度为___________。

2. 如图,电流在 $O$ 点的磁感应强度的大小为__________,其方向为_________。
\begin{figure}[ht]
\centering
\includegraphics[width=6cm]{./figures/5d271a49aae9f94c.png}
\caption{} \label{fig_NJUB07_1}
\end{figure}
3. 迈克尔逊干涉仪中,当 $M2$ 移动距离$\Delta d=0.322mm$ 时,测得某单色光的干涉条纹移动过$\Delta N=1024$ 条,则该单色光的波长为__________;若在 $M2$ 镜前插入一透明片,观测到 150 条条纹移过,若薄片的折射率为 1.632,所用单色光的波长为 $500nm$,则薄片的厚度为_______________。

4. 一个 50 匝的半径 $R=5.0cm$ 的通电线圈处于 $B=1.5(T)$的均匀磁场中,线圈中电流 $I=0.2A$,则该线圈的磁矩大小为____________;当线圈的磁矩与外磁场方向的夹角从 0 转到 $\pi$ 时,外磁场对线圈所作的功为______________。

5. 涡旋电场(或感生电场)的场方程是___________,其物理意义是_______。

6. 一立方体的静质量为 $m_0$,边长为 $l_0$,若该物体沿其一边的方向作以速度为 $u$的高速运动 , 则静止的观测者测得其体积为 $V=$____________ ,密度$\rho=$___________。

7. 一粒子静质量为 $m0$,其动能是静能的 $n$ 倍,则该粒子的运动质量为________
__________,运动速度大小为____________________。
\subsection{三、(12 分)}
如图,一长为 $l$,质量为 $M $的匀质细杆自由悬挂于通过其上端的光滑水平轴上,今有一质量为 $m$ 的子弹以水平速度 $v_0$ 射向杆的中心,并停留在杆内,此后杆的最大摆角恰好为 90°,问 $v_0$ 的大小是多少?
\begin{figure}[ht]
\centering
\includegraphics[width=6cm]{./figures/de868b2c9a8a437a.png}
\caption{} \label{fig_NJUB07_2}
\end{figure}
\subsection{四、(12 分)}
试画出理想气体的卡诺循环过程,并证明该循环的循环效率为$\eta=1-T_2/T_1 (T_1,T_2) $分别为高温热源和低温热源的温度)。
\subsection{五、(10 分)}
半径为 $R$ 的均匀带电的介质球,介电常数为$\xi$ ,球外为真空,求距球心为 $r$ 处的电势。
\subsection{六、(10 分)}
在一平板上放有质量为 $m=1.0kg$ 的物体,平板在竖直方向上作上下谐振动,周期 $T=0.5s$,振幅 $A=0.02m$。试求:(1)在位移最大时物体对平板的正压力;(2)平板应以多大的振幅振动时,才能使物体开始脱离平板?
\subsection{七、(12 分)}
匀磁场 $B$ 被局限在半径为 $R$ 的圆柱体内,磁场随时间的变化率,有一长为 $R$ 的导体 棒 $AC$ 放在磁场中,如图, 求:(1)导体棒中的电动势的大小与方向;(2)在 $A$ 点电子所受电场力的大小和方向。
\begin{figure}[ht]
\centering
\includegraphics[width=6cm]{./figures/810b93ad4a7f22df.png}
\caption{} \label{fig_NJUB07_3}
\end{figure}
\subsection{八、(12 分)}
空气中有一无限长直导线,通以 $I=4.0A$ 的电流,与之共面时一个通以电流 $I1=1.0A$ 的 矩形回路 MNPQM , $a=0.2m$ , $b=0.4m$ , $d=0.2m$ 。 求 :(1)MN 段所受安培力的大小和方向;(2)闭合电流所受的安培力合力的大小和方向;(3)矩形回路所受的磁力矩。
\begin{figure}[ht]
\centering
\includegraphics[width=6cm]{./figures/489fd1e97378d7c2.png}
\caption{} \label{fig_NJUB07_4}
\end{figure}
\subsection{九、(12 分)}
一光栅,在 $2.4cm$ 宽度上有 6000 条刻痕,其刻痕宽度是通光缝宽的二倍,现以波长 为 $632.8nm$ 的单色平行光垂直照射,求:(1)光栅常数;(2)最多能观察到哪些谱线,共有多少条?
\subsection{十、(12 分)}
一个电子用静电场加速后,其动能为 $0.25MeV$,求:(1)运动电子的质量;(2)电子的运动速度;(3)电子的德布罗意物质波波长。