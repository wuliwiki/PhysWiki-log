% 旋度的逆运算
% keys 散度|旋度
% license Xiao
% type Tutor

\begin{issues}
\issueTODO
\end{issues}

\pentry{矢量算符运算法则\upref{VopEq}}

\begin{theorem}{}\label{the_HlmPr2_1}
\footnote{参考 \cite{GriffE} 233 页。}令 $\bvec F$ 为无散场, 即
\begin{equation}
\div \bvec F = 0~.
\end{equation}
则 $\bvec F(\bvec r)$ 总能表示为另一个矢量场 $\bvec G(\bvec r)$ 的旋度, 即
\begin{equation}\label{eq_HlmPr2_4}
\bvec F = \curl \bvec G~.
\end{equation}
且 $\bvec G(\bvec r)$ 可以通过以下公式计算:
\begin{equation}\label{eq_HlmPr2_1}
\bvec G(\bvec r) = \frac{1}{4\pi}\int \bvec F(\bvec r') \cross \frac{\bvec R}{R^3} \dd{V'}~.
\end{equation}
其中 $\bvec G$ 通常被称为\textbf{矢势(vector potential)}, $\bvec r, \bvec r'$ 分别是坐标原点指向三维直角坐标 $(x, y, z)$ 和 $(x', y', z')$ 的位置矢量, $\bvec R = \bvec r - \bvec r'$, $R = \abs{\bvec R}$, 体积分 $\int\dd{V'} = \int\dd{x'}\dd{y'}\dd{z'}$ 的区域是空间中 $\bvec F$ 不为零的区域, $\cross$ 表示矢量叉乘\upref{Cross}。
\end{theorem}

\begin{corollary}{}
$\bvec F$ 是无散场的充分必要条件是它可以表示为另一个矢量场的旋度 $\curl \bvec G$。
\end{corollary}
证明: \autoref{the_HlmPr2_1} 提供了充分性。 必要性: 我们知道 $\div (\curl G) \equiv 0$ (链接未完成)。 证毕。

\autoref{the_HlmPr2_1} 在电动力学中有两个重要的应用: 一个是证明比奥萨法尔定律满足安培环路定律\upref{AmpLaw}, 另一个是证明磁矢势\upref{BvecA} $\bvec A$ 必定存在(因为磁场是无散场\upref{MagGau})。

\begin{corollary}{}\label{cor_HlmPr2_1}
在\autoref{the_HlmPr2_1} 中, 给 $\bvec G(\bvec r)$ 加上任意一个无旋场 $\bvec H(\bvec r)$ (满足 $\curl \bvec H = \bvec 0$), 也能使\autoref{eq_HlmPr2_4} 成立。 通过这种方法可以得到\autoref{eq_HlmPr2_4} 中所有可能的 $\bvec G(\bvec r)$。
\end{corollary}
\textbf{证明}: 第一句话证明显然。 第二句话: 若 $\bvec G$ 和 $\bvec G'$ 同时满足\autoref{eq_HlmPr2_4}, 那么相减得到 $\curl (\bvec G' - \bvec G) = \bvec 0$。 所以二者之差只能是无旋场。 证毕。

注意 $\bvec H$ 也可以表示为任意标量函数 $V(\bvec r)$ 的梯度 $\grad V$(引用未完成)。

我们可以认为\autoref{eq_HlmPr2_1} 是旋度运算的逆运算, 这可以类比不定积分\upref{Int}是求导的逆运算。 而无旋场 $\bvec H(\bvec r)$ 可以类比不定积分中的任意常数。 散度也有类似的逆运算\upref{DivInv}。

\begin{theorem}{}\label{the_HlmPr2_2}
令 $\bvec F(\bvec r)$ 为无散场, 则\autoref{eq_HlmPr2_1} 得到的 $\bvec G(\bvec r)$ 仍然是一个无散场。
\end{theorem}
显然, 给 $\bvec G(\bvec r)$ 加上一个任意的无散场 $\bvec H(\bvec r)$ 后仍然是一个无散场。 若需要满足\autoref{eq_HlmPr2_4}, 则 $\bvec H(\bvec r)$ 必须也是无旋的, 即 $\bvec H(\bvec r)$ 是调和场。

我们已经知道一个矢量场无论求几次旋度, 都一直是无散场。 而这个定理告诉我们任意无散场无论求几次 “逆旋度” 也都可以是无散场。

\subsection{证明定理 1}
我们只需要证明\autoref{eq_HlmPr2_1} 右边第一项求旋度等于 $\bvec F$, 使用\autoref{eq_VopEq_6}~\upref{VopEq}(展开的四项中对 $\bvec F$ 微分的两项为零, 因为 $\bvec F$ 是 $\bvec r'$ 的函数):
\begin{equation}\label{eq_HlmPr2_3}
\begin{aligned}
\curl \bvec G &= \frac{1}{4\pi}\int \curl \qty(\bvec F \cross \frac{\bvec R}{R^3}) \dd{V'}\\
&= \frac{1}{4\pi}\int\bvec F \qty(\div \frac{\bvec R}{R^3}) \dd{V'} -  \frac{1}{4\pi}\int(\bvec F \vdot \grad) \frac{\bvec R}{R^3} \dd{V'}~,
\end{aligned}
\end{equation}
其中第一个等号是因为 “对一个变量积分” 再 “对另一个变量求导” 这两个操作可以交换。 % 链接未完成

先来证明上式第二项为零: 被积函数中的 $x$ 分量为($\grad'$ 意味着对 $x', y', z'$ 求偏导, 另外注意 $\bvec F$ 是 $\bvec r'$ 的函数)
\begin{equation}\label{eq_HlmPr2_2}
\begin{aligned}
(\bvec F \vdot \grad) \frac{x' - x}{\abs{x'-x}^3} &= \bvec F \vdot \qty(\grad \frac{x' - x}{\abs{x'-x}^3}) = -\bvec F \vdot \qty(\grad' \frac{x' - x}{\abs{x'-x}^3})\\
&= - \grad' \vdot \qty(\bvec F \frac{x' - x}{\abs{x'-x}^3}) + (\grad' \vdot \bvec F) \frac{x' - x}{\abs{x'-x}^3}~,
\end{aligned}
\end{equation}
这里使用了\autoref{eq_VopEq_1}~\upref{VopEq}。 由于 $\bvec F(\bvec r')$ 是无散场, 最后一项为零。 $y, z$ 分量同理。 对上式做体积分得(使用散度定理\autoref{eq_Divgnc_13}~\upref{Divgnc})
\begin{equation}
- \int \grad' \vdot \qty(\bvec F \frac{x' - x}{\abs{x'-x}^3})\dd{V'} = -\oint \bvec F \frac{x' - x}{\abs{x'-x}^3} \dd{\bvec s}~,
\end{equation}
积分曲面是体积分区域的边界曲面。 由于我们假设 $\bvec F$ 只有在体积分内部(不包括边界)不为零, 所以该式为零。

再来看\autoref{eq_HlmPr2_3} 右边第一项, 有(\autoref{eq_CEfDiv_2}~\upref{CEfDiv})
\begin{equation}
\div \frac{\bvec R}{R^3} = 4\pi \delta^3(\bvec R)~.
\end{equation}
代入得
\begin{equation}
\curl \bvec G = \int \bvec F(\bvec r') \delta^3(\bvec r - \bvec r') \dd{V'} = \bvec F(\bvec r)~,
\end{equation}
证毕。

\subsection{证明定理 2}
使用\autoref{eq_VopEq_4}~\upref{VopEq}, 有
\begin{equation}
\begin{aligned}
&\div \bvec F = \frac{1}{4\pi} \int \div \qty(\bvec F \cross \frac{\bvec R}{R^3}) \dd{V'}\\
&= \frac{1}{4\pi} \int \frac{\bvec R}{R^3} \vdot (\curl \bvec F)\dd{V'} - \frac{1}{4\pi} \int \bvec F \vdot \qty(\curl \frac{\bvec R}{R^3})\dd{V'}~.
\end{aligned}
\end{equation}
注意 $\bvec F$ 是 $\bvec r'$ 的函数而不是 $\bvec r$ 的函数, 所以第一项中 $\curl \bvec F = \bvec 0$。 另外由于 $\curl(\bvec R/R^3) = \bvec 0$ (引用未完成) , 上式恒为零。 证毕。
