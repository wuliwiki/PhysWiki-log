% 布劳威尔不动点定理(综述)
% license CCBYSA3
% type Wiki

本文根据 CC-BY-SA 协议转载翻译自维基百科\href{https://en.wikipedia.org/wiki/Brouwer_fixed-point_theorem}{相关文章}。

布劳威尔不动点定理是拓扑学中的一个不动点定理,以 L. E. J.(Bertus)布劳威尔命名。该定理指出:**对于任意一个将非空紧致凸集映射到其自身的连续函数 $f$,总存在一点 $x_0$,使得 $f(x_0) = x_0$**。
最简单的形式是将一个闭区间 $I$(实数集中的)映射到自身的连续函数,或将一个闭圆盘 $D$ 映射到自身的情形。比这更一般的形式是:将欧几里得空间中的非空紧致凸子集 $K$ 映射到自身的连续函数。

在众多不动点定理中,\(^\text{[1]}\)布劳威尔不动点定理尤为著名,部分原因是它在数学的众多领域中都有广泛应用。在其原始领域中,这一结果是刻画欧几里得空间拓扑性质的关键定理之一,与约旦曲线定理、毛球定理、维数不变性定理以及博苏克–乌拉姆定理并列为拓扑学的基本定理之一。\(^\text{[2]}\)它还被用于证明关于微分方程的重要结论,因此通常出现在微分几何的入门课程中。此外,它也出现在一些出人意料的领域,比如博弈论。在经济学中,布劳威尔不动点定理及其推广——卡库塔尼不动点定理,是 20 世纪 50 年代经济学诺贝尔奖得主肯尼斯·阿罗和热拉尔·德布鲁提出的一般均衡存在性证明的核心工具。

这一定理最初是在研究微分方程的背景下被法国数学家们提出的,代表人物有昂利·庞加莱和夏尔·埃米尔·皮卡尔。要证明诸如庞加莱–本迪克松定理这样的结果,需要运用拓扑方法。这一领域在 19 世纪末开启,催生了定理的若干版本。对于 $n$ 维闭球中可微映射的情形,最早由雅克·阿达马于 1910 年给出证明;而对连续映射的一般情形,则由布劳威尔于 1911 年完成证明。\(^\text{[5]}\)
\subsection{陈述}
布劳威尔不动点定理有多种表述方式,取决于其应用的上下文以及推广的程度。最简单的形式如下:

\textbf{在平面上}

每一个从闭圆盘映射到其自身的连续函数至少有一个不动点。\(^\text{[6]}\)

这个结论可以推广到任意有限维度:

\textbf{在欧几里得空间中}

每一个从欧几里得空间中闭球映射到其自身的连续函数都有一个不动点。\(^\text{[7]}\)

稍微更一般一点的版本是:

\textbf{凸紧致集}

每一个从欧几里得空间中非空凸紧致子集 $K$ 映射到自身的连续函数都有一个不动点。\(^\text{[9]}\)

一个更加广义的形式通常以另一个名称广为人知:

\textbf{舍乌德尔不动点定理}

每一个从巴拿赫空间中非空凸紧致子集 $K$ 映射到自身的连续函数都有一个不动点。\(^\text{[10]}\)
\subsection{先决条件的重要性}
该定理仅对自同态函数(即定义域与值域相同的函数)成立,并且要求集合是非空、紧致(即有界并闭合)且凸的(或者与凸集同胚)。下面的示例将说明这些先决条件为何是必要的。
\subsubsection{函数 $f$ 作为自同态映射的情况}
考虑函数
$$
f(x) = x + 1~
$$
其定义域为 $[-1, 1]$,而值域为 $[0, 2]$。因此,$f$ 并不是一个**自同态映射**(即定义域和值域不同的函数)。
\subsubsection{有界性}
考虑函数
$$
f(x) = x + 1~
$$
这是一个从实数集 $\mathbb{R}$ 映射到自身的连续函数。由于它将每个点都向右平移,因此不可能存在不动点。空间 $\mathbb{R}$ 是凸的和闭合的,但它不是有界的。
\subsubsection{闭合性}
考虑函数
$$
f(x) = \frac{x + 1}{2}~
$$
这是一个从开区间 $(-1, 1)$ 映射到自身的连续函数。由于点 $x = 1$ 不属于该开区间,因此在定义域中不存在满足 $f(x) = x$ 的点,即该函数在开区间上没有不动点。集合 $(-1, 1)$ 是凸的、是有界的,但它不是闭合的。另一方面,函数 $f$ 在闭区间 $[-1, 1]$ 上确实有不动点,即$x = 1$闭区间 $[-1, 1]$ 是紧致的(即同时是闭的和有界的),而开区间 $(-1, 1)$ 则不是。
\subsubsection{凸性}
对于布劳威尔不动点定理而言,凸性并不是绝对必要的条件。因为该定理中涉及的属性(连续性、是否为不动点)在同胚变换下是不变的,所以布劳威尔不动点定理等价于那些将定义域要求为闭单位球 $D^n$ 的形式。出于同样的原因,定理也适用于任何与闭单位球同胚的集合(因此这些集合也是闭的、有界的、连通的、无洞的,等等)。

下面这个例子说明了:布劳威尔不动点定理不适用于存在“空洞”的定义域。考虑函数$f(x) = -x$它是一个从单位圆映射到自身的连续函数。由于对于单位圆上的任何点 $x$,都有 $-x \ne x$,因此 $f$ 没有不动点。类似的例子也适用于 $n$ 维球面(或者任何不包含原点的对称区域)。单位圆是闭合且有界的,但它存在一个空洞(因此不是凸的)。相反地,函数 $f$ 在单位圆盘内是有不动点的,因为它将原点映射为自己。

对布劳威尔不动点定理在“无洞”区域上的形式推广,可以通过 Lefschetz 不动点定理来导出\(^\text{[11]}\)。
\subsubsection{注}
该定理中的连续函数不要求是双射或满射。
\subsection{插图说明}
布劳威尔不动点定理有一些“现实世界”的形象例证。以下是几个例子:
\begin{enumerate}
\item 皱纸叠放\\
   拿两张大小相同的坐标方格纸,一张平铺在桌面上,另一张则不撕裂地揉皱,然后随意放置在那张平铺的纸上,只要揉皱纸的边界没有超出平铺纸的边缘,必然存在揉皱纸上的某个点正好位于平铺纸上相同坐标位置的正上方。
   这是布劳威尔不动点定理在二维情形($n = 2$)的一个推论,应用在这样一个连续映射上:它将揉皱纸上每个点的坐标,映射为其正下方平铺纸上对应点的坐标。
\item 地图定位\\
   拿一张国家的普通地图,将它摊开放置在该国家的某个位置上,总会存在一个“你在这里”的点,地图上的这个点刚好对应实际国家中的同一个点。
\item 鸡尾酒搅拌\\
   在三维情形中,布劳威尔不动点定理的一个结果是:无论你如何搅拌一杯美味的鸡尾酒(或奶昔),当液体静止下来时,总有一点液体回到了搅拌前在杯子里的同一位置。
   前提是:每个点最终的位置是其初始位置的连续函数;搅拌后液体仍然处于原始体积之内;杯子的形状以及液体表面维持一个凸形状的空间。如果点了“摇而非搅”的鸡尾酒,这种搅动过程打破了“凸性”条件(因为摇晃过程中的液体状态处于非凸的惯性运动封闭体积中),那么定理将不再适用,此时液体的每个点都可能被移动到一个不同于原位置的地方。
\end{enumerate}
\subsection{直观解释}
\subsubsection{布劳威尔的解释}
据说,这一定理起源于布劳威尔对一杯精品咖啡的观察。\(^\text{[12]}\)当他搅拌咖啡以溶解一块方糖时,他注意到:似乎总有一个点是静止不动的。他据此得出结论:在任何时刻,液面上总存在一个没有运动的点。\(^\text{[13]}\)不过,这个不动点未必是视觉上看起来静止的那个点,因为涡流的中心本身也会有轻微的移动。这个结果并不直观,因为原先的不动点在另一个不动点出现时可能会变得可动。

据说布劳威尔还补充道:“我可以用另一种方式来表述这个精彩的结果:我拿一张水平的纸片,再拿一张一模一样的纸,把它揉皱,然后摊开,放在第一张纸上。那么,揉皱纸上必定存在一个点,它正好和下面那张纸上的对应点重合。”\(^\text{[13]}\)布劳威尔所谓的“摊开”,是像用熨斗那样将其压平,但不消除折痕和皱纹。不同于咖啡杯的例子,这个揉皱纸的例子还展示了可能存在不止一个不动点。这也使得布劳威尔的不动点定理与其他定理(如斯特凡·巴拿赫的不动点定理)区分开来,后者往往保证不动点的唯一性。
\subsubsection{一维情况}
\begin{figure}[ht]
\centering
\includegraphics[width=6cm]{./figures/403d2b1a5ea49cb6.png}
\caption{} \label{fig_BLWRbd_1}
\end{figure}
在一维中,这一结果是直观且容易证明的。设有一个连续函数 $f$,定义在闭区间 $[a, b]$ 上,且函数值也落在该区间内。所谓函数有一个不动点,就是说它的图像(图中深绿色曲线)与定义在同一区间 $[a, b]$ 上的恒等函数 $x \mapsto x$(图中浅绿色对角线)相交。

直观地看,任何从正方形左边到右边的连续曲线,必然会与那条绿色对角线相交。为了证明这一点,考虑函数 $g(x) = f(x) - x$。那么在端点 $a$ 处有 $g(a) \ge 0$,在 $b$ 处有 $g(b) \le 0$。根据**介值定理**,函数 $g$ 在区间 $[a, b]$ 上必有一个零点;这个零点就是一个不动点。

据说布劳威尔曾这样表达这个观点:“我们不去研究一个表面,而是用一根绳子来说明这个定理。我们先把绳子拉直,再重新折叠它,然后将折叠后的绳子压平。一定存在一个点,其在压平后的位置与其在原始未折叠绳子中的位置保持一致。”\(^\text{[13]}\)
