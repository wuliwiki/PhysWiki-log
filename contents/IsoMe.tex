% 等距变换
% license Usr
% type Tutor

\pentry{伴随映射\nref{nod_Sample}}{nod_592f}\pentry{正交空间与辛空间\nref{nod_OrSp}}{nod_2cce}


\begin{definition}{}
$(V,f),(V',f')$是两个正交空间或辛空间。若线性映射$\sigma:(V,f)\rightarrow (V',f')$是双射且保内积不变,即\begin{equation}
f(\bvec x,\bvec y)=f'(\sigma \bvec x,\sigma \bvec y)~,
\end{equation}
则称$\sigma $是\textbf{等距映射}(isometry)。若$\sigma:(V,f)\rightarrow (V,f)$,则称之为等距变换。
\end{definition}
由于等距变换是线性映射,因此保加法和数乘运算,是全体向量的自同构映射。由定义式可知,等距变换复合依然是等距变换。因此全体等距变换的集合可记作$\opn{Aut}(V,f)$,表明其自同构成群和保内积的性质。从正交变换的\autoref{def_ortho_1}~\upref{ortho}可知,正交变换是度量矩阵为$E$的等距变换,因此我们可以拓展正交变换的定义为\textbf{非退化}正交空间的等距变换,并称\textbf{非退化}辛空间的等距变换为\textbf{辛变换}。

等距变换的保内积性质意味着其必然受度量矩阵限制。设$\sigma,f$的矩阵表示分别为$A$和$G$,任意向量$\bvec x,\bvec y\in V$,则有$\bvec x^{T}G\bvec y=(A\bvec x)^TG(A\bvec y)$,也就是说:
\begin{equation}
A^TGA=G~.
\end{equation}

若$(V,f)$是非退化的正交空间和辛空间,则其上的线性变换都可以诱导出伴随变换,所以若设等距变换的伴随变换为$\sigma':V\rightarrow V$,则有$f(\sigma(\bvec x),\bvec y)=f(\bvec x,\sigma'(\bvec y))$,因此其伴随变换的形式必然也受度量矩阵约束。
\begin{theorem}{}
设$(V,f)$为n维\textbf{非退化正交(辛)空间},$\sigma$是双射线性变换,$\sigma$是其伴随变换。则$\sigma$是等距变换当且仅当$\sigma'\sigma=1$
\end{theorem}
\textbf{证明:}

设$\bvec x,\bvec y$是任意两个向量,由定义得:$f(\sigma(\bvec x),\sigma(\bvec y))=f(\bvec x,\sigma'\sigma(\bvec y))=f(\bvec x,\bvec y)$。

因此,$f(\bvec x,\sigma'\sigma(\bvec y)-\bvec y)=0$对任意$\bvec x$成立。由于非退化线性空间的根只有零向量,所以$\sigma'\sigma(\bvec y)=\bvec y$,得证。

欧几里得空间上的正交变换有一系列结论,比如\autoref{lem_ortho_2}~\upref{ortho}可以拓展到非退化正交(辛)空间:
\begin{theorem}{}
$(V,f)$为\textbf{非退化正交(辛)空间},若$W$是等距变换$\sigma$的不变子空间,则其正交补$W^{\bot}$也为该变换的不变子空间。
\end{theorem}
\textbf{证明:}

为证明方便,接下来用$(,)$表广义内积运算,即$(\bvec  x,\bvec y)\equiv f( \bvec x,\bvec  y) $。

设任意$\bvec x\in W,\bvec y\in W$,则由定义得$(\sigma (\bvec x),\sigma(\bvec y))=(\bvec x,\bvec y)=0$。由于$W$是$\sigma$的不变子空间,则$\sigma (\bvec x)\in W$。因此$\sigma(\bvec y)\in W^{\bot}$,得证。


非退化线性空间是可以有退化的子空间的。比如$f(\bvec e_1,\bvec e_1)=-1,f(\bvec e_2,\bvec e_2)=1$则$\bvec e_1+\bvec e_2$对应的一维子空间便是退化的。根据\autoref{the_OrSp_1}~\upref{OrSp}和\autoref{the_OrSp_2}~\upref{OrSp}可知,若$W$是非退化的,则$V$内总存在一组基使得$V=W\oplus W^{\bot}$。则在该基下,等距变换是比较简洁的块对角形式。