% 物质导数(实质导数)
% keys 物质导数|当地导数|迁移导数

\begin{issues}
\issueDraft
\end{issues}

\subsection{物质导数定义}

在流体力学中,物质导数实际上是\textbf{拉格朗日法下某变量对时间的导数},但它可以表示成\textbf{对欧拉变量的全导数}形式,因其特殊性,常用大写的微分符号来表示,称为物质导数(Substantial  Derivative),或实质导数,又或者随体导数。设 $\bvec \Phi$ 是流体的某种性质,物质导数的一般形式为:
\begin{equation}
\frac{\mathrm{D}\bvec \Phi }{\mathrm{D} t}=\pdv{\bvec \Phi}{t}+(\bvec V \vdot \Nabla)\bvec \Phi~.
\end{equation}
公式中,等号左边的项为物质导数,等号右边的第一项称作当地导数,第二项称作对流导数或者牵连导数,这两项是欧拉法下的描述。

\subsection{推导过程}
