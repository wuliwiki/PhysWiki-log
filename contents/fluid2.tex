% 物质导数(实质导数)
% keys 物质导数|当地导数|迁移导数
% license Usr
% type Tutor

\begin{issues}
\issueDraft
\end{issues}

\subsection{物质导数定义}

在流体力学中,物质导数实际上是\textbf{拉格朗日法下某变量对时间的导数},但它可以表示成\textbf{对欧拉变量的全导数}形式,因其特殊性,常用大写的微分符号来表示,称为物质导数(Substantial  Derivative),或实质导数,又或者随体导数。设 $\bvec \Phi$ 是流体的某种性质,物质导数的一般形式为:
\begin{equation}
\frac{\mathrm{D}\bvec \Phi }{\mathrm{D} t}=\pdv{\bvec \Phi}{t}+(\bvec V \vdot \Nabla)\bvec \Phi~.
\end{equation}
公式中,等号左边的项为物质导数,等号右边的第一项称作当地导数,第二项称作对流导数或者牵连导数,这两项是欧拉法下的描述。

\subsection{推导过程}
\pentry{复合函数的偏导、链式法则(多元微积分)\nref{nod_PChain}}{nod_2980}
可以直接从链式法则推导。

\begin{equation}
\begin{aligned}
\frac{\mathrm{D} \bvec \Phi}{\mathrm{D}t} &= \frac{\mathrm{D} \bvec \Phi(x,y,z,t)}{\mathrm{D}t} = \pdv{\bvec \Phi}{x} \dv{x}{t} + \pdv{\bvec \Phi}{y} \dv{y}{t} + \pdv{\bvec \Phi}{z} \dv{z}{t} + \pdv{\bvec \Phi}{t} \\
&= \pdv{\bvec \Phi}{t} + \left(\dot x \pdv{x} + \dot y \pdv{y} + \dot z \pdv{z}\right) \bvec \Phi\\
&= \pdv{\bvec \Phi}{t} + ({\bvec V \cdot \nabla})\bvec \Phi~.
\end{aligned}
\end{equation}
一般称第一项为局部变化,第二项为对流变化。