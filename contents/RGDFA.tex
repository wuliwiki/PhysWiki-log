% 简单刚体系统的静力学分析
% keys 静力学 刚体 系统 受力分析
% license CCBYSA3
% type Tutor

\footnote{本文参考了张娟著《理论力学》.}本文简要介绍分析简单平面刚体系统的静力学分析。

\subsection{轻绳与轻杆模型}
\subsubsection{轻绳}
\begin{figure}[ht]
\centering
\includegraphics[width=4cm]{./figures/0e39769fc12b745b.pdf}
\caption{轻绳提供沿绳方向的力} \label{fig_RGDFA_1}
\end{figure}
拉紧的轻绳只能受拉而不能受压。

拉紧的轻绳向与其相连的刚体提供拉力,但不能提供压力(轻绳不能被压缩);如果轻绳被拉直,那么轻绳提供的拉力方向总是平行于绳。

\subsubsection{轻杆}
\begin{figure}[ht]
\centering
\includegraphics[width=4cm]{./figures/a4d1e8e12054a040.pdf}
\caption{轻杆提供的力不一定平行于杆} \label{fig_RGDFA_2}
\end{figure}
轻杆可以受压或受拉,同时轻杆提供的力的方向也不一定平行于轻杆。

\subsection{常见约束条件}
\subsubsection{接触面 Surface Constraint}
\begin{figure}[ht]
\centering
\includegraphics[width=6cm]{./figures/36c8e0d41de592a0.pdf}
\caption{接触面的支持力总是垂直于公切线} \label{fig_RGDFA_3}
\end{figure}

放置于平坦接触面的刚体受一个垂直于公切面的支持力。如果接触面粗糙,还可以提供一个平行于接触面的静摩擦力。
\subsubsection{小车模型 Roller Support}
\begin{figure}[ht]
\centering
\includegraphics[width=6cm]{./figures/ee1720a29df4a480.pdf}
\caption{小车提供垂直接触面的力} \label{fig_RGDFA_5}
\end{figure}
小车可以为与其相连的杆提供一个垂直接触面向上的支持力,但是不能提供一个垂直表面向下的拉力;如果接触面是粗糙的,那还能提供一个平行于接触面的静摩擦力。
\subsubsection{铰链模型 Fixed Hinge}
\begin{figure}[ht]
\centering
\includegraphics[width=6cm]{./figures/cee1a07677348143.pdf}
\caption{铰链提供任意方向的力} \label{fig_RGDFA_6}
\end{figure}
铰链可以为与其相连的杆提供一个任意方向的力,但是不能提供力偶。在初步的受力分析中,力的方向不能确定,因此可以先记为两个互相垂直的分力。
\subsubsection{钉子模型 Fixed Rod}
\begin{figure}[ht]
\centering
\includegraphics[width=5cm]{./figures/53d794287a5cc7e1.pdf}
\caption{钉子提供力偶与任意方向的力} \label{fig_RGDFA_7}
\end{figure}
杆钉入墙中的部分可以提供一个任意方向的力与一个力偶。同铰链一样,力的方向无法立刻确定,先记为两个互相垂直的分力。

\subsection{系统、内力与外力}
\begin{figure}[ht]
\centering
\includegraphics[width=8cm]{./figures/e1c8c9a7a7d96e15.png}
\caption{灵光一现(图片来自Pixabay)} \label{fig_RGDFA_15}
\end{figure}

你或许已经了解过系统、内力与外力\upref{PSys}的相关内容。不过,在此之前,你最好复习一下相关概念。

\textbf{系统}是指人为选定的一组物体、组件。系统的选取是\textsl{任意}的,即你应该根据具体问题的不同,以适合解决问题的方式来选取系统。有时,为了研究方便,我们还能在系统中再选取子系统,并分析各个子系统之间的相互作用。例如,我们可以选取灯泡(以及其中的灯丝、装填的惰性气体等)作为一个系统。

\textbf{内力}指受力与施力物体均在系统之内的力。例如,灯头对灯丝施加对支持力就是内力,因为灯头与灯丝都在灯泡这一系统内。

而\textbf{外力}的施力物体在系统之外。例如,灯泡所受的重力(施力物体是地球,在系统之外),电线对灯泡的拉力(施力物体是电线,也在系统之外)都是外力。

我们将在下文探讨为何要选定系统。

\subsection{主动力与约束力}
先简要定义约束力与主动力:
\begin{figure}[ht]
\centering
\includegraphics[width=6cm]{./figures/2d4b201fea5b88f8.pdf}
\caption{主动力与约束力} \label{fig_RGDFA_8}
\end{figure}
\begin{itemize}
\item 约束力:由于约束条件而产生的力,例如桌面对物体产生的支持力,绳子对物体提供的拉力等。
\item 主动力:由于其他外界因素而产生的力,例如重力、外加的压力等。
\end{itemize}

\subsection{受力分析}
总体思路是系统法与隔离法,即先分析系统的整体受力,再逐个分析系统内各个组件的受力。分析系统受力不是一蹴而就,经常是一个反复试错、修正的过程。
\begin{itemize}
\item 先把系统视作一个整体,进行受力分析。此时,系统中各刚体之间的约束力是内力,均可被忽略。可以想象将系统装进一个箱子,这样外界就只看到了一个箱子以及与箱子连接的部分,而对箱子内部的情况一无所知。这样,我们就能暂时不用考虑系统中各组件间复杂的内力。例如,分析灯泡整体的受力时,由于灯泡内部灯头对灯丝施加的支持力等都是内力,因此暂不需考虑。
\begin{figure}[ht]
\centering
\includegraphics[width=12cm]{./figures/4389f5fb1afb0d2f.pdf}
\caption{先把系统视作一个整体,进行受力分析。例如,此处灯泡所受的外力是向下的重力与电线向上的拉力} \label{fig_RGDFA_16}
\end{figure}

\item 随后逐个分析系统内各个组件的受力。一般而言,先分析受到主动力的组件,以及只受两个力的组件(由二力平衡定理立刻得知该两力等大、反向、作用点共线)。
\item 在分析一个组件时,标注主动力,
\begin{figure}[ht]
\centering
\includegraphics[width=6cm]{./figures/5b680e14ff24d763.pdf}
\caption{“标注主动力”} \label{fig_RGDFA_12}
\end{figure}
\item 并根据约束条件标注约束力(注意到,此时系统中其他组件对该组件的约束力不能再被忽视),
\begin{figure}[ht]
\centering
\includegraphics[width=6cm]{./figures/151d78bf4eb12210.pdf}
\caption{“标注约束力”} \label{fig_RGDFA_13}
\end{figure}
\item 最终根据刚体平衡条件计算,并得到各个约束力。
\end{itemize}

\textbf{特别要注意的是},根据牛顿第三定律,若A对B施加约束力,那么B对A也会施加一个等大反向的约束力。也就是说,若A对B施加的约束力已被确定,那么B对A施加的约束力也随之确定。标注组件的约束力时必须注意到这一点,以避免引入多余变量。

\subsection{静力平衡方程}\label{sub_RGDFA_1}
\pentry{刚体的静力平衡\upref{RBSt}}{nod_3dcc}
当系统处于静力平衡时,其中每一个刚体都处于静力平衡。而对于一个刚体,处于平衡的条件是合力为0,且力矩和为0。或许你已经知道,若合力为0但力矩和不为0,那么刚体会开始加速旋转。

\subsubsection{合力}
对于合力,一般把所有力分解至水平与竖直方向,在这两个方向上的合力分别为0. 力是有方向的,在这种平面情况下,以正负号代表方向。可选取水平向右、垂直向上为正方向。
\begin{equation}
\sum F_x=\sum F_y=0~.
\end{equation}
如果一个力的计算结果是负数,那么说明力的实际方向与假定的相反。

\subsubsection{力矩和}
对于力矩和,一般选取未知力较多的点作为参考点,因为作用点在该点上的未知力关于该点的力矩均为0。同时,计算力矩时,也是一般把力分解至水平与竖直方向,分别计算力矩。力矩也是有方向的,在这种平面情况下,以正负号代表方向。可使用右手法则确定力矩的正负:拇指放在参考点,四指指向力的方向。拇指向纸面内为正,朝纸面外为负。
\begin{equation}
\sum M=0~.
\end{equation}
有时,也可以选取多个不同的点作为参考点,并列出多个 力矩和 方程。

\begin{example}{框架结构的受力}
\begin{figure}[ht]
\centering
\includegraphics[width=6cm]{./figures/e8b5db491adee782.pdf}
\caption{框架结构} \label{fig_RGDFA_9}
\end{figure}
如图,杆的重力均不计,分析框架的受力

先分析系统的整体受力。但此时未知力太多,无法得到有价值的信息。

\begin{figure}[ht]
\centering
\includegraphics[width=5cm]{./figures/1abe96b4909a9e09.pdf}
\caption{AB杆受力} \label{fig_RGDFA_10}
\end{figure}

再分析AB杆的受力。显然,AB杆只受两个力(分别由两个铰链提供),根据二力平衡定理,两力等大、反向、作用点共线。

\begin{figure}[ht]
\centering
\includegraphics[width=8cm]{./figures/fe30e69b55f6afa1.pdf}
\caption{BC杆受力} \label{fig_RGDFA_11}
\end{figure}
再分析BC杆的受力。注意根据牛顿第三定律,B处力的方向已经可以确定。

最后,根据各组件力的平衡条件列方程组,即可解得$\bvec F_A, \bvec F_B, \bvec F_C$。例如,在BC杆中,可列出平衡方程:

x方向合力为0: $F_{Cx}-P-F_{B} \cos \theta=0~.$

y方向合力为0: $F_{Cy}-P-F_{B}\sin \theta=0~.$

关于C点的力矩和为0: $P \cdot l_{CP} + F_{By} \cdot l_{CB} = 0~.$

共有3个未知量($F_{Cx}, F_{Cy}, F_B$)与3个方程,原则上可以解出全部未知力。
\end{example}
% \begin{exercise}{“反重力积木”?}
% \begin{figure}[ht]
% \centering
% \includegraphics[width=10cm]{./figures/RGDFA_14.png}
% \caption{“反重力积木”} \label{fig_RGDFA_14}
% \end{figure}
% 试受力分析该“反重力积木”,并说明其实它并没有反重力。\footnote{图片来自网络}
% \end{exercise}
