% 多体薛定谔方程
% keys 波函数|薛定谔方程|双缝干涉|库仑力
% license Xiao
% type Tutor

\begin{issues}
\issueDraft
\end{issues}

\pentry{单粒子薛定谔方程\upref{QMndim}}{nod_2f48}

量子力学假设 $N$ 个粒子的波函数记为
\begin{equation}
\Psi(\bvec r_1, \dots, \bvec r_N, t)~.
\end{equation}
如果每个粒子都延直线运动, 那么某时刻波函数就是 $N$ 维的, 若每个粒子在三维空间运动, 那么就是 $3N$ 维的。

含时薛定谔方程变为
\begin{equation}
-\frac{\hbar^2}{2m} \sum_i\laplacian_i \Psi + V(\bvec r_1, \dots, \bvec r_N) \Psi = \I \hbar \pdv{t}\Psi~.
\end{equation}


\begin{example}{两个粒子的库仑作用}
考虑一维波函数, $\Psi_1(x,t)$ 表示一个从左向右移动的波包, $\Psi_2(x,t)$ 表示从右到左移动的波包, 那么把它们叠加起来之后的波函数 $\Psi_1 + \Psi_2$ 代表什么? 是两个粒子的碰撞吗? 不是的,是同一个粒子和自身的干涉。 又如电子双缝干涉实验中,两个缝隙发出的波或波包互相干涉,是指两个不同的电子的干涉吗? 不是的, 是同一个电子。 双缝干涉实验强调一次只发射一个电子。既然是同一个电子,那么即使它的波函数由两个波包组成, 也不会存在库仑力, 只能互相发生干涉而已, 不存在一个波包把另一个推开的可能性。

如果要考虑两个直线运动的带电粒子(假设质量相同)在库仑力作用下发生速度的变化, 就要考虑双粒子波函数 $\Psi(x_1, x_2, t)$。 注意这是一个二维波函数, 而不是两个一维波函数相加。 相互作用(库仑力)体现在薛定谔方程的势能项中
\begin{equation}
-\frac{\hbar^2}{2m}\laplacian \Psi + \qty(V(x_1) + V(x_2) + \frac{q_1 q_2}{4\pi\epsilon_0\abs{x_2 - x_1}}) \Psi = \I\hbar \pdv{t}\Psi~.
\end{equation}
例如考虑一维简谐振子势能中的两个粒子, 那么 $V(x_i) = ax_i^2/2$, 如果除了相互作用没有外部势能, 那么 $V(x_i) = 0$。

现在考虑两个直线运动的可区分(非全同\upref{IdPar})带电粒子($V(x_i) = 0$)以相同的初速度靠近并在库仑力作用下反弹, 可以假设初始的二维波函数 $\Psi(x_1, x_2, t)$ 是一个二维波包, 波包中心延着轨迹 $x_2 = -x_1$ 从左上方向坐标原点方向几乎匀速地运动。 注意这并不是两个一维波包的运动, 而是一个二维波包的运动。

从另一个角度来说,这个问题完全等效于二维平面 $x$-$y$ 上单个粒子的波包在二维势能 $V(x,y) = q_1q_2/\abs{y-x}$ 中的运动。 画出该势能的等势线会发现它们是和初始波包的运动轨迹 $y-x$ 垂直的, 且越靠近原点势能越大, 也就是初始波包一直在走上坡路。 那么初始波包在该势能的作用下会减速,并延着原来的路径反弹。 这就体现了第一个粒子 $x_1$ 先从负无穷靠近 $0$ 再反弹回负无穷, 第二个粒子 $x_2$ 则从正无穷靠近 $0$ 再反弹回正无穷, 也就是两个粒子在相遇以前就已经被库仑力反弹了。
\end{example}
