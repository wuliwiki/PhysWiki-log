% 南京理工大学  2015 量子真题
% license Usr
% type Note

\textbf{声明}:“该内容来源于网络公开资料,不保证真实性,如有侵权请联系管理员”

\subsection{简答题}

1. 哪些实验现象揭示了微观粒子的波粒二象性(至少写出两条)。

2. 薛定谔方程和定态薛定谔方程分别为_______、________。

3. 写出量子力学工大基本假设中的任意两个。

4. 量子力学中能量算符和动量算符分别为_______、________。

5. 证明关系式 $[\hat{x}, \hat{p}_x] = i\hbar$。。

\subsection{计算题}

1. 有一波函数 $\Psi(r, \theta, \phi) = \frac{1}{\sqrt{4 \pi}} \left( \frac{1}{a_0} \right)^{\frac{3}{2}} 2 e^{\frac{r}{a_0}}$,求 (1) 在点附近体积元 $dr$ 内找到粒子的概率;(2) 在 $r \to r + dr$ 球壳内找到粒子的概率;(3) 在什么位置发现粒子的概率最大。

2. 设在 $H^0$ 表象中,$\hat{H}$ 的矩阵为
$\hat{H} = \begin{pmatrix}E_1^0 & 0 & a \\\\0 & E_2^0 & b \\\\a & b & E_3^0\end{pmatrix}$
其中 $E_1^0 < E_2^0 < E_3^0$。试用微扰论求能量的二级修正。

3. 利用波尔—索末菲量子化条件计算匀场磁场中的圆周运动的电子的可能能量和轨道半径。

(波尔—索末菲量子化条件为 $\oint p dq = (n + \frac{1}{2}) h$,$p$ 为广义动量, $q$ 为对应的广义坐标 )

4. 用波恩近似法求粒子在势场 $U(r) = \begin{cases} 
\frac{Ze^2}{r} - \frac{r}{c} & \text {r < r_0 }\\\\
\qquad0 & \text{r > r_0 }
\end{cases}$ 中散射的微分散射截面。其中
$c = \frac{r_0^2}{Ze_3^2}$

5. 求 $\hat{S}_z = \frac{\hbar}{2} \begin{pmatrix}
0 & 1 \\\\
1 & 0
\end{pmatrix}$, $\hat{S}_y = \frac{\hbar}{2} \begin{pmatrix}
0 & -i \\\\
i & 0
\end{pmatrix}$ 的本征值和相应的本征函数。

6. 假设一氢原子的状态为
$\psi = \begin{pmatrix}\frac{1}{4} R_{32}  Y_{21}(\theta, \varphi) \\\\\frac{\sqrt{15}}{4} R_{32}  Y_{2-1}(\theta, \varphi)\end{pmatrix}$\\
(1) 求轨道角动量$Z$分量 $\hat{L}_z$ 和自旋角动量 $\hat{S}_z$ 的平均值;\\
(2) 求总磁矩 $\hat{M} = -\frac{e}{2\mu}\hat{L} - \hat{S}$ 的 $  Z$ 分量的平均值。

7.电子在半径无限金属表面受到的势场可用如图所示的简化模型加以描述,
\begin{equation}
V(x) =
\begin{cases}
 -V_0,& x < 0 \\\\
\qquad0, & x > 0
\end{cases}~.
\end{equation}

分别计算电子在金属-真空界面的反射和透射概率。(假设电子的能量 $E$ 满足 $-V_0 < E < 0$)
