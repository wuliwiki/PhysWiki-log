% 反射和折射、斯涅尔定律
% keys 光学|反射|折射|斯涅尔定律
% license Xiao
% type Tutor

% 这是二级词条

\pentry{折射定律\nref{nod_Snel}, 平面波的复数表示\nref{nod_PWave}}{nod_210d}
% 未完成

首先证明 $A_1 \E^{\I \bvec k_1 \vdot \bvec r_1} + A_2 \E^{\I \bvec k_2 \vdot \bvec r_2} = A_3 \E^{\I \bvec k_3 \vdot \bvec r_3}$ 在 $z = 0$ 平面上处处成立得条件是

\begin{enumerate}
\item $\bvec k_1, \bvec k_2, \bvec k_3$ 共面
\item $k_1 \sin \theta_i = k_2 \sin \theta_r$, 两边乘以 $c/\omega$ 得 $n \sin\theta_i = n' \sin \theta_r$。 即斯涅尔定律。
\end{enumerate}
斯涅耳定律完全由波的本质决定, 与电磁场的特性没有关系。 任何满足以上形式的平面波都可以用该公式。
