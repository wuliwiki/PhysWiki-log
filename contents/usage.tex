% 小时百科使用说明
% license Xiao
% type Tutor

这是给普通读者的说明,给作者的说明见\href{https://wuli.wiki/editor}{编辑器首页}。

\subsection{页面导航}
百科中每篇文章中的公式和图标等的编号从 1 开始, 较长的公式或表格有可能超出屏幕范围, 可以用光标或手指拉动公式即可. 例如:
\begin{equation}\label{eq_usage_1}
\sin x = x - \frac{1}{3!} x^3 + \frac{1}{5!} x^5 - \frac{1}{7!} x^7+\frac{1}{9!} x^9-\frac{1}{11!} x^{11}+\frac{1}{13!} x^{13}-\frac{1}{15!} x^{15}+\frac{1}{17!} x^{17}-\frac{1}{19!} x^{19}+ \order{x^{21}}~.
\end{equation}

\textbf{返回:}如果点击的链接跳转到同一页面的不同位置(如脚注,公式引用等), 可以用快捷键 “Alt + 左方向键” 返回跳转以前的位置.

\textbf{搜索:}若要在单个页面内搜索, 用快捷键 Ctrl + F, 若要全站搜索, 请使用\href{https://wuli.wiki}{网站首页}的搜索按钮.

\textbf{脚注:}当点击一个脚注跳到页面底部,再点击脚注的数字编号可以回到之前的位置。

\subsection{引用}
为方便读者, 百科中设置了大量的链接, 其中文章链接的形式如“\enref{导数简介}{Der}”,同一页面内公式的链接如“\autoref{eq_usage_1} ”,其他文章中的公式链接如“\autoref{eq_Der_3}~\upref{Der}”,图表等链接的格式相同.

在 pdf 中, 右上角中括号中的数字代表被引用文章的页码。

\subsection{预备知识和知识树}
\begin{figure}[ht]
\centering
\includegraphics[width=14.25cm]{./figures/a8d215c416edfe14.png}
\caption{文章菜单} \label{fig_usage_2}
\end{figure}

在每个页面中,我们假设您已经掌握了预备知识中列出的页面,以及预备知识的预备知识等等。如果您因为没有这么做而看不懂某个页面,我们无能为力。 但如果您已经掌握了预备知识却仍然看不懂一个页面,请在评论中反馈,我们会尽快改进。您的反馈对内容的创作至关重要。如果有的预备知识显示为灰色,说明它已经包含在其他预备知识中,只是列出来作为强调。

您也可以点击某个页面的 “知识树” 按钮将该文章的所有预备知识绘制成树状图。 直接访问\href{https://wuli.wiki/tree/}{知识树主页}可以查看百科中所有文章构成的总知识树。每个页面中的 “知识树” 按钮相当于在选中树中某个节点并点击 “选为目标”。

如果您想知道当前页面被哪些页面作为预备知识,您也可以选中树种某个节点然后点击 “选为起点”。
