% 函数(高中)
% 函数|定义域|值域|二元函数

\pentry{集合(高中)\upref{HsSet}}

\begin{issues}
\issueDraft
\issueTODO
\end{issues}

% 高中的 函数 不应该需要 映射 作为预备知识
本篇文章的预设读者是,对于函数的定义感到熟悉但又有些模糊,希望进一步了解函数的高中学生。

函数最开始是用于研究曲线的工具。高中时期函数往往出现在平面直角坐标系上,不同的函数就对应着不同的曲线,你也许有些困惑——说不清二者之间的区别和联系。比如$f(x)=x^2$,对于每一个$x_1$,都能够找到对应的$f(x_1)$,进而得到平面上对应的点$(x_1,f(x_1))$,这是它们的联系。而区别在于,不应认为曲线就是我们口中的函数,只能认为,一个函数可以确定一条曲线。

在高中时期,一个函数通常指的是一个计算式,输入一个数字,然后通过计算式计算以输出一个数字。数学家们在上述内涵中,进一步地抽象得到更加广泛的定义,以此让函数这一概念能够更加精确、更加广泛的描述事物。

我们可以将函数认作是对输入和输出的关系的描述,我们所见到的\textsl{计算式}就是告诉我们如果将输入得到输出的方法。用工厂来比喻,我们向自动化工厂的入料口中添加原材料,经过一系列复杂的加工得到了产品,其中的加工方法、流程就可以粗略地看作是一个函数。又或者更加具体的说,让若干水果变成水果沙拉的函数是,一份水果沙拉的制作方法+我们灵巧的双手。在原有的计算式的理解上更进一步,得到了计算机学科中对函数的理解。
\begin{enumerate}
\item 从内容上,我们将输入、输出从数字拓宽到了更多事物
\item 从数量上,我们不局限于单一输入,而是能够同时输入多种事物
\end{enumerate}


\begin{figure}[ht]
\centering
\includegraphics[width=5cm]{./figures/42b6ece8604a4742.png}
\caption{函数} \label{fig_functi_1}
\end{figure}

但是上述理解仍然不是数学上的理解,在数学中,我们将函数的概念进一步抽象。抛开对象之间如何实现转换的过程,而仅仅在两个事物之间建立对应关系——将集合中的全体元素,向另一个集合中的元素建立对应法则,一个对应法则就称为一个函数。
$$f:\{1,2,3,4\}\to\{1,2,3,4,5\}$$
其中具体的对应法则为$f(1)=2,f(2)=3,f(3)=1,f(4)=1$,这样我们就打造了一个函数$f$。其中左侧的集合$\{1,2,3,4\}$称为\textbf{定义域},右侧的集合$\{1,2,3,4,5\}$称为\textbf{上域},全体定义域经过对应关系得到的全体元素的集合为$\{1,2,3\}$,称为\textbf{值域或像}。在这个记号下,我们将常见的抛物线方程记为
\begin{equation}
f:\mathbb{R}\to{\mathbb{R}},x\mapsto{x^2}
\end{equation}
我们已经从原有的函数概念中抽象得到了不同的概念,为了区分前后,我们为后来的解释取一个新的名字——\textbf{映射}。

在这个角度,我们可以

%广义来说, 任何映射都可以叫做\textbf{函数(function)}。 所以我们也可以用映射的记号表示函数。 例如 $f: \mathbb R \to \mathbb R$ 表示定义域为实数集, 值为实数的函数, 通常记为 $f(x)$。 又例如 $f: \mathbb R^2 \to \mathbb C$ 表示实变量和复值的二元函数 $f(x_1, x_2)$。 注意其中 $\mathbb R^2$ 表示笛卡尔积(\autoref{eq_Set_1}~\upref{Set}) $\mathbb R \times \mathbb R$。

\subsection{单射满射双射}

上文中,我们已经粗糙地展示了映射的内涵,在这个基础上补充一些扩展资料。

\begin{definition}{单射}
设$f:A\to{B}$,任意两个不同的集合A中的元素$a,b\in{A}$,$a\not={b}$,都有$f(a)\not={f(b)}$,则称$f$是一个单射。
\end{definition}

\begin{figure}[ht]
\centering
\includegraphics[width=5cm]{./figures/2acb45dfe0b6742b.png}
\caption{单射}\label{fig_functi_2}
\end{figure}

\begin{definition}{满射}
设$f:A\to{B}$,任意集合B中的元素$c\in{B}$,都存在$a\in{A}$,使得$f(a)=c$。则称$f$是一个满射。
\end{definition}

\begin{figure}[ht]
\centering
\includegraphics[width=5cm]{./figures/d9926f81f5c41467.png}
\caption{满射} \label{fig_functi_3}
\end{figure}


\begin{definition}{双射}
设$f:A\to{B}$,$f$既是单射又是满射,则称$f$是一个双射或者一一对应。
\end{definition}

\begin{figure}[ht]
\centering
\includegraphics[width=5cm]{./figures/d5d47614fcd85f2b.png}
\caption{双射} \label{fig_functi_4}
\end{figure}

注:对于两个只有有限个元素的集合之间的映射,如果是双射,那么说明两个集合中的所有元素都可以两两的唯一对应$f(a)=b,f^{-1}(b)=a$,并且两个集合的元素数量是相等的。


\subsection{复合函数}
\addTODO{复合函数}


\subsection{函数的性质}
以后我们会看到一些用极限\upref{Lim}和导数\upref{Der}描述的性质。 例如连续性\upref{contin}, 一致连续 % 未完成
, 可导。
