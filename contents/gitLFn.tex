% Git-LFS 笔记
% license Usr
% type Note

\begin{issues}
\issueDraft
\end{issues}

\begin{itemize}
\item 更复杂的解决方案,见 Git Scalar(linux 支持 2020/10 才完成,而且需要 Windows 版本的 git 才可以实现后台 maintenance 服务,另外可以选择使用 GVSF 虚拟文件系统) 和 Git Annex\upref{gitanx} 等。
\item git-lfs 的原理很简单,对于超过一定尺寸的文件,把它们替换成一个文本文件里面是该文件的 hash 和其他信息 (该文本文件叫做 pointer file)。 然后用传统的 git 来储存小文件以及 pointer file。 而大文件使用服务器上的另外服务储存,不在传统的 git 上游文件夹,甚至可以是一个完全不同的服务器。 本地仓库的 \verb`.git/lfs` 文件夹也可以储存大文件的缓存, 而 working tree 中可以选择保留 pointer 或者 checkout 真正的文件。
\begin{lstlisting}[language=none,caption=example\_pointer\_file]
version https://git-lfs.github.com/spec/v1
oid sha256:24b9da6552252987aa493b52f8696cd6d3b00373f2ed1f20c5908b07cfe4f2c0
size 12345
\end{lstlisting}
\item 被 lfs 处理的大文件即使改变一点,也没有 delta 储存,而是会把整个文件都新上传到服务器,占用服务器上新文件大小的额外空间。
\item github 和 gitlab 都支持 LFS, Linux/Windows/MacOS 也支持。
\item 上游 repo 只支持网络不支持本地文件系统,但应该可以通过 \verb`loop back` 端口使用本地文件系统。
\item 第一次 clone 的时候,默认大文件都不下载,只下载 pointer 文件。
\end{itemize}
