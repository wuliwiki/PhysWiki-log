% 常微分方程
% 微积分|简谐振子|微分方程|常微分方程

\begin{issues}
\issueDraft
\end{issues}

\pentry{简谐振子\upref{SHO}}

作为一个引入的例子, 我们首先看“简谐振子\upref{SHO}” 中的\autoref{eq_SHO_1}。 一般来说, 含有函数 $y(x)$ 及其高阶导数 $y^{(n)}$, 和自变量 $x$ 的等式叫做\textbf{常微分方程}(简称微分方程\footnote{这里的 “常” 强调未知函数只有一个变量, 用于区别多元微积分中的“偏微分方程”。}), 即
\begin{equation}
f(y^{(N)}, y^{(N-1)}, \dots, y, x) = 0
\end{equation}

上式中的最高阶导数为 $N$ 阶, 所以可以把上式叫做 $N$ 阶微分方程。 注意方程中必须出现 $y^{(N)}$, 剩下的 $y^{(N-1)}, \dots, y, x$ 可以只出现部分或不出现。所有能使微分方程成立的函数 $f(x)$ 都是方程的\textbf{解}, 如果能找到含有参数的函数 $f(x,C_1, \dots , C_N)$, 使所有可能的解都可以通过给 $C_i$ 赋值来表示, 那么这就是函数的\textbf{通解}。

有一些微分方程的解法是显然的, 例如描述自由落体运动\upref{ConstA} 的微分方程为 $\dv*[2]{y}{t} = g$ (假设 $y$ 轴竖直向下)。 要解这个方程, 只需对等式两边进行两次不定积分即可得到通解为 $y = C_1 + C_2 t + gt^2/2$。 一般来说, 如果 $N$ 阶微分方程具有 $y^{(N)} = f(x)$ 的形式, 只需进行 $N$ 次积分即可得到通解。

另一些方程是\textbf{可以分离变量}的, 我们来看“受阻落体\upref{RFall}” 这个例子。 若方程可分离变量, 只需先分离变量, 再对等式两边求不定积分即可找到通解。



\subsection{一阶线性微分方程}
一阶线性微分方程\upref{ODE1}。

\subsection{二阶线性微分方程}

\addTODO{链接一下 Wolfram Alpha 和 Mathematica 解常微分方程}
