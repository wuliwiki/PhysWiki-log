% 常微分方程

\pentry{简谐振子\upref{SHO}}

作为一个引入的例子, 我们首先看“简谐振子\upref{SHO}” 中的\autoref{SHO_eq1}。 一般来说, 含有函数 $y(x)$ 及其高阶导数 $y^{(n)}$, 和自变量 $x$ 的等式叫做\textbf{常微分方程}, 即
\begin{equation}
f(y^{(N)}, y^{(N-1)}, \dots, y, x) = 0
\end{equation}

上式中的最高阶导数为 $N$ 阶, 所以可以把上式叫做 N阶微分方程。 所有能使微分方程成立的函数 $f(x)$ 都是方程的解, 如果能找到含有参数的函数 $f(x,C$, 使所有可能的解都可以通过给 $C_i$ 赋值来表示, 那么这就是函数的通解。