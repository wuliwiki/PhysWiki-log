% 哈尔滨工业大学 2002 年硕士物理考试试题
% keys 哈尔滨工业大学|考研|物理|2002年
% license Copy
% type Tutor

\textbf{声明}:“该内容来源于网络公开资料,不保证真实性,如有侵权请联系管理员”

\begin{enumerate}
\item 把点电荷$Q$从很远处移相内、外半径依次为$R_1,R_2$的导体球壳并穿过球壳上一小孔一直移至球心处,求此过程中所需要做多少功。
\item 两同心导体球壳半径$R_1<R_2$,球壳间充满导电率为$\sigma$、介电常数为$\varepsilon$的均匀介质。设$t=0$时刻内球壳带电$Q$,试计算:\\
(1)介质中的电流强度;\\
(2)电流总共产生多少焦耳热。
\item 有人认为地磁来源于地球中心处的小电流环,已知地极附近的磁感应强度为$B$,地球的半径为$R$,试用毕奥-萨伐尔-拉普拉斯定律求小电流环的磁矩。
\item 一个细而薄的圆柱壳体长为$l$,半径为$a$,面电荷密度为$\sigma$。今圆柱壳以匀角加速度$\beta$绕中心轴转动,若不计边缘效应,试求:\\
(1).圆柱壳内的磁场;\\
(2).圆柱壳内的电场;\\
(3).圆柱壳内的磁能与电能。
\item 迈克尔逊干涉仪用波长$589.3nm$的钠黄光作光源,视场中心为亮点,此外还能看到10个亮环,今移动一臂中的反射镜,发现有10个亮环向中心收缩而消失,即中心级次减小10,此时视场除中央亮点外还剩5个亮环,试求开始时中央亮点的干涉级,反射镜移动的距离,以及反射镜移动后视场中最外那个亮环的干涉级。
\item 一平面透射光栅,当用白光垂直照射时,能够在$30$°角衍射方向上观察到波长 $600nm$ 的第二级主极大干涉条纹,并能在该处分辨$\Delta \lambda=0.005nm$的两条光谱线,但在此方向上测不到波长$400nm$的第三级主极大条纹,求:\\
(1)光栅常数$d$和总缝数$N$;\\
(2)光栅的缝宽$a$和缝间距$b$;\\
(3)光栅的总宽度;\\
(4)对 $400nm $的单色光能看到哪些谱级。
\item 两个偏振方向正交放置的偏振片,以光强为$I_0$的自然单色光照射,若在其中插入另一块偏振片,求:\\
(1)若透过的光强为$I_0/8$,插入的偏振片方位角;\\
(2)若透过的光强为$0$,插入的偏振片方位角;\\
(3)能否找到合适的方位,使透过的光强为$I_0/2$;\\
(4)若在其中插入一块$ 1/4$波片,其光轴与第一块偏振片的偏振方向成 $30$°角,出射光的强度为多少。
\item 一双凸透镜的第一、二折射面的曲率半径分别为$r_1=20cm,r_2=25cm$。已知它在空气中的焦距为$f_1=20cm$,现将其置于如图的方玻璃水槽中并在其前的水中置高为$1cm$ 的小物体,它距透镜为$100cm$,求:\\
(1)通过透镜所成像的位置、大小,虚像还是实像。\\
(2)若用眼睛在玻璃外A处观察,该像的视位置与槽壁的距离$l$。设玻璃槽壁的折射忽略不计,水的折射率为$n=1.33$。
\begin{figure}[ht]
\centering
\includegraphics[width=10cm]{./figures/e2875571428f1101.png}
\caption{} \label{fig_HGD02_1}
\end{figure}
\end{enumerate}