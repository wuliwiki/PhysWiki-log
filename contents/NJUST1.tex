% 南京理工大学 2011 量子真题
% license Usr
% type Note

\textbf{声明}:“该内容来源于网络公开资料,不保证真实性,如有侵权请联系管理员”

\subsection{简答题(请考生在下列12题中选作10题,每题6分,共60分):}

1. 一粒子的波函数为 $\psi (\mathbf{r}) = \psi (x, y, z)$,写出粒子位于 $x \sim x + d x$ 间的几率;用球坐标表示,粒子波函数表示为 $\psi (r, \theta, \varphi)$,写出粒子在球壳 $(r, r + \mathrm{d}r)$ 中被测到的几率,及在 $(\theta, \varphi)$ 方向的立体角 $\mathrm{d}\Omega$ 内找到粒子的几率;

2. 何谓正常塞曼效应?何谓反常塞曼效应?何谓斯塔克效应?

3.问下列算符是否是厄米算符,并指明原因:\\
(1)$\quad \ddot{x}_i,$\\ (2)$\quad \frac{1}{2}(\ddot{p}_i + \dot{p}_i \dot{x}_i)$

4. 写出在 $\hat{S}^2$ 和 $\hat{S}_z$ 的共同表象中泡利矩阵的表示式。

5. 一体系统由两个全同的玻色子组成,玻色子之间无相互作用。玻色子只有两个可能的单粒子态,分别为 $\varphi_1(q_1)$, $\varphi_j(q_1)$ 和 $\varphi_1(q_2)$, $\varphi_j(q_2)$。问体系可能的状态有几个?它们的波函数怎样用单粒子波函数构成?

6.对一个量子体系进行某一力学量的测量值,测最结果与表示力学量算符有什么关系?两个力学量同时具有确定值的条件是什么?

7.多粒子系的一个基本特征是什么?何谓全同粒子?

8. 下列波函数所描写的状态是否为定态?并说明其理由。

 (1) $\psi_1(x,t) = \varphi(x) e^{\frac{i \xi t}{\hbar}} + \varphi(x) e^{-\frac{i \xi t}{\hbar}}$

 (2) $\psi_2(x,t) = u(x) e^{\frac{i (x - \xi t)}{\hbar}} + v(x) e^{-\frac{i (x + \xi t)}{\hbar}}$

9.已知 $\vec{L} \times \vec{L} = i\hbar \vec{L}$,问 $L_x, L_y, L_z$ 是否一定不能同时测定?说明原因或举例说明。

10.与自由粒子相联系的波是什么波?与经典波有何差异?

11.若两个力学量 $\hat{A}, \hat{B}$ 的对易关系式为 $[\hat{A}, \hat{B}] = i\hat k$,写出其测不准关系的严格表示式。

12.写出轨道磁矩与轨道角动量的关系,自发角动量与自旋磁矩的关系。

\subsection{计算题(请考生在下列7题中任选6题,每题15分,共90分):}

1. 由下列定态波函数数计算几率流密度:

(1)$\psi_1 = \frac{1}{r} e^{ikr}$ 


(2)$\psi_2 = \frac{1}{r} e^{-ikr}$

从结果说明 $\psi_r$ 表示向外传播的球面波, $\psi_\ell$ 表示向内(即向原点)传播的球面波。
$\text{[提示:在球坐标中}\quad \nabla = \mathbf{e}_r \frac{\partial}{\partial r} + \mathbf{e}_\theta \frac{1}{r} \frac{\partial}{\partial \theta} + \mathbf{e}_\varphi \frac{1}{r \sin \theta} \frac{\partial}{\partial \varphi}\text{]}]$

2. 证明 $\psi (x,y,z) = x + y + z$ 是角动量算符 $l^2$ 的本征值为 $2\hbar^2$ 的本征函数。

3. 证明下列对易关系 $[ \hat{y}, \hat{z} ] = i\hbar \hat{x}$, $[\hat{z}, \hat{x}] = i\hbar \hat{y}$, $[\hat{x}, \hat{y}] = i\hbar \hat{z}$ 和 $[l_i, p_j] = 0$。

4.线性谐振子在初始时刻处于归一化状态:
$\psi(x) = \frac{1}{\sqrt{5}} \psi_0(x) + \frac{1}{\sqrt{2}} \psi_2(x) + c_5 \psi_5(x)$

式中 $\psi_n(x)$ 表示谐振子第 $n$ 个定态波函数,求:

(1)系数 $c_5$?

(2)写出 $t$ 时刻的波函数;

(3)$t=0$ 时刻谐振子的能量的可能取值及其相应几率,并求其平均值

5. 求自旋角动量在 $(\cos \alpha, \cos \beta, \cos \gamma)$ 方向上的投影
$[\hat{S}_n = \hat{S}_x \cos \alpha + \hat{S}_y \cos \beta + \hat{S}_z \cos \gamma]$
的本征值和本征函数。

