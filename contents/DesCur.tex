% 可取曲线(变分学)
% 泛函|可取曲线

Hamilton 分析方法是现代理论物理的通用方法,只要给定相应的作用量,就可以通过Hamilton 分析构建一门理论,并判断理论的自洽性及对称性和相应的守恒量等诸多性质。经典力学如此,相对论如此,量子力学如此,杨米尔斯理论亦可如此。要掌握 Hamilton 分析,必要的数学准备是逃不开的。当然,我们不可能去搞懂每一个数学上的细节,因为每一个数学细节背后往往都有一门深厚的学问。同时也不该漏掉一些重要的数学概念,因为往往一个艰巨的物理问题后面只是一个简单的数学原理。我们力求在用到数学的地方,都有一个较合理的解释,以便更清楚的看清问题,把握问题的实质。

变分学是 Hamilton 分析绕不过的一道坎,这也是目前首要的任务之一。当然,我们只挑取以后将用到的一部分内容。现在开始吧!
\subsection{泛函}
\begin{figure}[ht]
\centering
\includegraphics[width=8cm]{./figures/DesCur_1.pdf}
\caption{过给定两点之间的曲线} \label{DesCur_fig1}
\end{figure}
在变分法中,我们须研究这样的关系,其中因变数的值是由函数所确定的。比如研究连接给定两点 $A(x_A,y_A),B(x_B,y_B)$ 的任意曲线的长度,因变数的值“曲线的长度”是由连接 $A,B$ 两点的曲线的形状决定的。设连接 $A,B$ 两点的曲线的方程为
\begin{equation}
y=y(x)
\end{equation}
并设横坐标 $x$ 在区间 $x_A\leq x\leq x_B$ 上变动,而函数 $y(x)$ 在着区间内有连续的微商 $y'(x)$ 。于是曲线的长度 $J$ 等于
\begin{equation}
J=\int_{x_A}^{x_B} \sqrt{1+y'^2}\dd x
\end{equation}
当函数 $y(x)$ 改变时,曲线的长度 $J$ 也将改变。所以, $J$ 是依赖于函数 $y(x)$ 的。如果 $J$ 的值随着某一类函数中的函数 $y(x)$ 而确定,我们就可以写成
\begin{equation}
J=J[y(x)]
\end{equation}
通过这个例子,便可引出泛函的概念。
\begin{definition}{泛函}
设 $y(x)$ 是给定的某类函数。如果对于这类函数 $y(x)$ 中的每一个函数,有某数 $J[y(x)]$ 与之对应,那么我们说 $J[y(x)]$ 是这类函数 $y(x)$ 的\textbf{泛函}。
\end{definition}
\subsection{可取曲线}\label{DesCur_sub1}
为将多元函数在极值点处微分等于零的原理,推广到泛函上去,需要了解可取曲线概念。

在函数 $f(x)$ 中,自变数 $x$ 可取到的所有数构成的集合叫作函数 $f(x)$ 的定义域。同样,在变分学中,为求得使泛函给出极值的曲线,有可取曲线的概念。显然,为寻求使某一泛函达到极值的曲线,须首先指出关于函数定义区域内的曲线族的性质,以便在这族曲线中寻求给出极值的曲线,这族曲线叫做变分问题的\textbf{可取曲线}。

可取曲线的确定随问题的性质而改变。比如在初等变分法中,可取曲线是连接两定点的平面曲线;在等周问题中,可取曲线必须有定长。

对于积分 $\int F(x,y,y')\dd x$ 所表示的曲线函数,我们暂且只限于这种曲线 $y=y(x)$ 所组成的可取曲线族,这种函数 $y=y(x)$ 和它的一级微商都是连续的。

这样我们可对可取曲线在两方面加于限制:一方面在函数理论的性质上限制它(例如,代表曲线的函数的连续性及其微商的连续性)。另一方面在问题的本身的性质上限制它(例如,在等周问题中可取曲线长度是相等的),改变问题的性质,就得到新的问题,它需要新的方法来解决。