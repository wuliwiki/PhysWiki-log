% 电流产生磁场
% keys 电流|磁场|线圈|电磁学
% license Xiao
% type Tutor

\addTODO{这属于科普部分,主要定性说明, 电流周围会产生闭合磁场, 随距离减弱, 线圈内部会有近似直线的磁场, 另外提一下两个线圈的那个装置叫做什么?画两个图}

在很长的一段时间内,科学家们认为电学和磁学是两个不同的分支,电学是研究正电荷负电荷的学科,导体内定向移动的电荷导致电流,库仑发现了电荷之间的相互作用的定律——\enref{库仑定律}{ClbFrc},法拉第提出电场概念表示电荷激发的一种力场。而磁学是研究磁体之间的相互作用,磁铁有正负两极,产生磁场,磁场会对位于其中的铁磁性物质或者磁铁有一定相互作用。那时的人们还无法想象,电和磁能在一个统一的理论框架下共同描述。

1820 年 4 月的一天,丹麦科学家奥斯特发现通电导线附近,指南针的指向会受到影响。经过反复的实验奥斯特最终确定了电流周围会产生磁场,这对那时的学界带来了巨大的震撼,电学和磁学从此成为了两个联系紧密的学科,人们开始通过更精细的定性定量实验研究它们之间的联系。
\subsection{奥斯特实验与电流的磁效应}
奥斯特将一个指针放在通电直导线的附近,发现指针的 $S$-$N$ 会受到力的作用,最终朝向与电流方向垂直的切向方向,而当导线不通电的时候,指针放在导线附近不会受到力的作用。因此奥斯特推断,通电直导线附近会激发一个环绕导线的磁场。如果将磁场线画出来,磁场线会形成一个环绕导线的一个闭合回路,离电流越远的地方磁场越弱,离电流越近的地方磁场越强。

磁场的方向可以利用右手定则确定,将大拇指指向通电直导线的方向,四根手指就表示磁场的方向。

\begin{figure}[ht]
\centering
\includegraphics[width=12cm]{./figures/96b34ecfda1131e0.png}
\caption{通电导线产生的磁场} \label{fig_CurMag_1}
\end{figure}

根据\enref{麦克斯韦方程}{MWEq} 静磁学情况下的公式 $\curl \bvec B=\mu_0 \bvec J$ 以及 \enref{stokes 公式}{Stokes},可以推断出通电直导线附近的磁场大小与距离的关系式是:
\begin{equation}
2\pi R B = \mu_0 I\quad \Rightarrow\quad B=\frac{\mu_0 I }{2\pi R}~.
\end{equation}
其中 $R$ 为与通电直导线的垂直距离。可以看到磁场的大小与距离成反比。
\subsection{通电螺线管}
当通电导线弯曲后,环绕导线的磁场仍然存在,方向仍然可以用右手定则确定。一个电流方向为逆时针的闭合线圈,由电流的磁效应导致的磁场方向是:磁感线从线圈内穿出,从线圈外绕回去回到线圈内部,这相当于一个很薄的磁铁,上边是 $N$ 极,下面是 $S$ 极。

如果将这些许多通电的闭合线圈叠在一起,形成一个通电螺线管,线圈内的磁场大小就会增强,整个通电螺线管就相当于一个\textbf{电磁铁},可以用\textbf{电流大小}和\textbf{线圈匝数}来调控电磁铁产生的磁场大小。