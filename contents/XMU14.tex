% 厦门大学 2014 年 考研 量子力学
% license Usr
% type Note

\textbf{声明}:“该内容来源于网络公开资料,不保证真实性,如有侵权请联系管理员”

\subsection{一、}
(1)下列波函数所描述的状态是否为定态?并说明。

①$\phi(x,t)=\varphi(x)e^{\frac{i}{\hbar}Et}+\varphi(x)e^{\frac{-i}{\hbar}Et}$

②$\phi(x,t)=\varphi(x)_1e^{\frac{i}{\hbar}(px-Et)}+\varphi(x)_2e^{\frac{-i}{\hbar}(px+Et)}$

2)已知算符 $A,B,C$ 中,$A$ 和 $B$ 对易,且 $A$ 和 $C$ 也对易,问 $B$ 和 $C$ 是否一定对易?举例说明你的结论。

(3)对于全同粒子体系,什么是全同性原理?描写分别由电子和光子组成的全同粒子体系的波函数的特点。

(4)什么是电子的自旋?电子自旋与轨道角动量有什么不同之处?

(5)写出电磁场中带电粒子的薛定谔方程:它是否具有规范不变性?若有,指出在规范变换下波函数应作何变换?
\subsection{二、}
设质量为 m 的粒子处于下列一维无限深势阱中,
$$V(x,y)=\begin{cases}
0,&0 < x <a \\\\
\infty ,& X < 0,  x>a 
\end{cases}~
$$
已知初始时刻粒子的波函数为$\phi(X,0)=AX(a-X)$

(1)求归一化系数 $A$;

(2)给出粒子的能量本征态$\phi_n(X)$及其能级$E_n$ ;

(3)求测得粒子处于能量本征态$\phi_n(X)$的概率;

(4)求 $t>0$ 时刻粒子的波函数$\phi(X,t)$,只要求给出计数表达式。
\subsection{三、}
已知哈密顿量 $\hat{A}$ 的本征矢为 $|n\rangle$, 本征值为 $E_n$, 若 $\hat{A}$ 的本征矢组 $\{|n\rangle\}$ 构成正交归一的完备基, 定义一个算符 $\hat{U}(m,n)=\ket{m}\bra{n}$

(1) 计算对易式 $\left[\hat{H}, \hat{U}(m,n)\right]$;

(2) 证明 $\hat{U}(m,n)\hat{U}^+(p,q) = \delta_{nq} \hat{U}(m,p)$;

(3) 计算 $\hat{U}(m,n)$ 的迹 $\mathrm{Tr}\{\hat{U}(m,n)\}$;

(4) 若算符 $\hat{A}$ 的矩阵元为 $A_{mn} = \langle m|\hat{A}|n\rangle$,证明:

① $\hat{A} =\sum_{mn} A_{mn} \hat{U}(m,n)$

②$A_{pq} =T_r\left\{\hat{A}\hat{U}^+(p,q)\right\}$
\subsection{四、}
一个质量为 $m$ 的一维粒子在如下势场中运动
$$V(X)=\frac{K}{2}(X-X_0)^2+V_0~$$
其中 $k, x_0, V_0$ 为已知量

(1)给出该离子系统的定态波函数以及相应的能量本征值;

(2)讨论系统的能及间距对 $k,x_0$ 以及 $V_0$ 的依赖关系;

(3)该系统有无可能存在非束缚态?
\subsection{五、}
氢原子处于下列状态
$$\Phi(r, s_z) = A \begin{pmatrix}
\sqrt{\frac{3}{5}} \phi_{100}(\vec{r}) + \sqrt{\frac{2}{5}} \phi_{211}(\vec{r}) \\\\
\sqrt{\frac{2}{5}} \phi_{100}(\vec{r}) - \sqrt{\frac{2}{5}} \phi_{200}(\vec{r})
\end{pmatrix}~
$$
其中 $\phi_{nlm}$ 为能量本征函数,$\hat{H} \phi_{nlm} = E_n \phi_{nlm}$ ,求

\begin{enumerate}
  \item 归一化常数 $A$;
  \item 轨道角动量 $L_z$ 的平均值;
  \item 同时测量 $E = E_2$,$L^2 = 2\hbar^2$ 的概率;
  \item 电子处于总角动量 $j = 3/2$,$m_j = 3/2$ 的概率。
\end{enumerate}
\subsection{六、}
已知微扰前体系的哈密顿量 $H_0$ 存在一系列非简并能级 $E_n^{(0)}$,相应的能量本征态记为 $|n\rangle$ ($n = 0, 1, 2, 3, \cdots$)。给定厄米算符 $A, B$,以及 $C = i[B, A]$,并且 $A, B, C$ 在微扰前基态 $|0\rangle$ 下的平均值均为已知,记为 $A_0, B_0, C_0$。假设体系受到微扰作用,微扰算符可以表示为 $H' = i\lambda[A, H_0]$ ($\lambda$ 为实数小量)

求:

(1) 基态波函数的一阶修正;

(2) 算符 $B$ 在修正后基态下的平均值(准确至 $\lambda$ 量级)。
