% 多元狄拉克 delta 函数
% keys delta 函数|正交曲线坐标系
% license Xiao
% type Tutor

\begin{issues}
\issueDraft
\end{issues}

\pentry{狄拉克 delta 函数\upref{Delta}, 正交曲线坐标系\upref{CurCor}}{nod_7bc9}

在三维直角坐标系中, 
\begin{equation}
\delta(\bvec r) = \delta(x)\delta(y)\delta(z)~.
\end{equation}
这样就有
\begin{equation}
\int \delta(\bvec r) \dd{V} = 1~.
\end{equation}

\subsection{其他坐标系}
若要把 $\delta(\bvec r - \bvec r_0)$ 转换到其他正交曲线坐标系中, 直接使用
\begin{equation}
\delta(\bvec r - \bvec r_0) = \frac{1}{\abs{\bvec r_u(u_0)\bvec r_v(v_0)\bvec r_w(w_0)}}\delta(u - u_0)\delta(v - v_0)\delta(w - w_0)~,
\end{equation}
其中 $\bvec r_u(u_0)\bvec r_v(v_0)\bvec r_w(w_0)$ 分别是 $\bvec r$ 在 $u_0,v_0,w_0$ 对 $u, v, w$ 的偏导数。

例如球坐标系中
\begin{equation}
\bvec r_r = 1 ~,\qquad \bvec r_\theta = r~, \qquad \bvec r_\phi = r\sin\theta~.
\end{equation}
有
\begin{equation}
\delta(\bvec r - \bvec r_0) = \frac{1}{r_0^2\abs{\sin\theta_0}}\delta(r - r_0)\delta(\theta - \theta_0)\delta(\phi - \phi_0)
\quad (r_0, \theta_0 \ne 0)~.
\end{equation}
但当 $r_0$ 或 $\theta_0$ 为零时怎么办呢? 用 $\delta^2$ 来表示! 但如何证明?
\addTODO{……}
