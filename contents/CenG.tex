% 重心
% keys 重力|质心|质心系|力矩
% license Xiao
% type Tutor

\begin{issues}
\issueDraft
\end{issues}

\pentry{质心\nref{nod_CM}, 力矩\nref{nod_Torque}}{nod_3bfb}

我们来将\autoref{ex_CM_1}  拓展到一般情况

我们先来定义重心: 一个刚体在均匀重力场中的\textbf{重心(center of gravity)}, 就是它所受的重力关于重心产生的合力矩为零的点。

可以证明, 质心就是重心。

\subsection{力矩}
\addTODO{刚体在计算力矩时可以看成重力都集中在质心一点 (参考刚体摆例题)。 可得刚体通过某点被悬挂并静止时, 重心必然在该点或其正下方。由此可以测量刚体的质心位置(两条直线确定一点)。}
