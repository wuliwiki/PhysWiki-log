% 电场波动方程
% keys 麦克斯韦方程组|波动方程|电场|介质|电位移矢量
% license Xiao
% type Tutor

\begin{issues}
\issueDraft
\issueOther{需要增加复数形式}
\end{issues}

\pentry{麦克斯韦方程组(介质)\nref{nod_MWEq1}, 矢量算符运算法则\nref{nod_VopEq}, 平面波\nref{nod_PWave}}{nod_3df3}

从麦克斯韦方程组出发,我们证明电磁场传播具有波动性。

麦克斯韦方程组为:

\begin{equation}
\begin{aligned}
& \div \bvec D = \rho ~, \\
& \div \bvec B = 0 ~, \\
& \curl \bvec E = -\frac{\partial \bvec B}{\partial t} ~, \\
& \curl \bvec H = \bvec j + \frac{\partial \bvec D}{\partial t} ~.
\end{aligned}
\end{equation}

将 $\bvec D$ 和 $\bvec H$ 转化为 $\bvec E$ 和 $\bvec B$:

\begin{equation}
\begin{aligned}
& \div \bvec E = \frac{\rho}{\epsilon} ~, \\
& \div \bvec B = 0 ~, \\
& \curl \bvec E = -\frac{\partial \bvec B}{\partial t} ~, \\
& \curl \bvec B = \mu(\bvec j + \epsilon \frac{\partial \bvec E}{\partial t}) ~.
\end{aligned}
\end{equation}

讨论在无限大各向同性均匀介质中的情况,此时 $\epsilon$ 和 $\mu$ 均为常数,并且在远离辐射源的地方,不存在自由电荷和传导电流,即 $\rho = 0$ 和 $\bvec j = 0$,此时方程组化为:

\begin{equation}
\begin{aligned}
& \div \bvec E = 0 ~, \\
& \div \bvec B = 0 ~, \\
& \curl \bvec E = -\frac{\partial \bvec B}{\partial t} ~, \\
& \curl \bvec B = \epsilon \mu\frac{\partial \bvec E}{\partial t} ~.
\end{aligned}
\end{equation}

取第3式的旋度,并将第4式代入:

\begin{equation}
\curl(\curl\bvec E) = \Nabla (\div \bvec E) - \laplacian \bvec E = -\pdv{t} (\curl\bvec B) = -\epsilon \mu \pdv[2]{t} \bvec E~.
\end{equation}

整理得:

\begin{equation}
\laplacian \bvec E - \epsilon \mu \frac{\partial^2 \bvec E}{\partial t^2} = 0 ~.
\end{equation}

令:

\begin{equation}
v = \frac{1}{\sqrt{\epsilon \mu}} ~.
\end{equation}

所以我们得到:

\begin{equation}
\laplacian \bvec E - \frac{1}{v^2} \pdv[2]{\bvec E}{t} = 0 ~.
\end{equation}

这就是电场的波动方程。同理我们可以得到磁场的波动方程。将两式列出,我们得到了\textbf{波动方程}:

\begin{equation}
\begin{aligned}
& \laplacian \bvec E - \frac{1}{v^2} \pdv[2]{\bvec E}{t} = 0 ~, \\
& \laplacian \bvec B - \frac{1}{v^2} \pdv[2]{\bvec E}{t} = 0 ~.
\end{aligned}
\end{equation}

电场的各个分量分别满足三维波动方程。%\addTODO{链接}

它的解为平面波
\begin{equation}\label{eq_EWEq_1}
\bvec E(\bvec r, t) = \bvec E_0 \cos(\bvec k\vdot \bvec r - \omega t)~,
\end{equation}
其中 $\omega = c\abs{\bvec k} = ck$。 而通解是这些平面波的任意线性组合。 注意如果 $\bvec E_0$ 中存在平行于 $\bvec k$ 的分量, 那么 $\div \bvec E \ne 0$, 所以二者必须垂直, 即 $\bvec E \vdot \bvec k = 0$。 电场的通解可表示为
\begin{equation}
\bvec E(\bvec r, t) = \int \bvec E_0(\bvec k) \cos(\bvec k\vdot \bvec r - \omega k t) \dd[3]{k}~.
\end{equation}

根据 $\curl \bvec E = -\pdv*{\bvec B}{t}$, 可求出\autoref{eq_EWEq_1} 伴随的磁场为
\begin{equation}
\bvec B(\bvec r, t) = \bvec B_0 \cos(\bvec k \vdot \bvec r - \omega t)~.
\end{equation}
其中 $\bvec B_0$ 的模长为 $\abs{\bvec B_0} = \abs{\bvec E_0}/c$, 于 $\bvec E_0$ 垂直, 方向满足 $\uvec E \cross \uvec B = \uvec k$。 可见\textbf{电磁波是横波}。

\subsection{介质中}

非线性光学中一般认为介质具有 $\mu = \mu_0$,且假设 $\div\bvec E = 0$ 仍然成立

介质中没有自由电荷或自由电流。

类似真空情况的推导过程,有
\begin{equation}
\curl(\curl\bvec E) = -\pdv{t} (\curl\bvec B) = -\mu_0 \pdv{t} (\curl\bvec H)
= -\mu_0 \pdv[2]{t} \bvec D~.
\end{equation}

把电位移矢量的定义 $\bvec D = \epsilon_0\bvec E + \bvec P$ 代入上式,化简为

\begin{equation}
\curl(\curl\bvec E) = -\pdv{t} (\curl\bvec B) = -\mu_0 \pdv{t} (\curl\bvec H)
= -\mu_0 \pdv[2]{t} \bvec D~.
\end{equation}
