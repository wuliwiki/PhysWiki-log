% 分圆域
% 分圆域|cyclotomy|域扩张|单位根|本原根|原根|primitive element

\addTODO{预备知识待确定}



$k$次\textbf{单位根}即形如$x^k-1$的多项式在$\mathbb{C}$中的根,可以理解为$1$的$k$次根.$\pm 1$都是$1$的$2^k$次根,$\pm \I$都是$1$的$4$次单位根,而$\omega=\frac{1}{2}\qty(-1+\I\sqrt{3})$是$1$的$3$次根.

本节将使用Galois理论来处理单位根及其最小多项式对有理数域的扩张,作为Galois理论的应用.


\subsection{分圆多项式}

任意$k$次单位根都形如$\exp{2\pi n/k}$,其中$n$取从$1$到$n$的全体整数,即包括了所有$k$次单位根.如$4$次单位根构成的集合是$\{1, -1, \I, -\I\}$,而这四个元素又可以两两分组:$\pm 1$是$2$次单位根\footnote{更细节些,$1$是$1$次单位根.},$\pm \I$是$4$次单位根.显然两组性质是不同的:$\mathbb{Q}(\pm 1)=\mathbb{Q}\subsetneq \mathbb{Q}(\pm\I)$.我们将这分类表述为以下定义:

\begin{definition}{本原单位根}

当$k$与$n$互素时,称单位根$\exp{2\pi n/k}$是\textbf{本原(primitive)}的.

\end{definition}

由互素的概念易知,$k$次\textbf{本原单位根}的全体幂,包含了全体$k$次\textbf{单位根};而非本原的单位根则不然.因此由分裂域的知识可知,非本原的单位根,其最小多项式不可能是$x^k-1$.为此,我们需要明确本原根所属的多项式.

\begin{definition}{分圆多项式}

设$\{\varepsilon_i\}_{i\in S}$是全体$k$次\textbf{本原单位根}的集合,则称
\begin{equation}
\Phi_k(x)\in \mathbb{C}[x] = \prod_{i\in S}(x-\varepsilon_i)
\end{equation}
为$k$次\textbf{分圆多项式(cyclotomic polynomials)}.

\end{definition}




\begin{example}{分圆多项式的例子}\label{Cycltm_ex1}

\begin{equation}
\Phi_1(x) = x-1
\end{equation}

\begin{equation}
\Phi_2(x) = x+1
\end{equation}

\begin{equation}
\Phi_3(x) = (x^3-1)/(x-1) = x^2+x+1
\end{equation}

\begin{equation}
\Phi_4(x) = (x^4-1)/(x^2-1) = x^2+1
\end{equation}

\begin{equation}
\Phi_5(x) = (x^5-1)/(x-1) = x^4+x^3+x^2+x+1
\end{equation}

注意后三个例子里,都除以了一个多项式.为什么这么做?

\end{example}

\begin{exercise}{}
请尝试算几个\autoref{Cycltm_ex1} 中未列出的分圆多项式的例子.
\end{exercise}

下面列出分圆多项式的几个基本性质.

\begin{theorem}{}
任取正整数$k, q\geq 2$,则$\Phi_k(q)\in\mathbb{R}$,且$\Phi_k(q)>q-1$.
\end{theorem}

\textbf{证明}:

首先证明$\Phi_k(x)\in\mathbb{R}[x]$.



\textbf{证毕}.













