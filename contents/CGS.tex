% 厘米—克—秒单位制
% 厘米|克|秒|单位|量纲|转换|国际单位|达因

\textbf{厘米—克—秒单位制}也叫 CGS 单位制(分别取 centimeter, gram, second 的首字母). 我们令 CGS 与 SI 单位制之间的转换常数为 $\beta_\square$, 物理量用角标 $c$ 加以区分. 例如 $x = \beta_x x_c$. 则 $\beta_x = 1\Si{cm}$, $\beta_m = 1\Si{g}$, $\beta_t = 1\Si{s}$.
为了满足 $a = \ddot x$, 有
\begin{equation}
\beta_a = \beta_x/\beta_s^2 = 1\Si{cm/s^2} = 0.01 \Si{m/s}
\end{equation}
为了满足牛顿定律
\begin{equation}
F = ma
\end{equation}
代入得
\begin{equation}
\beta_F = \beta_m \beta_x/\beta_s^2 = 1\Si{g \cdot cm/s^2} = 1\Si{dyn} = 1\times 10^{-5} \Si{N}
\end{equation}
其单位叫做\textbf{达因(dyne)}, 记作 $\Si{dyn}$.
