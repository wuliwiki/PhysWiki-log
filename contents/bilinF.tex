% 双线性型
% keys 双线性形式|bilinear form|双线性函数|bilinear function
% license Xiao
% type Tutor
\pentry{线性映射\upref{LinMap}}



给定域$\mathbb{F}$上的线性空间$V$,若函数$f:V\times V\to \mathbb{F}$对于两个自变量都满足线性性(称为“双线性性”),则称$f$为一个\textbf{双线性函数(bilinear function)}、\textbf{线性$2$-函数(linear $2$-function)}或者\textbf{双线性形式(bilinear form)},也译作\textbf{双线性型}。


所谓双线性性,即对于任意$\bvec{u}_i, \bvec{v}_j\in V$和$a_i, b_j\in\mathbb{F}$,都有
\begin{equation}
\begin{aligned}
&f(a_1\bvec{u}_1+a_2\bvec{u}_2, b_1\bvec{v}_1+b_2\bvec{v}_2)\\
={}& a_1b_1f(\bvec{u}_1, \bvec{v}_1)+a_2b_1f(\bvec{u}_2, \bvec{v}_1)+a_1b_2f(\bvec{u}_1, \bvec{v}_2)+a_2b_2f(\bvec{u}_2, \bvec{v}_2)~. 
\end{aligned}
\end{equation}

如果固定两个自变量中的一个,如固定第二个自变量$\bvec{v}$,则$f$可以看成是$V\to \mathbb{F}$的单自变量函数,显然对于这个自变量,$f$满足线性性:
\begin{equation}
f(a_1\bvec{u}_1+b_1\bvec{u}_2, \bvec{v})=a_1f(\bvec{u}_1, \bvec{v})+a_2f(\bvec{u}_2, \bvec{v})~. 
\end{equation}



%若关于矢量空间 $V$ 函数 $V\times V \to \mathbb F$ 的函数 $f(x,y)$ 满足当 $x,y$ 之一取常矢量时, $f$ 是另一个矢量的线性函数, 则 $f(x, y)$ 称为双线性的。



双线性形式是一种\textbf{张量}\upref{Tensor},具体来说是$(0, 2)$型张量。\textbf{内积}\upref{InerPd}是一种双线性形式,具体来说是\textbf{正定}的双线性形式。

















