% 科学计数法、数量级(高中)
% license Xiao
% type Tutor

\subsection{科学计数法}
\textbf{科学计数法}通常表示为
\begin{equation}\label{eq_OrdMag_1}
x \e{n}~.
\end{equation}
其中 $x$ 是一个小数, 满足 $1 \leqslant \abs{x} < 10$。 $n$ 是一个整数($n=0$ 时 $10^{0} = 1$ 可以省略不写)。

例如 $1.23\e{4} = 12300$, $1.23\e{-4} = 0.000123$。

\autoref{eq_OrdMag_1} 中的 $n$ 可以看作 $x$ 的小数点需要移动的位数。 $n > 0$ 则向右移动, $n < 0$ 则向左移动, $n=0$ 则不移动($10^{0} = 1$)。

小技巧:若保证 $1 \le \abs{x} < 10$, $x\e{-n}$ 前面一共有 $n$ 个零。 例如 $1.23\e{-4} = 0.000123$ 前面有 4 个零。

在计算机领域中, $\times 10^\square$ 一般简单表示为 \verb`e` (指数的英文是 exponent), 例如 $1.23\e{4}$ 表示为 \verb`1.23e4`,$1.23\e{4}$ 表示为 \verb`1.23e-4`(负指数无需括号)。 在非正式的书写中,这种记号也可以提高效率。

另外一些非正式文章中由于作者懒得设置上标或者排版错误,也可能会用 \verb`^` 代表上标, 用 \verb`*` 代替乘号, 例如写成 1.23*10^4。 甚至如果一些作者复制粘贴后没有检查,也可能会直接显示为 1.23*10 4。

\subsection{数量级}
\footnote{参考 Wikipedia \href{https://en.wikipedia.org/wiki/Order_of_magnitude}{相关页面}。}简单来说,我们可以认为能四舍五入到 $10^b$ 的所有数,都具有数量级 $b$ (一个整数)。 例如 $1.23\e{4}$ 和 $6.7\e{3}$ 具有数量级 $4$, 也可以说具有 $10^4$ 的数量级。 也就是 $5\e{b-1}$ 到 $5\e{b}$ 之间。

一种可能的计算方法是, 在 $10^{b-1/2}$ 到 $10^{b+1/2}$ 之间的数具有数量级 $b$。 也就是从 $\sqrt{10}\e{b-1}$ 到 $\sqrt{10}\e{b}$ 之间的数, 或者 $3.16\e{b-1}$ 到 $3.16\e{b}$ 之间。

但一般来说相邻数量级的区分并没有那么严格, 不严谨地说,$9.9\times 10^b$ 的数量级也可以认为是 $b$。
