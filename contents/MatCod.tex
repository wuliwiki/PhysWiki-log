% Matlab Coder 代码生成笔记
% license Usr
% type Note

\begin{itemize}
\item 官方文档在\href{https://ch.mathworks.com/help/coder/index.html?s_tid=CRUX_lftnav}{这里}
\item 首先要安装 Matlab Coder 工具箱
\item 当前使用 Matlab 2024b 进行测试
\item 还需要独立编译器,Windows 上这里用 MinGW,在 Home 菜单中选 \verb`Add-Ons > Get Add-Ons` 即可。这里可以安装和卸载不同官方工具箱(无需安装包),也可以装 MinGW。按照提示下载安装即可。
\item 命令行运行 \verb`mex -setup`, \verb`mex -setup C++`, \verb`mex -setup FORTRAN`。会提示如 \verb`MEX configured to use 'MinGW64 Compiler (C++)' for C++ language compilation.`
\item 现在就可以在 \verb`APPS` 菜单里面搜索 \verb`Matlab Coder`,会有 GUI 引导进行代码生成。需要选择入口函数,入口函数参数类型。
\item 使用以下简单的单文件代码测试
\begin{lstlisting}[language=matlab,caption=test1.m]
% 代码生成的入口函数,arg* 限制为字符串,ret 限制为非负整数
function ret = test1(a, b)

% 输出
% disp('hello, world');
disp('hello, world');

% 修改入参
% b = b+1;
b = b+1;

% 简单表达式
% c = a + b;
c = a + b;

ret = "";
% disp('a = '); ret = ret + 'a = ' + newline;
disp('a = '); ret = ret + 'a = ' + newline;
% disp(a); ret = ret + sprintf('%0.5e',a) + newline;
disp(a); ret = ret + sprintf('%0.5e',a) + newline;
% disp('b = '); ret = ret + 'b = ' + newline;
disp('b = '); ret = ret + 'b = ' + newline;
% disp(b); ret = ret + sprintf('%0.5e',b) + newline;
disp(b); ret = ret + sprintf('%0.5e',b) + newline;
% disp('c = '); ret = ret + 'c = ' + newline;
disp('c = '); ret = ret + 'c = ' + newline;
% disp(c); ret = ret + sprintf('%0.5e',c) + newline;
disp(c); ret = ret + sprintf('%0.5e',c) + newline;

% 判断
% if c < 0
if c < 0
    % d = b + sqrt(-c);
    d = b + sqrt(-c);
% elseif c > 0
elseif c > 0
    % d = a + sqrt(c);
    d = a + sqrt(c);
else
    % error('c = 0.');
    error('c = 0.');
end

% disp('d = '); ret = ret + 'd = ' + newline;
disp('d = '); ret = ret + 'd = ' + newline;
% disp(d); ret = ret + sprintf('%0.5e',d) + newline;
disp(d); ret = ret + sprintf('%0.5e',d) + newline;

% 循环
% for n = 1:5
for n = 1:5
    % c = c + n;
    c = c + n;
end

% disp('c = '); ret = ret + 'c = ' + newline;
disp('c = '); ret = ret + 'c = ' + newline;
% disp(c); ret = ret + sprintf('%0.5e',c) + newline;
disp(c); ret = ret + sprintf('%0.5e',c) + newline;

% 子函数调用
% e = f1(a, b);
e = f1(a, b);

% disp('e = '); ret = ret + 'e = ' + newline;
disp('e = '); ret = ret + 'e = ' + newline;
% disp(e); ret = ret + sprintf('%0.5e',e) + newline;
disp(e); ret = ret + sprintf('%0.5e',e) + newline;

% 正常退出
end

% 子函数
function z = f1(x, y)
% z = x^2 + y^2;
z = x^2 + y^2;
end
\end{lstlisting}
\item 有几个问题要注意就是,代码生成是经过优化的。 例如如果不把 \verb`a, b` 作为参数而是直接在代码内赋值,它们很可能就会被优化掉,例如直接在 c++ 生成 \verb`c = 3`。
\item 非整数的 \verb`num2str()` 提示不支持生成代码
\item \verb`disp()` 似乎会被忽略,把每个中间变量 \verb`disp()` 似乎不能防止被优化,测试中的 \verb`sprintf()` 可以, 可以猜测 \verb`printf()` 也行。
\item 在生成路径下 \verb`mex` 路径是用于测试的编译好的 mex 文件 \verb`test1_mex.mexw64` 和编译所需的 cpp 和头文件。 \verb`lib` 路径是给终端用户生成的 cpp 和头文件。
\item \verb`mex` 模式下的 cpp 源码非常类似于我们的兼容模式,例如 \verb`disp()` 是支持的,其入参的确是 \verb`bxArray`,另外两个入参应该是 m 文件中的具体行号列号,以及调用栈信息?
\item 另外做一些矩阵和线性代数运算
\begin{lstlisting}[language=matlab]
function ret = test2

% 简单矩阵操作
% h = 1e-5; ret = "";
h = 1e-5; ret = "";
% a = rand(2,2)*h+1;
a = rand(2,2)*h+1;
% ret=ret+ sprintf('sum(a) = %0.5e\n', sum(a(:)));
ret=ret+ sprintf('sum(a) = %0.5e\n', sum(a(:)));
% b = rand(2,3)*h+2;
b = rand(2,3)*h+2;
% ret=ret+ sprintf('sum(b) = %0.5e\n', sum(b(:)));
ret=ret+ sprintf('sum(b) = %0.5e\n', sum(b(:)));
% c = rand(2,5)*h+3;
c = rand(2,5)*h+3;
% ret=ret+ sprintf('sum(c) = %0.5e\n', sum(c(:)));
ret=ret+ sprintf('sum(c) = %0.5e\n', sum(c(:)));
% d = [a b; c];
d = [a b; c];
% ret=ret+ sprintf('sum(d) = %0.5e\n', sum(d(:)));
ret=ret+ sprintf('sum(d) = %0.5e\n', sum(d(:)));
% d([2,4], 1:2:5) = 4;
d([2,4], 1:2:5) = 4;
% d1 = [...]
d1 = rand(4,5)*h + ...
    [1     1     2     2     2
     4     1     4     2     4
     3     3     3     3     3
     4     3     4     3     4];
% ret=ret+ sprintf('sum(d1) = %0.5e\n', sum(d1(:)));
ret=ret+ sprintf('sum(d1) = %0.5e\n', sum(d1(:)));
% max_abs = max(abs(d(:) - d1(:)));
max_abs = max(abs(d(:) - d1(:)));
% if max_abs > 1e-4
if max_abs > 1e-4
    % ret=ret+ sprintf('error: max_abs = %0.5e\n', max_abs);
    ret=ret+ sprintf('error: max_abs = %0.5e\n', max_abs);
end

% 加减乘除、取矩阵元
% a = rand(3, 4);
a = rand(3, 4);
% ret=ret+ sprintf('sum(a) = %0.5e\n', sum(a(:)));
ret=ret+ sprintf('sum(a) = %0.5e\n', sum(a(:)));
% b = rand(3, 4);
b = rand(3, 4);
% ret=ret+ sprintf('sum(b) = %0.5e\n', sum(b(:)));
ret=ret+ sprintf('sum(b) = %0.5e\n', sum(b(:)));
% c = sqrt(a.^2 + b.^2);
c = sqrt(a.^2 + b.^2);
% ret=ret+ sprintf('sum(c) = %0.5e\n', sum(c(:)));
ret=ret+ sprintf('sum(c) = %0.5e\n', sum(c(:)));
% for i = 1:3
for i = 1:3
    % for j = 1:4
    for j = 1:4
        % if c(i, j) ~= sqrt(a(i,j)^2 + b(i, j)^2)
        if c(i, j) ~= sqrt(a(i,j)^2 + b(i, j)^2)
            ret=ret+ sprintf('error: c(i, j) = %0.5e\n', c(i, j));
        end
    end
end

% 解线性方程组
% A = rand(30,30);
A = rand(30,30);
% ret=ret+ sprintf('sum(A) = %0.5e\n', sum(A(:)));
ret=ret+ sprintf('sum(A) = %0.5e\n', sum(A(:)));
% X = rand(30,100);
X = rand(30,100);
% ret=ret+ sprintf('sum(X) = %0.5e\n', sum(X(:)));
ret=ret+ sprintf('sum(X) = %0.5e\n', sum(X(:)));
% B = A * X;
B = A * X;
% ret=ret+ sprintf('sum(B) = %0.5e\n', sum(B(:)));
ret=ret+ sprintf('sum(B) = %0.5e\n', sum(B(:)));
% X1 = A \ B;
X1 = A \ B;
% ret=ret+ sprintf('sum(X1) = %0.5e\n', sum(X1(:)));
ret=ret+ sprintf('sum(X1) = %0.5e\n', sum(X1(:)));
% if norm(X-X1) > 1e-10
if norm(X-X1) > 1e-10
    % ret=ret+ sprintf('error: norm(X-X1) = %0.5e\n', norm(X-X1));
    ret=ret+ sprintf('error: norm(X-X1) = %0.5e\n', norm(X-X1));
end

% 本征问题
% A = A + A.';
A = A + A.';
% ret=ret+ sprintf('sum(A) = %0.5e\n', sum(A(:)));
ret=ret+ sprintf('sum(A) = %0.5e\n', sum(A(:)));
% [V, D] = eig(A);
[V, D] = eig(A);
% err = max(max(abs(A*V - V*D)));
err = max(max(abs(A*V - V*D)));
% if err > 1e-10
if err > 1e-10
    % ret=ret+ sprintf('error: err = %0.5e\n', err);
    ret=ret+ sprintf('error: err = %0.5e\n', err);
end
end
\end{lstlisting}
\item 由于生成的 c++ 代码和 m 代码肉眼对应较复杂,建议每行都注释一下,注释会生成到正确的位置
\item 默认把矩阵乘法直接生成循环
\item 默认把 block 操作也都生成循环
\item 线性代数运算如 \verb`\`, 即 \href{https://www.mathworks.com/help/releases/R2024b/matlab/ref/double.mldivide.html}{mldivide},文档里面有 \textbf{C/C++ Code Generation} 字样的,就是支持直接生成源码,但可能有一些限制。
\item 从 \verb`namespace` 来看,好些代码应该是直接从 reference lapack 的源码里面摘抄过来的,剥离出来的好处就是不用把整个 lapack 编译成 dll,所以预期编译出来的 exe 体积会非常小
\begin{lstlisting}[language=matlab]
// Function Definitions
namespace coder {
namespace internal {
namespace reflapack {
double xzlarfg(int n, double &alpha1, double x[3])
\end{lstlisting}
\item 关于临时变量的优化:例如计算 \verb`A-B`,如果 \verb`A` 不再被使用,那么会直接用循环中用 \verb`A[i] -= B[i]`
\end{itemize}
