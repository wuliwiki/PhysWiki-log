% 厄米共轭算符的映射结构
% keys 线性无关|列秩|行秩|零空间|解空间|正交归一|基底
% license Xiao
% type Tutor

\pentry{矩阵与矢量空间\nref{nod_MatLS}, 正交子空间\nref{nod_OrthSp}, 线性映射的结构\nref{nod_MatLS2}, 厄米共轭\nref{nod_HerMat}}{nod_4db7}

\begin{figure}[ht]
\centering
\includegraphics[width=10cm]{./figures/35f3e5ecab1b0bb6.pdf}
\caption{\autoref{the_RCrank_1} 中的韦恩图, 三角形表示线性空间, $X_0\cap X_1$ 和 $Y_0\cap  Y_1$ 都是零向量。} \label{fig_RCrank_1}
\end{figure}

\begin{theorem}{}\label{the_RCrank_1}
令 $X, Y$ 为有限维线性空间, 线性算符 $A:X \to Y$ 的零空间为 $X_0$, 其厄米共轭\upref{HerMat}(也叫自伴算符) $A\Her: Y \to X$ 的零空间(\autoref{def_LinMap_2}~\upref{LinMap})为 $Y_0$; 令两算符的值空间(\autoref{def_LinMap_3}~\upref{LinMap})分别为 $Y_1 = A(X)$, $X_1 = A\Her(Y)$。 那么 $X_0, X_1$ 是 $X$ 中的正交补(\autoref{def_OrthSp_1}~\upref{OrthSp}), $Y_0, Y_1$ 是 $Y$ 中的正交补。 即
\begin{equation}
X = X_0 \oplus X_1~, \qquad X_0 \perp X_1~,
\end{equation}
\begin{equation}
Y = Y_0 \oplus Y_1~, \qquad Y_0 \perp Y_1~.
\end{equation}
其中 $\oplus$ 表示直和\upref{DirSum}, $\perp$ 表示两空间正交\upref{OrthSp}。
\end{theorem}
证明见下文。

\begin{corollary}{}
\autoref{the_RCrank_1} 中 $X_1, Y_1$ 维数相同, 满足(见\autoref{fig_RCrank_1} )
\begin{equation}\label{eq_RCrank_3}
A(X_1) = Y_1~, \qquad A\Her (Y_1) = X_1~,
\end{equation}
且这两个映射都是双射。
\end{corollary}
根据\autoref{the_MatLS2_1}~\upref{MatLS2}, 
由于值空间的维数只能小于或等于定义域空间的维数, 所以 $X_1, Y_1$ 维数相同。

\begin{corollary}{}\label{cor_RCrank_1}
矩阵的行秩等于列秩\upref{MatRnk}。
\end{corollary}
证明: 令矩阵 $\mat A$ 的算符为 $A$, 该矩阵的列秩就是 $Y_1$ 的维数。 而 $\mat A$ 的行秩等于 $\mat A\Her$ 的列秩, 也就是 $X_1$ 的维数。 而 $X_1, Y_1$ 维数相同。 证毕。

\subsection{证明}
以下证明\autoref{the_RCrank_1} 中 $Y_1$ 的正交补就是 $Y_0$ ($X_0, X_1$ 的证明同理)。 令 $Y_1$ 的正交补为 $Y'_0$, 那么 $y \in Y'_0$ 的充分必要条件是 $y$ 和 $Y_1$ 中所有矢量都正交, 即
\begin{equation}\label{eq_RCrank_1}
\braket{Ax}{y} = 0 \qquad (\forall x \in X)~.
\end{equation}
而根据零空间的定义 $y \in Y_0$ 的充分必要条件是 $A\Her y = 0$, 即
\begin{equation}\label{eq_RCrank_2}
\braket*{x}{A\Her y} = 0 \qquad (\forall x \in X)~,
\end{equation}
而%根据(\addTODO{链接}),
\autoref{eq_RCrank_1} 和\autoref{eq_RCrank_2} 等效。 所以 $Y'_0 = Y_0$。 证毕。
