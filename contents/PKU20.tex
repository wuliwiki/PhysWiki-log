% 北京大学 2020 年强基计划招生考试数学试题
% license Usr
% type Art


\textbf{声明}:“该内容来源于网络公开资料,不保证真实性,如有侵权请联系管理员”


%tst
北京大学 2020 年强基计划招生考试数学试题\\
选择题共20小题;在每小题的四个选项中,只有一项符合题目要求,请把正确的的代号填在表格中, 选对得 5 分, 选错或不选得 0 分.\\
1. 正实数 $x, y, z, w$ 满足 $x \geq y \geq w$ 和 $x+y \leq 2(z+w),$ 则 $\frac{w}{x}+\frac{z}{y}$ 的最小值等于\\
(A) $\frac{3}{4}$
(B) $\frac{7}{8}$
(C) 1
(D) 前三个答案都不对\\
2. 在 $(2019 \times 2020)^{2021}$ 的全体正因数中选出若干个, 使得其中任意两个的乘积都不是平方数, 则最多可选因数个数为\\
$(\mathrm{A}) 16$
(B) 31
(C) 32
(D) 前三个答案都不对\\
3. 整数列 $\left\{a_{n}\right\}_{n \geq 1}$ 满足 $a_{1}=1, a_{2}=4,$ 且对任意 $n \geq 2$ 有 $a_{n}^{2}-a_{n+1} a_{n-1}=2^{n-1},$ 则 $a_{2020}$ 的个位数字是\\
$(A) 8$
$\begin{array}{ll}\text { (B) } 4 & \text { (C) } 2\end{array}$
(D) 前三个答案都不对\\
4. 设$a, b, c, d$ 是方程 $x^{4}+2 x^{3}+3 x^{2}+4 x+5=0$ 的 4 个复根, 则 $\frac{a-1}{a+2}+\frac{b-1}{b+2}+\frac{c-1}{c+2}+\frac{d-1}{d+2}$
的值为\\
$(A)-\frac{4}{3}$
$\begin{array}{ll}\text { (B) }-\frac{2}{3} & \text { (C) } \frac{2}{3}\end{array}$
(D) 前三个答案都不对\\
5. 设等边三何形 ABC 的边长为 1, 过点 $C$ 作以 $A B$ 为直径的圈的切线交 $A B$ 的延长线与点 $D, A D>B D,$ 则三角形 $B C D$ 的面积为\\
(A) $\frac{6 \sqrt{2}-3 \sqrt{3}}{16}$
(B) $\frac{4 \sqrt{2}-3 \sqrt{3}}{16}$
(C) $\frac{3 \sqrt{2}-2 \sqrt{3}}{16}$
(D) 前三个答案都不对\\
6. 设 $x, y, z$ 均不为 $\left(k+\frac{1}{2}\right) \pi,$ 其中 $k$ 为数数, 已知$\sin (y+z-x), \sin (x+z-y), \sin (x+
y-z )$ 成等差数列, 则依然成等差数列的是\\
$(\mathrm{A}) \sin x, \sin y, \sin z$
(B) $\cos x, \cos y, \cos z$
(C) $\tan x, \tan y, \tan z$
(D) 前三个答案都不对\\
7. 方程 $19 x+93 y=4 x y$ 的整数解个数为\\
$\begin{array}{lll}\text { (A) } 4 & \text { (B) } 8 & \text { (C) } 16\end{array}$
(D) 前三个答案都不对\\
8. 从圆 $x^{2}+y^{2}=4$ 上的点向椭圆 $C: \frac{x^{2}}{2}+y^{2}=1$ 引切线, 两个切点间的线段称为切点弦, 则椭圆 $C$ 内不与任何切点弦相交的区 域面积为\\
(A) $\frac{\pi}{2}$
(B) $\frac{\pi}{3} \quad(C) \frac{\pi}{4}$
(D) 前三个答余都不对\\
9. 使得 $5x+12 \sqrt{xy}\leq a(x+y)$ 对所有正实数 $x,y$ 都成立的实数 $a$ 的最小值为\\
(A) 8
(B) 9
(C)10
(D) 前三个答余都不对\\
10. 设$P$为单位立方体 $A B C D-A_{1} B_{1} C_{1} D_{1}$ 上的一点, 则 $P A_{1}+P C_{1}$ 的最小值为\\
(A) $\sqrt{2+\sqrt{2}}$
(B) $\sqrt{2+2 \sqrt{2}} \quad(C) 2-\frac{\sqrt{2}}{2}$
(D) 前三个答案都不对\\
11. 数列 $\left\{a_{n}\right\}_{n \geq 1}$ 满足 $a_{1}=1, a_{2}=9,$ 且对任意 $n \geq 1$ 有 $a_{n+2}=4 a_{n+1}-3 a_{n}-20,$ 其前$n$项和为$S_{n}$,则函数$S_{n}$的最大值等于\\
(A) 28
(B) 35
(C) 47
(D) 前三个答案都不对\\
12. 设直线 $y=3 x+m$ 与椭圆 $\frac{x^{2}}{25}+\frac{y^{2}}{16}=1$ 交于 $A, B$ 两点, $O$ 为坐标原点, 则三角形 $O A B$ 面积的最大值为\\
(A) 8
(B) 10
(C) 12
(D) 前三个答案都不对\\
13. 正整数 $n \geq 3$ 称为理想的, 若存在正整数 $1 \leq k \leq n-1$ 使得 $C_{n}^{k-1}, C_{n}^{k}, C_{n}^{k+1}$ 构成等差数列, 其中 $C_{n}^{k}=\frac{n !}{k !(n-k) !}$ 为组合数, 则不超过 2020 的理想数个数为\\
$(\mathrm{A}) 40$
(B) 41
(C) 42
(D) 前三个答案都不对\\
14. 在 $\triangle A B C$ 中, $\angle A=150^{\circ}, D_{1}, D_{2}, \ldots, D_{2020}$ 依次为边 $B C$ 上的点, 且$\mathrm B D_{1}=$
$D_{1} D_{2}=D_{2} D_{3}=\cdots=D_{2019} D_{2020}=D_{2020} C,$ 设 $\angle B A D_{1}=\alpha_{1}, \angle D_{1} A D_{2}=\alpha_{2}, \ldots,$
$\angle D_{2019} A D_{2020}=\alpha_{2020}, \angle D_{2020} A C=\alpha_{2021},$ 则 $\frac{\sin a_{1} \sin a_{3} \cdots \sin \alpha_{2021}}{\sin a_{2} \sin \alpha_{4} \cdots \sin \alpha_{2020}}$ 的值为\\
(A) $\frac{1}{1010}$
$\begin{array}{ll}\text { (B) } \frac{1}{2020} & \text { (C) } \frac{1}{2021}\end{array}$
(D) 前三个答案降不对\\
15. 函数 $\sqrt{3+2 \sqrt{3} \cos \theta+\cos ^{2} \theta}+\sqrt{5-2 \sqrt{3} \cos \theta+\cos ^{2} \theta+4 \sin ^{2} \theta}$ 的最大值为\\
(A) $\sqrt{2}+\sqrt{3}$
(B) $2 \sqrt{2}+\sqrt{3}$
(C) $\sqrt{2}+2 \sqrt{3}$
(D) 前三个答案都不对\\
16. 方程 $\sqrt{x+5-4 \sqrt{x+1}}+\sqrt{x+2-2 \sqrt{x+1}}=1$ 的实根个数为\\
$\begin{array}{lll}\text { (A) } 1 & \text { (B) } 2 & \text { (C) } 3 & \text { (D) 前三个答案都不对 }\end{array}$\\
17. 凸五边形 $A B C D E$ 的对角线 $C E$ 分别与对角线 $B D$ 和 $A D$ 交于点 $F$ 和 $G,$ 已知 $B F: F D=5: 4, A G: G D=1: 1, C F: F G: G E=2: 2: 3, S_{\triangle C F D}$ 和 $S_{\triangle A B E}$ 分
别为 $\triangle C F D$ 和 $\triangle A B E$ 的面积, 则 $S_{\triangle C F D}: S_{\triangle A B E}$ 的值等于\\
(A) 8: 15
(B) 2: 3
(C) 11: 23
(D) 前三个答案都不对\\
18. 设 $p, q$ 均为不超过 100 的正整数, 则有有理根的多项式 $f(x)=x^{5}+p x+q$ 的个数为\\
(A) 99
(B) 133
(C) 150
(D) 前三个答案都不对\\
19. 满足对任意 $n \geq 1$ 有 $a_{n+1}=2^{n}-3 a_{n}$ 且严格递增的数列 $\left\{a_{n}\right\}_{n \geq 1}$ 的个数为\\
(A) 0
(B) 1
(C) 无穷多个
(D) 前三个答案都不对\\
20. 设函数 $f(x, y, z)=\frac{x}{x+y}+\frac{y}{y+z}+\frac{z}{z+x},$ 其中 $x, y, z$ 均为正实数, 则有\\
(A) $f$ 既有最大值也有最小值
(B) $f$ 有最大值但无最小值
(C) $f$ 有最小值但无最大值
三个答案都不对

