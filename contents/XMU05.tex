% 厦门大学 2005 年 考研 量子力学
% license Usr
% type Note

\textbf{声明}:“该内容来源于网络公开资料,不保证真实性,如有侵权请联系管理员”

\subsection{(15 分)}
简答和计算下列问题:
\begin{enumerate}
    \item 为什么表示力学量的算符必须是厄米算符?
    \item 试计算 [$\hat{L}_x ,\hat{H}$]=?其中 $\hat{L}_x$ 为轨道角动量在 $x$ 分量,$\hat{H}$ 为中心力场的哈密顿量。
    \item 设二维各向同性谐振子处于第一激发态,试写出其能级和简并度,
\end{enumerate}
\subsection{(20分)}
试在 $\hat{S}_z$ 对角的表象中:
\begin{enumerate}
    \item 矩阵 $\hat{S}_z = \frac{\hbar}{2}\begin{pmatrix} 0 & 1 \\\\ 1 & 0 \end{pmatrix}$ 的本征值和所属的本征函数.
    \item 在$\hat{S}_z$的本征值为$\frac{\hbar}{2}$的本征态中,测$\hat{S}_y$的可能值及相应几率。
\end{enumerate}
\subsection{(20分)}
用测不准关系系统地描述薛定谔方程的零点能:
$$\overline{( \Delta x )^2} \cdot \overline{( \Delta p_x )^2 }\geq \frac{\hbar^2}{4}, \quad
\varphi_n(x) = N_n e^{-\frac{\alpha^2 x^2}{2}} H_n(\alpha x)~$$
\subsection{(20分)}
设有两个质量均为 $m$,自旋为0的非全同粒子,在一维无限深势阱:
\[U(x) = \begin{cases}       \infty, & x<0,x>a, \\\\   0, & 0>x>a.    \end{cases}~\]
中运动。两个子之间的相互作用相差量 $- g \delta(x_1- x_2)$ 可作为微扰处理 (其中 $g$ 很小,是正的常数),试求准确到一级修正的体系的能量表达式。

