% C++ 多线程笔记(std::thread)
% license Xiao
% type Note

\begin{itemize}
\item 在 Linux 或其他 POSIX 系统上 \verb`std::thread` 底层通常使用 \enref{pthread}{pthred}。
\item \verb`#include <thread>`, \verb`#include <mutex>`
\item \verb`std::thread th(函数, arg1, arg2, ...)`; 创建一个线程, 调用\verb`函数`(可以是函数指针, 函数对象, lambda), \verb`arg*` 是\verb`函数`的变量。 但如果 \verb`函数` 有参数是 reference, 对应的 \verb`arg*` 需要用 \verb`std::ref(变量)` 或者 \verb`std::cref(变量)`
\item \verb`th.join()` 可以让当前线程休眠等待某个线程自己退出再让当前线程继续运行。
\item 声明一个全局变量 \verb`std::mutex m` (互斥锁,mutual exclusive lock)可以避免多个线程操作同一数据。 该类型不可以 copy 只能通过 reference 传参。
\item \verb`m.lock()` 给 mutex 上锁, 如果已经被别人上锁就暂停该线程并等待解锁。 \verb`m.try_lock()` 试图上锁, 如果已经被别人上锁就返回 \verb`false`。
\item 如果同一个线程试图给 mutex 连续上锁两次, 结果未定义, 但通常会导致该线程卡住且一直锁下去(dead lock)。
\item \verb`m.unlock()` 解锁,以便别的线程上锁。
\item 为了避免在 \verb`m.lock()` 和 \verb`m.unlock()` 之间发生 throw(这样就无法自动解锁), 通常不直接调用他们, 而是用 \verb`std::lock_guard<std::mutex> guard(m)`。 constructor 相当于 lock, destructor 相当于 unlock。
\item \verb`std::unique_lock<std::mutex>` 类似于 \verb`lock_guard` 但更灵活。 可以用 \verb`.lock()` 和 \verb`.unlock()` 多次手动上锁和解锁。
\item \verb`std::this_thread::sleep_for(std::chrono::seconds(2));` 可以让某个线程休眠若干时间
\item \verb`sleep_for` 不是 busy wait 而是像 \verb`cin` 等待时一样运行 cpu 执行别的东西,也就是让系统接管,直到某个事件发生然后再继续。
\item 相比之下, \textbf{自旋锁}就是指一个线程一直用一个空循环不断检查锁的状态直到解锁。
\end{itemize}

例程(编译时要给 linker 加上 \verb`-l pthread`):
\begin{lstlisting}[language=cpp]
#include <thread>
#include <mutex>
#include <chrono>
using namespace std;

mutex mut;

// A dummy function
void myfun(int id, int *i)
{
    if (id == 1)
        this_thread::sleep_for(chrono::milliseconds(100));
    lock_guard<mutex> guard(mut);
    printf("id = %d, i = %d\n", id, *i);
    int j = *i+1;
    if (id == 2)
        this_thread::sleep_for(chrono::milliseconds(100));
    *i = j;
}

int main()
{
    int i = 0;
    thread th1(myfun, 1, &i);
    thread th2(myfun, 2, &i);
    myfun(0, &i);
    th1.join();
    th2.join();
    return 0;
}
\end{lstlisting}

另外也可以把 \verb`mut` 声明为 \verb`myfun()` 中的一个 \verb`static` 变量。

另外注意 \verb`mutex` 不光适用于 \verb`std::thread`, 对任何其他 threading 库例如 \verb`pthread` 或者 \enref{OpenMP}{OpenMP} 都有效。

\addTODO{互斥锁、条件锁、读写锁以及自旋锁分别什么时候用?}
% 参考 https://www.zhihu.com/question/66733477

\subsubsection{thread\_local 变量}
\begin{itemize}
\item 在全局变量, 以及函数和类中的 static 变量的声明中把 \verb`static` 替换为 \verb`thread_local`(后者已经包含了前者,但 \verb`static thread_local` 也是允许的),可以让其在每个线程中保持自己独立的 copy, 使函数变得 thread safe。 例如上面的 \verb`myfun` 中如果声明了 \verb`thread_local static vector<int> a;`, 那么每个线程第一次调用该函数时就会生成一个独立的变量 \verb`a`, 若一个线程多次调用该函数, 那么每次调用时 \verb`a` 都会有上次调用结束前的值。
\end{itemize}

\subsection{std::condition\_variable}
\begin{itemize}
\item \verb`condition_variable` 用于 block 一个或者多个线程(也就是让它睡觉,腾出 CPU 算力做别的事情),直到某个另外的线程发送一个信号,可以由操作系统唤醒其中一个或全部。 每个线程被 block 以前需要检查某个条件是否满足,如果已经满足就不会被 block, 不满足就 block, 直到被某个 \verb`notify` 唤醒。 但唤醒可能是假唤醒(跟操作系统有关,即使不 notify 也可能会唤醒),所以唤醒以后还要再确认一遍条件。
\item 如果没有线程正在 \verb`wait()`, \verb`notify` 作废,不会保留到以后。
\item 每次确认条件需要使用 mutex 确保条件不会被中途改变。 所以需要检查前手动锁 mutex, \verb`wait()` 在 block 开始时会释放 mutex。 唤醒后(包括假唤醒)会重新锁 mutex 再退出 \verb`wait()`, 手动检查完条件后还可以手动对条件变量做一些更改,完了再手动释放 mutex。
\item \verb`wait()` 的第二个可选参数可以直接把一个 lambda 函数放进去,用于返回条件,这相当于把 \verb`wait()` 手动放入一个 \verb`while(条件)` 循环。
\item 一个著名的应用是 producer/consumer 问题,也就是一个队列,若干 producer 线程往里添加,若干 consumer 处理队列中的元素,处理完的元素删掉:
\end{itemize}

\begin{lstlisting}[language=cpp]
#include <iostream>
#include <thread>
#include <mutex>
#include <condition_variable>
#include <queue>

using namespace std;
mutex mtx;
condition_variable cv_producer;
condition_variable cv_consumer;
queue<int> buffer;
const unsigned int max_buffer_size = 10;
int product_id = 0;

void producer(int id) {
    for (int i = 0; i < 4; ++i) {
		unique_lock<mutex> lck(mtx); // locked
		int this_product_id = product_id++;
		cout << "+ Producer [" << id << "] producing "
            << this_product_id << "\n";
		lck.unlock();
		// do some heavy work here in parallel
        this_thread::sleep_for(chrono::milliseconds(200));
		lck.lock();
        auto wake_cond = []{ return buffer.size() < max_buffer_size; };
        // wakes when notified and wake_cond is true
        // do nothing (no waiting, no unlock) if condition is already true
        // 如果 wait() 第一次检查 wake_cond 不要求上锁,那么可能在检查到休眠之间,
        // 别人生产了一个且发送了无效的 notify,那这里就可能醒不来了。
        // 所以 buffer 必须要在 lock 的时候修改和检查
        cv_producer.wait(lck, wake_cond);
        // lock re-acquired on wake (or ramain locked)
        // followings not thread-safe without lock
        // buffer might already be full if not locked
        buffer.push(this_product_id);
        cout << "  Producer [" << id << "] produced "
            << this_product_id << "\n";
        // guaranteed waking at least one consumer (lock unnecessary)
        // do nothing is all consumers are awake
        cv_consumer.notify_one();
        // destructor unlocked loc
    }
}

void consumer(int id) {
    for (int i = 0; i < 2; ++i) {
        unique_lock<mutex> lck(mtx); // locked
        auto wake_cond = []{ return !buffer.empty(); };
        // wakes when notified and wake_cond is true
        // do nothing (no waiting, no unlock) if condition is already true
        cv_consumer.wait(lck, wake_cond);
        // lock re-acqured on wake (or ramain locked)
        // not thread-safe if unlocked
        // (buffer might be poped by others and empty if unlocked)
        int item = buffer.front();
        buffer.pop();
        cout << "- Consumer [" << id << "] received " << item << "\n";
        // loc.unlock();
        // do some heavy work here in parallel
        this_thread::sleep_for(chrono::milliseconds(200));
        // loc.lock();
        cout << "  Consumer [" << id << "] processed " << item << "\n";
        cv_producer.notify_one(); // notify one producer
        // destructor unlocked loc
    }
}

int main() {
    thread producers[2];
    thread consumers[4];

	int tid = 0;
    for (int i = 0; i < 2; ++i)
        producers[i] = thread(producer, tid++);

    for (int i = 0; i < 4; ++i)
        consumers[i] = thread(consumer, tid++);

    for (auto& p : producers) p.join();
    for (auto& c : consumers) c.join();

	cout << "final buffer size: " << buffer.size() << endl;

    return 0;
}

\end{lstlisting}

\subsection{自旋锁}
c++ 没有定义自旋锁,我们可以简单实现一个:
\begin{lstlisting}[language=cpp]
#include <atomic>
#include <thread>
#include <iostream>

class SpinLock {
private:
    std::atomic_flag lockFlag = ATOMIC_FLAG_INIT;

public:
    void lock() {
        while (lockFlag.test_and_set(std::memory_order_acquire)) {
            // Spin-wait (busy wait) until the lock is released
        }
    }

    void unlock() {
        lockFlag.clear(std::memory_order_release);
    }
};

void exampleFunction(SpinLock& spinLock) {
    spinLock.lock();
    std::cout << 123 << "234" << std::endl;
    spinLock.unlock();
}

int main() {
    SpinLock spinLock;
    std::thread t1(exampleFunction, std::ref(spinLock));
    std::thread t2(exampleFunction, std::ref(spinLock));

    t1.join();
    t2.join();

    return 0;
}
\end{lstlisting}
