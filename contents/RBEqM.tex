% 刚体的运动方程
% 刚体|运动方程|惯性张量|旋转矩阵|角动量|力矩

\pentry{刚体的平面运动方程\upref{RBEM}, 惯性张量\upref{ITensr}}

一般情况下下刚体的运动方程要比平面运动复杂许多, 但我们仍然可以将运动分解为质心的运动以及刚体绕质心的旋转, 前者由合力决定, 所以仍然有(\autoref{eq_RBEM_1}~\upref{RBEM})
\begin{equation}\label{eq_RBEqM_8}
M\bvec a_c = \sum_i \bvec F_i~.
\end{equation}
所以相对于平面运动, 该问题的困难在于绕质心转动的计算。 虽然角动量定理(\autoref{eq_AMLaw_1}~\upref{AMLaw})仍然成立, 但惯性张量\upref{ITensr} $\mat I$ 随时间的变化会使问题复杂得多。 下面我们会看到, 刚体绕任意固定点转动的角动量定理可以记为
\begin{equation}\label{eq_RBEqM_1}
\bvec\tau = \dot{\bvec L} = \bvec\omega \cross \bvec L + \mat I \dot{\bvec\omega}~.
\end{equation}
该式又被称为刚体转动的\textbf{欧拉方程(Euler's equation)}, 其中 $\bvec L = \mat I\bvec\omega$ (\autoref{eq_ITensr_3}~\upref{ITensr})是刚体的角动量, $\mat I$ 是惯性张量。 $\dot{\bvec\omega} = \dv*{\bvec\omega}{t}$ 是矢量角加速度矢量, 角速度和角加速度的关系可以类比速度和加速度\upref{VnA}。 本文中符号上方一点都表示对时间求导, 对矩阵和列矢量求导就是对每个元分别求导。  对比定轴转动的\autoref{eq_RBEM_2}~\upref{RBEM}, 转动惯量变为了惯性张量, 且多了一项 $\bvec\omega \cross \bvec L$。

\subsection{定点转动的方程}
我们接下来假设刚体可以绕坐标原点自由转动, 而原点未必是刚体的质心\footnote{例如我们考虑陀螺的运动时, 可以把它与地面的接触点作为旋转点(坐标原点), 这会比把运动分解为质心的运动和绕质心的转动更方便。 但注意此时惯性张量也必须是关于旋转点而不是关于质心的, 详见\autoref{exe_ITensr_2}~\upref{ITensr}。}。 当参考系为非惯性系时, 需要考虑惯性力带来的力矩\footnote{在平动参考系中, 如果取刚体的质心为原点, 可以证明惯性力产生的合力矩为零。}。

沿用\autoref{sub_ITensr_1}~\upref{ITensr}中的符号规范, 令体坐标系中惯性张量为 $\mat I_0$, 体坐标系到实验室坐标系的旋转矩阵为 $\mat R$, 那么 $\mat R$ 和 $\bvec \omega$ 完整描述了刚体绕原点转动的\textbf{状态}, 其中 $\mat R$ 确定了刚体每一点的位置, 而 $\bvec \omega$ 进一步确定了刚体上每一点的速度。 若已知刚体在初始时间的运动状态, 且力矩 $\bvec \tau$ 是一个关于时间和刚体运动状态的已知函数 $\bvec \tau(t, \mat R, \bvec\omega)$, 那么求解以下运动方程可以得到刚体接下来的运动
\begin{equation}\label{eq_RBEqM_6}
\dot{\bvec \omega} = \mat R \mat I_0^{-1} \mat R\Tr \qty[\bvec \tau  - \bvec\omega\cross(\mat R \mat I_0 \mat R\Tr \bvec\omega)]~,
\end{equation}
\begin{equation}\label{eq_RBEqM_4}
\dot{\mat R} = \mat\Omega \mat R~.
\end{equation}
其中
\begin{equation}\label{eq_RBEqM_5}
\mat\Omega = \pmat{0 & -\omega_z & \omega_y \\ \omega_z & 0 & -\omega_x\\ -\omega_y & \omega_x & 0}~.
\end{equation}
\autoref{eq_RBEqM_6} 和\autoref{eq_RBEqM_4} 组成一个非线性一阶常微分方程组, 写成标量的形式共有 12 条, 未知数分别为 $\omega_x, \omega_y, \omega_z$, $R_{i,j}$ 共 12 个。

事实上旋转矩阵 $\mat R$ 的 9 个矩阵元中只有三个自由度\upref{DoF}, 如果我们能用三个变量表示 $\mat R$, 就可以得到 6 元方程组, 例如使用三个欧拉角, 详见 “刚体定点旋转的运动方程(欧拉角)\upref{RigEul}”。 另一种折衷的方法是用 4 元数, 即用 4 个变量表示 $\mat R$, 可以得到相对简单的 7 元方程组, 详见 “刚体运动方程(四元数)\upref{RBEMQt}”。

\subsubsection{推导}
\pentry{旋转矩阵的导数\upref{RotDer}}

\autoref{eq_RBEqM_4} 的含义和推导见 “旋转矩阵的导数\upref{RotDer}”。 而\autoref{eq_RBEqM_6} 就是\autoref{eq_RBEqM_1} 的变形:
\begin{equation}\label{eq_RBEqM_7}
\dot{\bvec \omega} = \mat I^{-1} (\bvec\tau - \bvec\omega\cross\bvec L)~.
\end{equation}
这里的角动量要用惯性张量来计算(\autoref{eq_ITensr_3}~\upref{ITensr} 和\autoref{eq_ITensr_6}~\upref{ITensr})
\begin{equation}\label{eq_RBEqM_2}
\bvec L = \mat I \bvec \omega = \mat R \mat I_0 \mat R\Tr \bvec \omega~.
\end{equation}
其中 $\mat I_0$ 不随时间变化, $\bvec L$, $\bvec \omega$ 和 $\bvec R$ 都是时间的函数。 另外 $\mat I = \mat R \mat I_0 \mat R\Tr$ 的逆矩阵是 $\mat I^{-1} = \mat R \mat I_0^{-1} \mat R\Tr$, $\mat I_0^{-1}$ 是 $\mat I_0$ 的逆矩阵。 \autoref{eq_RBEqM_2} 代入\autoref{eq_RBEqM_7} 就是\autoref{eq_RBEqM_6}。

要证明\autoref{eq_RBEqM_1}, 从角动量定理(\autoref{eq_AMLaw_1}~\upref{AMLaw})得
\begin{equation}\label{eq_RBEqM_3}
\bvec \tau = \dot{\bvec L} = \dot{\mat I} \bvec\omega + \mat I \dot{\bvec\omega}~.
\end{equation}
其中
\begin{equation}
\begin{aligned}
\dot{\mat I}\bvec\omega &= \dv{t}(\mat R \mat I_0 \mat R\Tr)\bvec\omega
= \dot{\mat R} \mat I_0 \mat R\Tr \bvec\omega + \mat R \mat I_0 \dot{\mat R}\Tr \bvec\omega\\
&= \mat \Omega \mat R \mat I_0 \mat R\Tr \bvec\omega + \mat R \mat I_0 \mat R\Tr \mat \Omega\Tr \bvec\omega
= \mat \Omega \mat I \bvec\omega - \mat I \mat \Omega \bvec\omega\\
&= \bvec \omega \cross \bvec L - \mat I (\bvec\omega\cross\bvec\omega)
= \bvec \omega \cross \bvec L~.
\end{aligned}
\end{equation}
证毕。

\subsection{体坐标系中的定点转动方程}
\pentry{刚体的惯量主轴\upref{PrncAx}}
在体坐标系中列出转动方程往往可以简化计算。 这时往往令体坐标系的三个轴与刚体的三个主轴\upref{PrncAx}重合, 这样 $\mat I_0$ 就是三个主转动惯量构成的对角矩阵
\begin{equation}
\mat I_0 = \pmat{I_1 & 0&0 \\ 0& I_2 &0 \\ 0& 0& I_3}, \qquad
\mat I_0^{-1} = \pmat{1/I_1 & 0&0 \\ 0& 1/I_2 &0 \\ 0& 0& 1/I_3}~.
\end{equation}
令角(加)速度矢量和力矩在实验室坐标系中的坐标为
\begin{equation}\label{eq_RBEqM_10}
\begin{aligned}
&\bvec\omega_0 = \mat R\Tr \bvec\omega = (\omega_{01}, \omega_{02}, \omega_{03})~,\\
&\bvec\alpha_0 = \mat R\Tr \dot{\bvec\omega} = (\alpha_{01}, \alpha_{02}, \alpha_{03})~,\\
&\bvec\tau_0 = \mat R\Tr\bvec\tau = (\tau_{01}, \tau_{02}, \tau_{03})~.
\end{aligned}
\end{equation}
注意此时式中的 $\bvec\omega$ 和 $\dot{\bvec\omega}$ 仍然是刚体相对于实验室坐标系的角(加)速度, 只是使用了体坐标系中的基底来计算坐标。 令
\begin{equation}
\dot{\bvec\omega}_0 = \dv{\bvec\omega_0}{t} = (\dot\omega_{01}, \dot\omega_{02}, \dot\omega_{03})~,
\end{equation}
可以证明
\begin{equation}\label{eq_RBEqM_12}
\dot{\bvec\omega}_0 = \bvec\alpha_0~.
\end{equation}
又令
\begin{equation}\label{eq_RBEqM_11}
\mat\Omega_0 = \pmat{0 & -\omega_{03} & \omega_{02} \\ \omega_{03} & 0 & -\omega_{01}\\ -\omega_{02} & \omega_{01} & 0}~,
\end{equation}
那么转动方程\autoref{eq_RBEqM_6} 和\autoref{eq_RBEqM_4} 分别简化为
\begin{equation}\label{eq_RBEqM_9}
\dot{\bvec\omega}_0 = \mat I_0^{-1} (\bvec \tau_0  - \bvec\omega_0\cross \mat I_0 \bvec\omega_0)~,
\end{equation}
\begin{equation}\label{eq_RBEqM_15}
\dot{\mat R} = \mat R \mat \Omega_0~.
\end{equation}
把\autoref{eq_RBEqM_9} 写成分量形式得
\begin{equation}\label{eq_RBEqM_16}
\leftgroup{
&\dot\omega_{01} = [\tau_{01} - (I_3-I_2)\omega_{03}\omega_{02}]/I_{1}\\
&\dot\omega_{02} = [\tau_{02} - (I_1-I_3)\omega_{01}\omega_{03}]/I_{2}\\
&\dot\omega_{03} = [\tau_{03} - (I_2-I_1)\omega_{01}\omega_{02}]/I_{3}
}~.\end{equation}

\subsubsection{推导}
现在推导\autoref{eq_RBEqM_9}。 用\autoref{eq_RBEqM_10} 的定义整理\autoref{eq_RBEqM_6} 得
\begin{equation}\label{eq_RBEqM_13}
\bvec\alpha_0 = \mat I_0^{-1} \qty[\bvec \tau_0  - (\mat R\Tr\mat\Omega\mat R) \mat I_0 \bvec\omega_0]~.
\end{equation}
令
\begin{equation}\label{eq_RBEqM_14}
\mat\Omega_0 = \mat R\Tr\mat\Omega\mat R~,
\end{equation}
容易发现\footnote{推导详见\autoref{sub_CrosMt_1}~\upref{CrosMt}。} $\mat\Omega_0$ 就是 $\bvec\omega_0\cross$ 代表的矩阵(\autoref{eq_RBEqM_11})。 接下来,
\begin{equation}
\dot{\bvec\omega}_0
= \dv{t}(\mat R\Tr \bvec\omega)
= \dot{\mat R}\Tr \bvec\omega + {\mat R}\Tr \dot{\bvec\omega}
= -\mat R\Tr\mat\Omega\bvec\omega + \bvec\alpha_0~.
\end{equation}
其中 $\mat\Omega\omega = \bvec\omega\cross\bvec\omega = \bvec 0$, 这就证明了\autoref{eq_RBEqM_12}。 代入\autoref{eq_RBEqM_13} 就证明了\autoref{eq_RBEqM_9}。

最后, 结合\autoref{eq_RBEqM_4} 和\autoref{eq_RBEqM_14} 就得到\autoref{eq_RBEqM_15}。
