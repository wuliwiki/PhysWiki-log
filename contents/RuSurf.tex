% 直纹面(古典微分几何)
% ruled surface|曲率|微分几何|参数|坐标系|curvature|reparametrization

\pentry{高斯曲率和平均曲率\upref{GHcurv}}

直纹面可以说是直线和曲面的完美结合体,作为一个曲面,即使高斯曲率不为零,也能在任何一个点处找到一条直线,这条直线是曲面的子集.直纹面可以通过让一条直线在空间中扫过来构造,不过我们在这儿肯定是要给出准确的数学描述,以方便深入研究.

\begin{definition}{直纹面}
空间中的一条直线可以由两个向量组成,一个位置向量 $\alpha(t)$ 和一个方向向量 $\omega(t)$.不同的 $t$ 对应不同的直线,而每个 $t$ 对应的直线可以表示为 $\{\alpha(t)+v\omega(t)|v\in\mathbb{R}\}$.随着参数 $t$ 变化,直线的位置和方向都连续变化,也就是说,$\alpha$ 和 $\omega$ 随着 $t$ 连续变化.由此容易想到,可以使用两个参数来描述一个直纹面,将其局部坐标系写成如下形式:
\begin{equation}
\bvec{x}(t, v)=\alpha(t)+\omega(t)v
\end{equation}
\end{definition}


这里的两个参数确定直纹面上一个点的过程,可以理解为 $t$ 确定了是在哪一条直线上,而 $v$ 确定了是在直线上的哪个位置.

不失一般性地,为了方便,我们不妨设方向向量 $\omega(t)$ 恒为单位向量.这样的设置能保证 $\omega(t)\cdot\omega'(t)=0$ 恒成立.当 $\omega'(t)$ 为零的时候,得到的直纹面就被称为\textbf{柱形面(cylindrical surface)};当 $\omega'(t)$ 不为零的时候,意味着方向必然在变,得到的就是\textbf{非柱形面(noncylindrical surface)}.

直纹面上的任意一条曲线 $\beta(t)$,都可以用一个\textbf{标量函数}$u(t)$ 来唯一表示:\begin{equation}
\beta(t)=\alpha(t)+\omega(t)u(t)
\end{equation}

直纹面上每个点处都可以画出一条直线,作为曲率为零的曲率曲线.我们现在关心的是曲率不为零的曲率曲线怎么求.

设这个曲线是 $\beta(t)=\alpha(t)+\omega(t)u(t)$.由于形状算子是切空间中的线性映射,而主曲率可以视为形状算子的特征值,不属于同一特征值的向量彼此垂直,因此 $\beta(t)$ 的切向量必然处处和直线部分垂直.也就是说,我们要求有
\begin{equation}
\beta'(t)\cdot\omega'(t)=0
\end{equation}
恒成立.

代入计算并注意到 $\omega(t)\cdot\omega'(t)=0$,可得 $u$ 的表达式\begin{equation}
u=-\frac{\alpha'\cdot\omega'}{\omega'^2}
\end{equation}
由此唯一确定了 $\beta$ 的表达式.

值得一提的是,虽然 $u$ 的表达式里出现了位置向量 $\alpha$,而同一个直纹面可以由不同的位置向量移动过程来构造,但位置向量的选择并不会影响 $u$ 的表达式.




