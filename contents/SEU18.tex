% 东南大学 2018 年 考研 量子力学
% license Usr
% type Note

\textbf{声明}:“该内容来源于网络公开资料,不保证真实性,如有侵权请联系管理员”

\subsection{(共30分,时小题:分)列断題(以下叙述是否正确)}
\\(1)坐标表象中的波函数可以完全描述一个量子态,但动量空间的波函数不能完全描述一个量子态;\\
(2)若体系具有空间反演对称性,则其能量本征态一定是偶宇称态;\\
(3)对任何力学体系,只要将经典哈密顿函数中的所有动力学变量换位相应的力学量算符,就可以得到正确描述该体系量子力学效应的哈密顿算算符;\\
(4)自由粒子的宇称,能量,动量,角动量均为守恒量;\\
(5)除特殊情况外,幺正算符一般不是力学量算符;\\
(6)对于磁场中的带电粒子,薛定谔方程和所有可观测量均具有规范不变性;\\
(7)若一个体系的量子态空间未付为3,则所有可能的量子态数目为3;
(8)任何时空变换对称性均对应于一个守恒量;\\
(9)某些双原子分子的振动光谱线强度随频率的分布表现出强弱交替现象,这与全同粒子系统量子态的交换对称性或反对称性有关。\\
(10)碱金属原子光谱的双线结构与电子的自旋-轨道耦合效应有关。\\
\subsection{(共30分,每小题3分)选择题:}

1. 以下关于Pauli算符的等式,哪个是错误的:\\
   (A) $\hat{\sigma}_z^2 = 1$; \\
   (B) $\hat{\sigma}_x \hat{\sigma}_y = i\hat{\sigma}_z$; \\
   (C) $\hat{\sigma}_x \hat{\sigma}_y = -\hat{\sigma}_y \hat{\sigma}_x$; \\
   (D) $\hat{\sigma}_x \hat{\sigma}_y = \hat{\sigma}_y \hat{\sigma}_x$\\

2.设$\hat{a}^\dagger$为谐振子的上升算符,能量本征态为$|n\rangle (n=0,1,\cdots)$,则$\hat{a}^\dagger \ket{n}$等于\\
   (A) $0$;\\ 
   (B) $|n\rangle$; \\
   (C) $\sqrt{n+1}|n+1\rangle$; \\
   (D) $\sqrt{n-1}|n-1\rangle$\\

3. 设体系的基态能为$E_0$,在某个量子态下的能量平均值为$\bar{H} $,则必有\\
   (A) $\bar{H} \geq E_0$; \\
   (B) $\bar{H} \leq E_0$; \\
   (C) $\bar{H} = E_0$; \\
   (D) $\bar{H}< E_0$\\

4.设 $|\psi\rangle = \hat{A}\hat{B}|\phi\rangle$,则 $\langle \psi |$ 的表达式为\\
   (A) $\hat{B} \hat{A}\ket{\phi}$;; \\
   (B) $ \hat{A}^\dagger \hat{B}^\dagger \ket{\phi}$; \\
   (C) $\langle \phi | \hat{B}^{\dagger} \hat{A}^{\dagger}$; \\
   (D)  $\langle \phi | \hat{B} \hat{A}$\\

5. 设$\hat{A}^\dagger \hat{A}$的本征值为$\alpha$,则有\\
   (A) $\alpha<0$; \\
   (B) $\alpha>0$; \\
   (C) $\alpha=0$; \\
   (D) $\alpha>0$\\

6. 无自旋粒子在三维空间的势场中,守恒量完全集可选为:\\
   (A) $\{\hat{x}, \hat{y}, \hat{z}\}$; \\
   (B)  $\{\hat{p}_x, \hat{p}_y,\hat{p}_z\}$; \\
   (C) $\{\hat{H},\hat{l}^2,  \hat{l}_z\}$; \\
   (D) $\{\hat{l}_x, \hat{l}_y,\hat{l}_z\}$\\

7. 在动量表象中,动量算符$\hat{p}$对其本征值$\rho_0$对应的本征函数\\
   (A) $\delta(\rho-\rho_0)$; \\
   (B) $\exp(i\rho_0 x/\hbar)$; \\
   (C) $\sin(\rho_0 x/\hbar)$; \\
   (D) $\exp(\rho_0 x/\hbar)$\\

8.在坐标系 $(\hat{x}, \hat{y})$ 中,的共同本征函数为:\\
   (A) $\delta(x-x_0)e^{i\lambda y}$; \\
   (B) $\delta(x-x_0)\delta(y-y_0)$; \\
   (C) $e^{i\lambda x} \delta(y-y_0)$; \\
   (D) $e^{i\lambda x + i\mu y}$\\

9.轨道角动量为$\hat{l}=\hat{r}\times \hat{p}$,以下哪个等式是错误的\\
   (A) $[\hat{l}_x, \dot{r} \cdot \dot{p}] = 0$; \\
   (B) $[\hat{l}_x, \dot{r}^2] = 0$; \\
   (C) $[\hat{l}_r, \dot{p}^2] = 0$; \\
   (D) $[\hat{l}_x, \hat{l}_y] = 0$\\

\subsection{(共30分,每小题3分)填空题。}
\begin{enumerate}
    \item 设 $\dot{A}$ 为厄米算符, $\psi$ 为体系的任一可能状态,测运算 $A \{\psi\} = \langle \psi | A | \psi \rangle $ 在限制条件 $\langle \psi | \psi \rangle = 1$ 取极值的充分必要条件为 $(1)$。
    
    \item 考虑两个电子的自旋态空间,偶合表象与非耦合表象的基矢变换关系为:
\[
\ket{1,+1}= \ket{\uparrow\uparrow},\ket{1,-1}=\ket{\downarrow\downarrow}, |1,0\rangle = (|\uparrow\downarrow\rangle + |\downarrow\uparrow\rangle)/\sqrt{2}, \quad |0,0\rangle =(2) .~
\]
    
    \item 电子态空间的基矢取为力学量完全集$\{\hat{r}\hat{\sigma}_z\}$ 的共同本征态$r(\sigma)$,则 $r(\sigma)$ 满足$\langle r(\sigma) | r'(\sigma') \rangle =(3)$,
 
    \item 设 $\lambda=\langle \psi |\hat{A}^\dagger \hat{B} | \phi \rangle$ 其中$\hat{A}$和$\hat{B}$是两个线性算符,则$\lambda =(4)$
    
    \item 设$\ket{lm}$是角动量{$\hat{i}^2,l_z$}的共同本征态,则{$\hat{l}_x,\hat{l}_y$}.的共同本征态为 $(5)$。
    
    \item 已知空间平移算符为 $D = \exp(-ip \cdot a/\hbar)$,则 $\hat D = (6)$。
    
    \item 壁于力学中的能量-时间不确定关系可表示为$(7)$。
    
    \item 考虑正常 Zeeman 效应,一条分黄线在磁场下分裂为 $(8)$ 条光谱线。
    
    \item 体系的 3 个同类费米子组成,每个粒子可处于 3 个单粒子态中的任一个,则体系可能的量子态数目为 $(9)$。
    
    \item 设 $\psi(x)$ 是实纯态波函数,测 $\psi(\pm\infty) = 0$ 。
\end{enumerate}

\subsection{10分}
粒子处于二维各向异性谐振子势中,$V(x, y) = m(\omega_x^2 x^2 + \omega_y^2 y^2)/2$ ($\omega_x \neq \omega_y$),试求能级及其简并度。
\subsection{10分}
两个自旋均为0的无相互作用的全同玻色子处于谐振子势$V = m\omega^2 r^2 / 2$中,试求体系的最低3条能级及简并度。
\subsection{10分}
设含$z$个质子的原子核发生$\beta^-$衰变,使原子核增加1个质子而减少1个中子(核的质量保持不变)。假设衰变过程非常快,使得$1s$电子的状态保持为

    \[
    \psi^z_{100} = \left[ \sqrt{\pi}(a/z)^{3/2} \right]^{-1} \exp(-zr/a), \quad (a = \hbar^2/\mu e^2)~
    \]

    但该状态不再是新原子的能量本征态,试求该电子处于新原子的$1s$态的几率。
\subsection{10分}
设两个自旋$\frac{1}{2}$的非全同粒子体系的$\hat{H} = (\hbar \omega/2)\sigma_1 \cdot \sigma_2$。 
    \begin{enumerate}
        \item 试求体系的能级。
        \item 若$t = 0$时的量子态为$|\psi_0\rangle = \ket{\uparrow \downarrow}$,试求$t$时刻的量子态$|\psi_t\rangle$。
    \end{enumerate}
\subsection{10分}
一维无限深方势阱$ (0 < x < a) $中的粒子。设势哈密顿为:
    \[
    H' = 2\lambda x/a, \quad (0 < x < a/2), \quad H' = 2\lambda(1 - x/a), \quad (a/2 < x < a).~
    \]
    试求基态能的一级修正。
    
\subsection{10分}
粒子处于一维谐振子势 $V = m\omega^2x^2/2$ 中,归一化的基态试探波函数为
\[
\psi_{\lambda} = \left(\frac{\lambda \alpha^2}{\pi}\right)^{1/4} \exp\left(-\frac{\lambda }{2}\alpha^2 x^2\right), \quad \left(\alpha = \sqrt{\frac{m\omega}{\hbar}}\right),\int_{-\infty}^{\infty} dx \, \exp(-x^2) = \sqrt{\pi}.~
\]
其中 $\lambda$ 为变分参数。试求 $\psi_{\lambda}$ 态的能量平均值 $\bar{H}(\lambda)$,并求 $\lambda$ 的最佳值。