% 角动量的叠加(量子力学)
% license Xiao
% type Tutor
\subsection{回顾经典总角动量}
$N$ 粒子经典体系总,每个粒子对于 $O$ 点的总角动量为每一粒子对于 $O$ 的角动量矢量 $\bvec L_i=\bvec r_i\cross\bvec p_i$ 之和:
\begin{equation}
\bvec L = \sum_{i=1}^N \bvec L_i~.
\end{equation}
\begin{exercise}{思考题}
举例思考为什么在相互作用粒子的体系中,总的角动量是时间守恒量,也就是说为什么\textbf{总角动量是常量}?
\end{exercise}
\subsection{两个粒子自旋 $1/2$ 角动量的叠加}
考虑两个粒子的自旋分别为 $s_1$ 和 $s_2$,假设第二个粒子处于态 $\ket{s_1,m_1}$,第二个粒子处于态 $\ket{s_1,m_1}$。我们将\textbf{组合态(composite state)}记作 $\ket{s_1,s_2,m_1,m_2}$,于是和之前类似的有:
\begin{align}
S_{(1)}^2\ket{s_1,s_2,m_1,m_2} = \hbar^2 s_1(s_1+1)\ket{s_1,s_2,m_1,m_2}~,\\
S_{(2)}^2\ket{s_1,s_2,m_1,m_2} = \hbar^2 s_2(s_2+1)\ket{s_1,s_2,m_1,m_2}~,\\
S^{(1)}_z\ket{s_1,s_2,m_1,m_2} = m_1\hbar\ket{s_1,s_2,m_1,m_2}~,\\
S^{(2)}_z\ket{s_1,s_2,m_1,m_2} = m_2\hbar\ket{s_1,s_2,m_1,m_2}~.
\end{align}


假设氢原子处于基态(这样就不用考虑轨道角动量),其电子和质子都是自旋为 $1/2$ 的粒子。这两个粒子都可以自旋向上或者自旋向下,也就是说由四种自旋的可能性:
\begin{equation}
\chi^{(1)}_+\chi^{(2)}_+\equiv\uparrow\uparrow;\ \chi^{(1)}_+\chi^{(2)}_-\equiv\uparrow\downarrow;\ \chi^{(1)}_-\chi^{(2)}_+\equiv\downarrow\uparrow;\ \chi^{(1)}_-\chi^{(2)}_-\equiv\downarrow\downarrow~.
\end{equation}
更严格地讲,每一个粒子是处在上自旋和下自旋线性组合的状态,两个粒子构成的体系是上面
四个态的线性组合态。你应当对我们接下来的操作和推导产生怀疑,不过在介绍严格可靠的形式化之前,我们先用没有严格数学依据和不优雅的箭头做一个初步的介绍。其中 $\chi^{(1)}$ 也就是第一个箭头代表电子自旋,第二个 $\chi^{(2)}$ 代表质子自旋。

我们将这个氢原子的总角动量定义为:
\begin{equation}
\bvec S \equiv \bvec S^{(1)}+\bvec S^{(2)}~.
\end{equation}
上面的四个组合态都是 $S_z$ 的本征态,也就有:
\begin{equation}
S_z\chi_1\chi_2=(S_z^{(1)}+S_z^{(2)})\chi_1\chi_2=(S_z^{(1)}\chi_1)\chi_2+\chi_1(S_z^{(2)}\chi_2)~.
\end{equation}
\textbf{注意:}$S_z^{(1)}$ 仅作用于 $\chi_1$;$S_z^{(2)}$ 仅作用于 $\chi_2$。那么我们就得到了该系统 $S_z$ 的本征值:$m\hbar$ 中的 $m=m_1+m_2$:
\begin{align}
m=1: \ \uparrow\uparrow\equiv \ket{\frac{1}{2},\frac{1}{2},\frac{1}{2},\frac{1}{2}}~,\\
m=0: \ \uparrow\downarrow\equiv \ket{\frac{1}{2},\frac{1}{2},\frac{1}{2},\frac{-1}{2}}~,\\
m=0: \ \uparrow\downarrow\equiv \ket{\frac{1}{2},\frac{1}{2},\frac{-1}{2},\frac{1}{2}}~,\\ 
m=-1:\ \downarrow\downarrow\equiv \ket{\frac{1}{2},\frac{1}{2},\frac{-1}{2},\frac{-1}{2}}~.
\end{align}
你或许已经感到疑惑了,$m$ 不应该是从整数 $-s$ 到 $s$ 这样改变的吗?为什么会出现一个额外的 $m=0$ 的态。

那么我们先来思考如何得到 $m=0$ 的态,自然我们需要对其上一个态运用降阶算符:$S_-=S_-^{(1)}+S_-^{(2)}$ 也就是有:
\begin{equation}
S_-(\uparrow\uparrow)=(S_-^{(1)}\uparrow)\uparrow+\uparrow(S_-^{(2)})=\hbar(\uparrow\downarrow+\uparrow\downarrow)~.
\end{equation}


这样我们将得到 $s=1$ 的三个态,并将其称作三重态,因为它有对应的三个 $|s\ m\rangle$ 态分别为:
\begin{align}
&|1 \ 1\rangle =\uparrow\uparrow~,\\
&|1 \ 0\rangle=\frac{1}{\sqrt{2}}(\uparrow\downarrow+\downarrow\uparrow)~,\\
&|1 -1\rangle=\downarrow\downarrow~.
\end{align}
此外,还有一个单重态 $s=0,m=0$:
\begin{equation}
|0\ 0\rangle = \frac{1}{\sqrt{2}}(\uparrow\downarrow-\downarrow\uparrow)~.
\end{equation}
为了证实这一结论,回顾:
\begin{equation}
S^2\ket{s, m} = \hbar^2 s(s+1)\ket{s, m} ~.
\end{equation}
可知我们需要证明三重态是 $\bvec S^2$ 的本征值为 $2\hbar^2$ 所对应的本征向量。单态是 $\bvec S^2$ 的本征值为 $0$ 所对应的本征向量。根据:
\begin{equation}
S^2=(\bvec S^{(1)}+\bvec S^{(2)})\cdot (\bvec S^{(1)}+\bvec S^{(2)})=( S^{(1)})^2+( S^{(2)})^2+2\bvec S^{(1)}\cdot \bvec S^{(2)}~.
\end{equation}
根据矩阵形式所表示的 $S_x,S_y,S_z$ 可得:
\begin{align}
\bvec S^{(1)}\cdot \bvec S^{(2)}\ket{\uparrow,\downarrow}&=(S_x^{(1)}\uparrow)(S_x^{(2)}\downarrow)+(S_y^{(1)}\uparrow)(S_y^{(2)}\downarrow)+(S_z^{(1)}\uparrow)(S_z^{(2)}\downarrow)\\
&=(\frac{\hbar}{2}\downarrow)(\frac{\hbar}{2}\uparrow)+(\frac{\I\hbar}{2}\downarrow)(\frac{-\I\hbar}{2}\uparrow)+(\frac{\hbar}{2}\uparrow)(\frac{-\hbar}{2}\downarrow)\\
&=\frac{\hbar^2}{4}(2\downarrow\uparrow-\uparrow\downarrow)~.
\end{align}
同理有:
\begin{equation}
\bvec S^{(1)}\cdot \bvec S^{(2)}\ket{\downarrow,\uparrow} = \frac{\hbar^2}{4}(2\uparrow\downarrow-\downarrow\uparrow)~,
\end{equation}
也就有:
\begin{equation}
\bvec S^{(1)}\cdot \bvec S^{(2)}\ket{1,0}=\frac{\hbar^2}{4}\frac{1}{\sqrt{2}}(2\downarrow\uparrow-\uparrow\downarrow+2\uparrow\downarrow-\downarrow\uparrow)=\frac{\hbar^2}{4}\ket{1,0} ~,
\end{equation}
和
\begin{equation}
\bvec S^{(1)}\cdot \bvec S^{(2)}\ket{0,0}=\frac{\hbar^2}{4}\frac{1}{\sqrt{2}}(2\downarrow\uparrow-\uparrow\downarrow-2\uparrow\downarrow+\downarrow\uparrow)=-\frac{3\hbar^2}{4}\ket{0,0} ~,
\end{equation}
根据
\begin{equation}
\bvec S^2\chi_+=\frac{3}{4}\hbar^2\chi_+,\quad \bvec S^2\chi_-=\frac{3}{4}\hbar^2\chi_-~,
\end{equation}
可得我们最后需要的结论:
\begin{equation}
\bvec S^2\ket{1,0}=\left(\frac{3\hbar^2}{4}+\frac{3\hbar^2}{4}+\frac{2\hbar^2}{4}\right)=2\hbar^2\ket{1,0}~.
\end{equation}
因此 $\bvec S^2$ 的本征值 $2\hbar^2$ 所对应的本征态为 $\ket{1,0}$,同样还有:
\begin{equation}
\bvec S^2\ket{0,0}=\left(\frac{3\hbar^2}{4}+\frac{3\hbar^2}{4}-2\frac{3\hbar^2}{4}\right)=0\ket{0,0}~,
\end{equation}
不难证明 $\ket{1,1},\ket{1,-1}$ 也是 $S^2$ 本征值为 $2\hbar^2$ 所对应的本征态。
\subsection{普遍情况下粒子角动量的叠加}
我们在这里将尝试学会查阅克莱布斯-戈登系数表,一种“神秘学”的技巧遵循“不知道为啥”的步骤,奇迹般的得到角动量叠加的可能值和概率。如果有人问你这些魔法的步骤背后的原理,你可以告诉他:这是群论中的分解旋转群的两个不可约表示的直积为不可约表示的直和。比如,上面我们将两个自旋为 $s=1/2$ 叠加,得到了自旋 $s=1$(三重态)和 $s=0$(单重态)的过程,可以严格的被数学上的张量积形式化描述:
\begin{equation}
(s=\frac{1}{2})\otimes(s=\frac{1}{2})=(s=1)\oplus (s=0)~.
\end{equation}
当然,如果你还不满足于此请参阅角动量的叠加 2(量子力学)\upref{AMAdd}。

接下来,我们首先讨论将自旋为 $s_1$ 和 $s_2$ 的两个粒子叠加所得到的总自旋 $s$ 为:
\begin{equation}
s=(s_1+s_2),(s_1+s_2-1),(s_1+s_2-2),\cdots,|s_1-s_2|~.
\end{equation}
也就是说比如有两个粒子,一个自旋为 $2$,另一个粒子的自旋为 $3/2$,如果我们将其组合在一起可得到总自旋 $s$ 的可能值为:$7/2,5/2,3/2,1/2$.

我们还是可以大概理解这个魔法的,当两个粒子的自旋同向平行时,总自旋有最大值; 当两个粒子的自旋反向平行时, 总自旋有最小值。其他的可能值以整数 $1$ 依次递减。

那么接下来我们自然就要问道:有着总自旋 $s$ 和 $S_z$ 的本征值 $m\hbar$ 的组合态 $\ket{s,m}$ 和 $\ket{s_1,m_1}\ket{s_2,m_2}$ 之间有着什么样的联系?它们当然是线性组合在一起的:
\begin{equation}
\ket{s,m}=\sum_{m_1+m_2=m}C^{s_1s_2s}_{m_1m_2m}\ket{s_1,m_1}\ket{s_2,m_2}~.
\end{equation}
其中的常系数 $C^{s_1s_2s}_{m_1m_2m}$ 我们称之为\textbf{克莱布希-高登(Clebsch-Gordan)系数},查阅和实用它就是正在的魔法之处了。
\begin{figure}[ht]
\centering
\includegraphics[width=12cm]{./figures/eda0efa2bb085341.pdf}
\caption{例子图示} \label{fig_AdAngM_2}
\end{figure}
接下来我会按照一步步特定的步骤找到结果。首先假设有两个粒子,一个的自旋为 $s_1 = 2$ 另一个自旋为 $s_2 = 1$,那么假设它们的总自旋 $s=3$,并且 $m=0$。那么我们首先找到 $2\times 1$ 的那个表格\autoref{fig_AdAngM_2}  。然后在最上面的方格上找到 $3\quad 0$ 的那高亮的一列,那些分数就是我们所要找的系数(记得带上根号)。
现在我们就有了:
\begin{equation}
\ket{3,0}=\frac{1}{\sqrt{5}}\ket{2,m_1}\ket{1,m_2}+\sqrt{\frac{3}{5}}\ket{2,m_1}\ket{1,m_2}+\frac{1}{\sqrt{5}}\ket{2,m_1}\ket{1,m_2}~,
\end{equation}
然后我们看到最左侧的小方格就是 $m_1,m_2$ 分别对应(系数)概率的可能值。最后得到:
\begin{equation}
\ket{3,0}=\frac{1}{\sqrt{5}}\ket{2,1}\ket{1,-1}+\sqrt{\frac{3}{5}}\ket{2,0}\ket{1,0}+\frac{1}{\sqrt{5}}\ket{2,-1}\ket{1,1}~.
\end{equation}
也就是说,如果我们测量 $S_z^{(1)}$ 得到 $\hbar$ 的概率为 $1/5$,得到 $0$ 的概率为 $3/5$,得到 $-\hbar$ 的概率为 $1/5$;测量 $S_z^{(2)}$ 得到 $-\hbar$ 的概率为 $1/5$,得到 $0$ 的概率为 $3/5$,得到 $\hbar$ 的概率为 $1/5$。

\begin{exercise}{思考题}
为什么我们不去测一测 $S_x,S_y$?提示:思考它们是否和 $S^2$ 对易?
\end{exercise}

现在我们换个问题:如果有自旋为 $s_1=3/2$ 和自旋为 $s_2= 1 $ 的粒子,而且知道第一个粒子的 $ m_1 =1/2$ ,第二个
粒子的 $ m_2 =0$,那么我们测量总自旋将会得到什么概率和可能值?

实际上这个表格还可以适用于逆展开:
\begin{equation}
\ket{s_1,m_1}\ket{s_2,m_2}=\sum_{s}C^{s_1s_2s}_{m_1m_2m}\ket{s,m}~.
\end{equation}
与之前的步骤类似,不过这里我们就不能再继续看到最上面的方格,而是要查阅\textbf{左侧方格}中有 $1/2\quad 0$ 的那高亮一\textbf{行}。\autoref{fig_AdAngM_2} 
由此我可以得到:
\begin{equation}
\ket{\frac{3}{2},1,\frac{1}{2},0}=\sqrt{\frac{3}{5}}\ket{\frac{5}{2},\frac{1}{2}}+\frac{1}{\sqrt{15}}\ket{\frac{3}{2},\frac{1}{2}}-\frac{1}{\sqrt{3}}\ket{\frac{1}{2},\frac{1}{2}}~.
\end{equation}
因此测量总自旋 $s$,得到 $5/2$ 的概率为 $3/5$,或得到 $3/2$ 的概率为 $1/15$,$1/2$ 的概率为 $1/3$。在克莱布希-高登系数表中
任意一行的平方和为(概率之和)当然为 $1$。
\begin{figure}[ht]
\centering
\includegraphics[width=14.25cm]{./figures/f28e01477f8df53e.pdf}
\caption{\textbf{CG系数表格},注意:被夹着的方格子的每一个项都要带根号,负号放在根号外面} \label{fig_AdAngM_1}
\end{figure}

当然如果你懒得查表,那你可以使用python直接得出,比如下面是测量 $s_1=3/2,m_1=1/2,s_2=1,m_2=0$ 得到 $s=1/2,m=1/2$ 的概率为数值的 $-1/\sqrt{3}$.
\begin{lstlisting}[language=python]
from sympy.physics.quantum.cg import CG
from sympy import S
CG(j1=S(3)/2, m1=S(1)/2, j2=S(1), m2=S(0), j3=1/2, m3=1/2).doit()
\end{lstlisting}

\subsection{自旋角动量和轨道角动量的叠加}
等等。你应该要感到疑惑。对于给定的这两个粒子,其自旋角动量与空间波函数无关,而其轨道角动量与空间波函数无关。
