% 中山大学 2019 年研究生入学物理考试试题
% keys 中山大学|考研|物理|2019年
% license Copy
% type Tutor


\textbf{声明}:“该内容来源于网络公开资料,不保证真实性,如有侵权请联系管理员”
\subsection{简答题}
\begin{enumerate}
\item “牛顿第二定律”的含义是什么?它的基础是什么?
\item  Mayer 公式$C_P-C_v=R>0$,请用热力学第一定律解释它的物理意义。
\item 长条形的电介质在外电场中极化后,两端出现等量异号电荷。若把它截成两半后并撤去外电场问,这两个半截的电介质是否带电?为什么?
\item 为什么窗玻璃在日光照射下我们观察不到于涉条纹?
\end{enumerate}
\subsection{计算题}
\begin{enumerate}
\item 有一半圆形的光滑槽,质量为$ M$,半径为$R$,放在光滑的水平面上。一个小物体质量为 $m$,可以在槽内自由滑动。开始时半圆槽静止,小物体静止于$A$处,如图所示。试求:\\
(1)当小物体滑到C点处($\theta$角)时,小物体$m$相对于槽和槽相对于地的速度的大小。\\
(2)当小物体滑到最低点$B$时,槽移动的距离$S_1$。
\begin{figure}[ht]
\centering
\includegraphics[width=8cm]{./figures/502836025938d905.png}
\caption{} \label{fig_SY19_1}
\end{figure}
\item 一定量的单原子分子理想气体,从初态$A$ 出发,沿图示直线过程变到另一状态$B$,又经过等容、等压两过程回到状态$ A$。求:\\
(1)$A-B$,$B-C$,$C-A$ 各过程中系统对外所作的功$W $,内能增量及所吸收的热量$Q$。\\
(2)该热机的效率。
\item 两圆圈半径为 $R$,平行地共轴放置,圆心$O_1,O_2$相距为$a$,所载电流均为$I$,且电流方向相同。\\
(1)以$O_1O_2$连线的中点为原点 $O$,求轴线上坐标为$x$的任一点处磁感应强度。\\(2)试证明:当$a=R$时,$O$点处的磁场最为均匀(这样放置的一对线圈称为亥姆霍兹线圈,常用它获得近似均匀的磁场)。
\item 在空气中以白光垂直照射到厚度为$d$且均匀的肥皂膜上后反射,在可见光谱中观察到$\lambda_1=6300A$的干涉极大$\lambda_2=5250A$的干涉极小,且它们之间没有另外的干涉极小,求肥皂膜的厚度$d$为多少?(肥皂膜的折射率 $n=1.33$)
\end{enumerate}
\subsection{实验题}
\begin{enumerate}
\item 已知$P=x+y-z$,直接测量量的测量结果为$x=\bar x \pm \delta_{\bar x},y=\bar y \pm \delta_{\bar y},z=\bar z \pm \delta_{\bar z},$。求$P$的结果、标准误差和相对误差。
\item 下表是水的表面张力系数随温度变化的数据及差分表。试用牛顿内插公式求13.2°C时的表面张力系数。
\begin{figure}[ht]
\centering
\includegraphics[width=12cm]{./figures/8116244e7b2adc1b.png}
\caption{} \label{fig_SY19_2}
\end{figure}
\end{enumerate}