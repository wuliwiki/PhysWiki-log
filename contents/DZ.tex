% 电子
% license CCBYSA3
% type Wiki

(本文根据 CC-BY-SA 协议转载自原搜狗科学百科对英文维基百科的翻译)

\textbf{电子}是一种亚原子粒子,符号是$e^-$或者$\beta^-$,其电荷量是负一个单位的基本电荷。[1]电子属于第一代轻子族,且通常被认为是基本粒子,因为它们没有已知的成分或亚结构。电子的质量大约是质子质量的1/1836。 电子的量子力学性质包括半整数值的内禀角动量(自旋),单位为约化普朗克常数ħ。根据泡利不相容原理,作为费米子,没有两个电子可以占据相同的量子态。像所有基本粒子一样,电子表现出波粒二象性:它们可以与其他粒子发生碰撞,也可以像光一样发生衍射。比其他粒子(例如中子和质子)更容易通过实验观察到,因为电子的质量更低,对于给定的能量电子拥有更大的德布罗意波长。

电子在许多物理现象中起着重要作用,如电学、磁学、化学和热导率,电子还参与重力、电磁相互作用和弱相互作用。[2]因为电子带有电荷,所以它周围有一个电场,如果电子相对于某个观察者发生运动,这个观察者将观察到它产生一个磁场。其他来源所产生的电磁场将影响电子的运动,这种影响由洛伦兹力定律所描述。当电子被加速时,它们以光子的形式辐射或吸收能量。实验室仪器能够通过使用电磁场囚禁单个电子和电子等离子体。特殊的望远镜可以探测外层空间的电子等离子体。电子涉及许多应用,例如电子学、焊接、阴极射线管、电子显微镜、放射疗法、激光器、气体电离探测器和粒子加速器。

电子与其他亚原子粒子的相互作用在化学和核物理等领域都很有意义。原子核内带正电荷的质子和没有原子核的负电荷电子之间的库仑力相互作用允许它们共同组成原子。电离或负电荷电子与正电荷原子核之间的电荷量差异改变了原子系统的结合能。两个或多个原子之间的电子交换或共享是化学键形成的主要原因。1838年,英国自然哲学家理查德·拉明首先假设了一个不可分割的电荷量的概念来解释原子的化学性质。爱尔兰物理学家乔治·约翰斯顿·斯通尼在1891年将这种电荷命名为“电子”,而约瑟夫·汤姆孙和他的英国物理学家团队在1897年将它确定为粒子。电子也可以参与核反应,例如恒星中的核合成,在其中电子被称为贝塔粒子。电子可以通过放射性同位素的β衰变以及高能碰撞产生,后者的一个例子发生在宇宙射线进入大气层的时候。电子的反粒子被称为正电子;它与电子拥有很多相同的性质,除了它携带有与电子相反的电荷以及其他的荷。当一个电子与正电子碰撞时,两个粒子可以发生湮灭,并产生$\gamma$射线光子。

\subsection{历史}
\subsubsection{1.1 电力效应的发现}
古希腊人注意到琥珀在摩擦皮毛后会吸引小物体。与闪电一样,这种现象是人类最早记录的电学经验之一。[3]在1600年的论文《De Magnete》中,英国科学家威廉·吉尔伯特创造了新拉丁语术语electrica来指代那些性质类似琥珀、摩擦后会吸引小物体的物质。[4]electric和electricity这两个词都源自拉丁语ēlectrum(这也是同名合金的名称由来),来源于希腊语中的“琥珀”ἤλεκτρον(ēlektron)。
\subsubsection{1.2 两种电荷的发现}
17世纪初,法国化学家Charles Franç ois du Fay发现如果带电的金箔被用丝绸摩擦过的玻璃所排斥,那么同样的带电金箔就会被用毛皮摩擦过的琥珀所吸引。从这个和其他类似实验的结果中,他得出结论:电由两种电流体所组成,分别是来自被丝绸摩擦过的玻璃的玻璃液体以及来自被毛皮摩擦过的琥珀的树脂液体。这两种流体混合时可以互相中和。[4][5]美国科学家埃比尼泽·金纳斯利后来也独立得出了同样的结论。[6]十年后本杰明·富兰克林提出电不是来自不同类型的电流体,而是表现出过量(+)或不足(-)的单一电流体。他将这两种情况分别命名为正和负,这也是现代的电荷命名规则。[7]富兰克林认为电荷载体是正的,但他没有正确地识别哪种情况是电荷载体的盈余、哪种情况是不足。[8]

1838年至1851年间,英国自然哲学家理查德·拉明发展了原子由被亚原子粒子包围的物质核心所构成的观点,这些包围核心的亚原子粒子具有单位数量的电荷。[9]从1846年开始,德国物理学家威廉·韦伯建立了电流由带正电荷和负电荷的流体组成的理论,它们的相互作用受平方反比定律支配。在1874年研究了电解现象后,爱尔兰物理学家乔治·约翰斯顿·斯通尼提出存在“单一确定量的电”,即单价离子的电荷。借助于法拉第电解定律,他能够估算出这个基本电荷的数值e。[10]然而,斯通尼认为这些电荷永久地附着在原子上并且无法移除。1881年,德国物理学家赫尔曼·赫尔姆霍茨认为正电荷和负电荷都被分成基本部分,每个都“表现得像电的原子”。[11]

斯通尼在1881年创造了术语electrolion。十年后,他转而用electron来描述这些基本电荷,他在1894年写道:“...有人估计了这个最引人注目的基本电学单位的实际量,后来我大胆地提出了这个名字电子(electron)”。1906年有人提议改用electrion一词,但这个提议没有成功,这是因为亨德里克·洛伦兹更愿意继续使用electron这个词。[11][12]electron这个词是单词electric和ion的组合。[13]来自electron的后缀-on现在被用来为其他亚原子粒子命名,如质子或中子。[14][15]
\subsubsection{1.3 物质外自由电子的发现}
\begin{figure}[ht]
\centering
\includegraphics[width=8cm]{./figures/921ff5a58456bdae.png}
\caption{被磁场偏转成圆形的电子束[15]} \label{fig_DZ_17}
\end{figure}
约瑟夫·汤姆孙对电子的发现与许多物理学家几十年来对阴极射线的实验和理论研究密切相关。[11]在1859年研究稀薄气体的电导率时,德国物理学家朱利叶斯·普吕克尔观察到由阴极发射的辐射引起的磷光出现在阴极附近的管壁上,并且磷光区域可以通过施加磁场来移动。1869年,普吕克的学生希托夫发现置于阴极和磷光体之间的固体会在管的磷光区域上投下阴影。希托夫推断阴极发出笔直的射线,而磷光是由撞击管壁的射线所引起的。1876年,德国物理学家尤金·戈德斯坦证明这些射线被垂直于阴极表面发射出来,这区分了从阴极发射的射线和白炽光。戈德斯坦称这种射线为阴极射线。[16] [17]

19世纪70年代,英国化学家和物理学家威廉·克鲁克斯爵士开发了第一个内部具有高度真空的阴极射线管。[18]接着,他在1874年证明阴极射线可以转动放在它们的路径上的一个小桨轮。因此,他得出结论:这种射线带有动量。此外,通过施加磁场,他能够偏转这种射线,从而证明这种射线表现得好像带了负电荷。[17]1879年,他提出这些性质可以通过将阴极射线视为由处于第四种物质状态的带负电的气态分子组成来解释,在这里粒子的平均自由程太长以致于它们之间的碰撞可以被忽略。[16]

德国出生的英国物理学家阿瑟·舒斯特扩展了克鲁克斯的实验,他将金属板相对阴极射线平行放置,并在金属板之间施加电势。电场将射线偏转向带正电荷的板上,这进一步证明了这种射线带有负电荷。通过测量给定水平的电流的偏转量,舒斯特在1890年就能够估计出射线成分质荷比。然而,这产生了一个比预期大一千多倍的数值,所以当时他的计算很少得到认可。[17]

1892年,亨德里克·洛伦兹提出这些粒子(电子)的质量可能是它们电荷的结果。[19]

法国物理学家昂利·贝可勒尔在1896年研究自然发荧光矿物时发现,它们在没有暴露于外部能源的情况下就会发出辐射。这些放射性材料成为科学家们非常感兴趣的主题,这其中就包括后来发现它们所发射粒子的本质的新西兰物理学家欧内斯特·卢瑟福。他根据这些粒子穿透物质的能力将它们命名为α粒子和β粒子。[20]1900年,贝克勒尔证明了镭发出的β射线可以被电场偏转,并且它们的质荷比与阴极射线相同。[21]这一证据支持了电子作为原子成分存在的观点。[22][23]
\begin{figure}[ht]
\centering
\includegraphics[width=6cm]{./figures/1dbcc8852c33c51a.png}
\caption{约瑟夫·汤姆孙} \label{fig_DZ_1}
\end{figure}
1897年,英国物理学家约瑟夫·汤姆孙和他的同事约翰·汤森和H. A .威尔逊进行的实验表明阴极射线确实是一种独特的粒子,而不是以前认为的波、原子或分子。[24]汤姆森对这种粒子的电荷e和质量m都做了很好的估计,他发现阴极射线粒子(他称之为“微粒”)的质量可能是已知质量最小的离子(氢离子)的千分之一。[24]他证明了这种粒子的荷质比e/m不取决于阴极材料。他进一步证明,放射性材料、被加热材料和照明材料所产生的负电荷粒子是相同的。[24][24]科学界将这些粒子命名为电子,这主要是因为G. F. 菲茨杰拉德、J. 拉莫和H. A. 洛伦兹的倡导。[25]
\begin{figure}[ht]
\centering
\includegraphics[width=6cm]{./figures/9a283a42b0af6b49.png}
\caption{罗伯特·密立根} \label{fig_DZ_2}
\end{figure}
美国物理学家罗伯特·密立根和哈维·弗莱彻在他们1909年的油滴实验中更仔细地测量了电子的电荷,其结果发表于1911年。在这个实验中,他们使用电场来阻止带电的油滴因重力而下落。该装置可以测量少至1-150个离子的电荷,误差小于0.3\%。汤姆森的团队早些时候也做过类似的实验,[24]他们利用了电解产生的带电水滴云。1911年亚伯兰·约菲使用带电金属微粒独立地获得了与密立根相同的结果,并在1913年发表了他的结果。[26]然而,油滴比水滴更稳定,因为它们的蒸发速度更慢,所以更适合长时间的精确实验。[27]

大约在二十世纪初,人们发现,在某些条件下,快速移动的带电粒子会导致过饱和水蒸气沿其路径发生凝结。1911年,查理·威尔逊利用这一原理设计了他的威尔逊云雾室,这样他就可以拍摄带电粒子的轨迹,例如快速移动的电子。[28]
\subsubsection{1.4 原子理论}
\begin{figure}[ht]
\centering
\includegraphics[width=8cm]{./figures/d736d3f77e6f2f50.png}
\caption{玻尔原子模型,显示了能量由n所量子化的电子状态。下降到较低轨道的电子发射的光子的能量等于轨道之间的能量差。} \label{fig_DZ_3}
\end{figure}
到1914年,物理学家欧内斯特·卢瑟福、亨利·莫塞莱、詹姆斯·弗兰克和古斯塔夫·赫兹的实验已经在很大程度上确立了原子的结构,即被低质量电子包围的带正电荷的致密核。[29]1913年,丹麦物理学家尼尔斯·玻尔假设电子处于量子化的能量状态,其能量由电子绕核轨道的角动量所决定。通过发射或吸收特定频率的光子,电子可以在这些状态或轨道之间移动。通过这些量子化的轨道,他准确地解释了氢原子的谱线。[30]然而,玻尔模型未能解释谱线的相对强度,也未能解释更复杂的原子的光谱。[29]

吉尔伯特·牛顿·路易斯解释了原子之间的化学键,他在1916年提出两个原子之间的共价键是由它们之间共享的一对电子所维持的。[31]后来,在1927年,沃尔特·海特勒和弗里茨·伦敦为电子对形成和化学键给出了完整的量子力学解释。[32]1919年,美国化学家欧文·朗缪尔阐述了路易斯的原子静态模型,并提出所有电子都分布在连续的“(接近)同心的厚度相等的球形壳层中”。[33]接着,他将壳分成许多单元,每个单元包含一对电子。利用这个模型,朗缪尔能够定性地解释元素周期表中所有元素的化学性质,[32]根据周期定律,它们在很大程度上重复自身。[34]

1924年,奥地利物理学家沃尔夫冈·泡利观察到,原子的壳状结构可以用一组定义每个量子能量状态的四个参数来解释,只要每个状态被不超过一个电子占据。这种禁止一个以上的电子占据相同量子能量态的禁令被称为泡利不相容原理。[35]荷兰物理学家塞缪尔·古德施密特和乔治·乌伦贝克提供了解释第四个参数的物理机制,该参数有两个不同的可能值。1925年,他们提出电子除了轨道角动量之外,还拥有一个内禀角动量和磁偶极矩。[29][36]这类似于地球绕着太阳转动的同时也绕着自身的轴旋转。内禀角动量被称为自旋,它解释了以前用高分辨率光谱仪观察到的神秘的谱线分裂,这种现象被称为精细结构分裂。[37]
\subsubsection{1.5 量子力学}
\begin{figure}[ht]
\centering
\includegraphics[width=6cm]{./figures/7b9fbb20551d0820.png}
\caption{在量子力学中,原子中电子的行为由轨道所描述,轨道是概率分布而不是轨迹。在图中,阴影表示在该点“找到”具有对应于给定量子数的能量的电子的相对概率。} \label{fig_DZ_4}
\end{figure}
在1924年的论文《Recherches sur la théorie des quanta》(量子理论研究)中,法国物理学家路易·德布罗意假设所有物质都可以用类似光的方式被表示为德布罗意波。也就是说,在适当的条件下,电子和其他物质将显示出粒子或波的性质。粒子的微粒性质表现在任何给定时刻沿着其轨道它都在空间中具有一个局域的位置。[38]光的波动性质则表现在穿过平行狭缝进而产生干涉图案的时候。1927年,乔治·佩杰特·汤姆孙发现当电子束穿过薄金属箔时存在干涉效应。美国物理学家克林顿·戴维孙和雷斯特·革末在电子从镍晶体反射的过程中也观察到了干涉效应。[39]

德布罗意对电子的波动性质的预测导致埃尔温·薛定谔假设了电子在原子核影响下运动的波动方程。1926年,这个被称为薛定谔方程的方程成功地描述了电子波是如何传播的。[40]这个波动方程不是产生一个确定电子随时间变化的位置的解,而是可以用来预测在某个位置附近找到电子的概率,特别是在电子被束缚在空间中的位置附近,在这种情况下电子的波动方程不随时间变化。这种方法导致了量子力学的第二种表述(海森堡在1925年提出了第一种),薛定谔方程的解像海森堡方程一样提供了对氢原子中的电子能量状态的推导,这些状态跟玻尔在1913年首次推导得到的结果是一样的,并且已知可以再现氢光谱。[41]一旦自旋和多个电子间相互作用被描述出来,量子力学就可以预测原子序数大于氢的原子中的电子构型。[42]

1928年,在沃尔夫冈·泡利的工作的基础上,保罗·狄拉克通过对电磁场的量子力学的哈密顿表述应用相对论和对称性考虑而创建了一个电子模型,这就是遵循相对论的狄拉克方程。[43]为了解决他的相对论方程中的一些问题,狄拉克在1930年建立了一个真空模型,它是一个负能量粒子的无限海,后来被称为狄拉克海。这使他预测了正电子的存在,后者是与电子相对应的反物质。[44]这种粒子是由卡尔·安德森于1932年发现的,安德森提议将普通电子命名为negatons,而使用electron作为描述带正电荷和负电荷的变体的通用术语。

1947年威利斯·兰姆与研究生罗伯特·雷瑟福德合作发现氢原子的某些本应具有相同能量的量子态相对于彼此发生了位移,这种差异被称为兰姆位移。大约在同一时间,波利卡普·库施与亨利·福利一起发现电子的磁矩略大于狄拉克理论的预测值。这个微小的差异后来被称为电子的反常磁矩。这种差异后来被朝永振一郎、朱利安·施温格和理查德·费曼于20世纪40年代末所发展的量子电动力学所解释。[45]
\subsubsection{1.6 粒子加速器}
随着20世纪上半叶粒子加速器的发展,物理学家开始更深入地研究亚原子粒子的性质。[46]1942年,唐纳德·克尔斯特首次成功地尝试使用电磁感应加速电子。他最初的β加速器的能量达到2.3MeV,而随后的β加速器则达到300MeV。在1947年,同步辐射是用通用电气的一台70MeV的电子同步加速器被发现的。这种辐射是由电子以接近光速移动时穿过磁场的加速度引起的。[47]

第一台高能粒子对撞机是于1968年开始运行的ADONE,粒子束能量为1.5GeV。[48]这种设备在相反的方向上加速了电子和正电子,与用电子撞击静态目标相比,它有效地将粒子碰撞的能量增加了一倍。[49]欧洲核子研究中心(CERN)于1989年至2000年运行的大型电子正子对撞机(LEP)的碰撞能量达到209GeV,物理学家利用它对粒子物理标准模型进行了重要测量。[50][51]
\subsubsection{1.7 囚禁单个电子}
单个电子现在可以很容易地被限制在超小($L = 20 nm,W = 20 nm$)的CMOS晶体管中,后者运行于-269摄氏度(4K)到大约258摄氏度(15K)的低温下。[52]电子波函数分散于半导体晶格中,与价带电子的相互作用很小,因此可以通过将质量替换成有效质量张量而作为单粒子来处理。

\subsection{特征}
\subsubsection{2.1 分类}
\begin{figure}[ht]
\centering
\includegraphics[width=10cm]{./figures/b5c187a00339db95.png}
\caption{基本粒子的标准模型。电子(符号e)在左边。} \label{fig_DZ_5}
\end{figure}
在粒子物理学的标准模型中,电子属于被称为轻子的亚原子粒子,它们被认为是基本粒子。在所有带电轻子(或任何类型的带电粒子)中,电子的质量是最低的,属于第一代基本粒子。[53]第二代和第三代包含带电轻子($\mu$子和$\tau$子),它们与电子拥有相同的电荷、自旋和相互作用,但质量更大。轻子不同于物质的其他基本成分(即夸克),因为它们不参与强相互作用。轻子群的所有成员都是费米子,因为它们都有半奇整数的自旋。电子的自旋为12。[54]
\subsubsection{2.2 基本性质}
电子的不变质量近似为$9.109\times10^{-31}$ 千克,[55]或者$5.489\times10^-4$原子质量单位。根据爱因斯坦的质能公式,这个质量对应于0.511 MeV的静止能量。质子和电子的质量之比约为1836。[56][56]天文测量表明,正如标准模型预测的那样,质子电子质量比至少在宇宙年龄一半的时间内保持相同的值。[57]

电子的电荷量为$-1.602\times10^{-19}$ 库仑,[55]这被用作亚原子粒子的标准电荷单位,也称为基本电荷。这个基本电荷的相对标准不确定度为$2.2\times10^{-8}$。[55]在实验精度的限制内,电子的电荷等于质子的电荷,但符号相反。[58]由于e被用于表示基本电荷,电子通常被写成$e^-$,其中的负号代表负电荷。正电子的符号是$e^+$,因为它具有与电子相同的性质,但带有正电荷而不是负电荷。[54][55]

电子具有内禀角动量或自旋,大小为12。[55]由于这一性质,我们通常将电子称为一种自旋-12粒子。[54]这样的粒子的自旋大小是√32$\hbar$。而对任何轴上的自旋投影的测量结果只能是$\pm\hbar2$。除了自旋,电子还拥有沿着其旋转轴的内禀磁矩。[55]它大约等于一个玻尔磁子,[59]后者是一个等于$9.27400915(23)\times10^{-24}$ joules/ tesla的物理学常数。[55]自旋相对于电子动量的方向定义了一个称为螺旋度的基本粒子性质。[60]

电子没有已知的亚结构[61][61],并且它被认为是没有空间范围的带有点电荷的点粒子。[62]

电子半径问题是现代理论物理中一个具有挑战性的问题。承认电子半径有限的假设不符合相对论的前提。另一方面,点状电子(半径为零)使得电子的自能趋于无穷大,这会导致严重的数学困难。[62]对彭宁离子阱中一个单独电子的观察表明电子的半径的上限为$10^{-22}$米。[63]电子半径的上限$10^{-18}$米[64]可以由能量的不确定关系导出。还存在一个叫“经典电子半径”的物理学常数,它的值为$2.8179\times10^{-15} m$,大于质子的半径。这个术语来源于一个忽略了量子力学效应的简单计算;在现实中,所谓的经典电子半径与电子真正的基本结构没有什么关系。[65]

有一些基本粒子可以自发地衰变成质量较小的粒子。其中一个例子是$\mu$子,它们的平均寿命为$2.2\times10^{-6}$秒,可以衰变为电子$\mu$子中微子和电子反中微子。另一方面,从理论上来说,电子被认为是稳定的:电子是带有非零电荷的质量最小的粒子,因此它的衰变将违反电荷守恒定律。[66]电子平均寿命的实验下限是$6.6\times10^{28}$年,置信度为90\%。[67][68][69]
\subsubsection{2.3 量子性质}
如同所有粒子一样,电子可以表现出波的性质。这被称为波粒二象性,可以用双缝实验来证明。

电子的波动性质允许它同时穿过两个平行的狭缝,而不是像经典粒子那样只穿过一个狭缝。在量子力学中,一个粒子的类波性质可以在数学上描述为一个复值函数,即波函数,后者通常用希腊字母psi($\Psi$)来表示。当取该函数的绝对值的平方时,便得到了在某个位置附近观察到该粒子的概率——概率密度。[70]
\begin{figure}[ht]
\centering
\includegraphics[width=8cm]{./figures/36805e7d38d91a05.png}
\caption{一维盒中两个全同的费米子,量子态的反对称波函数的例子。如果粒子交换位置,波函数反转其符号。} \label{fig_DZ_6}
\end{figure}
电子是全同粒子,因为它们不能通过内禀的物理性质相互区分。在量子力学中,这意味着一对相互作用的电子必须能够交换位置而系统状态没有可观察到的变化。包括电子在内的费米子的波函数是反对称的,这意味着当两个电子交换时,波函数会改变符号,也就是说,$\Psi(r_1, r_2) = -\Psi(r_2, r_1)$,其中变量$r_1$和$r_2$分别对应于第一个电子和第二个电子。因为绝对值不会因符号变化而改变,所以这对应着相等的概率。玻色子(例如光子)则具有对称波函数。[70]

在反对称的情况下,电子相互作用的波动方程的解导致每对电子占据相同的位置或状态的概率为零。这是造成泡利不相容原理的原因,它阻止了任何两个电子占据相同的量子态。这个原理解释了电子的许多性质。例如,它导致束缚电子群在原子中占据不同的轨道,而不是在同一轨道上相互重叠。[70]
\subsubsection{2.4 虚粒子}
\begin{figure}[ht]
\centering
\includegraphics[width=8cm]{./figures/8c81f15fd70d547c.png}
\caption{在电子(左下方)附近随机出现的虚拟电子-正电子对的示意图} \label{fig_DZ_7}
\end{figure}
在一个简化的图像中,每个光子在一段时间内变成虚电子加上它的反粒子(虚正电子)的组合,它们在此后不久迅速彼此湮灭。[71]产生这些粒子所需的能量变化和它们存在时间的组合低于由海森堡不确定性原理$\delta E \delta t \ge \hbar$所表示的可检测性阈值。实际上,创造这些虚粒子所需的能量$\delta E$可以从真空中被“借用”一段时间$\delta t$,它们的乘积不超过约化普朗克常量($\hbar\approx 6.6\times10^{-16}$ eV·s)。因此,对于虚电子,$\delta t$最多是1.$3\times10^{-21} s$。[72]

当虚电子-正电子对存在时,来自电子周围电场的库仑力会使一个产生出来的正电子被原来的电子吸引,而一个产生出来的电子则会受到排斥。这导致了所谓的真空极化。实际上,真空就像一种相对介电常数超过1的介质。因此,电子的有效电荷实际上小于其真实值,并且有效电荷随着离电子距离的增加而减少。[73][74]这种极化在1997年由日本的TRISTAN粒子加速器实验所证实。[75]虚粒子对电子质量产生相似的屏蔽效应。[76]

与虚粒子的相互作用也解释了电子内禀磁矩与玻尔磁子之间约0.1\%的微小偏差(反常磁矩)。[59][77]这种预测值与实验值非常精确的吻合被认为是量子电动力学的伟大成就之一。[78]

在经典物理学中,作为点粒子的电子具有内禀角动量和磁矩这一点看起来好像是一个悖论,但这可以用虚光子在电子产生的电场中形成来解释。这些光子导致电子以抖动的方式四处移动(称为zitterbewegung),[79]这导致具有进动的净圆周运动。这种运动产生了电子的自旋和磁矩。[62][80]在原子中,虚光子的产生解释了在谱线中观察到的兰姆位移[73]。
\subsubsection{2.5 相互作用}
电子产生的电场对带正电荷的粒子(如质子)施加吸引力,对带负电荷的粒子施加排斥力。这种力在非相对论近似下的强度由库仑平方反比定律决定。[81]当电子运动时,它产生一个磁场。[70]安培-麦克斯韦定律将磁场与电子相对于观察者的大规模运动(电流)联系起来。感应的这一特性提供了驱动电动机的磁场。[82]任意移动带电粒子的电磁场由黎纳-维谢势表示,即使粒子的速度接近光速(相对论性的),它也是有效的。[81]
\begin{figure}[ht]
\centering
\includegraphics[width=6cm]{./figures/60ad796aaf803d50.png}
\caption{带电荷q的粒子(在左边)以速度v通过指向读者的磁场B。对于电子来说,q是负的,所以它沿着弯曲的轨迹向上移动。} \label{fig_DZ_8}
\end{figure}
当电子在磁场中移动时,它受到垂直于磁场和电子速度所定义的平面的洛伦兹力。这种向心力使电子沿着螺旋轨迹穿过磁场,其半径称为回转半径。这种弯曲运动的加速度诱导电子以同步辐射的形式辐射能量。[70][83]能量发射反过来导致电子的反冲,这种力被称为“亚伯拉罕-洛伦兹-狄拉克力”,它产生了减慢电子速度的摩擦。这个力是由电子自身的场对其自身的逆反应引起的。[84]
\begin{figure}[ht]
\centering
\includegraphics[width=6cm]{./figures/922ae329d9b2b5b8.png}
\caption{这里,轫致辐射是由电子e被原子核的电场偏转所产生的。能量变化E2 − E1决定了发射的光子的频率f。} \label{fig_DZ_9}
\end{figure}
在量子电动力学中,光子介导了粒子之间的电磁相互作用。恒速的孤立电子不能发射或吸收实光子,这样做会违反能量守恒和动量守恒。相反,虚光子可以在两个带电粒子之间传递动量。例如,虚光子的交换产生了库伦力。[85]移动的电子被带电粒子(如质子)偏转时会发射能量。电子的加速导致轫致辐射的发射。[86]

光子和孤立的(自由)电子之间的非弹性碰撞被称为“康普顿散射”。这种碰撞导致粒子之间的动量和能量转移,并改变光子的波长,其量被称为康普顿位移。这个波长偏移的最大幅度是h/mec,被称为康普顿波长。[87]对于电子来说,康普顿波长的值为$2.43\times10^{-12}$ m。[55]当光的波长较长时(例如,可见光的波长为0.4-0.7 微米),波长偏移变得可以忽略不计。光和自由电子之间的这种相互作用被称为汤姆孙散射或线性汤姆孙散射。[88]

两个带电粒子(如电子和质子)之间电磁相互作用的相对强度由精细结构常数给出。该值是由两种能量的比值形成的无量纲量:距离为一个康普顿波长时的静电吸引(或排斥)能量,以及电荷的静止能量。它的值为$\alpha \approx 7.297353\times10^{-3}$,大约等于1137。[55]

当电子和正电子碰撞时,它们彼此湮灭,产生两个或更多的伽马射线光子。如果电子和正电子的动量可以忽略不计,那么在湮灭产生总计能量为1.022MeV的两个或三个伽马射线光子之前,可以形成一个电子偶素。[89][90]另一方面,高能光子可以通过一种叫做成对产生的过程转换成一个电子和一个正电子,但这种转换必须在附近有带电粒子(例如原子核)存在的情况下才能进行。[91][92]

在电弱相互作用理论中,电子波函数的左手分量与电子中微子一起形成一个弱同位旋二重态。这意味着在弱相互作用中电子中微子的行为类似于电子。这个二重态的任何一个成员都可以通过发射或吸收W 玻色子来参与带电流相互作用并被转化成另一个成员。电荷在这个反应中是守恒的,因为W玻色子也携带电荷并抵消了相互作用过程中的任何净变化。带电流相互作用是放射性原子中β衰变现象的原因。电子和电子中微子都可以通过交换Z0 玻色子来参与中性流相互作用,这就是中微子-电子弹性散射的原因。[93]
\subsubsection{2.6 原子和分子}
\begin{figure}[ht]
\centering
\includegraphics[width=8cm]{./figures/2c1e4bce63ce4191.png}
\caption{从截面看,最初几个氢原子轨道的概率密度。束缚电子的能级决定了它占据的轨道,颜色反映了在给定位置找到电子的概率。} \label{fig_DZ_10}
\end{figure}
电子可以通过库伦吸引力被束缚于原子核周围。一个或多个电子被束缚在原子核上的系统被称为原子。如果电子的数量不同于原子核的电荷,那么这样的原子就叫做离子。束缚电子的波动行为由一个叫做原子轨道的函数所描述。每个轨道都有自己的一组量子数,如能量、角动量和角动量的投影,这些轨道的集合中只有一个离散集存在于原子核周围。根据泡利不相容原理,每个轨道最多可以被两个电子占据,它们的自旋量子数必须不同。

电子可以通过发射或吸收能量与势能差相匹配的光子在不同轨道之间转移。[94]轨道转移的其他方法包括与粒子(如电子)的碰撞以及俄歇效应。[95]为了逃离原子,电子的能量必须增加到高于其与原子的结合能。例如,这发生在光电效应中,在其中能量超过原子电离能的入射光子被电子吸收。

电子的轨道角动量是量子化的。因为电子是带电的,所以它产生与角动量成正比的轨道磁矩。原子的净磁矩等于所有电子和原子核的轨道磁矩和自旋磁矩的矢量和。与电子相比,原子核的磁矩可以忽略不计。占据相同轨道的电子(所谓的成对电子)的磁矩相互抵消。[96]

正如量子力学定律所描述的,原子之间的化学键是电磁相互作用的结果。[97]最强的键是通过原子之间的电子共享或转移形成的,从而允许形成分子。[98]在一个分子中,电子在几个原子核的影响下移动,并占据分子轨道,非常类似于它们可以占据孤立原子中的原子轨道。[98]这些分子结构中的一个基本因素是电子对的存在。这些电子具有相反的自旋,允许它们占据相同的分子轨道而不违反泡利不相容原理(很像原子中的情况)。不同的分子轨道具有不同的电子密度空间分布。例如,在成键对(即实际上将原子结合在一起的电子对)中,电子可以在原子核之间相对较小的区域中以最大概率被发现。相比之下,在非成键对中,电子分布在原子核周围的大区域中。[99]
\subsubsection{2.7 传导性}
\begin{figure}[ht]
\centering
\includegraphics[width=6cm]{./figures/420f1808375ea3e4.png}
\caption{闪电放电主要由电子流组成。[53]闪电所需的电势可以由摩擦电效应产生。[54][55]} \label{fig_DZ_11}
\end{figure}
如果一个物体有比平衡原子核正电荷所需的更多或更少的电子,那么该物体就有净电荷。当电子过量时,物体被称为带负电。当电子数量少于原子核中质子的数量时,该物体被称为带正电荷。当电子和质子的数量相等时,它们的电荷相互抵消,物体被称为电中性。宏观物体可以通过摩擦(摩擦电效应)产生不为零的净电荷。[100]

在真空中运动的独立电子被称为自由电子。金属中的电子也表现得好像是自由的一样。实际上,金属和其他固体中通常被称为电子的粒子是准电子——一种准粒子,它们与真实电子具有相同的电荷、自旋和磁矩,但质量可能不同。[101]当真空和金属中的自由电子移动时,它们产生一种电荷的净流动,称为电流,电流产生磁场。同样,变化的磁场也可以产生电流。麦克斯韦方程组用数学方法描述了这些相互作用。[102]

在给定温度下,每种材料具有电导率,当施加电势时,该电导率决定电流值。良导体的例子包括铜和金等金属,而玻璃和特氟隆则是不良导体。在任何电介质材料中,电子保持与它们各自的原子结合,材料表现为绝缘体。大多数半导体具有介于导电和绝缘两个极限之间的可变导电水平。[103]另一方面,金属具有包含部分填充电子带的电子带结构。这种能带的存在使金属中的电子表现得像是自由的或离域电子。这些电子与特定的原子没有关联,因此当施加电场时,它们可以像气体一样自由移动(称为费米气体)[104],就像自由电子一样。

由于电子和原子之间的碰撞,导体中电子的漂移速度为每秒几毫米的数量级。然而,材料中某一点的电流变化引起材料其他部分电流变化的速度(即传播速度)通常约为光速的75\%。[105]这是因为电信号以波的形式传播,速度取决于材料的介电常数。[106]

金属是相对较好的导热体,主要是因为离域电子可以在原子之间自由传输热能。然而,与电导率不同,金属的热导率几乎与温度无关。这在数学上由维德曼–弗兰兹定理表示,[104]其中指出热导率与电导率的比值正比于温度。金属晶格中的热无序增加了材料的电阻率,产生了电流的温度依赖性。[107]

当冷却到低于一个叫做临界温度的点时,材料会经历一个相变,在这个相变中,它们会失去对电流的所有电阻,这个过程被称为超导性。在 BCS理论中,被称为库柏对的电子对通过被称为声子的晶格振动将其运动耦合到附近的物质上,从而避免与通常产生电阻的原子发生碰撞。[108](库珀对的半径约为100nm,所以它们可以相互重叠)[109]然而,高温超导体的工作机制仍然不确定。

导电固体内部的电子本身是准粒子,当它们在接近绝对零度的温度下被紧密限制时,表现得好像它们已经分裂成另外三个准粒子:自旋子,轨道子和空穴子。[110][111]自旋子携带自旋和磁矩,轨道子携带其轨道位置,而空穴子携带电荷。
\subsubsection{2.8 运动和能量}
\begin{figure}[ht]
\centering
\includegraphics[width=6cm]{./figures/b79deb6a7e1c9f6e.png}
\caption{洛仑兹因子是速度的函数。它从1开始,随着v趋于c它趋于无穷大。} \label{fig_DZ_12}
\end{figure}
根据爱因斯坦的狭义相对论,当电子的速度接近光速时,从观察者的角度来看,它的相对论质量增加,从而使得在观察者的参照系内加速它变得越来越困难。电子的速度可以接近但永远达不到真空中的光速$c$。然而,当相对论性电子(即速度接近$c$的电子)被注入到电介质(例如水)中时,介质中的局部光速明显小于$c$,因此电子在介质中可以暂时比光传播得快。当它们与介质相互作用时,会产生一种微弱的光,这种光被称为契伦科夫辐射。[112]
狭义相对论的效应基于一个被称为洛仑兹因子的量,其定义为$\gamma=1/ \sqrt{1-v^2/c^2}$  ,其中$v$是粒子的速度。以速度运动$\nu$移动的电子的动能$K_e$是:
$$K_e=(\gamma-1)m_ec^2~$$
其中$m_e$是电子的质量。例如斯坦福线性加速器将电子加速到大约51GeV。[113]由于电子表现为波,在给定的速度下,它有一个特征德布罗意波长,它由$\lambda e = h/p$给出,这里$h$是普朗克常量,$p$是动量。[114]对于上面提到的51GeV电子来说,其波长约为$2.4\times10^{-17}$ m,小到足以探索远小于原子核尺寸的结构。[114]

\subsection{形成}
\begin{figure}[ht]
\centering
\includegraphics[width=8cm]{./figures/4ad0394e275acb0f.png}
\caption{电子和正电子的成对产生,由光子接近原子核引起。闪电符号代表虚光子的交换,因此电力在这里起作用。粒子之间的角度非常小。[1]} \label{fig_DZ_13}
\end{figure}
大爆炸理论是解释宇宙早期演化阶段的最被广泛接受的科学理论。[115]在大爆炸的第一毫秒,宇宙温度超过一百亿开尔文,光子的平均能量超过一百万电子伏特。这些光子能量足够大,可以相互反应形成电子和正电子对。同样,正电子-电子对相互湮灭并发射高能光子:
$$\gamma +\gamma\leftrightarrow e^+ +e^-~$$
在宇宙演化的这个阶段,电子、正电子和光子之间保持了平衡。然而,15秒钟后,宇宙的温度降至可能发生电子正电子形成的阈值之下。大多数幸存的电子和正电子相互湮灭,释放出伽马射线,短暂地重新加热了宇宙。[116]

由于仍然不确定的原因,在湮没过程中,粒子的数量超过了反粒子。因此,大约每十亿个电子-正电子对中就有一个电子存活下来。这种过量与质子相对于反质子的过量相匹配,在一种被称为重子不对称的条件下,导致宇宙的净电荷为零。[117][118]幸存的质子和中子开始与彼此相互反应——在称为“核合成”的过程中,形成氢和氦的同位素,并含有痕量的锂。这个过程在大约五分钟后达到顶峰。[119]任何剩余的中子都经历了半衰期大约为1000秒的负$\beta$衰变,在如下过程中释放出质子和电子,
$$n \to p +e^- +\nu_e~$$
在接下来的大约300000–400000 年间,过量的电子仍然能量过大以致于无法与原子核结合。[120]接下来是一个被称为复合的时期,这时中性原子形成,膨胀的宇宙对电磁辐射变得透明。[121]

大爆炸大约一百万年后,第一代恒星开始形成。[121]在恒星内部,恒星核合成导致原子核融合产生正电子。这些反物质粒子立即与电子湮灭并释放伽马射线。最终结果是电子数量稳定减少,而中子数相应增加。然而,恒星演化过程可以导致放射性同位素的合成。选定的同位素随后可以经历负β衰变,从原子核中发射电子和反中微子。[122]一个例子是钴60 ($^{60}Co$)同位素,其衰变形成镍-60 ($^{60}Ni$)。[123]
\begin{figure}[ht]
\centering
\includegraphics[width=8cm]{./figures/1400f9e856096e8f.png}
\caption{高能宇宙射线撞击地球大气层产生的大规模空气簇射} \label{fig_DZ_14}
\end{figure}
在其生命周期结束时,拥有超过大约20个太阳质量的恒星可以经历重力坍塌形成黑洞。[124]根据经典物理学,这些大质量恒星物体施加的引力足够强大,可以阻止任何东西(甚至电磁辐射)逃离史瓦西半径。然而,量子力学效应被认为潜在地允许在这个距离发射霍金辐射。电子(和正电子)被认为是在这些恒星残骸的事件视界中产生的。

当一对虚粒子(如电子和正电子)在事件视界附近产生时,随机空间定位可能导致其中一个出现在外部,这个过程被称为量子隧穿。黑洞的重力势可以提供能量,将虚粒子转化为实粒子,允许它辐射到空间中。[125]作为交换,这对粒子中的另一个成员被赋予负能量,这导致黑洞质量能量的净损失。霍金辐射的速率随着质量的降低而增大,最终导致黑洞蒸发,直到最后的爆炸。[126]

宇宙线是以高能量穿越空间的粒子。高达3.0×1020 eV的能量事件已经被记录下来了。[127]当这些粒子与地球大气层中的核子碰撞时,会产生一连串的粒子,包括$\pi$介子[128]。从地表观测到的宇宙辐射有一半以上由$\mu$介子组成。被称为$\mu$子的粒子是由$\pi$介子衰变在高层大气中产生的轻子。
$$\pi^- \to \mu^- +\nu_\mu~$$
$\mu$子反过来可以衰变形成电子或正电子。[129]
$$\mu^- \to e^- +\nu_e +\nu_\mu~$$

\subsection{观察}
\begin{figure}[ht]
\centering
\includegraphics[width=6cm]{./figures/97766292690f9942.png}
\caption{极光主要是由高能电子进入大气层中所引起的。[1]} \label{fig_DZ_15}
\end{figure}
远程观察电子需要检测它们的辐射能量。例如,在高能量环境中(如恒星的日冕),自由电子形成等离子体并通过轫致辐射过程辐射能量。电子气体可以经历等离子振荡,这是由电子密度的同步变化引起的波,它们产生能量发射,可以通过使用射电望远镜来探测.[130]

光子频率与它的能量成正比。当束缚电子在原子的不同能级之间跃迁时,它吸收或发射拥有特征频率的光子。例如,当原子被广谱辐射源照射时吸收谱线出现在透射辐射的光谱中。每个元素或分子都显示一组特征谱线,例如氢光谱系列。对这些谱线的强度和宽度的光谱学测量可以用来确定物质的组成和物理性质。[131][132]

在实验室条件下,单个电子的相互作用可以通过粒子探测器来观察,这允许测量特定的性质,例如能量、自旋和电荷。[133]保罗阱和彭宁离子阱的发展让我们能够将带电粒子长时间地囚禁在小区域内。这使得我们可以精确测量粒子的特性。例如,在一个例子中,彭宁离子阱被用来囚禁一个电子长达10个月之久。[133]电子的磁矩在1980年被测量到11位数的精度,这比任何其他物理常数都更精确。[134]

2008年2月,瑞典隆德大学的一个团队捕获了第一批电子能量分布的视频图像。科学家们使用了极短的闪光(称为阿秒脉冲),这使得电子的运动首次被观察到。[135][136]

固体材料中电子的分布可以通过角分辨光电发射光谱 (ARPES)来实现可视化。这种技术利用光电效应来测量倒易空间——一种用于推断原始结构的周期结构的数学表示。ARPES可用于确定材料中电子的方向、速度和散射。[137]

\subsection{等离子体应用}
\subsubsection{5.1 粒子束}
\begin{figure}[ht]
\centering
\includegraphics[width=6cm]{./figures/35bb8c483402a2e9.png}
\caption{在美国宇航局期间 风洞试验中,航天飞机的模型以电子束为目标,模拟了离子在重返期间使气体电离的效果。[1]} \label{fig_DZ_16}
\end{figure}
电子束被用于焊接。[138]它们允许高达$10^7$ W·cm$^{-2}$的能量密度穿过0.1–1.3 mm的狭窄的聚焦直径,并且通常不需要填充材料。这种焊接技术必须在真空中进行,以防止电子在到达目标之前与气体相互作用,并且它可以用于连接导电材料,后者在其他情况下被认为不适合焊接。[139][140]

电子束曝光(EBL)是一种分辨率小于微米的半导体蚀刻方法。[141]这种技术受到成本高、性能慢、需要在真空中操作电子束以及电子在固体中散射的趋势的限制。最后一个问题将分辨率限制在10nm左右。因此,EBL主要被用于生产少量特种集成电路。[142]

电子束处理被用于照射材料以改变它们的物理性质,或者医疗用品和食品消毒。[143]电子束流化或准熔化玻璃,在强辐射下温度没有显著升高:例如,强电子辐射导致粘度降低许多个数量级,活化能逐步降低。[144]

线性粒子加速器在放射治疗中产生用于治疗浅表肿瘤的电子束。电子疗法可以治疗诸如基底细胞癌的皮肤损伤,因为电子束在被吸收之前仅穿透有限的深度,通常在5-20MeV的电子能量范围内穿透距离高达5厘米。电子束可以用来补充对已经被X光照射的区域的处理。[145][146]

粒子加速器使用电场来推动电子及其反粒子达到高能。这些粒子通过磁场时会发出同步辐射。这种辐射强度对自旋的依赖性极化了电子束——这一过程被称为索科洛夫-特诺夫效应。极化电子束可以用于各种实验。同步加速器辐射也可以冷却电子束,以减少粒子的动量扩散。电子束和正电子束在被加速到所需能量时发生碰撞,粒子探测器观察由此产生的能量发射,这是粒子物理学的研究对象。[147]
\subsubsection{5.2 成像}
低能电子衍射(LEED)是一种用准直电子束轰击晶体材料然后观察所得衍射图样以确定材料结构的方法。电子所需的能量通常在20-200eV的范围内。[148]反射高能电子衍射(RHEED)技术使用以各种低角度发射的电子束的反射来表征晶体材料的表面。光束能量通常在8-20keV的范围内,入射角是1-4°。[149][150]

电子显微镜将聚焦的电子束导向样品。当电子束与材料相互作用时,一些电子会改变它们的性质,例如运动方向、角度、相对相位和能量。显微镜学家可以记录电子束中的这些变化,以产生材料的原子分辨率图像。[151]在蓝光下,传统的光学显微镜具有大约200nm的衍射极限分辨率。[152]相比之下,电子显微镜受到电子的德布罗意波长的限制。例如,对于在100,000伏特的电势上加速的电子来说,这个波长等于0.0037nm。[153]透射电子像差校正显微镜能够有低于0.05nm的分辨率,足以分辨单个原子。[154]这种能力使得电子显微镜成为高分辨率成像的有用的实验室仪器。然而,电子显微镜是维护成本很高的昂贵仪器。

电子显微镜主要有两种类型:透射电子显微镜和扫描电子显微镜。透射电子显微镜的功能类似于投影仪,电子束穿过一片材料,然后由透镜投射到摄影幻灯片或电荷耦合器件上。扫描电子显微镜像在电视机中一样,利用一束精细聚焦的电子通过被研究的样品产生图像。两种显微镜的放大倍数都在100倍至1,000,000倍或更高的范围内。扫描隧道显微镜使用电子从尖锐的金属尖端到研究材料的量子隧穿效应,并可以产生材料表面的原子分辨率图像。[155][156][157]
\subsubsection{5.3 其他应用}
在自由电子激光器(FEL)中,相对论性电子束穿过一对包含场指向交替方向的偶极磁体阵列的波动器。电子发射同步辐射,后者相干地与原来的电子发生相互作用,以共振频率强烈地放大辐射场。自由电子激光可以发射相干的高亮度电磁辐射,广阔的频率范围从微波一直到软X射线。这些设备被用于制造、通信和医疗应用中,例如软组织手术。[158]

电子在阴极射线管中很重要,阴极射线管已广泛用作实验室仪器、计算机屏幕和电视机的显示设备。[159]在光电倍增管中,撞击光阴极的每个光子都会引发电子雪崩,从而产生可检测的电流脉冲。[160] 真空管利用电子流操作电信号,在电子技术的发展中起着至关重要的作用。然而,它们在很大程度上已经被诸如晶体管之类的固态器件所取代。[161]

\subsection{笔记}
\begin{itemize}
\item The fractional version's denominator is the inverse of the decimal value (along with its relative standard uncertainty of 4.2×10−13 u).
\item The electron's charge is the negative of elementary charge, which has a positive value for the proton.
\item 这个量级是从自旋量子数获得的
\item 对于量子数$s=12$。
\item 请参见:Gupta, M.C. (2001). Atomic and Molecular Spectroscopy. New Age Publishers. p. 81. ISBN 978-81-224-1300-7.
\item Bohr magneton:$$\mu_B=\frac{e\hbar}{2m_e}~$$
\item 经典电子半径推导如下。假设电子的电荷均匀分布在整个球形体积中。因为球体的一部分会排斥其他部分,所以球体包含静电势能。假设该能量等于电子的静止能量,由狭义相对论定义(E = 喊麦$^2$)中。
根据静电学理论,半径为的球体的势能$r$充电$e$由下式给出:$$E_p=\frac{e^2}{8\pi\varepsilon_0 r}~$$在哪里$\varepsilon_0$是真空介电常数。对于一个和静止质量在一起的电子$m_0$,剩余能量等于:$$E_p=m_0c^2~$$在哪里$c$真空中的光速。让他们平等并解决$r$给出了经典的电子半径。
\item 来自非相对论性电子的辐射有时被称为回旋辐射。
\item 波长的变化$\delta\lambda$,取决于反冲的角度$\theta$,如下所示:$$\Delta\lambda=\frac{h}{m_ec}(1-cos\theta)~$$在哪里$c$真空中的光速$m_e$是电子质量。见Zombeck (2007: 393,396)。
\item The polarization of an electron beam means that the spins of all electrons point into one direction. In other words, the projections of the spins of all electrons onto their momentum vector have the same sign.
\end{itemize}

\subsection{参考文献}
[1]
^Coff, Jerry (2010-09-10). "What Is An Electron". Retrieved 10 September 2010..

[2]
^Anastopoulos, C. (2008). Particle Or Wave: The Evolution of the Concept of Matter in Modern Physics. Princeton University Press. pp. 236–237. ISBN 978-0-691-13512-0..

[3]
^Shipley, J.T. (1945). Dictionary of Word Origins. The Philosophical Library. p. 133. ISBN 978-0-88029-751-6..

[4]
^Benjamin, Park (1898), A history of electricity (The intellectual rise in electricity) from antiquity to the days of Benjamin Franklin, New York: J. Wiley, p. 315, 484–5, ISBN 978-1313106054.

[5]
^Keithley, J.F. (1999). The Story of Electrical and Magnetic Measurements: From 500 B.C. to the 1940s. IEEE Press. pp. 19–20. ISBN 978-0-7803-1193-0..

[6]
^Cajori, Florian (1917). A History of Physics in Its Elementary Branches: Including the Evolution of Physical Laboratories. Macmillan..

[7]
^"Benjamin Franklin (1706–1790)". Eric Weisstein's World of Biography. Wolfram Research. Retrieved 2010-12-16..

[8]
^Myers, R.L. (2006). The Basics of Physics. Greenwood Publishing Group. p. 242. ISBN 978-0-313-32857-2..

[9]
^Farrar, W.V. (1969). "Richard Laming and the Coal-Gas Industry, with His Views on the Structure of Matter". Annals of Science. 25 (3): 243–254. doi:10.1080/00033796900200141..

[10]
^Barrow, J.D. (1983). "Natural Units Before Planck". Quarterly Journal of the Royal Astronomical Society. 24: 24–26. Bibcode:1983QJRAS..24...24B..

[11]
^Arabatzis, T. (2006). Representing Electrons: A Biographical Approach to Theoretical Entities. University of Chicago Press. pp. 70–74, 96. ISBN 978-0-226-02421-9..

[12]
^Stoney, G.J. (1894). "Of the "Electron," or Atom of Electricity". Philosophical Magazine. 38 (5): 418–420. doi:10.1080/14786449408620653..

[13]
^"电子,n.2 "。在线牛津英语词典。2013年3月。牛津大学出版社。2013年4月12日访问[2].

[14]
^Soukhanov, A.H., ed. (1986). Word Mysteries & Histories. Houghton Mifflin. p. 73. ISBN 978-0-395-40265-8..

[15]
^Guralnik, D.B., ed. (1970). Webster's New World Dictionary. Prentice Hall. p. 450..

[16]
^Whittaker, E. T. (1951), A history of the theories of aether and electricity. Vol 1, Nelson, London.

[17]
^Leicester, H.M. (1971). The Historical Background of Chemistry. Courier Dover. pp. 221–222. ISBN 978-0-486-61053-5..

[18]
^DeKosky, R.K. (1983). "William Crookes and the quest for absolute vacuum in the 1870s". Annals of Science. 40 (1): 1–18. doi:10.1080/00033798300200101..

[19]
^弗兰克·维尔泽克:“生日快乐,电子”科学美国人,2012年6月。.

[20]
^Trenn, T.J. (1976). "Rutherford on the Alpha-Beta-Gamma Classification of Radioactive Rays". Isis. 67 (1): 61–75. doi:10.1086/351545. JSTOR 231134..

[21]
^Becquerel, H. (1900). "Déviation du Rayonnement du Radium dans un Champ Électrique". Comptes rendus de l'Académie des sciences (in 法语). 130: 809–815..

[22]
^布克瓦和沃里克(2001:90–91)。.

[23]
^Myers, W.G. (1976). "Becquerel's Discovery of Radioactivity in 1896". Journal of Nuclear Medicine. 17 (7): 579–582. PMID 775027..

[24]
^Thomson, J.J. (1897). "Cathode Rays". Philosophical Magazine. 44 (269): 293–316. doi:10.1080/14786449708621070..

[25]
^O'Hara, J. G. (Mar 1975). "George Johnstone Stoney, F.R.S., and the Concept of the Electron". Notes and Records of the Royal Society of London. Royal Society. 29 (2): 265–276. doi:10.1098/rsnr.1975.0018. JSTOR 531468..

[26]
^Kikoin, I.K.; Sominskiĭ, I.S. (1961). "Abram Fedorovich Ioffe (on his eightieth birthday)". Soviet Physics Uspekhi. 3 (5): 798–809. Bibcode:1961SvPhU...3..798K. doi:10.1070/PU1961v003n05ABEH005812.俄文原版出版物:Кикоин, И.К.; Соминский, М.С. (1960). "Академик А.Ф. Иоффе". Успехи Физических Наук. 72 (10): 303–321. doi:10.3367/UFNr.0072.196010e.0307..

[27]
^Millikan, R.A. (1911). "The Isolation of an Ion, a Precision Measurement of its Charge, and the Correction of Stokes' Law". Physical Review. 32 (2): 349–397. Bibcode:1911PhRvI..32..349M. doi:10.1103/PhysRevSeriesI.32.349..

[28]
^Das Gupta, N.N.; Ghosh, S.K. (1999). "A Report on the Wilson Cloud Chamber and Its Applications in Physics". Reviews of Modern Physics. 18 (2): 225–290. Bibcode:1946RvMP...18..225G. doi:10.1103/RevModPhys.18.225..

[29]
^Smirnov, B.M. (2003). Physics of Atoms and Ions. Springer. pp. 14–21. ISBN 978-0-387-95550-6..

[30]
^Bohr, N. (1922). "Nobel Lecture: The Structure of the Atom" (PDF). The Nobel Foundation. Retrieved 2008-12-03..

[31]
^Lewis, G.N. (1916). "The Atom and the Molecule". Journal of the American Chemical Society. 38 (4): 762–786. doi:10.1021/ja02261a002..

[32]
^Arabatzis, T.; Gavroglu, K. (1997). "The chemists' electron". European Journal of Physics. 18 (3): 150–163. Bibcode:1997EJPh...18..150A. doi:10.1088/0143-0807/18/3/005..

[33]
^Langmuir, I. (1919). "The Arrangement of Electrons in Atoms and Molecules". Journal of the American Chemical Society. 41 (6): 868–934. doi:10.1021/ja02227a002..

[34]
^Scerri, E.R. (2007). The Periodic Table. Oxford University Press. pp. 205–226. ISBN 978-0-19-530573-9..

[35]
^Massimi, M. (2005). Pauli's Exclusion Principle, The Origin and Validation of a Scientific Principle. Cambridge University Press. pp. 7–8. ISBN 978-0-521-83911-2..

[36]
^Uhlenbeck, G.E.; Goudsmith, S. (1925). "Ersetzung der Hypothese vom unmechanischen Zwang durch eine Forderung bezüglich des inneren Verhaltens jedes einzelnen Elektrons". Die Naturwissenschaften (in 德语). 13 (47): 953–954. Bibcode:1925NW.....13..953E. doi:10.1007/BF01558878..

[37]
^Pauli, W. (1923). "Über die Gesetzmäßigkeiten des anomalen Zeemaneffektes". Zeitschrift für Physik (in 德语). 16 (1): 155–164. Bibcode:1923ZPhy...16..155P. doi:10.1007/BF01327386..

[38]
^Falkenburg, B. (2007). Particle Metaphysics: A Critical Account of Subatomic Reality. Springer. p. 85. Bibcode:2007pmca.book.....F. ISBN 978-3-540-33731-7..

[39]
^Davisson, C. (1937). "Nobel Lecture: The Discovery of Electron Waves" (PDF). The Nobel Foundation. Retrieved 2008-08-30..

[40]
^Schrödinger, E. (1926). "Quantisierung als Eigenwertproblem". Annalen der Physik (in 德语). 385 (13): 437–490. Bibcode:1926AnP...385..437S. doi:10.1002/andp.19263851302..

[41]
^Rigden, J.S. (2003). Hydrogen. Harvard University Press. pp. 59–86. ISBN 978-0-674-01252-3..

[42]
^Reed, B.C. (2007). Quantum Mechanics. Jones & Bartlett Publishers. pp. 275–350. ISBN 978-0-7637-4451-9..

[43]
^Dirac, P.A.M. (1928). "The Quantum Theory of the Electron" (PDF). Proceedings of the Royal Society A. 117 (778): 610–624. Bibcode:1928RSPSA.117..610D. doi:10.1098/rspa.1928.0023..

[44]
^Dirac, P.A.M. (1933). "Nobel Lecture: Theory of Electrons and Positrons" (PDF). The Nobel Foundation. Retrieved 2008-11-01..

[45]
^"The Nobel Prize in Physics 1965". The Nobel Foundation. Retrieved 2008-11-04..

[46]
^Panofsky, W.K.H. (1997). "The Evolution of Particle Accelerators & Colliders" (PDF). Beam Line. 27 (1): 36–44. Retrieved 2008-09-15..

[47]
^Elder, F.R.; et al. (1947). "Radiation from Electrons in a Synchrotron". Physical Review. 71 (11): 829–830. Bibcode:1947PhRv...71..829E. doi:10.1103/PhysRev.71.829.5..

[48]
^Hoddeson, L.; et al. (1997). The Rise of the Standard Model: Particle Physics in the 1960s and 1970s. Cambridge University Press. pp. 25–26. ISBN 978-0-521-57816-5..

[49]
^Bernardini, C. (2004). "AdA: The First Electron–Positron Collider". Physics in Perspective. 6 (2): 156–183. Bibcode:2004PhP.....6..156B. doi:10.1007/s00016-003-0202-y..

[50]
^"Testing the Standard Model: The LEP experiments". CERN. 2008. Retrieved 2008-09-15..

[51]
^"LEP reaps a final harvest". CERN Courier. 40 (10). 2000..

[52]
^Prati, E.; De Michielis, M.; Belli, M.; Cocco, S.; Fanciulli, M.; Kotekar-Patil, D.; Ruoff, M.; Kern, D.P.; Wharam, D.A.; Verduijn, J.; Tettamanzi, G.C.; Rogge, S.; Roche, B.; Wacquez, R.; Jehl, X.; Vinet, M.; Sanquer, M. (2012). "Few electron limit of n-type metal oxide semiconductor single electron transistors". Nanotechnology. 23 (21): 215204. arXiv:1203.4811. Bibcode:2012Nanot..23u5204P. CiteSeerX 10.1.1.756.4383. doi:10.1088/0957-4484/23/21/215204. PMID 22552118..

[53]
^Frampton, P.H.; Hung, P.Q.; Sher, Marc (2000). "Quarks and Leptons Beyond the Third Generation". Physics Reports. 330 (5–6): 263–348. arXiv:hep-ph/9903387. Bibcode:2000PhR...330..263F. doi:10.1016/S0370-1573(99)00095-2..

[54]
^Raith, W.; Mulvey, T. (2001). Constituents of Matter: Atoms, Molecules, Nuclei and Particles. CRC Press. pp. 777–781. ISBN 978-0-8493-1202-1..

[55]
^CODATA的原始来源是Mohr, P.J.; Taylor, B.N.; Newell, D.B. (2008). "CODATA recommended values of the fundamental physical constants". Reviews of Modern Physics. 80 (2): 633–730. arXiv:0801.0028. Bibcode:2008RvMP...80..633M. CiteSeerX 10.1.1.150.1225. doi:10.1103/RevModPhys.80.633. CODATA中的单个物理常数可从以下网址获得:"The NIST Reference on Constants, Units and Uncertainty". National Institute of Standards and Technology. Retrieved 2009-01-15..

[56]
^"CODATA value: proton-electron mass ratio". 2006 CODATA recommended values. National Institute of Standards and Technology. Retrieved 2009-07-18..

[57]
^Murphy, M.T.; et al. (2008). "Strong Limit on a Variable Proton-to-Electron Mass Ratio from Molecules in the Distant Universe". Science. 320 (5883): 1611–1613. arXiv:0806.3081. Bibcode:2008Sci...320.1611M. doi:10.1126/science.1156352. PMID 18566280..

[58]
^Zorn, J.C.; Chamberlain, G.E.; Hughes, V.W. (1963). "Experimental Limits for the Electron-Proton Charge Difference and for the Charge of the Neutron". Physical Review. 129 (6): 2566–2576. Bibcode:1963PhRv..129.2566Z. doi:10.1103/PhysRev.129.2566..

[59]
^Odom, B.; et al. (2006). "New Measurement of the Electron Magnetic Moment Using a One-Electron Quantum Cyclotron". Physical Review Letters. 97 (3): 030801. Bibcode:2006PhRvL..97c0801O. doi:10.1103/PhysRevLett.97.030801. PMID 16907490..

[60]
^Anastopoulos, C. (2008). Particle Or Wave: The Evolution of the Concept of Matter in Modern Physics. Princeton University Press. pp. 261–262. ISBN 978-0-691-13512-0..

[61]
^Eichten, E.J.; Peskin, M.E.; Peskin, M. (1983). "New Tests for Quark and Lepton Substructure". Physical Review Letters. 50 (11): 811–814. Bibcode:1983PhRvL..50..811E. doi:10.1103/PhysRevLett.50.811..

[62]
^Curtis, L.J. (2003). Atomic Structure and Lifetimes: A Conceptual Approach. Cambridge University Press. p. 74. ISBN 978-0-521-53635-6..

[63]
^Dehmelt, H. (1988). "A Single Atomic Particle Forever Floating at Rest in Free Space: New Value for Electron Radius". Physica Scripta. T22: 102–110. Bibcode:1988PhST...22..102D. doi:10.1088/0031-8949/1988/T22/016..

[64]
^杰拉尔德·加布里埃尔斯 哈佛大学的网页.

[65]
^Meschede, D. (2004). Optics, light and lasers: The Practical Approach to Modern Aspects of Photonics and Laser Physics. Wiley-VCH. p. 168. ISBN 978-3-527-40364-6..

[66]
^Steinberg, R.I.; et al. (1999). "Experimental test of charge conservation and the stability of the electron". Physical Review D. 61 (2): 2582–2586. Bibcode:1975PhRvD..12.2582S. doi:10.1103/PhysRevD.12.2582..

[67]
^Agostini, M.; et al. (Borexino Collaboration) (2015). "Test of Electric Charge Conservation with Borexino". Physical Review Letters. 115 (23): 231802. arXiv:1509.01223. Bibcode:2015PhRvL.115w1802A. doi:10.1103/PhysRevLett.115.231802. PMID 26684111..

[68]
^J. Beringer (Particle Data Group); et al. (2012). "Review of Particle Physics: [electron properties]" (PDF). Physical Review D. 86 (1): 010001. Bibcode:2012PhRvD..86a0001B. doi:10.1103/PhysRevD.86.010001..

[69]
^Back, H.O.; et al. (2002). "Search for electron decay mode $e \to \gamma + \nu$ with prototype of Borexino detector". Physics Letters B. 525 (1–2): 29–40. Bibcode:2002PhLB..525...29B. doi:10.1016/S0370-2693(01)01440-X..

[70]
^Munowitz, M. (2005). Knowing, The Nature of Physical Law. Oxford University Press. ISBN 978-0-19-516737-5..

[71]
^Kane, G. (October 9, 2006). "Are virtual particles really constantly popping in and out of existence? Or are they merely a mathematical bookkeeping device for quantum mechanics?". Scientific American. Retrieved 2008-09-19..

[72]
^Taylor, J. (1989). "Gauge Theories in Particle Physics". In Davies, Paul. The New Physics. Cambridge University Press. p. 464. ISBN 978-0-521-43831-5..

[73]
^Genz, H. (2001). Nothingness: The Science of Empty Space. Da Capo Press. pp. 241–243, 245–247. ISBN 978-0-7382-0610-3..

[74]
^Gribbin, J. (January 25, 1997). "More to electrons than meets the eye". New Scientist. Retrieved 2008-09-17..

[75]
^Levine, I.; et al. (1997). "Measurement of the Electromagnetic Coupling at Large Momentum Transfer". Physical Review Letters. 78 (3): 424–427. Bibcode:1997PhRvL..78..424L. doi:10.1103/PhysRevLett.78.424..

[76]
^Murayama, H. (March 10–17, 2006). Supersymmetry Breaking Made Easy, Viable and Generic. Proceedings of the XLIInd Rencontres de Moriond on Electroweak Interactions and Unified Theories. La Thuile, Italy. arXiv:0709.3041. Bibcode:2007arXiv0709.3041M.—列出了一个电子的9\%质量差,其大小为普朗克距离。.

[77]
^Schwinger, J. (1948). "On Quantum-Electrodynamics and the Magnetic Moment of the Electron". Physical Review. 73 (4): 416–417. Bibcode:1948PhRv...73..416S. doi:10.1103/PhysRev.73.416..

[78]
^Huang, K. (2007). Fundamental Forces of Nature: The Story of Gauge Fields. World Scientific. pp. 123–125. ISBN 978-981-270-645-4..

[79]
^Foldy, L.L.; Wouthuysen, S. (1950). "On the Dirac Theory of Spin 1/2 Particles and Its Non-Relativistic Limit". Physical Review. 78 (1): 29–36. Bibcode:1950PhRv...78...29F. doi:10.1103/PhysRev.78.29..

[80]
^Sidharth, B.G. (2009). "Revisiting Zitterbewegung". International Journal of Theoretical Physics. 48 (2): 497–506. arXiv:0806.0985. Bibcode:2009IJTP...48..497S. doi:10.1007/s10773-008-9825-8..

[81]
^Griffiths, David J. (1998). Introduction to Electrodynamics (3rd ed.). Prentice Hall. ISBN 978-0-13-805326-0..

[82]
^Crowell, B. (2000). Electricity and Magnetism. Light and Matter. pp. 129–152. ISBN 978-0-9704670-4-1..

[83]
^Mahadevan, R.; Narayan, R.; Yi, I. (1996). "Harmony in Electrons: Cyclotron and Synchrotron Emission by Thermal Electrons in a Magnetic Field". The Astrophysical Journal. 465: 327–337. arXiv:astro-ph/9601073. Bibcode:1996ApJ...465..327M. doi:10.1086/177422..

[84]
^Rohrlich, F. (1999). "The Self-Force and Radiation Reaction". American Journal of Physics. 68 (12): 1109–1112. Bibcode:2000AmJPh..68.1109R. doi:10.1119/1.1286430..

[85]
^Georgi, H. (1989). "Grand Unified Theories". In Davies, Paul. The New Physics. Cambridge University Press. p. 427. ISBN 978-0-521-43831-5..

[86]
^Blumenthal, G.J.; Gould, R. (1970). "Bremsstrahlung, Synchrotron Radiation, and Compton Scattering of High-Energy Electrons Traversing Dilute Gases". Reviews of Modern Physics. 42 (2): 237–270. Bibcode:1970RvMP...42..237B. doi:10.1103/RevModPhys.42.237..

[87]
^Staff (2008). "The Nobel Prize in Physics 1927". The Nobel Foundation. Retrieved 2008-09-28..

[88]
^Chen, S.-Y.; Maksimchuk, A.; Umstadter, D. (1998). "Experimental observation of relativistic nonlinear Thomson scattering". Nature. 396 (6712): 653–655. arXiv:physics/9810036. Bibcode:1998Natur.396..653C. doi:10.1038/25303..

[89]
^Beringer, R.; Montgomery, C.G. (1942). "The Angular Distribution of Positron Annihilation Radiation". Physical Review. 61 (5–6): 222–224. Bibcode:1942PhRv...61..222B. doi:10.1103/PhysRev.61.222..

[90]
^Buffa, A. (2000). College Physics (4th ed.). Prentice Hall. p. 888. ISBN 978-0-13-082444-8..

[91]
^Eichler, J. (2005). "Electron–positron pair production in relativistic ion–atom collisions". Physics Letters A. 347 (1–3): 67–72. Bibcode:2005PhLA..347...67E. doi:10.1016/j.physleta.2005.06.105..

[92]
^Hubbell, J.H. (2006). "Electron positron pair production by photons: A historical overview". Radiation Physics and Chemistry. 75 (6): 614–623. Bibcode:2006RaPC...75..614H. doi:10.1016/j.radphyschem.2005.10.008..

[93]
^Quigg, C. (June 4–30, 2000). The Electroweak Theory. TASI 2000: Flavor Physics for the Millennium. Boulder, Colorado. p. 80. arXiv:hep-ph/0204104. Bibcode:2002hep.ph....4104Q..

[94]
^Mulliken, R.S. (1967). "Spectroscopy, Molecular Orbitals, and Chemical Bonding". Science. 157 (3784): 13–24. Bibcode:1967Sci...157...13M. doi:10.1126/science.157.3784.13. PMID 5338306..

[95]
^Burhop, E.H.S. (1952). The Auger Effect and Other Radiationless Transitions. Cambridge University Press. pp. 2–3. ISBN 978-0-88275-966-1..

[96]
^Jiles, D. (1998). Introduction to Magnetism and Magnetic Materials. CRC Press. pp. 280–287. ISBN 978-0-412-79860-3..

[97]
^Löwdin, P.O.; Erkki Brändas, E.; Kryachko, E.S. (2003). Fundamental World of Quantum Chemistry: A Tribute to the Memory of Per- Olov Löwdin. Springer. pp. 393–394. ISBN 978-1-4020-1290-7..

[98]
^Pauling, L.C. (1960). The Nature of the Chemical Bond and the Structure of Molecules and Crystals: an introduction to modern structural chemistry (3rd ed.). Cornell University Press. pp. 4–10. ISBN 978-0-8014-0333-0..
[99]

^Daudel, R.; et al. (1974). "The Electron Pair in Chemistry". Canadian Journal of Chemistry. 52 (8): 1310–1320. doi:10.1139/v74-201..

[100]
^Weinberg, S. (2003). The Discovery of Subatomic Particles. Cambridge University Press. pp. 15–16. ISBN 978-0-521-82351-7..

[101]
^Lou, L.-F. (2003). Introduction to phonons and electrons. World Scientific. pp. 162, 164. Bibcode:2003ipe..book.....L. ISBN 978-981-238-461-4..

[102]
^Guru, B.S.; Hızıroğlu, H.R. (2004). Electromagnetic Field Theory. Cambridge University Press. pp. 138, 276. ISBN 978-0-521-83016-4..

[103]
^Achuthan, M.K.; Bhat, K.N. (2007). Fundamentals of Semiconductor Devices. Tata McGraw-Hill. pp. 49–67. ISBN 978-0-07-061220-4..

[104]
^Ziman, J.M. (2001). Electrons and Phonons: The Theory of Transport Phenomena in Solids. Oxford University Press. p. 260. ISBN 978-0-19-850779-6..

[105]
^Main, P. (June 12, 1993). "When electrons go with the flow: Remove the obstacles that create electrical resistance, and you get ballistic electrons and a quantum surprise". New Scientist. 1887: 30. Retrieved 2008-10-09..

[106]
^Blackwell, G.R. (2000). The Electronic Packaging Handbook. CRC Press. pp. 6.39–6.40. ISBN 978-0-8493-8591-9..

[107]
^Durrant, A. (2000). Quantum Physics of Matter: The Physical World. CRC Press. pp. 43, 71–78. ISBN 978-0-7503-0721-5..

[108]
^Staff (2008). "The Nobel Prize in Physics 1972". The Nobel Foundation. Retrieved 2008-10-13..

[109]
^Kadin, A.M. (2007). "Spatial Structure of the Cooper Pair". Journal of Superconductivity and Novel Magnetism. 20 (4): 285–292. arXiv:cond-mat/0510279. doi:10.1007/s10948-006-0198-z..

[110]
^"Discovery About Behavior Of Building Block Of Nature Could Lead To Computer Revolution". ScienceDaily. July 31, 2009. Retrieved 2009-08-01..

[111]
^Jompol, Y.; et al. (2009). "Probing Spin-Charge Separation in a Tomonaga-Luttinger Liquid". Science. 325 (5940): 597–601. arXiv:1002.2782. Bibcode:2009Sci...325..597J. doi:10.1126/science.1171769. PMID 19644117..

[112]
^Staff (2008). "The Nobel Prize in Physics 1958, for the discovery and the interpretation of the Cherenkov effect". The Nobel Foundation. Retrieved 2008-09-25..

[113]
^Staff (August 26, 2008). "Special Relativity". Stanford Linear Accelerator Center. Retrieved 2008-09-25..

[114]
^de Broglie, L. (1929). "Nobel Lecture: The Wave Nature of the Electron" (PDF). The Nobel Foundation. Retrieved 2008-08-30..

[115]
^Lurquin, P.F. (2003). The Origins of Life and the Universe. Columbia University Press. p. 2. ISBN 978-0-231-12655-7..

[116]
^Silk, J. (2000). The Big Bang: The Creation and Evolution of the Universe (3rd ed.). Macmillan. pp. 110–112, 134–137. ISBN 978-0-8050-7256-3..

[117]
^Kolb, E.W.; Wolfram, Stephen (1980). "The Development of Baryon Asymmetry in the Early Universe". Physics Letters B. 91 (2): 217–221. Bibcode:1980PhLB...91..217K. doi:10.1016/0370-2693(80)90435-9..

[118]
^Sather, E. (Spring–Summer 1996). "The Mystery of Matter Asymmetry" (PDF). Beam Line. Stanford University. Retrieved 2008-11-01..

[119]
^Burles, S.; Nollett, K.M.; Turner, M.S. (1999). "Big-Bang Nucleosynthesis: Linking Inner Space and Outer Space". arXiv:astro-ph/9903300..
[120]

^Boesgaard, A.M.; Steigman, G. (1985). "Big bang nucleosynthesis – Theories and observations". Annual Review of Astronomy and Astrophysics. 23 (2): 319–378. Bibcode:1985ARA&A..23..319B. doi:10.1146/annurev.aa.23.090185.001535..

[121]
^Barkana, R. (2006). "The First Stars in the Universe and Cosmic Reionization". Science. 313 (5789): 931–934. arXiv:astro-ph/0608450. Bibcode:2006Sci...313..931B. CiteSeerX 10.1.1.256.7276. doi:10.1126/science.1125644. PMID 16917052..

[122]
^Burbidge, E.M.; et al. (1957). "Synthesis of Elements in Stars". Reviews of Modern Physics. 29 (4): 548–647. Bibcode:1957RvMP...29..547B. doi:10.1103/RevModPhys.29.547..

[123]
^Rodberg, L.S.; Weisskopf, V. (1957). "Fall of Parity: Recent Discoveries Related to Symmetry of Laws of Nature". Science. 125 (3249): 627–633. Bibcode:1957Sci...125..627R. doi:10.1126/science.125.3249.627. PMID 17810563..

[124]
^Fryer, C.L. (1999). "Mass Limits For Black Hole Formation". The Astrophysical Journal. 522 (1): 413–418. arXiv:astro-ph/9902315. Bibcode:1999ApJ...522..413F. doi:10.1086/307647..

[125]
^Parikh, M.K.; Wilczek, F. (2000). "Hawking Radiation As Tunneling". Physical Review Letters. 85 (24): 5042–5045. arXiv:hep-th/9907001. Bibcode:2000PhRvL..85.5042P. doi:10.1103/PhysRevLett.85.5042. hdl:1874/17028. PMID 11102182..

[126]
^Hawking, S.W. (1974). "Black hole explosions?". Nature. 248 (5443): 30–31. Bibcode:1974Natur.248...30H. doi:10.1038/248030a0..

[127]
^Halzen, F.; Hooper, D. (2002). "High-energy neutrino astronomy: the cosmic ray connection". Reports on Progress in Physics. 66 (7): 1025–1078. arXiv:astro-ph/0204527. Bibcode:2002RPPh...65.1025H. doi:10.1088/0034-4885/65/7/201..

[128]
^Ziegler, J.F. (1998). "Terrestrial cosmic ray intensities". IBM Journal of Research and Development. 42 (1): 117–139. doi:10.1147/rd.421.0117..

[129]
^Sutton, C. (August 4, 1990). "Muons, pions and other strange particles". New Scientist. Retrieved 2008-08-28..

[130]
^Gurnett, D.A.; Anderson, R. (1976). "Electron Plasma Oscillations Associated with Type III Radio Bursts". Science. 194 (4270): 1159–1162. Bibcode:1976Sci...194.1159G. doi:10.1126/science.194.4270.1159. PMID 17790910..

[131]
^Martin, W.C.; Wiese, W.L. (2007). "Atomic Spectroscopy: A Compendium of Basic Ideas, Notation, Data, and Formulas". National Institute of Standards and Technology. Retrieved 2007-01-08..

[132]
^Fowles, G.R. (1989). Introduction to Modern Optics. Courier Dover. pp. 227–233. ISBN 978-0-486-65957-2..

[133]
^Grupen, C. (2000). "Physics of Particle Detection". AIP Conference Proceedings. 536: 3–34. arXiv:physics/9906063. Bibcode:2000AIPC..536....3G. doi:10.1063/1.1361756..

[134]
^Ekstrom, P.; Wineland, David (1980). "The isolated Electron" (PDF). Scientific American. 243 (2): 91–101. Bibcode:1980SciAm.243b.104E. doi:10.1038/scientificamerican0880-104. Retrieved 2008-09-24..

[135]
^Mauritsson, J. "Electron filmed for the first time ever" (PDF). Lund University. Archived from the original (PDF) on March 25, 2009. Retrieved 2008-09-17..

[136]
^Mauritsson, J.; et al. (2008). "Coherent Electron Scattering Captured by an Attosecond Quantum Stroboscope". Physical Review Letters. 100 (7): 073003. arXiv:0708.1060. Bibcode:2008PhRvL.100g3003M. doi:10.1103/PhysRevLett.100.073003. PMID 18352546..

[137]
^Damascelli, A. (2004). "Probing the Electronic Structure of Complex Systems by ARPES". Physica Scripta. T109: 61–74. arXiv:cond-mat/0307085. Bibcode:2004PhST..109...61D. doi:10.1238/Physica.Topical.109a00061..

[138]
^Elmer, J. (March 3, 2008). "Standardizing the Art of Electron-Beam Welding". Lawrence Livermore National Laboratory. Retrieved 2008-10-16..

[139]
^Schultz, H. (1993). Electron Beam Welding. Woodhead Publishing. pp. 2–3. ISBN 978-1-85573-050-2..

[140]
^Benedict, G.F. (1987). Nontraditional Manufacturing Processes. Manufacturing engineering and materials processing. 19. CRC Press. p. 273. ISBN 978-0-8247-7352-6..

[141]
^Ozdemir, F.S. (June 25–27, 1979). Electron beam lithography. Proceedings of the 16th Conference on Design automation. San Diego, CA: IEEE Press. pp. 383–391. Retrieved 2008-10-16..

[142]
^Madou, M.J. (2002). Fundamentals of Microfabrication: the Science of Miniaturization (2nd ed.). CRC Press. pp. 53–54. ISBN 978-0-8493-0826-0..

[143]
^Jongen, Y.; Herer, A. (May 2–5, 1996). Electron Beam Scanning in Industrial Applications. APS/AAPT Joint Meeting. American Physical Society. 
Bibcode:1996APS..MAY.H9902J..

[144]
^Mobus, G.; et al. (2010). "Nano-scale quasi-melting of alkali-borosilicate glasses under electron irradiation". Journal of Nuclear Materials. 396 (2–3): 264–271. Bibcode:2010JNuM..396..264M. doi:10.1016/j.jnucmat.2009.11.020..

[145]
^Beddar, A.S.; Domanovic, Mary Ann; Kubu, Mary Lou; Ellis, Rod J.; Sibata, Claudio H.; Kinsella, Timothy J. (2001). "Mobile linear accelerators for intraoperative radiation therapy". AORN Journal. 74 (5): 700–705. doi:10.1016/S0001-2092(06)61769-9..

[146]
^Gazda, M.J.; Coia, L.R. (June 1, 2007). "Principles of Radiation Therapy" (PDF). Retrieved 2013-10-31..

[147]
^Chao, A.W.; Tigner, M. (1999). Handbook of Accelerator Physics and Engineering. World Scientific. pp. 155, 188. ISBN 978-981-02-3500-0..

[148]
^Oura, K.; et al. (2003). Surface Science: An Introduction. Springer. pp. 1–45. ISBN 978-3-540-00545-2..

[149]
^Ichimiya, A.; Cohen, P.I. (2004). Reflection High-energy Electron Diffraction. Cambridge University Press. p. 1. ISBN 978-0-521-45373-8..

[150]
^Heppell, T.A. (1967). "A combined low energy and reflection high energy electron diffraction apparatus". Journal of Scientific Instruments. 44 (9): 686–688. Bibcode:1967JScI...44..686H. doi:10.1088/0950-7671/44/9/311..

[151]
^McMullan, D. (1993). "Scanning Electron Microscopy: 1928–1965". University of Cambridge. Retrieved 2009-03-23..

[152]
^Slayter, H.S. (1992). Light and electron microscopy. Cambridge University Press. p. 1. ISBN 978-0-521-33948-3..

[153]
^Cember, H. (1996). Introduction to Health Physics. McGraw-Hill Professional. pp. 42–43. ISBN 978-0-07-105461-4..
[154]
^Erni, R.; et al. (2009). "Atomic-Resolution Imaging with a Sub-50-pm Electron Probe". Physical Review Letters. 102 (9): 096101. Bibcode:2009PhRvL.102i6101E. doi:10.1103/PhysRevLett.102.096101. PMID 19392535..

[155]
^Bozzola, J.J.; Russell, L.D. (1999). Electron Microscopy: Principles and Techniques for Biologists. Jones & Bartlett Publishers. pp. 12, 197–199. ISBN 978-0-7637-0192-5..

[156]
^Flegler, S.L.; Heckman Jr., J.W.; Klomparens, K.L. (1995). Scanning and Transmission Electron Microscopy: An Introduction (Reprint ed.). Oxford University Press. pp. 43–45. ISBN 978-0-19-510751-7..

[157]
^Bozzola, J.J.; Russell, L.D. (1999). Electron Microscopy: Principles and Techniques for Biologists (2nd ed.). Jones & Bartlett Publishers. p. 9. ISBN 978-0-7637-0192-5..

[158]
^Freund, H.P.; Antonsen, T. (1996). Principles of Free-Electron Lasers. Springer. pp. 1–30. ISBN 978-0-412-72540-1..

[159]
^Kitzmiller, J.W. (1995). Television Picture Tubes and Other Cathode-Ray Tubes: Industry and Trade Summary. Diane Publishing. pp. 3–5. ISBN 978-0-7881-2100-5..

[160]
^Sclater, N. (1999). Electronic Technology Handbook. McGraw-Hill Professional. pp. 227–228. ISBN 978-0-07-058048-0..

[161]
^Staff (2008). "The History of the Integrated Circuit". The Nobel Foundation. Retrieved 2008-10-18..