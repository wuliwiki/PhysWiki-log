% 电偶极子辐射

\begin{issues}
\issueDraft
\end{issues}

\pentry{电磁场推迟势\upref{RetPt0}}

\footnote{参考 \cite{GriffE}。}令原点处的电偶极子为
\begin{equation}
\bvec p(t) = p_0 \cos(\omega t) \uvec z~.
\end{equation}

使用洛伦兹规范, 在 $d \ll \lambda \ll r$ 的远场近似下 ($d$ 是偶极子的长度, $\lambda = \frac{2\pi}{k}$)
\begin{equation}
\varphi(r, \theta, t) = -\frac{p_0\omega}{4\pi\epsilon_0 c} \qty(\frac{\cos\theta}{r}) \sin[\omega(t - r/c)]~,
\end{equation}
\begin{equation}
\bvec A(r, \theta, t) = -\frac{\mu_0 p_0 \omega}{4\pi r} \sin[\omega(t - r/c)]\uvec z~.
\end{equation}
进而得
\begin{equation}
\bvec E = -\frac{\mu_0 p_0\omega^2}{4\pi} \qty(\frac{\sin\theta}{r})\cos[\omega(t - r/c)]\uvec \theta~~,
\end{equation}
\begin{equation}
\bvec B = -\frac{\mu_0 p_0\omega^2}{4\pi c} \qty(\frac{\sin\theta}{r})\cos[\omega(t - r/c)]\uvec \phi~.
\end{equation}

%看起来是对的,但和网上看到的又似乎不大一样,需要dalao check一下
\begin{figure}[ht]
\centering
\includegraphics[width=12cm]{./figures/75c77b08e8078860.png}
\caption{$Z=0$平面上的电场矢量场.原点附近由于远场近似不再有效而发散。笔者制作的\href{https://www.geogebra.org/m/xnputtwr}{一个可动的立体模型}(站外链接)} \label{fig_DipRad_4}
\end{figure}

\begin{figure}[ht]
\centering
\includegraphics[width=12cm]{./figures/30889a179f178d9b.png}
\caption{$X$轴上的电场矢量场。在远处辐射场行为类似平面电磁波\upref{VcPlWv} } \label{fig_DipRad_1}
\end{figure}

\begin{figure}[ht]
\centering
\includegraphics[width=12cm]{./figures/c8048e9ce90dcf8f.png}
\caption{$Y=0$上的电场矢量场。} \label{fig_DipRad_3}
\end{figure}
\subsubsection{绘制矢量场的.m代码}
\begin{lstlisting}[language=matlab]
clc
clear
T=2;%周期
w=2*pi/T;
v=1;%波速
k=w/v;
t=0;%时刻

[x y z] = meshgrid(-5:0.2:5,-5:0.2:5,-5:0.2:5);
%将轴的范围设成0,可得到截面上的矢量场.完整的矢量场太复杂了
[phi theta r]=cart2sph(x,y,z);
theta = pi/2-theta;
E = -(sin(theta)./r).*cos(w*t-k*r);%电场强度数值
a = E.*cos(theta).*cos(phi);
b = E.*cos(theta).*sin(phi);
c = E.*-sin(theta);

hold on
axis equal;
axis([-5 5 -5 5 -5 5]);

xlabel('X','fontsize',12,'fontname','Times New Roman');
ylabel('Y','fontsize',12,'fontname','Times New Roman');
zlabel('Z','fontsize',12,'fontname','Times New Roman');

quiver3(x,y,z,a,b,c,0);

line([-10 10], [0 0],[0 0]);

hold off

\end{lstlisting}


\subsubsection{辐射功率}
可用能流密度(坡印廷矢量\upref{EBS})描述辐射功率。
\begin{equation}
\bvec s(\bvec r, t) = \frac{1}{\mu_0} \bvec E \cross \bvec B = \frac{\mu_0p_0^2\omega^4}{16\pi^2c} \frac{\sin^2\theta}{r^2} \cos^2[\omega(t - r/c)] \uvec r~.
\end{equation}
时间平均值为
\begin{equation}
\ev{\bvec s} = \frac{1}{\mu_0} \bvec E \cross \bvec B
= \frac{\mu_0p_0^2\omega^4}{32\pi^2c} \frac{\sin^2\theta}{r^2} \uvec r~.
\end{equation}

\begin{figure}[ht]
\centering
\includegraphics[width=10cm]{./figures/6905e3954ff0b7a4.pdf}
\caption{$\ev{\bvec s}$的“等强度面”,内圈的$\ev{\bvec s}$强度更高.注意平行于偶极子的方向上没有能流.\href{https://www.geogebra.org/m/semmtxm5}{一个立体模型}(站外链接)} \label{fig_DipRad_2}
\end{figure}
总辐射功率(时间平均)为
\begin{equation}
\ev{P} = \oint \ev{\bvec s} \vdot \dd{\bvec a} = \frac{\mu_0 p_0^2 \omega^4}{12\pi c}~.
\end{equation}
总辐射功率是定值体现了能量守恒。如果总辐射功率不为定值,说明来自源的一部分能量在传播过程中被损耗掉了。
