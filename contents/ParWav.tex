% 量子散射(单粒子弹性)
% keys 散射|概率流|薛定谔|截面|微分截面|相移|S 矩阵|K 矩阵
% license Xiao
% type Tutor

\pentry{散射\nref{nod_Scater}, 概率流密度\nref{nod_PrbJ}}{nod_4d36}

\subsection{散射截面}

\footnote{本文参考 \cite{Bransden} 和 \cite{Burke} 相关章节。}本文使用原子单位制\upref{AU}。 散射截面 $\sigma$ 等于一定时间内被散射的粒子数除以单位截面入射的粒子数。那么从经典力学的角度,如果想象入射粒子流密度是均匀的, $\sigma$ 可以看做是一个障碍物(无远程作用)的最大横截面面积,微分截面 $\dv*{\sigma}{\Omega}$ 可以理解为单位立体角的散射截面。量子力学中,如果考虑单粒子以平面波入射,那么 $\sigma$ 等于被散射的概率流(概率/时间)除以入射的概率流密度(概率/时间/面积)。概率流定义为
\begin{equation}
\bvec j = \frac{\I }{2m}(\psi \grad \psi^* - \psi^* \grad \psi )~.
\end{equation}
\textbf{微分截面(differential cross section)} 是能量的函数, 可以用概率流定义为(假设散射是轴对称的)
\begin{equation}\label{eq_ParWav_3}
\dv{\sigma}{\Omega} = \lim_{r\to\infty} \frac{(\bvec j_{out} \vdot \uvec r) r^2}{\abs{\bvec j_{in}}}~,
\end{equation}
对所有方向积分得总\textbf{散射截面(scattering cross section)}
\begin{equation}
\sigma = \int \dv{\sigma}{\Omega} \dd{\Omega} ~.
\end{equation}

\subsection{短程势的边界条件}
在三维情况下, 每个能量 $E$ 的本征函数都是无穷维简并的。 且根据不同的边界条件我们可以获得不同的正交归一基底。 常见的边界条件如平面波入射。 这是一种物理意义较强的选择, 因为如果一个无穷远处的入射波包具有很窄的能量带宽, 我们就可以把它近似看作是平面波\footnote{如果不是, 那么入射波包可以由这些散射态叠加而成, 之后的时间演化就是这些散射态分别乘以 $\exp(-\I E t)$ 再叠加。 但由于这时能量没有精确的定义, 我们不能讨论微分截面关于能量的函数。}。 平面波经过散射后, 会向各个方向发射球面波。 于是规定波函数的边界条件为(以下用箭头表示 $r\to\infty$ 时函数的渐进形式)
\begin{equation}\label{eq_ParWav_1}
\psi_{\bvec k}^{(+)}(\bvec r) \rightarrow (2\pi)^{-3/2} \qty[\exp(\I \bvec k \vdot \bvec r) + f(k, \uvec r) \frac{\exp(\I k r)}{r}]~,
\end{equation}
$f(k, \uvec r)$ 是\textbf{散射幅}。 注意该边界条件只能对短程势能使用, 即满足
\begin{equation}
\lim_{r\to\infty} r V(r) = 0~
\end{equation}
的势能, 原因见下文。 显然\textbf{库仑势能不属于短}程势, 我们将在 “库仑散射\upref{CulmWf}” 讨论。 满足边界条件\autoref{eq_ParWav_1} 的波函数也会满足正交归一条件
\begin{equation}
\int \psi_{\bvec k'}^{(+)}(\bvec r)^* \psi_{\bvec k}^{(+)}(\bvec r) \dd[3]{r} = \delta(\bvec k - \bvec k')~.
\end{equation}
将\autoref{eq_ParWav_1} 代入\autoref{eq_ParWav_3} 可得微分截面就是散射幅的模方
\begin{equation}
\dv{\sigma}{\Omega} = \abs{f(k, \uvec r)}^2~.
\end{equation}

\subsection{光学定理}
由概率守恒对散射幅的约束, 可以得出\textbf{光学定理(Optical Theorem)}
\begin{equation}
\sigma = \frac{4\pi}{k} \Im[f(k, \uvec k)]~.
\end{equation}
光学定理的物理意义是: 球面波往其他方向发射出去的总概率流等于在 $\uvec k$ 方向抵消的概率流。
\addTODO{证明见 \cite{Bransden} 习题 12.2}

\subsection{分波展开}
\pentry{球坐标中的薛定谔方程\nref{nod_RYTDSE}}{nod_68d1}

除了\autoref{eq_ParWav_1} 的边界条件外, 我们也可以找到符合另一种常用边界条件的正交归一散射态, 即要求每个散射态同时是能量和角动量 $L^2, L_z$ 的本征态。 这与 “氢原子的定态波函数\upref{HWF}” 类似, 只是我们要求能量 $E > 0$。

这时每个能量同样有无穷维简并。 我们在球坐标中解方程。 先讨论简单的情况, 即势能为中心势能 $V(r)$, 薛定谔方程可以在球坐标中分离变量, 使波函数表示为 %链接未完成
\begin{equation}
\psi(\bvec r) = \frac{1}{r} \sum_l c_l \psi_{l,m}(r) Y_{l,m}(\uvec r)~,
\end{equation}
其中每一项被称为一个\textbf{分波(partial wave)}, 满足上述边界条件。 因为在中心势能 $V(r)$ 下, 使解偏微分方程变为解常微分方程, 即径向方程 %链接未完成
\begin{equation}
-\frac{1}{2m} \dv[2]{\psi_{l,m}}{r} + \qty[V(r) + \frac{l(l + 1)}{2mr^2}]\psi_{l,m} = \I \pdv{t} \psi_{l,m}~.
\end{equation}
\addTODO{这一段讲解有点问题, 要说明我们只需要 $m = 0$}

\subsection{相移}
对于短程势能 $V(r)$, 即
\begin{equation}
\lim_{r\to\infty} r V(r) = 0~.
\end{equation}
可以证明\footnote{不严谨的证明: 无穷远处相位的变化约等于 $\int_0^\infty [V(r) - l(l+1)/(2mr^2)]\dd{r}$, 只要满足 $rV(r) \to 0$ 这个积分就收敛。}短程势能在无穷远处对相位的影响可以忽略。 $\psi_l(r)$ 的渐进表达式就完全可以由\textbf{相移(phase shift)} $\delta_l$ 来描述。 由于微分截面也是在无穷远处定义的, 我们\textbf{可以直接由相移计算微分截面}。

我们定义相移 $\delta_l$ 满足
\begin{equation}
\psi_l(k, r) \to \sin\qty(kr - \frac{l\pi}{2} + \delta_l)~.
\end{equation}
为什么这么定义呢? 因为相移是相对的, 我们把 $V(r) \equiv 0$ 时的相位作为基准
\begin{equation}
\psi_l(k, r) = kr j_r(kr) \to \sin\qty(kr - \frac{l\pi}{2})~,
\end{equation}
可以证明(使用\autoref{eq_Pl2Ylm_1}~\upref{Pl2Ylm} 把\autoref{eq_ParWav_1} 中的平面波展开, 再对比系数), 将球面波叠加得到\autoref{eq_ParWav_1} 形式后散射幅为
\begin{equation}
f(k, \theta) = \sum_{l=0}^\infty f_l(k) P_l(\cos\theta)~,
\end{equation}
\begin{equation}
f_l(k) = \frac{2l+1}{k} \exp(\I\delta_l) \sin\delta_l~.
\end{equation}
注意这与 $\phi$ 无关, $f(\uvec r)$ 是一个轴对称的分布。 总截面也可以表示为每个分波的截面之和
\begin{equation}
\sigma(k) = \int \abs{f(k,\theta)}^2 \dd{\Omega} = \sum_{l=0}^\infty \sigma_l(k)~,
\end{equation}
\begin{equation}
\sigma_l(k) = \frac{4\pi}{2l+1} \abs{f_l(k)}^2 = \frac{4\pi}{k^2} (2l + 1) \sin^2 \delta_l(k)~.
\end{equation}
这说明, 散射截面只和相移 $\delta_l(k)$ 有关。 数学上可以证明对应同一列 $\delta_l(k)$ 的势能并不是唯一的。

最后, 容易证明
\begin{equation}
\sigma(k) = \frac{4\pi}{k} \Im f(k,\theta=0)~.
\end{equation}
这说明\textbf{入射方向的概率流减少等于被散射的概率流概率流}, 也被称为\textbf{光学定理(optical theorem)}。

\subsection{散射矩阵}
定义 $S$ 矩阵(虽然叫矩阵, 其实是一列函数)为
\begin{equation}
S_l(k) = \exp(2\I\delta_l)~,
\end{equation}
定义 $K$ 矩阵为
\begin{equation}
K_l(k) = \tan \delta_l~,
\end{equation}
二者关系为
\begin{equation}
S_l(k) = \frac{1 + \I K_l(k)}{1 - \I K_l(k)}~.
\end{equation}
称为矩阵的原因是, 在多通道散射(未完成)中, 需要考虑从不同通道入射和出射的情况, 这些量就成为了矩阵, 每个矩阵元都是一个函数。

\subsection{推导散射截面和相位的关系}
\pentry{平面波的球谐展开\nref{nod_Pl2Ylm}}{nod_44cf}
存在一组变换系数 $c_l$ 使
\begin{equation}\label{eq_ParWav_2}
\psi(\bvec r) \rightarrow \frac{1}{kr} \sum_l c_l \sin(kr - \frac{\pi l}{2} + \delta_l) Y_{l,0}(\bvec r)~
\end{equation}
满足边界条件\autoref{eq_ParWav_1}。 令\autoref{eq_ParWav_1} 中散射幅为
\begin{equation}\label{eq_ParWav_4}
f(k,\theta) = \sum_l (2l + 1) a_l(k) P_l(\cos \theta)~.
\end{equation}
假设我们已经在球坐标中解出了 $\psi_{k,l}$,即径向波函数 $R_{k,l}(r)$ 与相移, 如何获得 $f(k,\theta )$,即系数 $a_l(k)$ 呢? 把 $\psi_k$ 用 $\psi_{k,l}$ 基底展开,即对 $P_l$ 展开,再逐项对比系数即可。 首先展开平面波
\begin{equation}\label{eq_ParWav_12}
\E^{\I kz} = \sum_{l = 0}^\infty  \I^l (2l + 1) j_l(kr) P_l(\cos \theta) \rightarrow \sum_{l=0}^\infty  (2l + 1)\frac{\E^{\I kr} - \E^{-\I(kr - l\pi)}}{2\I kr} P_l(\cos \theta)~.
\end{equation}
将\autoref{eq_ParWav_4} 和\autoref{eq_ParWav_12} 代入\autoref{eq_ParWav_1}, 再逐项与\autoref{eq_ParWav_2} 对比,得
\begin{equation}
a_l(k) = \frac{1}{k} \sin\delta_l \exp(\I \delta_l)~,
\end{equation}
再次代入\autoref{eq_ParWav_4} 可得散射幅与相移的关系。

也可以得到系数
\begin{equation}
c_{l,m}^{(\pm)} = \sqrt{\frac{2}{\pi}} \I^l \E^{\pm \I \delta_l} Y_{l,m}^*(\uvec k)~,
\end{equation}
这比平面波的球谐展开\upref{Pl2Ylm}多了一个相位因子 $\E^{\pm \I \delta_l}$。

散射的边界条件可以有两种, 用 $\pm$ 区分。
\addTODO{有待数值验证}
\begin{equation}
\begin{aligned}
&\ket{\psi_{\bvec k}^{(\pm)}} = \frac{1}{(2\pi)^{3/2}}[\E^{\I \bvec k \vdot \bvec r} + f^{(\pm)}(\uvec k)\E^{\pm\I kr}]\\
&= \sqrt{\frac{2}{\pi}} \sum_{l=0}^{\infty} \I^l \E^{\pm\I \delta_l} j_l(kr + \delta_l) \sum_{m=-l}^l Y_{l,m}^*(\uvec k) Y_{l,m}(\uvec r)~.
\end{aligned}
\end{equation}
