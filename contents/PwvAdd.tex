% 平面电磁波的能量叠加
% 平面波|功率|能量密度|干涉

\pentry{真空中的平面电磁波\upref{VcPlWv}}

已知单个平面电磁波的平均能流密度为 $c\epsilon_0 E_0^2/2$, 那么两个同方向不同频率电磁波叠加后能流密度是多少呢?

考虑两个电磁波 $E_1 \sin(k_1 x - \omega_1 t)$ 和 $E_2 \sin(k_2x - \omega_2 t)$, 它们叠加后的能流密度是多少呢? 先看电磁场能量密度
\begin{equation}
\begin{aligned}
\rho_E &= \epsilon_0 [E_1 \sin(k_1 x - \omega_1 t) + E_2 \sin(k_2x - \omega_2 t)]^2\\
&= \epsilon_0 E_1^2 \sin^2(k_1 x - \omega_1 t) + \epsilon_0 E_2^2 \sin^2(k_2x - \omega_2 t)\\
&\quad  + 2\epsilon_0 E_1 E_2 \sin(k_1 x - \omega_1 t) \sin(k_2x - \omega_2 t)
\end{aligned}
\end{equation}
其中前两项是每个平面波各自的能量密度, 第三项是干涉项, 根据\autoref{TriEqv_eq10}~\upref{TriEqv} 可知他是两个简谐函数之和, 平均值为零。

所以 $N$ 个同方向不同频率平面波叠加后的平均能流密度等于它们各自能流密度之和
\begin{equation}
\abs{\bvec j} = \frac{1}{2}c\epsilon_0 \sum_i E_i^2
\end{equation}
