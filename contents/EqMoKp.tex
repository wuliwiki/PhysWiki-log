% 开普勒问题的运动方程

\pentry{开普勒问题\upref{CelBd}, 中心力场问题\upref{CenFrc}}

\footnote{参考 \cite{Goldstein} 第三章(未讨论双曲线), 以及 Wikipedia \href{https://en.wikipedia.org/wiki/Kepler's_equation}{相关页面}. }开普勒问题中若已知轨道形状\upref{Keple1}, 我们来计算质点在轨道上的位置如何关于时间变化. 直观上我们可以使用开普勒第二定律\upref{Keple}使用扫过的面积推导出从出发点到达任意位置所需的时间. 但这里介绍对半径做定积分的方法, 二者的结果是一样的.

由\autoref{CenFrc_eq8}~\upref{CenFrc}得
\begin{equation}
t = \sqrt{\frac{m}{2}} \int_{r_0}^r \frac{\dd{r'}}{\sqrt{E - k/r' - L^2/(2mr'^2)}}
\end{equation}
该式对任何圆锥曲线轨道都适用, 其中 $r_0$ 是轨道的近日点, 在标准的圆锥曲线方程(\autoref{Cone_eq5}~\upref{Cone})中对应 $\theta = \pi$. 令质点经过近日点时 $t= 0$. 把这个积分的结果 $t(r)$ 取反函数, 就可以得到 $r(t)$. 同理, 有
\begin{equation}
\dd{t} = \frac{mr^2}{L}\dd{\theta}
\end{equation}
将\autoref{Cone_eq5}~\upref{Cone}代入, 积分得
\begin{equation}
t = \frac{L^3}{mk^2} \int_{\pi}^\theta \frac{\dd{\theta'}}{(1 - e\cos \theta')^2 }
\end{equation}
以下我们令 $\Delta\theta$ 为某位置相对于近日点的极角增量, 即 $\Delta \theta = \theta - \pi$. 显然对任何圆锥曲线 $\Delta \theta$ 和 $t$ 的取值区间关于原点对称, 且由轨道的对称性可知 $\Delta\theta(t)$ 是一个奇函数. 以下默认质点绕中心天体逆时针转动, 所以 $\Delta\theta(t)$ 单调递增, 若要考虑顺时针, 取 $-\Delta\theta(t)$ 即可.

对\textbf{抛物线}($e = 1$), 假设逆时针运动, , 有
\begin{equation}\label{EqMoKp_eq3}
t = \frac{L^3}{2mk^2} \qty(\tan\frac{\Delta\theta}{2} +  \frac{1}{3}\tan^3 \frac{\Delta\theta}{2})\quad (-\pi<\Delta\theta<\pi)
\end{equation}

对于\textbf{椭圆}($e < 1$), 可以用一个参数\textbf{偏近点角(eccentric anomaly)} $\psi$ 来代替 $\Delta\theta$ 会更方便. $\psi$ 的定义为
\begin{equation}\label{EqMoKp_eq1}
r = a(1-e\cos\psi)
\end{equation}
其中 $a$ 是半长轴. 当 $\Delta\theta $ 从 $-\pi$ 变化到 $\pi$ 时, $\psi$ 也从 $-\pi$ 变化到 $\pi$, $\psi(\Delta\theta)$ 是一个递增的奇函数.
\begin{equation}\label{EqMoKp_eq5}
t = \sqrt{\frac{ma^3}{\abs{k}}} (\psi - e \sin\psi)
\end{equation}
该式被称为\textbf{开普勒方程(Kepler's equation)}, 开普勒第二定律也可以由该式验证.

对于\textbf{双曲线}($e>1, k<0$), \textbf{双曲偏近点角(hyperbolic eccentric anomaly)} $\xi$ 使用下式定义:
\begin{equation}\label{EqMoKp_eq2} % 已经验算正确
r = a(e\cosh\xi - 1) \qquad (\xi \in \mathbb R)
\end{equation}
$\xi$ 从 $-\infty$ 到 $\infty$ 的变化对应 $\Delta\theta$ 从 $-\pi+\gamma$ 到 $\pi-\gamma$ 变化, $\gamma$ 是双曲线渐进角(\autoref{Hypb3_eq1}~\upref{Hypb3} 中的 $\theta_0$). $\xi(\Delta\theta)$ 是一个递增奇函数.
\begin{equation}\label{EqMoKp_eq4} % 已经验算正确
t = \sqrt{\frac{ma^3}{\abs{k}}} (e\sinh\xi - \xi)
\end{equation}

对于\textbf{双曲线}($e>1, k>0$), 我们可以把圆锥曲线标准方程(\autoref{Cone_eq5}~\upref{Cone})旋转 $180^\circ$ 得
\begin{equation} % 已经验算正确
r = -\frac{p}{1 - e\cos\theta} \quad (-\gamma<\theta<\gamma)
\end{equation}
类似地, 有\footnote{推导见\lstinline|chrome-extension://efaidnbmnnnibpcajpcglclefindmkaj/viewer.html?pdfurl=https%3A%2F%2Fcomethunter.lamost.org%2Fscwrk%2FTHECAL%2Ffkepler.pdf&clen=69977&chunk=true|}
\begin{equation} % 已经验算正确
r = a(e\cosh\xi + 1) \qquad (\xi \in \mathbb R)
\end{equation}
此时近日点变为 $\theta = 0$, 故 $\Delta\theta = \theta \in (-\gamma, \gamma)$
\begin{equation}\label{EqMoKp_eq6} % 已经验算正确
t = \sqrt{\frac{ma^3}{\abs{k}}} (e\sinh\xi + \xi)
\end{equation}

\subsection{推导}
\addTODO{……}
