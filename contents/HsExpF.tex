% 指数函数(高中)
% keys 指数|指数函数|自然常数
% license Xiao
% type Tutor

\begin{issues}
\issueDraft
\end{issues}

%已基本完成,但行文结构需要调整。

\pentry{函数\nref{nod_functi},函数的性质\nref{nod_HsFunC},幂运算与幂函数\nref{nod_power}}{nod_d767}

指数函数在日常生活中十分常见,它能够形象地描述一种增长速度不断加快的模式。例如,细菌的繁殖就是一个典型的指数增长过程:如果某种细菌每小时分裂一次,每次分裂后数量就会翻倍,那么细菌的数量会从起初的几百个迅速增至成千上万。类似地,银行存款的利息增长也是指数式的:本金存入后,按固定的比例产生利息,利息再生息,逐渐积累。这种增长过程就像滚雪球一样,短时间内便可能增长到一个令人惊讶的数量。

指数函数的本质就是描述描述增长速度与自身数值直接相关的现象。在这种增长模式下,随着数量增大,其增长速度也不断加快。与幂函数仅以固定速率增长不同,指数函数的增长速度会随着数量的增加而加快,特别是在数量达到一定规模时,这种增长几乎是飞跃式的。在现实生活中,大多数处于理想环境的增长模式都可以归纳为指数增长,直到受到环境因素的变化或达到系统的承载能力,才会改变。除了细菌繁殖和利息积累的例子外,指数函数还广泛用于描述类似的变化,例如放射性元素的衰减、人口增长、疾病传播等。这些现象的共同点是:初期的变化可能并不显著,但随着时间推移,变化会迅速积累,产生极大的影响。

\subsection{指数函数}

回看幂运算的\aref{定义}{def_power_1},如果将底数作为参数,指数作为自变量的函数就称为指数函数,指数函数的名称指的就是自变量的在指数位置上,注意不要与幂函数相混淆。

\begin{definition}{指数函数}
形如
\begin{equation}
f(x) = a^x~.
\end{equation}
的函数称作\textbf{指数函数(exponential function)},其中 $a\in\mathbb (0,1)\cup(1,+\infty)$。
\end{definition}

这里之所以如此要求参数$a$,是因为负数的实数幂次在非常多点上是未定义的,造成函数的定义域不连续,难以研究\footnote{类比狄利克雷函数可知,这个函数无法画出图像。}。而$a=1$和$a=0$的情况是平凡的,分别是直线$y=1$和$y=0(x\neq0)$,它们仅会在后面的研究中作为一个参考基准。

\subsection{指数函数的性质}

有了幂函数的经验,同样,下面根据$a(a>0)$的性质讨论指数函数的性质。

\subsubsection{定义域和值域}

根据指数函数参数的限定,$x$可以取任意实数。此时根据幂运算的法则,$f(x)>0$恒成立,即函数的值域是$(0,+\infty)$。根据$0^a=1$可知,函数恒过定点$(0,1)$。

\subsubsection{单调性}

对于$f(x)=a^x$,任取定义域上的两点$x_1>x_2$,则平均变化率为

\begin{equation}\label{eq_HsExpF_1}
\frac{f(x_1)-f(x_2)}{x_1-x_2}=\frac{a^{x_2}(a^{x_1-x_2}-1)}{(x_1-x_2)}~.
\end{equation}

由于\autoref{eq_HsExpF_1} 的分母和$a^{x_2}$为正,因此讨论$a^{x_1-x_2}$和$1$的关系。由于$x_1>x_2$,$x_1-x_2>0$。为了讨论方便,取$1^{x_1-x_2}$。由于幂函数在参数为正时在第一象限内是增函数,因此$a>1$时,$a^{x_1-x_2}>1^{x_1-x_2}$,$0<a<1$时,$a^{x_1-x_2}<1^{x_1-x_2}$。

综上,$a>0$时,\autoref{eq_HsExpF_1} 的值大于$0$,函数在定义域上是递增的;反之$0<a<1$时,函数则是递减的。

\subsubsection{不同的$a$的函数图象的关系}

当$a_1>a_2$时,有$a_1-a_2>0$,讨论两个函数间的关系\footnote{下面的写法表示参数是$a_1,a_2$。}:

\begin{equation}
f(x;a_1)-f(x;a_2)=(a_1)^x-(a_2)^x=(a_2)^x\left(\left(\frac{a_1}{a_2}\right)^x-1\right)~.
\end{equation}

由于$\displaystyle (a_2)^x>0$,取$p=\frac{a_1}{a_2}>1$,下面讨论$p^x$和$1$的关系。对$x>0$时,$p^x>1$,$x<0$时,$\displaystyle p^x=\left(\frac{1}{p}\right)^{|x|}<1$。

形象地来说,在$x>0$时,如果从下到上画一条垂线,则会先穿过$a$值较小的函数图象,后穿过$a$值较大的函数图象;在$x<0$时从下到上画一条垂线,则会先穿过$a$值较大的函数图象,后穿过$a$值较小的函数图象。

由于$1^x=1$,根据上面的分析,当$a>1$时,$x<0$时$0<y<1$,$x>0$时$y>1$;当$0<a<1$时,$x<0$时$y>1$,$x>0$时$0<y<1$。

仔细对比,可以发现这里“单调性”和“不同的$a$的函数图象的关系”的讨论与幂函数讨论的表达式形式正好相反。

\subsubsection{其他性质}

由于$\displaystyle \left(\frac{1}{a}\right)^{-x}=a^{x}$,所以如果代入$(-x,y)$到$\displaystyle\left(\frac{1}{a}\right)^{x}$的表达式中,得到的它关于$y$轴对称的函数是$a^{x}$。一般地,根据指数运算的性质,有$f(x;a^b)=\left(a^b\right)^x=a^{bx}=f(bx;a)$\footnote{$f(x;a^b)$这种写法表示函数的参数是$a^b$,变量是$x$。}。假设 $c = a^b$,则有 $f(x; c) = f(bx; a)$,即任意的指数基数 $a$ 都可以通过\aref{伸缩变换}{sub_HsFunC_4}来表示所有的指数函数,这也表明不同指数函数之间存在放缩关系。

顺便一提,$x$轴(也就是直线$y=0$)是函数的一条渐近线,分别对应$a>1$,$x$趋于$-\infty$时和$1>a>0$,$x$趋于$+\infty$时。

事实上,若$a>1$,则$a^x$在$x$趋于$-\infty$时,趋于$0$;在$x$趋于$+\infty$时,趋于$+\infty$。若$1>a>0$,则$a^x$在$x$趋于$-\infty$时,趋于$+\infty$;在$x$趋于$+\infty$时,趋于$0$。这是根据幂运算的性质得到的。同样关于“趋于”、“无穷”、“\enref{渐近线}{Asmpto}”这三个词,现在只需要有一个感性的理解就可以了,它是符合几何直观的。关于他们的具体内涵,会在大学阶段学习。

\subsubsection{函数图像}

根据上面的分析可以得到两类函数的图像,分别是$a>1$和$0<a<1$情况的。
\addTODO{图像}
函数是光滑的,并且$|\ln(a)|$\footnote{这里的符号是\enref{对数}{Ln},此处如果看不懂可以先跳过。}越大,图像越会靠近$y$轴;$|\ln(a)|$越小,图像越会靠近直线$y=1$。

\subsection{自然常数$\E$}

到这里就需要引入一个新的常数——自然常数$\E$了。他是继小学接触的$\pi$之后,在数学科目中第二个出现的常数,它们之间有很多相似之处,也有很紧密的关系。

\subsubsection{引入}

自然常数一般会通过下面这个例子来引入。假设在银行存入$1$元,银行承诺年利率为$100\%$,利息的计算公式是“$\text{利息}=\text{本金}\times\text{年利率}\times\text{存款年数(时间)}$”。下面的计算不要关注每次计算的得到的数值,而是要关注计算过程的变化整合。

最简单的情况是银行一年只结算一次利息,这时年末得到的收入就是$1\times100\%\times1=1$。这样,一年后1元会变成$1+1\times100\%=1\times(1+100\%)^1=2$元。

如果要求银行“每半年结算一次利息” ,这样计算的好处是,第一次结算之后的利息会作为本金参与到第二次的计算中。于是第一次结算时,利息为$\displaystyle1\times100\%\times\frac{1}{2}=0.5$。第二次计息时的本金变成了$\displaystyle1+1\times100\%\times\frac{1}{2}=1.5$。第二次的利息就是$\displaystyle(1+1\times100\%\times\frac{1}{2})\times100\%\times\frac{1}{2}=\frac{1.5}{2}=0.75$。于是最终的收入变成了$\displaystyle1+1\times100\%\times\frac{1}{2}+(1+1\times100\%\times\frac{1}{2})\times100\%\times\frac{1}{2}=\frac{1.5}{2}=2.25$。相比第一次多出来的那$0.25$就是因为计息次数变多带来的\footnote{因此银行也会调低短期存储的利率,以降低短期储蓄的收入来保证长期投资的权益。}。整理一下其实就是$1\times(1+\frac{100\%}{2})^2=2.25$。

可以自己试一试“每三个月一次”和“每个月一次”,整理后从银行取出的金额$A(n)$都会变成如下的形式:
\begin{equation}
A(n)=a\times\left(1+\frac{1}{n}\right)^n~.
\end{equation}
其中$a$是本金,$n$是一年计息的次数。

上面算过$A(1)=2,A(2)=2.25$,另外上面提到的“每三个月一次”和“每个月一次”分别对应$A(4)\approx 2.4414$、$A(12)\approx 2.6130$。可以看出$A(n)$似乎是不断递增的,于是不禁要问,这个函数会一直递增吗?有没有最大值?通过计算器计算可知每天一次、小时一次和每秒一次分别是$A(365)\approx 2.7146$、$A(8760)\approx2.7181$、$A(31536000)\approx 2.718266$。看上去好像逐渐停止在小于$2.72$的某个数字了,但如果仅凭这样的直觉,可能会出错。

数学家们证明了,这个函数是一直递增的,而且有一个上限,称这个函数的上限\footnote{确切的说叫上确界。}为\textbf{自然常数(Natural Constant)},也称作\textbf{欧拉数(Euler's number)},记作$\E$,定义为:
\begin{equation}
\E = \lim_{n \to \infty} \left( 1 + \frac{1}{n} \right)^n~.
\end{equation}
如果看不懂这个表达方式没问题,只需要理解它是上面那个存款过程的最终结果,是一个常数,就可以。它描述了增长速度的极限,不仅与利息有关,它还出现在很多自然现象中,简单来说,$\E$代表着在不受限制的情况下,某种东西增长到最快时能达到的程度。

\subsubsection{对比}

自然常数$\E \approx 2.71828$,它和早已在小学时就接触过的$\pi$有许多相似点。

他们都是无理数,这意味着它们不能表示为两个整数的比值。它们的小数部分是无限且不循环的,也就是说,在任何整数进制中它们都永远不会终止或重复。

他们也都是超越数,意思是它们不能作为任何\aref{有理方程}{sub_HsEquN_1}的解。换句话说,这比无理数的要求更加严格。它们不仅不能表示为整数之比,也不能通过各阶的根式表示。$\E$的超越性由查尔斯·埃尔米特(Charles Hermite)在1873年证明,$\pi$的超越性由费迪南德·冯·林德曼(Ferdinand von Lindemann)在1882年证明。

二者都可以用无穷展开的方式来表示,下面给出两个常见的展开方式\footnote{关于求和符号可以参考\enref{求和符号(高中)}{SumSym},关于阶乘可以参考\enref{阶乘(高中)。}{factor}}:
\begin{equation}
\pi=4\sum_{n=0}^\infty\frac{(-1)^i}{2i+1}~.
\end{equation}
\begin{equation}
\E=\sum_{n=0}^\infty\frac{1}{i!}~.
\end{equation}

$\E$的定义有很多种方式,除了之前提到的广为了解的极限定义。下面将给出另一个定义:$\E$ 是使得
\begin{equation}
f'(x) = f(x)~.
\end{equation}
成立的指数函数的底数,这意味着以 $\E$ 为底的指数函数是唯一的能够保持自身增长速度不变的函数。

\subsection{指数爆炸}

\textbf{指数爆炸(exponential growth )}指的是函数值随自变量呈指数级别的快速增长,它的显著特征是初期增速缓慢,但随后会急剧加速。指数函数的增长速度非常快,对于初等函数而言,当参数 x 足够大时,指数函数的增长速度是最快的。具体来说,若参数 $a > 1$,在第一象限内($x > 0$)的典型函数增长速度从慢到快通常满足以下顺序:

\begin{equation}
 a < \log_a{x} <x^a < a^x~.
\end{equation}

式子中,常数 $a$ 是一个固定值,不随 $x$ 改变,或者说不增加。\enref{对数函数}{Ln} $\log_a{x}$ 的值在 $x$越大时,仍在增加,但增速会越来越慢,仅略大于不增。\enref{幂函数}{power} $x^a$ 和指数函数 $a^x$的增长速度都会随着 $x$ 增加,$x^a$ 增速逐渐加快,但$x^a$比指数函数 $a^x$ 慢,一般认为相较于指数函数,幂函数是线性或近似线性的。指数增长会呈现“爆炸式”的加速,远超其他初等函数。

这是一个一般规律,在定性地判断函数值和趋势时会比较有效,熟练使用会在很多题目中快速确定解题方向。

\begin{example}{已知函数$f(x)=(x-2)e^x,g(x)=a(x-1)^2$,讨论$a$与二者交点的关系。}
下面进行定性分析:

易知$f(0)=-2,f(2)=0$,且在函数趋于$-\infty$时,函数趋于$0$,在函数趋于$+\infty$时,函数趋于$+\infty$(这里的无穷判断就与指数爆炸相关,当两个函数相乘时,在无穷处一般可以根据上面的增长速度判定。)。而$g(x)$是一个开口方向及敞口大小与$a$相关的抛物线,$|a|$越大,敞口越小,唯一的零点在$x=1$处。

当$a>0$时,开口向上,此时两个函数只能有一个交点。

首先,二者必然相交。因为在$x=1$处,显然$f(1)<0=g(1)$,而$f(x)$的增长速度比$g(x)$快,哪怕$g(x)$的敞口再小,总有一个点之后,$f(x)$会在$g(x)$上方。又因为相交后,二者就会快速分开,所以二者只能相交一次。

当$a>0$时,开口向上,此时两个函数有两个交点。

从上面的分析,$f(2)=0,f(-\infty)=0,f(1)<0$,而$g(2)<0,g(-\infty)<0,g(1)=0$。很显然,在$(-\infty,1),(1,2)$两个区间上,各自会存在零点。同时,由于$f(x)$在这一段上与$g(x)$相比近乎一条水平直线,于是每个区间上也只会存在一个零点。

\end{example}

\subsection{柯西函数方程}

\textbf{柯西函数方程 (Cauchy functional equation)}是柯西提出的是数学分析中具有加性和乘性特征的几个方程。它们的形式如下:
\begin{enumerate}
\item $f(x+y)=f(x)+f(y)$
\item $f(xy) = f(x) f(y)$
\item $f(x+y)=f(x)f(y)$
\item $f(xy) = f(x)+f(y)$
\end{enumerate}

刚看到可能觉得有点吓人,下面以第一个作为例子表示一下它的含义:函数$f(x)$满足,任取两个自变量的值时,这两个自变量$x,y$的和$x+y$对应的函数值$f(x+y)$与他们对应的函数值$f(x),f(y)$的和$f(x)+f(y)$相等。这里的$y$只表示某一个可以任意赋值的,与$x$无关的自变量。

由于他们无关,可以任意给他们赋值来研究它的性质:
\begin{itemize}
\item 令$y=x$,有$f(2x)=2f(x)$,进而更多地得到对自然数$n$有$f(nx)=nf(x)$,代入$x=1$有$f(n)=nf(1)$。
\item 令$x=y=0$,有$f(0)=2f(0)$,因此$f(0)=0$,即函数恒过定点$(0,0)$。
\item 令$y=-x$,有$f(0)=f(x)+f(-x)$,又因为$f(0)=0$,从而$-f(x)=f(-x)$,这说明函数应该是奇函数。
\item 令$\displaystyle x=\frac{q}{p}$,即$q=px$,从而$f(q)=f(px)$。由于$p$是整数,根据上面的性质有$f(px)=pf(x)$。又因为$q$为整数,有$f(q)=qf(1)$。综上$\displaystyle f(x)=\frac{f(q)}{p}=\frac{q}{p}f(1)=xf(1)$
\end{itemize}

上面的方法是研究此类抽象函数的一种普遍方法,通过上面的推理已经可以证明$f(x)=xf(1)$对任意有理数$x$都成立,设$f(1)=a$的话,可知函数表达式为$f(x)=ax$。事实上,这个表达式对$x\in\mathbb{R}$都成立,不过证明就超过高中范畴了。

其他的方程也可以通过类似的方法进行研究,不过好在,这些方程已经被很多人研究过,下面直接给出各个函数的某种解的形式\footnote{可能有其他的解,但不在此处的讨论范畴内。},可以自行带入验证:

\begin{enumerate}
\item $f(x+y)=f(x)+f(y)\implies f(x)=ax$(正比例函数、线性函数)
\item $f(xy)=f(x)f(y)\implies f(x)=x^a$(\enref{幂函数}{power})
\item $f(x+y)=f(x)f(y)\implies f(x)=a^x$(指数函数)
\item $f(xy)=f(x)+f(y)\implies f(x)=\log_ax$(\enref{对数函数}{Ln})
\end{enumerate}

其中$a$是一个参数。因此,也可以说,在某个视角上,这几种函数根本的性质就是柯西函数方程描述的样子。
