% C++ 标准库常用算法
% license Xiao
% type Tutor

\begin{issues}
\issueDraft
\end{issues}

\begin{itemize}
\item \href{https://en.cppreference.com/w/cpp/container/vector/assign}{vector::assign}: 
\item \href{https://cplusplus.com/reference/algorithm/copy/}{copy}: 复制一段元素 \verb`copy(iter1_beg, iter1_end, iter_result);`
\item \href{https://cplusplus.com/reference/algorithm/move/}{move}: \verb`move (iter_beg, iter_end, iter_result);` 对每个元素用 \verb`std::move`
\begin{lstlisting}[language=cpp]
template<class InputIterator, class OutputIterator>
  OutputIterator move (InputIterator first, InputIterator last,
      OutputIterator result) {
  while (first!=last) {
    *result = std::move(*first); ++result; ++first;
  }
  return result;
}
\end{lstlisting}
\item \href{https://cplusplus.com/reference/algorithm/sort/}{sort}: \verb`sort(iter_beg, iter_end, Compare(可选))` 升序排列, 其中 compare 可以是函数指针, functor, lambda 等。 \verb`iter` 需要支持随机访问。
\item \href{https://cplusplus.com/reference/algorithm/find/}{find}: \verb`iter = find(iter_start, iter_end, elm)` 寻找第一个符合的元素, 如果找到返回 iterator, 否则返回 \verb`iter_end` 元素需要定义 \verb`operator==`, iterator 需要支持 forward。
\item \verb`swap(容器1, 容器2)`, 一般来说是不会逐个元素 copy, 只会交换容器指针。 数组除外。
\begin{lstlisting}[language=cpp]
template <class T> void swap (T& a, T& b)
{ T c(std::move(a)); a=std::move(b); b=std::move(c); }
template <class T, size_t N> void swap (T (&a)[N], T (&b)[N])
{ for (size_t i = 0; i<N; ++i) swap (a[i],b[i]); }
\end{lstlisting}
\end{itemize}
