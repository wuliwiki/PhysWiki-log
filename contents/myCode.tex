% 小时百科程序框架简介

网页版百科分为两个界面:\textbf{原界面} \verb|wuli.wiki/online| 由静态 html 网页构成,每个词条的网址如 \verb|wuli.wiki/online/AU.html| 其中 AU 是词条的 id, 每个词条都用固定不变的 id, 即使词条被重命名也不会变.  \textbf{新界面} \verb|wuli.wiki/book| 相当于在原界面上套壳,每天爬取 \verb|wuli.wiki/online/| 中的内容进行更新, 增加了评论,搜索等功能, 目录也更为美观. 新界面中词条的网址如 \verb|wuli.wiki/book/AU|.
词条编辑器 \verb|wuli.wiki/editor| 是一个相对独立的模块,百科词条的作者使用编辑器用 LaTeX 语言写作, 编辑器把 LaTeX 源码转为 \verb|wuli.wiki/online/| 目录中的 html 页面.

所有代码通过 \href{https://github.com/wuliwiki}{GitHub 账号}管理. 以下是各个仓库的功能
\begin{itemize}
\item \href{https://github.com/MacroUniverse/littleshi.cn}{littleshi.cn}: 网站中所有静态页面, 服务器路径 \verb|/var/www/littleshi.cn|
\item \href{https://github.com/MacroUniverse/PhysWiki}{PhysWiki}: 百科词条的源代码, 使用 LaTeX 语言, 支持用 TeXlive 直接编译 pdf, 服务器路径 \verb|/var/www/PhysWiki|
\item \href{https://github.com/MacroUniverse/PhysWikiScan}{PhysWikiScan}: 一个后台 C++ 命令行程序, 用于把 PhysWiki 中的 LaTeX 源码文件转为 littleshi.cn 中的静态的百科 html 页面, 可以单个转换也可以批量转换.服务器路径 \verb|/var/www/PhysWikiScan|
\item \href{https://github.com/MacroUniverse/littleshi.cn-server}{littleshi.cn-server}: 百科编辑器 wuli.wiki/editor 的源码, 调用 PhysWikiScan 程序. 服务器路径 \verb|/var/www/editor|, 使用 node.js 开发.
\item \href{https://github.com/MacroUniverse/PhysWiki-backup}{PhysWiki-backup}: 用于存放百科词条备份,每个词条页面根据该备份记录给作者排序.服务器路径 \verb|/var/www/PhysWiki-backup|
\item \href{https://github.com/MacroUniverse/wuliwiki-web}{wuliwiki-web}:网站中除了词条编辑器外的所有动态页面,以及后台. 当前在服务器中用 docker 运行.
\item \href{https://github.com/MacroUniverse/wuliwiki-app}{wuliwiki-app}: 百科移动端 app, 用 flutter 开发, 支持安卓、iPhone、iPad
\end{itemize}
