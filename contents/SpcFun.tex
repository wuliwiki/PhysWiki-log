% 函数求值
% 数值计算|函数|解析|泰勒展开|切比雪夫|连续分数

函数求值归根结底, 就是找到一个解析公式, 用加减乘除表示一个函数。 如果算法没有用到近似, 有限精度和任意精度都是一样的, 只是每个数占的内存不同罢了。 任意精度的加减乘除和开根号参考 Numerical Recipes 最后一节。

最常见的展开有三种, 分别是泰勒展开, 渐进展开, 以及连续分数。 这些展开在 \href{http://functions.wolfram.com}{functions.wolfram.com} 都可以找到。

接下来是如何计算级数或连续分数。 求级数的方法一般是用
\begin{equation}
f_N(x) = \sum_{n = 0}^N c_n x^n = (\dots ((c_n x + c_{n-1})x + c_{n-2} \dots )x + c_0
\end{equation}
这么做误差和直接对多项式求和一样, 但计算量却少了很多。 将上式中的 $x$ 换成 $1/x$ 就是渐进展开的形式。 注意我们可以在计算开始前估计误差, 根据精度要求得到我们需要的项数。

显然, $\abs{x}$ 越小时泰勒展开收敛得越快, 而 $\abs{x}$ 越大时渐进展开收敛得越快。

再来看连续分数
\begin{equation}
f_N(x) = b_0 + \dfrac{a_1}{b_1 + \dfrac{a_2}{b_2 + \dots \dfrac{a_N}{b_N}}}
\end{equation}
也可以表示为
\begin{equation}
f_N(x) = b_0 + \frac{a_1}{b_1 +} \frac{a_2}{b_2 +} \frac{a_3}{b_3 +} \dots \frac{a_N}{b_N}
\end{equation}
其中 $a_n$ 和 $b_n$ 都可以是 $x$ 的函数。

乍看之下连续分数只能从右往左求, 其实不然。 由递归法可以证明
\begin{equation}
f_n = \frac{A_n}{B_n}
\end{equation}
其中
\begin{equation}
A_n = b_n A_{n-1} + a_n A_{n-2} \qquad
B_n = b_n B_{n-1} + a_n B_{n-2}
\end{equation}
\begin{equation}
A_{-1} = 1 \qquad B_{-1} = 0
\qquad A_0 = b_0 \qquad B_0 = 1
\end{equation}
还有一种 Steed's 方法, 详见 Numerical Recipes。 正向求和的好处是可以判断什么时候开始收敛。
据说特定情况下连续分数收敛较快, 但具体什么时候用还有待考察。

另外两种不明觉厉的算法分别是\textbf{切比雪夫多项式(Chebyshev Polynomial)}和\textbf{算数几何平均(Agorithmic-Geometric Mean, AGM}, 后者也通常被用于计算高精度的 $\pi$, 且有二次收敛。
