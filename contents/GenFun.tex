% 复值测度与广义函数
% keys Lebesgue测度|广义函数|绝对连续|Lebesgue积分|狄拉克函数|狄拉克测度|Dirac函数|Dirac测度
% license Xiao
% type Tutor

\pentry{Lebesgue 积分\nref{nod_Lebes1},映射\nref{nod_map}}{nod_2c05}

在\textbf{集合的测度(实变函数)}\upref{SetMet}中,我们了解了Lebesgue外测度。它是将“开集的体积”进行推广而得来的,其定义思路也有明显的物理对应:用容器去衡量被容纳物的大小。具象的概念虽然容易想象,但因其具象也束缚了概念范围,所以我们在得到一个新的概念时,总会想把它的核心特征抽离出来,抽象出一个更一般的概念,看看能不能引申出有趣的理论。

Lebesgue外测度该怎么抽象呢?抛去其定义的方式不谈,它就是用来衡量“集合的体积”的,对不对?体积是一个数字,那我们就把测度看成是给各集合赋予一个数字,也就是“集合函数”\footnote{注意这个术语的意思:集合函数是指把集合映射到数字上的映射。值域是数字的映射,通常又称为函数。}。再考虑一些集合体积所具有的性质,我们可以构造出这样一个定义:

\begin{definition}{非负测度}\label{def_GenFun_1}
设 $S$ 是一个集合,$\mathcal{A}$ 是由 $S$ 的子集构成的一个 $\sigma$-代数\footnote{即用 $\mathcal{A}$ 中元素进行任意多次的交、并、差、补等运算,结果仍在 $\mathcal{A}$ 中。}。称映射 $\mu:\mathcal{A}\to [0, +\infty]$ 是 $(S, \mathcal{A})$ 上的一个\textbf{非负测度(non-negative measure)},如果它满足:
\begin{enumerate}
\item $\mu(\varnothing)=0$;\\
\item 对于两两不交的至多可数个 $A_i\in\mathcal{A}$,有
\begin{equation}
\mu\qty(\bigcup_i A_i) = \sum_i \mu(A_i)~.
\end{equation}
\end{enumerate}

% 通常将\textbf{非负测度}简称为\textbf{测度(measure)}。

三元组 $(S, \mathcal{A}, \mu)$ 也被称为一个\textbf{测度空间(measure space)}。

\end{definition}

注意,定义中测度 $\mu$ 的值域是 $[0, +\infty]$ 而非 $[0, +\infty)$,意味着测度值也可以取广义实数 $+\infty$。由于测度值非负,因此在不至于混淆的时候,也可以用 $\infty$ 代替 $+\infty$。

有非负测度,意味着测度的概念还可以继续推广:

\begin{definition}{复值测度}
将非负测度的定义域从 $[0, +\infty]$ 改为 $\mathbb{C}$,即得到\textbf{复值测度}。
\end{definition}

\begin{example}{计数测度}
设 $S$ 为任意集合,定义 $(S, 2^S)$ 上的一个测度 $\mu$ 如下:
\begin{equation}
\mu(A)=
\leftgroup{
    \abs{A},\quad \abs{A}<\aleph_0\\
    +\infty,\quad \abs{A}\geq\aleph_0
}~
\end{equation}
称其为 $(S, 2^S)$ 上的\textbf{计数测度}。
\end{example}

\autoref{def_GenFun_1} 比Lebesgue测度的定义抽象多了,因此有必要举一个和Lebesgue测度截然不同的例子,来体现其“更一般更抽象”的特点:

\begin{example}{Dirac测度}
令 $\mathcal{B}$ 为 $\mathbb{R}^n$ 中的Borel集合\footnote{即用开区间进行任意多次交、并、差、补等运算获得的集族。}。取 $\mathbb{R}^n$ 中的一点 $x_0$,定义 $(\mathbb{R}^n, \mathcal{B})$ 上的测度 $\sigma_{x_0}$ 为:
\begin{equation}
\sigma_{x_0}(A)=\leftgroup{
    1, \quad x_0\in A\\
    0, \quad x_0\not\in A
}~
\end{equation}
称之为\textbf{Dirac测度}。
\end{example}

Dirac测度显然连平移不变性都不具备,和Lebesgue外测度截然不同。你可能也注意到了,Dirac测度和Dirac函数看起来很相似——没错,Dirac函数本质上就不是一个函数,而是一个测度。但这话乍一听很奇怪,函数是给集合上每个点赋予一个数字,测度是给各子集赋予数字,两者是怎么联系起来的呢?这就需要利用Lebesgue外测度作为脚手架,来衔接二者了。





\subsection{函数与测度}




\begin{definition}{测度的绝对连续}
设 $\mu$ 和 $\nu$ 都是 $(S, \mathcal{A})$ 上的测度,且只要 $\nu(A)=0$,就必有 $\mu(A)=0$,那么称测度 $\mu$ 关于测度 $\nu$ 绝对连续,记为 $\mu\ll\nu$。

如果 $\mu\ll\nu$ 且 $\nu\ll\mu$,则称 $\mu$ 与 $\nu$ 等价。
\end{definition}

\addTODO{需补充或引用\textbf{Radon-Nikodym定理}相关文章。}


Radon-Nikodym定理显示,如果 $\mu\ll\nu$,且 $\nu$ 是 $\sigma$\textbf{-有限测度}\footnote{即 $S$ 可以表示为最多可数个 $B_i\in\mathcal{A}$ 的并集,且 $\nu(B_i)<+\infty$ 恒成立。},那么 $\mu$ 就可以和一个函数 $f:S\to [0, +\infty]$ 相联系起来:
\begin{equation}
\mu(A) = \int_A f \dd \nu~.
\end{equation}

我们可以取所研究的二元组 $(S, \mathcal{A})$ 为 $(\mathbb{R}^n, \mathcal{B})$,$\nu$ 为Lebesgue外测度,于是 $\int_A f \dd \nu$ 就是 $f$ 的Lebesgue积分。我们可以把 $f$ 称为 $\mu$ 的关于 $\nu$ 的\textbf{密度(density)}。

反过来,对于一个Lebesgue可积函数 $f$,我们总可以定义与之相关的测度:
\begin{equation}
\mu(A) = \int_A f(x) \dd x~.
\end{equation}

如果 $f$ 是测度 $\mu$ 的密度,那么任意函数 $\varphi$ 关于测度 $\mu$ 的积分就可以写为:
\begin{equation}
\int_A \varphi \dd \mu = \int_A f(x)\varphi(x) \dd x~.
\end{equation}

于是,我们可以将测度 $\mu$ 引申为一个泛函:
\begin{equation}
\mu(\varphi) = \int_{\mathbb{R}^n} f(x)\varphi(x) \dd x~.
\end{equation}

今后,我们将不再区分测度 $\mu$ 及其关于Lebesgue测度的密度 $f$。泛函 $\mu(\varphi)$ 有时候也表示为 $f(\varphi)$。











