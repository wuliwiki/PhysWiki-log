% 迈克尔·法拉第(综述)
% license CCBYSA3
% type Wiki

本文根据 CC-BY-SA 协议转载翻译自维基百科\href{https://en.wikipedia.org/wiki/Stokes\%27_theorem}{相关文章}。

迈克尔·法拉第(Michael Faraday,/ˈfærədeɪ, -di/;1791年9月22日-1867年8月25日)是一位英国物理学家和化学家,对电磁学和电化学的研究作出了重要贡献。他的主要发现包括电磁感应、抗磁性和电解等基本原理。尽管法拉第接受的正式教育很少,他作为一个自学成才的人,成为历史上最有影响力的科学家之一。[1] 通过研究直流电导体周围的磁场,法拉第在物理学中确立了电磁场的概念。他还发现了磁性可以影响光线,并揭示了两者之间的内在联系。[2][3] 他同样发现了电磁感应、抗磁性和电解定律的基本原理。他发明的电磁旋转装置奠定了电动机技术的基础,并且主要由于他的努力,使电的应用在技术中成为可能。[4]

作为一名化学家,法拉第发现了苯,研究了氯的包合水合物,发明了早期版本的本生灯以及氧化数系统,并推广了“阳极”、“阴极”、“电极”和“离子”等术语。最终,法拉第成为英国皇家学会的首任兼最重要的富勒讲座化学教授(Fullerian Professor of Chemistry),这一职位是终身制的。

法拉第是一位实验科学家,他以清晰简洁的语言表达自己的想法。他的数学能力并不高深,仅限于基础代数,甚至未涉猎三角学。詹姆斯·克拉克·麦克斯韦(James Clerk Maxwell)基于法拉第及其他人的工作,总结出一组方程,这些方程被认为是所有现代电磁现象理论的基础。关于法拉第使用“力线”的方法,麦克斯韦写道,这表明法拉第“实际上是一位非常高水平的数学家——未来的数学家可以从他那里获得宝贵而富有成果的方法。”[5] 国际单位制中电容的单位“法拉”(farad)就是以他的名字命名的。

阿尔伯特·爱因斯坦在书房的墙上挂着法拉第的画像,与艾萨克·牛顿和詹姆斯·克拉克·麦克斯韦的画像并列。[6] 物理学家欧内斯特·卢瑟福(Ernest Rutherford)曾说:“当我们考虑他的发现的广度与深度,以及它们对科学与工业发展的影响时,再大的荣誉也不足以表达对法拉第——这一有史以来最伟大的科学发现者之一——的敬意。”[1]
\subsection{传记}
\subsubsection{早年生活}
迈克尔·法拉第于1791年9月22日出生在萨里的纽因顿巴茨(Newington Butts),该地区现为伦敦南华克区的一部分。[7][8] 他的家境贫困。他的父亲詹姆斯·法拉第(James Faraday)是一个基督教格拉塞派(Glasite sect)的成员。1790年冬天,詹姆斯带着妻子玛格丽特(原姓哈斯特威尔,Margaret Hastwell)[9]和两个孩子从西摩兰郡的奥思吉尔(Outhgill)搬到伦敦。在奥思吉尔,詹姆斯曾是村庄铁匠的学徒。[10] 迈克尔是在次年秋天出生的,是家中四个孩子中的第三个。年幼的法拉第仅接受过最基本的学校教育,因此他不得不自学成才。[11]

14岁时,法拉第成为布兰德福德街(Blandford Street)当地书籍装订匠兼书商乔治·里博(George Riebau)的学徒。[12] 在七年的学徒期内,法拉第阅读了许多书籍,包括艾萨克·沃茨(Isaac Watts)的《提升心智》(*The Improvement of the Mind*)。他充满热情地实践了书中提出的原则和建议。[13] 在此期间,法拉第与同伴们在“城市哲学会”(City Philosophical Society)中进行讨论,并参加了关于各种科学主题的讲座。[14] 他对科学,特别是对电学产生了浓厚的兴趣。法拉第尤其受到简·马西特(Jane Marcet)所著的《化学对话》(*Conversations on Chemistry*)一书的启发。[15][16]
\subsubsection{成年生活}
\begin{figure}[ht]
\centering
\includegraphics[width=6cm]{./figures/80600dcd950a25f1.png}
\caption{1842年托马斯·菲利普斯绘制的法拉第肖像} \label{fig_FLD_1}
\end{figure}
1812年,20岁的法拉第在学徒期结束时,参加了英国皇家学会和皇家学院著名化学家汉弗里·戴维(Humphry Davy)以及“城市哲学会”创始人约翰·塔图姆(John Tatum)的讲座。这些讲座的许多门票由皇家爱乐学会的创始人之一威廉·丹斯(William Dance)赠送给了法拉第。随后,法拉第将他根据这些讲座笔记整理成的一本300页的书寄给了戴维。戴维很快回信,态度友好且积极。1813年,戴维在一次与三氯化氮的实验中发生事故,导致视力受损,他决定雇用法拉第作为助手。巧合的是,皇家学院的一位助理约翰·佩恩(John Payne)被解雇了,而戴维正被要求寻找替代者,因此他于1813年3月1日任命法拉第为皇家学院的化学助理。[2] 不久之后,戴维委托法拉第准备三氯化氮样品,他们两人在处理这种极为敏感的物质时都在一次爆炸中受了伤。[17]

1821年6月12日,法拉第与萨拉·巴纳德(Sarah Barnard,1800–1879)结婚。[18] 他们通过各自的家庭在桑德曼教会(Sandemanian church)中相识。结婚一个月后,法拉第向桑德曼教会的会众表白了他的信仰。他们没有子女。[7] 法拉第是一位虔诚的基督徒,其所属的桑德曼教派是苏格兰教会的一个分支。婚后多年,他曾在少年时参加的教会中担任执事,并两次担任长老。他所在的教会位于巴比肯(Barbican)的保罗巷(Paul's Alley)。1862年,这座教堂搬迁到了北伦敦伊斯灵顿的巴恩斯伯里格罗夫(Barnsbury Grove)。法拉第在担任第二任长老的最后两年时间里一直服务于这个新地点,直到辞去职务。[19][20] 传记作者指出,“对上帝和自然统一性的强烈信念贯穿了法拉第的生活和工作。”[21]
\subsubsection{晚年生活}
1832年6月,牛津大学授予法拉第名誉民法博士学位。在他的一生中,他因对科学的贡献被提议授予爵士头衔,但由于宗教原因,他拒绝了这一荣誉。他认为根据《圣经》的教义,积累财富和追求世俗奖励是错误的,并表示自己更愿意一直保持“普通的法拉第先生”身份。[22] 法拉第于1824年当选为皇家学会会士,但他两次拒绝担任会长职务。[23] 1833年,他成为皇家学院首任富勒讲座化学教授(Fullerian Professor of Chemistry)。[24]  

1832年,法拉第被选为美国艺术与科学学院的外籍名誉会员。[25] 1838年,他被选为瑞典皇家科学院外籍院士。1840年,他当选为美国哲学会会员。[26] 1844年,他成为法国科学院八位外籍院士之一。[27] 1849年,他被选为荷兰皇家研究院的准会员,该研究院两年后更名为荷兰皇家艺术与科学学院,随后法拉第被授予外籍院士身份。[28]

1839年,法拉第经历了一次神经崩溃,但最终他重新投入到对电磁学的研究中。[29] 1848年,在阿尔伯特亲王的建议下,法拉第获赠了一栋位于米德尔塞克斯汉普顿宫的恩典与厚爱之屋(grace and favour house),完全免除了所有费用和维护支出。这栋房屋是原来的石匠总监之家(Master Mason's House),后来被称为法拉第之家(Faraday House),现为汉普顿宫路37号(No. 37 Hampton Court Road)。1858年,法拉第退休后便搬到这里居住。[30]

法拉第为英国政府提供了许多服务项目,但当政府要求他就克里米亚战争(1853–1856)中化学武器的生产提供建议时,法拉第以伦理原因拒绝参与。[31] 他还拒绝了出版其讲座的提议,认为如果没有现场实验的配合,讲座内容的影响力会减弱。他在给出版商的一封信中写道:“我一直热爱科学胜过金钱,因为我的职业几乎完全是个人的,我无法承担变得富有的代价。”[32]

1867年8月25日,法拉第在汉普顿宫的住所中去世,享年75岁。[33] 在数年前,他拒绝了死后安葬于威斯敏斯特教堂的提议,但在那里靠近艾萨克·牛顿墓旁有一块纪念牌以纪念他。[34] 法拉第被安葬在海格特公墓的异议者(非英国国教徒)区。[35]
\subsection{科学成就}  
\subsubsection{化学}
法拉第最早的化学工作是在亨弗里·戴维的助手岗位上进行的。法拉第参与了氯的研究;他发现了氯和碳的两种新化合物:六氯乙烷,他通过乙烯的氯化反应制得,以及四氯化碳,它是通过前者的分解产生的。他还进行了关于气体扩散的初步实验,这一现象最早由约翰·道尔顿指出。托马斯·格雷厄姆和约瑟夫·洛施密特更充分地揭示了这一现象的物理重要性。法拉第成功地液化了几种气体,研究了钢的合金,并制造了几种新的光学用玻璃。后来,其中一种重玻璃样本变得具有历史意义;当玻璃置于磁场中时,法拉第确定了光的偏振平面旋转。这块样本也是第一个被磁铁的极性排斥的物质。

法拉第发明了一种早期形式的本生灯,这种装置至今仍在世界各地的科学实验室中作为便捷的热源使用。法拉第在化学领域做了大量工作,发现了化学物质如苯(他称之为氢的双碳化物),并液化了气体如氯。气体的液化帮助确立了气体是具有极低沸点的液体蒸气这一概念,并为分子聚集概念提供了更加坚实的基础。1820年,法拉第报告了首次合成由碳和氯组成的化合物C2Cl6和CCl4,并于次年发布了他的研究结果。法拉第还确定了氯包合水合物的成分,该物质是亨弗里·戴维在1810年发现的。法拉第还因发现电解法则而闻名,并推广了许多术语,如阳极、阴极、电极和离子,这些术语大部分是由威廉·惠威尔提出的。

法拉第是第一个报告后来被称为金属纳米粒子的科学家。1847年,他发现金胶体的光学特性与对应的大块金属有所不同。这可能是第一次报告量子尺寸效应的观察,可以认为这是纳米科学的诞生。
\subsubsection{电学与磁学}  
法拉第最著名的研究领域是电学与磁学。他的第一次实验记录是用七个英国半便士硬币、七片锌片以及六块浸泡在盐水中的纸片堆叠在一起,构建了一个伏打电堆。[47] 通过这个电堆,他将电流通过硫酸镁溶液,成功地分解了这种化学化合物(该实验记录在他写给阿博特的第一封信中,时间是1812年7月12日)。[47]

1821年,在丹麦物理学家和化学家汉斯·克里斯蒂安·厄尔斯特发现电磁现象之后不久,戴维和威廉·海德·沃拉斯顿曾尝试设计电动机,但未成功。[3] 法拉第与两位科学家讨论了这个问题后,着手制造了两种设备,用以产生他所称的“电磁旋转”。其中一台设备,现在被称为同极电动机,利用围绕一根延伸到汞池中的电线的圆形磁场产生了持续的圆周运动,池中放置了一个磁铁;当电池提供电流时,电线就会围绕磁铁旋转。这些实验和发明奠定了现代电磁技术的基础。在兴奋之中,法拉第发布了结果,但没有提及他与沃拉斯顿或戴维的合作。这引发了英国皇家学会内的争议,导致他与戴维的导师关系紧张,也可能是法拉第被分配到其他活动的原因,这使得他在接下来的几年里未能参与电磁研究。[49][50]

从1821年最初的发现开始,法拉第继续他的实验室工作,探索材料的电磁性质并积累所需的经验。1824年,法拉第简要地建立了一个电路,研究磁场是否能调节相邻电线中的电流流动,但他并未发现这种关系。[51] 这一实验与三年前他进行的光与磁体的类似实验得出了相同的结果。[52][53] 在接下来的七年里,法拉第将大部分时间花在完善他用于光学质量(重)玻璃的配方——铅硼硅酸盐玻璃上,[54] 他在未来将其用于连接光与磁性的研究中。[55] 在空闲时间,法拉第继续发表他在光学和电磁学方面的实验研究;他与在欧洲旅行中遇到的科学家保持通信,这些科学家也在从事电磁学的研究。[56] 在戴维去世后的两年,即1831年,法拉第开始了他一系列重要的实验,在这些实验中他发现了电磁感应,并在1831年10月28日的实验日志中记录道:“我正在用皇家学会的大磁铁做许多实验。”[57]

法拉第的突破出现在他将两根绝缘的电线绕在铁环上时,他发现当电流通过一根线圈时,另一根线圈中会感应出短暂的电流。[3] 这一现象现在被称为互感。[58] 该铁环-线圈装置至今仍在皇家学会展出。在随后的实验中,他发现如果将磁铁穿过一根线圈,线圈中会产生电流。如果线圈在静止的磁铁上移动,电流也会流动。他的实验表明,变化的磁场会产生电场;这一关系由詹姆斯·克拉克·麦克斯韦以法拉第定律的形式进行数学建模,后来成为四个麦克斯韦方程之一,并且逐渐发展成今天被称为场论的广义化理论。[59] 法拉第随后利用他所发现的原理,构建了电力发电机,这一装置是现代发电机和电动机的前身。[60]

1832年,他完成了一系列旨在研究电的基本性质的实验;法拉第使用“静电”、电池和“动物电”来产生静电吸引、电解、磁性等现象。他得出结论,认为与当时的科学观点相反,各种“电”的分类是虚幻的。法拉第提出,实际上只有一种“电”存在,电量和强度(电流和电压)的变化会产生不同的现象群体。[3]

在职业生涯接近尾声时,法拉第提出电磁力扩展到导体周围的空旷空间。[59] 这一观点被他的同行科学家所拒绝,法拉第未能亲眼见证他提出的这一假设最终被科学界接受。直到半个世纪后,电力才开始应用于技术领域,位于伦敦西区的萨沃伊剧院,安装了由约瑟夫·斯旺爵士开发的白炽灯泡,成为世界上第一个用电照明的公共建筑。[61][62] 皇家学会记录称,“法拉第在1831年发明了发电机,但直到近50年后,所有技术(包括此处使用的约瑟夫·斯旺的白炽灯泡)才广泛应用。”[63]
\subsubsection{反磁性}
1845年,法拉第发现许多材料表现出对磁场的微弱排斥作用,这一现象他称之为“反磁性”。[65]

法拉第还发现,线性偏振光的偏振平面可以通过施加与光传播方向对齐的外部磁场来旋转。这个现象现在被称为法拉第效应。[59] 1845年9月,他在笔记本中写道:“我终于成功地照亮了一个磁曲线或力线,并且使一束光线磁化。”[66]

在他晚年的1862年,法拉第使用分光镜寻找光的另一种变化,即通过施加磁场改变光谱线。然而,他当时使用的设备不足以明确地确定光谱变化。彼得·齐曼后来使用改进的设备研究了同样的现象,并在1897年发布了他的研究成果,因其成功而获得了1902年诺贝尔物理学奖。在1897年的论文[67]和诺贝尔领奖演讲中,齐曼都提到了法拉第的工作。[68]

\textbf{法拉第笼}  

在他关于静电的研究中,法拉第的冰桶实验表明,电荷仅存在于带电导体的外部,外部电荷对导体内部的任何物体没有影响。这是因为外部电荷重新分布,使得从它们发出的内部电场相互抵消。这种屏蔽效应在今天被称为法拉第笼中得到了应用。[59] 1836年1月,法拉第在四个玻璃支架上放置了一个12英尺见方的木框,并加上了纸墙和金属网。他然后走进框内并对其充电。当他从电荷笼中走出来时,法拉第证明了电是一种力,而不是当时人们认为的不可测量的流体。[4]
\subsection{皇家学会与公共服务}
法拉第与英国皇家学会有着长期的联系。1821年,他被任命为皇家学会院舍的助理监督。[69] 1824年,他被选为皇家学会会员。[7] 1825年,他成为皇家学会实验室的实验室主任。[69] 六年后的1833年,法拉第成为英国皇家学会首任富勒化学教授,这一职位终身任命,且无需承担讲课义务。他的赞助人和导师是约翰·“疯子”·富勒,正是富勒为法拉第在皇家学会创设了这一职位。[70]

除了在皇家学会进行化学、电学和磁学等领域的科学研究外,法拉第还承担了许多为私人企业和英国政府提供的服务项目,这些工作往往既繁琐又耗时。包括调查煤矿爆炸事件、作为专家证人在法庭作证,以及大约在1853年与来自Chance Brothers的两名工程师一起,为该公司准备高质量的光学玻璃,后者用于其灯塔。1846年,法拉第与查尔斯·莱尔共同编写了关于达勒姆郡哈斯威尔煤矿一次严重爆炸的详尽报告,爆炸造成95名矿工死亡。[71] 他们的报告是一次严谨的法医调查,表明煤尘在爆炸的严重性中起到了重要作用。[71] 这是首次将爆炸与煤尘相关联,法拉第在一次讲座中演示了如何通过通风来防止煤尘爆炸。该报告本应警告煤矿老板煤尘爆炸的危险,但这一风险在60多年里被忽视,直到1913年桑赫尼德煤矿灾难发生。[71]

作为一位在拥有强大海洋利益的国家中的受人尊敬的科学家,法拉第花费大量时间参与诸如灯塔的建设和运行、以及保护船底免受腐蚀等项目。他的工作室至今仍位于伦敦的特里尼提浮标码头,位于链条和浮标仓库之上,紧邻伦敦唯一的灯塔,这里是他为灯塔进行电光照明实验的地方。[72]

法拉第还积极参与今天所称的环境科学或工程领域。他调查了斯旺西的工业污染,并曾就皇家铸币厂的空气污染问题提供咨询。1855年7月,法拉第写信给《泰晤士报》,讨论泰晤士河的污浊状况,这封信导致了《Punch》杂志经常刊载的讽刺漫画。(另见《大恶臭》)。[73]

法拉第协助规划和评审1851年伦敦海德公园的大博览会展品。[74] 他还就国家画廊的艺术收藏品清洁和保护工作提供建议,并于1857年担任国家画廊选址委员会成员。[75][76] 教育是法拉第另一个服务领域;他在1854年在皇家学会讲授过相关主题的讲座,[77] 并于1862年出席了一个公立学校委员会,发表了他对英国教育的看法。法拉第还对公众迷恋桌面转动、催眠术和通灵术发表了负面意见,批评了公众和国家的教育体系。[78][79][80]

在他著名的圣诞讲座之前,法拉第曾在1816年至1818年期间为城市哲学学会讲授化学课程,以此来提高他讲座的质量。[81]

在1827年至1860年间,法拉第在伦敦皇家学会为年轻人举办了一系列19场圣诞讲座,这一系列讲座至今仍在继续。讲座的目标是向公众介绍科学,激发他们的兴趣,并为皇家学会筹集资金。这些讲座在伦敦上流社会的社交日程中是重要的活动。通过多封写给密友本杰明·阿博特的信件,法拉第概述了他关于讲课艺术的建议,他写道:“一开始应该点燃一把火,并保持它以不间断的光辉持续到最后。”[82] 他的讲座既充满欢乐又富有童趣,他喜欢用各种气体填充肥皂泡(以判断它们是否具有磁性),但这些讲座也富有哲理。在讲座中,他鼓励观众思考他实验的原理:“你们很清楚冰是漂浮在水面上的……为什么冰会漂浮?想一想这个问题,进行哲学思考。”[83] 他讲座的主题涉及化学和电学,具体包括:1841年:《化学基础》,1843年:《电学的基本原理》,1848年:《蜡烛的化学史》,1851年:《吸引力》,1853年:《伏打电学》,1854年:《燃烧的化学》,1855年:《常见金属的独特性质》,1857年:《静电》,1858年:《金属的特性》,1859年:《物质的各种力及它们之间的关系》。[84]
\subsection{纪念活动}
迈克尔·法拉第的雕像位于伦敦萨沃伊广场,英国工程技术学会外。法拉第纪念碑由野兽派建筑师罗德尼·戈登设计,1961年完工,位于象与城旋转交叉口,靠近法拉第的出生地伦敦纽因顿·巴茨。法拉第学校位于特里尼提浮标码头,他的工作室依旧屹立在链条和浮标仓库上方,紧邻伦敦唯一的灯塔。[85] 法拉第花园是位于伦敦沃尔沃斯的一个小公园,距离他在纽因顿·巴茨的出生地不远。它位于南华克区法拉第选区内。迈克尔·法拉第小学位于沃尔沃斯的艾尔斯伯里庄园。[86]

伦敦南岸大学的一座建筑,容纳了该校电气工程系,被命名为法拉第翼楼,因为它靠近法拉第在纽因顿·巴茨的出生地。1960年,拉夫堡大学的一座礼堂以法拉第命名。在其餐厅入口处有一座铜雕,雕刻了电气变压器的标志,礼堂内还挂有法拉第的肖像,二者均为纪念法拉第。爱丁堡大学科学与工程校园的一座八层楼建筑以法拉第命名,布鲁内尔大学最近建造的一座住宿楼也以法拉第命名,斯旺西大学的主工程楼以及北伊利诺伊大学的教学与实验物理楼也以法拉第命名。英国前法拉第站在南极也以他命名。[87]

以法拉第命名的街道可以在许多英国城市找到(例如:伦敦、法夫、斯温登、贝辛斯托克、诺丁汉、惠特比、柯克比、克劳利、新bury、斯旺西、艾尔斯伯里和史蒂夫尼奇),以及法国(巴黎)、德国(柏林-达赫勒姆、赫姆斯多夫)、加拿大(魁北克市、魁北克省;深河、安大略省;渥太华、安大略省)、美国(纽约布朗克斯区、弗吉尼亚州雷斯顿)、澳大利亚(维多利亚州卡尔顿)和新西兰(霍克湾)找到。[89][90][91]

一块皇家艺术学会蓝色纪念牌,于1876年揭幕,纪念法拉第在伦敦玛丽勒本区的布兰福德街48号。[92] 从1991年到2001年,法拉第的画像出现在英国银行发行的E系列20英镑纸币的背面。他被描绘为在皇家学会讲授演讲,旁边是磁电火花装置。[93] 2002年,在英国全国投票后,法拉第被评为BBC“100位伟大英国人”榜单中的第22位。[94]

法拉第也在皇家邮政发行的邮票上得到纪念。1991年,作为电学的先驱,他与其他三位领域的先驱(查尔斯·巴贝奇(计算机)、弗兰克·惠特尔(喷气发动机)和罗伯特·沃森-瓦特(雷达))一起出现在了科学成就主题邮票中。[95] 1999年,在“法拉第的电学”邮票中,他与查尔斯·达尔文、爱德华·詹纳和艾伦·图灵一起出现在了“世界改变者”系列中。[96]

法拉第科学与宗教研究所以这位科学家的名字命名,他将信仰视为其科学研究的一个重要组成部分。该研究所的标志也基于法拉第的发现。该研究所于2006年由约翰·坦普尔顿基金会捐赠200万美元建立,旨在进行学术研究,促进科学与宗教之间互动的理解,并提高公众对这两个领域的认知。[97][98]

法拉第研究所,成立于2017年,是一个独立的能源存储研究机构,也以法拉第的名字命名。[99] 该组织是英国主要的电池科学与技术研究项目,致力于教育、公众参与和市场研究。[99]

法拉第的生平及其在电磁学方面的贡献是2014年美国科学纪录片系列《宇宙:时空之旅》第十集《电的男孩》中的主要主题,该集在福克斯电视台和国家地理频道播出。[100]

作家奥尔德斯·赫胥黎在一篇题为《皮特拉马拉之夜》的文章中写道:“他始终是自然哲学家。发现真理是他唯一的目标和兴趣……即使我能成为莎士比亚,我想我仍然会选择成为法拉第。”[101] 玛格丽特·撒切尔在一次演讲中称法拉第为她的“英雄”,并在皇家学会表示:“他的工作价值应该高于所有股市上的股票总市值!”她从皇家学会借来法拉第的半身雕像,并将其放置在唐宁街10号的大厅内。[4]
\subsection{以法拉第的名字命名的奖项 } 
为了纪念他对科学的巨大贡献,多个机构设立了以法拉第名字命名的奖项。包括:
\begin{itemize}
\item IET 法拉第奖章[102]  
\item 伦敦皇家学会迈克尔·法拉第奖[103]  
\item 物理学会迈克尔·法拉第奖章和奖项[104]  
\item 皇家化学学会法拉第讲座奖[105]
\end{itemize}
\subsection{画廊}
\begin{figure}[ht]
\centering
\includegraphics[width=6cm]{./figures/4b2d5f88da1e39e2.png}
\caption{1826年左右的年轻迈克尔·法拉第肖像} \label{fig_FLD_2}
\end{figure}
\subsection{参考书目}
法拉第的书籍,除了《化学操作》之外,都是科学论文集或讲座的记录。[106] 自他去世以来,法拉第的日记已被出版,还有几本他的书信集以及法拉第与戴维一起旅行时(1813–1815)的旅行日志。
\begin{itemize}
\item 法拉第,迈克尔(1827)。《化学操作:化学学生的指导手册》。约翰·穆雷出版社。第二版,1830年,第三版,1842年。  
\item 法拉第,迈克尔(1839)。《电学实验研究》,第一卷和第二卷。理查德与约翰·爱德华·泰勒出版社;第三卷,理查德·泰勒与威廉·弗朗西斯,1855年。  
\item 法拉第,迈克尔(1859)。《化学与物理实验研究》。泰勒与弗朗西斯出版社。ISBN 978-0-85066-841-4。  
\item 法拉第,迈克尔(1861)。W. 克鲁克斯(编)。《六讲:蜡烛的化学史》。格里芬、博恩与公司出版社。ISBN 978-1-4255-1974-2。  
\item 法拉第,迈克尔(1873)。W. 克鲁克斯(编)。《自然界中的各种力》。查托与温达斯出版社。  
\item 法拉第,迈克尔(1932–1936)。T. 马丁(编)。《日记》。G. 贝尔出版社。ISBN 978-0-7135-0439-2。——出版了八卷;另见2009年出版的法拉第日记。  
\item 法拉第,迈克尔(1991)。B. 鲍尔斯与L. 西蒙斯(编)。《好奇心得到完全满足:法拉第的欧洲旅行 1813–1815》。电气工程师协会出版社。  
\item 法拉第,迈克尔(1991)。F.A.J.L. 詹姆斯(编)。《法拉第书信集》,第一卷。INSPEC公司。ISBN 978-0-86341-248-6。——第二卷,1993年;第三卷,1996年;第四卷,1999年。  
\item 法拉第,迈克尔(2008)。爱丽丝·詹金斯(编)。《法拉第的思维练习:摄政时期伦敦的手工艺人文艺沙龙》。利物浦大学出版社。  
\item 《自然界各种力的六讲及其相互关系》。伦敦;格拉斯哥:R. 格里芬,1860年。  
\item 《气体的液化》,爱丁堡:W.F. 克莱出版社,1896年。  
\item 《法拉第与肖恩贝因的书信 1836–1862》。附注释、评论和当代书信参考。伦敦:威廉姆斯与诺尔盖特出版社,1899年。(杜塞尔多夫大学和州立图书馆的数字版)
\end{itemize}
\subsection{另见}
\begin{itemize}
\item 法拉第(单位)– 物理常数:一摩尔电子的电荷
\item 法医工程学 – 与法律干预相关的故障调查
\item 尼古拉·特斯拉 – 塞尔维亚裔美国工程师和发明家(1856–1943)
\item 氢技术时间轴
\item 低温技术时间轴
\item 泽曼效应 – 磁场中的光谱线分裂
\end{itemize}
\subsection{参考文献}
\begin{enumerate}
\item Rao, C.N.R. (2000). 《理解化学》。Universities Press. ISBN 81-7371-250-6. 第281页.
\item Chisholm, Hugh, 编. (1911). "Faraday, Michael" 《大英百科全书》。第10卷(第11版)。剑桥大学出版社。第173-175页。1911年版《大英百科全书》.
\item "Archives Michael Faraday biography – The IET". theiet.org.
\item "法拉第笼:从维多利亚时代的实验到斯诺登时代的偏执". 《卫报》. 2017年5月22日.
\item Maxwell, James Clerk (2003). Niven, W. D. (编). 《詹姆斯·克拉克·麦克斯韦的科学论文》第II卷. Dover Publications. ISBN 978-0-486-49561-3.
\item "如何英国科学家启发并确保爱因斯坦在历史上的地位". BBC Science. 2024年5月3日检索.
\item James, Frank A. J. L. (2011) [2004]. "Faraday, Michael (1791–1867)" 《牛津国家传记词典》在线版. 牛津大学出版社. doi:10.1093/ref:odnb/9153. (需要订阅或英国公共图书馆会员资格).
\item 关于法拉第生平的简明叙述,参见《Every Saturday: A Journal of Choice Reading》第三卷,1873年由Osgood & Co.出版,第175-183页.
\item Jerrold, Walter (2018). 《迈克尔·法拉第:科学家》。Books on Demand. ISBN 3734011124. 第11页.
\item 其含义是詹姆斯通过加入该教派发现了其他工作的机会。詹姆斯于1791年2月20日加入伦敦聚会所,并不久后带着家人搬到伦敦。见Cantor,第57-58页.
\item "关于迈克尔·法拉第的答案"。Answers. 2023年2月23日检索.
\item Open Plaques第19号铭牌.
\item Jenkins, Alice (2008). 《迈克尔·法拉第的精神练习:摄政时代伦敦的一圈手工业者》。牛津大学出版社. 第213页. ISBN 978-1846311406.
\item James, Frank (1992). "迈克尔·法拉第,城市哲学学会与艺术学会" 《RSA期刊》。第140卷(5426):192-199. JSTOR 41378130.
\item Lienhard, John H. (1992). "迈克尔·法拉第" 《我们的创造力的引擎》。第741集。NPR。KUHF-FM休斯顿。第741集:迈克尔·法拉第(文字记录).
\item Lienhard, John H. (1992). "简·马塞特的书" 《我们的创造力的引擎》。第744集。NPR。KUHF-FM休斯顿。第744集:简·马塞特的书(文字记录).
\item Thomas,第17页
\item 圣保罗大教堂附近的圣圣母教堂的登记册记录了他们许可证的发放日期为6月12日。证人是莎拉的父亲爱德华。结婚发生在1837年《婚姻和登记法》实施之前16年。见Cantor,第59页.
\item Cantor,第41-43页,60-64页,277-280页.
\item Paul’s Alley位于Barbican南侧10座房屋的位置。见Elmes(1831)《英国大都市的地理辞典》第330页.
\item Baggott, Jim (1991年9月2日). "迈克尔·法拉第的神话:迈克尔·法拉第不仅是英国最伟大的实验家之一。更深入地研究这个人和他的工作揭示出他也是一个聪明的理论家". 《新科学家》. 2008年9月6日检索.
\item West, Krista (2013). 《金属与类金属基础》。Rosen Publishing Group. ISBN 1-4777-2722-1. 第81页.
\item Todd Timmons (2012). "西方科学的创造者:从哥白尼到沃森和克里克的24位先驱的作品和言论". 第127页.
\item "法拉第被任命为首位富勒化学教授". 皇家学会. 2017年10月16日. 2020年8月5日存档. 2017年10月16日检索.
\item "1780–2010年会员手册:F章节" (PDF). 美国艺术与科学学会. 第159页. 2016年5月27日存档. 2016年9月15日检索.
\item "美国物理学会会员历史". search.amphilsoc.org. 2021年4月9日检索.
\item Gladstone, John Hall (1872). 《迈克尔·法拉第》。伦敦:Macmillan and Co. 第53页. 法拉第法国科学院.
\item "M. Faraday (1791–1867)". 荷兰皇家艺术与科学学院. 2015年7月17日检索.
\item Bowden, Mary Ellen (1997). 《化学成就者:化学科学的人性面》。化学遗产基金会. ISBN 0-941901-12-2. 第30页.
\item "特威克纳姆博物馆关于法拉第和法拉第之家"; twickenham-museum.org.uk. 2014年8月14日访问.
\item Croddy, Eric; Wirtz, James J. (2005). 《大规模杀伤性武器:全球政策、技术与历史百科全书》。ABC-CLIO. 第86页. ISBN 978-1-85109-490-5.
\item "法拉第致威廉·史密斯 1859年1月3日". Epilson.ac.uk. 2024年7月12日检索.
\item Open Plaques第2429号铭牌.
\item 'The Abbey Scientists' Hall, A.R. 第59页:伦敦;Roger & Robert Nicholson;1966年.
\item 《杰出物理学家:从伽利略到汤川》。剑桥大学出版社. 2004年. 第118-119页.
\item Hadfield, Robert Abbott (1931). "关于法拉第的‘钢与合金’的研究". 《皇家学会哲学学报:A系列,数学或物理特征论文》. 230 (681–693): 221–292. doi:10.1098/rsta.1932.0007.
\item Akerlof, Carl W. "法拉第旋转" (PDF). 2023年11月29日检索.
\item Jensen, William B. (2005). "本生燃烧器的起源" (PDF). 《化学教育杂志》. 82 (4): 518. Bibcode:2005JChEd..82..518J. doi:10.1021/ed082p518. 2005年5月30日存档.
\item Faraday (1827), 第127页.
\item Faraday, Michael (1821). "关于氯和碳的两种新化合物,以及碘、碳和氢的一个新化合物". 《哲学学报》。111: 47–74. doi:10.1098/rstl.1821.0007. S2CID 186212922.
\item Faraday, Michael (1859). 《化学与物理的实验研究》。伦敦:理查德·泰勒和威廉·弗朗西斯. 第33–53页. ISBN 978-0-85066-841-4.
\item Williams, L. Pearce (1965). 《迈克尔·法拉第:传记》。纽约:基本书籍. 第122–123页. ISBN 978-0-306-80299-7.
\item Faraday, Michael (1823). "关于氯化物的水合物". 《科学季刊》。15: 71.
\item Faraday, Michael (1859). 《化学与物理的实验研究》。伦敦:理查德·泰勒和威廉·弗朗西斯. 第81–84页. ISBN 978-0-85066-841-4.
\item Ehl, Rosemary Gene; Ihde, Aaron (1954). "法拉第的电化学定律与当量重量的测定" (PDF). 《化学教育杂志》。31(5月):226–232. Bibcode:1954JChEd..31..226E. doi:10.1021/ed031p226.
\item "纳米技术的诞生". Nanogallery.info. 2006. 2007年7月25日检索. 法拉第尝试解释他金色混合物中鲜艳色彩的原因,称已知现象似乎表明,金粒子大小的微小变化会导致多种不同的颜色结果。
\item Mee, Nicholas (2012). 《希格斯力:打破对称的力量使世界变得有趣》。第55页.
\item Faraday, Michael (1844). 《电学实验研究》。第二卷。库里尔出版社. ISBN 978-0-486-43505-3. 参见第4页.
\item Hamilton, 第165–171页,第183页,第187–190页.
\item Cantor, 第231–233页.
\item Thompson, 第95页.
\item Thompson, 第91页. 该实验室条目展示了法拉第追求光与电磁现象之间联系的努力,1821年9月10日.
\item Cantor, 第233页.
\item Thompson, 第95–98页.
\item Thompson, 第100页.
\item 法拉第最初的感应实验工作发生在1825年11月下旬。他的工作深受同胞欧洲科学家安培、阿拉戈和厄尔斯特德的研究影响,这一点可以从他的日记中看出。Cantor,第235–244页.
\item Gooding, David; Pinch, Trevor; Schaffer, Simon (1989). 《实验的用途:自然科学研究》。剑桥大学出版社. ISBN 0-521-33768-2. 第212页.
\item Van Valkenburgh (1995). 《基础电学》。Cengage Learning. ISBN 0-7906-1041-8. 第4–91页.
\item 《化学领域伟大先驱的生平与时代(从拉瓦锡到桑格)》。《世界科学》. 2015. 第85, 86页.
\item "迈克尔·法拉第的发电机". 皇家学会. 2017年10月15日.
\item "萨沃剧院", 《泰晤士报》,1881年10月3日. "在昨天下午的《耐心》演出中,进行了一项有趣的实验,这是萨沃剧院首次使用电光照明。自剧院开幕以来,电光一直用于观众席。新的照明方式取得了圆满成功,其对舞台艺术发展的重要性几乎无法过度评价。整个演出过程中,光线保持稳定,效果在视觉上超越了煤气灯,服装的颜色——这是‘美学’歌剧的重要元素——看起来和白天一样真实清晰。使用的是斯旺白炽灯,完全不再使用煤气灯光。"
\item "萨沃剧院是伦敦最好的住宿地点之一"。《今日美国》。2024年7月6日检索。萨沃剧院是世界上第一座完全由电力照明的公共建筑,拥有丰富的发明和丑闻历史。
\item "迈克尔·法拉第在伦敦的旅行"。《皇家学会》。2024年7月6日检索。
\item "亨利·阿德拉德的雕刻细节,基于马尔与波利布兰克约1857年的早期照片"。英国国家肖像画廊:NPR。
\item James, Frank A.J.L (2010). 《迈克尔·法拉第:非常简短的介绍》。牛津大学出版社. ISBN 0-19-161446-7. 第81页。
\item Day, Peter (1999). 《哲学家的树:迈克尔·法拉第文选》。CRC出版社. ISBN 0-7503-0570-3. 第125页。
\item Zeeman, Pieter (1897). "磁化对物质发光性质的影响"。《自然》。55 (1424): 347. Bibcode:1897Natur..55..347Z. doi:10.1038/055347a0.
\item "皮特·齐曼,诺贝尔讲座"。2008年5月29日检索。
\item "迈克尔·法拉第(1791–1867)"。《皇家学会》。2014年2月20日检索。
\item Jones, Roger (2009). 《什么是谁?:以人为名的事物及其来源》。Troubador出版社. 第74页。
\item "19世纪意外爆炸的原因"。《皇家学会》。2020年9月8日检索。
\item Smith, Denis (2001). 《伦敦与泰晤士河谷》。托马斯·特尔福德出版社. ISBN 0-7277-2876-8. 第236页。
\item Faraday, Michael (1855年7月9日). "泰晤士河的状态",《泰晤士报》。第8页。
\item 《迈克尔·法拉第的信件:1849–1855年》,第4卷。IET。1991年。第xxxvii页。
\item "第21950号"。《伦敦公报》。1856年12月16日,第4219页。
\item Thomas, 第83页
\item 皇家学会;Whewell, William;Faraday, Michael;Latham, Robert Gordon;Daubeny, Charles;Tyndall, John;Paget, James;Hodgson, William Ballantyne;Lankester, E. Ray(埃德温·雷)(1917)。 《科学与教育:在皇家学会的讲座》。美国国会图书馆。W. Heinemann出版社。第39–74页【51】。
\item Faraday, Michael (1853年7月2日). "桌子转动"。《插图伦敦新闻》。第530页。
\item Thompson, Silvanus Phillips (1898). 《迈克尔·法拉第:他的生活与工作》。康奈尔大学图书馆。伦敦,卡塞尔出版社。第250–252页。
\item James, Frank A.J.L; Faraday, Michael (1991). 《迈克尔·法拉第的信件》。第4卷。伦敦:电气工程师协会。第xxx–xxii页。ISBN 978-0-86341-251-6。
\item Lan, B.L. (2001). "迈克尔·法拉第:维多利亚时代英国的讲座之王"。《物理教师》。39 (1): 32–36. Bibcode:2001PhTea..39...32L. doi:10.1119/1.1343427。
\item Hirshfeld, Alan (2006). 《迈克尔·法拉第的电气生活》。纽约:沃克与公司;ISBN 0-8027-1470-6
\item Seeger, R.J. (1968). "迈克尔·法拉第与讲座艺术"。《今日物理》。21 (8): 30–38. Bibcode:1968PhT....21h..30S. doi:10.1063/1.3035100。
\item "圣诞讲座的历史"。《皇家学会》。2017年6月9日存档。2024年10月16日检索。
\item Fisher, Stuart (2012). 《英国的河流:河口、潮道、港口、湖泊、湾和峡湾》。A&C Black. ISBN 1-4081-5583-4. 第231页。
\item 迈克尔·法拉第小学。2012年3月29日存档于Wayback Machine。michaelfaradayschool.co.uk
\item "法拉第(站F)的历史"。英国南极考察局。2023年2月23日检索。
\item "1933年10月3日 — 阿尔伯特·爱因斯坦在皇家阿尔伯特大厅发表他的最后一场欧洲演讲"。皇家阿尔伯特大厅。2017年10月15日检索。
\item McNamara, John (1991). 《柏油中的历史》。哈里森,纽约:哈伯山图书。第99页。ISBN 0-941980-15-4。
\item Sir Andrew Clarke (1824–1902)。澳大利亚传记词典。2024年3月28日检索。
\item "法拉第中心"。Faradaycentre.org。2020年9月8日检索。
\item "迈克尔·法拉第(1791–1867)"。英国遗产。2012年10月23日检索。
\item "撤销的纸币参考指南"。英格兰银行。2011年6月10日存档。2008年10月17日检索。
\item "BBC – 英国伟人 – 前100名"。互联网档案馆。2002年12月4日存档。2017年7月19日检索。
\item "‘科学成就’邮票"。应用科学博物馆藏品。2022年9月30日检索。
\item "期刊:世界改变者(1999年9月21日)"。BFDC。2022年9月30日检索。
\item "法拉第科学与宗教研究所:跨学科研究与项目"。templeton.org。2012年1月11日存档。
\item 关于我们。2009年12月13日存档于Wayback Machine,法拉第研究所。
\item "法拉第研究所"。法拉第研究所。2020年12月25日检索。
\item Overbye, Dennis (2014年3月4日). "萨根的继任者重新启动《宇宙》"。《纽约时报》。2014年6月17日检索。
\item Huxley, Aldous (1925). 《皮特拉马拉之夜》。收录于:《沿路:一位游客的笔记与散文》。纽约:乔治·H·多兰出版社。
\item "IET法拉第奖章"。剑桥圣约翰学院。2022年7月20日检索。
\item "迈克尔·法拉第奖与讲座 | 皇家学会"。royalsociety.org。2023年11月30日。
\item "金奖奖章"。金奖奖章 | 物理学会。
\item "RSC法拉第讲座奖"。www.rsc.org。
\item Hamilton, 第220页。
\end{enumerate}
\subsection{来源}
\begin{itemize}
\item Cantor, Geoffrey (1991). 《迈克尔·法拉第,圣门派信徒与科学家》。麦克米伦出版社。ISBN 978-0-333-58802-4。
\item Hamilton, James (2004). 《发现的一生:迈克尔·法拉第,科学革命的巨人》。纽约:兰登书屋。ISBN 978-1-4000-6016-0。
\item Thomas, J.M. (1991). 《迈克尔·法拉第与皇家学会:人的天才与地方的魅力(平装版)》。CRC出版社。ISBN 978-0-7503-0145-9。
\item Thompson, Silvanus (1901). 《迈克尔·法拉第,他的生活与工作》。伦敦:卡塞尔公司。ISBN 978-1-4179-7036-0。
\end{itemize}
\subsection{进一步阅读}
\subsubsection{传记}
\begin{itemize}
\item Agassi, Joseph (1971). 《法拉第作为自然哲学家》。芝加哥:芝加哥大学出版社。ISBN 978-0226010465。
\item Ames, Joseph Sweetman (编.) (约1900). 《感应电流的发现》。第2卷。纽约:美国书籍公司(1890年)。
\item Bence Jones, Henry (1870). 《法拉第的生平与信件》。费城:J.B. Lippincott公司。
\item The British Electrical and Allied Manufacturers Association (1931). 《法拉第》。爱丁堡:R. & R. Clark有限公司。
\item Gladstone, J.H. (1872). 《迈克尔·法拉第》。伦敦:麦克米伦出版社。
\item Gooding, David; James, Frank A.J.L. (1985). 《法拉第重现:关于迈克尔·法拉第(1791–1867)生平与工作的论文集》。英国汉茨郡巴辛斯托克;纽约:麦克米伦出版社;斯托克顿出版社。ISBN 978-0-333-39320-8。
\item Gooding, David; Cantor, Geoffrey; James, Frank A. J. L. (1996). 《迈克尔·法拉第》。纽约阿默斯特:人类图书出版社。ISBN 978-1-57392-556-3。
\item Gooding, David; Tweney, Ryan D. (1991). 《迈克尔·法拉第的《化学笔记、提示、建议与追求目标》1822年》。伦敦:P. Peregrinus,联合工程与技术学会出版。ISBN 978-0-86341-255-4。
\item Hamilton, James (2002). 《法拉第:生命》。伦敦:哈珀·柯林斯出版社。ISBN 978-0-00-716376-2。
\item Hirshfeld, Alan W. (2006). 《迈克尔·法拉第的电气生活》。沃克公司。ISBN 978-0-8027-1470-1。
\item Russell, Colin A. (编,欧文·金格里奇) (2000). 《迈克尔·法拉第:物理学与信仰(牛津科学系列传记)》。纽约:牛津大学出版社。ISBN 978-0-19-511763-9。
\item Thomas, John Meurig (1991). 《迈克尔·法拉第与皇家学会:人的天才与地方的魅力》。布里斯托尔:希尔格出版社。ISBN 978-0-7503-0145-9。
\item Tyndall, John (1868). 《法拉第作为发现者》。伦敦:朗曼公司。
\item Williams, L. Pearce (1965). 《迈克尔·法拉第:一部传记》。纽约:基础书籍出版社。
\end{itemize}
\subsection{外部链接}
\subsubsection{传记}
\begin{itemize}
\item 皇家学会(英国)网站上的法拉第传记
\item 《法拉第作为发现者》 作者:约翰·廷达尔,古腾堡计划(可下载)
\item 《迈克尔·法拉第的基督教品格》
\item 《迈克尔·法拉第的生平与发现》 作者:J. A. 克劳瑟,伦敦:基督教知识推广协会,1920年
\end{itemize}
\subsection{其他资源}  
\begin{itemize}
\item [Michael Faraday的作品](https://www.gutenberg.org/)(Project Gutenberg)  
\item [有关Michael Faraday的作品或文献](https://archive.org/)(Internet Archive)  
\item [Michael Faraday的作品](https://librivox.org/)(LibriVox,公共领域有声书)  
\item [Michael Faraday的完整通信录](https://www.examplelink.com)(基于Frank James的标准版,提供所有信件的可搜索全文)  
\item 与Sir John Cadogan讨论自Faraday以来苯的在线视频播客  
\item [Faraday与Schoenbein的信件 1836–1862](https://www.examplelink.com)(附注释、评论和参考当时的信件,1899年完整版PDF下载)  
\item [Faraday学校](https://www.trinitybuoywharf.com/)(位于Trinity Buoy Wharf,New Model School Company Limited网站)  
\item 在YouTube上的《化学人物:Michael Faraday》节目(Chemical Heritage Foundation)
\end{itemize}
