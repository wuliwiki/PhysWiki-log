% 定向
% 取向

本文翻译自Encyclopedia of Mathematics(数学百科)的\href{https://encyclopediaofmath.org/wiki/Orientation}{Orientation}\footnote{Volume form. Encyclopedia of Mathematics. URL: http://encyclopediaofmath.org/index.php?title=Volume_form&oldid=32331.}词条.

\subsection{一般概念}

在传统数学中,一个\textbf{定向(orientation)}(或译作\textbf{取向})是指一种坐标系的等价划分,如果两个坐标系\textbf{正相关(positively related)}则是等价的.

对于有限实线性空间$\mathbb{R}^n$,一个坐标系由一组基确定,而两组基等价的条件是\textbf{转移矩阵}\upref{TransM}的行列式为正数.这个等价关系划分出两个等价类.对于复数的情况,即$\mathbb{C}^n$,任取其复基$\{e_1, \cdots, e_n\}$,则能导出实基$\{e_1, \cdots, e_n, \I e_1, \cdots, \I e_n\}$,从而可以将其视为$\mathbb{R}^{2n}$.任意两个复基分别导出的实基就是正相关的(也就是说,复结构定义了$\mathbb{R}^{2n}$上的定向).

在一条线、一个面或者更一般的实\textbf{仿射空间}\upref{AfSp}$E^n$上,一个坐标系由一个点(原点)和一组基给定,坐标系的变换由一个平移(改变原点)和一个基变换给定.坐标系的变换是正的,当且仅当基变换的转移矩阵行列式为正数.(举个例子:基向量的偶置换.)两个坐标系定义的定向相同,当且仅当其中一个可以连续地变为另一个,即存在由参数$t\in[0, 1]$给定的一族坐标系$O_t, e_t$关于$t$是连续的,则$O_0, e_0$到$O_1, e_1$的变换就是连续的.在$n-1$维超平面上的\textbf{反射(reflection)}映射能反转定向,即将一个定向中的坐标系映入另一个定向.

坐标系的等价类也能用不同的\textbf{几何体(geometric figures)}\footnote{译注:geometric figures指任何点、线、面等构成的集合,是几何空间的子集.}来定义.如果一个几何体$X$按照某种规则与一个坐标系关联,那么它的镜像在同一个规则下就与取向不同的另一个坐标系关联,于是$X$(以及给定的那个规则)就定义了一个定向.比如说,在仿射平面$E^2$上,一个给定了方向的圆就定义了一个定向,其中正定向里的代表坐标系就是原点在圆心处、中点在圆上的两个向量,而第一个向量到第二个向量沿着给定方向走的角度最小\footnote{译注:原文比这还绕口.总之,给定的方向就是规定逆时针或者顺时针之类的方向.}.在$E^3$中,可以用一根螺丝来作参考系\footnote{译注:这里原文改成参考系(frame)了,译者也很疑惑.},令第一个基向量沿着螺丝旋进的方向,而第二个和第三个基向量之间的旋转则沿着螺丝旋进时旋转的方向.一个基(或称参考系)也可以用著名的\textbf{右手定则}来定义,即用右手大拇指、食指和中指来确定\footnote{译注:即向量叉乘的记忆法则,食指指向前方,中指向掌心弯折,大拇指翘起,则食指方向叉乘中指方向,所得方向就是大拇指所指方向.}.

如果给定了$E^n$的一个定向,那么每一个半空间$E^n_+$就定义了边界面$E^{n-1}$上的一个定向.比如说,如果$E^n$的定向中后$n-1$个基向量都落在$E^{n-1}$中,而第一个基向量指入$E^n_+$,那么后$n-1$个基向量就定义了$E^{n-1}$上的一个定向.在$E^n$中,也可以用一个$n$维\textbf{单形}\footnote{译注:见\textbf{单纯形与复形}\upref{SimCom}.}($E^2$中的三角形,$E^3$中的三角锥)的顶点顺序来定义,即将原点选为第一个顶点,基向量则是从顶点顺次指向其它顶点的向量.同一组顶点的两个顺序属于统一定向,当且仅当它们之间是偶置换关系.一个给定了顶点顺序(至多差一个偶置换)的单形称为\textbf{定向的(oriented)}.一个$n$维定向单形的每一个$(n-1)$-面$\sigma^{n-1}$都有一个诱导定向:如果第一个顶点不在$\sigma^{n-1}$中,那么剩下的顶点顺序就被定义为$\sigma^{n-1}$的正向.

在一个\textbf{连通}\upref{Topo3}的\textbf{流形}\upref{Manif}$M$上,坐标系以一个\textbf{图册(atlas)}的形式出现:一组覆盖了$M$的图.如果各图之间的变换都是正的,那么称这些图构成的图册是定向的.对于一个微分流形来说,这意味着任何两个图之间的Jacobi矩阵处处为正.如果存在一个定向的图册,那么称$M$是可定向的.此时,全体定向图册的集合被分成两个等价类,两个图册等价当且仅当在它们俩中各任取一个图,这两图之间的变换都是正的.如此选择的等价类就被称为该流形的一个定向,选择方式可以是先选择一个图或者一个点$x_0$上的局部定向(包含$x_0$的连通图自动分为两个等价类).对于微分流形,可以通过选择$x_0$处切平面的基来定义局部定向(比如说,圆上的旋转方向可以通过给定一个切向量来给定).如果$M$有边界且已定向,那么边界也是可定向的,比如说按照下列规则定向:在边界的一个点上,选择一组用于给$M$定向的基向量,第一个基向量从边界$\partial M$指向内部,其它的基向量则在边界的切空间中,那么后面这些切向量则定义了边界上的一组定向基.


在任何道路$q:[0, 1\to M]$上,我们可以选择一串图(覆盖该道路),使得两个相邻的图都是正连通的.这样一来,点$q(0)$处的定向就决定了$q(1)$处的定向,而且只需要道路是连续的且起点、终点确定,即可有此定义.如果$q$是个回路,即$q(0)=q(1)=x_0$,那么当按上述方式决定的$q(1)$定向与$q(0)$的相反,则称$q$是一个\textbf{反转定向回路}\footnote{译注:原文为If $q$ is a loop, i.e. $q(0)=q(1)=x_0$, then $q$ is called an orientation-reserving loop if these orientations are opposite. 根据句意和原文下一句判断,这里是原作者笔误,reserving应为reversing.}.于是,我们得到了一个从基本群$\pi_1(M, x_0)$到一个二元群的同态:只要让反转定向回路映射到$-1$即可.通过这一同态,我们能定义一个覆盖,对于不可定向流形来说此覆盖是2-层(two-sheeted)覆盖.我们说它是orienting的(因为覆盖空间会是可定向的)\footnote{译注:It是啥?那个manifold?}.这一同态还能定义$M$上的一个线丛(line bundle),当且仅当$M$可定向时它是平凡的.对于微分流形$M$,这可以定义为$n$次外微分形式的丛$\bigwedge^n(M)$,当且仅当流形可定向时,它有一个非零的截面,且这样的截面同时定义了$M$上的一个体积形式和定向.这个丛有一个特征映射$k:M\to \mathbb{R}P^n$.流形$M$可定向当且仅当其特征$\mu\in H^{n-1}(M; \mathbb{Z})$不为零,此特征是对偶于$\mathbb{R}P^{n-1}\subseteq \mathbb{R}P^{n}$的特征的像.它对偶于一个cycle,即$\mathbb{R}P^{n-1}$在映射$k$下在一般位置所取的(taken in general position)的预像,其支撑集(support)是整个流形.这个cycle就叫orienting的,因为它的补是可定向的:如果用此cycle切开$M$,就可以得到一个可定向流形.$M$本身也是可定向的(不可定向的),当且仅当这么切了以后能得到一个不连通流形(连通补).比如,在$\mathbb{R}P^2$中,投影线$\mathbb{R}P^1$就可以当作一个orienting cycle.

% Along any path q:[0,1]→M, a chain of charts can be chosen such that two neighbouring charts are positively connected. Thus, an orientation at the point q(0) defines an orientation at the point q(1), and this relation depends on the path q only up to its continuous deformation when its ends are fixed. If q is a loop, i.e. q(0)=q(1)=x0, then q is called an orientation-reserving loop if these orientations are opposite. A homomorphism of the fundamental group π1(M,x0) into a group of order 2 arises: The orientation-reversing loops are sent to −1, while the others are sent to +1. Through this homomorphism a covering is created, which is a two-sheeted covering in the case of a non-orientable manifold. It is said to be orienting (since the covering space will be orientable). This same homomorphism defines a line bundle over M which is trivial if and only if M is orientable. For a differentiable M it can be defined as the bundle Λn(M) of differential forms of order n. It has a non-zero section only in the orientable case and then such a section simultaneously defines a volume form on M and an orientation. This bundle has a classifying mapping k:M→RPn. The manifold M is orientable if and only if the class μ∈Hn−1(M;Z) which is the image of the class dual to RPn−1⊂RPn, is not equal to zero. It is dual to a cycle whose support is the manifold which is the pre-image of RPn−1 under the mapping k, taken in general position. This cycle is said to be orienting, since its complement is orientable: If M is cut by means of the cycle, then an orientable manifold is obtained. M is itself orientable (non-orientable) if and only if a disconnected manifold (a connected complement) is obtained after the cut. For example, in RP2, a projective line RP1 serves as orienting cycle.



一个作了单纯剖分的流形$M$(或者一个伪流形)是可定向的,当且仅当可以把所有$n$维单纯形都定向,使得任意两个有公共$n-1$维面的单纯性在此公共面上诱导的定向相反.给定一条$n$维单纯性的闭链,其相邻单纯形的公共面是$n-1$维度的,那么称此链为\textbf{反转定向}的,如果作为起点和终点的两个单纯形在公共面上导出的定向相同,而其它相邻单纯形则导出相反定向.

\addTODO{未完待续}
我们也可以用同调论的语言来定义定向:

An orientation can be defined in the language of homology theory thus: For a connected orientable manifold without boundary, the homology group Hn(M;Z) (with closed supports) is isomorphic to Z, and the choice of one of the two generators defines an orientation. This is also true for a connected manifold with boundary, using Hn(M,∂M;Z). In the first instance, orientability is a homotopy invariant of M, while in the second, of the pair (M,∂M). So, the Möbius strip and the annulus have one and the same homotopy type but a different one if one considers the boundary. A local orientation of the manifold can also be defined by the choice of generators in the group Hn(M,M∖x0;Z), isomorphic to Z. The homological interpretation of orientation enables this concept to be applied to generalized homology manifolds (cf. Homology manifold).






%未完待续.这玩意儿是真的长啊.















