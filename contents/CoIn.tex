% 协变性和不变性
% keys 协变性|不变性|物理定律
% license Usr
% type Tutor

\footnote{A.Zee,Einstein Gravity in a Nutshell.}协变性和不变性是相对论中会遇到的术语,本节给出它们具体的定义。
\subsection{概念的引入}
物理定律经常被表达为一个矢量等于另一个矢量,例如,Newton定律 
\begin{equation}
m\bvec a=\bvec F.~
\end{equation}
若换一参考系(重选基底),它和原参考系(或坐标系)由坐标变换相联系。设 $R$ 是这两参考系下矢量的变换矩阵,即若 $\bvec x$ 是旧参考系下表达的矢量,则新参考系下表达的该矢量 $\bvec x'$ 和旧参考系下的 $\bvec x$ 关系为 $\bvec x'=R\bvec x$。将 $R$ 作用于Newton定律,就有
\begin{equation}
mR\bvec a=R\bvec F.~
\end{equation}
由于加速度是矢量,因此新参考系下的加速度为 $\bvec a'=R\bvec a$。假设 $\bvec F$ 像一个矢量一样变换(虽然这里对力使用了矢量相同的符号,但是力的定义并不清晰,因此在不同坐标变换下不一定是一个矢量),那么新坐标下 $\bvec F'=R\bvec F$。于是在新坐标系下,就有
\begin{equation}
m\bvec a'=\bvec F'.~
\end{equation}
即两坐标系下的牛顿定理具有相同的形式。应该注意的是,质量是标量的例子,即在坐标变换下不变。如果改变的话,那么Newton定律在坐标变换下就不是不变的了,从而导致某个参考系比另一个更可取,这是不可接受的,物理学应当建立在平等的思想上(可视为某种假设)。

Newton定律是协变的,是指方程的两边在坐标变换下按照同一变换方式变换,即若左边的量 $x$ 在新坐标系下为 $x'f(x)$ (映射 $f$ 由新旧坐标系确定),那么右边的量 $y$ 在新坐标系下由 $y'=f(y)$ 确定。然而,由Newton定律表达的物理是不变的,即独立于通过坐标变换联系的参考系。若物理依赖于你如何偏头,那么就相当的麻烦了。物理不应该取决于物理学家,但物理学家具有使用不同方式表达物理的自由。

\subsection{定义}
每一物理学都具有一些最基本的物理定律,对应的物理学是这些基本定律下逻辑推演。物理定律在数学上表达为方程的形式,对应基本物理定律的方程称为\textbf{基本方程}。注意基本定律是那些不依赖于参考系选择的定律,即若基本定律表达的是物理量 $x,y$ 的关系 $y=f(x)$,在任一参考系下测得的物理量 $x,y$ 为 $x_1,y_1$,则恒有 $y_1=f(x_1)$。
\begin{definition}{协变性、不变性}\label{def_CoIn_1}
设 $x=(x_1,\ldots,x_m),y=(y_1,\ldots,y_n)$ 是某一物理学的基本物理量,
\begin{equation}
y=f(x)~
\end{equation}
是该物理学的基本方程。记 $x_1,y_1$ 是一参考系下对应的物理量 $x,y$ , $x_2,y_2$ 是另一参考系下对应的物理量 $x,y$ ,设 $R$ 是从前一参考系到后一参考系之间的变换,使得基本方程左边的物理量满足 $y_2=R(y_1)$。若基本方程右边的物理量在两参考系下仍有关系 $x_2=R(x_1)$,即成立
\begin{equation}
y_2=f(x_2),\quad R(f(x_1))=f(R(x_1)).~
\end{equation}
则称基本方程 $y=f(x)$ 是\textbf{协变}的(covariant),而变换 $R$ 称为(在该物理学下)是\textbf{物理对称}的 (symmetry of physics),并称在变换 $R$ 下物理是\textbf{不变}的(invariant)。
\end{definition}

值得注意的是 $y=f(x)$ 可写为 $g(x,y):=y-f(x)=0$。因此协变性相当于 $g(R(x),R(y))=0$,即在坐标系变换下,$0$ 不会转变为非0。
\begin{example}{}
若某一物理学的基本方程是关于矢量之间的方程,即具有形式 $\bvec u=\bvec v$,记 $\bvec w=\bvec u-\bvec v$,且坐标系变换是对应矢量空间的线性变换 $R$,那么不变性就恰好表达了明显的数学事实:若 $\bvec w=0$,则 $R(\bvec w)=0$。(严格来说,这里的 0 应写为 $\bvec 0$)。
\end{example}














