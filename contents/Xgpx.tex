% 能量色散 X 射线光谱学
% license CCBYSA3
% type Wiki

(本文根据 CC-BY-SA 协议转载自原搜狗科学百科对英文维基百科的翻译)
\begin{figure}[ht]
\centering
\includegraphics[width=6cm]{./figures/3bd3a331a0e45635.png}
\caption{EDXRF用于金属分析和涂层厚度测量。} \label{fig_Xgpx_2}
\end{figure}
\textbf{能量色散X射线光谱学(EDS,EDX,EDXS或XEDS),有时称为能量色散X射线分析(EDXA)或能量色散X射线微量分析(EDXMA}),是一种用于样品元素分析或化学表征的分析技术。它基于某种X射线激发源和样品间的相互作用。其表征能力在很大程度上归因于一个基本原理,即每种元素都具有一种独特的原子结构,使其在电磁发射光谱中产生一组独特的峰[2] (这也是光谱学的主要原理)。

为了激发样品中特有的X射线的发射,一束高能带电粒子束如电子或质子(见PIXE-质子激发X射线发射分析),或一束X射线,聚焦于被研究样品上。静止时,样品中的每个原子包含基态(或未激发)电子,它们处于离散能级或被原子核约束的电子层中。入射束可以激发内壳中的电子,将其从壳中射出,同时在电子所在的位置产生电子空穴。然后,来自外壳的高能电子填充空穴,高能外壳和低能内壳之间的能量差便以X射线的形式释放出来。从样品发出的X射线的数量和能量可通过能量色散光谱仪来测量。由于X射线的能量体现了两个壳层之间的能量差以及发射元素原子结构的特有性,EDS便能测量样品的元素组成[2]。

\subsection{设备}
EDS装置的四个主要组件是
\begin{enumerate}
\item 激发源(电子束或X射线束)
\item X射线探测器
\item 脉冲处理器
\item 分析仪
\end{enumerate}
电子束激发源用于电子显微镜、扫描电子显微镜(SEM)和扫描透射电子显微镜(STEM)。X射线束激发源用于X射线荧光(XRF)光谱仪。探测器用于将X射线能量转换成电压信号;该信号被发送到脉冲处理器,该处理器测量信号并将其传递到分析仪上进行数据显示和分析。过去最常见的探测器是用液氮冷却到低温的硅(锂)探测器。现在,较新的系统通常配备带有珀尔帖冷却系统的硅漂移探测器(SDD)。

\subsection{技术拓展}
\begin{figure}[ht]
\centering
\includegraphics[width=6cm]{./figures/64b0bc2676fbfba3.png}
\caption{EDS原理} \label{fig_Xgpx_1}
\end{figure}
迁移到内壳层填充新生成空穴的电子释放的多余能量不仅仅是发射出X射线。通常,多余能量会转移到更外壳层的第三个电子使之射出,而非产生X射线。这种射出的粒子被称为俄歇电子(Auger electron),其分析方法被称为俄歇电子光谱学(AES)。

X射线光电子能谱学(XPS)是EDS的另一个近亲,它以类似于AES的方式利用发射电子。关于发射电子的数量和动能的信息被用来确定这些已释放的电子的结合能,该能量是元素特异性的,于是可以对样品进行化学表征。

EDS通常与其光谱对应物WDS(波长色散X射线光谱)形成对比。WDS与EDS的不同之处在于,它利用X射线在特殊晶体上的衍射将原始数据分成频谱成分(波长)。WDS的光谱分辨率比EDS高得多。WDS还避免了EDS中的伪像(假峰值、放大器噪声和颤噪)相关的问题。

\subsection{能谱仪的准确性}
EDS可用于确定样品中存在哪些化学元素,并可用于估计它们的相对丰度。EDS也能准确地测量金属涂层的多层涂层厚度及分析各种合金。样品成分定量分析的准确性受到各种因素的影响。许多元素具有重叠的X射线发射峰(例如钛Kβ和钒Kα、锰Kβ和铁Kα)。被测组分的准确性也受到样品性质的影响。X射线由样品中被入射光束充分激发的任何原子产生。这些X射线向各个方向发射(各向同性),因此它们可能不会全部从样本中逸出。X射线从样本中逸出并被用于检测和测量的可能性,取决于X射线的能量以及它到达探测器前穿过的材料的成分、数量和密度。由于这种X射线吸收效应和类似效应,从测得的X射线发射光谱中准确估计样品成分需要采用定量校正程序,有时被称为矩阵校正。[2]

\subsection{新兴技术}
EDS探测器有一种新的趋势称为硅漂移探测器(SDD)。SDD由高电阻率硅片组成,硅片中电子被驱动到一个小的收集阳极。其优点在于该阳极电容极低,从而使处理时间更短并使信号产出更高。SDD的好处包括:
\begin{enumerate}
\item 高计数率和处理效率
\item 在高计数率下比传统硅(锂)探测器分辨率更高
\item 更低的停滞时间(处理X射线事件所花费的时间)
\item 更快的分析能力和更精确的X射线图谱或数秒内收集的粒子数据
\item 能够在相对较高的温度下储存和操作,无需液氮冷却
\end{enumerate}
SDD芯片的电容与检测器的有效面积无关,因此可以使用大得多的SDD芯片(40 mm2或更大),这使得计数采集率更高。大面积芯片的其他优势包括:
\begin{enumerate}
\item 扫描电镜(SEM)束电流的最小化使分析条件下的成像得以优化
\item 减少样品损坏
\item 更小的光束相互作用和高速地图下空间分辨率的优化
\end{enumerate}
当引起研究兴趣的X射线能量超过约30keV时,由于探测器制动功率的降低,传统硅基技术的量子效率很低。由碲化镉(CdTe)和碲化锌镉(CdZnTe)等高密度半导体制成的探测器在更高能量的X射线下具有更高的量子效率,并且能够在室温下工作。单元件系统以及新近出现的像素化成像探测器,如HEXITEC系统,能够在100keV下实现1\%的能量分辨率。

近年来,一种基于超导微量热计的不同类型的EDS探测器也已经商业化。这项新技术将EDS的同步探测能力与WDS的高光谱分辨率结合起来。EDS微量热计由两部分组成:吸收器和超导转变边缘传感器(TES)温度计。前者吸收样品发出的X射线,并将这种能量转化为热量;后者随后测量由于热量流入而引起的温度变化。EDS微量热计历来有许多缺点,包括计数率低和检测器面积小。计数率因依赖量热计电路的时间常数而受阻。探测器面积一定要小,以保持热容量小并最大化热灵敏度(分辨率)。然而,采用数百个超导EDS微量热计阵列可以提高计数率并增大检测器面积,并且这项技术的重要性正与日俱增。

现在有一种新的EDS探测器的趋势,称为硅漂移探测器(SDD)。SDD由高电阻率硅片组成,电子被驱动到一个小的收集阳极。优点在于该阳极的电容极低,从而利用更短的处理时间并允许非常高的产量。SDD的好处包括:\textsl{[来源请求]}
\begin{enumerate}
\item 高计数率和加工,
\item 在高计数率下比传统硅(锂)探测器分辨率更高,
\item 较低的停滞时间(用于处理x光事件的时间),
\item 更快的分析能力和更精确的x光图或几秒钟内收集的粒子数据,
\item 能够在相对较高的温度下储存和操作,无需液氮冷却。
\end{enumerate}
因为SDD芯片的电容独立于检测器的有效面积,所以可以使用更大的SDD芯片(40 毫米2 或者更多)。这允许更高的计数率收集。大面积芯片的其他优势包括:\textsl{[来源请求]}
\begin{enumerate}
\item 最小化扫描电镜束电流,允许在分析条件下优化成像,
\item 减少样品损坏和
\item 更小的光束相互作用和更高的高速地图空间分辨率。
\end{enumerate}
当感兴趣的X射线能量超过约30 keV时,传统硅基技术由于探测器阻止本领的减小而遭受量子效率低下的困扰。由高密度半导体如碲化镉(CdTe)和碲锌镉(CdZnTe)生产的探测器在更高的X射线能量下提高了效率,并且能够在室温下工作。单元件系统,以及最近的像素化成像检测器,如HEXITEC系统,能够在100 keV下实现1\%数量级的能量分辨率。

近年来,一种基于超导微量热计的不同类型的EDS检测器也已经商业化。这项新技术结合了EDS的同时探测能力和WDS的高光谱分辨率。EDS微量热计由两部分组成:吸收器和超导过渡边缘传感器(TES)温度计。前者吸收样品发出的x光并将这些能量转化为热量;后者测量随后由于热量流入引起的温度变化。 EDS微量热计历来有许多缺点,包括计数率低和探测器面积小。计数率因依赖量热计电路的时间常数而受阻。探测器面积必须小,以保持热容量小并最大化热灵敏度(分辨率)。然而,通过数百个超导eds微量热计阵列的实施,计数率和探测器面积得到了提高,并且这一技术的重要性日益增加。

\subsection{参考文献}
[1]
^Corbari, L; et al. (2008). "Iron oxide deposits associated with the ectosymbiotic bacteria in the hydrothermal vent shrimp Rimicaris exoculata" (PDF). Biogeosciences. 5 (5): 1295–1310. doi:10.5194/bg-5-1295-2008..

[2]
^Joseph Goldstein (2003). Scanning Electron Microscopy and X-Ray Microanalysis. Springer. ISBN 978-0-306-47292-3. Retrieved 26 May 2012..