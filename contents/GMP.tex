% GNU Multiple Precision(GMP)库笔记
% license Xiao
% type Note

\begin{issues}
\issueDraft
\end{issues}

\href{https://gmplib.org/}{GMP} 可能是最常用的大整数库。 \enref{Boost}{Boost} 的 Multiprecision 库也提供 C++ 的 wrapper (可以选择后台用 GMP 或者其他库), 封装得更为友好。

\subsection{数据结构}
\begin{itemize}
\item \verb`mpz_t` 大整数(\verb`z` 表示整数), \verb`mpq_t` 大分数, 浮点数 \verb`mpf_t`
\item \textbf{limb} 的意思是大整数数据的一个 word (4 或 8 字节), 类型为 \verb`mp_limb_t`。
\item 原理参考\href{https://gmplib.org/manual/Integer-Internals#Integer-Internals}{这篇文档}。
\end{itemize}

\begin{lstlisting}[language=cpp]
#ifdef _LONG_LONG_LIMB
typedef unsigned long long int	mp_limb_t;
typedef long long int		mp_limb_signed_t;
#else
typedef unsigned long int	mp_limb_t;
typedef long int		mp_limb_signed_t;
#endif

typedef mp_limb_t *mp_ptr;
typedef const mp_limb_t *mp_srcptr;

/* For reference, note that the name __mpz_struct gets into C++ mangled
   function names, which means although the "__" suggests an internal, we
   must leave this name for binary compatibility.  */
typedef struct
{
  int _mp_alloc;		/* Number of *limbs* allocated and pointed
				   to by the _mp_d field.  */
  int _mp_size;			/* abs(_mp_size) is the number of limbs the
				   last field points to.  If _mp_size is
				   negative this is a negative number.  */
  mp_limb_t *_mp_d;		/* Pointer to the limbs.  */
} __mpz_struct;

typedef __mpz_struct MP_INT;    /* gmp 1 source compatibility */
typedef __mpz_struct mpz_t[1];
\end{lstlisting}

\begin{itemize}
\item \verb`_mp_size` limb 的个数(至少是 1), 类型是 \verb`mp_size_t` (\verb`long` 或者 \verb`int`)。 负数的 \verb`_mp_size` 代表 mpz 是负的, 零代表 mpz 为零, 此时 \verb`_mp_d` 无定义。
\item \verb`_mp_d` limb 的指针(little endian), \verb`_mp_d[0]` 是最不重要的 limb, \verb`_mp_d[ABS(_mp_size)-1]` 最重要的 limb。
\verb`_mp_alloc` 和 \verb`vector.capacity()` 类似。 有 \verb`_mp_alloc >= ABS(_mp_size)`, 如果需要更多会重新 allocate。
\end{itemize}
