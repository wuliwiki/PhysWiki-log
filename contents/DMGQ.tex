% 多模光纤
% license CCBYSA3
% type Wiki
(本文根据 CC-BY-SA 协议转载自原搜狗科学百科对英文维基百科的翻译)

多模光纤是一种主要用于短距离通信的光纤。典型的多模式链路在长达$600$米($2000$英尺)的链路长度上具有$10 Mbit/s$至$10 Gbit/s$的数据传输速率。多模光纤具有相当大的纤芯直径,这就使得其能够传播多种光模式,并且由于模式色散而限制了传输链路的最大长度。

\subsection{应用}
用于多模光纤通信的设备比用于单模光纤的设备便宜。[1] 典型的传输速度与传输距离的极限关系为在传输距离高达$2km$情况下速度极限为$100 Mbit/s$,在$1000m$的情况下速度极限为$1 Gbit/s$,而$550m$的情况下速度极限为$10 Gbit/s$。

由于其高容量和可靠性,多模光纤通常在建筑物中作为主要的通信应用工具。越来越多的用户正收获到通过光纤将桌面或区域间联系起来所带来的用户们之间越来越近的通信交互的好处。符合标准的体系结构(如集中布线和光纤到电信机柜)使用户能够通过将电子设备集中在电信机房而不是在每层都安装有源电子设备来利用光纤的距离传输能力。

多模光纤用于向/从微型光纤光谱设备(光谱仪、光源和采样附件)传输光信号,并在第一台便携式光谱仪的开发中发挥了重要作用。

多模光纤也用于高功率光的传输,如激光焊接。
\subsection{与单模光纤的比较}
多模光纤和单模光纤的主要区别在于前者的纤芯直径要大得多,通常为$50-100$微米;且其能够传播更大波长的光。由于纤芯大,及大数值孔径的可能性,多模光纤比单模光纤具有更高的“聚光”能力。实际上,较大的纤芯尺寸简化了连接,还允许使用低成本电子器件,例如工作在$850$纳米和$1300$纳米波长的发光二极管(LEDs)和垂直腔面发射激光器(VCSELs)(电信中使用的单模光纤通常工作在$1310$或$1550$纳米 [2])。然而,与单模光纤相比,多模光纤带宽-距离乘积限制更低。因为多模光纤比单模光纤具有更大的纤芯尺寸,所以它支持多种传播模式;因此,多模光纤受到模式色散的限制,而单模却不受。

与多模光纤一起使用的LED光源有时会产生一系列不同波长的光波,并且每个波长的波以不同的速度传播。这种色散是对多模光纤电缆可用长度的另一种限制。相比之下,用于驱动单模光纤的激光器则产生单一波长的相干光。由于模式色散,多模光纤比单模光纤具有更高的脉冲展宽速率,这就限制了多模光纤的信息传输能力。

单模光纤通常用于高精度的科学研究,因为将光限制在单一传播模式下,可以使其聚焦在一个强烈的衍射受限的点上。

电缆的外皮颜色有时可用于区分多模电缆和单模电缆。在非军事领域中应用的TIA-598C标准建仪电缆外皮的颜色具体取决于所适用的光纤的种类,单模光纤使用黄色外皮,多模光纤使用橙色或浅绿色外皮。 一些供应商使用紫色来区分更高性能的OM4通信光纤和其他类型的光纤。[3]

\subsubsection{3.1 比较 }
\begin{table}[ht]
\centering
\caption{多模光纤上以太网变体的最小覆盖范围}\label{tab_DMGQ1}
\begin{tabular}{|c|c|c|c|c|c|c|c|c|}
\hline
种类 & 最小模式带宽 
850 / 953 /1300nm& 快速以太网100BASE-FX & 1 Gb (1000 Mb)以太网1000BASE-SX & 1 Gb (1000 Mb)以太网1000BASE-LX & 10 Gb以太网 & 40 Gb以太网
40GBASE-SWDM4& 40 Gb以太网40GBASE-SR4 &00千兆以太网100千兆以太网接口-SR10 \\
\hline
FDDI (62.5/125) &160 / – / 500 MHz·km& 2000 m[8] &220 m[9] & 550 m[10] (需要模式调节跳线)[11][12]& 26 m[13] & 不支持& 不支持 & 不支持 \\
\hline
OM1 (62.5/125) & 200 / – / 500 MHz·km &2000 m[8]& 275 m[9] & 550 m[10] (需要模式调节跳线)[11][12] & 33 m[8]& 不支持 & 不支持 & 不支持 \\
\hline
OM2 (50/125)& 500 / – / 500 MHz·km &2000 m[8]& 2000 m[8] & 550 m[10] (需要模式调节跳线)[11][12]& 82 m[14]&不支持 &不支持& 不支持\\
\hline
OM3(50/125)*激光优化* & 1500 / – / 500 MHz·km & 2000 m[8] & 550 m[14] &550 m (不应使用模式调节跳线)[14]& 300 m[8]& 240m[15]
双工液晶显示器& 100 m[14]
(330米QSFP+ eSR4[16]) & 100 m[14]\\
\hline
OM4(50/125)*激光优化* &3500 / – / 500 MHz·km& 2000 m[8] & 550 m[14]&550 m (不应使用模式调节跳线)[14] & 400 m[17]&350m[15]
双工液晶显示器& 150 m[14]
(550米QSFP+ eSR4[16])&150 m[14]\\
\hline
OM5 (50/125)“宽带多模式”,适用于短波WDM[18] & 3500 / 1850 / 500  MHz·km & 2000 m[8] & 550 m[14]&550 m (不应使用模式调节跳线)[14] & 400 m[17]&350m[15]
双工液晶显示器& 150 m[14]
(550米QSFP+ eSR4[16])&150 m[14]\\
\hline
\end{tabular}
\end{table}
1.OFL Over-Filled Launch for 850/953 nm / EMB Effective Modal Bandwidth for 1310 nm
\subsection{环绕通量}
国际电工委员会61280-4-1(现为TIA-526-14-B)标准定义了环形通量,该标准规定了测试光注入尺寸(适用于各种光纤直径),以确保纤芯不会被过度填充或填充不足,从而实现可更多次重复(且变量更小)的链路损耗测量。[19]

\subsection{参考文献}
[1]Telecommunications Industry Association. "Multimode Fiber for Enterprise Networks". Retrieved Jun 4, 2008..

[2]ARC Electronics (Oct 1, 2007). "Fiber Optic Cable Tutorial". Retrieved March 4, 2015..

[3]Crawford, Dwayne (Sep 11, 2013). "Who is Erika Violet and what is she doing in my data center?". Tech Topics. Belden. Retrieved Feb 12, 2014..

[4]British FibreOptic Industry Association. "Optical Fibers Explained" (PDF). Retrieved Apr 9, 2011..

[5]"Fiber Optics Overview". Retrieved 2012-11-23..

[6]"Meeting Report #14" (PDF). Telecommunications Industry Association..

[7]Kish, Paul (2010-01-01). "Next generation fiber arrives". # Cabling Networking Systems. Business Information Group..

[8]"Fiber optic cable color codes". Tech Topics. The Fiber Optic Association. Retrieved Sep 17, 2009..

[9]IEEE 802.3-2012 Clause 38.3.

[10]IEEE 802.3 38.4 PMD to MDI optical specifications for 1000BASE-LX.

[11]Cisco Systems, Inc (2009). "Cisco Mode-Conditioning Patch Cord Installation Note". Retrieved Feb 20, 2015..

[12]As with all multi-mode fiber connections, the MMF segment of the patch cord should match the type of fiber in the cable plant (Clause 38.11.4)..

[13]"Cisco 10GBASE X2 Modules Data Sheet". Cisco. Retrieved June 23, 2015..

[14]Furukawa Electric North America. "OM4 - The next generation of multimode fiber" (pdf). Retrieved May 16, 2012..

[15]"40GE SWDM4 QSFP+ Optical Transceiver | Finisar Corporation". www.finisar.com (in 英语). Retrieved 2018-02-06..

[16]"40G Extended Reach with Corning Cable Systems OM3/OM4 Connectivity with the Avago 40G QSFP+ eSR4 Transceiver" (pdf). Corning. 2013. Retrieved 14 August 2013..

[17]"IEEE 802.3". Retrieved 31 October 2014..

[18]"TIA Updates Data Center Cabling Standard to Keep Pace with Rapid Technology Advancements". TIA. 2017-08-09. Retrieved 2018-08-27..

[19]Goldstein, Seymour. "Encircled flux improves test equipment loss measurements". Cabling Installation & Maintenance. Retrieved 1 June 2017..