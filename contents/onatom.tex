% 一维单原子链晶格
% keys 声子|一维晶格|振动
% license Xiao
% type Tutor

\pentry{小振动\nref{nod_Oscil}}{nod_3dc0}

为了研究复杂的三维晶格的性质,我们可以先从较为简单的结构入手,例如研究一维单原子链的情形,并且假定\textbf{仅相邻原子之间存在相互作用}。设一排 N 个相同的原子组成的简单晶体,原子质量为 $m$,相邻两个原子的相互作用能是它们之间距离的函数:$V(r)$。当晶体\textbf{处于平衡位置}时,相邻两原子间距离为 $a$,那么原子略微偏离平衡位置可以产生\enref{小振动}{Oscil},根据理论力学的相关知识,一维单原子链作为 $N$ 个自由度的力学体系,共有 $N$ 个独立简谐振动模式,对这些简谐振动模式的研究可以帮助我们认识一维晶格振动的情形。

为了衡量相邻原子间弹性恢复力的程度,我们把相互作用能 $V(r)$ 在 $a$ 附近进行傅里叶展开:
\begin{equation}
V(a+\delta)=V(a)+\frac{1}{2}\beta \delta^2+\text{高阶项}~,
\end{equation}
$\beta$ 被称为\textbf{力常数}。让我们\textbf{暂时地忽略高阶项},那么相邻两原子间的作用力大小为
\begin{equation}
F=-\frac{\dd V}{\dd \delta}=-\beta\delta~.
\end{equation}

\subsection{格波解与色散关系}
我们先忽略原子链的边界情况,即假设它是无限长的。设第 $n$ 个原子偏离平衡位置的位移为 $\mu_n$,我们可以根据\enref{牛顿运动定律}{New3}列出方程
\begin{equation}\label{eq_onatom_1}
m \ddot \mu_n = \beta(\mu_{n+1}-\mu_n)-\beta(\mu_n - \mu_{n-1})~.
\end{equation}
为了找到相互独立且正交的简谐振动模式,我们假定方程具有“格波”形式的特殊解
\begin{equation}\label{eq_onatom_2}
\mu_{n}(q)=Ae^{i(\omega t-naq)}~,
\end{equation}
$q$ 为波数。代入\autoref{eq_onatom_1} 可以求得
\begin{equation}\label{eq_onatom_3}
\omega^2=\frac{4\beta}{m}\sin^2 \qty(\frac{1}{2}aq)~.
\end{equation}
这表明对于任意波数 $q$ 可以解出简振模的频率 $\omega$,而 $\omega$ 与 $q$ 的函数关系被称为\textbf{色散关系}。

相邻两个原子间距离为 $a$,这表明 $q$ 和 $q+2\pi/a$ 对应同一种格波。为了方便起见,我们令 $-\pi/a<q<\pi/a$。这个取值范围是这个一维简单晶格的\textbf{布里渊区}。

前面我们讨论的是无穷长的晶格,现在让我们把目光转向 $N$ 个原子的情形。采用\textbf{周期性边界条件}(玻恩-卡曼条件),即设这个原子链首尾相连形成一个环,这时振动模式的色散关系应当与实际情形差不多。由于首尾相连,\autoref{eq_onatom_1} 仍然成立,所以仍可以设格波解。所以 \autoref{eq_onatom_2} 应满足周期性条件,$\mu_n(0)=\mu_n(N)$。因此 $q$ 的取值不再是连续的,而是离散的:
\begin{equation}
q=0,\pm \frac{2\pi}{Na},\pm 2\cdot \frac{2\pi}{Na},\cdots, (-\pi/a<q<=\pi/a)~.
\end{equation}
$q$ 一共有 $N$ 种取值,色散关系仍为 \autoref{eq_onatom_3} ,所以共有 $N$ 种简谐振动模式。当原子数量 $N$ 很大时,一维原子链近似于连续介质,$q$ 的取值就是准连续的。
\subsection{声子与元激发}
\pentry{量子简谐振子(升降算符法)\nref{nod_QSHOop}}{nod_6fd3}

根据量子力学基本原理,简正振动的能级并非连续的,而是量子化的。波数为 $q$ 频率为 $\omega$ 的振动模的能量本征值只能为
\begin{equation}
\epsilon_n=\qty(n+\frac{1}{2})\hbar\omega~.
\end{equation}
能量激发的单元是 $\hbar\omega$,因此能级 $\epsilon_n$ 可以看作是 $n$ 个频率为 $\omega$ 的带有能量 $\hbar\omega$ 的\textbf{声子}。声子不是真实的粒子,而是多体系统集体运动的激发单元(称为“元激发”),是“准粒子”。用准粒子的概念可以更好地分析和理解晶体的各种物理性质。

在前面的推导种我们对势能展开时忽略了高阶项,同时忽略了晶格边界的效应,因此不同的简正模相互独立,不同声子互不干扰。然而如果考虑高阶项的影响,拉格朗日方程将多出“\textbf{非谐项}”,这使得不同声子之间可以有相互作用,即“碰撞”。声子的碰撞正是晶格的\textbf{热传导}的原因。此外晶格中如果掺杂了杂志,则可能会产生\textbf{局域模}或\textbf{共振模},声子在传播过程中也可能与之“碰撞”,因此杂质对热传导系数也有影响。

我们可以对“\textbf{声子气体}”作研究来探索晶格的热力学性质。由统计力学的\enref{玻色爱因斯坦分布}{MBsta}可以知道声子的频率分布函数,进而求得晶格振动所导致的内能与热容。\enref{晶格热容的爱因斯坦理论}{EScap}与\enref{晶格热容的德拜理论}{Debye}正是运用了这一思想。
