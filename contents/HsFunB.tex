% 函数回顾(高中)
% keys 初中|函数|正比例|反比例|二次
% license Xiao
% type Tutor

\begin{issues}
\issueDraft
\end{issues}

在初中阶段,函数的概念初步展示了变量之间的关系。初中接触的函数主要包括正比例函数、反比例函数和一次函数、二次函数。接下来会在介绍实数和坐标系的概念后,逐一回顾每种函数的特性和相关概念。

\subsection{实数与坐标系}

函数的基础是\textbf{数字(number)},而函数的图像则依赖于坐标系的表现。在介绍函数前,下面会先从“运算”的视角来重新审视所学过的数字,并介绍坐标系的相关概念。

最简单也最广为人知的数字,莫过于\textbf{自然数(natural number)}。自然数是每个人开对数字最开始的概念,自然数指的是从$0$开始\footnote{也有领域认为数字从$1$开始。}一个接一个地排列的数。这个排列的过程,也称作\textbf{递增(increment)}或\textbf{后继(successor)}。在后继的基础上,人们抽象出了加法运算,又从加法抽象出了乘法运算。而自然数对于加法和乘法都是\textbf{封闭的(closed)},也就是说任意两个自然数的和或积都是自然数。

随着对数字的进一步需求,会遇到表示“少于零”的情况,比如温度计上的零度以下温度,银行账户的负债等。这就引入了负数的概念。此时,减法也可视作与后者相反数的和。\textbf{整数(integer)}包括所有自然数和负数,也使得它不仅对加法和乘法封闭,对减法也封闭。这是一次数的扩充。

整数虽然很有用,但当需要表示更精确的数量,比如半个苹果或三分之一米时,仅用整数就不够了。这时,引入了分数\footnote{小数是一种分数的表示方法,尽管在小学就已经学习了,但小数本身很复杂,此处不讨论。}。此时,除法也可视作与后者倒数的积。当然,定义也要求$0$没有倒数\footnote{$0$有倒数也不是不可以,但是引入之后造成的麻烦会很大,因此,数学领域放弃了这个设定。},后面的讨论涉及到除法时,也均不包含$0$作除数的情况。\textbf{有理数(rational number)}\footnote{“有理数”这个词的翻译本身是有问题的,译者当初可能混淆了“合理”和“可比”的词义。更贴切的译法应是“可比数”,也符合表示成两数之比的含义。但由于此翻译已经广为使用,难以纠正。}包括所有整数和分数。而这时,有理数也可以统一表示成两个整数$m,n$之比$\displaystyle\frac{n}{m}\quad(m \neq 0)$。此时,运算又一次扩张,有理数不仅对加法、乘法和减法封闭,对除法也封闭。

随着数学中所要描述的事物逐渐增多,人们逐步发现一些最初被认为可以用有理数表示的数实际上无法如此表示。例如,正方形的对角线长度和圆的周长与直径的比值。最初,人们期望这些数可以用两个整数之比来表达。但研究发现,写成整数之比只不过是一种近似,和真实的数值并不相等。在公元前400年左右就已经发现这件事了,但直到17世纪人们才广泛接受这件事。而具体把这一部分数字定义清楚已经是19世纪末的事情了\footnote{当然,既然这件事是这么晚才搞清楚,在高中阶段一定不会涉及它的明确定义了。}。最终,数学家们用“极限”运算把\textbf{实数(real number)}确定了下来。由于高中不涉及极限,因此只要了解实数包含有理数以及不能用整数比来表示的无理数,无理数对所有运算都不封闭,而实数对于加、减、乘、除都封闭就可以了。另外提及一下,实数对极限运算\footnote{注意,极限为无穷事实上是极限不存在的一种情况。}也是封闭的,这称为实数的\textbf{完备性(completeness)}。

实数除了对上面的运算封闭,还具有\textbf{顺序性(orderliness)}和\textbf{稠密性(density)}。顺序性是指任意两个实数可以比较大小,稠密性是指任意两个实数中间都会有另一个实数。后面会学习的复数就不具备顺序性,而前面介绍的有理数、整数等则不具备稠密性。稠密性保证了实数可以一一对应到直线上的所有点,这时的直线称作\textbf{数轴(number line)}。数轴由原点、单位长度以及正方向来唯一确定。

\textbf{坐标系(coordinate system)}是一种帮助将几何图形与数字相互关联的数学工具。法国数学家\textbf{勒内·笛卡尔(René Descartes)}在数学研究中,将两条数轴的原点重叠,并将其正交(即相互垂直)放置,创造了坐标系。这就是初中阶段学习过的\textbf{笛卡尔坐标系(Cartesian coordinate system)},也称为\textbf{直角坐标系(rectangular coordinate system)}。

引入坐标系后,平面上的任何一点都可以通过一个\enref{有序数对}{CartPr} $(x, y)$ 来表示。借助这种表示法,几何形状可以通过数对来分析和研究,这一方式称为\enref{解析几何}{JXJH}。而当数对中的值对应于函数的变量及其结果时,几何图形就成为了函数的图像。因此,坐标系不仅为函数的图像提供了清晰的视觉表达,还使得人们可以通过几何图形直观地观察函数的性质,例如其变化趋势、最大值和最小值等。

通常,直角坐标系中,两条数轴称为$x$轴和$y$轴,且向右的方向为$x$轴的正方向,向上为$y$轴的正方向。数轴将平面分为四个区域,称为\textbf{象限(quadrant)}。其中,第一象限是两个坐标都为正的区域,之后按逆时针方向依次为第二、第三和第四象限。

\subsection{正比例函数与一次函数}


\textbf{正比例函数(proportional function)}的表达式为 
\begin{equation}
y = kx,\quad(k\neq0)~.
\end{equation}

其中 k 是一个常数,称它为“斜率”。它表示的是每单位的 x 增加,会带来 y 的变化。当 k 为正时,y 随着 x 增加而增加;当 k 为负时,y 随着 x 增加而减少。斜率 k 决定了这条直线的倾斜程度。k 越大,直线越陡峭;k 越小,直线越平缓。正比例函数的图像是一条经过原点的直线。这是因为当 x 为 0 时,y 也必然为 0,所以图像一定会穿过原点。


图像特点:无论 正比例函数的图像总是一条直线,与k 的取值无关。

\textbf{一次函数(linear function)}
一次函数是最常见的线性关系。正比例函数是一种最简单的线性关系。

定义与一般形式:一次函数的表达式是  y = mx + b ,其中 m 是斜率,决定了直线的倾斜度;b 是截距,表示直线与 y 轴的交点。这条直线不一定经过原点,b 决定了它在 y 轴上截取的位置。

图像特点:一次函数的图像是一条直线。m 控制了直线的倾斜程度,b 则决定了直线在 y 轴上的起点。因此,一次函数可以表示更灵活的线性关系。

\subsection{反比例函数}

、\textbf{反比例函数(inversely proportional function)}
对应的是两条双曲线。你还学过如何根据反比例函数的表达式,通过已知的点来求解函数的值。

中心对称性

矩形面积相同

反比例函数描述了一种“此增彼减”的关系。

定义与表达式:反比例函数的标准形式为  y = \frac{k}{x} ,其中 k 是常数。当 x 增加时,y 减少;当 x 减少时,y 增加。这意味着 x 和 y 之间存在一种反向变化的关系。

图像特点:反比例函数的图像是一条双曲线,分别位于坐标轴的两侧,并且关于原点对称。这种对称性表明,x 值变号时,y 值也会相应地变号。

矩形面积不变性:反比例函数的一个特殊性质是,对于任意点 (x, y),x 和 y 的乘积总是等于常数 k。这类似于矩形的面积总是等于长乘以宽,即使长宽发生变化,只要乘积不变,面积就保持不变。


\subsection{二次函数}

二次函数(quadratic function)的图像可能与  x  轴有两个交点,并且具有一个对称轴和一个最低点。

轴对称性:图像上对称点到对称轴的距离相等,且连线与对称轴垂直。

关于二次函数和一元二次方程的关系参见\enref{因式分解与一元二次方程}{quasol}。

二次函数描述了一种曲线关系,常用于表示自然界中的抛物运动。

定义与标准形式:二次函数的标准形式为  y = ax^2 + bx + c ,其中 a, b, c 是常数。a 的正负决定了抛物线的开口方向:当 a 为正,开口向上;当 a 为负,开口向下。

图像特点:二次函数的图像是一条抛物线,具有明显的对称性。抛物线的对称轴可以帮助找到对称的两个点以及最高点或最低点。其顶点是抛物线的最高点(当 a < 0)或最低点(当 a > 0)。

二次函数与一元二次方程的关系:二次函数图像与 x 轴的交点是方程的解,即求得的 x 值可以使得 y 为零。这在很多实际问题中帮助找到最优值或临界点。
