% 数字与函数回顾(高中)
% keys 初中|函数|正比例|反比例|二次|实数|坐标系
% license Xiao
% type Tutor

在初中阶段,函数的概念初步展示了变量之间的关系。初中接触的函数主要包括正比例函数、反比例函数和一次函数、二次函数。接下来会在介绍实数和坐标系的概念后,逐一回顾每种函数的特性和相关概念。

\subsection{实数与坐标系}

函数的基础是\textbf{数字(number)},而函数的图像则依赖于坐标系的表现。在介绍函数前,下面会先从“运算”的视角来重新审视所学过的数字,并介绍坐标系的相关概念。

最简单也最广为人知的数字,莫过于\textbf{自然数(natural number)}。自然数是每个人开对数字最开始的概念,自然数指的是从$0$开始\footnote{也有领域认为数字从$1$开始。}一个接一个地排列的数。这个排列的过程,也称作\textbf{递增(increment)}或\textbf{后继(successor)}。在后继的基础上,人们抽象出了加法运算,又从加法抽象出了乘法运算。而自然数对于加法和乘法都是\textbf{封闭的(closed)},也就是说任意两个自然数的和或积都是自然数。

随着对数字的进一步需求,会遇到表示“少于零”的情况,比如温度计上的零度以下温度,银行账户的负债等。这就引入了负数的概念。此时,减法也可视作与后者相反数的和。\textbf{整数(integer)}包括所有自然数和负数,也使得它不仅对加法和乘法封闭,对减法也封闭。这是一次数的扩充。

整数虽然很有用,但当需要表示更精确的数量,比如半个苹果或三分之一米时,仅用整数就不够了。这时,引入了分数\footnote{小数是一种分数的表示方法,尽管在小学就已经学习了,但小数本身很复杂,此处不讨论。}。此时,除法也可视作与后者倒数的积。当然,定义也要求$0$没有倒数\footnote{$0$有倒数也不是不可以,但是引入之后造成的麻烦会很大,因此,数学领域放弃了这个设定。},后面的讨论涉及到除法时,也均不包含$0$作除数的情况。\textbf{有理数(rational number)}\footnote{“有理数”这个词的翻译本身是有问题的,译者当初可能混淆了“合理”和“可比”的词义。更贴切的译法应是“可比数”,也符合表示成两数之比的含义。但由于此翻译已经广为使用,难以纠正。}包括所有整数和分数。而这时,有理数也可以统一表示成两个整数$m,n$之比$\displaystyle\frac{n}{m}\quad(m \neq 0)$。此时,数字又一次扩张,有理数不仅对加法、乘法和减法封闭,对除法也封闭。

有理数具备两个重要特性:顺序性和稠密性。\textbf{顺序性(orderliness)}指的是任意两个有理数都可以进行大小比较,因此实数形成了一个有序的数集。而\textbf{稠密性(density)}则意味着在任意两个有理数之间,总能找到另一个有理数,甚至是无穷多个有理数,比如$\displaystyle {1\over n},{2\over n}$之间,永远有一个$\displaystyle {3\over2n}$,以及$\displaystyle {1\over n},{3\over 2n}$之间的$\displaystyle {5\over4n}$,$\displaystyle {3\over 2n},{2\over n}$之间的$\displaystyle {7\over4n}$等等,可以一直写下去,他们都在$\displaystyle {1\over n},{2\over n}$之间。与此形成对比的是,后面会学习到的\enref{复数}{CplxNo}并不具备顺序性,而前面介绍的整数则不具备稠密性。

人们一度认为有理数就包含了全部的数,但随着数学的发展,要描述的事物逐渐增多,人们逐渐意识到,有些数其实无法用有理数来表示。例如,正方形的对角线长度和圆的周长与直径的比值。最初,数学家们用有理数的形式来表达这些数,即以为这些数都能写成两个整数的比值。然而,深入研究后发现,这种方法得到的其实只是近似值,而无法精确表示这些数的真实值。这一发现可以追溯到公元前400年,但直到17世纪,人们才普遍接受了这一事实。至于对这类数的明确定义,直到19世纪末才得以完成\footnote{当然,既然这件事是这么晚才搞清楚,在高中阶段一定不会涉及它的精确定义。}。最终,数学家们明确了“极限(limit)”运算的含义,进而定义了\textbf{实数(real number)}\footnote{定义实数的方法非常多,而且他们彼此之间等价。}。由于高中阶段不涉及极限的具体内容,仅需知道实数包含两类数:一类是有理数,另一类是无法用整数比表示的无理数。无理数对之前提到的加、减、乘、除运算都不封闭,但实数在这些运算下是封闭的。另外,实数还对极限运算封闭,一特性称为实数的\textbf{完备性(completeness)}\footnote{有些人会认为有些极限是无穷,而无穷不是实这数。但其实,极限为无穷只是一种极限不存在的情况。}。

除了上面的性质,实数还具备一个有理数不具备的重要的性质:\textbf{连续性(continuity)},这使得实数能够与直线上的所有点建立一一对应关系,构成\textbf{数轴(number line)}。数轴的定义基于一个确定的原点、单位长度和正方向,这三个因素唯一地确定了数轴在几何中的位置和方向。法国数学家勒内·笛卡尔(René Descartes)在数学研究中,将两条数轴的原点重叠,并将其正交(即相互垂直)放置,创造了\textbf{坐标系(coordinate system)}。这就是初中阶段学习过的\textbf{笛卡尔坐标系(Cartesian coordinate system)},也称为\textbf{直角坐标系(rectangular coordinate system)}。

引入坐标系后,平面上的任何一点都可以通过一个\enref{有序数对}{CartPr} $(x, y)$ 来表示。借助这种表示法,几何形状可以通过数对来分析和研究,这一方式称为\enref{解析几何}{JXJH}。而当数对中的值对应于函数的变量及其结果时,几何图形就成为了函数的图像。因此,坐标系不仅为函数的图像提供了清晰的视觉表达,还使得人们可以通过几何图形直观地观察函数的性质,例如其变化趋势、最大值和最小值等。

通常,直角坐标系中,两条数轴称为$x$轴和$y$轴,且向右的方向为$x$轴的正方向,向上为$y$轴的正方向。数轴将平面分为四个区域,称为\textbf{象限(quadrant)}。其中,第一象限是两个坐标都为正的区域,之后按逆时针方向依次为第二、第三和第四象限。

\subsection{正比例函数与一次函数}

\begin{definition}{正比例函数}
形如
\begin{equation}
y = kx\quad(k\neq0)~.
\end{equation}
的函数称为\textbf{正比例函数(proportional function)}。
\end{definition}

这里只要求$k\neq0$,称为正比例是因为函数与自变量的关系形式,而非参数的正负。正比例函数的图像是一条经过原点的直线,且正比例函数的图像总是一条直线,与$k$的取值无关\footnote{有些函数的形状是与参数的取值有关的。}。由于 $x=0$ 时,$y$ 也必然为 $0$,所以图像一定会穿过原点。$k=0$时,正比例函数退化为$y=0$,即$x$轴。

参数$k$控制了直线的倾斜程度,或者说它量化了正比例函数的倾斜程度。这种对倾斜程度量化的值被抽象出来,就是斜率。

\begin{definition}{斜率}
对平面上相异两点,以两点坐标差之比,或两点连线与坐标轴夹角的正切,来表示两点所在直线关于某坐标轴(通常是$x$轴)倾斜程度的量,称为\textbf{斜率(Slope)},通常记作$k$或$m$,即:
\begin{equation}
k = \frac{\Delta y}{\Delta x}={y_2-y_1\over x_2-x_1}~.
\end{equation}
或
\begin{equation}
k =\tan\theta~.
\end{equation}
其中,$(x_1,y_1),(x_2,y_2)$为相异两点的坐标,$\theta$为直线与坐标轴夹角,称作\textbf{倾角(angle of inclination)}。
\end{definition}

从定义可以看出,斜率表示的是每单位的$x$增加带来$y$的变化,进而:
\begin{itemize}
\item $k$的符号决定了直线的倾斜方向:$k$为正时,表示直线向上倾斜,即$y$随着$x$的增加而增加;反之,$k$为负时,表示直线向下倾斜,$y$随着$x$的增加而减少。
\item $|k|$决定了直线的倾斜程度:$|k|$越大,直线越陡峭,$x$的变化带来的$y$的变化越大;反之,$|k|$越小,直线越平缓,$x$的变化带来的$y$的变化越小。
\item 特别地,如果一条直线和$x$轴垂直时$\theta=90^{\circ}$,此时$\tan\theta$为无穷或者说不存在。
\end{itemize}

在研究函数时,关注的主要就是其变化规律,因此“斜率”的概念非常重要,斜率也与\enref{导数}{HsDerv}的概念紧密相连。在多维空间中斜率的概念会推广为\textbf{\enref{梯度}{Grad}(gradient)},用于描述更复杂的函数的变化速率。

正比例函数相当于描述了三点要求:
\begin{enumerate}
\item 函数的图像是直线;
\item 函数要过原点;
\item 函数的斜率为$k$。
\end{enumerate}
如果放松第二个要求,或者说打开第二个约束,则得到了一次函数。

\begin{definition}{一次函数}
形如
\begin{equation}
y = kx+b\quad(k\neq0)~.
\end{equation}
的函数称为\textbf{一次函数(linear function,或线性函数)}。
\end{definition}

一次函数的图像是一条直线,是最常见的线性关系之一。参数$b$决定了直线与$y$轴的交点,可以理解为当变量$x$对$y$没有影响时,$y$所处的“默认状态”。这个默认状态即为截距。正比例函数相较于一次函数的特殊性也体现在截距$b$的取值上。

\begin{definition}{截距}
函数与坐标轴相交的点所对应的坐标值称为\textbf{截距(intercept)}。由于函数要求每个$x$值只能对应一个$y$值,一般\textbf{y轴截距(Y-intercept)}是唯一的。通常,当讨论截距时,指的就是$y$轴截距。相比之下,$x$轴的截距一般并不唯一,称为零点;而对于\aref{单调函数}{def_HsFunC_1}(尤其是直线),在$x$轴的截距若存在则唯一,称为\textbf{x轴截距(X-intercept)}。
\end{definition}

换句话说,截距表示直线或曲线在坐标轴上“停留”的位置。需要注意的是,截距不是“距离”,其数值可以是任意实数。如果直线某坐标轴平行,则认为这个坐标轴上的截距未定义或不存在。截距的作用包括以下几个方面:
\begin{itemize}
\item 快速了解直线或曲线在坐标系中的位置,在某些实际问题中,截距有着明确的物理意义。
\item 在某些直线或曲线表达式中,如:斜截式、截距式等,截距可以直接用于求解或确定表达式。
\item 对于许多实际问题,当$x$只能为非负值时,$x=0$对应的函数值(即截距)通常代表了函数的初始状态或初始条件。
\item 在回归分析中,即便$x=0$时没有实际意义,截距项仍然是模型中的重要组成部分。如果没有截距项,模型会强制回归线通过原点,可能导致斜率的估计值出现偏差。
\end{itemize}

\subsection{反比例函数与二次函数}

反比例函数与二次函数的形状都是\enref{圆锥曲线}{conic},原因是它们都涉及到二次关系:在反比例函数中是$xy$,二次函数中是$x^2$。

\begin{definition}{反比例函数}
形如
\begin{equation}
y={p\over x}\quad(p\neq0)~.
\end{equation}
的函数称为\textbf{反比例函数(inversely proportional function)}。
\end{definition}

反比例函数的图像是一条双曲线,双曲线的两支分别位于关于原点对称的象限:当$p>0$,函数图像在第一、三象限;当$p<0$,函数图像在第二、四象限。图像本身也关于原点中心对称。$x$轴和$y$轴是函数的渐近线。

经过变形,可以得到$xy=p$,即$|x||y|=|p|$,这说明反比例函数描述了一种“此增彼减”的关系:当$|x|$增加时,$|y|$减少;当$|x|$减少时,$|y|$增加。同时,也说明反比例函数有一个性质:不论函数上的点$(x,y)$如何变化,$|x|\cdot|y|$总是定值,即函数上的任意点作坐标轴的垂线,与原点共同构成的矩形面积为定值。

\begin{definition}{二次函数}
形如
\begin{equation}
y = ax^2 + bx + c\quad(a\neq0)~.
\end{equation}
的函数称为\textbf{二次函数(quadratic function)}。
\end{definition}

二次函数的形状是一条抛物线。$a$的正负决定了抛物线的开口方向:$a>0$时,开口向上;$a$时,开口向下。抛物线与$y$轴的交点,即截距,是$y=c$。通过变形可以得到与定义等价的表达形式:

\begin{equation}
\begin{split}
y &= ax^2 + bx + c\\
&=a(x-(-{b\over2a}))^2-{b^2-4ac\over4a^2}
\end{split}~.
\end{equation}

从这个表达形式可以看出,函数存在一条对称轴$\displaystyle x=-{b\over 2a}$,图像上对称点到对称轴的距离相等,且连线与对称轴垂直。函数的最值点就在对称轴处,此时$\displaystyle y_0=-{b^2-4ac\over4a^2}$。通过讨论开口方向以及最值点的正负,可以知道函数是否与$x$轴有交点,交点的个数是几个。这个讨论过程经常出现,也是更复杂的表达式求解的基础,需要熟练掌握。

\begin{table}[ht]
\centering
\caption{二次函数与$x$轴的交点关系}\label{tab_HsFunB1}
\begin{tabular}{|c|c|c|c|}
\hline
& $y_0>0$即$b^2-4ac<0$ & $y_0=0$即$b^2-4ac=0$ & $y_0<0$即$b^2-4ac>0$ \\
\hline
$a>0$ & 无交点 & 一个交点 & 两个交点 \\
\hline
$a<0$ & 两个交点 &一个交点&无交点 \\
\hline
\end{tabular}
\end{table}

二次函数图像与 $x$ 轴的交点是方程的解,这在涉及距离、方差等概念的实际问题中可以帮助找到最优值或临界点。关于二次函数和一元二次方程的具体关系参见\enref{因式分解与一元二次方程}{quasol}。
