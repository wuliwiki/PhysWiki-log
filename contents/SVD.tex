% 奇异值分解(SVD)
% keys SVD|singular value decomposition|奇异值分解|正交矩阵|正交变换|线性代数|对角矩阵
% license Xiao
% type Tutor


\pentry{正规算子\nref{nod_NmOpt}, 相似变换和相似矩阵\nref{nod_MatSim}}{nod_98b8}

本节默认在\textbf{复数域及酉空间}下讨论。
\subsection{定理陈述} 
\begin{theorem}{SVD}
给定复数域$\mathbb{C}$上的矩阵 $\mat M$,则存在同样尺寸的对角矩阵$\mat D$和酉矩阵$\mat P$、$\mat Q$,使得
\begin{equation}
\mat{M} = \mat{P\Her DQ}~. 
\end{equation}
称$\mat{D}$的对角元为$\mat{M}$的\textbf{奇异值(singular value)},$\mat{P\Her DQ}$为$\mat{M}$的\textbf{奇异值分解(singular value decomposition)},简称为 \textbf{SVD}。
\end{theorem}


\subsection{证明过程}

我们先从\textbf{可逆方阵}证起,然后再过渡到不要求可逆的\textbf{任意复矩阵}。
\begin{theorem}{}\label{the_SVD_1}
对于$n$阶\textbf{可逆方阵}$\mat M$,存在\textbf{酉方阵}$\mat P,\mat Q$和\textbf{可逆对角矩阵}$\mat D$使得
\begin{equation}\label{eq_SVD_1}
\mat M=\mat P\Her \mat D\mat Q~.
\end{equation}

\end{theorem}
\textbf{证明:}因为$\mat M\Her \mat M$是半正定厄米矩阵,必存在一组标准正交基使之对角化,设为$\mat  A$,对应的过渡矩阵为酉方阵 $\mat Q$,则该过程表示为$\mat {Q}\mat M\Her \mat{MQ}\Her=\mat{A}$。左右取行列式可证$\mat A$是可逆矩阵。在这组基下,对对角元计算后,我们可以得到$\mat A^{\frac{1}{2}},\mat A^{-\frac{1}{2}}$,显然互逆。

我们可以把$\mat M$表示为$\mat {MQ}\Her \mat{A}^{-\frac{1}{2}}\mat A^{\frac{1}{2}}Q$,和\autoref{eq_SVD_1} 比较,我们需要验证$\mat {MQ}\Her \mat A^{-\frac{1}{2}}$是酉方阵。

设$\mat P\Her =\mat M\mat Q\Her \mat A^{-\frac{1}{2}}$,则$\mat P=\mat A^{-\frac{1}{2}}\mat Q\mat M\Her $,则
\begin{equation}
\begin{aligned}
\mat P\Her \mat P&=\mat {MQ}\Her  \mat A^{-\frac{1}{2}}\mat A^{-\frac{1}{2}}\mat Q\mat M\Her \\
&=\mat M\mat Q\Her (\mat Q\Her )^{-1}(\mat M)^{-1}(\mat M\Her )^{-1}\mat Q^{-1}\mat Q\mat M\Her \\
&=\mat E~,
\end{aligned}
~.
\end{equation}
定理得证。

在\autoref{the_SVD_1} 的陈述里$\mat D$也是可逆矩阵,这并非偶然。考虑到$\mat P^{-1}=\mat P\Her $,该定理实际上是$\mat M$的广义相似变换。如下图所示,线性映射$f:V\rightarrow W$在不同基下有不同表示。设$\{\bvec e_i\},\{\bvec \theta_i\}$是$V,W$的旧基,且$f$在该基下的表示为$\mat M$。$\mat P,\mat Q$是对应$W,V$的从旧基到新基的过渡矩阵,则$\mat D$就自然是$f$在新基下的表示了。作为$f$的不同表示,$\mat M$与$\mat N$的行列数相同,秩也相同。
\begin{figure}[ht]
\centering
\includegraphics[width=6cm]{./figures/d1e0436da4c251f4.png}
\caption{} \label{fig_SVD_2}
\end{figure}
如此一来,我们似乎可以拓展该定理至$\mat M$不可逆的情况。

\begin{lemma}{}\label{lem_SVD_1}
对于任意矩阵$\mat M$,存在酉矩阵$\mat Q',\mat S$和\textbf{满秩}矩阵$\mat N$,使得
\begin{equation}\label{eq_SVD_2}
\mat M=\mat S\Her \left(\begin{array}{cc}
N & 0 \\
0 & 0
\end{array}\right) \mat Q'~.
\end{equation}
\end{lemma}

\addTODO{引用定理“非退化情况下,子空间的标准正交基可拓展到全空间”。}

换一个方式表述,\autoref{lem_SVD_1} 实际上是在说对于酉空间的任意线性映射$f$,原像和像空间存在一组标准正交基,使得$f$在该基下是\autoref{eq_SVD_2} 所示的块对角形式。

考虑到$\opn{dim}V=\opn{dim}\opn{ker}f+\opn{dim}\opn{Im}f$,我们可以在$\opn{ker}f$内选一组标准正交基,然后扩展到全空间。设$k=\opn{dim}\opn{ker}f,s=\opn{dim}\opn{Im}f$。则在该组基下,$f$的后$k$列必为$0$。

同理,$\opn{Im}f$作为$W$的子空间,也可以找到一组标准正交基,并拓展到其正交补。因而除去前$s$行,其他行都为$0$。如此我们就证明了$f$具有\autoref{eq_SVD_2} 所示的块对角形式,其中$\mat N$必为满秩的$s\times s$\textbf{方阵},即其可逆。如若不然,$\opn{Im}f\cap\opn{ker}f\neq\{\bvec 0\}$。

现在我们来证明奇异值分解定理。
对于\textbf{任意复矩阵}$\mat M$,我们可以分解为\autoref{lem_SVD_1} 的形式。那么对于\textbf{满秩}$\mat N$矩阵,自然而然的,我们可以利用\autoref{the_SVD_1} 进行进一步的分解。设该分解表示为$\mat N=\mat T\Her \mat D'\mat F$,其中$\mat D'$为可逆对角矩阵。

构造复矩阵$\mat T_1,\mat S_1$,其中$\mat E_1,\mat E_2$是用作填充的单位矩阵,比如$\mat E_1$使得$\mat T_1$的行数等于$\mat S\Her $的列数。
\begin{equation}
\mat T_1\Her =\pmat{\mat T\Her &0\\0&\mat E_1},\quad \mat F_1=\pmat{\mat F&0\\0&\mat E_2}
~.
\end{equation}
该定义是合理的,因为
\begin{equation}
\pmat{\mat  N&0\\0&0}=\pmat{\mat  T\Her &0\\0&\mat E_1}\pmat{\mat D'&0\\0&0}\pmat{\mat S&0\\0&\mat  E_2}\equiv \mat T_1\Her\mat  D\mat F_1~.
\end{equation}

易证$\mat T_1\Her ,\mat F_1$是酉矩阵,可见$\mat D$便是定理里要求的对角矩阵,$\mat P\Her =\mat S\Her \mat T_1\Her ,\mat Q=\mat  F_1\mat Q'$。
