% 电势 电势能
% 电势|电势能|电荷|电场|做功

% 推导 1/r 势能, 计算连续电荷的势能, 例如带电球壳的势能
\pentry{力场 势能\upref{V}}

\subsection{电势能}
我们先来看电势能. 如果空间中存在某种分布的静电场(关于位置的矢量函数) $\bvec E(\bvec r)$, 一个电荷为 $q$ 的点电荷在任意位置 $\bvec r$ 都会受到电场力
\begin{equation}
\bvec F(\bvec r) = q \bvec E(\bvec r)
\end{equation}
这样我们就得到了一个由电场力构成的力场, 注意该力场与电场成正比. 

从某点 $\bvec r_1$ 沿着某个路径移动到另一点 $\bvec r_2$, 电场对它做功为
\begin{equation}\label{QEng_eq4}
W = q \int_{\bvec r_1}^{\bvec r_2} \bvec E(\bvec r) \vdot \dd{\bvec r}
\end{equation}

要讨论势能的概念, 我们首先需要确保这个力场是一个保守场, 即只有上式的结果只于初末位置有关而与路径无关时才可能存在一个势能函数. 以后我们会看到, 当空间中没有随时间变化的磁场时, 电场必定是一个保守场, % 链接未完成
所以我们接下来只讨论这种情况.

记有了势能函数(关于位置的标量函数)为 $E_p(\bvec r)$, 就有 “力场对物体做功等于势能减少”, 或者 “势能增加等于力场做的负功” 即
\begin{equation}\label{QEng_eq5}
E_p(\bvec r_2) - E_p(\bvec r_1) = -W = -q \int_{\bvec r_1}^{\bvec r_2} \bvec E(\bvec r) \vdot \dd{\bvec r}
\end{equation}

\subsection{电势}
注意以上公式中, 无论是做功还是势能都与电荷量 $q$ 有关(成正比). 为了更直接地描述电场本身的性质, 我们可以把这些公式两边都除以 $q$. 并将 $E_p/q$ 称为\textbf{电势}, 记为 $\bvec V(\bvec r)$ 这样\autoref{QEng_eq5} 就变为
\begin{equation}
V(\bvec r_2) - V(\bvec r_1) = - \int_{\bvec r_1}^{\bvec r_2} \bvec E(\bvec r) \vdot \dd{\bvec r}
\end{equation}
同样, 该积分只与初末位置有关而与路径无关. 这样一来, 电势就只是电场的属性而与电荷量无关了.

% 未完成: 例题: 匀强电场的电势, 点电荷的电势.


\subsection{点电荷的电势和电势能}
与万有引力同理, 两个点电荷之间的库仑力产生的势能为
\begin{equation}\label{QEng_eq2}
V(r) = k \frac{q_1 q_2}{r_{12}} = \frac{1}{4\pi\epsilon_0} \frac{q_1 q_2}{r_{12}}
\end{equation}
当两电荷异号时, $q_1 q_2$ 为负数, 此时与万有引力一样, 距离越近, 势能越小. 两电荷同号时 $q_1 q_2$ 为正, 距离越远势能越小.

\begin{example}{}
两个电荷量为 $Q$ 的电荷分别被固定在 $(-c, 0)$ 和 $(c, 0)$ 两点处, 另一质量为 $m$ 电荷量为 $q$ 的点电荷从 $(0, a)$ 延直线移动到 $(0, b)$, 试问它的动能增加了多少?

解: 由\autoref{QEng_eq2} 可知, 点电荷 $q$ 在 $(0, y)$ 处的电势能分别为
\begin{equation}
V = \frac{2kQq}{\sqrt{y^2 + c^2}}
\end{equation}
所以动能增加等于势能减少, 即
\begin{equation}
\Delta E_k = V_a - V_b = 2kQq \qty(\frac{1}{\sqrt{a^2 + c^2}} - \frac{1}{\sqrt{b^2 + c^2}})
\end{equation}
\end{example}

\subsection{连续电荷分布产生的电势}
\begin{equation}
V(\bvec r) = \frac{1}{4\pi\epsilon_0} \int \frac{\rho(\bvec r)}{\abs{\bvec r - \bvec r'}} \dd[3]{r'}
\end{equation}

\begin{example}{均匀带电球}
令球的半径为 $R$, 电荷密度为 $\rho$, 令无穷远处为零势点, 求均匀带电球内外的电势分布.

%未完成
\end{example}
