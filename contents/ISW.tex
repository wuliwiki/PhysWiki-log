% 无限深方势阱
% keys 定态薛定谔方程|无限深势阱|量子力学|波函数|归一化
% license Xiao
% type Tutor
% 插入 app https://wuli.wiki/apps/QMISW.html

\pentry{量子力学的基本原理(科普)\upref{QM0}, 定态薛定谔方程\upref{SchEq}, 二阶常系数齐次微分方程\upref{Ode2}}

无限深势阱通常被作为量子力学中定态薛定谔方程的第一个例子。 这是因为它具有非常简单的解析解。 薛定谔方程具有解析解的情况屈指可数,其中无限深势阱和简谐振子\upref{QSHOop}要数在之后的教学中最常出现的例子了。

只考察质量为 $m$ 的粒子沿 $x$ 方向的运动情况。% 有没有地方可以说明一下矢量空间, 我们只在 x 空间解决问题
势能函数为
\begin{equation}
V(x) =
\begin{cases}
0 \quad &(0 \leqslant x \leqslant a)\\
+\infty  &(x < 0 \ \text{或}\  x > a)~.
\end{cases}
\end{equation}

\begin{figure}[ht]
\centering
\includegraphics[width=9cm]{./figures/3c41d1217dc56a53.pdf}
\caption{无限深势阱} \label{fig_ISW_1}
\end{figure}

在经典力学中,这对应的情景就是一个弹性小球在两面无限硬度的墙之间的运动(无摩擦)。为了更好地把经典力学和量子力学的物理图像对应起来,我们可以假设粒子的初始波函数是势阱内一个带有初速度的波包,在势阱内部来回反弹\upref{ISWmat}。但从更量子力学的角度来说,我们应该先看看它的能量本征态(也就是束缚态),毕竟求解波包运动的第一步就是求解后者。

\subsection{束缚态}
定态薛定谔方程为\upref{SchEq}
\begin{equation}\label{eq_ISW_2}
-\frac{\hbar ^2}{2m} \dv[2]{x} \psi(x) + V(x)\psi(x) = E\psi(x)~.
\end{equation} 

这里先直接给出结论,证明见文末。 无限深势阱没有散射态\footnote{这是因为散射态的能量只能取 $E > V(\pm\infty)$。},但有无穷个非简并束缚态。也就是说\autoref{eq_ISW_2} 有无穷个解(记为 $\psi_n(x)$),且对应的能量本征值 $E$ 各不相同(如果同一个 $E$ 对应多个线性无关解,就叫做简并),记为 $E_n$。

\begin{equation}\label{eq_ISW_3}
E_n = \frac{\pi^2 \hbar^2}{2m a^2} n^2 \qquad (n = 1,2,3\dots)~.
\end{equation}
\begin{equation}\label{eq_ISW_1}
\psi_n(x) = \sqrt{\frac{2}{a}} \sin(\frac{n\pi }{a} x)~.
\end{equation}

一些教材中也会把势阱中间作为坐标原点, 这相当于把这里的势能 $V(x)$ 和所有 $\psi_n(x)$ 都向左移动 $a/2$。 能级仍然保持不变, 波函数变为(为了简洁适当乘以 $-1$, 物理意义不变)
\begin{equation}
\psi_n(x) = \leftgroup{
    &\sqrt{\frac{2}{a}}\sin(\frac{n\pi}{a} x)\qquad (\text{偶数 } n)\\
    &\sqrt{\frac{2}{a}}\cos(\frac{n\pi}{a} x)\qquad (\text{奇数 } n)~,
}\end{equation}
\begin{figure}[ht]
\centering
\includegraphics[width=8cm]{./figures/197f3970f51a81a9.pdf}
\caption{基态波函数,注意第 $n$ 个函数中间有 $n-1$ 个节点\upref{SchEq}。} \label{fig_ISW_2}
\end{figure}

事实上我们可以通过函数平移\upref{FunTra}的方法把势阱移动到任何位置。

把波函数用\autoref{eq_ISW_1} 的基底展开, 是三角傅里叶级数的一个变形(详见\autoref{eq_FSTri_8}~\upref{FSTri}), 注意虽然看上去少了 $\cos$ 基底, 但它的确是完备的。

\addTODO{束缚态的意义,复习一下}

\subsection{推导} 
先考虑势阱内部($0 \leqslant x \leqslant a$, $V = 0$),方程变为
\begin{equation}
-\frac{\hbar^2}{2m} \dv[2]{x} \psi(x) = E\psi(x) ~,
\end{equation}
这是二阶常系数齐次微分方程。通解为
\begin{equation}
\psi(x) = C_1\cos(kx) + C_2 \sin(kx)~, \qquad
k = \frac{\sqrt{2mE}}{\hbar}~.
\end{equation} 
通解也可以写成指数函数 $\psi(x) = C\E^{\I kx}$, 加上边界条件后得到的束缚态一样。

现在讨论边界条件: 在有限深势阱束缚态中将会看到,如果势阱外部势能是有限值,波函数将会按照指数函数衰减,势能越高衰减得越快。而现在势阱外部势能为无穷大,就可以直接认为波函数在势阱外部始终为零。所以边界条件为
\begin{equation}
\psi(0) = 0, \quad \psi(a) = 0~.
\end{equation}
这两个条件代入以上通解中,解得
\begin{equation}
C_2 = 0, \quad k = \frac{n\pi}{a}  \qquad (n = 1,2,3\dots)~.
\end{equation}

$C_1$ 的取值暂时不能确定,但先将通解写为 $\psi(x) = C\sin(\frac{n\pi }{a}x)$, 常数 $C$ 就可以通过波函数的归一化%(链接未完成)
来确定:
\begin{equation}
1 = \int_{-\infty }^{+\infty } \abs{\psi(x)}^2 \dd{x}  = \int_{-a}^{+a} \abs{C\sin(\frac{n\pi }{a} x)}^2 \dd{x}  = \abs{C}^2 \frac{a}{2}~.
\end{equation}
严格来说, $C$ 可以是复数,解为 $C = \sqrt{2/a} \E^{\I\theta}$。 但是为了方便通常把归一化常数中的相位因子$\E^{\I\theta}$ 默认为 $1$。 所以归一化的波函数为
\begin{equation}
\psi(x) = \sqrt{\frac{2}{a}} \sin(\frac{n\pi }{a}x)~.
\end{equation}
另外,由
\begin{equation}
\frac{\sqrt{2mE}}{\hbar} = k = \frac{n\pi }{a}~,
\end{equation}
可以得出能级是离散的结论。即
\begin{equation}
E_n = \frac{\pi^2\hbar^2}{2m a^2} n^2 \quad (n = 1,2,3\dots)~.
\end{equation}
