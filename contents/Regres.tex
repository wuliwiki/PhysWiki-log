% 回归
% keys 回归 机器学习
% license Xiao
% type Tutor

\textbf{回归}(Regression),在统计学中,是一种用于估计自变量(机器学习中称特征、属性)和因变量(机器学习中称标签)之间相关关系的分析方法[1]。回归分析的过程是确定最能够代表数据趋势的直线或者曲线[2]。所求得的回归直线或曲线,又可以称为拟合直线或拟合曲线。

回归也是一种机器学习中的基本建模方法。当所需要预测的数据是\textbf{连续型}数值时,该学习任务就是回归任务,须要用到回归方法,所求得的模型可以称为回归模型。这点是回归与\enref{分类}{Class}的主要区别。分类模型所预测的值是离散型数据。

值得注意的是,回归模型所输出的数据是连续性数值,但其输入数据可以是离散型数据。在实际的回归分析过程中,可以对离散型的特征(因变量)做连续化处理。

\textbf{回归分析步骤}
\begin{enumerate}
\item 数据收集: 收集包含自变量和因变量的数据集。
\item 建模: 选择适当的回归模型,例如线性回归、多元回归或非线性回归。
\item 拟合模型: 使用统计方法或机器学习算法拟合模型,找到最佳的参数值,使模型最好地适应数据。拟合过程通常涉及到最小化损失函数,例如最小二乘法,以找到最优的模型参数。
\item 评估模型: 使用各种评估指标(如均方误差、决定系数等)来评估模型的性能。
\item 预测: 应用建立的模型进行未来观测值的预测。
\end{enumerate}

\begin{table}[ht]
\centering
\caption{睡眠数据集}\label{tab_Regres1}
\begin{tabular}{|c|c|c|c|c|c|c|c|c|c|c|}
\hline
编号 & 性别 & 年龄 & 职业 & 睡眠时间(小时) & BMI指数 & 心率 & 舒张压 & 收缩压 & 每日走路步数 & 睡眠障碍 \\\hline
1 & 男 & 27 & 软件工程师 & 6.1 & 超重 & 77 & 83 & 126 & 4200 & 无 \\
\hline
2 & 男 & 28 & 医生 & 6.2 & 正常 & 75 & 80 & 125 & 10000 & 无 \\
\hline
3 & 女 & 30 & 护士 & 6.4 & 正常 & 78 & 86 & 130 & 4100 & 睡眠暂停 \\
\hline
4 & 男 & 29 & 教师 & 6.3 & 肥胖 & 82 & 90 & 140 & 3500 & 失眠 \\
\hline
\end{tabular}
\end{table}

我们来举一个例子。表1所示的是一个简单的睡眠数据集。例如,我们想通过人的年龄、睡眠时间、每日步数来预测心率。那么,该任务是一个简单的回归任务。因为,待预测的标签数据是心率,其数据类型是连续性。因此,该任务显然是一个回归任务,须要建模回归模型。

下面用数学形式来表示该回归任务。输入空间可以表示为:$\bvec X=\{\text{年龄},\text{睡眠时间},\text{每日步数}\}$。输出空间表示为:$\bvec Y=\{\text{心率}\}$。回归模型是:$f: \bvec X \rightarrow \bvec Y$。

回归模型可以有多种选择。线性模型是最常用,也是最基本的回归模型。如果特征(自变量)和标签(因变量)之间存在明显的线性相关关系,则可以采用线性模型来建模。采用线性模型的回归称为\enref{线性回归}{LiGr}。如果数据之间的相关关系较为复杂,可以采用非线性回归,或者高阶回归。


% 编写计划:
% 1.回归的数学推导过程
% 2.回归分析的软件操作过程或者代码编写,可以贴一点图,并放上一些代码


\subsubsection{参考文献:}
\begin{enumerate}
\item \verb`https://en.wikipedia.org/wiki/Regression_analysis`
\item \verb`https://www.britannica.com/topic/regression-statistics`
\end{enumerate}