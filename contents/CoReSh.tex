% 宇宙学红移
% keys FRW 度规|红移|哈勃常数
% license Usr
% type Tutor
\pentry{Friedmann-Robertson-Walker (FRW) 度规\nref{nod_FRW}}{nod_7dd5}
\begin{issues}
\issueMissDepend
\end{issues}

\pentry{FRW 度规 \nref{nod_FRW}}{nod_6c42}
由于物体在宇宙中传播的过程中,宇宙也在加速膨胀,所以为了准确的测量物体在宇宙传播过程中的物理量,我们需要引进宇宙学红移的概念。

\subsection{光子的红移}
假设位于共动坐标$r_1$处的光源在$t_1$时刻发出光子,原点处的我们在$t_0$时刻收到,不失一般性,假设$\dd\theta=\dd\phi=0$,那么根据光子走过的世界线为$0$这一事实,并结合FRW度规,我们有
\begin{equation}\label{eq_CoReSh_4}
s=\int^{t_0}_{t_1}\frac{c}{a(t')}\dd t'=\int_0^{r_1}\frac{1}{\sqrt{1-k(r/R)^2}}\dd r~.
\end{equation}

间隔一小段时间后,即在$t_1+\dd t_1$时,光源处再次发射出光子,而我们在$t_0+\dd t_0$时刻检测到,则
\begin{equation}\label{eq_CoReSh_3}
\int^{t_0+\dd t_0}_{t_1+\dd t_1}\frac{c}{a(t')}\dd t'=\int_0^{r_1}\frac{1}{\sqrt{1-k(r/R)^2}}\dd r~.
\end{equation}

由于间隔时间极短,用\autoref{eq_CoReSh_3} 减去\autoref{eq_CoReSh_4} ,可以解得:

\begin{equation}
\begin{aligned}
a(t_0)\dd t_0&=a(t_1)\dd t_1\\
\frac{a(t_0)}{a(t_1)}&=\frac{\dd t_1}{\dd t_0}~.
\end{aligned}
\end{equation}

假设我们检测的是连续传播的电磁波,$\dd t_0,\dd t_1$对应的是两个连续波峰的时间间隔,那么因为波长与该时间间隔成正比可知:
\begin{equation}\label{eq_CoReSh_1}
\lambda_0=\frac{a(t_0)}{a(t_1)}\lambda_1~,
\end{equation}
因为宇宙在加速膨胀 $a(t_0)>a(t_1)$, 所以容易得 $\lambda_0>\lambda_1$。也就是说,宇宙的加速膨胀使得我们测得的电磁波波长更长,这就是所说的红移效应。因此,观测到红移意味着星系都在远离我们。


\subsection{红移因子}
\begin{definition}{}
为衡量红移效应的大小,我们定义\textbf{红移因子(redshift parametre)}为
\begin{equation}
z=\frac{\lambda_0-\lambda_1}{\lambda_1}~,
\end{equation}
\end{definition}
显然从\autoref{eq_CoReSh_1} 我们可以推出
\begin{equation}
\begin{aligned}
1+z&=\frac{a(t_0)}{a(t)}=\frac{1}{a(t_1)}\\
a(t_1)&=\frac{1}{1+z}
\end{aligned}~,
\end{equation}
一般地我们设现在的宇宙尺度因子 $a(t_0)=1$。在有了红移因子的定义后,我们就多了一种描述宇宙内物体位置的方式,

\subsection{哈勃常数}
我们把 $t_1$ 时刻的宇宙尺度因子 $a(t_1)$ 以现在的时刻 $t_0$ 为原点作泰勒展开,可得
\begin{equation}
a(t_1)=a(t_0)(1+(t-t_0)H_0+\cdots)~.
\end{equation}
则 $z=H_0(t_0-t_1)$,这里我们定义了\textbf{哈勃常数(Hubble Constant)} $H_0$. 显然我们可以发现红移参量与光走过的距离 $d=c(t_0-t_1)$ 成正比
\begin{equation}
z\simeq\frac{H_0d}{c}~.
\end{equation}

哈勃常数常常被定义为以下数值
\begin{equation}
H_0 \equiv100 h {\rm kms}^{-1}{\rm Mpc}^{-1}~.
\end{equation}
其中 $h$ 测得的数值为\footnote{12Planck 2013 Results – Cosmological Parameters [arXiv:1303.5076]}
\begin{equation}
h\sim 0.67 \pm 0.01~.
\end{equation}
