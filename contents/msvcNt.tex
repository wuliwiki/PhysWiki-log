% MSVC (微软 C++ 编译器)笔记
% license Usr
% type Note

参考 \verb`CMakeTest` 仓库中的 \verb`msvc-dyn-lib` 子文件夹。 CMake 如果适配了 VS,且系统安装了 VS,那么可以用构造程序
\begin{lstlisting}[language=none]
cmake -DCMAKE_BUILD_TYPE=Debug -G "Visual Studio 16 2019" ..
cmake --build . --verbose
\end{lstlisting}

\subsection{常识}
\begin{itemize}
\item \verb`cl.exe` 是编译器,默认安装路径 \verb`C:/Program Files (x86)/Microsoft Visual Studio/2019/Community/VC/Tools/MSVC/14.29.30133/bin/Hostx64/x64/cl.exe`。 注意微软命令行程序的选项以 \verb`/` 开头
\item \verb`MSBuild.exe` 是微软自家的构建程序(类似于 make)它会根据 cmake 生成的 \verb`.sln` 和 \verb`.vcxproj` (类似于 \verb`Makefile`) 调用 \verb`cl.exe` 和 \verb`link.exe` 等,默认路径 \verb`C:/Program Files (x86)/Microsoft Visual Studio/2019/Community/MSBuild/Current/Bin/MSBuild.exe`。
\item 其实静态库,和动态导入库,都是 \verb`.lib` 拓展名, 真正的动态库 \verb`.dll` 编译链接时用不上只有运行时才需要。
\end{itemize}

\subsection{CMakeLists 适配}
\begin{itemize}
\item utf-8 编码文件中的中文注释需要加个编译器选项 \verb`utf-8`,否则可能会被误认为是源码
\item 如果 cmake 不做特别适配,编译出的库默认不以 \verb`lib` 开头,找链接库时也是。
\item \verb`min, max` 在 windows SDK 中定义成为宏, 需要声明一个宏才可以禁止声明。
\end{itemize}
