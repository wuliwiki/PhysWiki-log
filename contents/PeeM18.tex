% 2018 年考研数学试题(数学一)
% keys 考研|数学
% license Xiao
% type Tutor

\subsection{选择题}
1.下列函数中,在 $x=0$ 处不可导的是 $(\quad)$\\
(A)$f(x)=\abs{x} \sin \abs{x} \qquad$  (B)$f(x)=\abs{x} \sin \sqrt{\abs{x}}$\\
(c)$f(x)=\cos \abs{x} \qquad \quad$  (D)$ f(x)=\cos \sqrt{\abs{x}}$

2.过点$(1,0,0),(0,1,0)$,且与曲面$z=x^2+y^2$相切的平面为 $(\quad)$\\
(A)$z=0$ 与 $x+y-z=1$\\
(B)$z=0$ 与 $2x+2y-z=2$\\
(C)$x=y$ 与 $x+y-z=1$\\
(D)$x=y$ 与 $2x+2y-z=2$

3.$\displaystyle \sum_{n=0}^\infty (-1)^n \frac{2n+3}{(2n+1)!}$ = $(\quad)$ \\
(A)$\sin 1+\cos 1 \qquad$  (B)$2\sin 1+\cos 1$ \\
(C)$2\sin 1+2\cos 1  \quad$(D)$2\sin 1+3\cos 1$

4.设$\displaystyle M=\int_{-\frac{\pi}{2}}^\frac{\pi}{2}\frac{(1+x)^2}{1+x^2}\dd{x},N=\int_{-\frac{\pi}{2}}^\frac{\pi}{2}\frac{1+x}{e^x}\dd{x},K=\int_{-\frac{\pi}{2}}^\frac{\pi}{2}(1+\sqrt{\cos x})\dd{x}$,则 $(\quad)$ 。\\
(A)$M>N>K \quad$ (B)$M>K>N \quad$ (C)$K>M>N \quad$ (D)$K>N>M$

5.下列矩阵中,与矩阵 $\pmat{1&1&0\\0&1&1\\0&0&1}$ 相似的为 $(\quad)$\\
(A)$\pmat{1&1&-1\\0&1&1\\0&0&1} \quad$
(B)$\pmat{1&0&-1\\0&1&1\\0&0&1} \quad$
(C)$\pmat{1&1&-1\\0&1&0\\0&0&1} \quad$
(D)$\pmat{1&0&-1\\0&1&0\\0&0&1}$

6.设$\mat A,\mat B$为$n$阶矩阵,记$r(\mat X)$为矩阵$\mat X$的秩,$(\mat X,\mat Y)$表示分块矩阵,则 $(\quad)$ \\
(A)$r(\mat A,\mat {AB})=r(\mat A) \qquad \qquad$
(B)$r(\mat A,\mat {BA})=r(\mat A) \quad$\\
(C)$r(\mat A,\mat B)$=max $ \{r(\mat A),r(\mat B)\}$
(D)$r(\mat A,\mat B)=r(\mat A \Tr,\mat B \Tr)$

(A)
(B)
(C)
(D)