% Julia 通用操作
% 通用操作

本文授权转载自郝林的 《Julia 编程基础》. 原文链接:\href{https://github.com/hyper0x/JuliaBasics/blob/master/book/ch08.md}{第 8 章 容器:字典与集合}.


\subsubsection{8.4 通用操作}

说到适用于各种容器的函数,其实我在前面已经提到过一些.比如,\verb|empty!|函数能够清空作为参数值的那个容器.这里的参数值不仅可以是字典,也可以是集合、数组等可变的容器.

\verb|empty!|函数还有一个名为\verb|empty|的孪生函数.这个函数不会对参数值进行任何的修改,而且会返回一个与参数值一模一样但不包含任何元素值的结果值.另外,还有一个名称很相近的函数\verb|isempty|,它可以判断一个容器是否为空.

在很多时候,仅仅判断一个容器是否为空是远远不够的,我们还需要知道一个容器总共容纳了多少个元素值或键值对.这时就需要用到\verb|length|函数了.如果想要获得容器的元素类型,那么可以调用\verb|eltype|函数.我们在前面用过这个函数.当向它传入字典的时候,它会返回字典中键值对的具体类型.另外,我们若想知道一个值是否是某个容器中的元素值、键或键值对,那么就可以通过调用\verb|in|函数得到答案.

此外,还有一些函数或操作可以应用于可索引对象和/或可迭代对象.我在前面讲索引与迭代的时候已经有过说明,因此就不再赘述了.

当然了,我在这里不可能罗列出所有的相关函数和操作.如果你需要的通用操作我没在这里讲到,那么你可以通过某种渠道向我提问,或者去查阅官方的文档,也可以到Julia社区的论坛里求助.