% 中国科学院大学 2020 年考研 811 量子力学
% keys 考研|量子力学|真题
% license Copy
% type Tutor

\textbf{声明}:“该内容来源于网络公开资料,不保证真实性,如有侵权请联系管理员”

一、
	(共 30 分)考虑一维束缚态。
	\begin{enumerate}
		\item
		证明 $\langle\psi(x, t) \mid \psi(x, t)\rangle$ 不随时间变化, 此处波函数 $\psi$ 不必是定态。
		\item 
		证明对于定态, 动量的期望值为零。
		\item 
		证明如果粒子在 $t=0$ 时刻处于定态, 则在以后时刻永远保持定态。
		
		
		
	\end{enumerate}

二、	(共 30 分)设波函数$\psi(x)=(1 / \sqrt{2 \pi}) e^{i p(x+\beta) \hbar}$, 而 $\hat{x}, \hat{p}$ 分别为 $x$ 方向的坐标和动量算符,其中 $\beta$ 为实常数。
\begin{enumerate}
	\item
说明 $\psi(x)$ 是否为 $\hat{p}$ 的归一化本征态。
\item 
证明 $\left\langle x^{\prime}\right| e^{i \alpha \hat{p} / \hbar}=\left\langle x^{\prime}+\alpha\right|$, 及 $e^{i \alpha \hat{p} / \hbar}\left|x^{\prime}\right\rangle=\left|x^{\prime}-\alpha\right\rangle$, 其中 $\alpha$ 为实常数。
\item 
化简算符 $e^{i \alpha \hat{p} / \hbar} \hat{x} e^{-i \alpha \hat{p} / \hbar}$ 。
\item 
化简算符 $e^{i \alpha \hat{p} / \hbar} \hat{x}^{2} e^{-i \alpha \hat{p} / \hbar}$ 。

\end{enumerate}


三、( 共 30 分 ) 一个无自旋粒子的波函数为 $\psi=K(x+i y+2 z) e^{-\alpha r} $, 此处 $r=\sqrt{x^{2}+y^{2}+z^{2}}$, 其中 $K, \alpha$ 为实常数。(球谐函数: $Y_{0}^{0}=\sqrt{\frac{1}{4 \pi}}$, $ Y_{1}^{0}=\sqrt{\frac{3}{4 \pi}} \cos \theta$, 
$Y_{1}^{\pm 1}=\mp \sqrt{\frac{3}{8 \pi}} \sin \theta e^{\pm i \phi} $)
\begin{enumerate}
	\item
	求粒子的总角动量。
	\item 
	求角动量 $z$ 分量即 $\hat{L}_{z}$ 的期望值, 及测得 $L_{z}=\hbar$ 的概率。
	\item 
	求发现粒子在 $(\theta, \varphi)$ 方向上 $\text{d}  \Omega$ 立体角内的概率。
	
	
	
\end{enumerate}

四、(共 30 分)
\begin{enumerate}
	\item
	一个电子在 $t=0$ 的时刻处于自旋态 $\chi=\frac{1}{3}\left(\begin{array}{c}1-2 i \\ 2\end{array}\right)$ 。在 $t>0$ 时刻,在外界加一个磁场 $\vec{B}=B_{0}\left(\sin \theta \hat{e}_{x}+\cos \theta \hat{e}_{z}\right)$, 此时电子的哈密顿量为 $\hat{H}=-2 \mu_{B} \hat{\vec{S}} \bullet \vec{B}$, 其中 $\hat{\vec{S}}$ 为自旋算符, $\mu_{B}$ 为玻尔磁子,求此粒子在任意 $t$ 时刻的波函数。
	\item 
	考虑两个自旋为 $\frac{1}{2}$ 的粒子处于磁场中,此时系统的哈密顿量为
	\[ \hat{H}=a \hat{\sigma}_{1 z}+b \hat{\sigma}_{2 z}+c_{0} \hat{\bar{\sigma}}_{1} \bullet \hat{\vec{\sigma}}_{2} ~, \]
	其中 $a, b, c_{0}$ 为常数, $\hat{\vec{\sigma}}$是泡利算符,前两项为粒子处于磁场中的势能,最后
	一项为两粒子自旋-自旋相互作用能。求系统的能级。
\end{enumerate}

五、(共 30 分)考虑谐振子问题。
\begin{enumerate}
	\item
	一维谐振子的哈密顿量为 $\hat{H}=\frac{\hat{p}^{2}}{2 m}+\frac{1}{2} k \hat{x}^{2}$,
	证明由不确定性关系得到的能量最小值与该谐振子的基态能量一致。
	\item 
	若(1)中的基态波函数是高斯型 $e^{-\beta x^{2}}$, 用变分法求 $\beta$ 。
	\item 
	利用升、降算符写出(1)中的第一激发态的波函数(不必归一)。
	\item 
	对于三维各向同性谐振子 ,第一激发态是三重简并的。现有一微扰
	\[ 
	\hat{H}^{\prime}=b \hat{x} \hat{y}=\frac{\hbar b}{2 m \omega}\left(\begin{array}{ccc}0 & 1 & 0 \\ 1 & 0 & 0 \\ 0 & 0 & 0\end{array}\right) ~.
	 \]
	写出该微扰引起的第一激发态的能级分裂。
\end{enumerate}
