% 矩阵与映射
% license Xiao
% type Tutor


注:为方便计,本篇采取爱因斯坦求和约定。即$a^i\bvec x_i=a^1\bvec x_1+a^2\bvec x_2+...$
\subsection{线性映射及其矩阵表示}
$V$和$W$是域$\mathbb F$上的两个线性空间。若映射$f:V\rightarrow W$保加法和数乘运算(即“线性”),即对于任意$\bvec x,\bvec y \in V,,a,b\in\mathbb F$有:
\begin{equation}
f(a\bvec x+b\bvec y)=af(\bvec x)+bf(\bvec y)
~,\end{equation}
则称$f$是\textbf{线性映射}(linear map),以$\mathcal L(V,W)$表示所有$V\rightarrow W$上的线性映射。可以验证,该集合是一个线性空间。一般称$f\in \mathcal L(V,V)$为\textbf{为线性变换}(linear transformation),全体\textbf{可逆线性变}换用$GL(V)$表示。

由于矩阵对向量作用,也是保线性运算不变的,因此矩阵和映射之间可以一一对应起来。矩阵的乘法就是映射的复合。
具体怎么对应呢,设$\{\bvec e_i\}$为$V$上的一组基,$f\in\mathcal L(V,W)$,其矩阵表示为$M$\footnote{$m^i_j$表示第$i$行$j$列的矩阵元}。对于任意向量$\bvec x=a^i\bvec e_i\in V$有:
\begin{equation}
f(a^i\bvec e_i)=a^if(\bvec e_i)=m^j_i a^i~.
\end{equation}
因此,矩阵的第$i$列是$f(\bvec e_i)$。选定$W$的一组基,那么$f(\bvec e_i)$总可以表示为$W$里基向量的线性组合。
\begin{example}{}
$\{\bvec x_1,\bvec x_2\}$是$V$上的一组基,$\{\bvec y_1,\bvec y_2,\bvec y_3\}$为$W$上的一组基。线性映射$f:V\rightarrow W$对基向量的作用为:$f(\bvec x_1)=2\bvec y_2+\bvec y_3\,,f(\bvec x_2)=2\bvec y_1-3\bvec y_3$,则有
\begin{equation}
f(2\bvec  x_1-\bvec x_2)=-2\bvec y_1+4\bvec y_2+5\bvec y_3~,
\end{equation}其矩阵表示为
\begin{equation}
\begin{pmatrix}
  0& 2\\
  2& 0\\
 1 &-3
\end{pmatrix}\times \begin{pmatrix}
 2 &-1
\end{pmatrix}=\begin{pmatrix}
-2 \\
 4\\
5
\end{pmatrix}~.
\end{equation}
\end{example}
\subsection{矩阵的坐标变换}
在矩阵论里,我们已经学过用相似变换得到线性变换在不同基下的表示。现在用矩阵表示不同线性空间之间的线性映射,我们依然可以得到矩阵在不同基下的表示。

设$V,W$的旧基分别为$\{\bvec e_i\},\{\bvec{\theta}_i\}$;新基分别为$\{\bvec{x}_i\},\{\bvec{y}_i\}$;$f:V\rightarrow W$在旧基下的表示为$M$,在新基下的表示为$N$;从$V$的旧基到其新基的过渡矩阵为$Q$,$W$的则为$S$。现在推导$M,N$的联系。

对于任意向量$\bvec v=b^i\bvec x_i\in V$,我们有:
\begin{equation}\label{eq_lnal04_1}
\begin{aligned}
f(b^i\bvec x_i)&=b^if(\bvec x_i)\\
&=b^iN^j_i\bvec y_j\\
&=b^iN^j_iS^a_j\bvec\theta_a~,
\end{aligned}
\end{equation}
由于对于$\bvec x_i$有:$\bvec x_i=Q_i^q\bvec e_i$,代入第一行,则得:
\begin{equation}
\begin{aligned}
b^if(\bvec x_i)&=b^if(Q_i^q\bvec e_q)\\
&=Q_i^qb^if(\bvec e_q)\\
&=Q_i^qb^iM^a_q\bvec \theta_a~.
\end{aligned}
\end{equation}
$b^i\theta_a$前的系数必相等,综合\autoref{eq_lnal04_1} 得:
\begin{equation}
SN=MQ~,
\end{equation}
即$N=S^{-1}MQ$。
\subsection{线性映射的核与象}

\begin{definition}{}
设$V,W$为域$\mathbb F$上的线性空间,$f:V\rightarrow W$为线性映射。

记$\opn{ker}f=\{\boldsymbol x\in V|f(\boldsymbol x)=\boldsymbol 0\}$,称作线性映射$f$的核(kernel)。记$\opn{Im}f=\{f(\boldsymbol x)|\boldsymbol x\in V\}$,称作线性映射$f$的象(image)
\end{definition}
\begin{exercise}{}
$f,V,W$的定义同上。验证核与象分别是$V$及$W$的子空间。
\end{exercise}
关于核与象,有两个好用的结论。
\begin{itemize}
\item 核$\opn{ker}f=\{\boldsymbol 0\}\Longleftrightarrow f$是单射。
\item 若象$\opn{Im}f=W\Longleftrightarrow f$是满射
\end{itemize}
在此只证明第一个结论。

proof.

先验证充分条件。反证该映射并非单射,及至少存在两个向量映射到同一个向量,设为$\boldsymbol{x,y}$,那么我们有
\begin{equation}
f(\boldsymbol{x}-\boldsymbol{y})=f(\boldsymbol x)-f(\boldsymbol y)=\boldsymbol 0~,
\end{equation}
由于核只有向量$0$,因此$\boldsymbol {x}=\boldsymbol{y}$

再验证必要条件。假设存在一个非$0$向量映射到$0$,即$f(a^i\boldsymbol x_i)=0$,则$-f(a^i\boldsymbol x_i)=f(-a^i\boldsymbol x_i)=0$,与假设矛盾,证毕。
可见,第一条结论能成立多亏了该同态映射是线性的,这也是线性空间的一个好处。

线性空间的向量构成加法群,因而也有同态定理:
\begin{theorem}{}
设$V,W$是域$\mathbb F$上的有限维线性空间。$f:V\rightarrow W$为线性映射。则有:
\begin{equation}
V/\opn{ker}f\cong \opn{Im} f~,
\end{equation}
\end{theorem}
Proof.

由于有限维线性空间的同构只需要维度相同,所以我们只需要构建基之间的映射即可。

设$\{\boldsymbol e_i\}$为$\opn{ker}f$上的一组基,扩充为$V$上全空间的基:$\{\boldsymbol e_i\}\cup \{\boldsymbol \theta_i\}$。由于核中元素都被映射为0.只要证明象的维度与$|\{\boldsymbol \theta_i\}|$一致即可。由于$f(a^i\boldsymbol e_i+b^i\boldsymbol \theta_i)=b^if(\boldsymbol \theta_i)$,而$f(\boldsymbol \theta_i)$是线性无关的,不然$\opn{span}\{\boldsymbol \theta_i\}$就会有$\opn{ker}f$的元素,与假设矛盾。证毕。
该证明同时也引出了以下定理:
\begin{lemma}{}
对于线性空间$V$和其上的线性映射$f$,我们有
\begin{equation}
\mathrm{dim}V=\mathrm{dim}\,\opn{ker}f+\mathrm {dim}\,\opn{Im}f~,
\end{equation}
\end{lemma}
利用该定理,我们可以证明一条关于秩的定理:
\begin{theorem}{}
给定两个$n$阶方阵$A$和$B$,若$AB=0$,我们有
\begin{equation}
rank\,A+rank\,B\le n~,
\end{equation}
\end{theorem}
proof.

设$A,B$对应线性空间$V$的线性变换为$f_A,f_B$,该定理又可理解为$\opn{Im}f_A+\opn{Im}f_B\le n$。

$AB=0$意味着$\opn{Im}f_B\le \opn{ker}f_A$,由定理2得:$\opn{ker}f_A=n-\opn{Im}f_A$,移项证毕。
\subsection{矩阵的秩}
由前文知,若$f$是一个把$n$维向量映射到$m$维向量的线性映射,那么该线性映射可以表示为一个$m$行$n$列的矩阵。由于象的维度=矩阵的列秩=行秩,因此线性映射的性质与矩阵的行向量组、列向量组的秩紧密关联。
\begin{theorem}{}
$f:V\rightarrow W$,对应$x\rightarrow Ax$。则:
\begin{itemize}
\item $f$是单射$\Leftrightarrow\,A$的列向量组线性无关。
\item $f$是满射$\Leftrightarrow\,A$的行向量组线性无关。
\item $f$是双射$\Leftrightarrow\,A$是可逆矩阵。
\end{itemize}
\end{theorem}
proof.

设$\opn{dim}V=n,\opn{dim}W=m$,${\bvec e_i}$为$V$的一组基,任意向量$\bvec x=a^i\bvec e_i\in V$,则:

$f$是单射$\Leftrightarrow\opn{ker}f=0\Leftrightarrow a^if(\bvec e_i)=\bvec 0$只有零解$\Leftrightarrow a^i=0$,得证。

$f$是单射$\Leftrightarrow \opn{dim}\opn{Im}f=m\Leftrightarrow$行秩为$m$,得证。

结合上述两点可证第三点。