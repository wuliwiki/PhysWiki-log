% 点集的内部、外部和边界
% 点集|内部|外部|边界|拓扑|开集|孤立点|闭集

\pentry{拓扑空间\upref{Topol}}
拓扑空间定义了什么样的集合算开集,而开集的补集被称为闭集。除此之外,还有一些集合是既不开也不闭的。比如说,在 $\mathcal{R}$ 上定义的度量拓扑中,开区间都是开集,闭区间和孤立点都是闭集,但是半开半闭区间两者都不属于。

\subsection{基本概念}
\begin{definition}{}\label{def_Topo0_1}
给定拓扑空间 $(X, \mathcal{T})$ 以及它的一个子集 $A$。
\begin{itemize}
\item 如果对于 $x\in X$,存在一个开集 $V\in \mathcal{T}$ 使得 $x\in V\subset A$,那么称这个 $x$ 是 $A$ 的一个\textbf{内点(interior point)}。
\item 如果存在一个开集 $V\in \mathcal{T}$ 使得 $x\in V\subset A^C$,那么称这个 $x$ 是 $A$ 的一个\textbf{外点(exterior point)}。
\item 如果一个 $x$ 既不是 $A$ 的外点也不是内点,那么称 $x$ 是 $A$ 的一个\textbf{边界点(boarder point)}。

\item 全体内点构成的集合,称为 $A$ 的\textbf{内部(interior)},记为 $A^\circ$。
\item 全体外点构成的集合,称为 $A$ 的\textbf{外部(exterior)}。
\item 全体边界点构成的集合,称为 $A$ 的\textbf{边界(boarder)}。
\item 如果包含点 $x\in X$ 的任何开集 $U_x$,都含有和 $x$ 不相同的、$A$ 的元素:$(U_x-\{x\})\cap A\not=\varnothing$,那么我们说 $x$ 是 $A$ 的一个\textbf{聚点}或者\textbf{极限点(limit point)}。
\end{itemize}
\end{definition}

外点就是“补集的内点”。

聚点的概念在数学分析和高等微积分里就会出现。在微积分中,一个集合 $A\subset\mathbb{R}$ 的聚点 $x$,就是不管取多小的 $r>0$ 作为半径,总有某个\textbf{不是}$x$ 的点 $x_r\in A$ 落在 $(x-r, x)\cup(x, x+r)$ 里面。$A$ 自己的点 $x\in A$ 当然是 $A$ 的聚点,但是 $x\not\in A$ 也可以是聚点。比如说,区间 $(0,1)$ 的全体聚点构成的集合,就是 $[0,1]$。

有了这些定义,我们就可以更直观地理解开集和闭集了:

\begin{exercise}{}\label{exe_Topo0_1}
给定 $(X, \mathcal{T})$ 以及它的一个子集 $A$。那么,$A$ 是开集当且仅当 $A$ 的点都是内点;$A$ 是闭集当且仅当 $A$ 的聚点都在 $A$ 里。证明留做习题。
\end{exercise}

同样地,使用区间为例子来理解开集、闭集和内点、聚点的关系会非常直观。

由定义,对于任何拓扑空间的子集 $A$,外点都不是聚点,因此聚点要么是内点、要么是边界点。

对于 $A$,它的聚点构成的集合被称为 $A$ 的\textbf{导出集(induced set)}或者\textbf{导集},在本书中记为\footnote{有的教材中,$A'$ 代表的是补集,但在本书中我们使用 $A^C$ 表示补集了,于是 $A'$ 可以腾出来表示导集。}$A'$。

\begin{example}{}
\begin{itemize}
\item 令 $A=\{\frac{1}{n}|n\in\mathbb{Z}^+\}$,那么 $A'=\{0\}$。
\item 令 $A=(0,1)\cup(1,2)$,那么 $A'=[0,2]$
\end{itemize}
\end{example}

\begin{exercise}{}
证明:任何集合的导集都是闭集。
\end{exercise}

在数学分析中,我们会使用包含点 $x$ 的任意长度的开区间,来描述 $x$ 的“附近”。这一概念在拓扑学中被拓展为“邻域”:

\begin{definition}{邻域}
给定拓扑空间 $X$ 和 $x\in X$,则任意开集 $U_x\ni x$,被称为 $x$ 的一个\textbf{邻域(neighborhood)}。
\end{definition}

注意,这里定义的邻域必须是开集。在紧致性\upref{Topo2}一节中,我们会提到一个容易混淆的概念,“紧邻域”,它并不一定是开集,因此不一定是邻域。紧邻域的名字是有些混淆,但是没有办法,约定俗成这么叫了\footnote{事实上,一种常见的表达方式是在“邻域”两字前加其它修饰,这样表达的往往不是邻域(即不是开集),而是包含某个邻域的集合。“紧邻域”表示包含一个邻域的紧集,“闭邻域”表示包含一个邻域的闭集,它们往往都不是开集,从而不是邻域。}。



\subsection{闭包}

给定任意拓扑空间 $X$ 的任意子集 $A$,那么 $A^\circ$ 是 $A$ 所包含的最大的开集。对称地,包含 $A$ 的最小的闭集被称为 $A$ 的\textbf{闭包(closure)},记为 $\bar{A}$。闭包很容易得到:

\begin{theorem}{}
给定任意拓扑空间 $X$ 的任意子集 $A$,则有 $\bar{A}=A\cup A'$。
\end{theorem}
\begin{corollary}{}\label{cor_Topo0_1}
考虑到 $A$ 的边界点要么在 $A$ 中,要么是 $A$ 的聚点;$A$ 的聚点要么在 $A$ 中,要么是 $A$ 的边界点;故 $\bar{A}$ 等于 $A$ 和 $A$ 的边界的并。
\end{corollary}

以区间作为例子:如果 $A=(0,1)$,那么 $A'=[0,1]$,$\bar{A}=[0,1]$。

以圆盘作为例子:如果 $D=\{(x,y)\in\mathbb{R}^2|x^2+y^2<1\}$,那么 $D'=\bar{D}=\{(x,y)\in\mathbb{R}^2|x^2+y^2\leq1\}$。尝试在 $x^2+y^2=1$ 的圆盘上添加部分点进 $D$,得到的 $D$ 还是具有相同的闭包。这是因为这些添加的点都是在最初的 $D$ 的边界上的。

\subsection{稀疏和稠密}
\pentry{集合的极限\upref{SetLim}}
给定任意拓扑空间 $X$ 的任意子集 $A$,如果 $A^\circ=\varnothing$,即 $A$ 没有内点,那么称 $A$ 是\textbf{稀疏(sparse)}的;如果 $\bar{A}=B\subset X$,也就是说 $B$ 是 $A$ 和 $A$ 的边界的并,那么称 $A$ 在 $B$ 中是\textbf{稠密(dense)}的。

无论取什么开集,它都不可能被稀疏的集合完全包含,总有遗漏的点,所以有了“稀疏”的名称。而当 $A$ 在 $B$ 中稠密时,不可能找一个开集来圈出 $B$ 中完全和 $A$ 不相交的一块区域,所以有了“稠密”的名称。

全体有理数的集合 $\mathbb{Q}$ 在 $\mathbb{R}$ 中既稠密,又稀疏。

\begin{example}{Cantor集}\label{ex_Topo0_2}
令 $A_0=[0,1]$,也就是说,$A_0$ 是一段长度为 $1$ 的线段。把这条线段的中间挖去 $1/3$,得到 $A_1=[0,1/3]\cup[2/3,1]$。现在 $A_1$ 有了两段长为 $1/3$ 的线段,同样地,分别挖去它们中间的 $1/3$,得到 $A_2$。以此类推,$A_n$ 有 $2^n$ 条长度为 $1/3^n$ 的线段,把每条线段的中间挖掉 $1/3$,就可以得到 $A_{n+1}$。取这一列集合的极限 $C=\lim_{n\rightarrow\infty}A_n$,所得到的 $C$ 被称为\textbf{Cantor集},这是一个非常重要的集合,常用来作特例来阐释各种集合的性质。

这是一个稀疏集,同时又是一个\textbf{完备集(perfect set)}\footnote{导集还是自身的集合,称为完备集或者完美集。}另外,在测度论中,这个集合的测度是0. 
\end{example}
