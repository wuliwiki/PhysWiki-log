% jQuery 笔记
% license Xiao
% type Note

\begin{issues}
\issueDraft
\end{issues}

\pentry{JavaScript 入门笔记\upref{JS}}

\begin{itemize}
\item jQuery 命令通常在 document ready 内部使用 \verb|$(document).ready(function(){/*jQuery 命令*/});|, 另一种等效的格式为 \verb|$(function(){/*jQuery 命令*/})|
\item Selector: \verb|$("p")| 选择所有 \verb|<p>...</p>| 元素。 例如: \verb|$(document).ready(function(){$("button").click(function(){$("p").hide();});});| 其中 \verb|click(回调函数)| 设置按钮的回调函数, 即 \verb|$("p").hide()|, 隐藏所有段落。 相反, \verb|.show()| 重新显示。
\item \verb|.hide(speed, callback)| 其中 \verb|speed| 是毫秒, \verb|callback| 是完成后的回调
\item \verb|.toggle()| 或者 \verb|.toggle(speed, callback)| 让隐藏的元素显示, 显示的元素隐藏
\item \verb|$("#id")| 选中所有指定 id 的元素
\item \verb|$(".class")| 选中所有指定 class 的元素
\end{itemize}

事件
\begin{itemize}
\item \verb|$(document).ready()| 网页渲染完成
\item \verb|.click()| 按下, \verb|.dbclick()| 双击
\item \verb|.mouseenter()| 鼠标进入元素。 例子: \verb|$(document).ready(function(){$("#p1").mouseenter(function(){做一些事});});|
\item \verb|.mouseleave()| 离开元素, \verb|.mousedown()| 按下鼠标左键, \verb|.mouseup()| 松开鼠标左键。
\item \verb|.hover(fun1,fun2)| 相当于 \verb|.mouseenter(fun1); .mouseleave(fun2);|
\item \verb|<input>| 元素有 \verb|$("input").focus()| 和 \verb|.blur()|, 分别是光标选中和离开
\item 动画没有执行完就可以执行下面的代码(异步)
\end{itemize}

\subsection{DOM manip}
\begin{itemize}
\item Document Object Model (DOM)
\item allows programs and scripts to dynamically access and update the content, structure, and style of a document
\item \verb|attr(属性)| 获取元素的属性, 如 \verb|$("#id").attr("href")|
\end{itemize}

一个获取鼠标坐标的例程
\begin{lstlisting}[language=js]
<!DOCTYPE html>
<html>
<head>
	<script src=
	"https://ajax.googleapis.com/ajax/libs/jquery/3.5.1/jquery.min.js"></script>
	<script>
		$(document).mousemove(function(event) {
			$("#btn1").text(event.pageX + ", " + event.pageY);
		});
	</script>
</head>
<body>
	<button id="btn1">Show Text</button>
</body>
</html>
\end{lstlisting}
