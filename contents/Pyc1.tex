% Python 环境搭建
% keys Python
% license Usr
% type Tutor

\begin{itemize}
\item \autoref{sub_Pyc1_1} Windows系统
\item \autoref{sub_Pyc1_2} MAC系统
\item \autoref{sub_Pyc1_3} Unix 和 Linuxxt
\end{itemize}

\subsection{Windows}\label{sub_Pyc1_1} 

\subsubsection{1.1 安装 Python:}
访问 \href{https://www.python.org/downloads/windows/}{Python 官网关于 Windows 下载},一般就下载  Windows installer,其中“Stable Releases”是指稳定版本,推荐下载。
\begin{figure}[ht]
\centering
\includegraphics[width=14.25cm]{./figures/a331410dd57d19b6.png}
\caption{Python 官网关于 Windows} \label{fig_Python_1}
\end{figure}
% 安装以后在开始菜单中搜索程序名如 \verb`Python 3.8 (64-bit)`, 点击后即可打开 Python 命令行。 安装包也会自动安装 \verb`pip3`, 可以在 Powershell 或者 cmd 中输入 \verb`pip3 --version` 查看。
% \addTODO{pip3 是什么?没介绍}

\subsubsection{1.2 库(包)的简介与Pip3}

\begin{itemize}
\item 库(包)是Python的扩展,是Python基本功能的延拓(具体会在后文解释)
\item Pip3 是 Python3 包管理工具,该工具提供了对Python 包的查找、下载、安装、卸载的功能。\footnote{注意:Python 3.4+ 以上版本都自带 pip3 工具}
\end{itemize}

你可以通过以下命令来判断是否已安装:

按住 Win + R 键输入 cmd 回车,进入 命令行 界面。

% pip --version     # Python2.x 版本命令

\begin{lstlisting}[language=bash]
pip3 --version    # Python3.x 版本命令
\end{lstlisting}

如果你还未安装,则可以使用以下方法来安装:

\begin{lstlisting}[language=bash]
$ sudo python3 get-pip.py    # 运行安装脚本。
\end{lstlisting}

\subsubsection{1.3 环境变量配置}

按住 Win + R 键输入 cmd 回车,进入 命令行 界面。(注:一般安装包会自动配置)

\begin{lstlisting}[language=bash]
path=%path%;C:\Python 
\end{lstlisting}

\subsubsection{1.4 Python IDE 安装 —— PyCharm}

IDE 简要来说就是让你编程更方便快捷轻松的软件

优点:安装插件简单方便、插件相对丰富

缺点:占用内存相对大

访问 \href{http://www.jetbrains.com/pycharm/download/}{PyCharm 官方下载地址}.其中你可以选择免费的社区版(Community),但是如果你需要用于科学和Web Python开发的话,建议你下一个专业版(Professional)。

\begin{figure}[ht]
\centering
\includegraphics[width=14.25cm]{./figures/22079997aae6351e.png}
\caption{PyCharm 官网} \label{fig_PyIDE_1}
\end{figure}

下载完后,PyCharm默认是英文,你可以在插件里面下载汉化包。(最好下一个)【File—Settings—Plugins—Chinese(Simplified)……】

\begin{figure}[ht]
\centering
\includegraphics[width=14.25cm]{./figures/112483a36b9e277a.png}
\caption{PyCharm 效果图} \label{fig_PyIDE_2}
\end{figure}

下一章 \enref{Python 基本语法}{Pyc2}

\subsection{MAC}\label{sub_Pyc1_2}

\subsubsection{2.1 安装 Python:}
MAC 系统都自带有 Python2.7 环境,你可以在链接 \href{https://www.python.org/downloads/mac-osx/}{Python 官网关于 mac 下载} 上下载最新版安装 Python 3.x。你也可以参考源码安装的方式来安装。

\subsubsection{2.2 库(包)的简介与Pip3}

\begin{itemize}
\item 库(包)是Python的扩展,是Python基本功能的延拓(具体会在后文解释)
\item Pip3 是 Python3 包管理工具,该工具提供了对Python 包的查找、下载、安装、卸载的功能。\footnote{注意:Python 3.4+ 以上版本都自带 pip3 工具}
\end{itemize}

你可以通过以下命令来判断是否已安装:

按住 Win + R 键输入 cmd 回车,进入 命令行 界面。

% pip --version     # Python2.x 版本命令

\begin{lstlisting}[language=bash]
pip3 --version    # Python3.x 版本命令
\end{lstlisting}

如果你还未安装,则可以使用以下方法来安装:

\begin{lstlisting}[language=bash]
$ sudo python3 get-pip.py    # 运行安装脚本。
\end{lstlisting}、

\subsubsection{2.3 Python IDE 安装 —— PyCharm}

IDE 简要来说就是让你编程更方便快捷轻松的软件

优点:安装插件简单方便、插件相对丰富

缺点:占用内存相对大

访问 \href{http://www.jetbrains.com/pycharm/download/}{PyCharm 官方下载地址}.其中你可以选择免费的社区版(Community),但是如果你需要用于科学和Web Python开发的话,建议你下一个专业版(Professional)。


\begin{figure}[ht]
\centering
\includegraphics[width=14.25cm]{./figures/22079997aae6351e.png}
\caption{PyCharm 官网} \label{fig_Pyc1_1}
\end{figure}

下载完后,PyCharm默认是英文,你可以在插件里面下载汉化包。(最好下一个)【File—Settings—Plugins—Chinese(Simplified)……】

\begin{figure}[ht]
\centering
\includegraphics[width=14.25cm]{./figures/112483a36b9e277a.png}
\caption{PyCharm 效果图} \label{fig_Pyc1_2}
\end{figure}

下一章 \enref{Python 基本语法}{Pyc2}

\subsection{Unix 和 Linux}\label{sub_Pyc1_3} 

你可以访问 \href{https://www.python.org/downloads/source/}{Python 官网关于 Linux 下载},选择适用于 Unix/Linux 的源码压缩包。
\begin{figure}[ht]
\centering
\includegraphics[width=14.25cm]{./figures/83c4a69337735cb6.png}
\caption{Python 官网关于 Linux} \label{fig_Python_2}
\end{figure}

也可以直接用命令行下载

按住 Win + R 键输入 cmd 回车,进入 命令行 界面。

\begin{lstlisting}[language=bash]
wget https://www.python.org/ftp/python/3.7.6/Python-3.7.6.tgz
\end{lstlisting}

创建安装目录(你想放哪就放哪)
\begin{lstlisting}[language=bash]
mkdir -p /usr/local/python3
\end{lstlisting}

解压
\begin{lstlisting}[language=bash]
tar -zxvf Python-3.7.6.tgz
\end{lstlisting}

编译安装
\begin{lstlisting}[language=bash]
# gcc 环境、zlib 库和 readline-devel 包
yum -y install gcc
yum -y install zlib*
yum install readline-devel
# 配置
cd Python-3.7.6
./configure --prefix=/usr/local/python3
# 编译安装
make && make install
\end{lstlisting}

建立软链接
\begin{lstlisting}[language=bash]
ln -s /usr/local/python3/bin/python3.7 /usr/bin/python3
ln -s /usr/local/python3/bin/pip3.7 /usr/bin/pip3
\end{lstlisting}

测试安装
\begin{lstlisting}[language=bash]
# 返回 Python 3.7.6(版本)
python3 --version
# 命令行输出
python3
......
print("你好")
\end{lstlisting}

\subsubsection{3.2 库(包)的简介与Pip3}

\begin{itemize}
\item 库(包)是Python的扩展,是Python基本功能的延拓(具体会在后文解释)
\item Pip3 是 Python3 包管理工具,该工具提供了对Python 包的查找、下载、安装、卸载的功能。\footnote{注意:Python 3.4+ 以上版本都自带 pip3 工具}
\end{itemize}

你可以通过以下命令来判断是否已安装:

按住 Win + R 键输入 cmd 回车,进入 命令行 界面。

% pip --version     # Python2.x 版本命令

\begin{lstlisting}[language=bash]
pip3 --version    # Python3.x 版本命令
\end{lstlisting}

如果你还未安装,则可以使用以下方法来安装:

\begin{lstlisting}[language=bash]
$ sudo python3 get-pip.py    # 运行安装脚本。
\end{lstlisting}

\subsubsection{3.3 环境变量配置}

在 csh shell: 输入

\begin{lstlisting}[language=bash]
setenv PATH "$PATH:/usr/local/bin/python"
\end{lstlisting}

回车。在 bash shell (Linux) 输入 :

\begin{lstlisting}[language=bash]
export PATH="$PATH:/usr/local/bin/python" 
\end{lstlisting}

回车。在 sh 或者 ksh shell 输入:

\begin{lstlisting}[language=bash]
PATH="$PATH:/usr/local/bin/python" 
\end{lstlisting}

按下 Enter。(注意: /usr/local/bin/python 是 Python 的安装目录。)

\subsubsection{3.4 Python IDE 安装 —— PyCharm}

IDE 简要来说就是让你编程更方便快捷轻松的软件

优点:安装插件简单方便、插件相对丰富

缺点:占用内存相对大

访问 \href{http://www.jetbrains.com/pycharm/download/}{PyCharm 官方下载地址}.其中你可以选择免费的社区版(Community),但是如果你需要用于科学和Web Python开发的话,建议你下一个专业版(Professional)。

\begin{figure}[ht]
\centering
\includegraphics[width=14.25cm]{./figures/22079997aae6351e.png}
\caption{PyCharm 官网} \label{fig_Pyc1_3}
\end{figure}

下载完后,PyCharm默认是英文,你可以在插件里面下载汉化包。【File—Settings—Plugins—Chinese(Simplified)……】(最好下一个)
\begin{figure}[ht]
\centering
\includegraphics[width=14.25cm]{./figures/112483a36b9e277a.png}
\caption{PyCharm 效果图} \label{fig_Pyc1_4}
\end{figure}

下一章 \enref{Python 基本语法}{Pyc2}
