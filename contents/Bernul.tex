% 伯努利方程
% 流体|密度|压强|功率|横截面

% \begin{issues}
% \issueDraft
% \end{issues}

伯努利方程是关于不可压缩的流体的方程。\footnote{参考 Wikipedia \href{https://en.wikipedia.org/wiki/Bernoulli-principle}{相关页面}。}

假设液体不可被压缩、没有粘滞性、与管壁也没有摩擦阻力,那么处处满足伯努利方程
\begin{equation}
\frac{v^2}{2} + gz + \frac{p}{\rho} = \text{常数}~.
\end{equation}
其中 $g$ 是重力加速度, $z$ 是高度, $p$ 是液体的压强, $\rho$ 是液体的密度

可以根据伯努利原理设计液体测速计等设备。

\subsubsection{推导}

\begin{figure}[ht]
\centering
\includegraphics[width=12cm]{./figures/35e282fa716a725a.png}
\caption{伯努利方程的推导 (参考\href{https://en.wikipedia.org/wiki/Bernoulli-principle}{相关页面})}\label{fig_Bernul_1}
\end{figure}
\footnote{推导过程还参考了安宇教授等的《大学物理》课程}如图, 一根管子的粗细不同两部分的横截面面积分别为 $A_1, A_2$, 压强分别为 $p_1, p_2$, 高度分别为$h_1, h_2$;其中流过的液体密度为$\rho$, 管内流速分布不随时间变化, 在  1,2 两处的速度分别为 $v_1, v_2$。

考虑 $A_1, A_2$ 之间的这段液体, 假设一段 $\Delta t$ 时间内, 起左端和右端分别移动了 $s_1, s_2$。
根据不可压缩的假设,流入水管的水量等于流出水管的水量,$A_1v_1\Delta t=A_2v_2\Delta t=V$,即$m1=m2$。 这个过程中这段液体的机械能改变了多少呢? 机械能包括动能和重力势能。 由于中间深蓝色的部分的机械能保持不变, 所以可以等效视为 1 处的一小截液体移动到了 2 处。

浅蓝色的两段液体的机械能为
\begin{equation}
E_1=\frac{1}{2}mv_1^2+mgh_1, \qquad
E_2=\frac{1}{2}mv_2^2+mgh_2~,
\end{equation}
所以 $A_1,A_2$ 间的液体在时间 $\Delta t$ 内机械能增量为 $E_2 - E_1$。

再考虑液体压力的做功。$A_1,A_2$ 之间的液体向右移动时, $A_1$ 处的压强对其做正功, $A_2$ 处的压强对其做负功。
\begin{equation}
W_1=p_1v_1A_1\Delta t=p_1V, \qquad
W_2=-p_2v_2A_2 \Delta t=-p_2V ~,
\end{equation}
根据机械能定理 $W_1 + W_2 = E_2 - E_1$, 代入得
\begin{equation}
p_1V+\frac{1}{2}mv_1^2+mgh_1=p_2V+\frac{1}{2}mv_2^2+mgh_2~.
\end{equation}
事实上1,2的位置可以任意选取,因此任意位置都有
\begin{equation}
pV+\frac{1}{2}mv^2+mgh=\text{常数}~,
\end{equation}
两边同除以 $V\rho$ 有
\begin{equation}
\frac{p}{\rho} + \frac{v^2}{2} + gh = \text{常数}~.
\end{equation}

\addTODO{这样的推导如何拓展到开放空间的情况呢? 举例: 水龙头下的乒乓球, 香蕉球, 机翼, 两张纸中间吹气}
