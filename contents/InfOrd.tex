% 无穷小、无穷大和阶数(极简微积分)
% license Xiao
% type Tutor

\begin{issues}
\issueTODO
\end{issues}

\pentry{\enref{函数的极限(极简微积分)}{FunLim}}{nod_4ff1}

\subsection{无穷小、无穷大}
首先,我们要严格区分一个给定的实数和无穷大或无穷小之间的区别。 一个给定的实数是指一个具体的值,例如 $3.2\e{-100}$ 或者 $1.6\e{1000}$。 无论一个确定的实数多么大或多么小,都不能说它是无穷。

无穷小是一个过程,具体来在求函数 $f(x)$ 的极限时,如果
\begin{equation}\label{eq_InfOrd_1}
\lim_{x\to \square} f(x) = 0~,
\end{equation}
$f(x)$ 就可以叫做无穷小。 这里的 $\square$ 代表一个实数或者 $\pm\infty$。如果我们说一个符号或者函数是无穷小,那么就暗含者我们在讨论某个\autoref{eq_InfOrd_1} 这样的极限。

反之,如果(定义见\autoref{eq_FunLim_1})
\begin{equation}
\lim_{x\to \square} f(x) = \pm\infty~,
\end{equation}
就说 $f(x)$ 是无穷大。

\begin{exercise}{}
说明以下极限中的表达式是无穷小(即极限等于 0)
\begin{itemize}
\item $\lim\limits_{x\to 0} x$
\item $\lim\limits_{x\to 0} x^n$ 其中 $n$ 是一个正整数
\item $\lim\limits_{x\to 0} \sqrt{x}$
\item $\lim\limits_{x\to 0} \sin x$
\item $\lim\limits_{x\to \infty} 1/x$
\end{itemize}
如果把以上的 $\lim\limits_{x\to 0}$ 和 $\lim\limits_{x\to \infty}$ 互换,哪些会变为无穷大?
\end{exercise}

\subsection{高阶无穷小、高阶无穷大}
在某个极限 $\lim\limits_{x\to \square}$ 的语境下,如果 $f(x)$ 是无穷小,我们可以定义它的阶数。

\addTODO{复制过来}
