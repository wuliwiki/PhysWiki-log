% 线性空间的同态与同构
% license Xiao
% type Tutor


与群运算的同态映射类似,线性空间的同态映射能保证运算结构不变。
\begin{definition}{}
给定域$\mathbb F$上的两个线性空间$V_1,V_2$。如果存在一个映射$f:V_1\rightarrow V_2$,使得对于任意$a,b\in \mathbb F$及任意$\boldsymbol{x_1,x_2}\in V_1,V_2$,都有
\begin{equation}
f(a\boldsymbol {x_1}+b\boldsymbol {x_2})=af(\boldsymbol {x_1})+bf(\boldsymbol {x_2})~,
\end{equation}
则称$f$是$V_1$到$V_2$的一个同态(homomorphism)。
\end{definition}
如果该同态还是一个双射,则称之为同构(isomorphism)。此时记$V_1\cong V_2$
从上述定义我们可以知道,同构是特殊的同态。同态映射意味着“先线性运算再映射”和“先映射再线性运算”的结果是相同的。
线性空间的同态映射又称线性映射(linear mapping),线性指的是“加性”($f(x+y)=f(x)+f(y)$)和“齐性”($f(ax)=af(x),a\in \mathbb F$)。显然,线性函数就是线性映射的一种。

那么怎么判断两个线性空间是否同构呢?如果已经存在同态映射,那么这个问题实际上是在进一步问:如何建立线性空间的双射?线性空间的良好性质使得下述定理成立,进而让我们能够迅速判断同构是否存在。
\begin{theorem}{}
给定域$\mathbb F$上的两个线性空间$V_1,V_2$,它们同构当且仅当维度相同
\end{theorem}
Proof。
证明思路是建立basis之间的双映射。设$\{\boldsymbol{e_i}\},\{\boldsymbol{\theta_i}\}$分别是$V_1,V_2$的一组基。那么我们有
\begin{equation}
f(a^i\boldsymbol e_i)=f(a^1\boldsymbol e_1+a^2\boldsymbol e_2+...)=a^if(e_i)=a^i\theta_i~,
\end{equation}
显然,在该映射下,$V_1$中的任意向量都能被唯一表示为$V_2$中的向量,单射成立,反之亦然,所以这映射是满的。因此,这是一个双射。以上是有限维情况的证明,但也可以适用于无限维,只要保证基于基的对应即可。证毕。