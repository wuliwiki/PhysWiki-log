% 传播子(量子力学)
% keys 路径积分|狄拉克符号|薛定谔方程|时间演化算符
% license Xiao
% type Tutor

\pentry{时间演化算符(量子力学)\upref{TOprt}}

本节采用自然单位制,$\hbar=1$。


\subsection{传播子的概念}


考虑一个$t=0$时刻的初始量子态,记为$\ket{s}$。设有哈密顿算符$H$,则时间演化算符是$\E^{-\I Ht}$。设$H$的本征态为$\ket{a}$,对应本征值为$E_a$。

用$H$\textbf{的本征态}来展开$\ket{s}$,得到$\ket{s}=\sum_a\ket{a}\braket{a}{s}$,则有
\begin{equation}\label{eq_PpgtQM_1}
\bra{\bvec{x}}\E^{-\I Ht}\ket{s} = \sum_a\bra{\bvec{x}}\E^{-\I Ht}\ket{a}\braket{a}{s} = \sum_a\braket{\bvec{x}}{a}\braket{a}{s}\E^{-\I E_at}~.
\end{equation}

用\textbf{位置本征态}$\ket{\bvec{x}}$来展开$\ket{s}$,得到$\ket{s}=\int\mathrm{d}^3 x\ket{\bvec{x}}\braket{\bvec{x}}{s}$,则有
\begin{equation}\label{eq_PpgtQM_2}
\braket{a}{s} = \int\mathrm{d}^3x\braket{a}{\bvec{x}}\braket{\bvec{x}}{s}~.
\end{equation}

将\autoref{eq_PpgtQM_2} 代入\autoref{eq_PpgtQM_1} ,设$\ket{\bvec{y}}$是位置本征态,$\psi(\bvec{y}, t)$是$\ket{s}$在时刻$t$的波函数,则有
\begin{equation}\label{eq_PpgtQM_3}
\ali{
    \psi(\bvec{y}, t)&=\bra{\bvec{y}}\E^{-\I Ht}\ket{s}\\
    &=\sum_a\braket{\bvec{y}}{a}\braket{a}{s}\E^{-\I E_at}\\
    &=\sum_a\braket{\bvec{y}}{a}\E^{-\I E_at}\int\mathrm{d}^3x\braket{a}{\bvec{x}}\braket{\bvec{x}}{s}\\
    &=\int\mathrm{d}^3x\sum_a\braket{\bvec{y}}{a}\E^{-\I E_at}\braket{a}{\bvec{x}}\braket{\bvec{x}}{s}~,
}
\end{equation}

则定义$K(\bvec{y}, \bvec{x}, t)$为$\sum_a\braket{\bvec{y}}{a}\E^{-\I E_at}\braket{a}{\bvec{x}}$。根据\autoref{eq_PpgtQM_3} ,这意味着
\begin{equation}
\psi(\bvec{y}, t) = \int\mathrm{d}^3x K(\bvec{y}, \bvec{x}, t)\psi(\bvec{x}, 0)~,
\end{equation}
称$K$为\textbf{传播子(propagator)}。

注意,传播子也可以写成
\begin{equation}
\ali{
K(\bvec{y}, \bvec{x}, t) &= \bra{\bvec{y}}\qty(\sum_a\ket{a}\E^{-\I E_at}\bra{a})\ket{\bvec{x}} 
\\&= \bra{\bvec{y}}\qty(\E^{-\I Ht}\sum_a\ket{a}\bra{a})\ket{\bvec{x}}\\
&= \bra{\bvec{y}}\E^{-\I Ht}\ket{\bvec{x}}~.
}
\end{equation}

实际计算中,$\E^{-\I Ht}\ket{\bvec{x}}$可能难以算出,因此也常插入$H$的本征态构造的恒等算符$\sum_a\ket{a}\bra{a}$或$\int\dd a\ket{a}\bra{a}$来方便计算。


\subsection{传播子的性质}




对传播子$K$关于时间求偏导,得到:
\begin{equation}\label{eq_PpgtQM_5}
\ali{
    \I\partial_t K &= \bra{\bvec{y}}\qty(\I\partial_t\E^{-\I Ht})\ket{\bvec{x}}\\
    &= \bra{\bvec{y}}\qty(H\E^{-\I Ht})\ket{\bvec{x}}\\
    &= HK~,
}
\end{equation}
即$K$\textbf{满足薛定谔方程}。

另外,注意到
\begin{equation}
\ali{
    K(\bvec{y}, \bvec{x}, 0) &= \braket{\bvec{y}}{\bvec{x}} = \delta^3(\bvec{y}-\bvec{x})~,
}
\end{equation}
即$K(\bvec{y}, \bvec{x}, 0)$可以看成是$t=0$时位置本征态$\ket{\bvec{x}}$的波函数$\psi_{\bvec{x}}(\bvec{y}, 0)$。

由于量子态的演化遵循薛定谔方程,因此综上所述,$K(\bvec{y}, \bvec{x}, t)$可以视为一个自然演化的波函数$\psi_{\bvec{x}}(\bvec{y}, t)$,其初态为$\ket{\bvec{x}}$。



\begin{example}{一维自由粒子}\label{ex_PpgtQM_1}

考虑一个\textbf{一维自由}粒子,显然$[H, p]=0$。具体地,$p=-\I\hbar\partial_x$,$H=\frac{p^2}{2m}=\frac{\hbar^2}{2m}\partial_x^2$。

设$\ket{p_a}$是$p$与$H$的共同本征态,其中下标$a$取值范围为整个实数集,且$p\ket{p_a}=a$,这还意味着$H\ket{p_a}=a^2/2m$。

再注意到$\ket{p_a}$的位置表象波函数为
\addTODO{引用关于$\ket{p_a}$的波函数归一化讨论的词条。}
\begin{equation}
\psi_a(x) = \braket{x}{p_a} = \frac{1}{\sqrt{2\pi}}\exp(\I ax)~.
\end{equation}


于是能计算出传播子
\begin{equation}\label{eq_PpgtQM_4}
\ali{
    K(y, x, t) &= \bra{y}\qty(\E^{-\I Ht}\int\ket{p_a}\bra{p_a}\dd a)\ket{x}\\
    &= \bra{y}\qty(\int\E^{-\I a^2t/2m}\ket{p_a}\bra{p_a}\dd a)\ket{x}\\
    &= \int\E^{-\I a^2t/2m}\braket{y}{p_a}\braket{p_a}{x}\dd a\\
    &= \frac{1}{2\pi}\int\E^{-\I a^2t/2m}\E^{\I a(y-x)}\dd a~,
}
\end{equation}

以下是\autoref{eq_PpgtQM_4} 积分的计算。

令$A=t/2m$,$B=x-y$,则
\begin{equation}\label{eq_PpgtQM_6}
\ali{
    \int\E^{-\I a^2t/2m}\E^{\I a(y-x)} \dd a &= \int \E^{-\I\qty(Aa^2+Ba)}\dd a\\
    &= \E^{\I\frac{B^2}{4A}}\int \E^{-\I\qty(\sqrt{A}a+\frac{B}{2\sqrt{A}})^2}\dd a\\
    &= \E^{\I\frac{B^2}{4A}}\cdot\frac{1}{\sqrt{A}}\int \E^{-\I b^2}\dd b~.
}
\end{equation}
根据\textbf{高斯积分}\upref{GsInt}\autoref{eq_GsInt_5}~\upref{GsInt} ,或者如果你对高斯积分向复数的拓展有疑虑的话,根据\textbf{留数定理}\upref{ResThe}\autoref{ex_ResThe_3}~\upref{ResThe},知$\int \E^{-\I b^2}\dd b = \sqrt{\frac{\pi}{\I}} = \sqrt{\frac{\pi}{2}}(1-\I)$。代入\autoref{eq_PpgtQM_6} 得
\begin{equation}\label{eq_PpgtQM_7}
\ali{
    \int\E^{-\I a^2t/2m}\E^{\I a(y-x)} \dd a &= \frac{1}{2\pi}\cdot\E^{\I\frac{B^2}{4A}}\cdot\frac{1}{\sqrt{A}}\cdot \sqrt{\frac{\pi}{2}}(1-\I)\\
    &= \frac{1}{2\pi}\exp\qty(\frac{m\I\qty(x-y)^2}{2t})\sqrt{\frac{\pi m}{t}}(1-\I)\\
    &= \exp\qty(\frac{m\I\qty(x-y)^2}{2t})\sqrt{\frac{m}{4\pi t}}(1-\I)~.
}
\end{equation}
\autoref{eq_PpgtQM_7} 也可以写为
\begin{equation}\label{eq_PpgtQM_9}
\int\E^{-\I a^2t/2m}\E^{\I a(y-x)} \dd a = \exp\qty(\frac{\I m\qty(x-y)^2}{2t})\sqrt{\frac{m}{2\pi t\I}}~.
\end{equation}

\end{example}



\begin{exercise}{三维自由粒子}
你已经学会了计算一维自由粒子的传播子(\autoref{ex_PpgtQM_1} ),现在请仿照算出三维自由粒子的传播子。

答案是
\begin{equation}\label{eq_PpgtQM_10}
\bra{\bvec{y}}\E^{-\I H t}\ket{\bvec{x}} = \qty(\frac{m}{2\pi t\I})^{3/2}\exp\qty(\frac{\I m(\bvec{x}-\bvec{y})^2}{2t})~.
\end{equation}
\end{exercise}





%需要加简谐振子的例子吗?这个例子计算太过复杂了。




\subsection{传播振幅}


给定测量算符$X$,$X$的本征值为$x_a$的本征矢记为$\ket{a}$。

给定量子态$\ket{s}$,一段时间$t$后它会变成$\E^{-\I Ht}\ket{s}$。此时若我们对它进行$X$测量,则得到$x_a$的概率为
\begin{equation}\label{eq_PpgtQM_8}
\abs{\bra{a}\E^{-\I Ht}\ket{s}}^2~.
\end{equation}
这可以理解为,时间$t$后,$\ket{s}$演化为$\ket{a}$的概率。

据以上结论,我们定义\textbf{传播振幅},用于描述一个系统一段时间后演化为另一个系统的概率:

\begin{definition}{传播振幅}\label{def_PpgtQM_1}
给定量子态$\ket{s}$和$\ket{a}$,定义从$\ket{s}$,经过时间$t$后到$\ket{a}$的\textbf{传播振幅}为$\bra{a}\E^{-\I Ht}\ket{s}$。

传播振幅的模方,\autoref{eq_PpgtQM_8} ,即为传播的概率。
\end{definition}

由\autoref{def_PpgtQM_1} 可见,传播子可以理解为一种跃迁振幅,即初始处于$\bvec{x}$位置的粒子,在时间$t$后出现在$\bvec{y}$处的概率振幅。




\subsection{经典量子力学的局限性}

根据传播振幅的概念,\autoref{eq_PpgtQM_9} 和\autoref{eq_PpgtQM_10} 意味着对于任意给定的位置$\bvec{y}$和$\bvec{x}$,自由粒子总能在任意短的时间$t$内从一处传播到另一处(传播振幅的模方不为零),意即粒子允许瞬间移动。这是不符合相对论要求的因果性的。

直接将哈密顿量$H$改为相对论形式$H=\sqrt{\bvec{p}^2+m^2}$也无济于事(\textsl{An Introduction to Quantum Field Theory}\cite{Peskin},2.1节)。

对于该问题的讨论,请参见
\addTODO{写了相关词条后引用}














