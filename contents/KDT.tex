% KD-Tree
% keys 数据结构|计算机
% license Usr
% type Wiki

KD-Tree (K-Dimension Tree)是一种空间划分数据结构,顾名思义,可以对 $k$ 维空间进行划分。它常被用于高维空间中的搜索,比如范围搜索和最近邻搜索,时间复杂度由 $k$ 和总节点数 $n$ 共同保障。一般当 $n$ 远大于 $2^k$ 时,应用 KD-Tree 的效果是最好的。

\begin{definition}{二叉搜索树}
又称 BST(Binary Search Tree),是一种特殊的二叉树,对于二叉树上每个节点以及其上权值,满足:
\begin{enumerate}
\item 若二叉搜索树的左子树不为空,则其左子树上所有点的权值均小于其根节点的值。
\item 若二叉搜索树的右子树不为空,则其右子树上所有点的权值均大于其根节点的值。
\item 二叉搜索树的左右子树均为二叉搜索树。
\end{enumerate}
特别的,定义空树也为二叉搜索树。
\end{definition}

KD-Tree 具有\textbf{二叉搜索树}的形态,二叉搜索树上的每个节点都对应 $k$ 维空间内的一个点。其每个子树中的点都在一个 $k$ 维的超长方体内,这个超长方体内的所有点也都在这个子树中。

由主定理可以分析得到,KD-Tree 的复杂度是 $\mathcal O\left(n^{1-\frac 1k}\right)$ 的。

\subsection{建树}

已经知道了这 $n$ 个 $k$ 维空间中的点的坐标,如何建立一颗对应的 KD-Tree?有如下方法:

\begin{itemize}
\item 考虑当前要分割的点集,这些点都在一个 $k$ 维超长方体内。若这超长方体内有且仅有一个点,返回这个点。
\item 若这个超长方体内有多于 $1$ 个点,考虑对这个超长方体进行分割:
\item 1. 选择 $k$ 个维度中的一个。
    \item 2. 选择一个分割点,这选择的这一维度上的值小于这个点的归入一个超长方体(左子树),其余的归入另一个超长方体(右子树)。
    \item 3. 将选择的点作为这棵子树的根节点,递归建立左右子树并在过程中维护出需要的信息。
\end{itemize}

显然如果选择维度是随机的、选择的点也是随机的,这数据结构的复杂度将没有保障。

对于步骤 $1$,我们经常需要讨论各个维度的临近情况,所以希望能在 KD-Tree 每相邻的 $k$ 个深度里都分别有这 $k$ 个维度。故我们考虑这选取维度的方法是轮流选取 $k$ 个维度。另外有一种“随机化”的方法是先将 $1$ 至 $k$ 连续赋值给一个数组,后将这个数组随机打乱,然后轮流取这个数组里的每个元素对应的维度。

对于步骤 $2$,考虑“二分”,每次取的这个分割点位于中位是最佳的。不难想到使用这优化后将限制 KD-Tree 的深度在 $\mathcal O\left(\log n\right)$ 量级,更准确地说其实是在 $\log n + \mathcal O(1)$ 量级。

\subsection{高维操作}

在维护高维矩形区域内的所有点的某些信息过程中,记录下每个节点子树内\textbf{每一维度}上的坐标的最大值和最小值。

\begin{itemize}
\item 如果当前子树对应的矩形与所求矩形完全相离,也就是没有交点,则直接返回,不继续搜索这棵子树的子树;

\item 如果当前子树对应的矩形完全包含在所求矩形内,返回当前子树内所有点的权值和;否则,判断当前点是否在所求矩形内,更新答案并递归在左右子树中查找答案(这过程类似于\enref{线段树}{STree})。
\end{itemize}

\subsection{插入操作}

类似于二叉搜索树的,对于要被插入的节点,根据各个维度上的划分确定它在左子树还是在右子树,之后递归插入即可。

\subsection{删除操作}
如果要删除的节点是叶子节点,直接递归访问到然后直接删除,再在回溯过程中更新这访问的整条链的信息就可以。

首先递归找到这节点,然后考虑这个节点的左右子树,可以在左子树上取这一维度上最大的点代替这点,然后递归重建左子树。如果没有左子树,就考虑右子树的这维度上最小的点即可。

处理完以这节点为根节点的子树后,从这节点回溯的过程中更新整棵树的信息。


\subsection{最邻近点}
注意:KD-Tree 解决该问题的复杂度最坏是 $\mathcal O(n)$ 的,只不过由于剪枝而快速很多。

具体剪枝的方法是:
维护一个子树中的所有节点在每一维上的坐标的取值范围。假设当前已经找到的最近点对的距离是 $\mathscr x_0$,如果查询的点到子树内所有点都包含在内的超长方体的\textbf{最近}距离大于等于 $\mathscr x_0$,则在这个子树内一定没有答案,搜索时不进入这个子树。

另外也可以考虑启发式搜索,估价函数可以取为查询点到子树对应的超长方体的距离。对于当前点对的距离,若估价函数大于等于当前答案,就无需访问此子树。对于两颗子树都小于答案的情况,优先访问距离小的子树。

