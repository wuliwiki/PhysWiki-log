% 约翰·蓝道尔(综述)
% license CCBYSA3
% type Wiki

本文根据 CC-BY-SA 协议转载翻译自维基百科 \href{https://en.wikipedia.org/wiki/John_Randall_(physicist)}{相关文章}。

\begin{figure}[ht]
\centering
\includegraphics[width=6cm]{./figures/fc0ea7384a703f1d.png}
\caption{} \label{fig_YHldr_1}
\end{figure}
约翰·特顿·兰德尔爵士(Sir John Turton Randall,FRS FRSE,\(^\text{[2]}\)1905年3月23日-1984年6月16日)是一位英国物理学家和生物物理学家,以对腔体磁控管的重大改进而闻名。腔体磁控管是厘米波雷达的重要组成部分,也是盟军在第二次世界大战中取得胜利的关键技术之一,同时也是微波炉的核心部件。\(^\text{[3][4]}\)

兰德尔与哈里·布特合作,制造出一种能够以10厘米波长发射微波脉冲的电子管。\(^\text{[3]}\)对于他们发明的重要性,不列颠哥伦比亚大学维多利亚分校军事史教授大卫·齐默曼指出:“磁控管依然是所有类型短波无线电信号的关键电子管。它不仅通过使我们能够开发机载雷达系统改变了战争进程,它仍然是今天微波炉核心的关键技术。腔体磁控管的发明改变了世界。”\(^\text{[3]}\)

兰德尔还曾领导伦敦国王学院的团队研究DNA的结构。他的副手莫里斯·威尔金斯与剑桥大学卡文迪许实验室的詹姆斯·沃森和弗朗西斯·克里克共同因DNA结构的解析获得了1962年诺贝尔生理学或医学奖。他的其他团队成员还包括罗莎琳德·富兰克林、雷蒙德·戈斯林、亚历克斯·斯托克斯和赫伯特·威尔逊,他们都参与了DNA结构研究。
\subsection{教育与早年生活}
约翰·兰德尔于1905年3月23日出生在兰开夏郡纽顿利威洛斯,是悉尼·兰德尔(Sidney Randall,苗圃及种子商人)与其妻汉娜·考利(Hannah Cawley,约翰·特顿[John Turton]之女,后者是当地煤矿经理)唯一的儿子,也是三名孩子中的长子。\(^\text{[2]}\)他在阿什顿因梅克菲尔德(文法学校和曼彻斯特维多利亚大学接受教育,1925年获得物理学一级荣誉学位和毕业奖学金,1926年获得理学硕士学位。\(^\text{[2]}\)

1928年,他与多丽丝·达克沃斯结婚。
\subsection{职业与研究}
1926年至1937年间,兰德尔在通用电气公司位于温布利的实验室从事研究工作,他在为放电灯研发发光粉方面发挥了重要作用。 他还积极研究这些发光机制。\(^\text{[2]}\)

到1937年,他已被公认为英国该领域的领先研究者,并获得了伯明翰大学英国皇家学会奖学金,在马克·奥利芬特领导的物理系与莫里斯·威尔金斯合作,研究磷光的电子陷阱理论。\(^\text{[5][6][7][8]}\)
\subsubsection{磁控管}
\begin{figure}[ht]
\centering
\includegraphics[width=6cm]{./figures/6040822fe73927cf.png}
\caption{伯明翰大学彭廷物理楼} \label{fig_YHldr_2}
\end{figure}
当战争在1939年爆发时,英国海军部找到奥利芬特,探讨是否有可能建造一种能在微波频率下工作的无线电发射源。这类系统能够使雷达探测到小型物体,例如潜艇的潜望镜。当时在萨福克海岸博兹西庄园的空军部雷达研究人员也表达了对10厘米波段系统的兴趣,因为这将大幅减小发射天线的尺寸,使其能够更轻松地安装在飞机机头,而不像现有系统那样需要安装在机翼和机身上。\(^\text{[9]}\)
\begin{figure}[ht]
\centering
\includegraphics[width=6cm]{./figures/b978444bc8af013b.png}
\caption{原始的六腔磁控管} \label{fig_YHldr_3}
\end{figure}
奥利芬特开始使用速调管开展研究,这是一种由拉塞尔和西格德·瓦里安在1937年至1939年间发明的设备,也是当时唯一已知能够高效产生微波的系统。然而,当时的速调管功率非常低,奥利芬特的工作重点是大幅提高其输出功率。如果这项工作成功,就会产生一个次要问题:速调管仅是放大器,因此需要一个低功率的信号源供其放大。奥利芬特指派兰德尔和哈里·布特解决微波振荡器的研制问题,让他们探索使用微型巴克豪森–库尔茨管作为信号源,这种设计已用于超高频(UHF)系统。他们的研究很快表明,这种管在微波频率范围内无法带来改进。\(^\text{[10]}\)速调管研究很快遇到瓶颈,研制出的设备只能产生约400瓦的微波功率,这仅够用于测试,但远远达不到用于实际雷达系统所需的数千瓦功率水平。

1939年11月,由于没有其他项目需要负责,兰德尔和布特开始考虑这一问题的解决方案。当时已知的另一种微波设备是分阳极磁控管,这种设备能够产生少量功率,但效率低下,且输出通常比速调管更低。然而,他们注意到这种设备相较于速调管有一个巨大优势:速调管的信号依赖于电子枪产生的电子束流,而电子枪的电流能力决定了设备最终能处理的功率上限。相比之下,磁控管使用的是普通的热阴极灯丝阴极,这种系统已被广泛用于产生数百千瓦功率的无线电系统中。这似乎为实现更高功率提供了更可行的途径。\(^\text{[10]}\)

现有磁控管的问题不在于功率,而在于效率。在速调管中,电子束会穿过一个金属盘状谐振腔,谐振腔的机械结构会影响电子,使其加速和减速,从而释放出微波。这种方法效率较高,功率受限于电子枪。而在磁控管中,谐振腔被两块带相反电荷的金属板取代,通过磁场使电子在板间运动以实现交替加速。这种方式在理论上对可加速电子数量没有限制,但微波释放过程的效率极低。

随后,两人开始考虑:如果将磁控管的两块金属板替换为谐振腔,会发生什么情况?这实际上是将现有的磁控管和速调管概念结合起来。与传统磁控管一样,磁场会使电子沿圆形轨道运动,使它们依次经过各个谐振腔,从而比使用金属板时更高效地产生微波。他们回忆起海因里希·赫兹(Heinrich Hertz)曾使用金属线圈作为谐振腔(不同于速调管中的盘状谐振腔),这使他们想到可以在磁控管中心周围放置多个谐振腔。更重要的是,这些线圈的数量和尺寸在理论上没有真正限制。通过将线圈延伸成圆筒,可以显著提高系统功率,而功率承载能力则取决于电子管的长度。增加谐振腔数量还可以提高效率,因为每个电子在运动轨道中能与更多谐振腔发生作用。唯一的实际限制来自所需的频率和电子管的物理尺寸。\(^\text{[10]}\)

使用常见实验室设备研发的第一个磁控管由一个铜块制成,铜块上钻有六个孔以形成谐振回路,然后将其放入钟罩中抽真空,再将其置于他们能找到的最大马蹄形磁铁的磁极之间。1940年2月,首次测试新型腔体磁控管时产生了400瓦功率,且在一周内功率提升至超过1000瓦。\(^\text{[10]}\)随后,这一设计展示给了GEC的工程师,并请他们协助进一步改进。GEC引入了多种工业新方法来更好地密封电子管和改善真空环境,同时增加了带氧化物涂层的新型阴极,使其能够承载更大的电流。这些改进使功率提升至10千瓦,与当时现有雷达系统中使用的传统电子管系统功率相当。磁控管的成功彻底革新了雷达的发展,自1942年起几乎所有新型雷达设备均开始采用磁控管。

1943年,兰德尔离开奥利芬特位于伯明翰的物理实验室,前往剑桥大学卡文迪许实验室任教一年。1944年,兰德尔被任命为圣安德鲁斯大学自然哲学教授,并开始在一笔海军部小额资助下(与莫里斯·威尔金斯合作)筹划生物物理研究项目。\(^\text{[11]}\)
\subsubsection{伦敦国王学院}
1946年,兰德尔被任命为伦敦国王学院物理系主任。 随后,他转任国王学院威特斯通物理学讲席教授,在那里医学研究委员会(MRC)设立了生物物理研究部门,由兰德尔担任主任(现称为国王学院兰德尔细胞与分子生物物理中心)。 在他担任主任期间,罗莎琳德·富兰克林、雷蒙德·戈斯林、莫里斯·威尔金斯、亚历克斯·斯托克斯和赫伯特·R·威尔逊在此进行的实验工作,最终促成了DNA结构的发现。他将雷蒙德·戈斯林分配给富兰克林作为博士研究生,共同使用X射线衍射研究DNA结构。\(^\text{[12]}\)根据戈斯林回忆,约翰·兰德尔在推动双螺旋结构研究中的作用不容低估。戈斯林对此深有感触,以至于在2013年DNA双螺旋发现60周年纪念期间,他专门致信《泰晤士报》表达这一观点。\(^\text{[13]}\) 兰德尔坚信DNA承载着遗传密码,并组建了跨学科团队以证明这一点。正是兰德尔指出,由于DNA主要由碳、氮和氧组成,与照相机中空气中的原子成分基本相同,因此会导致X射线发生漫反射,导致底片曝光发雾。为此,他指示戈斯林使用氢气置换相机内的空气以避免这一问题。\(^\text{[13]}\)

1962年,莫里斯·威尔金斯与詹姆斯·沃森和弗朗西斯·克里克共同获得诺贝尔生理学或医学奖;而罗莎琳德·富兰克林已于1958年因癌症去世。

除了X射线衍射研究之外,该研究部门还开展了由物理学家、生物化学家和生物学家组成的广泛研究项目。通过使用新型光学显微镜,他们于1954年提出了肌肉收缩的滑动丝机制这一重要假说。兰德尔还成功推动了国王学院生物科学教学的整合。\(^\text{[2]}\)

1951年,他亲自领导组建了一个大型跨学科团队,专门研究结缔组织蛋白胶原的结构和生长。他们的研究有助于阐明胶原分子的三链结构。 兰德尔本人专注于电子显微镜的使用,首先研究精子的精细结构,随后重点转向胶原的研究。\(^\text{[2]}\)1958年,他发表了一项关于原生动物结构的研究。\(^\text{[2]}\)他还组建了一个新研究小组,以原生动物纤毛作为模式系统,通过关联突变体的结构与生化差异来分析形态发生过程。
\subsubsection{个人生活与晚年}
兰德尔于1928年与多丽丝结婚,她是煤矿测量员乔赛亚·约翰·达克沃斯的女儿。\(^\text{[2]}\)他们于1935年育有一子,名为克里斯托弗。\(^\text{[2]}\)

1970年,他搬到爱丁堡大学工作,并在那里组建了一个研究小组,应用多种新的生物物理方法,例如在重水离子溶液中对蛋白质晶体进行相干中子衍射研究,通过中子衍射和散射研究各种生物大分子问题,如蛋白质残基与氘之间的质子交换过程。
\subsubsection{荣誉与奖项}
\begin{figure}[ht]
\centering
\includegraphics[width=6cm]{./figures/a4ecf342c93fc584.png}
\caption{伯明翰大学——彭廷物理楼——蓝色纪念牌} \label{fig_YHldr_4}
\end{figure}
\begin{itemize}
\item 1938年,兰德尔被曼彻斯特维多利亚大学授予理学博士学位。\(^\text{[14]}\)
\item 1943年,他因腔体磁控管的发明(与哈里·布特共同完成)获得英国皇家艺术学会托马斯·格雷纪念奖。
\item 1945年,他因磁控管发明获得伦敦物理学会达德尔奖章与奖金,并与布特分享了英国发明奖励皇家委员会颁发的奖金。
\item 1946年,他当选为英国皇家学会院士(FRS),\(^\text{[2]}\)同年荣获皇家学会休斯奖章。
随后他(与布特共同)因磁控管的研究继续获得荣誉:1958年获宾夕法尼亚州富兰克林研究所约翰·普赖斯·韦瑟里尔奖章,1959年获费城市约翰·斯科特奖章。\(^\text{[2]}\)
\item 1962年,他被授予爵士爵位;1972年被选为爱丁堡皇家学会院士(FRSE)。
\end{itemize}
\subsection{参考文献}
\begin{enumerate}
\item Wilkins, Maurice Hugh Frederick(1940年)。《磷光衰减定律与固体中的电子过程》。ethos.bl.uk(博士论文)。伯明翰大学。OCLC 911161224。
\item Wilkins, M. H. F.(1987年)。“约翰·特顿·兰德尔,1905年3月23日-1984年6月16日”,《英国皇家学会院士传记回忆录》,第33卷:493–535。doi:10.1098/rsbm.1987.0018。JSTOR 769961。PMID 11621437。S2CID 45354172。
\item “改变世界的公文包”,BBC,2017年10月20日。
\item “关键参与者:J. T. Randall – 莱纳斯·鲍林与DNA竞赛:文献历史”,osulibrary.oregonstate.edu。
\item Garlick, G. F. J.; Wilkins, M. H. F.(1945年)。“短周期磷光与电子陷阱”,《皇家学会会刊A:数学、物理与工程科学》,第184卷(999期):408–433。Bibcode:1945RSPSA.184..408G。doi:10.1098/rspa.1945.0026。ISSN 1364-5021。
\item Randall, J. T.; Wilkins, M. H. F.(1945年)。“磷光与电子陷阱 I:陷阱分布研究”,《皇家学会会刊A》,第184卷(999期):365–389。Bibcode:1945RSPSA.184..365R。doi:10.1098/rspa.1945.0024。ISSN 1364-5021。
\item Randall, J. T.; Wilkins, M. H. F.(1945年)。“磷光与电子陷阱 II:长周期磷光的解释”,《皇家学会会刊A》,第184卷(999期):390–407。Bibcode:1945RSPSA.184..390R。doi:10.1098/rspa.1945.0025。ISSN 1364-5021。
\item Randall, J. T.; Wilkins, M. H. F.(1945年)。“多种固体的磷光研究”,《皇家学会会刊A》,第184卷(999期):347–364。Bibcode:1945RSPSA.184..347R。doi:10.1098/rspa.1945.0023。ISSN 1364-5021。
\item Bowen, Edward George(1998年)《雷达岁月》,CRC出版社,第143页,ISBN 978-0-7503-0586-0。
\item Boot, H. A. H.; Randall, J. T.(1976年)。“腔体磁控管的历史注记”,《IEEE电子器件汇刊》,第23卷(第7期):724。Bibcode:1976ITED...23..724B。doi:10.1109/T-ED.1976.18476。
\item “细胞:剖析新解剖学”,伦敦国王学院。2016年3月18日归档,2016年1月19日检索。
\item 见兰德尔致富兰克林的信件。
\item Attar, Naomi(2013年4月25日),“雷蒙德·戈斯林:让基因结晶的人”,《基因组生物学》,第14卷(4期):402。doi:10.1186/gb-2013-14-4-402。PMC 3663117。PMID 23651528。
\item “约翰·特顿·兰德尔”(1938年),曼彻斯特大学理学院档案,系列:D.Sc. 考官报告,1909年至1949年。英国曼彻斯特牛津路:曼彻斯特大学图书馆,曼彻斯特大学。2020年3月1日检索。
\end{enumerate}
\subsection{延伸阅读}
\begin{itemize}
\item Chomet, S.(编),《DNA:发现的起源》,1994年,伦敦,纽曼-半球出版社。
\item 威尔金斯,《双螺旋的第三人:莫里斯·威尔金斯自传》,ISBN 0-19-860665-6。
\item 里德利,《弗朗西斯·克里克:遗传密码的发现者(杰出人生)》,2006年7月首次在美国出版,同年9月在英国由哈珀柯林斯出版社出版,ISBN 0-06-082333-X。
\item 泰特,西尔维娅与詹姆斯,《不太可能的四大发现》,雅典娜出版社,2004年,ISBN 1-84401-343-X。
\item 沃森,《双螺旋:DNA结构发现的亲历记》,雅典娜出版社,1980年,ISBN 0-689-70602-2(初版出版于1968年)。
\end{itemize}
\subsection{外部链接}
\begin{itemize}
\item 丘吉尔档案中心收藏的约翰·兰德尔爵士文献资料
\end{itemize}
