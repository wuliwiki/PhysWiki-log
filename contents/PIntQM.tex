% 路径积分(量子力学)
% keys 传播子|跃迁振幅|关联振幅|概率振幅|高等量子力学
% license Xiao
% type Tutor

\pentry{传播子(量子力学)\nref{nod_PpgtQM},薛定谔绘景和海森堡绘景\nref{nod_HsbPic}}{nod_9a9b}

% 定义传播子$C_i(t) = \bra{\bvec{x}_{i+1}}\E^{-\I Ht}\ket{\bvec{x}_i}$。


\subsection{概念的引入}

为了方便,我们考虑二维时空的情况,即空间只有一维。

在初始时刻$t=0$时,一个粒子处于$x_0$位置,将它的态记为$\ket{x_0}$,其在位置空间的波函数为$\psi_0(x)=\braket{x}{x_0}=\delta(x-x_0)$。

时间过去$t_1$后,我们在$x_1$位置测量,发现粒子的概率振幅为$\bra{x_1}\E^{-\I H t_1}\ket{x_0}$。因此我们可以说,粒子在时间$t_1$后“出现”在\footnote{我们也可以说“传播”到。}$x_1$的概率密度是$\abs{\bra{x_1}\E^{-\I H t_1}\ket{x_0}}^2$。

同样地,时间过去$t_2>t_1$后,粒子在$x_2$位置的概率振幅为$\bra{x_2}\E^{-\I H t_2}\ket{x_0}$。

注意到$\int \ket{x_1}\bra{x_1}\dd x_1=1$,即恒等变换(其矩阵总是单位矩阵),因此我们可以把这个积分插入到任何位置,比如:
\begin{equation}\label{eq_PIntQM_1}
\ali{
    \bra{x_2}\E^{-\I H t_2}\ket{x_0} &= \bra{x_2}\int \ket{x_1}\bra{x_1}\dd x_1 \E^{-\I H t_2}\ket{x_0}\\
    &= \int \braket{x_2}{x_1}\bra{x_1} \E^{-\I H t_2}\ket{x_0}\dd x_1\\
    &= \int \bra{x_2}\E^{-\I H( t_2-t_1)}\ket{x_1}\bra{x_1} \E^{-\I H t_1}\ket{x_0}\dd x_1~.\\
}
\end{equation}

\autoref{eq_PIntQM_1} 数学上成立,但它有什么物理意义呢?

每个$\bra{x_2}\E^{-\I H( t_2-t_1)}\ket{x_1}\bra{x_1} \E^{-\I H t_1}\ket{x_0}$表达的是,粒子从$x_0$开始,$t_1$后出现在$x_1$的振幅,乘以从$x_1$开始,再过$t_2-t_1$后出现在$x_2$的振幅。而\autoref{eq_PIntQM_1} 是对这个表达式关于$x_1$遍历全空间求积分。

综上,\autoref{eq_PIntQM_1} 表达的是:求粒子从$x_0$出发、经过$t_2$后出现在$x_2$的概率振幅,等于先求出粒子经过$t_1$后传播到$x_1$后再从$x_1$经过$t_2-t_1$后传播到$x_2$的振幅,然后把所有可能的$x_1$遍历一遍,把得到的所有\textbf{路径}的振幅求积分。这个过程如\autoref{fig_PIntQM_1} 所示:

\begin{figure}[ht]
\centering
\includegraphics[width=10cm]{./figures/6f78e294c2f7073b.pdf}
\caption{路径积分的示意图。如图,计算粒子从$x_0$经过时间$t_2$后传播到$x_2$的概率振幅,相当于图中各路径的振幅关于$x_1$遍历整个空间求积分。也就是说,$x_1$取遍所有可能性,得到类似图中三条路径的所有路径,所有这些路径的振幅积分,即为所求。} \label{fig_PIntQM_1}
\end{figure}



同样地,我们可以把时间分成多段,产生更多的路径可能性,则我们所求的振幅$\bra{x_n}\mathcal{U}(t)\ket{x_1}$同样是所有这些可能路径的振幅之积分,如\autoref{fig_PIntQM_2} 所示。

\begin{figure}[ht]
\centering
\includegraphics[width=10cm]{./figures/3e64cedefe1413be.pdf}
\caption{将时间分成多段后,得到更多可能路径。} \label{fig_PIntQM_2}
\end{figure}


积分表达式为:
\begin{equation}\label{eq_PIntQM_5}
\ali{
    &\bra{x_n}\E^{-\I H t_n}\ket{x_0}\\
    =& \int \bra{x_n}\E^{-\I H( t_n-t_1 )}\ket{x_1}\bra{x_1} \E^{-\I H t_1}\ket{x_0}\dd x_1\\
    =& \int\int \bra{x_n}\E^{\I H (t_3-t_2)}\ket{x_2}\bra{x_2}\E^{-\I H( t_2-t_1 )}\ket{x_1}\bra{x_1} \E^{-\I H t_1}\ket{x_0}\dd x_1\dd x_2\\
    =& \cdots\\
    =&\int\int\cdots\int \dd x_1\dd x_2\cdots \dd x_{n-1}\times \\
    &\qty (\bra{x_n}\E^{-\I H( t_n-t_{n-1} )}\ket{x_{n-1}}\cdots\bra{x_2}\E^{-\I H( t_2-t_1 )}\ket{x_1}\bra{x_1} \E^{-\I H t_1}\ket{x_0})~.
}
\end{equation}



最后,当我们给时间所分的段数趋于无穷时,能得到所有可能的路径,如\autoref{fig_PIntQM_3} 所示。同样地,所有这些路径的振幅之积分就是我们所求的$\bra{x_n}\E^{-\I H t_n}\ket{x_0}$。



\begin{figure}[ht]
\centering
\includegraphics[width=10cm]{./figures/3cd3f93d61e9bfcd.pdf}
\caption{全体可能的路径。} \label{fig_PIntQM_3}
\end{figure}


\subsection{路径积分的计算}


\subsubsection{无穷小路径的概率振幅与经典作用量}

我们回到\autoref{fig_PIntQM_1} 的例子。任取$t_1$时刻一个给定的位置$x_1$,得到一条给定的折线路径。粒子沿着\textbf{这条路径}演化到$x_2$的概率振幅是\footnote{你可以理解为,在$t_1$时刻整个宇宙盖上了吸收粒子的材料,只有$x_1$处例外,然后立刻把材料取走,那么在$t_2$时刻在$x_2$找到粒子的概率振幅就是\autoref{eq_PIntQM_2} 。这可以和单缝衍射作类比。}
\begin{equation}\label{eq_PIntQM_2}
\bra{x_2}\E^{\I H (t_2-t_1)}\ket{x_1}\bra{x_1}\E^{\I H t_1}\ket{x_1}~,
\end{equation}
这是两条直线段路径的概率振幅之\textbf{积}。

% 应用上述事实,让时间分割趋于无穷,可以推知,一条任意路径的概率振幅,可以视为各\textbf{无穷小段}的概率振幅之积。于是,我们又要考虑无穷小近似了。

% \subsubsection{无穷小路径的概率振幅}

由于概率振幅是个数字,即路径的泛函,这让人想到一个重要的泛函,经典作用量:$S=\int_{\text{给定路径}}L\dd t$,其中拉格朗日函数$L=\frac{m\dot{x}^2}{2}-V(x)$。

折线段的作用量,是各直线段组分的作用量之\textbf{和};而\autoref{eq_PIntQM_2} 中,折线段的概率振幅,是各直线段组分的概率振幅之\textbf{积}。由此猜想,可以取作用量的指数,来对应概率振幅。

对于每条折线段(以\autoref{fig_PIntQM_2} 为例),记其$t_{i-1}$和$t_i$之间直线段的作用量为
\begin{equation}
S(t_{i}-t_{i-1})=\int^{t_i}_{t_{i-1}}\dd t L(x, \dot{x})~,
\end{equation}
给定\textbf{无穷小段}的作用量记为$S(\dd t)$。

我们猜测
\begin{equation}
\ali{
    \bra{x_1}\E^{\I H \dd t}\ket{x_0} &= \qty(\frac{1}{\omega(\dd t)})\exp\qty(\I S(\dd t))~,\\
    %&= \qty(\frac{1}{\omega(\Delta t)})\qty(1+\frac{\I}{\hbar}S(\dd t))
}
\end{equation}
其中$\omega$是比例和量纲的修正项。下面就讨论该猜想的合理性。

我们对自由粒子应用上述猜想。注意到自由粒子的拉格朗日函数为$L=mv^2/2$,而无穷小段可以近似于匀速直线运动,故无穷小段的作用量为(时间差$\Delta t$是小量,位置$x_0$到$x_1=x_0+v \Delta t$):
\begin{equation}\label{eq_PIntQM_6}
\ali{
    S(\Delta t) &= \int^{\Delta t}_0 \dd t\qty(\frac{mv^2}{2})\\
    &= \frac{mv^2}{2}\Delta t\\
    &= \frac{m}{2}\qty(\frac{x_1-x_0}{\Delta t})^2\Delta t~,
}
\end{equation}
于是
\begin{equation}\label{eq_PIntQM_3}
\ali{
    \exp\qty(\I S(\Delta t)) &= \exp\qty(\frac{\I m(x_1-x_0)^2}{2\Delta t})~.
}
\end{equation}
\autoref{eq_PIntQM_3} 右边的\textbf{指数}恰为\textbf{自由粒子传播子}\autoref{eq_PpgtQM_9}~\upref{PpgtQM}的指数部分!

因此,对于一维自由粒子,其无穷小段路径的概率振幅为
\begin{equation}\label{eq_PIntQM_4}
\bra{x_1}\E^{-\I H \dd t}\ket{x_0} = \sqrt{\frac{m}{2\pi\I \dd t}}\exp\qty(\I S(\dd t))~,
\end{equation}
其中$S$是该路径的经典作用量。

类似地,对于三维自由粒子,其无穷小段路径的概率振幅为
\begin{equation}
\bra{x_1}\E^{-\I H \dd t}\ket{x_0} = \qty(\frac{m}{2\pi\I \dd t})^{3/2}\exp\qty(\I S(\dd t))~.
\end{equation}

\subsubsection{一维自由粒子的路径积分}

考虑\autoref{fig_PIntQM_2} 中任意\textbf{给定路径}$\Gamma$,其起点与终点分别\textbf{固定}为$(x_0, t_0)$和$(x_n, t_n)$。

设时间段均匀划分为$n$段,即各$t_{i+1}-t_i=t_n/n$。当$n\to\infty$时,各$t_{i+1}-t_i$趋于$0$,则由\autoref{eq_PIntQM_5} 、\autoref{eq_PIntQM_6} 和\autoref{eq_PIntQM_4} ,粒子沿着这条路径的概率传播的振幅为
\begin{equation}
\ali{
    \bra{x_n}\E^{-\I H t_n}\ket{x_0} &= \lim_{n\to\infty} \int\int\cdots\int \dd x_1\dd x_2\cdots \dd x_{n-1}\times \\
    &\phantom{11}\qty (\bra{x_n}\E^{-\I H( t_n-t_{n-1} )}\ket{x_{n-1}}\cdots\bra{x_2}\E^{-\I H( t_2-t_1 )}\ket{x_1}\bra{x_1} \E^{-\I H t_1}\ket{x_0})\\
    &= \lim_{n\to\infty} \qty(\frac{m}{2\pi\I \dd t})^{n/2}\int\int\cdots\int \dd x_1\dd x_2\cdots \dd x_{n-1}\times \\
    &\phantom{11}\exp\qty(\I\frac{nm}{2t_n}\sum_{i=0}^{n-1}(x_{i+1}-x_i)^2)\\
    &\propto \int \mathcal{D}[x] \exp\left(i\int_0^T \frac{m}{2}\dot x^2\right)~.
}
\end{equation}
其中 $\mathcal{D}[x]$ 代表泛函积分测度,这里意味着对所有可能的固定两端 $x(t_0)=x_i,x(t_n)=x_f$ 的路径 $\Gamma$ 作积分。注意到 $S[x,\dot x]=\int_{t_0}^{t_n} \dd t\ m\dot x^2/2$ 为自由粒子沿路径 $\Gamma$ 计算得到的作用量,因此可以将上式简记为
\begin{equation}
\bra{x_f} e^{-iHT}\ket{x_i}=\mathcal{N}\cdot \int \mathcal{D}[x]|_{x(t_0)=x_i}^{x(t_n)=x_f} \exp(i S[x,\dot x])~.
\end{equation}
$\mathcal{N}$ 是依赖于泛函积分测度定义方式的一个常数因子,例如在上文中我们将 $t_0$ 到 $t_n$ 的时间划分为 $n$ 段,这导致了常数因子 $\mathcal{N}=(\frac{m}{2\pi\I \dd t})^{n/2}$。
\subsubsection{一维势场中粒子的路径积分}
对于一维势场 $V(x)$ 中的粒子,其哈密顿量为 $H=\frac{\hat{p}^2}{2m}+V(x)$,拉氏量为 $L=\frac{m}{2}\dot x^2-V(x)$。类似于自由粒子的无穷小段路径概率振幅的公式\autoref{eq_PIntQM_4},我们同样可以证明
\begin{equation}
\bra{x_1} e^{-iH \dd t} \ket{x_0} = \sqrt{\frac{m}{2\pi i \dd t}}\exp(iS(\dd t))~,
\end{equation}
其中
\begin{equation}
\exp(iS(\dd t))=\exp\left(i\Delta t\left(\frac{m(x_1-x_0)^2}{2\Delta t^2}-V\left(\frac{x_0+x_1}{2}\right)\right)\right)~.
\end{equation}
通过类似的推导,最终可以得到一维势场中粒子的路径积分公式:
\begin{equation}
\bra{x_f} e^{-iHT}\ket{x_i}=\mathcal{N}\cdot \int \mathcal{D}[x]|_{x(t_0)=x_i}^{x(t_n)=x_f} \exp(i S[x,\dot x])~,
\end{equation}
其中 $S[x,\dot x]=\int_{t_0}^{t_n} \dd t\ \left(m\dot x^2/2-V(x)\right)$ 为粒子沿这条路径所对应的作用量。
%未完。参考费曼的《量子力学与路径积分》继续。

\subsubsection{路径积分的经典极限}
在上面的推导中我们采取了自然单位制 $\hbar=1$。现在我们复原 $\hbar$,并考察路径积分公式在经典极限下的行为。
\begin{equation}
\bra{x_f} e^{-i\frac{H}{\hbar}T}\ket{x_i}=\mathcal{N}\cdot \int \mathcal{D}[x]|_{x(t_0)=x_i}^{x(t_n)=x_f} \exp(\frac{i}{\hbar}S[x,\dot x])~.
\end{equation}
在经典极限 $\hbar\rightarrow 0$ 下,路径积分公式中 $S[x,\dot x]$ 取极值的路径对结果的贡献最大,而远离极值的路径对应的 $\exp(\frac{i}{\hbar}S[x,\dot x])$ 相位角振荡会非常剧烈,以至于在经典极限下这些偏离作用量极值的路径的贡献相抵消。这也就是说,在经典极限 $\hbar\rightarrow 0$ 的情形下,粒子的运动将只由作用量泛函取极值的路径所决定,这样就从量子力学的情况退化到了经典力学体系,与我们所知的最小作用量原理\upref{HamPrn} 相符合。

\subsection{路径积分公式的 Dirac 方法推导}
下面我们利用 Dirac 的思路给出一个更简洁的对路径积分公式的证明方法。
\begin{theorem}{量子力学路径积分公式}\label{the_PIntQM_1}
    设有相互作用的哈密顿量
    \begin{equation}
        H=\frac{\hat p^2}{2m}+V(x)~,
    \end{equation}
    那么 $\bra{x_f}e^{-iHT}\ket{x_i}$ 的路径积分表达式为
    \begin{equation}
        \bra{x_f}e^{-iHT}\ket{x_i}=\mathcal{N}\cdot\int \mathcal D[x]|_{x(0)=x_i}^{x(T)=x_f} \exp\left(iS[x,\dot x]\right)~.
    \end{equation}
    其中 $S[x,\dot x]=\int_0^T d t L(x(t),\dot x(t))=\int_0^T d t \left[\frac{1}{2}m\dot{x}^2(t)-V(x(t))\right]$ 为粒子的作用量。
    $\mathcal D[x]$ 为泛函积分测度,当时间划分为 $N$ 个间隔时它可以被定义为 $=d x_1\cdots d x_{N-1}$($x_i$ 为第 $i$ 个时刻的粒子坐标),
    而此时常数 $\mathcal N=\left(\frac{m}{2\pi i a}\right)^{N/2}$(依赖于泛函积分测度的定义)。
\end{theorem}

将时间 $T$ 划分为 $N$ 段,并在计算的最后取 $N\rightarrow +\infty$ 的极限。设 $T=Na$。
\begin{equation}
\begin{aligned}
        &\bra{x_f}e^{-iHT}\ket{x_i}\\
        &=\int d x_1\cdots d x_{N-1}
        \bra{x_f}e^{-iH a}\ket{x_{N-1}}\bra{x_{N-1}}e^{-iH a}
        \ket{x_{N-2}}\bra{x_{N-2}}\cdots \ket{x_{1}}\bra{x_{1}}e^{-iH a}\ket{x_i}
        \\
        &=\int d x_1\cdots d x_{N-1}\frac{d p_1}{2\pi} \cdots \frac{d p_N}{2\pi}
        \bra{x_f}e^{-iH a}\ket{p_N}\langle p_N|x_{N-1}\rangle 
        \\
        &\quad \times\bra{x_{N-1}}e^{-iH a}
        \ket{p_{N-1}}\langle p_{N-1}|x_{N-2}\rangle
        \cdots \bra{x_{1}}e^{-iH a}\ket{p_1}\langle p_1 | x_i\rangle~.
\end{aligned}
\end{equation}

将 $e^{-iH a}=\exp[-ia[\hat p^2/2m + V(x)]]$ 改写为
\begin{equation}
        e^{-iH a}
        =\exp[-iaV(x)]\exp\left[-ia\frac{\hat p^2}{2m}\right]+O(a^2)~.
\end{equation}

当 $N\rightarrow \infty$ 时 $O(a^2)$ 的修正项可以忽略,代入上式可以得到
\begin{equation}
\begin{aligned}
        &\quad\bra{x_f}e^{-iHT}\ket{x_i}\\
        &=\int d x_1\cdots d x_{N-1}\int \frac{d p_1}{2\pi} \cdots \frac{d p_N}{2\pi}\\
        &\quad\times \exp(-iaV(x_f))\exp(-iap_N^2/2m) \exp(ip_Nx_f)\exp(-ip_N x_{N-1})
        \\
        &\quad \times \cdots \times 
        \exp(-iaV(x_1))\exp(-iap_1^2/2m) \exp(ip_1x_1)\exp(-ip_1 x_{i})
        \\
        &=\int d x_1\cdots d x_{N-1}\int \frac{d p_1}{2\pi} \cdots \frac{d p_N}{2\pi}
        \exp\left(ia\sum_{i=1}^N\left(p_i\frac{x_i-x_{i-1}}{a}-H(p_N,x_N) \right)\right)\\
        &\quad(x_0\equiv x_i,x_N\equiv x_f)~.
\end{aligned}
\end{equation}
为了使得关于 $p_i$ 的高斯积分收敛,我们可以手动的将 $p^2$ 视作 $p^2(1-i\epsilon)$,这样高动量处积分的贡献就被指数压低了\footnote{Gell-Mann-Low 定理\upref{pathi2}将说明这样做是合理的。}。
    利用高斯积分公式 $\int_{-\infty}^{\infty} dx e^{-x^2/A}=\sqrt{A\pi}$,可以将上面积分公式中的动量积分部分全部约去。最终可以得到
\begin{equation}
\begin{aligned}
        \bra{x_f}e^{-i\hat HT}\ket{x_i}&=
        \left(\frac{m}{2\pi i a}\right)^{N/2}
        \int d x_1\cdots d x_{N-1} 
        \exp\left( ia\sum_{i=1}^N \left( \frac{1}{2}m \left(\frac{x_i-x_{i-1}}{a}\right)^2 - V(x_i) \right) \right)
        \\
        &=\mathcal{N} \cdot 
        \int d x_1\cdots d x_{N-1} \exp(iS[x,\dot x])~.
\end{aligned}
\end{equation}
