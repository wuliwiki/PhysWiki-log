% 球谐函数表
% 球谐函数|勒让德多项式

\pentry{球谐函数\upref{SphHar}}

下面列出一些常见的球谐函数, 这和 Mathematica 和 Wolfram Alpha 的定义\footnote{例子: 在 \href{https://www.wolframalpha.com/}{wolframalpha.com} 中输入 \lstinline|SphericalHarmonics[2,1]|, 或者在 Mathematica\upref{Mma} 中输入 \lstinline|SphericalHarmonicY[2, 1, θ, ϕ]|} 一致。 Condon–Shortley 相位(\autoref{SphHar_sub1}~\upref{SphHar}) 体现在对于奇数 $m$ 的球谐函数表达式前的 $\mp$ 号。

\begin{equation}
Y_{l,m}(\theta, \phi) = \sqrt{\frac{2l + 1}{4\pi} \frac{(l - m)!}{(l + m)!}} P_l^m (\cos\theta) \E^{\I m\phi}~,
\end{equation}
\begin{equation}
l = 0 \qquad
Y_{0,0} = \sqrt{\frac{1}{4\pi}}
\end{equation}
\begin{equation}
l = 1 \qquad
\leftgroup{
Y_{1,0} &= \sqrt{\frac{3}{4\pi}} \cos\theta \\
Y_{1,{\pm 1}} &= \mp\sqrt{\frac{3}{8\pi}} \sin\theta \  \E^{\pm\I m\phi}
}\end{equation}
\begin{equation}
l = 2 \qquad
\leftgroup{
Y_{2,0} &= \sqrt{\frac{5}{16\pi}} (3\cos^2 \theta  - 1)\\
Y_{2,{\pm1}} &= \mp \sqrt{\frac{15}{8\pi}} \sin\theta \cos\theta \  \E^{ \pm \I m\phi}\\
Y_{2,{\pm 2}} &= \sqrt{\frac{15}{32\pi}} \sin ^2\theta  \  \E^{\pm 2\I m\phi}
}\end{equation}
\begin{equation}
l = 3 \qquad
\leftgroup{
Y_{3,0} &= \sqrt{\frac{7}{16\pi}} (5\cos^3 \theta  - 3 \cos \theta)\\
Y_{3,{\pm1}} &= \mp \sqrt{\frac{21}{64\pi}} \sin\theta (5\cos^2\theta - 1) \  \E^{ \pm \I m\phi}\\
Y_{3,{\pm 2}} &= \sqrt{\frac{105}{32\pi}} \sin ^2\theta \cos\theta  \  \E^{\pm 2\I m\phi}\\
Y_{3,{\pm 3}} &= \mp \sqrt{\frac{35}{64\pi}} \sin ^3\theta  \  \E^{\pm 3\I m\phi}
}\end{equation}