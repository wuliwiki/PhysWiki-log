% 兰州大学 2016 硕士研究生招生初试试题
% keys 热力学|统计管理|
% license Usr
% type Note


\textbf{声明}:“该内容来源于网络公开资料,不保证真实性,如有侵权请联系管理员”



一、(每小题 10 分,共 30 分)

1、为什么说熵是态函数?其物理意义是什么?

2、对于 1mol 范氏气体(范德瓦尔斯气体),经绝热自由膨胀过程,体积从 V 变为
2V,熵变为多少?

3、一定质量温度为 0°C的水和温度为 100°C的热源接触,水温达到 100°C°C。欲使
整个系统的熵保持不变,应如何使水温从 0°C升到 100°C?

二、(每小题 10 分,共 30 分)

1、什么是负温度?请从热力学基本微分方程出发说明负温度存在的条件。

2、实际气体在节流过程中,气体温度降低的条件是什么?

3、压强降落相同,气体在准静态绝热膨胀过程及节流过程中温度降落与初温的关系。

三、(每小题 10 分,共 30 分)

1、热容的定义是什么?热源热容为多少?

2、定性分析常温下电子对体系热容的贡献有多大。

3、用波尔兹曼统计推到理想气体的定压热容。

四、(每小题 10 分,共 30 分)

1、费米子在 T=0K 时如何占据能级?

2、T=0K 时最高能级与哪些物理量有关?

3、在 kBT<<μ (0)的情形下,热容、熵、自由能、化学势与温度有何关系?

五、(每小题 10 分,共 30 分)

1、什么是元激发方法?元激发的条件是什么?

2、写出至少 2 中元激发的例子,并描述其物理图像。

3、固体中某种准粒子遵从玻色分布,具有以下色散关系ω =Ak2,在低温范围内,
这种准粒子的激发所导致的热容与 T 有何关系?