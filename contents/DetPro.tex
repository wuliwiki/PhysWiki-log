% 行列式的性质
% 行列式|交换两行|绝对值|体积|元素|转置|外代数

\pentry{行列式\upref{Deter}}

\addTODO{应该并入行列式\upref{Deter}}

以下是行列式常见的性质, 我们从定义出发容易一一证明. 由于行列式的绝对值表示体积\upref{DetVol}, 我们也可以从几何上理解这些定理. 这些定理对复数元素的行列式同样适用,只是复数行列式不再具有直观的几何意义.

\begin{theorem}{ } \label{DetPro_the1}
若行列式中某行或某列全为 0, 其结果等于 0.
\end{theorem}
证明: 根据定义(\autoref{Deter_eq4}~\upref{Deter}), 行列式展开后, 相加的每一项都含有每一行(或每一列)的一个元素, 所以当某列全为 0, 则结果为 0.

几何理解: 若平行体的某条边长等于 0, 其体积也等于 0.

\begin{theorem}{ } \label{DetPro_the3}
矩阵的任意一列(或任意一行)乘以常数,行列式的值也要乘以该常数.
\end{theorem}
证明: 思路和\autoref{DetPro_the1} 的证明一样, 行列式展开后, 相加的每一项都含有每一行(或每一列)的一个元素, 所以当某列全部元素乘以常数, 则相加的每一项都乘以该常数.

几何理解: 将平行体任意一条边长乘以一个常数, 它的体积也需要乘以该常数.

\begin{theorem}{ }\label{DetPro_the6}
将行列式的两列交换, 结果取相反数.
\end{theorem}
证明: 根据行列式的定义, 交换两行或两列会给展开后的每一项增加一个逆序数, 导致行列式的值取相反数.

\begin{theorem}{}\label{DetPro_the4}
若将行列式的某行或某行的每一个元素都表示为两个数之和,那么它就可以表示为两个行列式相加:
\begin{equation}\label{DetPro_eq1}
\begin{aligned}
&\quad\begin{vmatrix}
a_{1,1} & a_{1,2} & \dots & a_{1,N}\\
\dots & \dots & \dots & \dots\\
b_{i,1} + c_{i,1} & b_{i, 2} + c_{i, 2} & \dots & b_{i,N} + c_{i,N}\\
\dots & \dots & \dots & \dots\\
a_{N,1} & a_{N,2} & \dots & a_{N,N}\\
\end{vmatrix}\\
&=
\begin{vmatrix}
a_{1,1} & a_{1,2} & \dots & a_{1,N}\\
\dots & \dots & \dots & \dots\\
b_{i,1} & b_{i, 2} & \dots & b_{i,N}\\
\dots & \dots & \dots & \dots\\
a_{N,1} & a_{N,2} & \dots & a_{N,N}\\
\end{vmatrix}
+
\begin{vmatrix}
a_{1,1} & a_{1,2} & \dots & a_{1,N}\\
\dots & \dots & \dots & \dots\\
c_{i,1} & c_{i, 2} & \dots & c_{i,N}\\
\dots & \dots & \dots & \dots\\
a_{N,1} & a_{N,2} & \dots & a_{N,N}\\
\end{vmatrix}
\end{aligned}
\end{equation}
列的情况也同理.
\end{theorem}
证明: 行列式展开后, 相加的每一项都含有每一行(或每一列)的一个元素, 所以显然成立.

几何理解: 若把平行体的一条边变为两条折线, 那么可以作出两个叠加的平行体, 体积之和等于原来的体积.

\begin{theorem}{ }\label{DetPro_the5}
把矩阵的第 $i$ 行(列)加上 “第 $j$ 行(列)乘任意常数 $\lambda$”,行列式的值不变.
\end{theorem}
证明: 根据\autoref{DetPro_the4} 和\autoref{DetPro_the3}, 我们可以把这样操作后的行列式拆分为两个行列式, 第一个与原来相同; 第二个在原来的基础上把第 $i$ 行替换为第 $j$ 行, 再把行列式的值乘以 $\lambda$. 而第二个行列式由于存在两行相同, 结果为 0.

几何理解: 以平行四边形为例, 由于其体积是底乘以高, 令 $\bvec v_1$ 为底, $\bvec v_2$ 在垂直于 $\bvec v_1$ 方向的投影为高, 则将 $\bvec v_2$ 变为 $\bvec v_2 + \lambda \bvec v_1$ ($\lambda$ 为常数)后高不变, 所以面积不变. 高维情况同理.

\begin{theorem}{ }
若行列式中的行矢或列线性相关, 行列式的值为 0.
\end{theorem}
线性相关意味着存在某一行 $i$, 可以表示为其他行的线性组合. 那么我们可以通过将这些其他的行依次乘以常数, 然后加到第 $i$ 行上, 使得第 $i$ 行全为 0. 根据\autoref{DetPro_the5}, 这么做不改变行列式的值, 而根据\autoref{DetPro_eq1}, 行列式的值为 0.

几何理解: 二维情况下两矢量线性相关意味着他们共线, 平行四边形面积为 0. 三维情况下线性相关意味着三个矢量共面, 平行四面体体积为 0. 高维情况也可类比.

\begin{theorem}{ } \label{DetPro_the2}
行列式的值为 0 当且仅当行列式中存在线性相关的列(行).
\end{theorem}

\begin{theorem}{ }\label{DetPro_the7}
矩阵\textbf{转置}(将所有 $a_{i,j}$ 与 $a_{j,i}$ 交换\upref{Mat})后行其列式的值不变.
\end{theorem}
这个定理没有显然的几何理解, 可以直接用行列式的定义证明(\autoref{Deter_eq4}~\upref{Deter}). 根据这个定理, 以上凡是涉及到 “行” 的定理和说明, 都可以替换为 “列”, 请读者自行回顾一次.

\begin{theorem}{}\label{DetPro_the8}
两方阵相乘的行列式等于他们的行列式相乘
\begin{equation}\label{DetPro_eq2}
\abs{\mat A \mat B} = \abs{\mat A}\abs{\mat B}
\end{equation}
\end{theorem}
证明: 容易证明对 $\mat A$ 做行变换或对 $\mat B$ 左列变换分别相当于对 $\mat A\mat B$ 做相同的行变换或列变换, 所以根据\autoref{DetPro_the5}, 变换后\autoref{DetPro_eq2} 两边的值都不改变. 用行变换和列变换把 $\mat A$ 和 $\mat B$ 都变为对角矩阵后, 证明显然(留做习题).

\begin{theorem}{}
若$A$的逆矩阵$A^{-1}$存在,那么
\begin{equation}
\abs{\bvec A}=\frac{1}{\abs{\bvec A^{-1}}}
\end{equation}
\end{theorem}
证明:
$\abs{\mat I} = \abs{\mat A \mat A^{-1}} = \abs{\mat A}\abs{\mat A^{-1}}=1$,所以
$\abs{\mat A}=\frac{1}{\abs{\mat A^{-1}}}$
%话说2,3,5的证明使用初等行、列变换的思路可能更为简洁,需要补上吗

\subsection{拓展}

行列式的代数性质可以抽象为\textbf{外代数}(见\autoref{AlgFie_ex1}~\upref{AlgFie} 以及例子后的解释),进而用于定义\textbf{外微分}\upref{ExtDer},描述微分形式的代数性质.
