% 函数的算符
% keys 算符|算子|微分算符

\pentry{导数\upref{Der}}
\addTODO{定积分算符}

\textbf{算符}(也叫\textbf{算子})可以理解为 “函数的函数”, 即一个函数经过算符作用可以得到另一个算符。 例如将一个函数乘以一个数 $\lambda$ 或另一个函数, 又例如对一个函数求导(\textbf{微分算符})等等。

\begin{example}{}
令算符 $\Q A$ 为
\begin{equation}
\Q A = \sin(x)~,
\end{equation}
那么它作用在函数 $f(x)$ 上就是
\begin{equation}
\Q A f(x) = \sin(x) f(x)~.
\end{equation}
令算符 $\Q B$ 为
\begin{equation}
\Q B = \dv{x} + 1~,
\end{equation}
那么
\begin{equation}
\Q B f(x) = \qty(\dv{x} + 1)f(x) = \dv{f(x)}{x} + f(x)~.
\end{equation}
令算符 $\Q C$ 为
\begin{equation}
\Q C = \Q A \Q B = \sin(x) \qty(\dv{x} + 1)~,
\end{equation}
那么
\begin{equation}
\Q C f(x) = \Q A [\Q B f(x)] = \sin(x)\dv{f(x)}{x} + \sin(x)f(x)~.
\end{equation}
令算符 $\Q D$ 为
\begin{equation}
\Q D = \Q A + \Q B~,
\end{equation}
那么
\begin{equation}
\Q D f(x) = \Q A f(x) + \Q B f(x)~.
\end{equation}
令算符 $\Q E$ 为
\begin{equation}
\Q E = {\Q B}^2~.
\end{equation}
相当于 $\Q B$ 对某函数作用两次, 即
\begin{equation}
\begin{aligned}
\Q E f(x) &= \Q B [\Q B f(x)] = \qty(\dv{x} + 1)\qty[\dv{f(x)}{x} + f(x)] \\
&= \dv[2]{f(x)}{x} + 2\dv{f(x)}{x} + f(x)~.
\end{aligned}
\end{equation}
所以也可以将这个过程简写为
\begin{equation}
\Q B^2 = \qty(\dv{x} + 1)^2 = \dv[2]{x} + 2\dv{x} + 1~.
\end{equation}
\end{example}

\subsection{线性}
若一个算符 $\Q A$ 对两个不同的函数 $f, g$ 和两个不同的常数满足
\begin{equation}
\Q A (f + g) = \Q A f + \Q A g~.
\end{equation}
是\textbf{线性}的。
