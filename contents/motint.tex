% 运动积分
% keys 拉格朗日方程|欧拉定理|广义动量|哈密顿量|守恒
% license Xiao
% type Tutor

\pentry{拉格朗日方程\nref{nod_Lagrng}, 齐次函数的欧拉定理\nref{nod_Homeul}}{nod_415a}

我们知道拉格朗日方程是关于广义坐标 $q_i(i=1,2,\cdots,s)$ 的二阶微分方程。在一些情况下,在系统运动过程中,存在 $q_i$ 和 $\dot{q}_i$ 的某些函数,它们不随时间而变,这些函数称为系统的\textbf{运动积分}(integral of motion)。运动积分是通常的守恒律,如动量守恒定律、角动量守恒定律、机械能守恒定律的概念的推广。可以看出,运动积分相对于拉格朗日方程而言降了一阶,是一阶的微分方程,所以运动积分有时也称为\textbf{第一次积分}。力学中的对称性是十分重要的,而运动积分的存在与否,与系统的对称性有密切的关系。一个系统如果有尽可能多的运动积分,将对问题的求解带来极大的方便。

\subsection{可遗坐标与广义动量积分}

若拉格朗日函数 $L$ 不含某个广义坐标 $q_\beta$,即 $\dfrac{\partial L}{\partial q_\beta}=0$,则这种广义坐标叫做\textbf{可遗坐标(ignorable coordinates)}, 也称为\textbf{循环坐标(cyclic coordinates)}。 于是, 拉格朗日方程写为
\begin{equation}
\frac{\mathrm{d}}{\mathrm{d} t}\left(\frac{\partial L}{\partial \dot{q}_{\beta}}\right)=0~.
\end{equation}
这是说,广义动量 $p_\beta=\dfrac{\partial L} {\partial \dot{q_\beta}}$ 是守恒的,
\begin{equation}
p_\beta= \text{常数(如 $L$ 不含有 $q_\beta$)}~.
\end{equation}
这叫做\textbf{广义动量积分}。

我们知道,若循环坐标 $q_\beta$ 是系统的整体平移坐标,即拉格朗日函数 $L $ 对于整体平移是不变的,可知广义动量积分归结为动量守恒定律。若拉格朗日函数不包含整体转动坐标,即拉格朗日函数 $L$ 对于整体转动是不变的(各向同性),则广义动量积分归结为角动量守恒定律。

在矢量力学中,动量守恒定律和角动量守恒定律是以牛顿第三定律为先决条件,即内力的矢量和为零、内力的力矩和为零,而\textbf{广义动量积分则并不以牛顿第三定律为先决条件}。这点在后续讨论电磁场时十分明显,很难根据电磁场与粒子的相互作用来谈牛顿第三定律。

\begin{example}{两个楔子的加速度}
\addTODO{整合到 “滑块和运动斜面问题(blkSlp.tex)” 中}
质量为 $M$ 的光滑大楔子置于光滑的水平桌面上,质量为 $m$ 的光滑小楔子沿着大楔子的光滑斜边滑下,如\autoref{fig_motint_1} 所示。求这两个楔子的加速度。
\begin{figure}[ht]
\centering
\includegraphics[width=7.5cm]{./figures/af098a90cd3f270b.pdf}
\caption{两个楔子} \label{fig_motint_1}
\end{figure}

解:大楔子可在水平方向运动,小楔子在大楔子斜边上运动系统有两个自由度。

取桌面上的固定点 $O$,把大楔子的质心相对于 $O$ 点的水平坐标记作 $X $。把小楔子的质心相对于大楔子斜边底端、沿着斜边计算的坐标记作 $q$(其实取的时候不需要一定是质心,物体上的任一点都可以)。这个系统的广义坐标就是 $X $ 和 $q $。

主动力是两个楔子所受的重力,它们都是势力。大楔子的势能在运动过程中不变,不需要考虑,只需要讨论小楔子的势能。计算动能需注意,小楔子的速度不仅仅是沿斜边的 $\dot q$,而且还有随着大楔子在水平方向运动的速度 $\dot X$。可得:
\begin{equation}
\begin{aligned} T &=\frac{1}{2} M \dot{X}^{2}+\frac{1}{2} m\left[v_{\text {水平 }}^{2}+v_{\text {竖直 }}^{2}\right] \\ &=\frac{1}{2} M \dot{X}^{2}+\frac{1}{2} m\left[(\dot{X}+\dot{q} \cos \theta)^{2}+\dot{q}^{2} \sin ^{2} \theta\right] \\ V &=m g q \sin \theta \\ L &=T-V=\frac{1}{2}(M+m) \dot{X}^{2}+\frac{1}{2} m \dot{q}^{2}+m \dot{X} \dot{q} \cos \theta-m g q \sin \theta~. \end{aligned}
\end{equation}

由拉格朗日方程
\begin{equation}
\begin{cases}
\dfrac{\mathrm{d}}{\mathrm{d} t}[M \dot{X}+m(\dot{X}+\dot{q} \cos \theta)]=0 \\ \dfrac{\mathrm{d}}{\mathrm{d} t}[m(\dot{X} \cos \theta+\dot{q})]+m g \sin \theta=0
\end{cases}~.
\end{equation}

第一个方程指出,这个系统在水平方向的动量守恒。事实上,$X$ 是可遗坐标,所以相应的广义动量守恒。

由运动方程解得大楔子的加速度
\begin{equation}
\ddot{X}=\frac{m g \sin \theta \cos \theta}{M+m \sin ^{2} \theta}~,
\end{equation}
以及小楔子相对于大楔子的加速度
\begin{equation}
\ddot{q}=-\frac{(M+m) g \sin \theta}{M+m \sin ^{2} \theta}~,
\end{equation}
\end{example}
从此例题可以再次看出用拉格朗日方法解题的优越性。

\subsection{广义能量积分}

把拉格朗日量 $L(q, \dot{q}, t)$ 对 $t$ 求全导数得
\begin{equation}
\dv{L}{t} = \sum_{i} \pdv{L}{q_i} \dot{q}_i + \sum_{i} \pdv{L}{\dot{q}_i} \dv{\dot{q}_i}{t} + \pdv{L}{t}~.
\end{equation}
对右边第一项的偏微分使用拉格朗日方程(\autoref{eq_Lagrng_1}~\upref{Lagrng}), 得
\begin{equation}
\begin{aligned}
\frac{\mathrm{d} L}{\mathrm{d} t} &=\sum_{i} \frac{\mathrm{d}}{\mathrm{d} t}\left(\frac{\partial L}{\partial \dot{q}_{i}}\right) \dot{q}_{i}+\sum_{i} \frac{\partial L}{\partial \dot{q}_{i}} \frac{\mathrm{d} \dot{q}_{i}}{\mathrm{d} t} + \frac{\partial L}{\partial t} \\
&= \frac{\mathrm{d}}{\mathrm{d} t}\left(\sum_{i} \pdv{L}{\dot q_i} \dot{q}_{i}\right) + \frac{\partial L}{\partial t}~,
\end{aligned}
\end{equation}
即
\begin{equation} \label{eq_motint_1}
\frac{\mathrm{d}}{\mathrm{d} t}\left(\sum_{i} \pdv{L}{\dot q_i} \dot{q}_{i}-L\right) = -\frac{\partial L}{\partial t}~.
\end{equation}
当拉格朗日量不显含时间时, 即 $\pdv*{L}{t} = 0$, 我们发现括号中的量是一个守恒量, 叫做\textbf{能量函数(energy function)}\footnote{$h$ 在数值上和以后要学的哈密顿量\upref{HamCan}相等, 但哈密顿量看作是 $q$、广义动量 $p$ 和 $t$ 的函数, 而能量函数看作是 $q, \dot q, t$ 的函数。}
\begin{equation} \label{eq_motint_2}
h = \sum_{i} \frac{\partial L}{\partial \dot{q}_{i}} \dot{q}_{i}-L~,
\end{equation}
该过程也叫做\textbf{雅可比积分(Jacobian integral)}。 在一定条件下这个量是系统的总能量, 详见“哈密顿正则方程\upref{HamCan}”。

\addTODO{下面的部分需要整(删)合(掉)}

我们需要清楚广义能量函数的意义。首先,势能 $V$ 是与广义速度无关的,因此 $H$ 的定义\autoref{eq_motint_2} 中的 $\dfrac{\partial L}{\partial \dot q_i}$。可代之以 $\dfrac{\partial T}{\partial \dot q_i}$。

设变换式 $\bvec{r}_{i}=\bvec{r}_{i}(q)$ 不显含时间,即 $\partial \bvec{r}_{i} / \partial t=0$, 则
\begin{equation}
\dot{\bvec{r}}_{i}=\sum_{i=1}^{s} \frac{\partial \bvec{r}_{i}}{\partial q_{i}} \dot{q}_{i}~.
\end{equation}
于是
\begin{equation}
\begin{aligned} T &=\sum_{i=1}^{n} \frac{1}{2} m_{i} \dot{\bvec{r}}_{i} \cdot \dot{\bvec{r}}_{i}=\sum_{i=1}^{n} \frac{1}{2} m_{i} \sum_{i=1}^{s} \dot{q}_{i} \frac{\partial \bvec{r}_{i}}{\partial q_{i}} \cdot \sum_{\beta=1}^{s} \frac{\partial \bvec{r}_{i}}{\partial q_{\beta}} \dot{q}_{\beta} \\ &=\sum_{i=1}^{n} \sum_{i=1}^{s} \sum_{\beta=1}^{s} \frac{1}{2} m_{i} \frac{\partial \bvec{r}_{i}}{\partial q_{i}} \cdot \frac{\partial \bvec{r}_{i}}{\partial q_{\beta}} \dot{q}_{i} \dot{q}_{\beta}~, \end{aligned}
\end{equation}
这是广义速度的二次齐次多项式。根据齐次函数的欧拉定理,有
\begin{equation} \label{eq_motint_3}
\sum_{i=1}^{s} \frac{\partial T}{\partial \dot{q}_{i}} \dot{q}_{i}=2 T~.
\end{equation}
其实这也可以直接验证:
\begin{equation}
\begin{aligned} \frac{\partial T}{\partial \dot{q}_{\gamma}} &=\sum_{i=1}^{n} \sum_{i=1}^{s} \frac{1}{2} m_{i} \frac{\partial \bvec{r}_{i}}{\partial q_{i}} \cdot \frac{\partial \bvec{r}_{i}}{\partial q_{\gamma}} \dot{q}_{i}+\sum_{i=1}^{n} \sum_{\beta=1}^{s} \frac{1}{2} m_{i} \frac{\partial \bvec{r}_{i}}{\partial q_{\gamma}} \cdot \frac{\partial \bvec{r}_{i}}{\partial q_{\beta}} \dot{q}_{\beta} \\ &=\sum_{i=1}^{n} \sum_{i=1}^{s} m_{i} \frac{\partial \bvec{r}_{i}}{\partial q_{i}} \cdot \frac{\partial \bvec{r}_{i}}{\partial q_{\gamma}} \dot{q}_{i} ~.\end{aligned}
\end{equation}
于是
\begin{equation}
\sum_{\gamma=1}^{s} \frac{\partial T}{\partial \dot{q}_{\gamma}} \dot{q}_{\gamma}=\sum_{i=1}^{n} \sum_{i=1}^{s} \sum_{\gamma=1}^{s} m_{i} \frac{\partial \bvec{r}_{i}}{\partial q_{i}} \cdot \frac{\partial \bvec{r}_{i}}{\partial q_{\gamma}} \dot{q}_{i} \dot{q}_{\gamma}=2 T~.
\end{equation}

由此,广义能量函数
\begin{equation}
H=\sum_{i=1}^{s} p_{i} \dot{q}_{i}-L=2 T-(T-V)=T+V~.
\end{equation}
这样,在变换式 $\bvec{r}_{i}=\bvec{r}_{i}(q)$ 不显含时间的条件下,动能是广义速度的二次齐次式,广义能量函数 $H $ 就是机械能。如果约束是非定常的,则变换式 $\bvec{r}_{i}=\bvec{r}_{i}(q, t)$ 难免显含时间。即使约束是稳定的,也可能由于选择了某些广义坐标(例如平移坐标系),变换式 $\bvec{r}_{i}=\bvec{r}_{i}(q, t)$ 显含时间 $t $。在变换式显含时间的情况下,
\begin{equation}
\dot{\bvec r}_{i}=\frac{\partial \bvec{r}_{i}}{\partial t}+\sum_{i=1}^{s} \frac{\partial \bvec{r}_{i}}{\partial q_{i}} \dot{q}_{i}~.
\end{equation}
于是
\begin{equation}
\begin{aligned} T=& \sum_{i=1}^{n} \frac{1}{2} m_{i}\left(\frac{\partial \bvec{r}_{i}}{\partial t}+\sum_{i=1}^{s} \frac{\partial \bvec{r}_{i}}{\partial q_{i}} \dot{q}_{i}\right) \cdot\left(\frac{\partial \bvec{r}_{i}}{\partial t}+\sum_{\beta=1}^{s} \frac{\partial \bvec{r}_{i}}{\partial q_{\beta}} \dot{q}_{\beta}\right) \\=& \sum_{i=1}^{n}\left\{\frac{1}{2} m_{i}\left(\frac{\partial \bvec{r}_{i}}{\partial t}\right)^{2}+\sum_{i=1}^{s} m_{i} \frac{\partial \bvec{r}_{i}}{\partial t} \cdot \frac{\partial \bvec{r}_{i}}{\partial q_{i}} \dot{q}_{i}\right.\\ &\left.+\sum_{i=1}^{s} \sum_{\beta=1}^{s} \frac{1}{2} m_{i} \frac{\partial \bvec{r}_{i}}{\partial q_{i}} \cdot \frac{\partial \bvec{r}_{i}}{\partial q_{\beta}} \dot{q}_{i} \dot{q}_{\beta}\right\} ~.\end{aligned}
\end{equation}
这包含三个部分,它们分别是广义速度的零次、一次、二次的齐次多项式,分别记作 $T_0,T_1,T_2$,即 $T=T_0+T_1+T_2$。根据齐次函数的欧拉定理,
\begin{equation}
\sum_{i=1}^{s} \frac{\partial T}{\partial \dot{q}_{i}} \dot{q}_{i}=0 T_{0}+1 T_{1}+2 T_{2}=T_{1}+2 T_{2}~.
\end{equation}
由此,广义能量函数
\begin{equation}
H=\sum_{i=1}^{s} p_{i} \dot{q}_{i}-L=\left(T_{1}+2 T_{2}\right)-\left(T_{0}+T_{1}+T_{2}-V\right)=T_{2}-T_{0}+V~.
\end{equation}

这样,在变换式显含时间的条件下,广义能量函数 $H $ 并非机械能,但是具有机械能的量纲,故名之为\textbf{广义能量}。

\begin{example}{直线运动汽车里的谐振子}
在匀速直线运动的汽车中,有一谐振子在光滑水平槽中往返振动。取沿振动方向的坐标为 $q$,原点在谐振子的平衡点。选汽车为参考系,于是它是惯性系。谐振子的拉格朗日函数 $L=T-V=m \dot{q}^{2} / 2-k q^{2} / 2$。易见 $\partial L / \partial t=0$,所以 $H $ 守恒。另一方面,由于动能 $T=m \dot{q}^{2} / 2$ 是广义速度 $\dot q $ 的二次单项式,所以 $H $ 就是机械能。诚然,$ H=p\dot q-L=(\partial L / \partial \dot{q}) \dot{q}-L=m \dot{q}^{2} / 2+k q^{2} / 2=T+V$。

改取地面为参考系,这也可认为是惯性系。如果谐振子的振动方向平行于汽车运动方向,则谐振子的 $L=T-V=m\left(\dot{q}+v_{0}\right)^{2} / 2-k q^{2} / 2=m \dot{q}^{2} / 2+m v_{0} \dot{q}+m v_{0}^{2} / 2-k q^{2} / 2$,其中 $v_0$ 是汽车的速度。因 $\partial L / \partial t=0$, 所以 $H $ 守恒。但动能 $T$ 不是 $\dot q $ 的二次齐次式,所以 $H $ 并非机械能。实际上我们可以通过写出 $H$ 的表达式来说明这一点:$H=p \dot{q}-L=m\left(\dot{q}+v_{0}\right) \dot{q}-L=m \dot{q}^{2} / 2-m v_{0}^{2} / 2+k q^{2} / 2 \neq T+V$。

如果汽车并非匀速,我们考察它的匀加速运动,即速度为 $at $。仍以地面为参考系,则谐振子的 $m(\dot{q}+a t)^{2} / 2-k q^{2} / 2$,这时 $\partial L / \partial t \neq 0$, 所以 $H $ 不守恒。另一方面,$T$ 不是 $\dot q $ 的二次齐次式,所以 $H$ 也不是机械能。实际上,$H=m \dot{q}^{2} / 2-m a^{2} t^{2} / 2+k q^{2} / 2 \neq T+V$。
\end{example}
