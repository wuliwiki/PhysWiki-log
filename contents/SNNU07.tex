% 陕西师范大学 2007 年 考研 量子力学
% license Usr
% type Note

\textbf{声明}:“该内容来源于网络公开资料,不保证真实性,如有侵权请联系管理员”

\subsection{[25 分]}
\begin{enumerate}
\item 量子力学的被函数与经典的波场有柯本质性的区别?
\item 已知 $\vec L \cross \vec L = \I \hbar \vec L$,试问$\vec L_x, \vec L_y,$和 $\vec L_z$ 是不是一定不同时间测定?说明其原由或举例说明。
\item 能够用微扰法求解微观体系运动的条件是什么?
\item 对于 $He$ 原子的单重态和三重态,它们的空间波函数满足什么样的对称性?
\item 解释碱金属原子$S$谱项的精细结构是单层的原因。
\end{enumerate}
\subsection{[20分]}
一维无限深方势阱中的粒子,设初始时刻 ($t = 0$) 处于
\[
\psi(x, 0) = [\psi_1(x) + \psi_2(x)]/\sqrt{2}~
\]
$\psi_1(x)$ 与 $\psi_2(x)$ 分别为基态和第一激发态。求:
\begin{enumerate}
\item $\psi(x, t)$和$\rho(x,t) = \psi^*(x,t)\psi(x,t)$
\item 能量平均值$\overline{H} $:
\item 能量平反的平均值$\overline{H}^2 $
\item 能量的涨落$\Delta E =\overline{\left\langle \left( H - \overline{H} \right)^2\right\rangle^{1/2}}$
\end{enumerate}
\subsection{[15 分]}
对称陀螺的哈密顿是\[\hat{H} = \frac{1}{2I_1} \left( \hat{L}_x^2 + \hat{L}_y^2 \right) + \frac{1}{2I_3} \hat{L}_z^2~\]
证明其算符是它的本征函数,并写出相应的本征值。
\subsection{[15 分]}
设在 $\hat H_0$ 表象中,力的矩阵表示为:
\[
\begin{pmatrix}
E_1^0 & 0 & a \\
0 & E_2^0 & b \\
a^. & b & E_3^0
\end{pmatrix}~
\]
其中 $E_1^0 < E_2^0 < E_3^0$,使用定态微扰论求能量的二级修正值。
\subsection{[30 分]}
有一无自旋的粒子,其波函数为
\[
\psi = k(x + y + 2z)e^{-\alpha r},~
\]
其中
\[
r = \sqrt{x^2 + y^2 + z^2},~
\]
且 $k$, $\alpha$ 是实常数。问:
\begin{enumerate}
\item 粒子的总角动量是多少?
\item 角动量$z$分量的平均值是多少?
\item 若角动量的 $z$ 分量 $L_z$ 被测量,问测得 $L_z = +\hbar$ 的几率是多少?
\end{enumerate}
(已知:
\[
Y_{00} = \sqrt{\frac{1}{4\pi}}, \quad Y_{10} = \sqrt{\frac{3}{4\pi}} \cos\theta, \quad Y_{1\pm1} = \sqrt{\frac{3}{8\pi}} \sin\theta e^{\pm i\phi}~
\])
\subsection{[20 分]}
在 $\sigma_z$ 表象中,求 $\sigma_x$ 的本征态。
\subsection{[25 分]}
在 $Na$ 原子光谱中,谱线 $D_1$ 来自于第一激发态 $3^2P_{1/2}$ 到基态 $3^2S_{1/2}$ 的跃迁,其波长为 5896 埃。当 $Na$ 原子放在磁场 $B$ 中时,$D_1$ 谱线将分裂成几条谱线,试用能级跃迁图表示。设磁场 $B$ 的强度为 $2.5T$。求分裂谱线中最短与最长的两条谱线之间的波长差。