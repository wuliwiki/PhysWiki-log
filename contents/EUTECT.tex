% 共晶、共析相图

\footnote{本文参考了Callister的Material Science and Engineering An Introduction,刘智恩的《材料科学基础》与Gaskell的Introduction to the Thermodynamics of Materials}
\pentry{匀晶相图\upref{ISOMOR}}
共晶转变:由一个液相生成两个固相 $L \rightarrow \alpha+\beta$

共析转变:由一个固相生成两个固相 $\gamma \rightarrow \alpha+\beta$

共晶转变与共析转变具有很多相似之处,因此本文主要介绍共晶转变,以Pb-Sn合金的平衡冷却为例。

在热力学中,往往只关心相的变化;但由于动力学因素,实际冷却时,各相往往形成一定的有序组织结构。本文一并简要讨论。

\subsection{共晶相图}
\begin{figure}[ht]
\centering
\includegraphics[width=12cm]{./figures/EUTECT_1.png}
\caption{典型的共晶系合金相图。注意固体部分(基本)是二相混合物。引用自Callister, Material and Engineering An Introduction} \label{EUTECT_fig1}
\end{figure}
根据具体的成分不同,共晶系合金的转变过程可分为以下几类。

\subsubsection{$\omega Sn=61.9\%$,共晶合金}
\begin{figure}[ht]
\centering
\includegraphics[width=14cm]{./figures/EUTECT_2.png}
\caption{共晶合金. 引用自Callister, Material and Engineering An Introduction} \label{EUTECT_fig2}
\end{figure}

共晶合金的成分正好等于共晶点(i点)所对应的成分 ($\omega_{Sn}=61.9\%$)。

全程的相转变:$L \rightarrow \alpha+\beta$

全程的组织转变:$L \rightarrow (\alpha+\beta)_{eutectic}$

i点处,发生共晶转变,由一个液相生成两个固相,并形成一定组织结构 $L \rightarrow (\alpha+\beta)_{eutectic}$
\begin{itemize}
\item 共晶转变是恒成分转变,即共晶转变全程中,先后结晶部分的成分一致。例如,共晶转变全程中,α相中Sn浓度始终为$18.3\%$。
\item 共晶转变是恒温转变: 相转变时,系统自由度f=2-3+1=0,因此相转变温度是定值,在相图上体现为三相区是直线。
\end{itemize}

i点以下,发生脱熔转变 $\alpha \rightarrow \beta_{II}, \beta \rightarrow \alpha_{II}$
\begin{itemize}
\item 随温度降低,α相溶解Sn、β相溶解Pb的能力均减弱。Sn将以β相固溶体的形式从α中析出,而Pb将以α相固溶体的形式从β中析出,称为二次相 $\alpha_{II},\beta_{II} $。
\item 脱熔转变生成的$\alpha_{II},\beta_{II} $结构与性质与$\alpha, \beta$完全相同。
\item 但是,αII,βII难以与共晶组织中α,β区分,可以不标出
\end{itemize}

\subsubsection{$18.3\%<\omega Sn<61.9\%$,亚共晶合金}
\begin{figure}[ht]
\centering
\includegraphics[width=14cm]{./figures/EUTECT_3.png}
\caption{亚共晶合金. 引用自Callister, Material and Engineering An Introduction} \label{EUTECT_fig3}
\end{figure}

亚共晶合金中,Sn的浓度小于共晶点所对应的成分,但大于Sn在$\alpha$中的最大固溶度 ($18.3\%<\omega_{Sn}<61.9\%$)。

全程的相转变:$L \rightarrow \alpha+\beta$

全程的组织转变:$L \rightarrow \alpha_{primary}+\beta_{II}+(\alpha+\beta)_{eutectic}$

k点处,部分液体先发生匀晶转变生成α相,此部分先生成的α相称为初生α相$L \rightarrow \alpha_{primary}$。α相中Pb的浓度高于液相浓度;由于Pb进入了初生α相,剩余液体中Pb浓度下降,Sn浓度上升;随着α相不断生成,液体的成分逐渐与共晶成分相同。

l-m点处,剩余液体的发生共晶转变并生成共晶组织 $L \rightarrow (\alpha+\beta)_{eutectic}$

m点以下,发生脱溶转变$\alpha \rightarrow \beta_{II}, \beta \rightarrow \alpha_{II}$。只有在α相中的βII脱熔相容易被观察到。

\subsubsection{$61.9\%<\omega Sn<97.8\%$,过共晶合金}
过共晶合金的成分大于共晶点所对应的成分,但小于Pb在$\beta$中的最大固溶度 ($61.9\%<\omega_{Sn}<97.8\%$)。

过共晶合金的转变过程与亚共晶合金类似。

\subsubsection{$\omega Sn<18.3\%$或$\omega Sn>97.8\%$,端部固溶体}
\begin{figure}[ht]
\centering
\includegraphics[width=10cm]{./figures/EUTECT_4.png}
\caption{端部固溶体. 引用自Callister, Material and Engineering An Introduction} \label{EUTECT_fig4}
\end{figure}
端部固溶体中,Sn的浓度小于Sn在$\alpha$中的最大固溶度($\omega_{Sn}<18.3\%$),或大于Pb在$\beta$中的最大固溶度($\omega_{Sn}>97.8\%$)。由于二者类似,此处只讨论前者。

全程的相转变:$L \rightarrow \alpha+\beta$

全程的组织转变:$L \rightarrow \alpha+ \beta_{II}$

端部固溶体的转变比较简单,全程不涉及共晶转变。

e点附近,先发生匀晶转变$L \rightarrow \alpha$。在f点附近,系统完全由α组成。

g点附近,发生脱熔转变$\alpha \rightarrow \beta_{II}$

\subsection{热力学}
\begin{figure}[ht]
\centering
\includegraphics[width=10cm]{./figures/EUTECT_5.png}
\caption{简化的共晶相图} \label{EUTECT_fig5}
\end{figure}
假设Sn完全不溶解于$\alpha$,Pb完全不溶解于$\beta$,即固相完全是Sn与Pb的机械混合物。当处于共晶点时(成分等于共晶成分,温度等于共晶点温度),固、液相中Sn,Pb的化学势分别相同。
\begin{align}
&\mu_{Pb,s}^*=\mu_{Pb,l,mixed}\\
&\mu_{Sn,s}^*=\mu_{Sn,l,mixed}\\
\end{align}
对于Pb,即
$$\mu_{Pb,s}^*=\mu_{Pb,l}^*+RT \ln x_{Pb,l}=\mu_{Pb,s}^*+\Delta G_{Pb, s\rightarrow l}+RT \ln x_{Pb,l}$$
即
\begin{equation}
x_{Pb,l}=e^{-\frac{\Delta G_{Pb, s\rightarrow l}}{RT}}
\end{equation}
同理
\begin{equation}
x_{Sn,l}=e^{-\frac{\Delta G_{Sn, s\rightarrow l}}{RT}}
\end{equation}
再加上条件
\begin{equation}
x_{Sn,l}+x_{Pb,l}=1
\end{equation}
原则上可解得共晶成分与温度。
