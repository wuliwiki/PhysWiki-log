% 闵可夫斯基时空中的能动张量

\pentry{动力学假设\upref{SRDyn},张量\upref{Tensor}}



\subsection{均匀等速尘埃云的动量通量}

\subsubsection{均匀等速尘埃云}

在任意参考系中,空间中分布若干质点.这些质点的集合,被称为一片\textbf{尘埃云(dust)},各质点被称为\textbf{尘埃粒子}.如果在某个参考系中,一片尘埃云的各质点都保持静止,那么我们称这个参考系是尘埃云的\textbf{自身系},称尘埃云为\textbf{等速尘埃云},因为这意味着在其它参考系中,尘埃粒子的速度都会是相同的.如果在某个参考系中,尘埃粒子的质量相同、在空间中均匀分布,那么我们称这片尘埃云是\textbf{均匀的}.

任何一片\textbf{均匀尘埃云},都可以看成是许多\textbf{均匀等速尘埃云}的叠加,只要把属于各个速度的尘埃粒子分别拿出来构成尘埃云即可.而任何一片尘埃云,也可以看成是局部均匀的.因此,研究均匀等速尘埃云的性质最为容易,也可以方便地拓展到任意尘埃云的性质中.

\subsubsection{尘埃云数量通量密度}

假设空间中有一片均匀等速尘埃云,在其自身系中各点的粒子数量密度都是 $n$,其中 $n$ 是一个实数.也就是说,在尘埃云的自身系中,在任何体积 $V$ 中,尘埃粒子的数量都是 $nV$.

取 $K_1$ 参考系作为观察者,设观察者认为尘埃云的速度是 $\bvec{v}=(v_x, v_y, v_z)^T$.由于尺缩效应,在 $K_1$ 中,空间各点尘埃粒子的数量密度变为 $n/\sqrt{1-v^2}$.

\begin{figure}[ht]
\centering
\includegraphics[width=10cm]{./figures/SRFld_1.pdf}
\caption{粒子数量通量示意图.绿色面表示计算粒子数量通量的参考平面 $A_{yz}$;一个粒子在单位时间后通过这个平面,当且仅当这个粒子现在正在图示体积内.该体积的四条红色边和尘埃云的速度方向平行,而左右两个平面之间的距离是 $v_x$,故体积为 $A_{yz}\cdot v_x$.} \label{SRFld_fig1}
\end{figure}

固定 $x$ 坐标,取此处 $y-z$ 平面上的一个单位面积 $A_{yz}$.如所示,在单位时间内,通过 $A_{yz}$ 的是体积 $A_{yz}\cdot v_x=v_x$ 内的粒子,数量是
\begin{equation}
\frac{n}{\sqrt{1-v^2}}\cdot v_x=\frac{nv_x}{\sqrt{1-v^2}}
\end{equation}

类似地,用 $i, j, k$ 代表任意的字母 $x, y, z$,那么对于固定的 $i$ 坐标,单位时间内通过 $j-k$ 平面上的单位面积的粒子数量是 $nv_i/\sqrt{1-v^2}$.通过一个平面的尘埃粒子数量,称为尘埃云通过这个平面的数量通量;而当所通过平面的面积是单位面积时,数量通量也可以称为数量密度的通量,或者数量通量的密度.

总结下来,如果取所考察平面的面积向量为 $\bvec{S}$,那么对于上述均匀等速尘埃云,单位时间内通过这个平面的粒子数量是 $\frac{n}{\sqrt{1-v^2}}\abs{\bvec{v}\cdot\bvec{S}}$.

\subsubsection{尘埃云动量通量密度}

尘埃云设定同上,并假设每个尘埃云粒子的质量是 $m$,那么它在 $K_1$ 中的动量是 $m/\sqrt{1-v^2}$.结合数量通量密度可知,在单位时间内通过面积 $\bvec{S}$ 的尘埃云动量是
\begin{equation}
\frac{n}{\sqrt{1-v^2}}\abs{\bvec{v}\cdot\bvec{S}}\cdot\frac{m}{\sqrt{1-v^2}}\bvec{v}=\frac{nm\bvec{v}}{1-v^2}\abs{\bvec{v}\cdot\bvec{S}}=\frac{nm}{1-v^2}\abs{\bvec{v}\cdot\bvec{S}}\cdot\pmat{v_x\\v_y\\v_z}
\end{equation}

固定 $x$ 坐标时,单位时间内通过单位面积的尘埃云动量的 $y$ 分量就是
\begin{equation}
\frac{nm}{1-v^2}v_xv_y
\end{equation}

一般地,固定 $i$ 坐标时,单位时间内通过单位面积的尘埃云动量的 $j$ 分量就是
\begin{equation}\label{SRFld_eq1}
\frac{nm}{1-v^2}v_iv_j
\end{equation}

\subsection{四动量通量密度和能量-动量张量}

上面所讨论的动量通量密度,在四维闵可夫斯基空间中可以表示为穿过单位时间、单位 $A_{yz}$ 面积的\textbf{三维“平面”}\footnote{多维几何学中,将 $n$ 维空间中的一个 $n-1$ 维子空间称为一个\textbf{超平面(hypersurface)},因此这里的三维平面,指的是四维空间中的三维超平面.}的世界线数量乘以各世界线代表的动量.

\autoref{SRFld_fig2} 是四维闵可夫斯基空间中的动量通量示意图,绿色的垂直线段表示单位时间长度里的单位面积 $A_{yz}$,相当于用一维线段表示了一个三维的体积;若干红色线段表示各尘埃粒子的世界线.动量通量密度,就是通过单位时间单位面积的世界线数量乘以各粒子的动量\footnote{这里有一个视错觉现象,即绿色线段看起来不是垂直的,而是微微倒向左边.}.

\begin{figure}[ht]
\centering
\includegraphics[width=6cm]{./figures/SRFld_2.pdf}
\caption{四维闵可夫斯基空间中的动量通量} \label{SRFld_fig2}
\end{figure}

既然都用到四维表示了,我们也可以把结论拓展到四动量,以及固定时间的情况.

\begin{figure}[ht]
\centering
\includegraphics[width=6cm]{./figures/SRFld_3.pdf}
\caption{固定时间时,穿过单位三维空间超平面的世界线示意图.} \label{SRFld_fig3}
\end{figure}

如果设各尘埃粒子的四速度是 $\bvec{U}=1/\sqrt{1-v^2}(1, v_x, v_y, v_z)^T$,可以定义一个矩阵 $\bvec{M}$,其第 $i$ 行 $j$ 列的元素代表“固定 $i$ 坐标时,单位时间内通过单位面积的尘埃云动量的 $j$ 分量”,那么就有

\begin{equation}\label{SRFld_eq2}
\bvec{M}=\frac{nm}{1-v^2}\pmat{1&v_x&v_y&v_z\\v_x&v_x^2&v_xv_y&v_xv_z\\v_y&v_xv_y&v_y^2&v_yv_z\\v_z&v_xv_z&v_yv_z&v_z^2}=nm\cdot\bvec{U}\bvec{U}^T
\end{equation}

这里的 $\bvec{U}$ 是一个列矩阵,也就是说,$\bvec{M}$ 是向量 $n\bvec{U}$ 和 $m\bvec{U}$ 的张量积.$\bvec{M}$ 因而是一个二阶张量,被称作\textbf{能量-动量张量}.


\subsection{一般尘埃云的能动张量}

均匀等速尘埃云是一种高度理想化的流体模型,现实中并不会存在.这个模型,如果在每个点处的质心系中观察,那么所有物质都是绝对静止的;现实中的流体模型,哪怕在每一个点处的质心系来观察,物质仍然有非零的速度分布.事实上,宏观情况下我们对流体的能动张量释义会有一些不同.

在时空流形上,固定时间时计算的四动量通量,也就是\autoref{SRFld_eq2} 中的第一列,恰是流体在这一时刻的四动量空间密度;固定某一空间面时计算的四动量通量,则是单位时间内四动量通过这个面的量.动量随时间的变化率,就是“力”,这是牛顿第二定律给出的定义.因此,我们推广时换用力的概念来定义能动张量.

由流体力学可知,流体作为一个宏观的统计模型,忽略了流体粒子之间的相互作用,只考虑由大量粒子构成的“流体元”之间的宏观相互作用.流体元之间可能有垂直接触面的压力,也可能有因摩擦产生的、平行于接触面的粘滞力.微观来看,两个流体元接触面上,从流体元 $A$ 流向流体元 $B$ 的三动量变化率,就是 $A$ 对 $B$ 的作用力;而牛顿第三定律就体现在,反过来从 $B$ 流向 $A$ 的三动量,一定时刻与之相反.

这就是均匀等速尘埃云所无法描述的现象,因为如果非要认为均匀等速尘埃云是非静止的流体,那么在任何一个面上,最多只会有单向流动的动量,也就是说没有反作用力.因此均匀等速尘埃云最多只能被认为是静止的流体.我们在计算均匀等速尘埃云的四动量通量时,并没有规定流出或流入流体元这两个方向上的动量交换,保留哪一个舍弃哪一个,而是都纳入计算,于是也无法体现出施力物体和受力物体的区别.因此在推广时,我们需要区分清楚流入流出方向,并分开计算.对于一个流体元,计算流入它的四动量通量,所得的是它受到的闵可夫斯基力;而流出它的通量,则是它施加于其它物体的力.

\begin{definition}{}
考虑一团实际的流体,对于每个时空点 $p$,取此处流体的局部质心系\footnote{即取包含此时空点 $p$ 的体积,计算体积中质心的轨迹,然后取体积趋于零的极限,所得的极限轨迹就是局部质心.},那么流体在该处的四动量通量就由一个嵌套矩阵 $T_{ab}$ 描述,其中各 $T_{ab}$ 是指在该处的一个小流体元在固定 $a$ 坐标的面上\textbf{受}闵可夫斯基力的 $b$ 分量.
\end{definition}


为了方便建立广义相对论的基础,我们不考虑太复杂的模型,而是将物质都理解为以下“没有粘滞力”的理想流体.


\begin{definition}{理想流体}
对于一个流体,如果它在任何时空点处,任何表面上所受的力都是垂直于该表面的,则称此流体为\textbf{理想流体(ideal fluid)}.
\end{definition}

这样一来,容易证明理想流体任何一个点上的 $T_{ab}$ 就可以写成:
\begin{equation}\label{SRFld_eq3}
T_{ab}=\pmat{\pmat{\rho&0&0&0}\\
\pmat{0&p&0&0}\\
\pmat{0&0&p&0}\\
\pmat{0&0&0&p}}\Tr
\end{equation}

其中 $\rho$ 是能量在空间中的密度,$p$ 就是流体在该处受到的压强.

但是这个嵌套矩阵只是局部质心系里描述流体受力,我们希望摆脱具体参考系的约束,抽象地在时空流形上讨论受力、能量和动量等概念,因此我们使用已知的摆脱了具体参考系约束的量(也就是张量场)来描述这个嵌套矩阵:
\begin{equation}\label{SRFld_eq4}
T_{ab}=\rho U_aU_b+P g_{ab}+P U_aU_b\\
\end{equation}

其中 $g_{ab}$ 是时空的度量张量场,$U_a$ 是局部质心的四速度的对偶,即 $U_a=U^ig_{ia}$.在局部质心系中,$g_{ab}=\opn{diag}(-1, 1, 1, 1)$,而 $U^{a}=(1, 0, 0 ,0)\Tr$,因此在局部坐标系中\autoref{SRFld_eq3} 和\autoref{SRFld_eq4} 没有冲突,而\autoref{SRFld_eq4} 则摆脱了坐标系限制.

能动张量能如何描述物质分布的性质呢?

\begin{definition}{能量密度}
令 $T_{ab}$ 为一理想流体在时空流形上的能动张量场,$u^a$ 是该流体在各时空点处局部质心系的四速度,则 $T_{ab}u^au^b$ 是该流体在该处的\textbf{能量密度}.
\end{definition}

这一定义是合理的.考虑到局部质心系中 $u^a=\pmat{1&0&0&0}\Tr$,根据理想流体的能动张量的定义,可以计算出 $T_{ab}u^au^b$ 就是能量密度分布,而这是张量场的抽象指标分布,不依赖于参考系的选择,所以是合理的推广表达.

\subsection{守恒定律}

理想流体的能动张量,满足方程
\begin{equation}\label{SRFld_eq5}
\partial^aT_{ab}=0
\end{equation}

\autoref{SRFld_eq5} 的理解可以类比向量场的散度 $\partial_av^a$.如果向量场 $v^a$ 描述了某种东西的流,比如说流体的质量流,那么 $\partial_av^a$ 就是流体质量的流失速率,在 $\partial_av^a=0$ 的地方自然就有质量密度守恒.

根据指标的升降法则,对于任何张量 $\bvec{T}$,我们定义 $\partial^a\bvec{T}=\eta^{ab}\partial_a\bvec{T}$,这样子我们就可以把\autoref{SRFld_eq5} 理解为 $\eta^{ai}\partial_i(T_b)^a$,相当于用余切向量场 $T_b$ 取代了光滑函数 $v$,形式不变,同样表达了某种量的守恒.

什么东西守恒呢?

利用\autoref{SRFld_eq4},$T_{ab}=\rho u_au_b+P g_{ab}+P u_au_b$,把\autoref{SRFld_eq1} 展开,我们能得到:
\begin{equation}\label{SRFld_eq6}
\begin{aligned}
0=\partial^aT_{ab}=&\partial^a(\rho u_au_b+P g_{ab}+P u_au_b)\\
=&u_au_b\partial^a\rho+\rho u_b\partial^au_a+\rho u_a\partial^au_b+\\
&g_{ab}\partial^ag_{ab}+P\partial^ag_{ab}+\\
&u_au_b\partial^aP+P u_b\partial^au_a+P u_a\partial^au_b
\end{aligned}
\end{equation}

\autoref{SRFld_eq6} 最右边很长,我们可以把它分成两部分,分别是在度量 $\eta_{ab}$ 下垂直和平行于 $u^b$ 的项.显然,其中的 $u_au_b\partial^a\rho+\rho u_b\partial^au_a+P u_b\partial^au_a$ 是平行于 $u^b$ 的,因为它们都是在 $u_b$ 前面乘以一个光滑函数的形式;我们断言,剩下的部分,是垂直于 $u^b$ 的,即与 $u^b$ 的内积为 $0$.

这是因为,剩下的那部分还可以再分成两部分,一个是 $(P+\rho)u_a\partial^au_b$,一个是 $(\eta_{ab}+u_au_b)\partial^aP$.考虑到 $u^b$ 按定义必须是单位向量,因此 $\partial_au^b$ 必定与 $u^a$ 垂直,即 $u^a\partial_au^b=0$,因此 $u^b(P+\rho)u_a\partial^au_b=0$;再由洛伦兹度规的定义,$u^au_b=1$,因此 $u^b(\eta_{ab}+u_au_b)=u_a-u_a=0$.综上,剩下这部分与 $u^b$ 的乘积就是 $0$\footnote{特别要注意的是 $u^b(\eta_{ab}+u_au_b)$ 这一部分,如果取 $\eta_{ab}=\opn{diag}\pmat{1&-1&-1&-1}$,那它就不再垂直于 $u^b$ 了.}.

将\autoref{SRFld_eq6} 最右边按上述讨论分成垂直和平行于 $u^b$ 的两部分后,就分别得到了两个独立的等式(适当进行指标升降以获得更直观的表达):

\begin{equation}\label{SRFld_eq7}
u^a\partial_a\rho+(\rho+P)\partial^au_a=0
\end{equation}

\begin{equation}\label{SRFld_eq8}
(P+\rho)u^a\partial_au_b+(\eta_{ab}+u_au_b)\partial^aP=0
\end{equation}

回归经典极限时,三速度 $\bvec{v}$ 很小,以至于 $u^\mu=\pmat{1, \bvec{v}}$,而流体之间的压强也远小于密度,那么\autoref{SRFld_eq7} 和\autoref{SRFld_eq8} 就分别变成

\begin{equation}\label{SRFld_eq9}
\frac{\partial}{\partial t}\rho+\Nabla\cdot(\rho\bvec{v})=0
\end{equation}

\begin{equation}\label{SRFld_eq10}
\rho(\frac{\partial}{\partial t}\bvec{v}+(\bvec{v}\cdot\Nabla)\bvec{v})=-\Nabla P
\end{equation}

\autoref{SRFld_eq9} 意味着任意点处质量密度 $\rho$ 的流失速率等于 $\rho\bvec{v}$ 的散度,即质量流就是 $\rho\bvec{v}$,因此是\textbf{质量守恒}的表达式.

\autoref{SRFld_eq10} 是流体的欧拉方程\footnote{见\textbf{流体运动的描述方法}\upref{fluid1}中的欧拉法.}.

\autoref{SRFld_eq9} 和\autoref{SRFld_eq10} 都是取经典极限进行的讨论,说明了\autoref{SRFld_eq5} 是经典力学的合理推广.对于一般的情况,$\partial^aT_{ab}=0$ 可以看成是描述了\textbf{四个}分量的守恒定律,即固定 $b$ 指标来考虑,一共有四种固定方法.对于固定的 $b$ 指标,$T_{ab}$ 描述的是流体在 $a$ 面上单位面积受力的 $b$ 分量,因此\autoref{SRFld_eq5} 描述了流体四动量的 $b$ 分量的守恒律.那么综合来看,该式描述的是四动量的守恒定律,即相对论动力学假设.







