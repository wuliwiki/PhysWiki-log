% 北京大学 2017 年 考研 量子力学
% license Usr
% type Note

\textbf{声明}:“该内容来源于网络公开资料,不保证真实性,如有侵权请联系管理员”

\subsection{一}
1.宽度为 $2L$ 的无限深势阱,范围为 $-L < x < L$,求能量本征态和相应的本征值。

2.已知 $t = 0$ 时处于基态,势阱宽度突然变为 $4L$,范围为 $-2L < x < 2L$,求随时间变化的波函数表达式 $\varphi(t)$,求处于变化后体系本征态能量的概率,求体系的能量平均值 $\overline{E(t)}$。
\subsection{二}
一个二维谐振子,哈密顿量为
$$\hat{H} = \frac{p_x^2}{2m} + \frac{1}{2} m \omega_x x^2 + \frac{p_y^2}{2m} + \frac{1}{2} m \omega_y y^2~$$

1.若 $\frac{\omega_x}{\omega_y} = \frac{3}{4}$,求第一个和第二个简并能级。

2.若 $\omega_x = \omega_y$,有两个自旋为 $\frac{1}{2}$ 的全同粒子处于此谐振子势场中,求体系最低三个能级,并给出简并度。
\subsection{三}
1.在 $\sigma_z$ 表象下,$\sigma_x$、$\sigma_y$、$\sigma_z$ 的矩阵表示及其归一化本征态、本征值。

2.在 $\sigma_x$ 表象下,$\sigma_y$ 和 $\sigma_z$ 的归一化本征态。
\subsection{四}
体系哈密顿量为
$\hat{H} = \hbar( |1\rangle \langle 2| + |2\rangle \langle 1| )$
其中 $|1\rangle$ 和 $|2\rangle$ 为体系两个正交归一的本征态。在 $t=0$ 时,算符
$\hat{O} = 3 |1\rangle \langle 1| - |2\rangle \langle 2|$
的平均值为 -1,求体系初始态及 $t > 0$ 时体系最初转化为 $|1\rangle$ 的时间。
\subsection{五}
考虑氢原子核不是点电荷,而是均匀带电的球体

1.体系能量基态本征函数为 $\psi = N e^{-\frac{r}{a}}$,求归一化系数 $N$ 和不确定度 $(\Delta \vec x)^2$ 和 $(\Delta r)^2$。

2.用微扰法求出这种改动对氢原子能量基态能量的一级修正。
\subsection{六}
一维粒子由右边入射,收到的势能为
$$V(x) = \begin{cases}
\infty & x < 0 \\
-\frac{gh^2}{2m} \delta(x-a) & x > 0
\end{cases}~$$
1.若能量 $E > 0$,求反射波与入射波之间的相位差,及相位差在低能和高能下的表现。

2.求存在束缚态得条件。