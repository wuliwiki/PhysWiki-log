% 麦克斯韦—玻尔兹曼分布的数值模拟

\begin{issues}
\issueDraft
\end{issues}

\pentry{二体碰撞\upref{TwoCld}}

最简单算法: 定时间步长, 若发现某两个小球重合,或与墙重合, 则就地完全弹性碰撞。

一些变体
\begin{itemize}
\item 支持三维
\item 支持重力
\item 不同形状的容器
\item 支持质量和大小不同的若干种小球
\item 真随机碰撞(似乎已完成): 随机找出一对对小球进行碰撞, 无论他们在哪里
\end{itemize}

一些优化
\begin{itemize}
\item (先不做)把盒子形划分成小盒子, 每一步把小球先按照小盒子分类, 然后再在每个盒子中判断两小球是否碰撞。 若一个小球出现在边角处, 则它同时属于多个小盒子。
\item (先不做)每一步找出速度最快的小球, 确保时间步长小于某长度除以该速度。
\end{itemize}

\begin{lstlisting}[language=matlab]
% 两个粒子的完全弹性碰撞
% size(v1)=size(v2)=size(m1)=size(2)=[N,3];
% A 是碰撞平面的法向量
% 当不输入 m1, m2 的时候, 默认他们相等.
function [v11, v22] = collision(v1, v2, A, m1, m2)
A = A/norm(A);
% 质心的速度
vc = (m1*v1+m2*v2)/(m1+m2);
% 质心系中的速度
vr1 = v1-vc;
vr2 = v2-vc;
% 计算关于碰撞平面的反射
vr11 = reflection(vr1,A);
vr22 = reflection(vr2,A);
% 原参考系中的速度
v11 = vr11+vc;
v22 = vr22+vc;
end
\end{lstlisting}

\begin{lstlisting}[language=matlab]
% 反射
% A 是法向量. 
% size(v1) = [N,3];
% v1 是入射的方向, v2 是反射的方向
function v2 = reflection(v1,A)
vnorm = sqrt(v1' * v1);
v1unit = v1 / vnorm;
cosine = v1unit' * A;
v2 = v1 - (2*vnorm*cosine)*A;
end
\end{lstlisting}

\begin{lstlisting}[language=matlab]
%麦克斯韦-波尔兹曼分布

% === 参数设置 ====
N = 100; % 总粒子数
m = 1e-3; % 粒子质量
R = 1e-3; % 粒子半径

% 容器规格
Size = [1,1,1];% [x长度, y长度, z长度].0<x<x长度, y...,z....
tmax = 10; Nt = 1000; % 模拟总时间
T = 10; % 温度
kb = 1.380650324; % 波尔兹曼常量

% 每个粒子的位置矢量
for i = 3:-1:1
    P(:,i) = rand(N,1)*Size(i);
end
% 每个粒子的速度矢量
V = rand(N,3);
V = vunit(V)*sqrt(3*kb*T/m); % 每个粒子都要有平均动能
v = vmag(V); % 速度绝对值

t = linspace(0, tmax, Nt); dt = t(2)-t(1);
for it = 1:Nt
    P = P + V*dt;
    [P, V] = do_reflection(P, V, R, xmax, ymax);
    [P, V] = do_collision(P, V, R);
    figure(1); scatter(P(:,1), P(:,2), P(:,3));
end
\end{lstlisting}
