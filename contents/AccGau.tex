% 加速度规范
% 速度规范|平移算符|Kramers-Henneberger|K-H 变换|规范变换|矢势|电磁波

\pentry{速度规范\upref{LVgaug}, 平移算符\upref{tranOp}}

本文使用原子单位制\upref{AU}。 首先注意\textbf{加速度规范(acceleration gauge)}并不是一种规范而只是薛定谔方程的一种变换, 说它是规范只是习惯上的叫法。 该变换也叫做 \textbf{Kramers-Henneberger 变换} 或 \textbf{K-H 变换}。

在速度规范\upref{LVgaug}下, 电场和矢势满足和库仑规范同样的关系(\autoref{eq_Cgauge_2}~\upref{Cgauge})
\begin{equation}\label{eq_AccGau_5}
\bvec {\mathcal{E}}(t) = -\pdv{\bvec A}{t}~.
\end{equation}
注意速度规范默认偶极子近似\upref{DipApr}, 即空间中电场和矢势都与位置无关。 一个电荷为 $q$ 的粒子在电磁波到来之前处于静止, 那么接下来它在电磁波作用下的位移为
\begin{equation}\label{eq_AccGau_3}
\bvec \alpha(t) = -\frac{q}{m}\int_{-\infty}^t \bvec A(t') \dd{t'}~.
\end{equation}
波函数变换是一个位移为 $\bvec \alpha$ 的平移
\begin{equation}\label{eq_AccGau_1}
\Psi_V(\bvec r, t) = \Psi_A(\bvec r - \bvec \alpha, t)~.
\end{equation}
也可以用平移算符\upref{tranOp}记为
\begin{equation}\label{eq_AccGau_2}
\Psi_V(\bvec r, t) = \E^{-\I \bvec \alpha \vdot \bvec p} \Psi_A(\bvec r, t)~.
\end{equation}
这相当于一个参考系变换, 我们把 $\Psi_A$ 所在的参考系叫做 \textbf{K-H 参考系(K-H frame)}, 是一个非惯性系。 注意严格来说\autoref{eq_AccGau_2} 要求波函数在整个空间无穷阶可导, 而\autoref{eq_AccGau_1} 却不用, 但习惯上我们只是把平移算符看成平移的一种记号, 并不要求波函数无穷阶可导。

为什么说 K-H 变换不是一个规范变换? 因为如果我们如果试图找到\autoref{eq_QMEM_3}~\upref{QMEM}中的 $\chi$, 会发现 $\chi = -\bvec \alpha \vdot \bvec p/q$, 而这是一个微分算符, 不是位置和时间的函数。

可以证明 K-H 系中, 哈密顿量变为
\begin{equation}
H_A = \frac{\bvec p^2}{2m} + V[\bvec r + \bvec \alpha(t)]~,
\end{equation}
其中 $V(\bvec r)$ 是不含时的势能函数(例如原子核对电子的库仑势能)。 我们可以把加速度规范想象成在 K-H 非惯性系中使用速度规范, 仍有
\begin{equation}
\bvec p = m\bvec v + q\bvec A = -\I \grad~,
\end{equation}
但这里的 $\bvec v$ 是相对于 K-H 系的。

注意这里不存在表示电场力的项, 我们可以理解为 K-H 参考系中的惯性力\upref{Iner}与电场力抵消了。

对应的薛定谔方程仍然是
\begin{equation}\label{eq_AccGau_4}
H_A \Psi_A = \I \pdv{t} \Psi_A~.
\end{equation}


加速度规范的一个简单应用见 Volkov 波函数\upref{Volkov}。
