% 平均值的不确定度
% keys 概率|概率分布|平均值|方差|标准差|不确定度|概率密度|数学期望|多元微积分|重积分

\begin{issues}
\issueDraft
\issueMissDepend
\end{issues}

\pentry{随机变量、概率密度函数\upref{RandF}}

例题: 比如说有一个概率分布 $y = f(x)$。  它的

平均值(数学期望)为 $\bar x = \int_{-\infty}^{+\infty} xf(x) \dd{x}$,  

方差为  $\sigma_x^2 = \int_{-\infty}^{+\infty} (x - \bar x)^2 f(x) \dd{x} = \qty(\int_{-\infty}^{+\infty} x^2 f(x) \dd{x} ) - \bar x^2 = \overline{x^2}  - \bar x^2~,$ 

标准差为 $\sigma_x = \sqrt{\int_{-\infty}^{+\infty} (x - \bar x)^2 f(x) \dd{x}} $。  

如果测量一个数据, 这三个值就可以用来衡量这个数据的特征。

但如果测量 $n$ 次平均值, 那显然平均值显然要比一次测量更可靠, ${\sigma_{\bar x}} < {\sigma_x}$。   各种教科书上都会给出 ${\sigma_{\bar x}} = \frac{1}{\sqrt n }{\sigma_x}$ 或者 $\sigma_{\bar x}^2 = \frac{1}{n}\sigma_x^2$。  那么这个公式到底怎么来的呢?

其实在上式中, $\sigma_{\bar x}^2$  的定义是
\begin{equation}
\sigma_{\bar x}^2 = \int_{-\infty}^{+\infty} \qty( \frac{1}{n}\sum_{i=1}^n x_i - \bar x)^2 f(x_1) f(x_2)\dots f(x_n) \dd{x_1}\dots \dd{x_n}~.
\end{equation}
下面就从这个定义证明 $\sigma_{\bar x}^2 = \frac{1}{n}\sigma_x^2$。 

先考虑两次测量, 即 $n = 2$ 的情况。 先后得到 $x_1, x_2$ 的概率密度是 $f_2(x_1, x_2) = f(x_1) f(x_2)$ 。 不难证明归一化:
\begin{equation}
\iint f(x_1) f(x_2) \dd{x_1} \dd{x_2} = \int f(x_1) \dd{x_1} \int f(x_2) \dd{x_2}  = 1 \times 1 = 1 ~,
\end{equation}

先看 $(x_1 + x_2)/2$ 的平均值, 令 $y = (x_1 + x_2)/2$。 
\begin{equation}\ali{
\bar y &= \iint \frac{x_1+x_2}{2} f(x_1) f(x_2) \dd{x_1} \dd{x_2}  \\
& = \frac12 \int {{x_1} f(x_1) \dd{x_1}} \int f(x_2) \dd{x_2}  + \frac12 \int f(x_1) \dd{x_1} \int x_2 f(x_2) \dd{x_2}\\
&= \frac{\bar x}{2} + \frac{\bar x}{2} = \bar x~.
}\end{equation}
结论是, 进行两次测量取平均值, 数学期望就是测量一次的数学期望。 这个结论是符合常识的。

根据同样的方法, 可以测量方差。
\begin{equation}\ali{
\sigma_{\bar x}^2 &= \iint \qty(\frac{x_1+x_2}{2} - \bar x)^2 f(x_1) f(x_2) \dd{x_1}\dd{x_2}\\
&= \frac12 \qty(\overline{x^2}  - \bar x^2) = \frac12 \sigma_x^2~,
}\end{equation}
所以 $\sigma_{\bar x}^2 = \frac12 \sigma_x^2$,  且  $\sigma_{\bar x} = \frac{1}{\sqrt 2 }\sigma_x~.$ 

对于 $n > 2$ 的情况, 利用求和符号和积分运算法则, 也很容易证明  $\sigma_{\bar x} = \frac{1}{\sqrt n} \sigma_x~.$ 

当 $N$ 很大时, 另一种证明方法是使用中心极限定理\upref{CLT}。
