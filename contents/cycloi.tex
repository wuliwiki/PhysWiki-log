% 摆线

\begin{issues}
\issueDraft
\end{issues}

其中 $a$ 是大于零的参数。 令方括号部分为 $\theta$, 该方程能写成 $\theta$ 的参数方程, 该曲线被称为\textbf{摆线(cycloid)}或\textbf{滚轮线}。 % 链接未完成
\begin{equation}
\leftgroup{
x &= a(\theta - \sin\theta)\\
y &= a(1 - \cos\theta)
}\end{equation}
\textbf{公式推导}
\begin{figure}[ht]
\centering
\includegraphics[width=10cm]{./figures/97cb21e45dd29d49.pdf}
\caption{摆线} \label{fig_cycloi_1}
\end{figure}
如\autoref{fig_cycloi_1} ,直角坐标系 $xOy$ 中,半径为 $a$ 的圆 $B$ 沿 $x$ 轴做无滑滚动,该圆与 $x$ 轴相切与 $A$ 点,圆上一动点 $M$ 在 $x$ 轴上投影为点 $D$ , 点 $C$ 为 $M$ 在线 $AB$ 上的垂点, 动点 $M$ 初始位置在坐标原点 $O$ , 其运动轨迹便是摆线, $\angle ABM=\theta$ ,设动点 $M$ 坐标为 $(x,y)$ ,则
\begin{equation}
\begin{aligned}
&OA=a\theta \\
&AD=a\sin\theta\\
&BC=a\cos(\pi-\theta)=-a \cos\theta
\end{aligned}
\end{equation}
由几何关系可知
\begin{equation}
\begin{aligned}
&x=OA-AD=a\theta-a\sin\theta=a(\theta-\sin\theta)\\
&y=a+BC=a-a\cos\theta=a(1-\cos\theta)
\end{aligned}
\end{equation}
这便是摆线的参数方程。
\addTODO{ 推导公式}


从 $\theta = 0$ 开始的任意一段摆线都是连接这两点的最速降线\upref{Brachi}。 
