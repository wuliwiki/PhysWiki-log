% 去噪扩散概率模型
% 扩散 马尔科夫

\textbf{去噪扩散概率模型}(Denoising Diffusion Probabilistic Models, DDPM)是一种参数化的马尔科夫链,通过变分推理的方法来训练。去噪扩散概率模型(后文简称扩散模型)是深度生成模型的一种,通常包含两个过程:第一是前向扩散过程,第二是反向的逆扩散过程。正反两个方向的马尔科夫链均由有限个时间步组成。其中,前向扩散过程就是一个无参数的马尔科夫链,而反向的逆扩散过程须要学习算法来训练模型。模型结构如图1所示。
\begin{figure}[ht]
\centering
\includegraphics[width=14cm]{./figures/c97bcfcdb960bc3f.png}
\caption{去噪扩散概率模型的结构} \label{fig_DDPM_1}
\end{figure}

设源数据为$X_0$,$t$为时间变量,某个扩散过程中有$T$个时间步.

前向扩散过程会逐渐向原数据添加小幅的高斯噪音,时间序列为:$t=0$ -> $1$ -> $2$ -> ... -> $T-1$ -> $T$.每一步所添加的高斯噪音的方差序列为:$\beta_1$, $\beta_2$, ..., $\beta_T$.

从$t-1$到$t$的转换概率为$q(\bvec x_t|\bvec x_{t-1})$,则从源数据到扩散过程最后一步的转换概率为:
\begin{equation}
q(\bvec x_{1:T})=\prod_{t=1}^{T}q(\bvec x_{t}|\bvec x_{t-1})
\end{equation}
其中,
\begin{equation}
q(\bvec x_{t}|\bvec x_{t-1})=\mathcal{N}(\bvec x_t;\sqrt{1-\beta_t}\bvec x_{t-1},\beta_tI)
\end{equation}

反向的逆扩散过程是一个参数化的马尔科夫过程,每一步转换概率的参数通过学习而获得。时间序列为:$t=T$ -> $T-1$ -> ... -> $1$ -> $0$.逆过程的起点为时间步t=T时的带有噪音的数据,其数据分布为$p_\theta(\bvec x_T)=\mathcal{N}(\bvec x_T;0,I)$.整个过程的转换概率为:
\begin{equation}
p_{\theta}(\bvec x_{0:T})=\prod_{t=1}^{T}p_{\theta}(\bvec x_{t-1}|\bvec x_{t})
\end{equation}
其中,
\begin{equation}
p_{\theta}(\bvec x_{t-1}|\bvec x_{t})=\mathcal{N}(\bvec x_{t-1};\mu_\theta(\bvec x_{t},t),\Sigma_\theta(\bvec x_t,t))
\end{equation}


参考文献:
\begin{enumerate}
\item J. Ho, A. Jain, and P. Abbeel, “Denoising Diffusion Probabilistic Models,” in Advances in Neural Information Processing Systems, 2020, vol. 33, pp. 6840–6851.
\end{enumerate}