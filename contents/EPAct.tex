% 端点可变的作用量
% keys 作用量的偏导数|作用量的全微分
% license Xiao
% type Tutor

\pentry{作用量原理\nref{nod_HamPrn}}{nod_f98c}
本文将证明几个个关于作用量
\begin{equation}
S=\int_{t_1}^{t_2}L\dd t~.
\end{equation}

的重要公式,其中有两个是通常遇到的,即作用量对末时刻的偏导数等于负的能量(哈密顿量),对末坐标的偏导数等于动量对应分量,或说作用量的梯度等于动量。

具体来说,要证下面公式:
\begin{equation}\label{eq_EPAct_1}
\begin{aligned}
&\pdv{S}{t^{(1)}}=H^{(1)}~,\\
&\pdv{S}{t^{(2)}}=-H^{(2)}~,\\
&\pdv{S}{{q}^{i(2)}}=p_i^{(2)}~,\\
&\pdv{S}{{q}^{i (1)}}=-p_i^{(1)}~,\\
\dd S=\sum_i p_i^{(2)}\dd {q}^{i (2)}-&H^{(2)}\dd t^{(2)}-\sum_i p_i^{(1)}\dd {q}^{i (1)}+H^{(1)}\dd t^{(1)}~.
\end{aligned}
\end{equation}
这里,上标 $(1),(2)$ 分别代表起点和终点对应值。

既然\autoref{eq_EPAct_1} 是关于端点的偏微分,这就是说这里的作用量实际上是端点可变的作用量。

可能读者已经疑惑了,作用量的自变量不应是个函数么?怎么这里的自变量是起止时刻和初末位置了。事实上,我们要找的作用量对应物理系统的演化,那么系统演化的曲线是使作用量取极值的曲线,而在端点和起止时刻确定时系统的演化我们认为只有一个,那么作用量就可看成这一极值曲线的两端点和对应起止时刻的函数。
\subsection{证明:}
这里的公式事实上和变分学的\enref{端点可变问题}{EPQue}中的一样,那里有更严格的证明,只需明确物理意义即可。然而,我们这里给出较之更适合物理人的证明,以避免深入了解变分学。

我们先证明关于端点的偏导数,即初末时刻不变,而仅有一端点变化时的情形。

注意到
\begin{equation}
\delta S=\left.\pdv{L}{\dot q^i}\delta q^i\right|_{t_1}^{t_2}+\int_{t_1}^{t_2}\qty(\pdv{L}{q}-\dv{}{t}\pdv{L}{\dot q})\delta q\dd t~.
\end{equation}
因为系统真实演化,那么拉格朗日方程(\autoref{eq_HamPrn_4}~\upref{HamPrn})成立,上式积分项为0。而初位置固定,即 $\delta q^{(1)}=0$,于是
\begin{equation}
\delta S=\sum_i\pdv{L}{{\dot q}^{i (2)}}\delta {q}^{i (2)}\Rightarrow \pdv{S}{{q}^{i (2)}}=\pdv{L}{{\dot q}^{i (2)}}~,
\end{equation}
注意动量定义
\begin{equation}
p_i:=\pdv{L}{\dot q^i}~.
\end{equation}
于是
\begin{equation}
\pdv{S}{{q}^{i (2)}}=p_i^{(2)}~.
\end{equation}

同样,若末时刻 $q^{(2)}$ 不变,仅初始刻 $q^{(1)}$ 变,就是
\begin{equation}
\pdv{S}{{q}^{i (1)}}=-p_i^{(1)}~.
\end{equation}

下面证对时间的偏导数,即初末位置不变,仅初末一时刻变化时的情形。

根据作用量 $S$ 的定义,有
\begin{equation}\label{eq_EPAct_3}
\dd{S}=L^{(2)}\dd t^{(2)}-L^{(1)}\dd t^{(1)}~.
\end{equation}
另一方面, $S$ 可看成初末位置和时间的函数,于是
\begin{equation}\label{eq_EPAct_4}
\begin{aligned}
\dd{S}&=\pdv{S}{t^{(1)}}\dd t^{(1)}+\sum_i\pdv{S}{{q}^{i (1)}}\;{\dot q}^{i (1)}\dd t^{(1)}+\pdv{S}{t^{(2)}}\dd t^{(2)}+\sum_i\pdv{S}{{q}^{i (2)}}\; {\dot q}^{i (2)}\dd t^{(2)} \\
&=\qty(\pdv{S}{t^{(1)}}-\sum_i p_i^{(1)} {\dot q}^{i (1)})\dd t^{(1)}+\qty(\pdv{S}{t^{(2)}}+\sum_i p_i^{(2)}{\dot q}^{i (2)})\dd t^{(2)}~.
\end{aligned}
\end{equation}
比较\autoref{eq_EPAct_3} ,\autoref{eq_EPAct_4} ,有
\begin{equation}\label{eq_EPAct_5}
\begin{aligned}
\pdv{S}{t^{(1)}}&=\sum_i p_i^{(1)}{\dot q}^{i (1)}-L^{(1)}~,\\
\pdv{S}{t^{(2)}}&=-\qty(\sum_i p_i^{(2)}{\dot q}^{i (2)}-L^{(2)})~.
\end{aligned}
\end{equation}
由哈密顿量 $H$ 定义:
\begin{equation}
H:=\sum_i p_i {\dot q^i}-L~.
\end{equation}
\autoref{eq_EPAct_5} 就成为
\begin{equation}\label{eq_EPAct_2}
\begin{aligned}
\pdv{S}{t^{(1)}}&=H^{(1)}~,\\
\pdv{S}{t^{(2)}}&=-H^{(2)}~.
\end{aligned}
\end{equation}
将\autoref{eq_EPAct_2} 带入\autoref{eq_EPAct_4} 就有
\begin{equation}
\dd{S}=H^{(1)}\dd t^{(1)}-\sum_ip_{(1)}\dd {q}^{i (1)}-H^{(2)}\dd t^{(2)}+\sum_i p_{(2)}\dd {q}^{i (2)}~.
\end{equation}

证明结束。
