% 乔丹代数(综述)
% license CCBYSA3
% type Wiki

本文根据 CC-BY-SA 协议转载翻译自维基百科\href{https://en.wikipedia.org/wiki/Jordan_algebra}{相关文章}。

在抽象代数中,乔丹代数是定义在一个域上的非结合代数,其乘法满足以下公理:
\begin{enumerate}
\item 交换律:$xy = yx$
\item 乔丹恒等式:$(xy)(xx) = x(y(xx))$
\end{enumerate}
在乔丹代数中,两个元素 $x$ 和 $y$ 的乘积也常记作 $x \circ y$,尤其是在需要避免与相关结合代数的乘法混淆时。

这些公理推出\(^\text{[[1]]}\)乔丹代数是幂结合的,即:$x^n = x \cdots x$的值与括号的放置方式无关。它们还推出\(^\text{[[1]]}\):$x^{m}(x^{n}y) = x^{n}(x^{m}y)$对所有正整数 $m$ 和 $n$ 都成立。因此,我们也可以等价地将乔丹代数定义为:一个交换的、幂结合的代数,并且对于任意元素 $x$,所有“与 $x^n$ 相乘”的运算彼此交换。

乔丹代数由帕斯夸尔·乔丹于 1933 年引入,目的是形式化量子电动力学中可观测量代数的概念。尽管很快就发现这种代数在该物理背景中并不实用,但此后它们在数学中找到了许多应用\(^\text{[[2]]}\)。这种代数最初被称为“r-数系统”,后来在 1946 年被亚伯拉罕·阿德里安·阿尔伯特改名为“乔丹代数”,并由他开始了对一般乔丹代数的系统研究。
\subsection{特殊乔丹代数}
首先注意,一个结合代数当且仅当它是交换的时才是乔丹代数。

给定任意结合代数 $A$(特征不为 2),可以利用相同的加法运算并引入一种新的乘法来构造一个乔丹代数 $A^+$,其乘法(乔丹积)定义为:
$$
x \circ y = \frac{xy + yx}{2}~
$$
这些乔丹代数及其子代数被称为特殊乔丹代数,而其他不是通过这种方式得到的乔丹代数称为例外乔丹代数。这一构造类似于与 $A$ 相关的李代数,其乘法(李括号)定义为交换子:$[x, y] = xy - yx$

Shirshov–Cohn 定理指出,任意由两个生成元构成的乔丹代数都是特殊的\(^\text{[3]}\)。与此相关,Macdonald 定理指出:任意在三个变量中的多项式,如果在其中一个变量上的次数为 1,并且它在每一个特殊乔丹代数中恒为零,那么它在所有乔丹代数中也恒为零\(^\text{[4]}\)。
\subsubsection{Hermitian 乔丹代数}
如果 $(A, \sigma)$ 是一个带有对合$\sigma$ 的结合代数,那么若 $\sigma(x) = x$ 且 $\sigma(y) = y$,就有:$\sigma(xy + yx) = xy + yx$因此,由对合所不变的所有元素(有时称为Hermitian 元素)组成的集合在 $A^+$ 中构成一个子代数,有时记作 $H(A, \sigma)$。
\subsection{例子}
\begin{enumerate}
\item 由所有自伴的实矩阵、复矩阵或四元数矩阵组成的集合,在乘法
$$
(xy + yx) / 2~
$$
下构成一个特殊乔丹代数。
\item 由所有 3×3 自伴的八元数矩阵组成的集合,在同样的乘法
$$
(xy + yx) / 2~
$$
\end{enumerate}
下构成一个 27 维的例外乔丹代数(之所以是例外的,是因为八元数不是结合的)。这是第一个Albert 代数的例子。它的自同构群是例外李群 $F_4$。在复数域上,这个代数在同构意义下是唯一的单纯例外乔丹代数\(^\text{[5]}\),因此它通常被称为“那个例外乔丹代数”。在实数域上,单纯例外乔丹代数共有三种同构类\(^\text{[5]}\)。
\subsection{导子与结构代数}
乔丹代数 $A$ 的一个导子是 $A$ 的一个自同态 $D$,满足:$D(xy) = D(x)y + xD(y)$所有导子构成一个李代数,记作 $\mathrm{der}(A)$。乔丹恒等式推出:如果 $x, y \in A$,则将 $z$ 映射为$x(yz) - y(xz)$的映射是一个导子。因此,$A$ 与 $\mathrm{der}(A)$ 的直和可以构成一个李代数,称为 $A$ 的结构代数,记作 $\mathrm{str}(A)$。

一个简单的例子由 Hermitian 乔丹代数 $H(A, \sigma)$ 提供。在这种情况下,若 $x \in A$ 且满足 $\sigma(x) = -x$,则 $x$ 定义了一个导子。在许多重要的例子中,$H(A, \sigma)$ 的结构代数就是 $A$ 本身。导子与结构代数也是蒂茨构造Freudenthal 魔方的一部分。
\subsection{形式实乔丹代数}
在实数域上的(可能是非结合的)代数,如果满足这样一个性质:若 $n$ 个平方的和为零,则每一个平方项都必须单独为零,就称为形式实。1932 年,乔丹尝试通过公理化的方式刻画量子理论,他提出:任何量子系统的可观测量代数都应当是一个形式实代数,并且是交换的($xy = yx$)且幂结合的(结合律在仅涉及 $x$ 的乘积中成立,从而任意元素 $x$ 的幂是无歧义的)。他证明了任何这样的代数都是一个乔丹代数。

并非所有乔丹代数都是形式实的,但乔丹、冯·诺伊曼和维格纳(Jordan, von Neumann & Wigner,1934)对有限维的形式实乔丹代数(也称为欧几里得乔丹代数)进行了分类。任何形式实乔丹代数都可以写成若干所谓单纯的代数的直和,而单纯代数本身不能以非平凡的方式再分解为直和。在有限维情形下,单纯形式实乔丹代数分为四个无限族,加上一个例外情形:
\begin{itemize}
\item $n \times n$ 自伴实矩阵的乔丹代数(同前所述)。
\item $n \times n$ 自伴复矩阵的乔丹代数(同前所述)。
\item $n \times n$ 自伴四元数矩阵的乔丹代数(同前所述)。
\item 由 $\mathbb{R}^n$ 自由生成,并满足关系式
$$
x^2 = \langle x, x \rangle~
$$
的乔丹代数,其中右侧是用 $\mathbb{R}^n$ 上的通常内积定义的。有时称为旋量因子或克利福德型乔丹代数。
\item $3 \times 3$ 自伴八元数矩阵的乔丹代数(同前所述,即例外乔丹代——Albert 代数)。在这些可能性中,到目前为止,自然界似乎只将 $n \times n$ 自伴复矩阵用作可观测量的代数。然而,旋量因子在狭义相对论中起着作用,而所有形式实乔丹代数都与射影几何有关。
\end{itemize}
\subsection{皮尔斯分解}
如果 $e$ 是乔丹代数 $A$ 中的一个幂等元(idempotent,满足 $e^2 = e$),并且 $R$ 表示“与 $e$ 相乘”的运算,那么有:
\begin{itemize}
\item
$$
R(2R - 1)(R - 1) = 0~
$$
\end{itemize}
因此,$R$ 的唯一特征值是 $0$、$1/2$、$1$。如果乔丹代数 $A$ 在特征不为 2 的域上是有限维的,这意味着它可以分解为三个特征子空间的直和:$A = A_0(e) \oplus A_{1/2}(e) \oplus A_1(e)$这种分解最早由乔丹、冯·诺伊曼和维格纳(1934)在完全实乔丹代数的情形下提出。后来,阿尔伯特(Albert,1947)在一般情形下进行了研究,并将其称为相对于幂等元 $e$ 的皮尔斯分解\(^\text{[6]}\)。
\subsection{特殊类型与推广}
\subsubsection{无限维乔丹代数}
1979 年,叶菲姆·泽尔马诺夫对无限维单纯(以及素的非退化)乔丹代数进行了分类。它们要么是Hermitian 型,要么是Clifford 型。特别地,唯一的例外单纯乔丹代数是有限维的 Albert 代数,其维数为 27。
\subsubsection{乔丹算子代数}
算子代数理论已经扩展到乔丹算子代数的情形。

C$^*$ 代数在乔丹算子代数中的对应物是 JB 代数,在有限维情形下称为欧几里得乔丹代数。实乔丹代数上的范数必须是完备的,并且满足以下公理:
$$
\|a \circ b\| \leq \|a\| \cdot \|b\|,\quad \|a^2\| = \|a\|^2,\quad \|a^2\| \leq \|a^2 + b^2\|~
$$
这些公理保证了乔丹代数是形式实的,即如果若干项的平方和为零,则这些项必须分别为零。JB 代数的复化称为乔丹 C$^*$ 代数(Jordan C$^*$-algebra)或 JB$^*$ 代数(JB$^*$-algebra)。它们在复几何中被广泛应用,用于将 Koecher 关于有界对称域的乔丹代数化处理方法推广到无限维情形。并非所有 JB 代数都可以实现为希尔伯特空间上自伴算子的乔丹代数,这与有限维情形完全一致;阻碍的核心正是那个例外的 Albert 代数。

乔丹代数中与冯·诺伊曼代数对应的概念是 JBW 代数。JBW 代数实际上是作为巴拿赫空间的JB 代数,并且它们是某个巴拿赫空间的对偶空间。冯·诺伊曼代数的大部分结构理论都可以移植到 JBW 代数中。特别地,JBW 因子(其中心为 $\mathbb{R}$)已经可以完全用冯·诺伊曼代数刻画。除了例外的 Albert 代数 外,所有的 JBW 因子都可以实现为在弱算子拓扑下闭合的希尔伯特空间上自伴算子的乔丹代数。其中的旋量因子可以直接由实希尔伯特空间构造出来。其他所有的 JBW 因子要么是某个冯·诺伊曼因子的自伴部分,要么是该因子在一个 2 阶 $*$-反自同构作用下的不动子代数\(^\text{[7]}\)。
\subsubsection{乔丹环}
乔丹环是乔丹代数的推广,只要求其定义在一般环上而非域上。
或者,也可以将乔丹环定义为满足乔丹恒等式的交换非结合环。
\subsubsection{乔丹超代数}
乔丹超代数由 Kac、Kantor 和 Kaplansky 引入;它们是一个 $\mathbb{Z} / 2$-分次代数$J = J_{0} \oplus J_{1}$其中 $J_{0}$ 是一个乔丹代数,而 $J_{1}$ 具有“类李”乘法,其结果在 $J_{0}$ 中\(^\text{[8]}\)。

任意一个 $\mathbb{Z} / 2$-分次结合代数$A = A_{0} \oplus A_{1}$在带分次的乔丹括号运算下成为一个乔丹超代数:
$$
\{x_{i}, y_{j}\} = x_{i}y_{j} + (-1)^{ij} y_{j}x_{i}~
$$
在特征为 0 的代数闭域上,Kac(1977)对乔丹单纯超代数进行了分类。
它们包括若干族以及一些例外代数,特别是 $K_{3}$ 和 $K_{10}$。
\subsubsection{J-结构}
J-结构的概念由 Springer(1998)引入,旨在利用线性代数群发展乔丹代数理论,并以乔丹逆元作为基本运算、以Hua 恒等式作为基本关系来设定公理。
当特征不等于 2 时,J-结构理论与乔丹代数理论本质上相同。
\subsubsection{二次乔丹代数}
二次乔丹代数是由 Kevin McCrimmon(1966)引入的对(线性)乔丹代数的推广。
它以线性乔丹代数的二次表示的基本恒等式作为公理,用于定义任意特征域上的二次乔丹代数。有限维单纯二次乔丹代数有一个与特征无关的统一描述:当特征不等于 2 时,二次乔丹代数理论可化归为线性乔丹代数理论。
\subsection{另见}
\begin{itemize}
\item 弗罗因登塔尔代数
\item 乔丹三重系统
\item 乔丹对
\item 坎托–科赫尔–蒂茨构造
\item 斯科尔扎簇
\end{itemize}
\subsection{注释}
\begin{enumerate}
\item Jacobson 1968,第 35–36 页,特别是 (56) 之前的备注与定理 8。
\item Dahn, Ryan(2023-01-01),"Nazis, émigrés, and abstract mathematics",*Physics Today*,76 (1):44–50,Bibcode:2023PhT....76a..44D,doi:10.1063/PT.3.5158。
\item McCrimmon 2004,第 100 页。
\item McCrimmon 2004,第 99 页。
\item Springer & Veldkamp 2000,第 §5.8 节,第 153 页。
\item McCrimmon 2004,第 99 页及以下,第 235 页及以下。
\item 参见:
\begin{itemize}
\item Hanche-Olsen & Størmer 1984
\item Upmeier 1985
\item Upmeier 1987
\item Faraut & Koranyi 1994
\end{itemize}
\item McCrimmon 2004,第 9–10 页。
\end{enumerate}
\subsection{参考文献}
\begin{itemize}
\item Albert, A. Adrian (1946), “On Jordan algebras of linear transformations”, Transactions of the American Mathematical Society, 59 (3): 524–555, doi:10.1090/S0002-9947-1946-0016759-3, ISSN 0002-9947, JSTOR 1990270, MR 0016759
\item Albert, A. Adrian (1947), “A structure theory for Jordan algebras”, Annals of Mathematics, Second Series, 48 (3): 546–567, doi:10.2307/1969128, ISSN 0003-486X, JSTOR 1969128, MR 0021546
\item Baez, John C. (2002), “The Octonions, 3: Projective Octonionic Geometry”, Bulletin of the American Mathematical Society, 39 (2): 145–205, doi:10.1090/S0273-0979-01-00934-X, MR 1886087, S2CID 586512. [在线 HTML 版本]
\item Faraut, J.; Koranyi, A. (1994), *Analysis on symmetric cones*, Oxford Mathematical Monographs, Oxford University Press, ISBN 0198534779
\item Hanche-Olsen, H.; Størmer, E. (1984), Jordan operator algebras*, Monographs and Studies in Mathematics, vol. 21, Pitman, ISBN 0273086197
\item Jacobson, Nathan (2008) [1968], Structure and representations of Jordan algebras, American Mathematical Society Colloquium Publications, vol. 39, Providence, R.I.: American Mathematical Society, ISBN 9780821831793, MR 0251099
\item Jordan, Pascual (1933), “Über Verallgemeinerungsmöglichkeiten des Formalismus der Quantenmechanik”, Nachr. Akad. Wiss. Göttingen. Math. Phys. Kl. I, 41: 209–217
\item Jordan, P.; von Neumann, J.; Wigner, E. (1934), “On an algebraic generalization of the quantum mechanical formalism”, Annals of Mathematics, 35 (1): 29–64, doi:10.2307/1968117, JSTOR 1968117
\item Kac, Victor G. (1977), “Classification of simple Z-graded Lie superalgebras and simple Jordan superalgebras”, *Communications in Algebra, 5 (13): 1375–1400, doi:10.1080/00927877708822224, ISSN 0092-7872, MR 0498755
\item McCrimmon, Kevin (1966), “A general theory of Jordan rings”, Proc. Natl. Acad. Sci. U.S.A., 56 (4): 1072–1079, Bibcode:1966PNAS...56.1072M, doi:10.1073/pnas.56.4.1072, JSTOR 57792, MR 0202783, PMC 220000, PMID 16591377, Zbl 0139.25502
\item McCrimmon, Kevin (2004), A taste of Jordan algebras, Universitext, Berlin, New York: Springer-Verlag, doi:10.1007/b97489, ISBN 978-0-387-95447-9, MR 2014924, Zbl 1044.17001, [勘误]
\item Ichiro Satake (1980), Algebraic Structures of Symmetric Domains, Princeton University Press, ISBN 978-0-691-08271-4. [评论]
\item Schafer, Richard D. (1996), An introduction to nonassociative algebras, Courier Dover Publications, ISBN 978-0-486-68813-8, Zbl 0145.25601
\item Zhevlakov, K.A.; Slin'ko, A.M.; Shestakov, I.P.; Shirshov, A.I. (1982) [1978], Rings that are nearly associative, Academic Press, ISBN 0-12-779850-1, MR 0518614, Zbl 0487.17001
\item Slin'ko, A.M. (2001) [1994], “Jordan algebra”, Encyclopedia of Mathematics, EMS Press
\item Springer, Tonny A. (1998) [1973], Jordan algebras and algebraic groups, Classics in Mathematics, Springer-Verlag, doi:10.1007/978-3-642-61970-0, ISBN 978-3-540-63632-8, MR 1490836, Zbl 1024.17018
\item Springer, Tonny A.; Veldkamp, Ferdinand D. (2000) [1963], *Octonions, Jordan algebras and exceptional groups, Springer Monographs in Mathematics, Berlin, New York: Springer-Verlag, doi:10.1007/978-3-662-12622-6, ISBN 978-3-540-66337-9, MR 1763974
\item Upmeier, H. (1985), Symmetric Banach manifolds and Jordan C$^*$-algebras, North-Holland Mathematics Studies, vol. 104, ISBN 0444876510
\item Upmeier, H. (1987), Jordan algebras in analysis, operator theory, and quantum mechanics, CBMS Regional Conference Series in Mathematics, vol. 67, American Mathematical Society, ISBN 082180717X
\end{itemize}
\subsection{进一步阅读}
\begin{itemize}
\item Knus, Max-Albert; Merkurjev, Alexander; Rost, Markus; Tignol, Jean-Pierre (1998), *The book of involutions*, Colloquium Publications, vol. 44, With a preface by J. Tits, Providence, RI: American Mathematical Society, ISBN 0-8218-0904-0, Zbl 0955.16001
\end{itemize}
\subsection{外部链接}
\begin{itemize}
\item PlanetMath 上的 Jordan algebra
\item PlanetMath 上的 Jordan-Banach 与 Jordan-Lie 代数
\end{itemize}