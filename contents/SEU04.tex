% 东南大学 2004 年 考研 量子力学
% license Usr
% type Note

\textbf{声明}:“该内容来源于网络公开资料,不保证真实性,如有侵权请联系管理员”

\subsection{简发题【30分,每题 5分]}
\begin{enumerate}
\item 简述 Bohr量子论的基本内客
\item 什么叫隧道效应 ?试解释之
\item 写出 schrodinger方程的钜阵形式,
\item 若势能 $V(\vec{r})$ 改变一个常量 $C$ 时,即 $V(\vec{r}) \rightarrow V(\vec{r}) + S$,粒子的波函数与时间无关的部分将改变吗?能量值改变否?
\item 设一维粒子的 Hamilton 量为 $H = p^2/2m + V(x)$,写出 $p$ 表象中 $x, p$ 和 $H$ 的“矩阵元”。
\item 写出 Fermi黄金规则,并解释之。
\end{enumerate}
\subsection{15分}
证明:在任何状态下平均值均为实的算符,必为厄密算符。
\subsection{15分}
证明:如果体系有两个彼此不对易的宇恒量,则体系能级一般是简并的
\subsection{15分}
设粒子处于半径无限高的势垒中:

\[
V(x) =
\begin{cases}
\infty, & x < 0 \\
-V_0, & 0 < x < a \\
0, & x > a
\end{cases}~
\]
求粒子的能量本征值。求至少存在一条束缚能级的条件。
\subsection{10分}
证明在分立得能量本征态下动量平均值为0。
\subsection{20分}
自旋为$\hbar/2$,内禀磁矩为$m_0$的粒子。\\
1. 在空间分布均匀但随时间改变得磁场$\vec{B}_(t)$中运动,证明粒子的波函数。\\
2. 同1类似,设磁场大小不变,但磁场在 $x y $ 平面中以下列规律变化$B_x = B \cos \omega t$, $B_y = B \sin \omega t$, $B_z = 0$),求粒子的自旋波函数。