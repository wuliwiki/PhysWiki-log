% 南京航空航天大学 2015 量子真题
% license Usr
% type Note

\textbf{声明}:“该内容来源于网络公开资料,不保证真实性,如有侵权请联系管理员”

\subsection{简答题(20 分,每题 10 分)}
①证明:薛定谔方程中如果 $V(x)$是偶函数,即 $V(-x)=V(x)$,那么波函数 $\psi(x)$总
可以取作偶函数或奇函数。

②经典物理中一个矢量 $\vec r$ 与自身的矢量积(叉乘)恒为零$\vec r \times \vec r\equiv
0$   ,对于量子力学
中矢量算符这一结论仍然普遍成立吗?试举例说明。

\subsection{二}
如果算符 $\hat{P}$ 满足等幂性,即 $\hat{P}^2 = \hat{P}$,那么我们称为 $\hat{P}$ 投影算符,试证明两
投影算符 $\hat{P}_1$ 、 $\hat{P}_1$ 之和 $\hat{P}_1+\hat{P}_2$ 也为为投影算符的重要条件为这两个投影算符对易$[\hat{P}_1,\hat{P}_2=0]$。(20 分)

\subsection{三}
质量为 $m$ 的粒子在一维无限深方势阱 $ V(x) = \begin{cases} 0, & 0 < x < a \\\\ \infty, & \text{others} \end{cases} $ 中运动,

1. 若已知 $ t = 0$ 时,该粒子状态为 $\psi(x,0) = \sqrt{\frac{1}{a}}(1 + \cos \frac{\pi x}{a})\sin \frac{\pi x}{a}$,求 $t > 0 $ 时刻该粒子的波函数;

2. 求 $ t > 0 $ 时刻测量到粒子的能量为 $\frac{n^2 \hbar^2}{2 m a^2}$ 的几率是多少?

3. 求 $t > 0 $ 时刻粒子的平均能量 $\bar{E}$和平均位置 $\bar{x}$。

(30 分)

\subsection{四}
质量为 $m$ 、电荷为 $q$ 的粒子在三维各向同性谐振子势 $$V(r) = m\omega^2(x^2 + y^2 + z^2)/2~$$ 中运动,同时受到一个沿 $x$ 方向的均匀电场 $E = E_0 x i$ 作用。求粒子的能量本征值和第一激发态的简并度。此时轨道角动量是否守恒?如回答是,则请写出此守恒力学量的表达式。(30分)

\subsection{五}
设氢原子处于状态
\begin{equation}
\Psi (r, \theta, \phi, s_z) = \frac{1}{2}
 \begin{pmatrix}
R_{21}(r)Y_{11}(\theta, \phi) \\\\
-\sqrt{3}R_{21}(r)Y_{10}(\theta, \phi)
\end{pmatrix}~
\end{equation}

1. 求轨道角动量 $\hat{L}_z$ 分量 $L_z$ 和自旋角动量 $\hat{\sigma}_z$ 分量 $\sigma_z$ 的平均值;

2. 求总磁矩 $\hat{\vec M} = -\frac{e}{2\mu} \hat{\vec L} - \frac{e}{\mu} \hat{\vec S}$ 的 $z$ 分量的平均值。(20 分)

\subsection{六}
一个二能级系统,哈密顿量为 $\hat{H}$,能级大小间隔为 $E_0$。现在此系统受到一个微扰 $\hat{H}'$。在 $\hat{H}$ 表象中,$\hat{H}'$ 的表示为 $\hat{H}'=\hat\lambda(\sigma_x + \sigma_y)$,其中 $\sigma_x,\sigma_y$ 是泡利矩阵,$\lambda$ 为实数。请算出系统受微扰后的能级间隔(精确到二级微扰修正)。(30 分)
