% 无理数(数论)
% keys 无理数
% license Usr
% type Tutor

\begin{definition}{有理数与无理数}
能表示为互素的整数 $a$ 与 $b\neq 0$ 的比值 $a/b$ 的数称为\textbf{有理数}。

非有理数,也就是不能表示为互素整数 $a$ 与 $b\neq 0$ 的比值 $a/b$ 的数称为\textbf{无理数}。
\end{definition}

\begin{theorem}{}
$\sqrt{2}$ 无理。
\end{theorem}
\textbf{证明}:假设若 $\sqrt 2$ 有理,则对于 $\sqrt 2 = a/b$,即方程 $a^2=2b^2$ 有整数解,且 $(a, b) = 1$。
而注意到等式右侧是 $2$ 的整数倍,是偶数,故 $a$ 也是偶数,即 $a=2c$,这会使得 $(2c)^2 = 2b^2$,可化为 $2c^2=b^2$,同样是的 $b$ 是偶数而 $a, b$ 就有公约数 $2$,与 $(a, b) = 1$ 矛盾!得证。

\begin{theorem}{}
对于整数 $N$,$\sqrt[m]{N}$ 无理,除非 $N$ 是某整数 $n$ 的 $m$ 次方。
\end{theorem}
\textbf{证明}:仍考虑 $\sqrt[m]{N} = a/b$,$(a, b) = 1$,可化为
\begin{equation}
a^m = N b^m, ~ (a, b) = 1 ~,
\end{equation}
而考虑 $a^m$ 的各素因子与次幂 $p^k$,则 $p^{mk} | (Nb^m)$,显然 $N$ 不能是 $p^{mk}$ 否则 $N$ 就是 $p^k$ 的 $m$ 次方,这就使得总会有 $p | b^m$ 从而 $p|b$,从而 $(a, b)$ 至少有 $p$ 而不等于 $1$,矛盾!故得证。

\begin{theorem}{}
对于关于 $x$ 的首 $1$ 整系数多项式
\begin{equation}
x^n + c_1 x^{n-1} + c_2 x^{n-2} + \cdots + c_n = 0 ~,
\end{equation}
若 $x_0$ 是一根,则 $x$ 要么是整数,要么 $x$ 就是无理数。
\end{theorem}
\textbf{证明}:我们可以假设 $c_n \neq 0$, 否则可以提取出因式 $x$ 而继续重复这操作直到 $c_n = 0$。

此时当 $x_0$ 有理时,设 $x_0 = a/b$ 且 $(a , b) = 1$,则可化为
\begin{equation}
a^n + c_1 a^{n-1} b + c_2 a^{n-2} b^2 + \cdots +  c_nb^n = 0~,
\end{equation}
这就说明 $b | a^n$,于是对于任何 $b$ 的素因子 $p$ 都有 $p|a^m$ 即 $p | a$,就使得 $(a, b)$ 至少为 $p$,矛盾!故 $b$ 无素因子,即 $b = 1$,而此时 $x_0$ 是整数。

故 $x_0$ 有理时必然为整数,即 $x_0$ 要么无理,要么是整数。证毕!