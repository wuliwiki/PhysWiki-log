% 函数的连续性
% keys 极限|微积分|连续函数|左连续|右连续
% license Xiao
% type Tutor

\pentry{极限\nref{nod_Lim}}{nod_ee22}

简单来说, \textbf{连续函数}定义为: 在某个区间内, 函数曲线是连续的。 例如常见的 $\sin x$, $\exp x$, $x^2$ 都在整个实数域上连续, 又例如 $\ln x$ 和 $1/x$ 在区间 $(0, \infty)$ 上连续, $\tan x$ 在所有 $x_n = (1/2 + n)\pi$($n$ 为自然数)处不连续, $1/x$ 在 $x = 0$ 处不连续。 但这只是一些简单的情况。 在一些情况下这种判断方法则显得不严谨, 例如函数
\begin{equation}
f(x) = \leftgroup{
    &\sin(1/x) \quad &(x \ne 0)\\
    &0  \quad &(x = 0)
}~.
\end{equation}
在原点处的连续性(不连续)根据这个定义不好判断。 所以我们需要一个更严谨的定义。

首先我们要讨论函数在一个点附近是否是连续的。这个概念的思想核心是,在函数曲线的某一点附近 $(x_0, f(x_0))$,无论我们要求 $f(x)$ 有多接近 $f(x_0)$,只要 $x$ 足够靠近 $x_0$,就一定能满足条件。比如说,如果定义函数 $f$ 为当 $x<0$ 的时候,$f(x)=0$,其它时候 $f(x)=1$,那么在 $x=0$ 这一点处 $f$ 就是跳跃的。如果我们要求的接近程度小于 $1$,那么无论 $x$ 多接近 $0$,只要 $x<0$,$f(x)$ 和 $f(0)=1$ 的距离就永远满足不了需要。

准确地描述以上“连续”的概念,如下所示:

\begin{definition}{函数在一点的连续性和区间的连续性}
函数 $f(x)$ 在某点 $x = x_0$ 处\textbf{连续}的定义是: 函数 $f(x)$ 在某点 $x=x_0$ 处连续,当且仅当对于任何精度要求 $\epsilon>0$,我们都可以找到一个对应的范围 $\delta>0$,使得只要 $|x-x_0|<\delta$,就有 $|f(x)-f(x_0)|<\epsilon$。 用极限符号来表示, 就是:
\begin{equation}
\lim_{x \to x_0} f(x) = f(x_0)~.
\end{equation}

如果一个函数在某区间的所有点都连续, 我们就说它\textbf{在这个区间连续}。
\end{definition}

注意这里要求 $x$ 从左边和右边趋近于 $x_0$ 时的极限(即左极限和右极限)都成立。 % 未完成: 极限文章中介绍左极限和右极限

\subsection{一致连续}

以上所定义的连续性是针对一个个点 $x_0$ 而言的,就算函数在每一个点都连续,我们也只能说这个函数是\textbf{逐点连续的(pointwise continuous)}。事实上,还有一种更强的连续性,它着眼于整体的性质,这就是\textbf{一致连续(uniformly continuous)}。它的准确定义如下:

\begin{definition}{一致连续}
如果函数 $f(x)$ 在区间 $S$ 上,对于任意精度 $\epsilon>0$,都存在对应的范围 $\delta>0$,使得只要 $|x_1-x_2|<\delta$,那么 $|f(x_1)-f(x_2)|<\epsilon$。

\end{definition}

一致连续着眼于整个区间的性质,而不是一个个点。显然,一致连续的函数肯定逐点连续,但是逐点连续的函数不一定一致连续, 我们举一个反例。

\begin{example}{}
例如考虑函数 $f(x)=1/x$,那么在区间 $(0, \infty)$ 上, $f$ 就是逐点连续的,但它并不一致连续;对于同样的精度要求 $\epsilon>0$ 和任何范围 $\delta>0$,只要 $0<x_1<\delta$,那么就总有一个足够小的 $x_2$ 使得 $|f(x_1)-f(x_2)|>\epsilon$, 毕竟当 $x_2$ 接近 $0$ 的过程中, $f(x_2)$ 的斜率绝对值会趋近于无穷大。
\end{example}

\begin{exercise}{连续但不一致连续的函数}
试证明 $1/x$ 在区间 $(0, +\infty]$ 以及 $x^2$ 在 $\mathbb R$ 都是连续的, 但不是一致连续的。
\end{exercise}

\begin{theorem}{}\label{the_contin_1}
设 $S$ 是 $\mathbb{R}$ 的一个区间。函数 $f$ 在 $S$ 上逐点连续的充分必要条件是,对于任何开区间 $A\subset \mathbb{R}$,满足 $f(x)\in A$ 的所有 $x$ 构成的集合,是若干开区间的并集。用紧凑的写法来表达就是,$f^{-1}(A)$ 是开区间的并集。
\end{theorem}

这个定理还可以等价地用闭区间来表达:$f$ 在 $S$ 上逐点连续的充分必要条件是,任何闭区间的逆映射是闭区间的并集。

\begin{figure}[ht]
\centering
\includegraphics[width=5cm]{./figures/c83e83057243bed8.png}
\caption{如图,红色水平线在 $f(X)$ 轴上划分出了一个开/闭区间,绿色垂直线是取反函数的过程,$x$ 轴上的靛蓝色线段就是取反函数的结果。从这个图可以直观地看出\autoref{the_contin_1} 的意义。} \label{fig_contin_1}
\end{figure}

在实数轴上,开集被定义为任何开区间的并集,而闭集是开集的补集。如果 $S$ 是 $\mathbb{R}$ 的一个子集,那么 $S$ 上的开集被定义为 $\mathbb{R}$ 的开集和 $S$ 的交集。这样一来,\autoref{the_contin_1} 就可以扩展为:$f$ 在 $S$ 上逐点连续,等价于对于任何开集 $A$,$f^{-1}(A)\cap S$ 是 $S$ 上的开集,等价于对于任何闭集 $B$,$f^{-1}(B)\cap S$ 是 $S$ 上的闭集。

用逆映射来刻画连续性是一个非常好用的方法。
