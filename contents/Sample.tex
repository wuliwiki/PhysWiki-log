% 小时百科文章示例
% keys 小时百科|文章编辑|latex|宏包
% license CCBYSA3
% type Tutor

\begin{issues}
\issueDraft       % 本文处于草稿阶段
\end{issues}

\pentry{麦克斯韦方程\nref{nod_maxswell}}


\subsection{等离子体的基本概念}

等离子体作为区别于固体、液体、气体而独立的物质第四态,其最大的特性就是内部相互作用力的改变——从固液气中的中性成分作用力占主导变为了电磁相互作用力占主导。等离子体由电子和离子组成,这些带电的粒子所携带的总电荷相等,也就是说体系整体上呈现电中性。等离子体中可以有中性成分吗?当然是可以的,但是考虑到“库仑力主导"的原则,电的作用要占到主导地位才能称得上等离子体,也就是说带电粒子占比要足够高或者说\textbf{电离度}足够高。
		
为了进一步地说明,我们不妨考察一个基本的物态相变的例子:对于某一固体,随着温度的升高,其内部分子的振动能量增加,原本的小尺度振荡被破坏,固体变成了可以自由流动变形的液体。液体在分子作用力作用下,其内部分子能被限制在一定体积中而不能随意逃离;但进一步加热液体,分子做热运动的动能进一步增大,突破分子作用力的限制而可以自由在空间中运动,这就成为了气体。不过这时,气体中的主导相互作用仍为分子或原子间的相互碰撞这样的中性作用。我们再对气体进一步加热,当其热运动的动能增加到其组分的第一电离能时,碰撞就能导致原子或分子的电离,产生正离子和电子。这一过程也就是\textbf{电离过程},而电离成分数密度与发生电离前中性成分数密度之比就是电离度$\alpha$。这是的气体可以称为部分电离的气体,如果温度进一步升高,电离度会随之增加,当库伦相互作用占绝对主导地位时,等离子态就出现了。
		
等离子体看上去只是强烈电离的气体,但是其物理性质却和气体截然不同,其最大的特点就是“牵一发而动全身”,库伦相互作用使原本过于松散随意的气体粒子变得更加循规蹈矩,也在它们之间建立了更加密切的联系,从而表现出一种\textbf{集体行为}。
		
综合以上,我们可以给出等离子体的定义如下:
		
\textbf{等离子体是带电粒子和中性粒子组成的表现出集体行为的一种准中性气体。}
		
接下来我们来展现这种集体行为的几个例子,并对准中性加以进一步的说明。

