% 包络和奇解
% 常微分方程

\pentry{包络线\upref{Velope},一阶隐式常微分方程\upref{ODEa4}}

\subsection{包络}

\begin{definition}{包络}

在 $x-y$ 平面上,定义一族曲线 $\{\Phi_c\}$,其中 $c$ 是一个连续参数。每条曲线 $\Phi_c$ 的表达式为 $f(x, y, c)=0$。

如果存在一条曲线 $\Phi$,它本身不是任何一条 $\Phi_c$,但是在 $\Phi$ 的每一点处都有一条 $\Phi_c$ 和它相切,那么我们就称 $\Phi$ 是曲线族 $\{\Phi_c\}$ 的\textbf{包络(envolope)}。

\end{definition}

\begin{example}{包络的例子}
考虑曲线族 $\{\Phi_c\}$,其中 $\Phi_c$ 的表达式为
\begin{equation}
-\frac{1}{5}x^2+cx+5c^2-y=0
\end{equation}

令曲线 $\Phi$ 的表达式为

\begin{equation}
-\frac{1}{4}x^2-y=0
\end{equation}

则容易验证,$\Phi$ 是 $\{\Phi_c\}$ 的包络。

这个例子正是\textbf{一阶隐式常微分方程}\upref{ODEa4}中的\autoref{ODEa4_eq6}~\upref{ODEa4}和\autoref{ODEa4_eq10}~\upref{ODEa4}。

\end{example}

关于包络的详细讨论请参见预备知识\textbf{包络线}\upref{Velope}。


\subsection{奇解}

\begin{definition}{微分方程的奇解}

如果微分方程 $\frac{\dd y}{\dd x}=f(x, y)$ 有一个解 $F(x, y)=0$,在这个解的每一个点上都至少还有另外一个不同的解,那么称 $F(x, y)$ 是该方程的\textbf{奇解(singular solution)}。

\end{definition}

由定义可知,奇解对应的曲线,就是一族解的包络线。这是因为奇解的每个点上都有的那个不同的解,一定和奇解相切,因为它们都满足同一个微分方程。同时容易看到,奇解是处处不满足解的唯一性定理的。

为了求微分方程的奇解,我们可以先求出其通解,把通解视为一族曲线,求出它们的包络线,则该包络线就是一个奇解。当然,如果通解曲线族没有包络线,这样的奇解就不存在。求包络线的方式参见预备知识\textbf{包络线}\upref{Velope}。

\begin{example}{}
求方程
\begin{equation}\label{EnvSol_eq1}
(\frac{\mathrm{d} y}{\dd x})^2+y^2-1=0
\end{equation}
的奇解。

将\autoref{EnvSol_eq1} 化为
\begin{equation}
\frac{1}{\pm\sqrt{1-y^2}}\dd y=\dd x
\end{equation}

积分后解得
\begin{equation}\label{EnvSol_eq4}
y=\pm\sin(x+c)
\end{equation}

令 $F(x, y, c)=\pm\sin(x+c)-y$,则 $F_c(x, y, c)=\pm\cos(x+c)$ 包络线的判别方程为
\begin{equation}\label{EnvSol_eq2}
\leftgroup{
    \pm\sin(x+c)-y=0\\
    \pm\cos(x+c)=0
}
\end{equation}

联立\autoref{EnvSol_eq2} 中两等式,得
\begin{equation}\label{EnvSol_eq3}
y=\pm 1
\end{equation}

则\autoref{EnvSol_eq3} 就是曲线族\autoref{EnvSol_eq4} 的包络线。

将\autoref{EnvSol_eq3} 代回\autoref{EnvSol_eq1} ,发现它也是其解。这就是一个奇解。


\end{example}

另一种判别方式如下所述:对于微分方程 $F(x, y, \frac{\dd y}{\dd x})=0$,写出方程组
\begin{equation}\label{EnvSol_eq5}
\leftgroup{
    F(x, y, c)=0\\
    F_c(x, y, c)=0
}
\end{equation}
则从\autoref{EnvSol_eq5} 中消去 $c$ 所得的曲线 $M(x, y)=0$ 就\textbf{有可能}是该微分方程的奇解。至于它是不是奇解,还需要代回原方程进行验证。
\addTODO{这个判别式是来自于一个存在与唯一性定理的,原理是违反唯一性定理要求从而存在奇解。需要有相关定理的词条。}


\begin{example}{}
方程
\begin{equation}\label{EnvSol_eq7}
\qty(\frac{\dd y}{\dd x})^2+2x\frac{\dd y}{\dd x}-y=0
\end{equation}
有奇解吗?

如果我们解方程的话,得到其参数形式的通解\footnote{见\autoref{ODEa4_ex2}~\upref{ODEa4}。}
\begin{equation}
\leftgroup{
    \begin{aligned}
    v^2x+\frac{2}{3}v^3&=C\\
    v^2+2xv&=y
    \end{aligned}
}
\end{equation}
很难把它表示成显式的解,从而不好求包络线。

但是,我们可以用\autoref{EnvSol_eq5} ,将 $F(x, y, c)=c^2+2xc-y$ 代进去即可得到满足判别式的曲线:
\begin{equation}\label{EnvSol_eq6}
y=-x^2
\end{equation}

把\autoref{EnvSol_eq6} 代回\autoref{EnvSol_eq7} 验证,发现并不是一个解,从而不是奇解。而奇解必须满足判别式,因此该方程没有奇解。

\end{example}























