% 麦克斯韦方程组(介质)
% keys 麦克斯韦方程组|介质磁导率
% license  Usr
% type Tutor

% 麦克斯韦方程组(介质)
% 麦克斯韦方程组|介质磁导率
\footnote{本文参考了\cite{GriffE}与周磊教授的《电动力学》讲义}
\begin{issues}
\issueDraft
\end{issues}

\pentry{麦克斯韦方程组\upref{MWEq}}{nod_ae6e}
介质中的麦克斯韦方程:
\begin{equation}\label{eq_MWEq1_1}\ali{
&\div\bvec D = \rho_f~,\\
&\curl\bvec E = -\pdv{\bvec B}{t}~,\\
&\div\bvec B = 0~,\\
&\curl\bvec H = J_f + \pdv{\bvec D}{t}~.
}\end{equation}

其中$\bvec D$为“电位移矢量”,$\bvec H$为“磁场强度”\footnote{有一些作者认为$\bvec D, \bvec H$仅仅是数学工具而没有实际的物理含义,因为电荷或电流最终只能感受到$\bvec E, \bvec B$这类“真正的场”。现在,不少人称呼$\bvec B$为“磁场强度”},$\rho_f$是自由电荷,$\bvec j_f$是自由电流\footnote{即除了因介质极化而产生的极化电荷、极化电流}. 根据"电介质 \upref{dieleS} \upref{Dielec}"与"磁介质\upref{MagMat}"的物理模型,他们被定义为
\begin{align}
\bvec D &= \epsilon_0 \bvec E + \bvec P~,\\
\bvec H &= \frac{\bvec B}{\mu_0} - \bvec M~.\\
\end{align}

%我的理解是这样的,需要dalao check一下。
至少目前,\autoref{eq_MWEq1_1} 是普适的,因为推导他的过程中我们没有过多地假设介质的性质(没有说明$\bvec P$与$\bvec E$的关系,等等)。

\subsection{均匀线性介质的特例}
在\textbf{各向同性、非铁磁性的均匀线性介质}中,极化电偶、电流密度与场\footnote{这里的场指总的场,即外场与极化场之和。对于电荷与介质而言,不论是外场还是极化场,他们的效果都是“真实而相同”的}之间有线性关系\footnote{或许你会感觉这个定义有点别扭。这有一定的\textsl{历史原因}。最早人们把$\bvec H$看作“真正的场”,因此定义了$\bvec M = \chi_m \bvec H$与$\bvec B = \mu \bvec H$}:
\begin{align}
\bvec P &= \epsilon_0 \chi_E \bvec E~,\\
\bvec M &= \frac{1}{\mu_0} \frac{\chi_B}{1+\chi_B}\bvec B~.\\
\end{align}

因此,有构成关系:
\begin{align}
\bvec D &= \epsilon \bvec E = \epsilon_0 \epsilon_r \bvec E = \epsilon_0(1 + \chi_E)\bvec E~,\\
\bvec H &= \frac{\bvec B}{\mu} = \frac{\bvec B}{\mu_0\mu_r} = \frac{\bvec B}{\mu_0(1 + \chi_B)}~.\\
\end{align}
其中$\epsilon_r$为相对介电常数,$\mu_r$为相对磁导率,是与物质种类有关的物理量。%与介质性质有关的量(极化电偶、极化电流等)都被包装为$\epsilon, \mu$了。

\textbf{此时},麦克斯韦方程还可以写为:
\begin{equation}\ali{
&\div\bvec E = \frac{\rho_f}{\epsilon}~,\\
&\curl\bvec E = -\pdv{\bvec B}{t}~,\\
&\div\bvec B = 0~,\\
&\curl\bvec B = \mu J_f + \mu\epsilon\pdv{\bvec E}{t}~.
}\end{equation}
注意他的适用范围(例如,不能在跨越两种不同的电介质的区域内使用),以及和“真空”麦克斯韦方程组的相似与不同
% 在各向同性线性介质中,有 $\bvec P = \chi_E \epsilon_0 \bvec E$,  $\bvec M = \chi_B \bvec H$。  代入上式得 $\bvec D = (1 + \chi_E)\epsilon_0\bvec E$ 和  $\bvec H = \frac{\bvec B}{(1 + \chi_B)\mu_0}$。 

% 定义相对介电常数为 $\epsilon_r = 1 + \chi_E$, 相对磁导率为 $\mu_r = 1 + \chi_B$, 则 $\bvec D = \epsilon_r\epsilon_0\bvec E$, $\bvec H = \frac{\bvec B}{\mu_r\mu_0}$,  
