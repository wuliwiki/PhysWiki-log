% 反氢原子
% license CCBYSA3
% type Wiki

(本文根据 CC-BY-SA 协议转载自原搜狗科学百科对英文维基百科的翻译)

\begin{figure}[ht]
\centering
\includegraphics[width=6cm]{./figures/d42565f64ca382e0.png}
\caption{反氢由一个反质子和一个正电子组成} \label{fig_FQYZ_2}
\end{figure}
\textbf{反氢(H)}是对应元素氢的反物质。普通的氢原子由电子和质子组成,而反氢原子则由正电子和反质子组成。科学家们希望能够通过研究反氢来揭示在可观测到的宇宙中物质比反物质更多的原因,这也就是众所周知的重子不对称问题。[1] 反氢是在粒子加速器中人工合成的。1999年,美国国家航空航天局(NASA)估计每克反氢物质的成本为62.5万亿美元(相当于今天的94万亿美元),使其成为生产成本最高的材料。[2] 这是由于每次实验的产量极低,而使用粒子加速器的机会成本却很高。

\subsection{实验历史}
20世纪90年代,首次在加速器中检测到热的反氢。ATHENA项目在2002年对冷氢进行了研究。2010年,反氢首次被欧洲核子研究中心 (CERN)[3][4] 的Antihydrogen Laser Physics Apparatus (代号:ALPHA)研究团队捕获,该团队随后测量了反氢的结构和其他重要性质。[5] ALPHA、AEGIS和GBAR计划进一步地降低温度、研究反氢原子。
\subsubsection{1.1 1S–2S跃迁测量}
2016年,ALPHA实验测量了反氢的两个最低能级1S–2S之间的量子跃迁。在实验分辨率范围内,这些结果与氢的结果相同,支持了物质-反物质的对称性和CPT对称性等观点。[6]

磁场存在时,从1S到2S的能级跃迁分裂成两个频率略有不同的超精细跃迁。该团队计算了限制体积中磁场存在下正常氢的跃迁频率,如下所示:

$f_{dd} =2 466 061 103 064 (2) kHz$

$f_{cc} =2 466 061 707 104 (2) kHz$

量子选择规则禁止S态之间的单光子跃迁,因此为了将基态正电子提升到2S能级,通过限制空间,并使其被调谐到计算跃迁频率一半的激光照射,以激发允许双光子吸收。

被激发到2S能态的反氢原子可以以几种方式之一进化:
\begin{itemize}
\item 可以通过发射两个光子,直接返回基态
\item 可以吸收另一个光子,使原子电离
\item 可以发射单个光子,并经由2P态返回基态——在这种情况下,正电子自旋可能翻转或保持不变。
\end{itemize}
电离和自旋翻转的结果都会导致原子脱离束缚。该研究团队计算出,假设反氢的行为与正常氢相似,与无激光的情况相比,大约一半的反氢原子会在共振频率暴露过程中丢失。当激光源调谐到低于一半跃迁频率200 kHz时,计算出的损耗与无激光情况下的损耗基本相同。

ALPHA研究团队制造了多批反氢原子,将其保持600秒,然后在1.5秒内逐渐缩小限制场,同时计算有多少反氢原子湮灭。该实验在三种不同的实验条件下进行:
\begin{itemize}
\item 共振:–将受限的反氢原子暴露在激光源下,在两次跃迁中,每次跃迁时间为300秒,激光源的调谐频率恰好为跃迁频率的一半。
\item 无共振:–将受限的反氢原子暴露在低于两个共振频率200 kHz的激光源下各300秒,
\item 无激光:–在没有任何激光照射的情况下限制反氢原子。
\end{itemize}
需要无共振和非激光这两种控制手段来确保激光照射本身不会导致湮灭,或者可以通过从限制容器表面释放正常原子,然后使这些原子与反氢结合来实现。

该团队对三个案例进行了11次测试,发现非共振和无激光测试之间没有显著差异,但是共振测试后检测到的事件数量下降了58\%。他们还能够在运行期间计算湮灭事件,并在共振运行期间发现更高的水平,同样在非共振运行和无激光运行之间没有显著差异。该结果与基于正常氢的预测非常一致,可以被“理解为为在200 ppt的精度下进行的CPT对称性测试”。[7]

\subsection{特征}
粒子物理的CPT定理预测反氢原子具有许多常规氢所具有的特征;即相同的质量、磁矩和原子态跃迁频率等。[8] 例如,被激发的反氢原子预计会发出与普通氢相同的颜色。反氢原子应该受到其他物质或反物质吸引,引力大小与普通氢原子相同。[3] 这一结论当在反物质具有负引力质量时将是不成立的,不过通常认为这种情况是极不可能发生的,尽管尚没有经验证明这一观点。[9]

当反氢与普通物质接触时,其成分会迅速消失。正电子与电子湮灭,产生伽马射线。另一方面,反质子由反夸克组成,反夸克与中子或质子中的夸克结合,将产生高能π介子,并很快衰变为μ子、中微子、正电子和电子。如果反氢原子悬浮在完美的真空中,它们理应可以无限期存活。

作为一种反元素,可以推断其与氢元素具有完全相同的性质。[10] 例如,在标准条件下,反氢是一种气体,与反氧结合形成反物质形态的水。

\subsection{制备}
1995年,欧洲核子研究中心(CERN)的沃尔特·奥勒特领导的一个小组制备了第一个反氢原子。[11] 生产过程使用了小查理·孟格、斯坦利·布罗德斯基和伊万·施密特·安德拉德等首先提出的方法。[12]

在LEAR项目中,来自加速器的反质子被射向氙原子团簇,产生电子-正电子对。反质子捕获正电子的概率约为10-19,因此根据计算,这种方法不适合大量生产,[13] [14][15][16] 费米国立加速器实验室(Fermilab)测量了略有不同的横截面,[17]其结果与量子电动力学的预测一致。[18] 然而这些研究都会产生或高能、或高温的反原子,因此不适合详细研究。

随后,欧洲核子研究中心(CERN)建造了反质子减速器(AD),以支撑低能量反氢的研究,并且用于测试基本对称性。该反质子减速器为几个欧洲核子研究组织提供服务。欧洲核子研究中心预计他们的设备每分钟能够生产1000万个反质子。[19]
\subsubsection{3.1 低能反氢}
欧洲核子研究中心的ATRAP和ATHENA项目联合开展的实验将Penning陷阱中的正电子和反质子聚集在一起,使得合成速度为每秒100个反氢原子。反氢原子首先由ATHENA项目在2002年开始研究,[20] 接着是ATRAP项目,[21]一直到2004年,百万数量级的反氢原子被制造出来。合成的原子具有相对较高的温度(几千K),并且会撞击到实验装置的壁面而湮灭。此外,大多数精度测试需要较长的观察时间。

ATHENA项目的后继者ALPHA项目是为了稳定捕获反氢而成立的。[19] 电中性时,它的自旋磁矩与不均匀磁场相互作用;一些原子会被镜面和多极场的结合所产生的最小磁性吸引。[22]

2010年11月,ALPHA联合机构宣布他们已经捕获了38个反氢原子,[23] 持续时间为六分之一秒,这是第一次捕获中性反物质。2011年6月,他们捕获了309个反氢原子,同一时刻捕获量多达3个,持续时间长达1000秒。[24] 然后他们研究了它的超精细结构、重力效应和电荷。ALPHA项目也将继续测量,同时,ATRAP、AEGIS和GBAR等实验也在持续开展。
\subsubsection{3.2 更大的反物质原子}
更大的反物质原子则更难产生,如反氘(antideuterium)、反氚(antitritium)、反氦(antihelium)等,反氘、[25][26] 反氦-3 ($^3He$)[27][28] 和反氦-4 ($^4He$)原子核[29] 的生成速率非常之快,以至于合成这些原子的过程中有着许多的技术障碍。

\subsection{参考文献}
[1]
^BBC News – Antimatter atoms are corralled even longer. Bbc.co.uk. Retrieved on 2011-06-08..

[2]
^"Reaching for the stars: Scientists examine using antimatter and fusion to propel future spacecraft". NASA. 12 April 1999. Retrieved 11 June 2010. Antimatter is the most expensive substance on Earth.

[3]
^Reich, Eugenie Samuel (2010). "Antimatter held for questioning". Nature. 468 (7322): 355. Bibcode:2010Natur.468..355R. doi:10.1038/468355a. PMID 21085144..

[4]
^eiroforum.org – CERN: Antimatter in the trap Archived 2月 3, 2014 at the Wayback Machine, December 2011, accessed 2012-06-08.

[5]
^Internal Structure of Antihydrogen probed for the first time. March 7, 2012..

[6]
^Castelvecchi, Davide (19 December 2016). "Ephemeral antimatter atoms pinned down in milestone laser test". Nature. Retrieved 20 December 2016..

[7]
^Ahmadi, M; et al. (19 December 2016). "Observation of the 1S–2S transition in trapped antihydrogen" (PDF). Nature. 541 (7638): 506–510. Bibcode:2017Natur.541..506A. doi:10.1038/nature21040. PMID 28005057..

[8]
^Grossman, Lisa (July 2, 2010). "The Coolest Antiprotons". Physical Review Focus. 26 (1)..

[9]
^"Antihydrogen trapped for a thousand seconds". Technology Review. May 2, 2011..

[10]
^Palmer, Jason (14 March 2012). "Antihydrogen undergoes its first-ever measurement" – via www.bbc.co.uk..

[11]
^Freedman, David H. "Antiatoms: Here Today . ." Discover Magazine..

[12]
^Munger, Charles T. (1994). "Production of relativistic antihydrogen atoms by pair production with positron capture". Physical Review D. 49 (7): 3228–3235. Bibcode:1994PhRvD..49.3228M. doi:10.1103/physrevd.49.3228..

[13]
^Baur, G.; Boero, G.; Brauksiepe, A.; Buzzo, A.; Eyrich, W.; Geyer, R.; Grzonka, D.; Hauffe, J.; Kilian, K.; LoVetere, M.; Macri, M.; Moosburger, M.; Nellen, R.; Oelert, W.; Passaggio, S.; Pozzo, A.; Röhrich, K.; Sachs, K.; Schepers, G.; Sefzick, T.; Simon, R.S.; Stratmann, R.; Stinzing, F.; Wolke, M. (1996). "Production of Antihydrogen". Physics Letters B. 368 (3): 251ff. Bibcode:1996PhLB..368..251B. doi:10.1016/0370-2693(96)00005-6..

[14]
^Bertulani, C.A.; Baur, G. (1988). "Pair production with atomic shell capture in relativistic heavy ion collisions" (PDF). Braz. J. Phys. 18: 559..

[15]
^Bertulani, Carlos A.; Baur, Gerhard (1988). "Electromagnetic processes in relativistic heavy ion collisions". Physics Reports. 163 (5–6): 299. Bibcode:1988PhR...163..299B. doi:10.1016/0370-1573(88)90142-1..

[16]
^Aste, Andreas; Hencken, Kai; Trautmann, Dirk; Baur, G. (1993). "Electromagnetic Pair Production with Capture". Physical Review A. 50 (5): 3980ff. Bibcode:1994PhRvA..50.3980A. doi:10.1103/PhysRevA.50.3980. PMID 9911369..

[17]
^Blanford, G.; Christian, D.C.; Gollwitzer, K.; Mandelkern, M.; Munger, C.T.; Schultz, J.; Zioulas, G. (December 1997). "Observation of Atomic Antihydrogen". Physical Review Letters. Fermi National Accelerator Laboratory. 80 (14): 3037. Bibcode:1997APS..APR.C1009C. doi:10.1103/PhysRevLett.80.3037. FERMILAB-Pub-97/398-E E862 ... p and H experiments.

[18]
^Bertulani, C.A.; Baur, G. (1998). "Antihydrogen production and accuracy of the equivalent photon approximation". Physical Review D. 58 (3): 034005. arXiv:hep-ph/9711273. Bibcode:1998PhRvD..58c4005B. doi:10.1103/PhysRevD.58.034005..

[19]
^Madsen, N. (2010). "Cold antihydrogen: a new frontier in fundamental physics". Philosophical Transactions of the Royal Society A. 368 (1924): 3671–82. Bibcode:2010RSPTA.368.3671M. doi:10.1098/rsta.2010.0026. PMID 20603376..

[20]
^Amoretti, M.; et al. (2002). "Production and detection of cold antihydrogen atoms". Nature. 419 (6906): 456–9. Bibcode:2002Natur.419..456A. doi:10.1038/nature01096. PMID 12368849..

[21]
^Gabrielse, G.; et al. (2002). "Driven Production of Cold Antihydrogen and the First Measured Distribution of Antihydrogen States" (PDF). Phys. Rev. Lett. 89 (23): 233401. Bibcode:2002PhRvL..89w3401G. doi:10.1103/PhysRevLett.89.233401..

[22]
^Pritchard, D. E.; Heinz, T.; Shen, Y. (1983). "Cooling neutral atoms in a magnetic trap for precision spectroscopy". Physical Review Letters. 51 (21): 1983. Bibcode:1983PhRvL..51.1983T. doi:10.1103/PhysRevLett.51.1983..

[23]
^Andresen, G. B. (ALPHA Collaboration); et al. (2010). "Trapped antihydrogen". Nature. 468 (7324): 673–676. Bibcode:2010Natur.468..673A. doi:10.1038/nature09610. PMID 21085118..

[24]
^Andresen, G. B. (ALPHA Collaboration); et al. (2011). "Confinement of antihydrogen for 1,000 seconds". Nature Physics. 7 (7): 558. arXiv:1104.4982. Bibcode:2011NatPh...7..558A. doi:10.1038/nphys2025..

[25]
^Massam, T; Muller, Th.; Righini, B.; Schneegans, M.; Zichichi, A. (1965). "Experimental observation of antideuteron production". Il Nuovo Cimento. 39: 10–14. Bibcode:1965NCimS..39...10M. doi:10.1007/BF02814251..

[26]
^Dorfan, D. E; Eades, J.; Lederman, L. M.; Lee, W.; Ting, C. C. (June 1965). "Observation of Antideuterons". Phys. Rev. Lett. 14 (24): 1003–1006. Bibcode:1965PhRvL..14.1003D. doi:10.1103/PhysRevLett.14.1003..

[27]
^Antipov, Y.M.; et al. (1974). "Observation of antihelium3 (in Russian)". Yadernaya Fizika. 12: 311..

[28]
^Arsenescu, R.; et al. (2003). "Antihelium-3 production in lead-lead collisions at 158 A GeV/c". New Journal of Physics. 5: 1. Bibcode:2003NJPh....5....1A. doi:10.1088/1367-2630/5/1/301..

[29]
^Agakishiev, H.; et al. (2011). "Observation of the antimatter helium-4 nucleus". Nature. 473 (7347): 353–6. arXiv:1103.3312. Bibcode:2011Natur.473..353S. doi:10.1038/nature10079. PMID 21516103..