% GCC 源码笔记
% license Usr
% type Note

\begin{itemize}
\item GCC 采用前后端分离,高级中间表示为 \textbf{GIMPLE(GNU SIMPLE)}, 低级中间表示为 \textbf{RTL(Register Transfer Language)}。
\item RTL 是部分机器相关的,也就是并不是完全机器无关。
\item GIMPLE 是 SSA(Static Single Assignment)形式的中间表示,使用 phi 函数,使用 \verb`goto` 和 basic block 进行流控制。 C 程序
\begin{lstlisting}[language=cpp]
int foo(int n) {
    int sum = 0;
    for (int i = 0; i < n; i++) {
        if (i % 2 == 0) {
            sum += i;
        }
    }
    return sum;
}
\end{lstlisting}
对应的 GIMPLE 代码是
\begin{lstlisting}[language=cpp]
foo (int n) {
  int sum;
  int i;
  int D.1234;
  int D.1235;
  int D.1236;

  sum = 0;
  i = 0;
  goto L2;

L1:
  D.1234 = i % 2;
  if (D.1234 == 0) goto L3; else goto L4;

L3:
  sum = sum + i;
  goto L4;

L4:
  i = i + 1;
  goto L2;

L2:
  if (i < n) goto L1; else goto L5;

L5:
  return sum;
}
\end{lstlisting}
另一个 C 源码
\begin{lstlisting}[language=cpp]
int bar(int x, int y) {
    int z;
    if (x > 0) {
        if (y > 0) {
            z = x + y;
        } else {
            z = x - y;
        }
    } else {
        z = y - x;
    }
    return z;
}
\end{lstlisting}
对应的 GIMPLE 使用了 phi 函数
\begin{lstlisting}[language=cpp]
bar (int x, int y) {
  int z_1;
  int z_2;
  int z_3;
  int z_4;
  int D.1234;
  int D.1235;

  D.1234 = x > 0;
  if (D.1234 != 0) goto L1; else goto L4;

L1:
  D.1235 = y > 0;
  if (D.1235 != 0) goto L2; else goto L3;

L2:
  z_1 = x + y;
  goto L5;

L3:
  z_2 = x - y;
  goto L5;

L4:
  z_3 = y - x;
  goto L5;

L5:
  # z_4 = PHI <z_1(L2), z_2(L3), z_3(L4)>
  return z_4;
}
\end{lstlisting}
\end{itemize}
