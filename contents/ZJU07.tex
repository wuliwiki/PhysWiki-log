% 浙江大学 2007 年 考研 量子力学
% license Usr
% type Note

\textbf{声明}:“该内容来源于网络公开资料,不保证真实性,如有侵权请联系管理员”

\subsection{第一题(50 分)简答题:}
\begin{enumerate}
  \item 写出泡利矩阵的形式。
  \item 量子力学中的可观察量算符为什么要求是厄米算符?
  \item 放射性指的是某些原子核中的更小粒子有一定的概率逃逸出来,你认为这与什么量子效应有关?
  \item 试求质量为 $m$ 的粒子处在长度为 $L$ 的一维盒子(可看成是无限深势阱)中,试求他对各壁的压力。
  \item 自发辐射和受激辐射的区别是什么?
  \item 写出测测不准关系,并简要说明其物理含义。
  \item 请分别列出下列三种能级对应的量子系统:
  \[    E_n' \propto \frac{1}{n^2}, \quad E_n'' \propto n^2, \quad E_n''' \propto n ~\]
  \item $\hat{H} = \hat{H_0} + \hat{H'}$,设 $\Psi_n$ 为 $\hat{H_0}$ 的能量本征值为 $E_n$ 的非简并本征函数,如果 $\hat{H'}$ 可作微扰,试写出能级的微扰修正公式(写到二级修正)。
\end{enumerate}
\subsection{第二题(25 分):}
有一个质量为$m$ 的粒子处在如下势阱中

\[
V(x) =
\begin{cases} 
    \infty & x < 0 \\
    -V_0 & 0 < x < 0 \\
    0 & a < x < a + b \\
    V_0 &  a + b < x
\end{cases}~
\]
这里\( V_0 > 0 \).

1. 求能级与波函数。

2. 你认为通过调整 \( a \) 和 \( b \) 中的哪一个参数值可以让势阱中的粒子有一定的概率穿透出来,为什么?
\subsection{第三、四题(25 分+25 分)从如下(A)、(B)、(C)中选做二个即可!}
\textbf{(A)} 求屏蔽库伦场 $W(r) = \frac{b}{r} e^{-r/a}$ 的微分散射截面(提示:可直接用中心势散射的玻恩近似公式的化简形式)。

\textbf{(B)} 用分波法求势场 $V(r) = 
\begin{cases}
 0 & r > a \\\\
\infty & r \leq a 
\end{cases}$ 散射的S波相移。

\textbf{(C)} 有一种冷原子具有两个能级简并的态 $|1\rangle$ 和 $|2\rangle$,最近科学家在他们的冷原子“暗态”实验中引入的激光场的效应描述为如下微扰哈密顿量,
$$H' = \begin{pmatrix} w_1 w_1 & w_1 w_2 \\\\ w_2 w_1 & w_2 w_2\end{pmatrix}~$$
求出该微扰哈密顿量引起的能级修正和所得对应本征态。
\subsection{第五题(25 分):}
电子被束缚在简谐振子势场中: \( V(r) = \frac{1}{2} m \omega^2 x^2 ~\) ,若引入

\[\hat{a} = \left(\frac{m \omega}{2 \hbar} \right)^{1/2} \left( \hat{x} + \frac{i}{m \omega} \hat{p} \right), \quad \hat{a}^\dagger = \left(\frac{m \omega}{2 \hbar} \right)^{1/2} \left( \hat{x} - \frac{i}{m \omega} \hat{p} \right),~\]

则有 \( H = \hbar \omega (\hat{a}^\dagger \hat{a} + \frac{1}{2}) ~\) ,并有

\[\hat{a}^\dagger \lvert n \rangle = \sqrt{n+1} \lvert n+1 \rangle , \quad \hat{a} \lvert n \rangle = \sqrt{n} \lvert n-1 \rangle ~\]

显然基态 \(\lvert 0 \rangle\) 应满足 \( \hat{a} \lvert 0 \rangle = 0 ~\) 。

1. 试求基态波函数 \( \psi_0 \) 和第 1 激发态的波函数 \( \psi_1 \)。

2. 如果该势阱中有两个电子(忽略它们间的相互作用),写出它们的基态波函数(提示:电子的自旋为 \(1/2\) 的全同粒子)。

3. 如果加入均匀磁场 \( B \),问当 \( B \) 很强,超过某临近临界 \( B \) 时,上述基态还会是基态吗?是具体求出 \( B \)。

