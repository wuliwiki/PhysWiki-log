% 东南大学 2008 年 考研 量子力学
% license Usr
% type Note

\textbf{声明}:“该内容来源于网络公开资料,不保证真实性,如有侵权请联系管理员”

\textbf{1.(15分)}量子体系守恒量完全集 $\{H, A, \dots\}$ 的归一化共同本征态为 $|n\rangle$,($n=1, 2, \dots$),即 $H|n\rangle = E_n|n\rangle$,$A|n\rangle = A_n|n\rangle$,……
\begin{enumerate}
    \item (1) 用 $|n\rangle$ 分别写出正交归一性和完备性的表达式;
    \item (2) 设 $t=0$ 时刻的归一化量子态为 $|\psi(0)\rangle = \sum_n c_n |n\rangle$,试求 $|\psi(t)\rangle$;
    \item (3) 求 $t$ 时刻的平均值 $E(t)$ 和 $A(t)$。
\end{enumerate}
\textbf{2.(15 分)}质量为 $m$ 的粒子作一维运动,处于势阱 $V(x) = -\gamma \delta(x)$ ,( $\gamma > 0$ )。求束缚态的归一化波函数和束缚态能级。

\textbf{3.(15 分)}设质量为 $m$ 的粒子以能量 $E > 0$ 从左入射,碰到下列势阱,求反射系数和透射系数。提示:几率流密度为 $j(x) =(\hbar/i2m)(\psi^*\psi'  - \psi^{*'}\psi)$。
$$V(x) = \begin{cases} 
-V_0, & x < 0 \\ 
0, & x > 0 
\end{cases}~$$

\textbf{4.(15 分)}
设粒子处在 $\hat L^2,\hat L_z$ 的共同本征态 $|lrm\rangle$。

\begin{enumerate}
    \item (1) 利用角动量的对易关系证明 $L_x = L_y = 0$。
    \item (2) 求 $\overline{(\Delta L_z)^2}$ 与 $\overline{(\Delta L_y)^2}$,其中 $\Delta L_z = \hat L_z -\hat L_z $, $\Delta L_y = \hat L_y -\hat L_y$。
\end{enumerate}

\textbf{5.(15分)}以下对称性是否导致一个守恒量,如果是,请指出相应的守量\\
(1)空间反演对称性: (2)空间平移对称性; (3)空间转动对称性:(4)时间反演对称性:(5)时间平移对称性。

\textbf{6.(15)}设缺金属原子的价电子处于中心力场 $V(r)$ 中,守恒量完全集 $\{\hat H, \hat{l}^2, l_z\}$ 的共同本征函数数为 $\psi_{nlm}(r, \theta, \phi) = R_{nl}(r)Y_{lm}(\theta, \phi)$,能量本征值为 $E_{nl}$。

\begin{enumerate}
    \item 若外加强磁场 $B$,求能量本征值。提示:价电子的朗德因子为
  $$H = \frac{\hat p^2}{2\mu} + V(r) + \omega_L \hat l_z, \quad (\omega_L = \frac{eB}{2\mu c})~$$
    \item 求有磁场 $B$ 时,原子从 $p$ 态($l = 1$)跃迁到 $s$ 态($l = 0$)的光谱线频率(跃迁过程保持 $n_e$ 不变)。
\end{enumerate}

\textbf{7.(15分)}设体系由 2 个全同粒子组成,每个粒子可处于 3 个单粒子态 $|i\rangle, |j\rangle, |k\rangle$ 中的任何一个,分别在以下两种情况下写出体系可能的量子态:
\begin{enumerate}
    \item 全同 Bose 子;
    \item 全同 Fermi 子。
\end{enumerate}

\textbf{8.(15分)}设电子自旋角动量算符为 $\hat{S} = (\hbar/2) \hat{\mathbf{\sigma}}$。$\hat{\sigma}_z$ 的正交归一本征态为 $|\pm\rangle$,即 $\hat{\sigma}_z|\pm\rangle = \pm|\pm\rangle$。求 $\hat{\sigma}_x|\pm\rangle, \hat{\sigma}_y|\pm\rangle, $。

\textbf{9.(15分)}设体系的未微扰哈密顿量 $H_0$ 和微扰哈密顿量 $H'$ 分别为

\begin{equation}
H_0 = \begin{pmatrix} E_1^{(0)} & 0 \\\\ 0 & E_2^{(0)} \end{pmatrix}, \quad (E_1^{(0)} \neq E_2^{(0)}), \quad H' = \begin{pmatrix} a & b \\\\ b^* & a \end{pmatrix}~
\end{equation}
试用微扰论求能级的修正(准确到二级近似)。\\
提示:非简并微扰论的能级修正公式为\\
\begin{equation}
E_k = E_k^{(0)} + \langle k | \hat H' | k \rangle + \sum_{n \neq k} \frac{\langle k | \hat H' | n \rangle \langle n |\hat H' | k \rangle}{E_k^{(0)} - E_n^{(0)}}~
\end{equation}

\textbf{10.(15分)}一质量为 $m$,空间位置固定的电子处于沿 $x$方向的均分磁场 $ B$方向 中,其哈密顿算符(不计轨道运动)为$\hat{H} = \frac{eB}{mc} \hat{S}_z $,其中 $\hat{S}_z$ 为电子自旋角动量算符。已知 $t = 0$ 时电子的自旋态为 $S_z$ 的本征态,相应的本征值为 $\hbar/2$ ,试求:
\begin{enumerate}
    \item  t (> 0) 时刻该电子的自旋态;
    \item  t (> 0) 时刻 $\hat{S}$的平均值。
\end{enumerate}

提示:泡利矩阵为

$$\sigma_x = \begin{pmatrix}0 & 1 \\\\1 & 0\end{pmatrix}, \quad\sigma_y = \begin{pmatrix}0 & -i \\\\i & 0\end{pmatrix}, \quad\sigma_z = \begin{pmatrix}1 & 0 \\\\0 & -1\end{pmatrix}~$$