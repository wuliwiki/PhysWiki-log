% 苏州大学 2011 年硕士物理考试试题
% keys 苏州大学|考研|物理|2011年
% license Copy
% type Tutor
\textbf{科目代码:832}
\begin{enumerate}
\item 两个长方形物体$A$和$B$紧靠放在光滑的水平桌面上,已知$m_A=2kg,m_B=3kg$,有一质量$ m=100g$的子弹以速率 $v_0=800m/s$水平射入长方体 $A$,经$t=0.01s$,又射入长方体$B$,最后停留在长方体$B$内未射出。设子弹射入$A$时所受的摩擦力$F_r=3x10^3N$,求:\\
(1)子弹在射入$A$的过程中,$B $受到$A$的作用力的大小;\\
(2)当子弹留在$ B$ 中时,$A$ 和$B$的速度大小。
\begin{figure}[ht]
\centering
\includegraphics[width=6cm]{./figures/eb79ab6f479f587b.png}
\caption{} \label{fig_SD11_3}
\end{figure}
\item 一长为$L$,质量为$m$的均质细棒,一端可绕固定的水平光滑轴$O$在竖直平面内转动,在$O$点还系有一长为$b(b<L)$的细绳,绳的另一端悬挂一质量也为$m$的小球。当小球悬线偏离竖直方向某一角度时,由静止释放。已知小球与细棒发生完全弹性碰撞,要使碰撞后小球刚好停止,问绳的长度$b$应为多少?
\begin{figure}[ht]
\centering
\includegraphics[width=6cm]{./figures/a9f3abbb6cbb23f2.png}
\caption{} \label{fig_SD11_4}
\end{figure}
\item 某火车驶过车站时,站台上的观测者测得火车汽笛频率由 $1200Hz$变到了 $1000 Hz$,设空气中声速为 $330m/s$,求该火车的速率。
\item 一列机械波沿$x$轴正向传播,$t=0$时的波形如图所示,已知波速为 $10m/s$,波长为 $2m$,求:\\
(1)波动方程;\\
(2)$P$点的振动方程及振动曲线;\\
(3)$P $点的坐标。
\begin{figure}[ht]
\centering
\includegraphics[width=8cm]{./figures/953a1ea5b481a5a4.png}
\caption{} \label{fig_SD11_5}
\end{figure}
\item 地面上有一固定的点电荷$ A$,在$A$的正上方有一带电小球$B$,$B$在重力和$A$的库仑斥力的作用下,在$A$上方$ H/2 $到$H$之间作往返的自由振动。试求$B$运动的最大速率$v_{max}$。
\item  如图所示,一根塑料棒带有均匀分布的电荷$-Q$,塑料棒被弯曲成120°半径为$r$的圆弧。建立如图所示的坐标轴,原点在圆弧的曲率中心$P$点。求$P$点的场强$E$(用$Q$和$r$表示)。
\begin{figure}[ht]
\centering
\includegraphics[width=6cm]{./figures/a0e0d2b178b1790f.png}
\caption{} \label{fig_SD11_2}
\end{figure}
\item 一长直圆柱形导线由内外两种导电材料构成,截面如图所示,导线外半径$R_2$,内半径$R_1$,内外导体的电导率和磁导率分别为可$\sigma_1,\mu_1$和$\sigma_2,\mu_2$。导线中沿轴向通以电流$I$,求内外导体的磁场强度。
\begin{figure}[ht]
\centering
\includegraphics[width=6cm]{./figures/fa28a53467a3c371.png}
\caption{} \label{fig_SD11_1}
\end{figure}
\item 一导体球半径为 $R_1$,外罩一半径为$ R_2$的同心薄导体球壳,外球壳所带总电荷为$q$,而内球的电势为$ U_0$,求内球的所带的电量$ q$为何值。
\item 一光子与自由电子碰撞,电子可能获得的最大能量为60keV,求入射光子的波长和能量。
\item 在实验室参照系中,某个粒子具有能量$E=3.2x10^{-10}J$、动量$p=9.4x10^{-19}kg.m/s$,求该粒子的静止质量、速率和在粒子静止的参照系中的能量。
\item 太阳表面的能量辐出度为$6.87*10^7 W/m^2$。若太阳辐射被地球完全吸收,相应的地球表面的辐射压强是多大?若太阳光辐射被完全反射,辐射压强又是多大?将计算出的辐射压强值与大气压力作一比较。(已知太阳半径 $R_S=6.96*10^8m$,地球半径 $R_E=6.37*10^6m$,地球到太阳的距离 $d=1.496*10^{11}m$)
\item 已知钾的截止频率为 $4.84x10^{14}Hz$,求:\\
(1)钾的逸出功;\\
(2)在波长为 330nm 的紫外光照射下,钾的遏止电势差。




\end{enumerate}
有关常数:$m_0=9.1*10^{-32}kg,\quad h=6.63x10^{-34}J.S,\quad \mu_0=8.85x10^{-12}F/m,\quad \mu_0=1.26x10^{-6} H/m,
\quad R_H=1.097x10^7/m$
