% 拉马努金和(数论)
% keys 拉马努金和|Ramanujan|Ramanujan Sum|数论
% license Usr
% type Tutor

\pentry{数论三角和与高斯和\nref{nod_ntrtre}}{nod_8219}

\textbf{这里探讨的是数论中的拉马努金和(Ramanujan's Sum),非级数等中的拉马努金求和。}

\begin{definition}{拉马努金和}
\textbf{拉马努金和(Ramanujan's Sum)} $c(m; n)$ 定义为:
\begin{equation}
c(m; n) = c_n(m) = \sum_{h=1, \gcd(h, n)=1}^{n} e\left(\frac{mh}{n}\right) ~.
\end{equation}
\end{definition}
类似 Gauss 和的,可以发现,只要 $h$ 取遍 $n$ 的一个缩系即可,记为 $h^*(n)$。
\begin{theorem}{}
\begin{equation}
c_n(m) = \sum_{h^*(n)} e\left(mh/n\right) ~.
\end{equation}

\end{theorem}

\pentry{单位根与本原单位根(数论)\nref{nod_priru}}{nod_49e2}
\begin{theorem}{}
利用本原单位根,可以将 Ramanujan 和表示为
\begin{equation}
c_q(m) = \sum \rho^m~,
\end{equation}
其中 $\rho$ 取遍 $q$ 的所有本原 $q$ 次单位根。
\end{theorem}

利用缩系的\autoref{the_linmod_3}~\upref{linmod},将可以得到 Ramanujan 和的另一性质。
\begin{theorem}{}
若 $(q, q') = 1$,则 $c_{qq'}(m) = c_q(m) c_{q'}(m)$。
\end{theorem}
这是因为利用这性质将直接有
\begin{equation}
c_q(m) c_{q'}(m) = \sum_{h^*(q), h'^*(q')} e\left[ m\left(\frac{h}{q} + \frac{h'}{q'}\right) \right] = \sum_{h^*(q), h'^*(q')} e\left( \frac{m(hq' + h'q)}{q q'} \right) = c_{qq'}(m)~.
\end{equation}
证毕!

% 由定义就不难有 $c(m; n) = \sum_{d | \gcd(m, n)} (\mu(n/d) d)$。

% 特别的,当 $n | m$ 时,$c(m; n) = \varphi(n)$,是欧拉函数。
