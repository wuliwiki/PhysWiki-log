% C++ 标准库常用功能笔记

\begin{issues}
\issueDraft
\end{issues}

这里列举标准库中最常用的功能。

\subsection{string}
详见 “C++ 字符串笔记\upref{CppStr}”。

\subsection{vector}
更多容器详见 “C++ 标准库常用容器\upref{STLstr}”
\begin{itemize}
\item \verb`v.size()` 检查大小
\item \verb`v.empty()` 检查是否为空
\item range based for loop \verb`for (auto &i : v)`
\item iterator 类似指针, \verb`auto a = v.begin()`, 生成第一个元素的 iterator, \verb`v.end()` 生成最后一个元素后的一个 iterator. \verb`*a` 获取 a 指向的元素, \verb`*(a + n)` 获取 a 后面的第 n 个元素. \verb`==` 只有在 iterator 指向同一个 vector 的同一个元素时才成立.
\end{itemize}

\subsection{ctime}
\begin{itemize}
\item \verb`clock()` 用于测量 CPU 时, 而不是真正的时间. 如果在 ubuntu 下用 OpenMP, 时间将是所有 CPU 的累加.
\begin{lstlisting}[language=cpp]
clock_t start, stop;
start = clock();
...
stop = clock();
\end{lstlisting}
\item \verb`time()` 用于测量从 1970 年某时刻起流逝的秒数, 但仅限于整数秒.
若要精确测量物理时间, 用 chrono, 这个头文件用起来要复杂得多, 见我在 nr3plus.h 中写的 tic(), toc() 函数.
\end{itemize}

\subsection{algorithm}
\begin{itemize}
\item 提供 \verb`max()`, \verb`min()`, \verb`swap()` 交换两个变量值。
\end{itemize}


\subsection{stdlib}
\begin{itemize}
\item \verb`int system("命令")` 在调用程序的环境(如 linux 的 bash,windows 的 cmd)中执行命令,例如获得文件列表 \verb`ls`, 当前目录 \verb`pwd` 等。 但这些命令返回的值不能直接获得,需要 redirect 到文件再读取文件。
\item \verb`int remove("文件名")` 可以删除文件。
\end{itemize}
