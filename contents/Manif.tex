% 流形
% 拓扑|流形|欧几里得

\pentry{拓扑空间\upref{Topol}}

\subsection{拓扑流形}

\textbf{(实)拓扑流形(real topological manifold)} 是一种拓扑空间, 其每个点都有一邻域与与欧几里得空间中的开集同胚 (homeomorphic)。如果这些欧几里得空间是 $n$ 维的, 那么就叫做 $n$ 维流形。 因此,一个拓扑流形可以看成是我们熟知的欧几里得空间“拼接”而成的。

\begin{definition}{图和图册}\label{def_Manif_1}
设 $N$ 是一个拓扑空间,满足Hausdorff分离性以及第二可数性\footnote{拓扑空间如果有一个可数拓扑基(即一个拓扑基,包含最多 $\aleph_0$ 个基本开集),则称之为第二可数的。}。如果存在开集 $U\in\mathcal{T}_N$ 和一个正整数 $n$,使得 $U$ 同胚于 $\mathbb{R}^n$,同胚映射为 $\varphi:U\rightarrow\mathbb{R}^n$,那么称 $(U,\varphi)$ 是 $N$ 上的一张\textbf{图(chart)}。如果图的一个集合 $\mathcal{A}=\{(U_\alpha, \varphi_\alpha)\}$ 覆盖了 $N$,即 $\bigcup\{U_\alpha\}=N$,那么称这个集合 $\mathcal{A}$ 是一个\textbf{图册(atlas)}。
\end{definition}

图 $(U, \varphi)$ 可以看成是给 $U$ 中各点 $x$ 赋予了一个坐标值 $\varphi(x)$。

\begin{definition}{实拓扑流形}\label{def_Manif_2}

设 $N$ 是一个\textbf{道路连通的}拓扑空间,满足Hausdorff分离性以及第二可数性,且有一个图册 $\mathcal{A}$,则称 $(N, \mathcal{A})$ 是一个\textbf{实拓扑流形(real topological manifold)},简称\textbf{拓扑流形(topological manifold)}。

\end{definition}

从这些定义可看到,流形是“局部地”和欧几里得空间同胚的数学对象。低维欧几里得空间可以很方便地用我们的几何直觉来理解,大大方便了建立对于流形的直觉。

要注意的是,以上定义只要求了流形的每个点附近都有领域,使之局部地同胚于欧几里得空间,这样的流形被称为拓扑流形,但并不是我们将在微分几何中讨论的重点;我们将来讨论的重点概念是“光滑流形”。

\begin{example}{$S^n$ 流形}

$n$ 维球面 $S^n$ 都是实流形。
\begin{itemize}
\item 记 $S^1=\{(\cos{2\pi t},\sin{2\pi t})\in\mathbb{R}^2|t\in[0, 1]\}$,那么我们可以通过挖去一个点后使用正切函数来构造图:$U=S^1-\{(1,0)\}=\{(\cos{2\pi t},\sin{2\pi t})\in\mathbb{R}^2|t\in(0, 1)\}$ 可以同胚于 $\mathbb{R}$,同胚映射为 $\varphi(\cos{2\pi t},\sin{2\pi t})=\tan{\pi t-\pi/2}$;类似地,$V=S^1-\{(-1, 0)\}=\{(\cos{2\pi t},\sin{2\pi t})\in\mathbb{R}^2|t\in(-1/2, 1/2)\}$ 也可以同胚于 $\mathbb{R}$,同胚映射为 $\phi(\cos{2\pi t},\sin{2\pi t})= \tan{\pi t}$。这样一来,$\{(U, \varphi), (V, \phi)\}$ 就是 $S^1$ 的两个图的集合,并且覆盖 $S^1$,因此这个集合是一个图册,从而得出 $S^1$ 是一个实拓扑流形。
\item 类似地,从 $S^n$ 中分别挖去两个不同的点所得到的 $U$ 和 $V$ 都可以同胚于 $\mathbb{R}^n$,从而得到图册。


\end{itemize}

\end{example}

\begin{example}{莫比乌斯带}
将莫比乌斯带表示为矩形纸带两边扭转后粘合,我们可以将其用如图所示的 $5$ 个开集 $U_i$ 覆盖,每个开集都同胚于 $\mathbb{R}^2$,因此我们可以利用它们来构建一个含有 $5$ 个图的图册。
\begin{figure}[ht]
\centering
\includegraphics[width=5cm]{./figures/a6d5d50818dc4cea.pdf}
\caption{莫比乌斯带和覆盖它的五个开集。注意 $U1$(红色)和 $U2$(绿色)在图示中有“两块”区域,当然它们实际上是相连的。} \label{fig_Manif_1}
\end{figure}
\end{example}

\subsection{光滑流形}

考虑一个拓扑流形 $N$ 的两个图 $(U, \varphi)$ 和 $(V, \phi)$。如果 $U\cap V\not=\varnothing$,那么在 $U\cap V$ 中的每个点就有两套坐标;当然,也可以把 $\varphi\circ\phi^{-1}:\phi(V)\rightarrow\varphi(U)$ 和 $\phi\circ\varphi^{-1}:\varphi(U)\rightarrow\phi(V)$ 看成 $\mathbb{R}^n$ 的开子集上的多元向量值函数。只要这个函数是可以任意进行微分的,就能方便我们研究。

\begin{definition}{光滑函数}
设 $f:\mathbb{R}^n\rightarrow\mathbb{R}$ 是一个 $n$ 元函数,如果 $\frac{\partial^{k_1+k_2+\cdots+k_n}}{\partial^{k_1}x^1\partial^{k_2}x^2\cdots\partial^{k_n}x^n} f$ 在一点 $x\in\mathbb{R}^n$ 处对于任意非负整数 $k_i$ 成立,即 $f$ 的任意阶偏导数存在,那么称 $f$ 在 $x$ 处光滑;如果 $f$ 处处光滑,也称它是一个\textbf{光滑函数(smooth function)},记为 $C^\infty$ 的函数。如果向量值函数的各分量函数都是光滑函数,那么也称这个向量值函数是一个\textbf{光滑映射(smooth map)}。
\end{definition}

当一个函数的 $m$ 阶偏导数存在并且连续时,我们说它是 $C^m$ 的,因此任意阶偏导数存在时自然被记为 $C^\infty$。今后我们将“光滑”和 $C^\infty$ 视为同义词,不加区分。

需要注意的是,光滑和解析不是等价的概念。解析函数一定光滑,但光滑函数不一定解析,一个典型例子是\autoref{ex_SmthM_1}~\upref{SmthM}。可参考幂级数\upref{anal}.

\begin{definition}{相容}
考虑一个拓扑流形 $N$ 的两个图 $(U, \varphi)$ 和 $(V, \phi)$。如果 $U\cap V\not=\varnothing$,且 $\varphi\circ\phi^{-1}:\phi(V)\rightarrow\varphi(U)$ 和 $\phi\circ\varphi^{-1}:\varphi(U)\rightarrow\phi(V)$ 都是光滑映射,那么我们称这两个图是\textbf{相容的(compatible)}。
\end{definition}

从拓扑流形例子可以看出,同一个拓扑空间 $N$ 可以有不同的图册,对应的虽然是同一空间,但按照定义却是不同流形。这意味着流形不仅仅是指空间 $N$ 本身,还指它的局部坐标系结构。这样把拓扑流形分类通常是无意义的,实际上我们需要一个更为统一的研究对象,那就是光滑流形。我们接连使用以下两个定义来定义光滑流形。

\begin{definition}{极大图册}
设 $(N, \mathcal{A})$ 是一个拓扑流形。如果 $N$ 的任何一个图,满足“只要它和 $\mathcal{A}$ 中\textbf{所有图}都相容,它就一定在 $\mathcal{A}$ 中”,那么就称 $\mathcal{A}$ 是一个极大图册。
\end{definition}

\begin{definition}{光滑流形}\label{def_Manif_3}
拥有极大图册的拓扑流形,被称为一个\textbf{光滑流形(smooth manifold)}。光滑流形的图册,被称为该流形上的一个\textbf{微分结构(differential structure)}。
\end{definition}

% 复流形移动到它自己的词条

这就是我们将来讨论的主要对象了。今后如无特别说明,小时百科中“流形”一词都特指“光滑流形”。

通常,为了方便,我们也会将流形 $(N, \mathcal{A})$ 简单记为 $N$。





