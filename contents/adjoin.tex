% 伴随算符
% keys 伴随算符|算符|算子|厄米共轭
% license Xiao
% type Tutor

\begin{issues}
\issueDraft
\end{issues}

\pentry{厄米共轭\nref{nod_HerMat}}{nod_34fe}

% 参考泛函分析笔记。 这里讲的是伴随, 不是自伴

若一个线性算符对应的矩阵是厄米矩阵, 那么它就是一个

有限维的线性算符可以表示为

无限维的情况下, 和对称算符有什么区别? 注意是定义域必须是希尔伯特空间的稠密子空间。


\begin{equation}
\braket{Au}{v} = \braket{u}{A\Her v}~
\end{equation}
适合无穷维的情况, 也包括有限维的情况。 有限维时就是矩阵的共轭转置。

symmetric 和 adjoint: 一个定义域变大, 另一个变小, 等两个相等时就 adjoint 了
\begin{figure}[ht]
\centering
\includegraphics[width=10cm]{./figures/bf574f5bddbb8af9.png}
\caption{草图} \label{fig_adjoin_1}
\end{figure}
