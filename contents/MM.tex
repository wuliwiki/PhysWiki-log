% MM 模型

% 抱歉,百科需要申请成为志愿者才能编辑

近年来,随着我国证券市场的发展,越来越多的投资者开始购买股票、基金、期货期权等金融产品.而在购买金融产品时,不仅需要关注收益,也要注重规避风险,要有足够的投资知识积累,具备一定的投资组合理论,才能实现投资者自身利益的最大化.

投资组合理论是由美国经济学家马科维茨提出的,接下来,我们将运用马科维茨理论的投资组合模型,进行最优投资组合的选择与分析.

\subsection{(一)马科维茨投资组合理论}

\textbf{1. 马科维兹模型的假设条件}

(1)投资者在考虑每一次投资选择时,其依据是某一持仓时间内的证券收益的概率分布.

(2)投资者是根据证券的期望收益率估测证券组合的风险.

(3)投资者的决定仅仅是依据证券的风险和收益.

(4)在一定的风险水平上,投资者期望收益最大;相对应的是在一定的收益水平上,投资者希望风险最小.

\textbf{2.	数据来源}

(1)使用WIND数据库

(2)选取“601166 兴业银行”“601668 中国建筑”“600028 中国石化”“600104 上汽集团”“601288 农业银行”“601319 中国人保”“600588 用友网络”“600050 中国联通”“601138 工业富联”“600016 民生银行”十只股票进行研究.

(3)选用20160601-20210531的周收益率和2\%的无风险收益率作为参考.

\textbf{3.	模型建立}

假设所有投资者都是理性的,首先确定风险投资权重组合,然后找出最优化的风险投资组合权重比例.

模型如下:

\begin{equation}
Max S=\frac {E(R_P)-r_f}{\sigma_P}
\end{equation}

其中,

\begin{equation}
E(R_P)=w_1E(R_1)+w_2E(R_2)+\cdots+w_{10}E(R_{10})
\end{equation}

\begin{equation}
r_f=2\%
\end{equation}

\begin{equation}
\sigma=\sqrt{{w_1}^2 {\sigma_1}^2 + {w_2}^2 {\sigma_2}^2 + \cdots +{w_{10}}^2 {\sigma_{10}}^2}
\end{equation}

\begin{equation}
w_1,w_2,\cdots,w_{10}>0
\end{equation}

$R_P$ 为股票组合的投资收益,$R_i$ 为第i支股票的收益,$w_i,w_j$ 为有价证券i、j的投资比例,$\sigma_P$ 为组合投资的方差(表示组合总投资风险).

模型的限制条件为:

(1)允许卖空:$\sum_{i=1}^{10} w_i=1$

(2)不允许卖空:$0\leq\sum_{i=1}^{10} w_i=1$

\subsection{(二)马科维茨模型的实现}

\textbf{1. 组合收益与风险计算}

采用十支股票的周收益率数据进行研究,样本容量为2312,数据来源于WIND数据库,样本区间为2016年6月1日~2021年5月31日.

\textbf{2. 有效前沿计算}

利用Matlab计算有效前沿函数,将多目标优化问题转换为单目标优化问题,即给定期望收益***,计算相应风险最小的组合,得到有效前沿上的一点(有效组合),给定一系列 $e_k$,可以有效描述出有效前沿.


\subsection{(三)马科维茨模型结果分析}

在画出投资组合的有效前沿后,可以得到最优投资组合的分析结果如下:

\textbf{1. 允许卖空时,}



\textbf{2. 不允许卖空时,}