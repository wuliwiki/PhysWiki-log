% Minkowski 泛函
% keys Minkowski泛函
% license Usr
% type Tutor

\pentry{齐次凸泛函\nref{nod_ConFul},凸集和凸体\nref{nod_ConSet}}{nod_bf82}

\begin{definition}{Minkowski泛函}\label{def_MinFul_1}
设 $L$ 是任一线性空间,$A$ 是\enref{核}{ConSet} 包含 0 的 $L$ 中的凸体(\autoref{def_ConSet_1}),则称泛函
\begin{equation}\label{eq_MinFul_1}
p_A(x)=\inf\qty{r|\frac{x}{r}\in A,r>0}~
\end{equation}
为凸体 $A$ 的\textbf{Minkowski泛函}。

\end{definition}

\begin{lemma}{}
设 $L,A,p_A$ 如\autoref{def_MinFul_1} 定义。 则任意 $x\in L,0<\epsilon\in\mathbb R$,只要 $r>p_A(x)$,就有 $x/r\in A$
\end{lemma}
\textbf{证明:}事实上,在 $p_A(x)=0$ 时 $x/r=0\in A$;反之 $p_A(x)\neq0$, 由 $p_A(x)$ 的(下确界的)定义, 一定存在 $p_A(x)\leq a<r$,使得 $x/a\in A$,否者$\inf\qty{r|\frac{x}{r}\in A,r>0}\geq r$。于是存在 $s\in A$,使得 $x=as$,因此 $x/r=(a/r)s\in A$ (因为 $A$ 是凸集,而 $(a/r)s$ 在 $0,s\in A$ 的线段上)。

\textbf{证毕!}

\begin{theorem}{}\label{the_MinFul_1}
Minkowski泛函\autoref{eq_MinFul_1} 是齐次凸的与非负的。
\end{theorem}

\textbf{证明:}\textbf{非负性:}对每一 $x\in L$,如果 $r$ 充分大,则元素 $x/r$ 属于 $A$。因此\autoref{def_MinFul_1} 定义的 $p(x)$ 是非负有限的(由于是取下确界)。现在来证明正齐次性。

\textbf{正齐次性:}若 $t>0$,则
\begin{equation}
\begin{aligned}
p_A(tx)=&\inf \{r|tx/r\in A,r>0\}=\inf \{r|x/(r/t)\in A,r>0\}\\
=&\inf \{tr'|x/r'\in A,r'>0\}=t\inf \{r'|x/r'\in A,r'>0\}\\
=&tp_A(x)\\.
\end{aligned}~
\end{equation}
\textbf{凸性:}设 $x_1,x_2\in L,\epsilon>0$ 任意,取 $r_i(i=1,2)$ 使得 $p_A(x_i)<r_i<p_A(x_i)+\epsilon$,则由\autoref{the_MinFul_1} ,$x_i/r\in A$。令 $r=r_1+r_2$,则点
\begin{equation}
\frac{x_1+x_2}{r}=\frac{r_1}{rr_1}x_1+frac{r_2}{rr_2}x_2~
\end{equation}
属于具有端点 $x_1/r_1$ 与 $x_2/r_2$ 的线段。由于 $A$ 的凸性,该线段属于 $A$,这也意味着点 $(x_1+x_2)/r\in A$。因此(由\autoref{def_MinFul_1} )
\begin{equation}
p_A(x_1+x_2)\leq r=r_1+r_2<p_A(x_1)+p_A(x_2)+2\epsilon.~
\end{equation}
 因为 $\epsilon>0$ 的任意性,则

 \begin{equation}
p_A(x_1+x_2)\leq p_A(x_1)+p_A(x_2).~
 \end{equation}
 由\autoref{the_ConFul_1},$p_A$ 凸。








\textbf{证毕!}

\begin{theorem}{}
若 $p(x)$ 是线性空间 $L$ 上任意非负齐次凸泛函,$k$ 是正数,则
\begin{equation}\label{eq_MinFul_2}
A=\{x|p(x)\leq k\}~
\end{equation}
是凸体,其核是包含 $0$ 的集 $\{x|p(x)<k\}$。如果\autoref{eq_MinFul_2} 中 $k=1$,则原泛函 $p(x)$ 是 $A$ 的Minkowski泛函。
\end{theorem}

\textbf{证明:}
\textbf{凸体:}设 $x,y\in A,\alpha+\beta=1,\alpha,\beta\geq0$,则(根据 $p$ 泛函的凸性)
\begin{equation}
p(\alpha x+\beta y)\leq\alpha p(x)+\beta p(x)\leq k.~
\end{equation}
这即表明 $A$ 是凸的。

\textbf{核:}设 $p(x)<k,t>0$ 与 $y\in L$,则(根据正齐次凸性和\autoref{the_ConFul_1})
\begin{equation}
p(x\pm ty)\leq p(x)+tp(\pm y)~.
\end{equation}
若 $p(-y)=p(y)=0$,则上式为 $p(x\pm ty)\leq p(x)<k$,即对一切 $t$,成立 $x\pm ty\in A$;反之,$p(y),p(-y)$ 必有一异于零,则对
\begin{equation}
t<\frac{k-p(x)}{\max\{p(y),p(-y)\}}~,
\end{equation}
成立
\begin{equation}
p(x)+tp(\pm y)\leq p(x)+(k-p(x))=k~.
\end{equation}
即 $x\pm ty\in A$。

\textbf{$k=1$ 是Minkowski泛函:}任意 $x/r\in A$,成立 $p(x/r)\leq 1$,进而由齐次性 $p(x)\leq r$,即 $p(x)$ 是 $\{r|x/r\in A,r>0\}$ 的下界。另一方面,若 $x/r\notin A$,则 $p(x)>r$,即 $p(x)$ 比  $\{r|x/r\in A,r>0\}$ 的一切下界都大,因此 $p(x)$ 是 $\{r|x/r\in A,r>0\}$ 的下确界。

\textbf{证毕!}
