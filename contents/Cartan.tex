% 埃利·嘉当(综述)
% license CCBYSA3
% type Wiki

本文根据 CC-BY-SA 协议转载翻译自维基百科\href{https://en.wikipedia.org/wiki/\%C3\%89lie_Cartan}{相关文章}。

\begin{figure}[ht]
\centering
\includegraphics[width=6cm]{./figures/d9fe03c9e302e304.png}
\caption{教授Élie Joseph Cartan} \label{fig_Cartan_1}
\end{figure}
埃利·约瑟夫·卡尔坦(Élie Joseph Cartan,法语发音:[kaʁtɑ̃];1869年4月9日 – 1951年5月6日)是法国一位具有深远影响的数学家,他在李群理论、微分系统(偏微分方程的无坐标几何形式化)和微分几何领域做出了基础性工作。他还对广义相对论和间接地对量子力学做出了重要贡献。普遍认为,他是20世纪最伟大的数学家之一。

他的儿子亨利·卡尔坦(Henri Cartan)是位于代数拓扑学领域的具有深远影响的数学家。
\subsection{生平}  
埃利·卡尔坦(Élie Cartan)于1869年4月9日出生在法国伊泽尔省的多洛米厄村(Dolomieu),父亲是约瑟夫·卡尔坦(Joseph Cartan,1837–1917),母亲是安妮·科塔兹(Anne Cottaz,1841–1927)。约瑟夫·卡尔坦是村里的铁匠;埃利·卡尔坦回忆说,他的童年是在“铁砧的敲打声中度过的,每天早晨天刚亮就开始了”,而且“他的母亲在那少数几分钟的时间里,解放了自己从照顾孩子和家务的繁忙中,便开始使用纺车”。埃利有一个姐姐让娜-玛丽(Jeanne-Marie,1867–1931),她成为了一名裁缝;一个弟弟莱昂(Léon,1872–1956),他成为了父亲的铁匠铺的工匠;以及一个妹妹安娜·卡尔坦(Anna Cartan,1878–1923),在埃利的影响下,她进入了高等师范学校(École Normale Supérieure),并选择了成为一名数学教师的职业,担任中学(lycée)的教师。

埃利·卡尔坦在多洛米厄的一个小学上学,并且是班级中最好的学生。曾任他老师的杜普依(M. Dupuis)回忆道:“埃利·卡尔坦是一个害羞的学生,但他眼中闪烁着一种不寻常的智慧光芒,并且具备了极好的记忆力。”当时伊泽尔的代表安托南·杜博(Antonin Dubost)参观了学校,并对卡尔坦的非凡才能印象深刻。他建议卡尔坦参加一项中学奖学金的竞赛。卡尔坦在杜普依老师的指导下准备了这次竞赛,并在十岁时顺利通过。他在维耶讷(Vienne)学院度过了五年(1880–1885),然后在格勒诺布尔(Grenoble)中学度过了两年(1885–1887)。1887年,他搬到巴黎的尚松·德·塞伊(Lycée Janson de Sailly)中学,学习科学,并在两年内结识了同班同学让-巴蒂斯特·佩朗(Jean-Baptiste Perrin,1870–1942),后者后来成为法国著名的物理学家。

卡尔坦于1888年进入巴黎高等师范学校(École Normale Supérieure),在那里他听取了查尔斯·埃尔米特(Charles Hermite,1822–1901)、朱尔·塔内里(Jules Tannery,1848–1910)、加斯顿·达尔布(Gaston Darboux,1842–1917)、保罗·阿佩尔(Paul Appell,1855–1930)、埃米尔·皮卡尔(Émile Picard,1856–1941)、爱德华·古尔萨(Édouard Goursat,1858–1936)和亨利·庞加莱(Henri Poincaré,1854–1912)等人的讲座,他特别推崇庞加莱的讲座。

卡尔坦于1891年从高等师范学校毕业后,被征召入法国军队,服役一年并晋升为中士。在接下来的两年里(1892–1894),卡尔坦返回高等师范学校,并在同班同学阿瑟·特雷斯(Arthur Tresse,1868–1958)的建议下,开始研究由威廉·基林(Wilhelm Killing)提出的简单李群分类问题。特雷斯曾在1888–1889年间师从索弗斯·李(Sophus Lie)。1892年,李受达尔布和塔内里的邀请来到巴黎,并首次与卡尔坦见面。

卡尔坦于1894年在索邦大学(Sorbonne)的理学院捍卫了他的博士论文《有限连续变换群的结构》。在1894至1896年间,卡尔坦在蒙彼利埃大学担任讲师;1896年至1903年,他在里昂大学理学院担任讲师。

1903年,卡尔坦在里昂结婚,妻子是玛丽-路易丝·比安科尼(Marie-Louise Bianconi,1880–1950);同年,卡尔坦成为南锡大学理学院的教授。1904年,卡尔坦的长子亨利·卡尔坦(Henri Cartan)出生,后来成为一位有影响力的数学家;1906年,他的次子让·卡尔坦(Jean Cartan)出生,后来成为一位作曲家。1909年,卡尔坦将家人迁往巴黎,并在索邦大学理学院担任讲师。1912年,卡尔坦根据庞加莱的推荐成为索邦大学的教授,并一直在此工作,直到1940年退休。在晚年,卡尔坦在巴黎高等师范学校女子部教授数学。

作为卡尔坦的学生,几何学家陈省身写道:[4]

通常,在[与卡尔坦会面]的第二天,我会收到他的信。他会说:“你走后,我又想了想你的问题……”——他有了一些结果,还有一些新问题,等等。他记得所有关于简单李群、李代数的论文,都是熟记于心。当你在街上遇到他,某个问题冒出来时,他会从口袋里拿出一个旧信封,写点东西,然后给你答案。有时我花了几个小时甚至几天才能得到同样的答案……我必须非常努力地工作。

1921年,他成为波兰科学院的外籍院士,1937年成为荷兰皇家艺术与科学学院的外籍院士。[5] 1938年,他参与了组织国际科学统一大会的国际委员会。[6]

他于1951年在巴黎因长期病患去世。

1976年,一座月球陨石坑以他的名字命名,原本它被命名为阿波罗纽斯D。
\subsection{工作}  
在《Travaux》中,卡尔坦将他的工作分为15个领域。使用现代术语,它们是:
\begin{enumerate}
\item 李群理论  
\item 李群的表示  
\item 超复数,除法代数  
\item 偏微分方程系统,卡尔坦–凯勒定理  
\item 等价理论  
\item 可积系统,延长理论与对合系统  
\item 无限维群和伪群  
\item 微分几何与运动框架  
\item 一般化的具有结构群和联络的空间,卡尔坦联络,鸿沟,魏尔张量  
\item 李群的几何和拓扑  
\item 黎曼几何  
\item 对称空间  
\item 紧群及其齐次空间的拓扑  
\item 积分不变量与经典力学  
\item 相对论,旋量  
\end{enumerate}
卡尔坦的数学工作可以描述为对可微流形上的分析发展的贡献,许多人现在认为这已成为现代数学的核心和最重要的部分,且他在这方面处于引领和推动的最前沿。这个领域集中在李群、偏微分方程系统和微分几何;这些领域,主要通过卡尔坦的贡献,现已紧密交织,并构成了一个统一且强大的工具。
\subsubsection{李群}  
在他的论文之后的三十年里,卡尔坦几乎是李群领域的唯一研究者。李群最初主要被李群认为是分析流形的分析变换系统,这些变换依赖于有限数量的参数。1888年,威廉·基林(Wilhelm Killing)开创了一个非常有成果的研究途径,他开始系统地研究李群本身,而不依赖于其可能作用于其他流形。那时(直到1920年),研究仅限于局部性质,因此基林研究的主要对象是李代数,李代数在纯代数的术语中恰好反映了局部性质。基林的伟大成就是确定了所有简单复李代数;然而,他的证明往往是有缺陷的,卡尔坦的论文主要致力于为局部理论提供严格的基础,并证明了基林所展示的每种简单复李代数类型下存在异常李代数。后来,卡尔坦通过明确解决两个基本问题,完成了局部理论的构建,为此他发展了全新的方法:简单实李代数的分类和所有简单李代数的不可约线性表示的确定,他为此引入了表示的权重的概念。正是在确定正交群的线性表示的过程中,卡尔坦于1913年发现了旋量,旋量后来在量子力学中发挥了重要作用。

1925年后,卡尔坦对拓扑问题越来越感兴趣。在魏尔关于紧群的精彩结果的启发下,他发展了研究李群全局性质的新方法;特别是,他证明了从拓扑上讲,一个连通的李群是欧几里得空间与紧群的积,并且对于紧李群,他发现可以从李群的李代数的结构中推导出其底层流形的可能基本群。最后,他勾画了一种确定紧李群Betti数的方法,再次将这个问题归结为对其李代数的代数问题,而这个问题已经被完全解决。
\subsubsection{李伪群}  
在解决了李群的结构问题之后,卡尔坦(沿用李的说法)提出了一个类似的问题,针对“无限连续群”,即现在所称的李伪群,这是李群的无限维类比(李群还有其他无限维的推广)。卡尔坦所考虑的李伪群是一个在空间子集之间的变换集合,包含恒等变换,并具有这样的性质:该集合中两个变换的复合结果(当复合可能时)仍属于同一个集合。由于两个变换的复合并不总是可能,因此这个变换集合不是一个群(而是现代术语中的群族),因此被称为伪群。卡尔坦只考虑那些在流形之间的变换,对于这些变换,不存在将流形划分为被所考虑变换互换的类的划分。这类变换的伪群被称为原始伪群。卡尔坦证明了每一个无限维的原始复分析变换伪群都属于六个类中的某一类:1)所有n个复变量的解析变换的伪群;2)所有n个复变量的解析变换,且其雅可比行列式为常数的伪群(即,将所有体积乘以相同复数的变换);3)所有n个复变量的解析变换,且其雅可比行列式为1的伪群(即,保持体积不变的变换);4)所有2n > 4个复变量的解析变换,且保持某一双重积分不变的伪群(辛伪群);5)所有2n > 4个复变量的解析变换,且将上述双重积分乘以一个复函数的伪群;6)所有2n + 1个复变量的解析变换,且将某一形式乘以一个复函数的伪群(接触伪群)。对于由实变量的解析函数定义的原始实变换伪群,也有类似的伪群类。
\subsubsection{微分系统}  
卡尔坦在微分系统理论中的方法或许是他最深刻的成就。打破传统,他从一开始就力图以完全不变的方式来表述和解决问题,独立于任何特定的变量和未知函数的选择。因此,他首次能够准确地定义什么是任意微分系统的“一般”解。他的下一步是尝试通过一种“延拓”方法来确定所有“奇异”解,该方法通过向给定系统中附加新的未知数和新方程,使得原系统的任何奇异解都成为新系统的一个一般解。尽管卡尔坦证明,在他处理的每一个例子中,他的方法都导致了所有奇异解的完全确定,但他并未成功地证明这对任意系统始终成立;这种证明是在1955年由仓石正武(Masatake Kuranishi)获得的。

卡尔坦的主要工具是外微分形式的微积分,他帮助创造并发展了这一工具,并在论文后的十年中将其应用于微分几何、李群、解析动力学和广义相对论等多个领域的问题。他讨论了大量的例子,以一种极为深邃的风格进行处理,这种风格只有通过他独特的代数和几何洞察力才能实现。
\subsubsection{微分几何}  
卡尔坦对微分几何的贡献同样令人印象深刻,可以说他使整个学科焕发了新的生机,因为黎曼和达尔布的最初工作正逐渐被枯燥的计算和微不足道的结果所淹没,就像一个世代之前初等几何和不变理论所遭遇的情况一样。他的指导思想是大幅扩展达尔布和里博考尔(Ribaucour)提出的“移动框架”方法,并赋予了其极大的灵活性和力量,远超出经典微分几何中所做的任何工作。用现代的术语来说,这种方法是将一个纤维束 \( E \) 与一个具有相同基底的主纤维束关联起来,在每一个基点上,纤维是作用在该点上 \( E \) 的纤维的群。如果 \( E \) 是基底上的切丛(自李群起,实际上被视为“接触元素”的流形),则对应的群是一般线性群(或在经典欧几里得几何或黎曼几何中为正交群)。卡尔坦能够处理其他类型的纤维和群,这使得我们可以把他视为第一个提出纤维丛概念的人,尽管他并没有明确给出定义。这个概念已经成为现代数学各个领域中最重要的概念之一,特别是在全球微分几何学和代数与微分拓扑学中。卡尔坦用它来提出他对联络的定义,这一概念现已被普遍使用,并且取代了1917年后若干几何学家的尝试,他们试图寻找一种比黎曼几何模型更为一般化的“几何”,可能更适合于沿着广义相对论的线索描述宇宙。

卡尔坦展示了如何利用他对联络的概念来获得更为优雅和简洁的黎曼几何呈现。然而,他对黎曼几何的主要贡献是对对称黎曼空间的发现和研究,这是少数几种数学理论的创始人同时也是其完成者的情况之一。对称黎曼空间可以通过不同方式定义,其中最简单的一种是假设在空间的每个点周围存在一个“对称性”,它是自反的,固定该点并保持距离。卡尔坦发现的意外事实是,可以通过简单李群的分类来给出这些空间的完整描述;因此,在数学的各个领域中,例如自同态函数和解析数论(看似与微分几何相去甚远),这些空间正在发挥越来越重要的作用,这一点并不令人惊讶。
\subsection{广义相对论的替代理论}  
卡尔坦提出了一个与广义相对论竞争的引力理论,即爱因斯坦-卡尔坦理论(Einstein–Cartan theory)。
\subsection{出版物}  
卡尔坦的论文被收录在他的《全集》6卷本中(巴黎,1952–1955)。有两篇优秀的讣告是由S. S. Chern和C. Chevalley撰写,发表于《美国数学会公告》58期(1952);以及J. H. C. Whitehead撰写,发表于《皇家学会讣告》(1952)。
\begin{itemize}
\item 卡尔坦,Élie(1894),《Sur la structure des groupes de transformations finis et continus》,博士论文,Nony  
\item 卡尔坦,Élie(1899),《Sur certaines expressions différentielles et le problème de Pfaff》,《École Normale Supérieure科学年报》第3系列(法文),16期,巴黎:Gauthier-Villars:239–332,doi:10.24033/asens.467,ISSN 0012-9593,JFM 30.0313.04  
\item 《Leçons sur les invariants intégraux》,Hermann,巴黎,1922  
\item 卡尔坦,Élie(1925),《La géométrie des espaces de Riemann》。巴黎,Gauthier-Villars(《数学科学纪实》,第9期)(法文):IV + 60。JFM 51.0566.01。  
\item 卡尔坦,Élie(1946),《Leçons sur la géométrie des espaces de Riemann》(第二版修订增补版)。巴黎:Gauthier-Villars。p. VIII,378。ISBN 287647008X。Zbl 0060.38101。  
\item 卡尔坦,Élie(1931),《La théorie des groupes finis et continus et l'analyse situs》。数学科学纪实(42):68。JFM 56.0370.08。  
\item 卡尔坦,Élie(1950),《Leçons sur la géométrie projective complexe》(第二版)。巴黎:Gauthier-Villars。p. VII + 325。Bibcode:1950lgpc.book.....C。MR 0041456。Zbl 0003.06801。  
\item 《La parallelisme absolu et la théorie unitaire du champ》,Hermann,1932  
\item 《Les Espaces Métriques Fondés sur la Notion d'Arie》,Hermann,1933  
\item 《La méthode de repère mobile, la théorie des groupes continus, et les espaces généralisés》,1935  
\item 《Leçons sur la théorie des espaces à connexion projective》,Gauthiers-Villars,1937  
\item 《La théorie des groupes finis et continus et la géométrie différentielle traitées par la méthode du repère mobile》,Gauthiers-Villars,1937  
\item 卡尔坦,Élie(1981)[1938],《The theory of spinors》,纽约:Dover Publications,ISBN 978-0-486-64070-9,MR 0631850  
\item 《Les systèmes différentiels extérieurs et leurs applications géométriques》,Hermann,1945  
\item 《Oeuvres complètes》,3部分6卷本,巴黎,1952至1955,1984年由CNRS重印:  
\item 第1部分:Lie群(2卷),1952  
\item 第2部分,第1卷:代数、微分形式、微分系统,1953  
\item 第2部分,第2卷:有限群、微分系统、等价理论,1953  
\item 第3部分,第1卷:杂项、微分几何,1955  
\item 第3部分,第2卷:微分几何,1955  
\item 《Élie Cartan and Albert Einstein: Letters on Absolute Parallelism, 1929–1932》(卡尔坦和爱因斯坦:关于绝对平行主义的信件,1929–1932),原文为法德文,英语翻译由Jules Leroy和Jim Ritter完成,Robert Debever编辑,普林斯顿大学出版社,1979年
\end{itemize}
\subsection{另见}
\begin{itemize}
\item 外微分  
\item 微分系统的可积性条件  
\item 各向同性直线  
\item CAT(k)空间  
\item 爱因斯坦–卡尔坦理论  
\item 厄米对称空间  
\item 运动框架  
\item 伪群  
\item 纯自旋子
\end{itemize}
\subsection{参考文献}  
\begin{enumerate}
\item O'Connor, John J.; Robertson, Edmund F., "Élie Cartan", 《麦克图尔数学史档案》,圣安德鲁斯大学  
\item Élie Cartan 见于数学家谱项目  
\item O'Connor, J J; Robertson, E F (1999). 《20世纪伟大的数学家》(PDF)。原文档案存档于 2020年11月25日。检索于 2024年1月7日。  
\item Jackson, Allyn (1998). "采访施盈申·陈" (PDF)。  
\item "Élie J. Cartan (1869–1951)",《荷兰皇家艺术与科学学院》。检索于2015年7月19日。  
\item Neurath, Otto (1938). "统一科学作为百科全书式整合"。《国际统一科学百科全书》。1 (1): 1–27。  
\item Knebelman, M. S. (1937). "书评: 《基于Arie概念的度量空间》",《美国数学学会公报》。43 (3): 158–159. doi:10.1090/S0002-9904-1937-06493-7. ISSN 0002-9904.  
\item Levy, Harry (1935). "书评: 《移动框架方法,连续群体理论与广义空间》",《美国数学学会公报》。41 (11): 774. doi:10.1090/s0002-9904-1935-06183-x.  
\item Vanderslice, J. L. (1938). "书评: 《关于投影联结空间的课程》",《美国数学学会公报》。44 (1, 第一部分): 11–13. doi:10.1090/s0002-9904-1938-06648-7.  
\item Weyl, Hermann (1938). "卡尔坦关于群体和微分几何的观点",《美国数学学会公报》。44 (9, 第一部分): 598–601. doi:10.1090/S0002-9904-1938-06789-4.  
\item Givens, Wallace (1940). "书评: 《自旋子理论》由Élie Cartan" (PDF),《美国数学学会公报》。46 (11): 869–870. doi:10.1090/s0002-9904-1940-07329-x.  
\item Ruse, Harold Stanley (1939年7月). "书评: 《自旋子理论课程》由E. Cartan",《数学公报》。23 (255): 320–323. doi:10.2307/3606453. JSTOR 3606453.  
\item Biedenharn, Lawrence C. (1968). "书评: 《自旋子理论》由Élie Cartan(翻译自1937年法文版)",《今日物理》。21 (7): 95–96. doi:10.1063/1.3035084.  
\item Thomas, J. M. (1947). "书评: 《外微分系统及其几何应用》",《美国数学学会公报》。53 (3): 261–266. doi:10.1090/s0002-9904-1947-08750-4.  
\item Cartan, Élie (1899), "Sur certaines expressions différentielles et le problème de Pfaff", 《巴黎高等师范学校学术年鉴》, 16: 239–332, doi:10.24033/asens.467  
\item "书评: Élie Cartan, 阿尔伯特·爱因斯坦:关于绝对平行性的信件,1929-1932,由罗伯特·德贝弗编辑",《原子科学家公报》。36 (3): 51. 1980年3月
\end{enumerate}
\subsection{外部链接}  
\begin{itemize}
\item 与Élie Cartan相关的媒体资源见于维基共享资源  
\item M.A. Akivis 和 B.A. Rosenfeld(1993)《Élie Cartan (1869–1951)》,由V.V. Goldberg翻译自俄文原著,美国数学学会 ISBN 0-8218-4587-X 。  
\item 施盈申·陈(1994)《书评:Akivis和Rosenfeld的《Élie Cartan》》,《美国数学学会公报》,30(1)  
\item 陈施盈申,克劳德·谢瓦雷(1952)"Élie Cartan与他的数学工作",《美国数学学会公报》,58(2):217–250。doi:10.1090/s0002-9904-1952-09588-4。 
\end{itemize} 
以下是他的部分书籍和文章的英文翻译:  
\begin{itemize}
\item "On certain differential expressions and the Pfaff problem"  
\item "On the integration of systems of total differential equations"  
\item 《关于积分不变量的课程》  
\item "The structure of infinite groups"  
\item "Spaces with conformal connections"  
\item "On manifolds with projective connections"  
\item "The unitary theory of Einstein–Mayer"  
\item "E. Cartan, Exterior Differential Systems and its Applications"(由M. Nadjafikhah翻译成英文)
\end{itemize}