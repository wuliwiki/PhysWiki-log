% 单摆(大摆角)
% keys 单摆|周期|椭圆积分|微分方程|机械能
% license Usr
% type Tutor

\pentry{单摆\nref{nod_Pend}, 椭圆积分\nref{nod_EliInt}}{nod_ec0c}

令最大摆角为 $\theta_0$, 能量守恒, 机械能为
\begin{equation}
E = \frac{1}{2} m l^2 \dot \theta^2 - mg l \cos\theta = - mg l \cos\theta_0~,
\end{equation}
所以 $\theta$ 处的角速度为
\begin{equation}
\dot{\theta} = \sqrt{\frac{2g}{l} (\cos\theta - \cos\theta_0) }~.
\end{equation}
令 $t = 0$ 时 $\theta = 0$ 且 $\dot{\theta} > 0$, 该微分方程的解% \addTODO{链接}
可以用椭圆积分\autoref{eq_EliInt_2}~\upref{EliInt} 表示
\begin{equation}\label{eq_SinPen_1}
t(\alpha) = \sqrt{\frac{l}{2g}} \int_0^{\alpha} \frac{\dd{\theta}}{\sqrt{\cos\theta - \cos\theta_0}}
= \sqrt{\frac{l}{g}} \csc\frac{\theta_0}{2} F\qty(\frac{\alpha}{2}, \csc\frac{\theta_0}{2})~,
\qquad (0 \leqslant \alpha \leqslant \theta_0)~.
\end{equation}
周期可以表示为从最低点第一次摆到最高点所需时间的 4 倍
\begin{equation}
T = 4t(\theta_0) = 4 \sqrt{\frac{l}{g}} \csc\frac{\theta_0}{2} F\qty(\frac{\theta_0}{2}, \csc\frac{\theta_0}{2})~.
\end{equation}
Wikipedia 给出的公式为
\begin{equation}\label{eq_SinPen_3}
T = 4 \sqrt{\frac{l}{g}} F\qty(\frac{\pi}{2}, \sin\frac{\theta_0}{2})~.
\end{equation}
此式可由\autoref{eq_SinPen_1} 作如下变换得到:
\begin{equation}\label{eq_SinPen_2}
t(\alpha)=\frac12\sqrt{\frac{l}{g}}\int_{0}^{\alpha}\frac{\mathrm{d}\theta}{\sqrt{\sin^2\frac{\theta_0}2-\sin^2\frac{\theta}2}}~.
\end{equation}
令
\begin{equation}
\sin \phi(\theta)=\frac{\sin^2\frac{\theta}2}{\sin^2\frac{\theta_0}2}~,
\end{equation}
则
\begin{equation}
\mathrm d\theta=2\frac{\sin\frac{\theta_0}2}{\cos\frac{\theta}2}\cos\phi(\theta)\mathrm d\phi
=2\frac{\sin\frac{\theta_0}2}{\sqrt{1-\sin^2\frac{\theta}2}}\sqrt{1-\sin^2\phi(\theta)}\mathrm d\phi
=2\frac{\sqrt{\sin^2\frac{\theta_0}2-\sin^2\frac{\theta}2}}{\sqrt{1-\sin^2\frac{\theta_0}2\sin^2{\phi(\theta)}}}\mathrm d\phi~.
\end{equation}
代入\autoref{eq_SinPen_2} 有
\begin{equation}
t(\alpha)=\sqrt{\frac{l}{g}}\int_{0}^{\phi(\alpha)}\frac{\mathrm{d}\phi}{\sqrt{1-\sin^2\frac{\theta_0}2\sin^2{\phi(\theta)}}}~.
\end{equation}
又$\phi(\theta_0)=\frac\pi2$,则
\begin{equation}
T=4t(\theta_0)=\sqrt{\frac{l}{g}}\int_{0}^{\frac\pi2}\frac{\mathrm{d}\phi}{\sqrt{1-\sin^2\frac{\theta_0}2\sin^2{\phi(\theta)}}}=4 \sqrt{\frac{l}{g}} F\qty(\frac{\pi}{2}, \sin\frac{\theta_0}{2})~,
\end{equation}
与\autoref{eq_SinPen_3} 相同。


(图未完成)(未完成:周期的级数展开, $\theta(t)$ 级数解)


\addTODO{泰勒展开一下}
\begin{figure}[ht]
\centering
\includegraphics[width=10cm]{./figures/05d241985e4e4ed5.png}
\caption{123} \label{fig_SinPen_1}
\end{figure}
\begin{figure}[ht]
\centering
\includegraphics[width=10cm]{./figures/d4c3d804ff1ce936.png}
\caption{请添加图片描述} \label{fig_SinPen_2}
\end{figure}
