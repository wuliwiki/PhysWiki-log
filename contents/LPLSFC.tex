% 拉普拉斯方程(综述)
% license CCBYSA3
% type Wiki

本文根据 CC-BY-SA 协议转载翻译自维基百科\href{https://en.wikipedia.org/wiki/Laplace\%27s_equation}{相关文章}。

在数学和物理学中,拉普拉斯方程是一个二阶偏微分方程,得名于皮埃尔-西蒙·拉普拉斯,他在1786年首次研究了其性质。通常写作:
\[
\nabla^2 f = 0~
\]
或
\[
\Delta f = 0~
\]
其中\(\Delta = \nabla \cdot \nabla = \nabla^2\)是拉普拉斯算符,\(^\text{[注1]}\)\(\nabla\cdot\)是散度算符(也表示为“div”),\(\nabla\)是梯度算符(也表示为“grad”),\(f(x, y, z)\)是一个二次可微的实值函数。因此,拉普拉斯算符将标量函数映射到另一个标量函数。

如果右边指定为一个给定函数,\(h(x, y, z)\)则我们有
\[
\Delta f = h~
\]
这被称为泊松方程,是拉普拉斯方程的一种推广。拉普拉斯方程和泊松方程是椭圆型偏微分方程的最简单例子。拉普拉斯方程也是赫尔姆霍兹方程的特例。

拉普拉斯方程解的一般理论称为势理论。拉普拉斯方程的二次连续可微解是调和函数,\(^\text{[1]}\)这些函数在多个物理学分支中非常重要,特别是在静电学、引力学和流体动力学中。在热传导的研究中,拉普拉斯方程是稳态热方程。\(^\text{[2]}\)一般来说,拉普拉斯方程描述了平衡状态或那些不显式依赖于时间的情况。
\subsection{在不同坐标系中的形式} 
在直角坐标系中,\(^\text{[3]}\)
\[
\nabla^2 f = \frac{\partial^2 f}{\partial x^2} + \frac{\partial^2 f}{\partial y^2} + \frac{\partial^2 f}{\partial z^2} = 0.~
\]
在柱面坐标系中,\(^\text{[3]}\)
\[
\nabla^2 f = \frac{1}{r} \frac{\partial}{\partial r} \left( r \frac{\partial f}{\partial r} \right) + \frac{1}{r^2} \frac{\partial^2 f}{\partial \phi^2} + \frac{\partial^2 f}{\partial z^2} = 0.~
\]
在球坐标系中,使用\( (r, \theta, \varphi) \)约定,\(^\text{[3]}\)
\[
\nabla^2 f = \frac{1}{r^2} \frac{\partial}{\partial r} \left( r^2 \frac{\partial f}{\partial r} \right) + \frac{1}{r^2 \sin \theta} \frac{\partial}{\partial \theta} \left( \sin \theta \frac{\partial f}{\partial \theta} \right) + \frac{1}{r^2 \sin^2 \theta} \frac{\partial^2 f}{\partial \varphi^2} = 0.~
\]
更一般地,在任意曲线坐标系\( (\xi^i) \)中,
\[
\nabla^2 f = \frac{\partial}{\partial \xi^j} \left( \frac{\partial f}{\partial \xi^k} g^{kj}\right) + \frac{\partial f}{\partial \xi^j} g^{jm}\Gamma_{mn}^n = 0,~
\]
或
\[
\nabla^2 f = \frac{1}{\sqrt{|g|}} \frac{\partial}{\partial \xi^i} \left( \sqrt{|g|} g^{ij} \frac{\partial f}{\partial \xi^j} \right) = 0, \quad (g = \det \{g_{ij}\}),~
\]
其中\( g_{ij} \)是相对于新坐标的欧几里得度量张量,\( \Gamma \) 表示其克里斯托费尔符号。
\subsection{边界条件}
\begin{figure}[ht]
\centering
\includegraphics[width=14.25cm]{./figures/e07e8bea5b6b38d4.png}
\caption{在一个环形区域(内半径\( r = 2 \),外半径 \( R = 4 \)上的拉普拉斯方程,具有狄里克雷边界条件\( u(r=2) = 0 \) 和 \( u(R=4) = 4 \sin(5 \theta)\)} \label{fig_LPLSFC_1}
\end{figure}
拉普拉斯方程的狄里克雷问题包括在某个区域\( D \)上找到一个解\( \varphi \),使得\( \varphi \)在\( D \)的边界上等于给定的某个函数。由于拉普拉斯算符出现在热方程中,因此这个问题的一个物理解释如下:根据给定的边界条件规格,固定区域边界上的温度。允许热量流动,直到达到一个稳态,在此状态下,区域内每个点的温度不再变化。然后,区域内的温度分布将由相应的狄里克雷问题的解给出。

拉普拉斯方程的诺伊曼边界条件指定的不是边界上\( D \)的函数\( \varphi \)本身,而是它的法向导数。从物理角度看,这对应于为一个向量场构造势,该向量场的效应仅在\( D \)的边界已知。以热方程为例,这相当于规定通过边界的热流。特别地,在绝热边界上,\( \varphi \)的法向导数为零。

拉普拉斯方程的解称为调和函数;它们在满足方程的区域内都是解析的。如果两个函数都是拉普拉斯方程(或任何线性齐次微分方程)的解,那么它们的和(或任何线性组合)也是解。这一性质被称为叠加原理,非常有用。例如,可以通过求和简单解来构造复杂问题的解。
\subsection{在二维中}  
拉普拉斯方程在直角坐标系中的形式为:
\[
\frac{\partial^2 \psi}{\partial x^2} + \frac{\partial^2 \psi}{\partial y^2} \equiv \psi_{xx} + \psi_{yy} = 0.~
\]
\subsubsection{解析函数}  
复数解析函数的实部和虚部都满足拉普拉斯方程。也就是说,如果\( z = x + iy \),并且
\[
f(z) = u(x, y) + iv(x, y),~
\]
那么\( f(z) \)为解析函数的必要条件是\( u \)和\( v \)可微,并且满足柯西-黎曼方程:
\[
u_x = v_y, \quad v_x = -u_y.~
\]
其中\( u_x \)是\( u \)对\( x \)的一阶偏导数。由此可得:
\[
u_{yy} = (-v_x)_y = -(v_y)_x = -(u_x)_x.~
\]
因此,\( u \)满足拉普拉斯方程。类似的计算表明,\( v \)也满足拉普拉斯方程。反过来,给定一个调和函数,它是一个解析函数\( f(z) \)的实部(至少在局部范围内)。如果试探函数的形式为
\[
f(z) = \varphi(x, y) + i \psi(x, y),~
\]
那么如果我们设定
\[
\psi_x = -\varphi_y, \quad \psi_y = \varphi_x.~
\]
这个关系并不能确定\( \psi \),而只是确定它的增量:
\[
d\psi = -\varphi_y\, dx + \varphi_x\, dy.~
\]
对于 \( \varphi \) 的拉普拉斯方程意味着 \( \psi \) 的可积性条件得以满足:
\[
\psi_{xy} = \psi_{yx},~
\]
因此,\( \psi \)可以通过线积分定义。可积性条件和斯托克斯定理意味着连接两点的线积分值与路径无关。由此得到的一对拉普拉斯方程解称为共轭调和函数。这个构造仅在局部有效,或者在路径不绕过奇点的情况下有效。例如,如果\( r \)和\( \theta \)是极坐标,且
\[
\varphi = \log r,~
\]
则对应的解析函数为
\[
f(z) = \log z = \log r + i \theta.~
\]
然而,角度\( \theta \)仅在不包含原点的区域内是单值的。

拉普拉斯方程和解析函数之间的紧密联系意味着,拉普拉斯方程的任何解都有各阶导数,并且可以展开为幂级数,至少在不包含奇点的圆内。这与波动方程的解形成鲜明对比,后者通常具有较少的光滑性。

幂级数和傅里叶级数之间有着密切的联系。如果我们在半径为\( R \)的圆内展开一个函数\( f \)为幂级数,这意味着:
\[
f(z) = \sum_{n=0}^{\infty} c_n z^n,~
\]
其中\( c_n \)是适当定义的系数,其实部和虚部分别为:
\[
c_n = a_n + i b_n.~
\]
因此,
\[
f(z) = \sum_{n=0}^{\infty} \left[ a_n r^n \cos n\theta - b_n r^n \sin n\theta \right] + i \sum_{n=1}^{\infty} \left[ a_n r^n \sin n\theta + b_n r^n \cos n\theta \right],~
\]
这是\( f \)的傅里叶级数。这些三角函数本身也可以展开,使用倍角公式。
\subsubsection{流体流动}   
设\( u \)和\( v \)分别为二维稳态不可压缩无旋流动的速度场的水平分量和垂直分量。不可压流动的连续性条件是:
\[
u_x + v_y = 0,~
\]
流动为无旋流的条件是:
\[
\nabla \times \mathbf{V} = v_x - u_y = 0.~
\]
如果我们定义函数\( \psi \)的微分为:
\[
d\psi = u\, dy - v\, dx,~
\]
那么连续性条件就是这个微分的可积性条件:得到的函数称为流函数,因为它在流线沿线是常数。\( \psi \)的一阶导数为:
\[
\psi_x = -v, \quad \psi_y = u,~
\]
而无旋条件意味着\( \psi \)满足拉普拉斯方程。与\( \psi \)共轭的调和函数\( \varphi \)称为速度势。柯西-黎曼方程意味着:
\[
\varphi_x = \psi_y = u, \quad \varphi_y = -\psi_x = v.~
\]
因此,每个解析函数都对应于平面上的稳态不可压缩、无旋、无粘的流体流动。实部是速度势,虚部是流函数。
\subsubsection{静电学}  
根据麦克斯韦方程组,二维空间中与时间无关的电场\( (u, v) \)满足:
\[
\nabla \times (u, v, 0) = (v_x - u_y) \hat{\mathbf{k}} = \mathbf{0},~
\]
和
\[
\nabla \cdot (u, v) = \rho,~
\]
其中\( \rho \)是电荷密度。麦克斯韦方程的第一条是微分方程
\[
d\varphi = -u\, dx - v\, dy~
\]
的可积性条件,因此电势\( \varphi \)可以构造为满足:
\[
\varphi_x = -u, \quad \varphi_y = -v.~
\]
麦克斯韦方程的第二条则意味着:
\[
\varphi_{xx} + \varphi_{yy} = -\rho,~
\]
这是泊松方程。拉普拉斯方程可以像在二维问题中一样,在三维静电学和流体流动问题中使用。
\subsection{在三维中}  
\subsubsection{基本解}  
拉普拉斯方程的基本解满足:
\[
\Delta u = u_{xx} + u_{yy} + u_{zz} = -\delta (x - x', y - y', z - z'),~
\]
其中狄拉克δ函数\( \delta \) 表示在点 \( (x', y', z') \) 集中的单位源。没有任何函数具有这个属性:事实上,它是一个分布,而不是一个函数;但可以把它看作是一个函数的极限,这些函数在空间上的积分为1,并且它们的支撑(函数非零的区域)收缩到一个点(参见弱解)。在定义基本解时,通常采用与常规不同的符号约定。这种符号选择通常方便使用,因为\( -\Delta \)是一个正算符。因此,基本解的定义意味着,如果\( u \)的拉普拉斯算符在包含源点的任意体积上积分,则有:
\[
\iiint_V \nabla \cdot \nabla u\, dV = -1.~
\]
拉普拉斯方程在坐标旋转下不变,因此我们可以预期,基本解可能是仅依赖于源点距离\( r \)的解。如果我们选择体积为以源点为中心,半径为\( a \)的球体,那么高斯的散度定理意味着:
\[
-1 = \iiint_V \nabla \cdot \nabla u\, dV = \iint_S \frac{du}{dr}\, dS = \left. 4\pi a^2 \frac{du}{dr} \right|_{r=a}.~
\]
由此得出:
\[
\frac{du}{dr} = -\frac{1}{4\pi r^2},~
\]
在一个以源点为中心,半径为\( r \)的球面上,因此:
\[
u = \frac{1}{4\pi r}.~
\]
注意,在相反的符号约定下(物理中使用的符号),这是由点粒子产生的势,对于一个反平方力定律力,在泊松方程的解中出现。类似的推理表明,在二维中:
\[
u = -\frac{\log(r)}{2\pi}.~
\]
其中\( \log(r) \)表示自然对数。注意,在相反的符号约定下,这是由点状汇聚体(参见点粒子)产生的势,这是二维不可压流动中欧拉方程的解。
\subsubsection{格林函数}  
格林函数是一个基本解,同时满足适当的边界条件。比如,\( G(x,y,z;x',y',z') \)可能满足  
\[
\nabla \cdot \nabla G = -\delta (x - x', y - y', z - z') \quad \text{在} \, V \, \text{中},~
\]
以及  
\[
G = 0 \quad \text{如果} \quad (x, y, z) \quad \text{在} \, S \, \text{上}.~
\]
现在,如果\( u \)是泊松方程在\( V \)中的任意解:
\[
\nabla \cdot \nabla u = -f,~
\]
并且\( u \)在\( S \)上取边界值\( g \),那么我们可以应用格林恒等式(散度定理的结果),它表明:
\[
\iiint_V \left[ G \nabla \cdot \nabla u - u \nabla \cdot \nabla G \right] \, dV = \iiint_V \nabla \cdot \left[ G \nabla u - u \nabla G \right] \, dV = \iint_S \left[ G u_n - u G_n \right] \, dS.~
\]
符号\( u_n \)和 \( G_n \)表示在边界\( S \)上的法向导数。考虑到\( u \)和\( G \)满足的条件,这个结果简化为:
\[
u(x',y',z') = \iiint_V Gf \, dV - \iint_S G_n g \, dS.~
\]
因此,格林函数描述了数据\( f \) 和 \( g \) 在 \( (x', y', z') \)处的影响。对于一个半径为\( a \)的球体内部的情况,格林函数可以通过反射法获得(Sommerfeld 1949):距离球心 \( \rho \) 的源点\( P \)沿其径向线反射到点\( P' \),该点与球心的距离为:
\[
\rho' = \frac{a^2}{\rho}.~
\]
注意,如果\( P \)在球体内部,则\( P' \)会在球体外部。此时,格林函数为:
\[
\frac{1}{4\pi R} - \frac{a}{4\pi \rho R'}.~
\]
其中\( R \)表示源点\( P \)到观察点的距离,\( R' \)表示反射点\( P' \) 到观察点的距离。这个格林函数表达式的一个结果是泊松积分公式。设\( \rho \)、\( \theta \)、和 \( \phi \)为源点\( P \)的球坐标。这里\( \theta \)表示与竖直轴的夹角,这与通常的美式数学符号相反,但与标准的欧洲和物理学惯例一致。那么,具有狄利克雷边界值\( g \)的拉普拉斯方程在球体内的解为(Zachmanoglou & Thoe 1986,第228页):
\[
u(P) = \frac{1}{4\pi} a^3 \left( 1 - \frac{\rho^2}{a^2} \right) \int_0^{2\pi} \int_0^\pi \frac{g(\theta', \phi') \sin \theta'}{\left( a^2 + \rho^2 - 2a \rho \cos \Theta \right)^{3/2}} d\theta' \, d\phi',~
\]
其中  
\[
\cos \Theta = \cos \theta \cos \theta' + \sin \theta \sin \theta' \cos(\phi - \phi')~
\]  
是\( (\theta, \phi) \)和 \( (\theta', \phi') \)之间夹角的余弦。这个公式的一个简单结果是,如果\( u \)是一个调和函数,那么\( u \)在球心的值是它在球面上的值的平均值。这个平均值性质立即意味着,一个非常数的调和函数不能在球内点达到其最大值。
\subsubsection{拉普拉斯的球面调和函数}
拉普拉斯方程在球坐标系中的形式为:\(^\text{[4]}\)
\[
\nabla^2 f = \frac{1}{r^2} \frac{\partial}{\partial r} \left( r^2 \frac{\partial f}{\partial r} \right) + \frac{1}{r^2 \sin \theta} \frac{\partial}{\partial \theta} \left( \sin \theta \frac{\partial f}{\partial \theta} \right) + \frac{1}{r^2 \sin^2 \theta} \frac{\partial^2 f}{\partial \varphi^2} = 0.~
\]
考虑寻找形如 \( f(r, \theta, \phi) = R(r) Y(\theta, \phi) \) 的解。通过变量分离,施加拉普拉斯方程后得到两个微分方程:
\[
\frac{1}{R} \frac{d}{dr} \left( r^2 \frac{dR}{dr} \right) = \lambda,~
\]
和
\[
\frac{1}{Y} \frac{1}{\sin \theta} \frac{\partial}{\partial \theta} \left( \sin \theta \frac{\partial Y}{\partial \theta} \right) + \frac{1}{Y} \frac{1}{\sin^2 \theta} \frac{\partial^2 Y}{\partial \varphi^2} = -\lambda.~
\]
第二个方程可以在假设\( Y \)具有形式\( Y(\theta, \phi) = \Theta(\theta) \Phi(\phi) \)的情况下简化。再次对第二个方程应用变量分离,得到一对微分方程:
\[
\frac{1}{\Phi} \frac{d^2 \Phi}{d \varphi^2} = -m^2~
\]
和
\[
\lambda \sin^2 \theta + \frac{\sin \theta}{\Theta} \frac{d}{d\theta} \left( \sin \theta \frac{d \Theta}{d \theta} \right) = m^2,~
\]
其中\( m \)是某个常数。先验地,\( m \)是一个复常数,但由于\( \Phi \)必须是一个周期函数,且其周期能整除\( 2\pi \),因此\( m \)必定是整数,且\( \Phi \)是复指数\( e^{\pm im\varphi} \)的线性组合。解函数\( Y(\theta, \phi) \)在球体的极点处(即 \( \theta = 0 \) 和 \( \theta = \pi \))是规则的。在边界点强加这一规则性后,第二个方程的解\( \Theta \)变成了一个 Sturm–Liouville 问题,这迫使参数\( \lambda \)具有形式\( \lambda = \ell (\ell + 1) \),其中 \( \ell \)是一个非负整数,且\( \ell \geq |m| \);这也可以通过轨道角动量的解释来说明。此外,通过变量变换\( t = \cos \theta \),该方程转化为勒让德方程,其解是与勒让德多项式\( P_{\ell}^m(\cos \theta) \)的倍数相关的。最后,\( R \)的方程有形式为\( R(r) = A r^\ell + B r^{-\ell - 1} \)的解;要求解在整个\( R^3 \)中是规则的,这强制\( B = 0 \)\(^\text{[注2]}\)。

在这里,假设解具有特殊形式\( Y(\theta, \varphi) = \Theta(\theta) \Phi(\varphi) \)。对于给定的\( \ell \) 值,存在 \( 2\ell + 1 \) 个独立的此类解,每个解对应一个整数\( m \),满足 \( -\ell \leq m \leq \ell \)。这些角向解是三角函数的积,这里表示为复指数和关联勒让德多项式:
\[
Y_{\ell}^{m}(\theta, \varphi) = N e^{im\varphi} P_{\ell}^{m}(\cos \theta),~
\]
它们满足
\[
r^2 \nabla^2 Y_{\ell}^{m}(\theta, \varphi) = -\ell (\ell + 1) Y_{\ell}^{m}(\theta, \varphi).~
\]
这里,\( Y_{\ell}^m \)被称为度\( \ell \)和阶\( m \)的球面调和函数,\( P_{\ell}^m \)是关联的勒让德多项式,\( N \)是归一化常数,\( \theta \) 和 \( \varphi \)分别表示纬度和经度。特别地,纬度 \( \theta \)或极角范围从北极的 0 到赤道的 \( \pi/2 \),再到南极的\( \pi \),而经度 \( \varphi \)或方位角可以取所有值,满足\( 0 \leq \varphi < 2\pi \)。对于固定的整数\( \ell \),特征值问题的每个解\( Y(\theta, \varphi) \)
\[
r^2 \nabla^2 Y = -\ell (\ell + 1) Y~
\]
是\( Y_{\ell}^m \)的线性组合。事实上,对于任何这样的解,\( r^\ell Y(\theta, \varphi) \)是球坐标中的一个齐次多项式的表达式,该多项式是调和的(见下文)。因此,通过计数维度,可以看出存在\( 2\ell + 1 \)个线性无关的此类多项式。

拉普拉斯方程在以原点为中心的球体中的一般解是球面调和函数的线性组合,每个函数乘以适当的尺度因子\( r^\ell \):
\[
f(r, \theta, \varphi) = \sum_{\ell = 0}^{\infty} \sum_{m = -\ell}^{\ell} f_{\ell}^m r^\ell Y_{\ell}^m(\theta, \varphi),~
\]
其中\( f_{\ell}^m \)是常数,\( r^\ell Y_{\ell}^m \)被称为固体调和函数。这样的展开式在球体内有效:
\[
r < R = \frac{1}{\limsup_{\ell \to \infty} |f_{\ell}^m|^{1/\ell}}.~
\]
对于\( r > R \),则选择具有负幂次的固体调和函数。在这种情况下,需要使用已知区域的解展开成洛朗级数(关于 \( r = \infty \)),而不是泰勒级数(关于 \( r = 0 \)),以匹配各项并找到\( f_{\ell}^m \)。
\subsubsection{静电学和磁静学}  
设 \( \mathbf{E} \) 为电场,\( \rho \) 为电荷密度,\( \varepsilon_0 \) 为真空的介电常数。那么,电场的高斯定律(麦克斯韦方程组的第一方程)在微分形式下为\(^\text{[5]}\):
\[
\nabla \cdot \mathbf{E} = \frac{\rho}{\varepsilon_0}.~
\]
现在,电场可以表示为电势 \( V \) 的负梯度:
\[
\mathbf{E} = -\nabla V,~
\]
如果电场是无旋的,即\(\nabla \times \mathbf{E} = \mathbf{0}\)则电场的无旋性也称为静电条件。\(^\text{[5]}\)
\[
\nabla \cdot \mathbf{E} = \nabla \cdot (-\nabla V) = -\nabla^2 V~
\]
\[
\nabla^2 V = -\nabla \cdot \mathbf{E}~
\]
将这个关系代入高斯定律中,我们得到电学的泊松方程\(^\text{[5]}\):
\[
\nabla^2 V = -\frac{\rho}{\varepsilon_0}.~
\]
在无源区域的特殊情况下,\( \rho = 0 \),泊松方程简化为电势的拉普拉斯方程。\(^\text{[5]}\)

如果静电势\( V \)在区域\( \mathcal{R} \)的边界上给定,则它是唯一确定的。如果\( \mathcal{R} \)被一层导电材料包围,且该材料的电荷密度\( \rho \) 已知,并且总电荷\( Q \)也已知,那么\( V \)也是唯一的。\(^\text{[6]}\)

对于磁场,当没有自由电流时,
\[
\nabla \times \mathbf{H} = \mathbf{0}.~
\]
因此,我们可以定义一个磁标量势 \( \psi \),其表达式为
\[
\mathbf{H} = -\nabla \psi.~
\]
根据 \( \mathbf{H} \) 的定义:
\[
\nabla \cdot \mathbf{B} = \mu_0 \nabla \cdot (\mathbf{H} + \mathbf{M}) = 0,~
\]
由此可以得出
\[
\nabla^2 \psi = -\nabla \cdot \mathbf{H} = \nabla \cdot \mathbf{M}.~
\]
类似于静电学,在无源区域,\( \mathbf{M} = 0 \),泊松方程简化为磁标量势的拉普拉斯方程,
\[
\nabla^2 \psi = 0.~
\]
一个不满足拉普拉斯方程以及边界条件的势是无效的静电或磁标量势。
\subsection{引力}  
设\( \mathbf{g} \)为引力场,\( \rho \)为质量密度,\( G \)为引力常数。那么,引力的高斯定律在微分形式下为\(^\text{[7]}\):
\[
\nabla \cdot \mathbf{g} = -4\pi G \rho.~
\]
引力场是保守场,因此可以表示为引力势 \( V \) 的负梯度:
\[
\mathbf{g} = -\nabla V,~
\]
\[
\nabla \cdot \mathbf{g} = \nabla \cdot (-\nabla V) = -\nabla^2 V,~
\]
\[
\implies \nabla^2 V = -\nabla \cdot \mathbf{g}.~
\]
利用引力的高斯定律的微分形式,我们得到
\[
\nabla^2 V = 4\pi G \rho,~
\]
这就是引力场的泊松方程。\(^\text{[7]}\)

在真空中,\( \rho = 0 \),我们得到
\[
\nabla^2 V = 0,~
\]
这就是引力场的拉普拉斯方程。
\subsection{在施瓦茨希尔德度量中}  
S. Persides\(^\text{[8]}\)在施瓦茨希尔德时空中,沿着常数\( t \)的超曲面解了拉普拉斯方程。使用标准变量\( r \)、\( \theta \)、\( \varphi \),解为:
\[
\Psi (r, \theta, \varphi) = R(r) Y_l(\theta, \varphi),~
\]
其中\( Y_l(\theta, \varphi) \)是球面调和函数,且
\[
R(r) = (-1)^l \frac{(l!)^2 r_s^l}{(2l)!} P_l \left( 1 - \frac{2r}{r_s} \right) + (-1)^{l+1} \frac{2(2l+1)!}{(l)!^2 r_s^{l+1}} Q_l \left( 1 - \frac{2r}{r_s} \right).~
\]
这里,\( P_l \)和 \( Q_l \)分别是第一类和第二类勒让德函数,而\( r_s \) 是施瓦茨希尔德半径。参数\( l \)是一个任意的非负整数。
\subsection{另见}  
\begin{itemize}
\item 6-球坐标,一个坐标系统,在该系统下拉普拉斯方程变得可以进行 \( R \)-分离  
\item 亥姆霍兹方程,拉普拉斯方程的一个广义  
\item 球面调和函数  
\item 求积域  
\item 势能理论  
\item 势流  
\item 贝特曼变换  
\item 恩肖定理使用拉普拉斯方程来表明稳定的静态铁磁悬浮是不可能的  
\item 向量拉普拉斯算子  
\item 基本解 
\end{itemize} 
\subsection{注释}  
\begin{enumerate}
\item Δ符号也常用来表示某个量的有限变化,例如,\(\Delta x = x_1 - x_2\). 它用于表示拉普拉斯算子的用法与这种用法不应混淆。  
\item 在物理应用中,通常选择在无穷远处消失的解,使得 \( A = 0 \)。这不会影响球面调和函数的角部分。
\end{enumerate}
\subsection{参考文献}  
\begin{enumerate}
\item Stewart, James. *Calculus: Early Transcendentals*. 第7版, Brooks/Cole, Cengage Learning, 2012. 第14章:偏导数. 第908页. ISBN 978-0-538-49790-9.  
\item Zill, Dennis G., 和 Michael R. Cullen. *Differential Equations with Boundary-Value Problems*. 第8版, Brooks/Cole, Cengage Learning, 2013. 第12章:矩形坐标系中的边值问题. 第462页. ISBN 978-1-111-82706-9.  
\item Griffiths, David J. *Introduction to Electrodynamics*. 第4版, Pearson, 2013. 内封面. ISBN 978-1-108-42041-9.  
\item 这里关于球面调和函数的方法可以参考 (Courant & Hilbert 1962, §V.8, §VII.5).  
\item Griffiths, David J. *Introduction to Electrodynamics*. 第4版, Pearson, 2013. 第2章:静电学. 第83-84页. ISBN 978-1-108-42041-9.  
\item Griffiths, David J. *Introduction to Electrodynamics*. 第4版, Pearson, 2013. 第3章:势能. 第119-121页. ISBN 978-1-108-42041-9.  
\item Chicone, C.; Mashhoon, B. (2011-11-20). "Nonlocal Gravity: Modified Poisson's Equation". *Journal of Mathematical Physics*. 53 (4): 042501. arXiv:1111.4702. doi:10.1063/1.3702449. S2CID 118707082.  
\item Persides, S. (1973). "The Laplace and Poisson Equations in Schwarzschild's Space-time". *Journal of Mathematical Analysis and Applications*. 43 (3): 571–578. Bibcode:1973JMAA...43..571P. doi:10.1016/0022-247X(73)90277-1.
\end{enumerate}
\subsection{来源}  
\begin{itemize}
\item Courant, Richard; Hilbert, David (1962). *Methods of Mathematical Physics*, Volume I, Wiley-Interscience.  
\item Sommerfeld, A. (1949). *Partial Differential Equations in Physics*. New York: Academic Press.  
\item Zachmanoglou, E. C.; Thoe, Dale W. (1986). *Introduction to Partial Differential Equations with Applications*. New York: Dover. ISBN 9780486652511.  
\end{itemize}
\subsection{进一步阅读}  
\begin{itemize}
\item Evans, L. C. (1998). *Partial Differential Equations*. Providence: American Mathematical Society. ISBN 978-0-8218-0772-9.  
\item Petrovsky, I. G. (1967). *Partial Differential Equations*. Philadelphia: W. B. Saunders.  
\item Polyanin, A. D. (2002). *Handbook of Linear Partial Differential Equations for Engineers and Scientists*. Boca Raton: Chapman & Hall/CRC Press. ISBN 978-1-58488-299-2.
\end{itemize}
\subsection{外部链接} 
\begin{itemize}
\item "拉普拉斯方程",《数学百科全书》,EMS出版社,2001年[1994]  
\item 在 EqWorld: The World of Mathematical Equations 上的拉普拉斯方程(特解和边值问题)。  
\item 来自 exampleproblems.com 的使用拉普拉斯方程的初边值问题示例。  
\item Weisstein, Eric W. "拉普拉斯方程"。MathWorld。  
\item 了解如何通过边界元法数值求解由拉普拉斯方程支配的边值问题。
\end{itemize}