% 费米子
% license CCBYSA3
% type Wiki

(本文根据 CC-BY-SA 协议转载自原搜狗科学百科对英文维基百科的翻译)

\begin{figure}[ht]
\centering
\includegraphics[width=6cm]{./figures/33787a12d530ccba.png}
\caption{恩利克·费密} \label{fig_Fermio_1}
\end{figure}

在 粒子物理学 中, \textbf{费米子}是一种遵循 费米-狄拉克统计 的粒子。这些粒子服从 泡利不相容原理。费米子包括所有种类的 夸克 和 轻子,以及所有由 奇数 个上述粒子所组成的 复合粒子,比如所 重子和许多 原子及 原子核。费米子不同于服从 玻色-爱因斯坦统计的 玻色子。

费米子可以是 基本粒子,如 电子,或者它也可以是 复合粒子,如 质子。在任何合理的情况 相对论的 量子场论下,根据 自旋统计定理 ,有 整数自旋 的粒子是 玻色子, 半整数自旋 的粒子是费米子。

除了自旋特性,费米子还有另一个特殊性质:它们拥有守恒重子或轻子量子数。因此,通常所说的自旋统计关系实际上是自旋统计-量子数关系。[1]

作为泡利排斥原理的结果,在任何时候一个费米子只可以占据一个特定的 量子态 。如果多个费米子具有相同的空间概率分布,那么每个费米子的至少一个性质,如自旋,必须是不同的。费米子通常与 物质 相关,而玻色子通常是粒子之间力的传递者,尽管在目前的粒子物理学中,这两个概念之间的区别还不清楚。 弱相互作用 费米子也可以在极端条件下显示玻色子行为。 低温下不带电荷的费米子显示 超流动性 ,带电的费米子粒子显示 超导性 。

复合费米子,如质子和 中子 是 日常物质 的关键组成部分。

费米子这个名字是由英国理论物理学家 保罗·狄拉克 创造的,它来自意大利物理学家 恩利克·费密 的姓氏 。[2]

\subsection{基本费米子}
标准模型 识别两种基本费米子: 夸克和 轻子。总之,这个模型区分了24种不同的费米子。有六个夸克(上夸克, 下夸克, 奇异夸克, 粲夸克, 底夸克 和 顶夸克 )和六个轻子(电子, 电子中微子, μ介子, μ子中微子, τ粒子 和 τ中微子),以及相应的 反粒子 。

数学上,费米子有三种类型:
\begin{itemize}
\item Weyl费米子(无质量)。
\item 狄拉克费米子(有质量)
\item Majorana 费米子(每个都有自己的反粒子)。
\end{itemize}
大多数标准模型费米子被认为是狄拉克费米子,尽管目前还不知道 中微子 是狄拉克费米子还是马约纳费米子(或两者都是)。狄拉克费米子可以被视为两个Weyl费米子的组合。[3] 在2015年7月,Weyl费米子已经在在 Weyl半金属 中被发现.

\subsection{ 复合费米子}
复合粒子(例如 强子、原子核和原子)可以是玻色子或费米子,这取决于它们的成分。更准确地说,取决于自旋和统计之间的关系,包含奇数费米子的粒子本身就是费米子。它将有半整数自旋。

例如:
\begin{itemize}
\item 重子,如质子或中子,包含三个费米子夸克,因此它是费米子。
\item 碳-13 原子的原子核包含六个质子和七个中子,因此是费米子。
\item 氦-3 原子($^3He$)由两个质子、一个中子和两个电子组成,因此它是费米子。
\end{itemize}
由简单粒子根据一个势组成而形成的复合粒子中玻色子的数量对该复合粒子是玻色子还是费米子没有影响。

复合粒子(或系统)的费米子或玻色子行为只能在很大的距离(与系统的大小相比)才能看到。在邻近的时候,根据复合粒子(或系统)的组成成分,其空间结构开始变得重要。

费米子成对松散结合时会表现出玻色子行为。这是氦-3中超导和超流性质的起源 :在超导材料中,电子通过交换 声子 在氦-3中形成 库珀对,库珀对是通过自旋涨落形成的。

准粒子的 分数量子霍尔效应 ,也被称为 复合费米子 , 是附着有偶数个量子化涡旋的电子。

\subsubsection{2.1 斯格明子}
在量子场论中,可能存在拓扑扭曲的玻色子场结构。他们是行为类似粒子的相干态(或者 孤波 ),虽然所有组成粒子都是玻色子,但他们具有费米子性质。这是由 托尼·斯凯姆 在20世纪60年代早期发现的,所以由玻色子组成的费米子被命名为 斯格明子。

斯凯姆最初的例子包括在三维球体上取值的场,最初的 非线性西格玛模型 描述了大距离尺度下 介子 的行为 。在斯凯姆的模型中,根据对 量子色动力学 (QCD)的 大$N$ 或者 线性 近似,质子和中子是π介子场中具有费米子性质的 拓扑孤子。[来源请求]

虽然斯凯姆的例子涉及$\pi$介子物理学,但在量子电动力学中还有一个更熟悉的例子 磁单极子。一种含有 最小可能磁荷 的玻色子单极子和一个玻色子版本的电子将形成费米子 双荷子。

有人用电弱区的斯凯姆场和希格斯场之间作类比[4] 来假设所有费米子都是斯格明子。这可以解释为什么所有已知的费米子都有重子或轻子量子数,并为泡利不相容原理提供了一个物理机制。

\subsection{参考文献}
[1]
^Physical Review D volume 87, page 0550003, year 2013, author Weiner, Richard M., title "Spin-statistics-quantum number connection and supersymmetry" arxiv:1302.0969.

[2]
^Notes on Dirac's lecture Developments in Atomic Theory at Le Palais de la Découverte, 6 December 1945, UKNATARCHI Dirac Papers BW83/2/257889. See note 64 on page 331 in "The Strangest Man: The Hidden Life of Paul Dirac, Mystic of the Atom" by Graham Farmelo.

[3]
^T. Morii; C. S. Lim; S. N. Mukherjee (1 January 2004). The Physics of the Standard Model and Beyond. World Scientific. ISBN 978-981-279-560-1..

[4]
^Weiner, Richard M. (2010). "The Mysteries of Fermions". International Journal of Theoretical Physics. 49 (5): 1174–1180. arXiv:0901.3816. Bibcode:2010IJTP...49.1174W. doi:10.1007/s10773-010-0292-7..


















