% 行列式唯一性定理
% 行列式|唯一性

\pentry{行列式的性质\upref{DetPro}}
在行列式的性质\upref{DetPro} 一节中,我们看到行列式 $\det \mat A$ 关于矩阵 $\mat A$ 的行是斜对称的(或反对称),即交换 $\mat A$ 的任意两行行列式变号; 且 $\det \mat A$ 是 $\mat A$ 的行的多重线性函数(下面解释),等等一些其它性质。事实上,任意函数只要满足斜对称性和多重线性性就天然具备其它性质,具有斜对称性和多重线性性的函数简称\textbf{斜对称多重线性函数}。

本节将证明:若 $\mathcal D$ 是矩阵 $\mat A$ 的斜对称多重线性函数,那么它将和 $\mat A$ 的行列式 $\det \mat A$ 成比例,比例系数为 $\mathcal D(\mat E)$ ($\mat E$ 为单位矩阵),即 $\mathcal D(\mat A)=\mathcal D(\mat E)\cdot \det \mat A$。那么若函数 $\mathcal D$ 还满足条件 $\mathcal D(\mat E)=1$,则 $\mathcal D(\mat A)=\det\mat A$。

简言之,若矩阵 $\mat A$ 的斜对称多重线性函数 $\mathcal D$ 满足条件 $\mathcal D(\mat E)=1$ ,那么 $\mathcal D$ 就为矩阵的行列式函数 $\det$。这便是行列式唯一性定理。本定理将有助于与行列式有关的其它理解,比如外代数和行列式的联系。

\subsection{两个引理}

先引入两个有助于证明的定理。

所有矩阵元为实数的 $n$ 阶矩阵 $\mat A$ 的集合一般记为 $M_n(\mathbb R)$ ,其中 $\mathbb R$ 是实数集。

为方便讨论矩阵的函数与矩阵行列的关系,我们引入下面两个记号:
\begin{equation}
\begin{aligned}
&\mat A_{(i)}=(a_{i1},a_{i2},\cdots,a_{in}),\quad i=1,2,\cdots, n\\
&\mat A^{(j)}=[a_{1j},a_{2j},\cdots,a_{nj}],\quad j=1,2,\cdots, n
\end{aligned}
\end{equation}
其中,$\mat A_{(i)},\mat A^{(j)}$ 分别表示矩阵 $\mat A=(a_{ij})$ 的第 $i$ 行和第 $j$ 列。于是矩阵 $\mat A$ 的函数 $\mathcal D(\mat A)$ 可记为 
\begin{equation}
\begin{aligned}
&\mathcal D(\mat A)=\mathcal D(\mat A_{(1)},\mat A_{(2)},\cdots,\mat A_{(n)})\\
&\mathcal D(\mat A)=\mathcal D(\mat A^{(1)},\mat A^{(2)},\cdots,\mat A^{(n)})\\
\end{aligned}
\end{equation}
当讨论 $\mathcal D$ 与矩阵行的关系时,往往用上式第一中记法,反之讨论 $\mathcal D$ 与矩阵列的关系时用第二种记法。

\begin{lemma}{}\label{lem_DetoD_1}
若任意函数 $\mathcal D:M_n(\mathbb R)\rightarrow \mathbb R$ 满足下面两个性质:
\begin{enumerate}
\item \textbf{斜对称性:}任意交换矩阵的两行函数 $\mathcal D$ 变号。即任意 $i\neq j\in \{1,2,\cdots,n\}$,有
\begin{equation}
\mathcal D(\mat A_{(1)},\cdots,\mat A_{(i)},\cdots,\mat A_{(j)},\cdots,\mat A_{(n)})=-\mathcal D(\mat A_{(1)},\cdots,\mat A_{(j)},\cdots,\mat A_{(i)},\cdots,\mat A_{(n)})
\end{equation}
\item \textbf{多重线性性:}函数 $\mathcal D$ 是矩阵 $\mat A$ 的任一行 $\mat A_{(k)}$ 的线性函数。即
\begin{equation}\label{eq_DetoD_1}
\begin{aligned}
&\mathcal D(\mat A_{(1)},\cdots,\alpha \mat A'_{(k)}+\beta\mat A''_{(k)},\cdots,\mat A_{(n)})\\
&=\alpha\mathcal D(\mat A_{(1)},\cdots,\mat A'_{(k)},\cdots,\mat A_{(n)})+\beta\mathcal D(\mat A_{(1)},\cdots,\mat A''_{(k)},\cdots,\mat A_{(n)}),\quad\forall\alpha,\beta\in\mathbb R
\end{aligned}
\end{equation}
\end{enumerate}
那么,$\mathcal D$ 亦满足:
\begin{enumerate}
\item $\mathcal D(\lambda\mat A)=\lambda^n\mathcal D(\mat A)$
\item 某 $\mat A$ 有一行为0,比如第 $i$ 行 $\mat A_{(i)}=\bvec 0$ ,则 $\mathcal D(\mat A)=0$
\item 若 $\mat A$ 有两行相同,比如 $\mat A_{(i)}=\mat A_{(j)},i\neq j$,则 $\mathcal D(\mat A)=0$
\item 在第 $i$ 行 $\mat A_{(i)}$加上 一常数 $\lambda$ 乘以任一 $j\neq i$ 行 $\mat A_{(j)}$,其值不变,即
\begin{equation}
\mathcal D(\mat A_{(1)},\cdots,\mat A_{(i)}+\lambda \mat A_{(j)},\cdots,\mat A_{(n)})=\mathcal D(\mat A_{(1)},\cdots,\mat A_{(i)},\cdots,\mat A_{(n)})
\end{equation}
\end{enumerate}
\end{lemma}
引理结论的1,2容易通过多重线性性\autoref{eq_DetoD_1} 证得!而结论3容易通过斜对称性得到,结论4通过多重线性性和结论3得到。

\begin{lemma}{}\label{lem_DetoD_2}
设 $\mat A$ 是一个 $n$ 阶上三角矩阵:
\begin{equation}
\begin{pmatrix}
&a_{11}&a_{12}&\cdots&a_{1n}\\
&0&a_{22}&\cdots&a_{2n}\\
&\vdots&\vdots&\vdots&\vdots\\
&0&0&\cdots&a_{nn}
\end{pmatrix}
\end{equation}
$\mat E$ 是单位矩阵,且 $\mathcal D$ 是满足\autoref{lem_DetoD_1} 条件的任意函数,则
\begin{equation}
\mathcal D(\mat A)=\mathcal D(\mat E)a_{11}a_{22}\cdots a_{nn}
\end{equation}
\end{lemma}

\textbf{证明:}显然,
\begin{equation}
\begin{aligned}
\mat A_{(1)}&=(a_{11},a_{12},\cdots,a_{1n})\\
\mat A_{(i)}&=(0,\cdots,0,a_{ii},\cdots,a_{in})\\
\mat A_{(n)}&=(0,\cdots,0,a_{nn})=a_{nn}(0,\cdots,0,1)=a_{nn}\mat E_{(n)}
\end{aligned}
\end{equation}
由 $\mathcal D$ 的多重线性性,有
\begin{equation}\label{eq_DetoD_2}
\begin{aligned}
\mathcal D(\mat A)&=\mathcal D(\mat A_{(1)},\cdots,\mat A_{(n-1)},a_{nn}\mat E_{(n)})\\
&=a_{nn}\mathcal D(\mat A_{(1)},\cdots,\mat A_{(n-1)},\mat E_{(n)})
\end{aligned}
\end{equation}

对\autoref{eq_DetoD_2} 最后一等式中 $\mathcal D$ 内的矩阵进行下面的变换,从矩阵的第 $i$ ($i\neq n$)行减去最后一行与 $a_{in}$ 的乘积,即第 $i$ 行变为 $\mat A_{(i)}-a_{in}\mat E_{(n)}$。这时最后一列的元素除第 $n$ 个都变为0,而第 $n-1$ 行变成 $a_{n-1,n-1}\mat E_{(n-1)}$。这时,由\autoref{lem_DetoD_1} 的4,$\mathcal D$ 的值不变,即
\begin{equation}
\mathcal D(\mat A)=a_{nn}a_{n-1,n-1}\mathcal D(\overline{\mat A}_{(1)},\cdots,\mat E_{(n-1)},\mat E_{(n)})
\end{equation}
重复同样的过程,最后即得
\begin{equation}
\mathcal D(\mat A)=a_{nn}\cdots a_{11}\cdot \mathcal D(\mat E)
\end{equation}

\textbf{证毕!}

显然行列式函数 $\det$ 是斜对称多重线性函数,所以具有\autoref{lem_DetoD_1} ,\autoref{lem_DetoD_2} 的性质。
\subsection{行列式唯一性定理}
现在来证明行列式的唯一性定理。
\begin{theorem}{行列式唯一性定理}
设 $\mathcal D:M_n(\mathbb R)\rightarrow \mathbb R$ 是斜对称多重线性函数,则
\begin{equation}
\mathcal D(\mat A)=\mathcal D(\mat E)\cdot \det \mat A
\end{equation}
特别的,在 $\mathcal D(\mat E)=1$ 时,$\mathcal D(\mat A)=\det \mat A$。
\end{theorem}
\textbf{证明:}因为借助2-型初等变换,即矩阵某一行加其它行乘任意常数,矩阵行列式不变;而借助1-型初等变换,即交换矩阵任意两行,行列式变号(\autoref{lem_DetoD_1} )。而矩阵都可用初等变换化为上三角矩阵(见高斯消元法\upref{GAUSS})。假设 $\overline{\mat A}$ 为 $\mat A$ 的上三角矩阵,在转化过程中初等1-型变换进行了 $q$ 次,则
\begin{equation}
\begin{aligned}
&\det{\mat A}=(-1)^q\det{\overline{\mat A}}=(-1)^q\overline{a}_{11}\cdots \overline{a}_{nn}\\
&\mathcal D(\mat A)=(-1)^q\mathcal{D}(\overline{\mat A})
\end{aligned}
\end{equation}
由\autoref{lem_DetoD_2}
\begin{equation}
\mathcal D(\overline{\mat A})=\overline{a}_{11}\cdots \overline{a}_{nn}\mathcal D(\mat E)
\end{equation}
故
\begin{equation}
\mathcal D(\mat A)=\mathcal D(\mat E)\cdot \det \mat A
\end{equation}

\textbf{证毕!}