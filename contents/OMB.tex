% 欧姆表
% license CCBYSA3
% type Wiki

(本文根据 CC-BY-SA 协议转载自原搜狗科学百科对英文维基百科的翻译)

\begin{figure}[ht]
\centering
\includegraphics[width=6cm]{./figures/89090fe7d673ffe0.png}
\caption{模拟欧姆表} \label{fig_OMB_1}
\end{figure}
\textbf{欧姆表}是一种测量电阻的电子仪器,电阻与电流成反比。微欧姆表专门测量微小电阻。兆欧姆表(注册商标又做高阻表),则专门测量较大的电阻。电阻的单位是欧姆($\Omega
$)。

\subsection{设计沿革}
最初的欧姆表是基于一种被称为“比率计”的仪表移动设计的,[1][2] 此移动类似于后来的欧姆表内用到的电流计式运动,它们的区别是比率计式运动利用一条导电“韧带”而不是细弹簧提供指针恢复力,这样不会产生净旋转力。此外,比率计式运动是由两个线圈决定的,一个线圈通过一串联电阻连接到电池电源,另一个通过另一电阻器和待测电阻元件连接到相同的电池电源。仪表上的指示值与通过两个线圈的电流之比成正比,而该比率由被测元件的电阻大小决定。这种设计有双重优点:首先,电阻值的指示完全独立于电池电压值(只要有一定电压提供),且不需要被调零。其次,尽管电阻刻度是非线性的,但在整个偏转范围内,刻度都能保持准确。除此之外,通过互换两个线圈,它可提供第二个测量范围,其刻度与第一个的相反。这种仪器的一个特点是,一旦测量通路断开它将随机指向一个电阻值(因为电源已断开)。这样的设计不容易被集成到万用表中,所以这种类型的欧姆表只能专门用来测量电阻。依靠手摇发电机运行的绝缘测试器也是基于同样的原理来确保其指示完全独立于实际电压。

后续的欧姆表设计则利用了一个小电池与待测电阻通过一个电流计串联来测量通过该电阻的电流。因为电池的固定电压值保证了随着电阻的增加,通过电流表的电流(以及偏转)会减小,所以该电流计的刻度可以转化为电阻刻度(利用欧姆定律)并以欧姆为单位来模拟欧姆表。该欧姆表自身形成回路,因此不能在组装好的电路中使用。这种设计比以前的设计更简单经济,并且易于集成到万用表中,因此是迄今为止最常见的模拟欧姆表类型。但这类欧姆表有两个固有的缺点。首先,在每次测量前,它都需要将两个端口短路并同时调整指针来调零。因为电池电压会随着时间的推移降低, 所以仪表内的串联电阻也需要降低来确保其在完全偏转时指向零。其次,由于第一条缺点的影响,一个给定的待测电阻的实际测量左右偏转幅度会随着内阻阻值的变化而变化,而仅在刻度尺中心的指示能始终保持正确,这就是为什么这类设计的欧姆表总是标明 “仅在标尺中心”精确。

一种更精确的欧姆表类型则采用了一个能使恒定电流值I通过电阻的电子电路与一个用来测量电阻两端电压值V的电路。这些测量值(I, V)经模拟数字转换器(adc)数字化之后,被一个微控制器或微处理器根据欧姆定律计算得出电阻值(V/I),随后此电阻值会被解码至显示器以向用户显示此时测出的电阻。这种类型的仪表同时测量了电流、电压和电阻,因此它们经常被用在数字万用表中。

\subsection{精密欧姆表}
对于极低电阻的高精度测量,上述类型的仪表是不够精确的。一部分是因为待测电阻太小,无法与欧姆表固有内阻(可以通过除以电流值估得)相比,所以指针本身的偏转变化将过小以至于无法准确读出;但最主要的原因是上述欧姆表测出的电阻值实际是导线电阻、接触电阻和被测电阻值的总和。为了减小它们带来的影响,精密欧姆表引入了四个端点,称为开尔文触点。其中两个端点允许电流输入和输出电表,而另外两个端点允许一电表测量待测电阻两端的电压降。在此设计中,电源通过外部的两端点对与待测电阻串联,而另一对端点则与测量电压降的电流计并联。这种设计使得由连接第一对端口的导线电阻和接触电阻引起的所有电压降都被仪表忽略,从而实现仅对待测电阻的精密测量。这种四端测量技术被称为开尔文传感,源自威廉·汤姆孙·开尔文勋爵(William Thomson, Lord Kelvin),他于1861年发明了开尔文电桥来测量极低的电阻。四端传感方法也可用于较低电阻的精确测量。

\subsection{参考文献}
[1]
^https://web.archive.org/web/20221025110754/http://www.g1jbg.co.uk/pdf/MeggerBK.pdf A pocket book on the use of Megger insulation and continuity testers..

[2]
^https://web.archive.org/web/20221025110754/http://www.prolexdesign.com/images/evohmmeter.jpg[永久失效连结] Illustration of type. Note the absence of any zero adjustment and the changed scale direction between ranges.[失效连结].