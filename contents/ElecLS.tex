% 电磁场中粒子的拉氏量
% 拉氏量|作用量|电磁势|拉格朗日方程
\pentry{拉格朗日电磁势\upref{EMLagP},电磁场张量\upref{EMFT}}

我们继续使用自然单位制,令 $\mu_0=\epsilon_0=c=1$ 来简化表达。依照习惯,上下标使用希腊字母如 $\mu, \nu$ 时,取值范围为 $\{0, 1, 2, 3\}$;使用拉丁字母如 $i, j$ 时,取值范围为 $\{1, 2, 3\}$。约定闵氏时空度规为 $(-1,1,1,1)$。

设一个在闵氏时空中自由运动的(不带电的)粒子,其作用量应当在洛伦兹变换下保持不变。唯一能写出的作用量是
\begin{equation}
S=\int L\dd t=-m\int \dd \tau=-m\int \sqrt{-\dd x^\mu \dd x_\mu}
\end{equation}
其中 $m$ 为粒子的静止质量,$\dd \tau$ 为粒子世界线的\textbf{固有时间}(或者说原时)间隔,即一小段微元的时空距离:
\begin{equation}
\dd \tau=\sqrt{-\dd x^\mu\dd x_\mu}=\sqrt{(\dd t)^2-(\dd x)^2-(\dd y)^2-(\dd z)^2}=\sqrt{1-v^2}\dd t
\end{equation}

所以自由粒子的拉氏量为
\begin{equation}\label{ElecLS_eq1}
L=-m\sqrt{1-v^2}
\end{equation}
如果取非相对论极限 $v\ll 1$,那么 $L_{\rm{non-rel}}=-m+\frac{1}{2}mv^2$,舍去常数项就可以得到 $L=\frac{1}{2}mv^2$,这与牛顿力学体系下自由粒子的拉格朗日量是一致的。
\subsection{电磁场中粒子的拉氏量}
现在考虑带电粒子的拉氏量。根据拉格朗日电磁势 中 \upref{EMLagP} 的证明,带电粒子在电磁场中的广义势为 $U=q(\phi-\bvec A\cdot \bvec v)$,根据 $U=T-U$,可以给\autoref{ElecLS_eq1} 增添一项,得到带电粒子的拉氏量
\begin{equation}\label{ElecLS_eq2}
L=-m\sqrt{1-v^2}-q\phi+q\bvec v\cdot \bvec A
\end{equation}

那么带电粒子的作用量为
\begin{equation}\label{ElecLS_eq3}
\begin{aligned}
S&=\int L\dd t=\int \qty(-m\sqrt{1-v^2}-q\phi +q\bvec v\cdot \bvec A)\dd t\\
&=\int (-m-q\gamma\phi+q\gamma\bvec v\cdot \bvec A)\dd \tau\\
&=\int (-m+qU^\mu A_\mu)\dd \tau
\end{aligned}
\end{equation}
或者可以利用 $U^\mu= \frac{\dd x^\mu}{\dd \tau}$ 将它改写为
\begin{equation}\label{ElecLS_eq4}
S=\int -m\dd \tau+qA_\mu \dd x^\mu
\end{equation}
其中 $A^\mu=(\phi,A_x,A_y,A_z)=(\phi,\bvec A)$ 是4-矢量势。带电粒子的作用量具有协变形式,这也是符合我们预期的—作用量具有洛伦兹不变性。

注意到 $A_\mu$ 是时空坐标 $x^\mu$ 的函数。保持时空舞台上的电磁场不变,即 $A_\mu$ 关于 $x^\mu$ 的函数不变,如果对 $x^\mu$ 作变分并保持初始和终止的时空坐标不变,根据最小作用量原理,$\delta S=0$,于是我们有
\begin{equation}
\begin{aligned}
\delta S&=\delta\qty( -m\int \sqrt{-\dd x_\mu \dd x^\mu} +q\int A_\mu \dd x^\mu)\\
&=-m\int \frac{-\dd x_\mu \delta (\dd x^\mu)}{\sqrt{-\dd x_\mu \dd x^\mu}}+q\int \qty(\delta(A_\mu)\dd x^\mu + A_\mu \delta(\dd x^\mu))\\
&=m\int \frac{\dd x_\mu}{\dd \tau}\dd(\delta x^\mu)+q\int\qty(\frac{\partial A_\mu}{\partial x_\nu})\dd x^\mu \delta x_\nu+q\int A_\mu \dd(\delta x^\mu)\\
&=-m\int\dd U_\mu \delta x^\mu+q\int \partial_\nu A_\mu \dd x^\mu\delta x^\nu-q\int \dd A_\mu \delta x^\mu\\
&=\int \qty(-m\frac{\dd U_\mu}{\dd \tau} + q\partial_\mu A_\nu \frac{\dd x^\nu}{\dd \tau}-q\partial_\nu A_\mu\frac{\dd x^\nu}{\dd \tau})\delta x^\mu\dd \tau\\
\end{aligned}
\end{equation}

上面的推导多次用了分部积分。我们定义\textbf{电磁场张量}\upref{EMFT} $F_{\mu\nu}=\partial_{\mu}A_\nu-\partial_\nu A_\mu$,那么
\begin{equation}
\delta S=\int\qty(-m\frac{\dd U_\mu}{\dd \tau}+qF_{\mu\nu}U^\nu)\delta x^\mu \dd \tau=0
\end{equation}
所以由最小作用量原理得到了洛伦兹力公式(也可以将 $\mu$ 指标上升变成 \autoref{EMFT_eq3}~\upref{EMFT} 的形式)
\begin{equation}
m\frac{\dd U_\mu}{\dd \tau}=qF_{\mu\nu} U^\nu
\end{equation}
\subsection{电磁场中粒子的哈密顿量}
\autoref{EMLagP_eq1}~\upref{EMLagP}描述了非相对论极限下电磁场中粒子的广义动量。在考论相对论粒子时,可以根据\autoref{ElecLS_eq2} 可以计算粒子的广义动量:
\begin{equation}
\bvec p=\frac{\partial L}{\partial \dot {\bvec q}}=\frac{m\bvec v}{\sqrt{1-v^2}}+q\bvec A=\bvec P+q\bvec A
\end{equation}
其中 $\bvec q$ 为广义坐标(取 $x,y,z$),$\bvec p$ 为与之共轭的广义动量(或者称正则动量),$\bvec P=\gamma m\bvec v$ 是粒子的相对论动量。

哈密顿量为
\begin{equation}
\begin{aligned}
H&=\bvec p \dot{\bvec q}-L=\frac{mv^2}{\sqrt{1-v^2}}+q\bvec v\cdot \bvec A-(-m\sqrt{1-v^2}-q\phi+q\bvec v\cdot \bvec A)\\
&=\frac{m}{\sqrt{1-v^2}}+q\phi=\sqrt{m^2+\qty(\frac{mv}{\sqrt{1-v^2}})^2}+q\phi\\
&=\sqrt{m^2+\bvec P^2}+q\phi\\
&=\sqrt{m^2+(\bvec p-q\bvec A)^2}+q\phi
\end{aligned}
\end{equation}
非相对论极限下 $v\ll 1$,则 $(\bvec p-q\bvec A)^2 \ll m^2$,此时哈密顿量为
\begin{equation}
H_{\rm{non-rel}}=\frac{1}{2m}(\bvec p-q\bvec A)^2+q\phi
\end{equation}