% 三维量子简谐振子(球坐标系)
% keys 量子力学|薛定谔方程|球坐标|球谐函数
% license Xiao
% type Tutor

\begin{issues}
\issueDraft
\end{issues}

\pentry{球坐标系中的径向方程\nref{nod_RadSE}, 简谐振子(升降算符)\nref{nod_QSHOop}}{nod_7ed9}

本文使用原子单位制\upref{AU}。 我们希望求解定态薛定谔方程(引用未完成)
\begin{equation}
-\frac{1}{2m} \laplacian {\Psi} + V(\bvec r)\Psi = E \Psi~,
\end{equation}
其中势能函数为
\begin{equation}
V(\bvec r) = \frac{1}{2}m\omega^2 r^2~.
\end{equation}
我们已知角向波函数是球谐函数\upref{RadSE} $Y_{l,m}(\uvec r)$。 只需解出方程(\autoref{eq_RadSE_1}~\upref{RadSE})即可
\begin{equation}\label{eq_SHOSph_1}
-\frac{1}{2m} \dv[2]{\psi_l}{r} + \qty[V(r) + \frac{l(l + 1)}{2mr^2}]\psi_l = E\psi_l~.
\end{equation}
总波函数和能级为
\begin{equation}
\psi_{n,l,m} = R_{n,l}(r) Y_{l,m}(\Omega)~,
\qquad
E_{n,l} = \qty(2n + l + \frac32) \omega~.
\end{equation}    

我们可以把\autoref{eq_SHOSph_1} 想象成一个一维势阱问题, 角动量量子数 $l$ 决定势能 $V(r) + l(l + 1)/(2mr^2)$ 在原点附近的形状。 由于 $V(r)$ 随 $r$ 无限变大, 所以总的束缚态个数是无限的, $n, l$ 也可以取任意非负整数。 但对于其他一些有限深的球对称势阱, 由于束缚态个数有限, $l$ 的取值存在上限。

令 $\beta = 1/\sqrt{m\omega}$, $x = r/\beta$, 则径向波函数为
\begin{equation}
R_{n,l}(r) = \frac{1}{\beta^{3/2} \pi^{1/4}} \sqrt{\frac{2^{n+l+2} n!}{(2n + 2l + 1)!!}} x^l L_n^{l+1/2}(x^2) \E^{-x^2/2}~,
\end{equation}
其中 $L_n^{l+1/2}$ 是连带拉盖尔多项式\upref{Laguer}。

前几个束缚态如下, 其中球谐函数产生的简并数为 $\sum_l (2l + 1)~.$
\begin{itemize}
\item $E = 3 \omega /2$ ( $\deg  = 1$ )
\begin{equation}
R_{0,0}(r) = \frac{1}{\beta^{3/2} \pi^{1/4}} 2\E^{-\frac12 x^2}~.
\end{equation}

\item $E = 5 \omega /2$ ( $\deg  = 3$ )
\begin{equation}
R_{0,1}(r) = \frac{1}{\beta^{3/2} \pi^{1/4}} \frac{2\sqrt 6}{3} x \E^{-\frac12 x^2}~.
\end{equation}

\item $E = 7 \omega /2$ ( $\deg  = 6$ )
\begin{equation}
R_{0,2}(r) = \frac{1}{\beta^{3/2} \pi^{1/4}} \frac{4}{\sqrt{15}} x^2 \E^{- x^2/2}~,
\end{equation}
\begin{equation}
R_{1,0}(r) = \frac{1}{\beta^{3/2} \pi^{1/4}} \frac{2\sqrt 6}{3} \qty(\frac32 - x^2) \E^{-x^2/2}~.
\end{equation}

\item $E = 9 \omega /2$ ( $\deg  = 8$ )
\begin{equation}
R_{0,3}(r) = \frac{1}{\beta^{3/2} \pi^{1/4}} 4\sqrt{\frac{2}{105}} x^3 \E^{-x^2/2}~,
\end{equation}
\begin{equation}
R_{1,1}(r) = \frac{1}{\beta^{3/2} \pi^{1/4}} \frac{4}{\sqrt{15}} \qty(\frac52 - x^2) x \E^{-x^2/2}~.
\end{equation}
\end{itemize}
