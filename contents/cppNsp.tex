% C++ 的 namespace (笔记)
% namespace|c++|declaration|using|directive

\begin{issues}
\issueDraft
\end{issues}

\begin{itemize}
\item 使用 \verb|namespace| 函数的一个重要例外就是, 如果调用函数的参数有 \verb|namespace|, 函数前面就不必要声明 namespace。 这样不同 namespace 中的同名函数就有可能冲突。 解决的办法就是还是在函数前声明 namespace (麻烦),或者直接更改函数名。 例子:
\begin{lstlisting}[language=cpp]
#include <iostream>
#include <cmath>
#include <complex>

namespace my {
	template< class T >
	std::complex<T> sin(const std::complex<T>& z)
	{
		return z;
	}
}

int main()
{
	using namespace my;
	std::complex<double> x(1,2);
	std::cout << sin(x) << std::endl; // 编译不通过, ambiguous sin
}
\end{lstlisting}
\item STL 中 include 的 c header 并没有任何 namespace, 例如 \verb|#include <iostream>| 以后, \verb|getchar()| 没有 namespace。 这是历史遗留问题。 一些数学函数也没有。
\item 定义 namespace 的大括号内部如果 using 其他 namespace 中的 name,那么这个 name 也会在这个 namespace 中出现。
\item 定义 namespace 的大括号中如果定义了一个 name,那么大括号外定义的相同的名字就会在大括号中被隐藏, 即使是不同的函数 overloading 也会被隐藏。 可以在函数名前面加 global namespace \verb|::| 来解决。
\item 如果有一个没有 namespace 的函数和一个有 namespace 的同 signature 函数, 使用 using declaration (\verb|using std::name;|) 不会发生冲突(没有 namespace 的函数被隐藏),但 using directive (\verb|using namespace std;|) 就会冲突!
\item 在 global namespace 中不要使用开头有下划线的名字。
\item 如果在 \verb|namespace A {}| 内部使用 \verb|using namespace B;| 在 \verb|namespace A {}| 外部使用 \verb|using A::xx;| 是安全的, 即不会引入 \verb|using namespace B|。但如果使用 \verb|using namespace A|, 则会引入 \verb|using namespace B|。
\item 即使在 \verb|namespace A {}| 内部使用 \verb|using namespace B;|  新定义的函数也不会与 B 中的同名函数 overload, 而是会直接覆盖 B 中的函数。
\end{itemize}
