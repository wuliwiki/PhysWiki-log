% 正交变换与对称变换
% license Usr
% type Tutor

本文用“$(x,y)$”表示对任意两个向量作内积。
\subsubsection{正交变换}
\begin{definition}{}
定义实内积空间$V$上的满射$\mathcal A$。对于任意$(x,y)\in V$,若有:
\begin{equation}
(x,y)=(\mathcal A x,\mathcal A y)~,
\end{equation}
则称$\mathcal A$是$V$上的\textbf{正交变换}。也就是说,正交变换是保内积不变的满射,从而保向量长度和向量之间的夹角不变。
\end{definition}
实际上,正交变换是线性变换。
\begin{theorem}{}
实内积空间$V$上的正交变换$A$一定是线性映射。
\end{theorem}
\textbf{证明:}
\begin{equation}
\begin{aligned}
|A(x+y)-(Ax+Ay)|^2&=(x+y)^2+x^2+y^2-2\left(A(x+y),Ax+Ay\right)\\
&=(x+y)^2+x^2+y^2-2(x+y,x)-2(x+y,y)\\
&=0
\end{aligned}
~,\end{equation}

因而$A(x+y)=Ax+Ay$。同理可得$A(kx)=kA(x)$。

设$\{\bvec e_i\}^k_{i=1}$是$V$上的一组基,由线性性可知,$\{A\bvec e_i\}^k_{i=1}$也是线性无关组。又因为$A$是满射,所以$\{A\bvec e_i\}^k_{i=1}$是$V$上的一组基,则该线性映射既单又满,是“同构映射”。由正交变换保内积可知,$A$\textbf{把标准正交基映射为标准正交基。}

总结上述讨论,易证:$A$是正交变换$\Longleftrightarrow A$\textbf{把标准正交基映射为标准正交基。}后者是正交矩阵的定义,可见正交矩阵是正交变换的矩阵表示。\textbf{正交矩阵保二次型不变即正交变换的内积定义。}

在欧几里得空间中,正交矩阵的行列式为$1$或者$-1$,常称行列式为$1$的正交变换为\textbf{第一类的(旋转)};行列式为$-1$的正交变换则是\textbf{第二类的}。
\subsubsection{对称变换}


