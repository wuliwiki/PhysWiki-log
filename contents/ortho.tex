% 正交变换与对称变换
% license Xiao
% type Tutor

本文用“$(\bvec x,\bvec y)$”表示对任意两个向量作内积。本文的“内积”是狭义上的\textbf{正定性对称双线性函数}。实际上,正交变换的概念可以拓展至配备任意非退化二次型的线性空间,可见正交矩阵一节\autoref{def_lnal05_1}。
\subsection{正交变换}
\begin{definition}{}\label{def_ortho_1}
定义\textbf{实内积}空间$V$上的满射$\mathcal A$。对于任意$\bvec x,\bvec y\in V$,若有:
\begin{equation}
(\bvec x,\bvec y)=(\mathcal A \bvec x,\mathcal A \bvec y)~,
\end{equation}
则称$\mathcal A$是$V$上的\textbf{正交变换}。也就是说,正交变换是保内积不变的满射,从而保向量长度和向量之间的夹角不变。
\end{definition}
实际上,正交变换是线性变换。
\begin{theorem}{}\label{the_ortho_1}
实内积空间$V$上的正交变换$\mathcal A$一定是线性映射。
\end{theorem}
\textbf{证明:}
\begin{equation}
\begin{aligned}
|\mathcal A(\bvec x+\bvec y)-(\mathcal A\bvec x+\mathcal A\bvec y)|^2&=(\bvec x+\bvec y)^2+\bvec x^2+\bvec y^2-2\left(\mathcal A(\bvec x+\bvec y),\mathcal A\bvec x+\mathcal A\bvec y\right)\\
&=(\bvec x+\bvec y)^2+\bvec x^2+\bvec y^2-2(\bvec x+\bvec y,\bvec x)-2(\bvec x+\bvec y,\bvec y)\\
&=0
\end{aligned}
~,\end{equation}

因而$\mathcal A(\bvec x+\bvec y)=\mathcal A\bvec x+\mathcal A\bvec y$。同理可得$\mathcal A(k\bvec x)=k\mathcal A(\bvec x)$。

\begin{theorem}{}
实内积空间$V$上的正交变换$\mathcal A$一定是双射。
\end{theorem}
\textbf{证明:}由于正交变换是满射,所以只需要利用线性证明其是单射即可。
设存在$\mathcal A\bvec x=\mathcal A\bvec y$,则$\mathcal A\bvec x-\mathcal A\bvec y=A(\bvec x-\bvec y)=\bvec 0$。又因为$(\bvec x-\bvec y,\bvec x-\bvec y)=(A(\bvec x-\bvec y),A(\bvec x-\bvec y))=\bvec 0$,由正定性可得$\bvec x=\bvec y$,得证。

设$\{\bvec e_i\}$是$V$上的一组标准正交基。由正交变换保内积可知,该基映射后依然是标准正交的。又因为正交变换是单射,所以映射后的向量组是线性无关的,所以\textbf{正交变换}$\mathcal A$\textbf{把标准正交基映射为标准正交基。}

也就是说:$\mathcal A$是正交变换$\Longleftrightarrow \mathcal A$\textbf{把标准正交基映射为标准正交基。}后者是正交矩阵的定义,可见正交矩阵是正交变换的矩阵表示。\textbf{正交矩阵保二次型不变的性质等价于正交变换的内积定义,}\footnote{广义内积即二次型决定的对称双线性函数:$(\bvec x,\bvec y)\equiv \frac{1}{2}(q(\bvec x+\bvec y)-q(\bvec x)-q(\bvec y))$,利用该性质可证明保二次型不变等价于内积不变。}或者说——

$\mathcal A$是正交变换$\Longleftrightarrow $对任意$\bvec x\in V$,线性变换$\mathcal A$满足$|\mathcal A\bvec x|=|\bvec x|$。\footnote{该充要条件并不能推广至广义内积空间。因为对于退化的二次型,保内积不变无法推出“将线性无关组映射为线性无关组”。}

令$O(n)$表示$n$维线性空间$V$上全体正交变换的集合。由定义可知,该集合具有如下性质:
\begin{enumerate}
\item 封闭性。若$A,B\in O(n)$,则$AB\in O(n)$;
\item 结合性。若$A,B,C\in O(n)$,则$A(BC)=(AB)C$;
\item 单位元存在性。$I\in O(n)$;
\item 逆元存在性。若$A\in O(n)$,则$A^{-1}\in O(n)$;
\end{enumerate}
所以,正交变换构成一个群。

在欧几里得空间中,正交矩阵的行列式为$1$或者$-1$,常称行列式为$1$的正交变换为\textbf{第一类的(旋转)},用$SO(n)$表示;称行列式为$-1$的正交变换为\textbf{第二类的}正交变换。
\subsubsection{正交变换的本征值}
由上述讨论可知,正交变换若有本征向量,则本征值的模长为$1$,从而保证向量的模长不变。又因为$O(3)$定义在实数域的线性空间上,因此本征值必为实数。
\begin{corollary}{}
 设$\lambda$为正交变换$A$的本征值,则$\lambda=\pm 1$。
\end{corollary}
\begin{corollary}{}
若$V$是奇数维线性空间,$A\in SO(3)$必有本征值$1$。
\end{corollary}
\textbf{证明:}
我们知道,在复数域上,矩阵行列式等于特征值的连乘。由于一元多项式方程里复根成对出现(若$\lambda_0$为本征值,则$\lambda_0^{*}$也为本征值),所以复根部分的本征值连乘结果必为$1$。于是剩下的奇数个实本征值连乘结果必为$1$,所以至少有一个本征值为$1$,得证。
\subsubsection{正交变换的块对角形式}
为了简化正交矩阵的形式,我们先来证明两个常用结论。
\begin{lemma}{}\label{lem_ortho_1}
设$f$为\textbf{实线性空间}$V$上的线性映射,存在$f$的不变子空间$W\subseteq V$,且$\opn{dim}W\in\{1,2\}$。
\end{lemma}
\textbf{证明:\footnote{参考Jier Peter的《代数学基础》}}

若$f$有若干个实本征值,则其对应的本征向量是$V$上的一维不变子空间。

将$n$维$V$复化为$U$。若$\{\bvec e_i\}$为$V$上的一组基,则任意$\bvec x\in U$可表示为:
\begin{equation}
\bvec x=a^i\bvec e_i+\mathrm i b^j\bvec e_j=\bvec u+\mathrm i\bvec v~.
\end{equation}
其中$a^i,b^i\in \mathbb R,\bvec u,\bvec v\in V$。显然$U\supseteq V$。
在复数域上,$f$有$n$个本征值。设有$k$个实本征值,则对应的本征向量是$V$上的一维不变子空间。设复本征值表示为$a+\mathrm ib$,对应本征向量表示为$\bvec x+\mathrm i \bvec y\,(\bvec x,\bvec y\in V)$。利用$f$的线性可得:
\begin{equation}
\begin{aligned}
f(\bvec x+\mathrm i \bvec y)&=(a+\mathrm i b)(\bvec x+\mathrm i \bvec y)\\
&=(a\bvec x-b\bvec y)+\mathrm i(a\bvec y+b\bvec x)\\
&=f(\bvec x)+\mathrm if(\bvec y)
\end{aligned}~.
\end{equation}
即:
\begin{equation}\label{eq_ortho_1}
\begin{aligned}
f(\bvec x)&=a\bvec x-b\bvec y\\
f(\bvec y)&=a\bvec y+b\bvec x
\end{aligned}~,
\end{equation}
所以$\opn{Span}\{\bvec x,\bvec y\}$为$f$在$V$上的一个二维不变子空间。
\begin{lemma}{}\label{lem_ortho_3}
若正交变换$A$有复根,对应本征向量$\bvec x=\bvec u+\mathrm i\bvec v$。则$\bvec u,\bvec v$是相互正交的本征向量。令$W=\opn{Span}\{\bvec u,\bvec v\}$,则有:
\begin{equation}\label{eq_ortho_2}
A|_W=\left[\begin{array}{rr}
\cos \phi & \sin \phi \\
-\sin \phi & \cos \phi
\end{array}\right]~.
\end{equation}
\end{lemma}
\textbf{证明:}
设复本征值$\lambda=a+b\mathrm i=\opn{cos}\phi+\mathrm i \opn{sin}\phi$,代入\autoref{eq_ortho_1} 便可得形式为\autoref{eq_ortho_2} 的$A|_W$。接下来只需要证明实部和虚部分量是标准正交即可。为方便计,把转置操作表示为'。

由于$k\bvec x$依然是$A$的本征向量,我们可以归一化其\textbf{实部分量},使得$\bvec u^2=1$。则归一化后的
\begin{equation}\label{eq_ortho_3}
\begin{aligned}
\bvec x^2&=(\bvec u+\mathrm i\bvec v)^2\\
&=1-\bvec v^2+2\mathrm i \bvec u\cdot\bvec v
\end{aligned}~.
\end{equation}

由于$A'=A^{-1}$且$A\bvec x=\lambda \bvec x$,左乘$A'$后得$\bvec x=\lambda A'\bvec x$,利用正交变换不改变向量模长,即$|\lambda|^2=1$,得$A'\bvec x=\lambda^*\bvec x$。所以
\begin{equation}
\bvec x'A'\bvec x=\lambda^* \bvec x'\bvec x=(Ax)'\bvec x=\lambda \bvec x'\bvec x~,
\end{equation}
因此$\bvec x^2=\bvec x'\bvec x=0$。代入\autoref{eq_ortho_3} 可得:
\begin{equation}
\bvec v^2=1,\,\bvec u\cdot \bvec v=0~,
\end{equation}
得证。
这两条引理说明,正交变换$A$的每个复本征值都对应一个实数域上的二维不变子空间,且$A$限制在该子空间上的形式总如\autoref{eq_ortho_2} 所示。若$A$为二阶矩阵且实数域上无特征根,则$A$唯一表示为该形式,乘以任意向量相当于旋转该向量。
\begin{exercise}{}
写出二阶正交矩阵的所有可能形式。
\end{exercise}
\begin{lemma}{}\label{lem_ortho_2}
设$A$为$V$上的正交变换,若$W$为其不变子空间,则$W^{\bot}$也是其不变子空间。
\end{lemma}
设$\{\bvec e_i\}$为$W$上的一组标准正交基,并扩展到全空间,使得$\opn{Span}\{\bvec {\theta}_i\}$张成$W^{\bot}$。由题设知$\{A\bvec e_i\}$依然张成$W$,由于$(A\bvec {\theta}_i,A\bvec e_i)=0$,因此$A\bvec{\theta}_i\in W^{\bot}$,得证。

由定理3可知,在$\{\bvec e_i\}\cup\{\bvec {\theta}_i\}$下,$A$有块对角形式:$A=A|_W\oplus A|_{W^{\bot}}$。实际上,该块对角形式可以进一步“细化”。
\begin{theorem}{}
设$A$为n维\textbf{实线性空间}$V$上的正交变换。则在$V$内存在一组\textbf{标准正交基},使得$A$具有如下块对角形式:
\begin{equation}
A=\left[\begin{array}{lllllll}
\lambda_1 & & & & & & \\
& \lambda_2 & & & & & \\
& & \ddots & & & 0 & \\
& & & \lambda_k & & & \\
& & & & S_1 & & \\
& & 0 & & & \ddots & \\
& & & & & & S_l
\end{array}\right]~,
\end{equation}
其中$\lambda_i=\pm 1(i=1,2...k)$,$S_i$为\autoref{eq_ortho_2},$\phi$由$\phi_i$代替。
\end{theorem}
\textbf{证明:}
若$n=1$,矩阵为一实数,设为a。由$(a\bvec x,a\bvec x)=\bvec x^2$可得,$a=\pm 1$。

若$n=2$且$A$没有本征值,由\autoref{lem_ortho_1} 可知,$A$在$V$上必有一二维不变子空间,$A$的形式就是旋转矩阵。
如果$A$有一本征值$\lambda_1$且$Ax=\lambda_1 \bvec x$,设单位向量$\bvec y\cdot \bvec x=0$,由\autoref{lem_ortho_2} 知,$\bvec y$也是$A$的本征向量。设$Ay=\lambda_2 \bvec y$,则
\begin{equation}
A=\left[\begin{array}{rr}
\lambda_1&0\\
0&\lambda_2
\end{array}
\right]~.
\end{equation}

若$A$无本征值,由\autoref{lem_ortho_3} 知其形式即二阶旋转矩阵,对应复根$\cos\phi-\I \sin\phi$。



接下来设$n>2$,用数学归纳法即可证明。若$A$无本征值,则必有一二维不变子空间,设为$S$,$A|_{S}$为\autoref{eq_ortho_2}。$A|_{S^{\bot}}$为正交矩阵,若$\lambda_1$为其本征值,则必定有一维不变子空间,设对应的本征向量为$\bvec x_1$,则可以扩展为$S^{\bot}$上的标准正交基,设$\bvec x_2$的正交补为$S_1$,则\begin{equation}
\left[\begin{array}{rrr}
A|_{S}&0&0\\
0&\lambda_1&0\\
0&0&A|_{S_1}
\end{array}
\right]~.
\end{equation}
则$\lambda_1$也是$A$的本征值,矛盾。故$A$由若干个形式如\autoref{eq_ortho_2} 的二阶矩阵直和而成。

若$A$有本征值,令其为$\lambda'_1$,则其本征向量可以扩展为全空间的标准正交基。在该本征向量的正交补空间讨论,若可以找到另一本征值$\lambda'_2$,则对应的本征向量又可以扩展为该正交补上的基矢组,经过相似变换得到$A$的新形式,$\lambda'_1$和$\lambda'_2$为对角元。以此类推,最后得到该定理的形式,$k$为$A$的实本征值数量。

\subsection{对称变换}
\begin{definition}{}
设$\mathcal B$为实线性空间上的线性变换,若对于任意$\bvec x,\bvec y\in V$有
\begin{equation}\label{eq_ortho_4}
(\bvec x,\mathcal B\bvec y)=(\mathcal B\bvec x,\bvec y)~,
\end{equation}
则称$B$为对称变换。
\end{definition}

若$V$为n维实线性空间,$B$为对称变换的矩阵表示。任选一组标准正交基,由\autoref{eq_ortho_4} 得:
\begin{equation}
\bvec x^TB\bvec y=\bvec x^TB^T\bvec y~.
\end{equation}
左乘$\bvec x$可得,
\begin{equation}
B\bvec y=B^T\bvec y~.
\end{equation}
因此,$B=B^T$,对称变换的矩阵表示是对称矩阵。
\begin{theorem}{}
$n$维欧式空间上的线性变换$\mathcal B$是对称变换当且仅当$\mathcal B$在任意标准正交基下的矩阵表示为对称矩阵。
\end{theorem}
前文的推导只要求基矢标准正交,因此该定理成立。

对称变换有类似正交变换的结论:
\begin{theorem}{}
若$U$是$B$的不变子空间,则$U^{\bot}$也是$B$的不变子空间。
\end{theorem}
\textbf{证明:}

任选$\bvec x\in U,\bvec y\in U^{\bot}$,则
\begin{equation}
(B\bvec y,\bvec x)=(\bvec y,B\bvec x)=0~,
\end{equation}
因而$B\bvec y\in U^{\bot}$,$ U^{\bot}$是$B$的不变子空间。
\begin{exercise}{}
设$\mathcal B$为$n$维欧式空间上的线性变换,则$\mathcal B$是正交变换$\Longleftrightarrow$存在一组标准正交基,使得$\mathcal B$在该基下的表示为对角矩阵。
\end{exercise}
下面罗列一些对称矩阵常见结论,$A$表示任一对称矩阵,证明略。
\begin{theorem}{}
$A$的特征值都为实数。
\end{theorem}
\begin{theorem}{}
若$a,b$是$A$的不同特征值,则对应的特征向量相互正交。
\end{theorem}
\begin{theorem}{}\label{the_ortho_2}
$A$一定可以正交对角化。即欧式空间必定存在一组标准正交基,使得$A$在这组基下是对角矩阵。
\end{theorem}
对于基域不为$2$的实线性空间,其上的二次型总可以写为对称矩阵的形式。因此\autoref{the_ortho_2} 相当于说:

\textbf{任一二次型都可以经正交合同变换变为标准二次型。}

上述常见定理的证明见\autoref{the_lialg_1}。