% 可定向曲面
% keys Jacobi矩阵|过渡矩阵|定向|曲面|orientation|surface

\pentry{三维空间中的曲面\upref{RSurf}}

对于曲面 $S$ 上一点 $p$,其附近可能存在两个不同的局部坐标系 $\bvec{x}:U_x\to V_x$ 和 $\bvec{y}:U_y\to V_y$,其中 $p\in V_x\cap V_y$。因此,这两个局部坐标系的交集非空。如果记 $W=\bvec{x}^{-1}(U_x)\cap\bvec{y}^{-1}(U_y)$,$\bvec{x}$ 和 $\bvec{y}$ 都是 $W\to V_x\cap V_y$ 的局部坐标系。

由于局部坐标系是同胚,我们由此得到了两个 $W$ 到自身的自同胚,$\bvec{x}\circ\bvec{y}$ 和 $\bvec{y}\circ\bvec{x}$。这两个自同胚都是二维欧几里得空间之间的映射,因此可以计算其Jacobi矩阵。回忆Jacobi矩阵的几何意义,我们发现它可以用来描述区域的方向——就是说,当Jacobi矩阵为正的时候,映射不会“翻转”被映射的区域,但是Jacobi矩阵为负的时候,区域则被映射“翻转”了。

由此我们可以严格讨论什么是可定向曲面了。

\begin{definition}{可定向曲面}
给定一个正则曲面 $S$,如果我们可以用一族局部坐标系 $\{\bvec{x}_i\}$ 完全覆盖 $S$\footnote{即对于任意 $p\in S$,总存在一个 $\bvec{x}_p$ 包含 $p$。},且任意两个局部坐标系之间,如果交集非空,则Jacobi矩阵必恒正的,那么我们说这是一个\textbf{可定向曲面(oriented surface)}。如果不存在这样的一族局部坐标系,那么我们说这个曲面是\textbf{不可定向的(nonorientable)}。
\end{definition}

莫比乌斯带就是一个常见的不可定向曲面。

判断一个曲面是否可定向,可以使用以下定理:

\begin{theorem}{曲面可定向的充要条件}
一个正则曲面 $S\subseteq \mathbb{R}^3$ 可定向,当且仅当,其上存在一个\textbf{可微的单位法向量场}。
\end{theorem}

这个条件包含三个部分,“可微”的,“单位”以及“法向量”。“法向量”意味着这个向量场里的每一个向量都垂直于曲面,“单位”意味着这些法向量的长度都是 $1$,而“可微”意味着,任取曲面的局部坐标系,这个法向量场可以看成坐标系里的一个向量值函数,而这个向量值函数是连续的。

这一充要条件还引出了表示曲面定向的方法:

\begin{definition}{曲面的定向}
可定向曲面 $S$ 上有且仅有两个不同的可微单位法向量场,都被称为 $S$ 的\textbf{定向(orientation)}。
\end{definition}

最常见的一种可定向曲面,是用函数定义的。

\begin{theorem}{}
如果一个正则曲面 $S$ 是由\textbf{连续可微}函数 $f:\mathbb{R}^3\to \mathbb{R}$ 定义的,即 $S=\{\bvec{r}\in\mathbb{R}|f(\bvec{r})=0\}$,且 $\bvec{r}$ 是 $f$ 的一个\textbf{正则点},那么 $S$ 必是可定向的。
\end{theorem}






