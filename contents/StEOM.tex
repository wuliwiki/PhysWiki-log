% 弦的运动方程
% license Usr
% type Tutor


\subsection{运动方程}
对作用量变分可以得到\textbf{能量动量张量}:
\begin{equation}
	T^{ab} = - 4\pi (- \gamma)^{-1/2} \frac{\delta}{\delta \gamma_{ab}} S_P = -\frac{1}{\alpha'} (h^{ab} - \frac{1}{2} \gamma^{ab} \partial_c X^\mu \partial^c X_\mu) ~.
\end{equation}

更直观的,可以写出 
\begin{equation}
	T_{ab} = \left(\begin{matrix}
		\frac{\dot X^2 + X'~^2}{2} & \dot X \cdot X' \\
		\dot X \cdot X' & \frac{\dot X^2 + X'~^2}{2}
	\end{matrix}\right) ~.
\end{equation}

能动张量是守恒的,即 $\nabla_a T^{ab} = 0$,这是 diff 不变性的结果。而 Weyl 不变性给出:
\begin{equation}
	\gamma_{ab} \frac{\delta}{\delta \gamma_{ab}} S_P = 0 \Rightarrow T_a^a = 0 ~.
\end{equation}
也就是说,$T^{ab} \gamma_{ab} = 0$。

对 $\gamma_{ab}$ 变分并利用最小作用量原理 $\delta S = 0$ 给出 $T_{ab} = 0$。此时也就是要求 $(\dot X \pm X')^2 = 0$,是一个非线性的约束,称为\textbf{ Virasoro 约束}。

此外,对 $X^\mu$ 变分给出的运动方程是 $\partial_a [\sqrt{-\gamma} \gamma^{ab} \partial_b X^\mu] = \det{-\gamma} \nabla^2 X^\mu = 0$。 也就是说自动的给出二阶线性波动方程:$\partial^2 X^\mu = 0\partial_\tau^2 X^\mu + \partial_\sigma^2 X^\mu = 0$。

现在考虑边界条件,这是另外的约束。为此分为开弦约束与闭弦约束。

\subsection{开弦约束}
考虑的是开弦时,也就是弦不闭合时,若规定 $\sigma \in [0, l]$,则开弦应当满足狄利克雷边界条件(对应开弦端点固定不动)或纽曼边界条件(对应开弦端点在约束下移动\footnote{具体的,可以考虑弦的两端分别连接到对应的无质量套筒上,这两个套筒都分别可以在两根互相平行的无限长直线上无摩擦地滑动。})。其中狄利克雷边界条件是 
\begin{equation}
	\left. \delta X^\mu \right|_{\sigma = 0, l} = 0~,
\end{equation}
而纽曼边界条件是 
\begin{equation}
	\left. \partial_\sigma X^\mu \right|_{\sigma = 0, l} = 0 ~.
\end{equation}

狄利克雷边界条件意味着开弦的端点看起来可以固定在一个“膜”上,这个膜称为\textbf{ D-膜}。

\subsection{闭弦约束}
闭弦是周期性的,闭合。不妨令 $\sigma \in [0, 2 \pi]$,而周期性要求:
\begin{equation}
	X^\mu(\tau, \sigma + 2\pi) = X^\mu(\tau, \sigma) ~.
\end{equation}
或者说,要求 $\sigma \in [0, l]$ 时,呈现周期性:
\begin{equation}
	X^\mu(\tau, 0) = X^\mu(\tau, l), ~ \partial^\sigma X^\mu(\tau, 0) = \partial^\sigma X^\mu(\tau, l), ~ \gamma_{ab}(\tau, 0) = \gamma_{ab}(\tau, l) ~.
\end{equation}