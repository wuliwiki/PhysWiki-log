% 动能、动能定理(单个质点)
% keys 动能|动能定理|功|功率|牛顿第二定律
% license Xiao
% type Tutor

\pentry{功、功率\nref{nod_Fwork},牛顿第二定律\nref{nod_New3}}{nod_29a4}
\subsection{质点的动能}
令质点的质量为 $m$,\enref{速度}{VnA}为 $\bvec v$,则质点的\textbf{动能}定义为
\begin{equation}
E_k = \frac12 m\bvec v^2 = \frac12 mv^2~.
\end{equation}
注意这里的 $\bvec v^2 = \bvec v \vdot \bvec v$ 表示速度矢量和自身的\enref{内积}{InerPd},结果等于其模长的平方。

\subsection{质点动能定理}
\begin{theorem}{质点动能定理}
\textbf{一段时间内质点动能的变化等于合外力对质点做的功}。
\begin{equation}
\sum W=\Delta E_k~.
\end{equation}

从变化率(即时间导数)的角度来看,动能定理也可以表述为\textbf{质点的动能变化率等于合外力对质点的功率}。
\begin{equation}
\sum P=\dv{E_k}{t}~.
\end{equation}
\end{theorem}

\subsubsection{动能定理的推导}
力对质点做功的\enref{功率}{Fwork}为
\begin{equation}\label{eq_KELaw1_2}
P = \dv{W}{t} =  \bvec F\vdot \dv{\bvec r}{t} = \bvec F\vdot\bvec v~,
\end{equation}
再来看动能的变化率
\begin{equation}
\dv{t} E_k = \frac12 m\dv{t} (\bvec v\vdot\bvec v)~.
\end{equation}
由“ 矢量内积的求导\upref{DerV}” \autoref{eq_DerV_5},$\dv*{(\bvec v\vdot\bvec v)}{t} = 2\bvec v\vdot \dv*{\bvec v}{t} = 2\bvec v\vdot\bvec a$,上式变为
\begin{equation}\label{eq_KELaw1_4}
\dv{t} E_k = m\bvec a\vdot\bvec v = \bvec F\vdot\bvec v~,
\end{equation}
最后一步使用了牛顿第二定律(\autoref{eq_New3_1})。 注意\autoref{eq_KELaw1_2} 与\autoref{eq_KELaw1_4} 相等,所以动能变化率等于合外力的功率。
