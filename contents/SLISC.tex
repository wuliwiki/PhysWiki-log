% SLISC 库简介
% keys C++|矩阵
% license Xiao
% type Tutor

\pentry{C++ 基础\nref{nod_Cpp0}}{nod_d183}

相对于 \enref{Fortran}{Fortra} 或者 \enref{Matlab}{Matlab} 等, 用 C++ 做数值计算的一个缺陷就是语言本身没有矩阵以及高维矩阵类。 但我们可以用第三方库或者自己写一个。 我们选择后者, 原因是为了教学需要我们需要保持代码的简单易读, 避免复杂的 C++ 语法。

本部分中我们将大量使用自编的 \textbf{SLISC(Scientific Library in Simple C++)} 矩阵库。 代码可以从 \href{https://github.com/MacroUniverse/SLISC}{GitHub 仓库}下载, 另外提供英文说明。 该库的特点是尽量不使用 C++ 的复杂语法(如模板)和复杂的类结构, 使代码便于阅读学习和修改, 同时又保持相对较高的性能。 SLISC 使用兼容性较高的 C++11 标准。

SLISC 库在矩阵类的基础上还实现了一些科学计算常用的功能, 如线性代数运算、 插值、 微分方程、 \enref{计时}{SliTim}、 \enref{文件和目录操作}{Sfile}、 \enref{矩阵文件读写}{matb}、 字符串处理、 特殊函数、 任意精度计算、 量子力学计算等。 当然, 一些功能基于其他项目如 \enref{Blas}{BLAS}、 \enref{Lapack}{Lapack}、 \enref{Intel MKL}{OneAPI}、 \enref{Arb}{ArbLib}、 \enref{GSL}{GSL}、 \enref{Eigen}{Eigen} 和 \enref{Arpack}{Arpkpp}。 在编译时可以通过选项来决定是否使用这些依赖。

一个例子(编译方法见 “\enref{SLISC 的编译和测试}{SLScom}”):
\begin{lstlisting}[language=cpp, caption=intro.cpp]
#include "SLISC/arithmetic.h"
#include "SLISC/disp.h"
#include "SLISC/cut.h"
#include "SLISC/matb.h"
int main()
{
	using namespace slisc;

	// === 矢量和矩阵 ===
	VecDoub u(3), v(3); // 三个元素的双精度矢量
	linspace(u, 0, 2); // u = [0,1,2]
	cout << "u = \n"; disp(u); // 显示矩阵
	copy(v, 3.14); // 所有元素 = 3.14
	copy(u, v); // 复制
	u += v; // 矢量 + 矢量
	v += 2; // 矢量 + 标量
	CmatDoub a, b(2, 3); // 双精度矩阵(行主序)
	a.resize(2, 3); // 改变大小(重新分配内存)
	a(0, 0) = 1.1; // 双指标
	a[3] = 9.9; // 单指标
	a.end() = 5.5; a.end(2) = 4.4; // 最后和倒数第二个矩阵元
	cout << "a 有 " << a.n0() << " 行和 " << a.n1()
	<< " 列, 共计 " << a.size() << " 个元素。" << endl;
	disp(a); // display

	// === 矩阵切割 ===
	SvecDoub bs = cut0(b, 0); // 切割矩阵的第一列
	bs[0] += 1; // 切割类型的使用方法和矩阵类似
	Doub s2 = sum(bs); // 求和
	DvecDoub bd = cut1(b, 1); // 切割第二行
	copy(cut1(b, 0), cut1(b, 1)); // 复制第二行到第一行

	// === 文件读写 ===
	save_matb(b, "b", "data0.matb"); // 把矩阵存为二进制文件
	// 新建二进制文件储存多个变量
	Matb matb("data.matb", "w");
	save(u, "u", matb); save(v, "v", matb);
	save(a, "a", matb); save(b, "b", matb);
	matb.close();
}
\end{lstlisting}

\subsection{主要特点}
以下列出 SLISC 的一些特性, 我们以后会详细介绍。
\begin{itemize}
\item 全部定义使用 \verb`namespace slisc`, 所有宏以 \verb`SLS_` 开头。
\item 尽量不使用模板(尤其是高级的模板技巧), 用 Matlab/Octave 生成代码。 初学用户不需要了解代码如何生成, 可以直接阅读或使用生成后的代码。
\item 实现了\enref{密矩阵}{MatSto}和一些\enref{稀疏矩阵}{SprMat}。 以对列主序的支持为主。
\item 实现了密矩阵的切割,并用于函数接口。 这会有少量的额外运算, 但使用起来却比指针方便得多。 用户可以自行选择使用哪种接口。
\item Debug 模式实现了详尽的尺寸和指标越界检查, 可以及时发现指标超出长度等错误。
\item 函数尽量首先使用和 Lapack 类似的指针接口实现, 然后再封装一层更友好的接口。
\item 为了保证性能, 函数内部尽量不改变矩阵尺寸, 这是因为内存的动态分配往往耗时较多。
\end{itemize}

在 SLISC 中, 我们希望把矩阵进行一定程度的封装, 但又几乎不损失性能。 许多人使用流行的矩阵库例如 Eigen, 但是其代码复杂, 错误信息不容易懂, 代码修改起来非常困难。 例如 Eigen 的 \verb`MatrixXd` 矩阵类有 6 次继承, 一个 2x2 的矩阵在 gdb 里面调试的时候显示出来的变量信息是这样的:
\begin{lstlisting}[language=cpp]
<Eigen::PlainObjectBase<Eigen::Matrix<double,-1,-1,0,-1,-1>>> = 
{<Eigen::MatrixBase<Eigen::Matrix<double,-1,-1,0,-1,-1>>> = 
{<Eigen::DenseBase<Eigen::Matrix<double,-1,-1,0,-1,-1>>> = 
{<Eigen::DenseCoeffsBase<Eigen::Matrix<double,-1,-1,0,-1,-1>, 3>> = 
{<Eigen::DenseCoeffsBase<Eigen::Matrix<double,-1,-1,0,-1,-1>, 1>> = 
{<Eigen::DenseCoeffsBase<Eigen::Matrix<double,-1,-1,0,-1,-1>, 0>> = 
{<Eigen::EigenBase<Eigen::Matrix<double,-1,-1,0,-1,-1>>> =
{<No data fields>}, <No data fields>}, 
<No data fields>}, <No data fields>}, <No data fields>}, <No data fields>},
m_storage = {m_data = 0x855ceb0, m_rows = 2, m_cols = 2}}, <No data fields>
\end{lstlisting}
实际上这里面的重点只有最后一行, 显示了矩阵在内存中的地址 \verb`m_data`, 行数 \verb`m_rows` 以及列数 \verb`m_cols`。 这样的库只适合直接拿来用, 不适合阅读、学习和修改, 尤其是对非计算机专业的同学来说。

\subsubsection{重新定义类型名称}
在 C++ 标准中, 一些基本类型在内存中的长度在不同计算机上可能会不同。 例如 \verb`long` 有时候和 \verb`int` 一样是 4 字节, 而另一些时候则和 \verb`long long` 一样是 8 字节。 因此, SLISC 仿照 Numerical Recipe \cite{NR3} 的方式, 定义自己的类型名称。
\begin{itemize}
\item \verb`Char` 是 \verb`char`, 即 1 字节字符或整数
\item \verb`Uchar` 是 \verb`unsigned char`, 即 1 字节无符号整数
\item \verb`Short` 是 2 字节整数
\item \verb`Int` 是 4 字节整数
\item \verb`Long` 是 \verb`Int` 或者 \verb`Llong`, 取决于宏 \verb`SLS_USE_INT_AS_LONG` 是否定义。 SLISC 库中, \verb`Long` 被用来作为矩阵的索引类型。
\item \verb`Llong` 是 8 字节整数
\item \verb`Float` 是 \verb`float`(4 字节浮点数)
\item \verb`Doub` 是 \verb`double`(8 字节浮点数)
\item \verb`Ldoub` 是 \verb`long double` (16 字节浮点数, 但并不一定是四精度, 取决于机器和编译器)
\item \verb`Qdoub` 是 16 字节浮点数, 标准的\enref{四精度}{FltCpp}(需要在编译时打开 \verb`quadmath` 选项)。
\item \verb`Fcomp` 是 \verb`std::complex<float>` (8 字节复数)
\item \verb`Comp` 是 \verb`std::complex<double>` (16 字节复数)
\item \verb`Lcomp` 是 \verb`std::complex<long double>` (32 字节复数)
\item \verb`Qcomp` 是 \verb`std::complex<Qdoub>` 32 字节复数, 标准的四精度。
\end{itemize}
另外我们还定义一些附加类型, 用于声明函数的参数。 例如 \verb`Doub_I` 是 \verb`const Doub`, 表示函数的输入参数。 又如 \verb`Doub_O` 是 \verb`Doub &`, 表示函数的输出参数(原来的值不会被使用)。 \verb`Doub_IO` 也是 \verb`Doub &`, 但表示既作为输入也作为输出(原来的值会被使用且被覆盖)。 以上的标量类型都同理, 在后面加 \verb`_I`, \verb`_O`, \verb`_IO` 分别表示 in, out, in-out。

以后还会看到矢量类和矩阵类, 但和标量不同, 为了避免不必要的复制, 它们都必须 pass by reference 而不是 pass by value。 例如矩阵类 \verb`CmatDoub` 的 \verb`CmatDoub_I` 是 \verb`const CmatDoub &`, \verb`CmatDoub_O` 和 \verb`CmatDoub_IO` 都是 \verb`CmatDoub &`。
