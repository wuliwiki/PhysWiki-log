% 帕夫努季·切比雪夫(综述)
% license CCBYSA3
% type Wiki

本文根据 CC-BY-SA 协议转载翻译自维基百科\href{https://en.wikipedia.org/wiki/Pafnuty_Chebyshev}{相关文章}。

\begin{figure}[ht]
\centering
\includegraphics[width=6cm]{./figures/97a93cba399441cc.png}
\caption{帕夫努季·李沃维奇·切比雪夫} \label{fig_PFNJ_1}
\end{figure}
帕夫努季·利沃维奇·切比雪夫(俄语:Пафну́тий Льво́вич Чебышёв,发音:[pɐfˈnutʲɪj ˈlʲvovʲɪtɕ tɕɪbɨˈʂof],1821年5月16日[俄历5月4日]—1894年12月8日[俄历11月26日])\(^\text{[3]}\)是一位俄罗斯数学家,被认为是俄罗斯数学的奠基人。

切比雪夫以其在概率论、统计学、力学以及数论领域的基础性贡献而著称。许多重要的数学概念以他的名字命名,包括切比雪夫不等式(可用于证明大数定律的弱形式)、伯特兰-切比雪夫定理、切比雪夫多项式、切比雪夫连杆机构以及切比雪夫偏差。
\subsection{音译}
切比雪夫这个姓氏在翻译时出现了多种不同的拼写方式,如:Tchebichef、Tchebychev、Tchebycheff、Tschebyschev、Tschebyschef、Tschebyscheff、Čebyčev、Čebyšev、Chebysheff、Chebychov、Chebyshov(据以俄语为母语的人说,这种拼写在英语中最接近旧俄语中的正确发音),以及 Chebychev,这是一种混合了英语和法语音译的方式,通常被认为是错误的。

在数学文献中,这种拼写混乱被认为是最著名的数据检索噩梦之一。目前,“Chebyshev” 这一英语拼写已被广泛接受,唯独法语国家仍偏好使用 “Tchebychev”。根据 ISO 9 国际音译标准,该姓氏的标准转写形式是 “Čebyšëv”。美国数学会在其《数学评论》中采用了 “Chebyshev” 这一拼写方式\(^\text{[4]}\)。

他的名字 Pafnuty 源自希腊语 Paphnutius(Παφνούτιος),而后者又来自科普特语 Paphnuty(Ⲡⲁⲫⲛⲟⲩϯ),意为“属于上帝的人”或简而言之“上帝之人”。
\subsection{传记}
\subsubsection{早年经历}
切比雪夫是家中九个孩子之一\(^\text{[5]}\),出生于卡卢加省博罗夫斯克地区的奥卡托沃村。他的父亲列夫·巴夫洛维奇是俄罗斯贵族和富有的地主。帕夫努季·列沃维奇最初由母亲阿格拉菲娜·伊万诺芙娜·波兹尼亚科娃在家中教授识字与写作,由堂姐阿夫多佳·昆提利安诺夫娜·苏哈列娃教授法语和算术。切比雪夫曾提到,他的音乐老师对他的教育也起了重要作用,因为她“将他的思维引导至精确与分析之道”。

切比雪夫在青少年时期受到特伦德伦堡步态的影响。从小他就跛行,需要依靠手杖行走,因此父母放弃了让他按照家族传统成为军官的念头。由于身体残疾,他无法参与许多儿童游戏,便将精力转向数学。

1832年,全家搬到了莫斯科,主要是为了让长子帕夫努季和帕维尔(后者成为律师)接受更好的教育。家庭教育继续进行,父母聘请了多位声誉卓著的教师,其中教授数学与物理的是莫斯科大学的高级教师普拉通·波戈列尔斯基【俄语维基】,他曾教授过未来作家伊万·图尔根涅夫等人。
\subsubsection{大学学习时期}
1837年夏天,切比雪夫通过了入学考试,并于同年9月开始在莫斯科大学哲学系第二分部攻读数学【来源请求】。他的老师包括N.D.布拉什曼、N.E.泽尔诺夫和D.M.佩列沃希奇科夫,其中对切比雪夫影响最大的是布拉什曼。布拉什曼教授他实用力学,并很可能向他介绍了法国工程师J.V.蓬斯莱的著作。1841年,切比雪夫因其于1838年完成的论文《方程根的计算》获得银奖。在这篇论文中,切比雪夫基于牛顿法推导出一种求解n次代数方程根的近似算法。同年,他以“最优秀候选人”身份完成学业。

1841年,切比雪夫的经济状况发生了重大变化。由于俄罗斯发生饥荒,他的父母被迫离开莫斯科。尽管家中已无力支持他继续学业,他仍决定坚持数学研究,并准备为期六个月的硕士考试。切比雪夫于1843年10月通过最终考试,1846年完成并答辩了他的硕士论文《概率论初等分析论文》。他的传记作者普鲁德尼科夫推测,切比雪夫是在了解到近期出版的一些关于概率论或俄罗斯保险业收益的书籍后,选择了这一研究方向。
\subsubsection{成年时期}
1847年,切比雪夫在圣彼得堡大学提交了题为《借助对数的积分法》的教学资格论文,由此获得在该校授课的权利。当时,莱昂哈德·欧拉的一些著作被P.N.富斯重新发现,并由维克托·布尼亚科夫斯基编辑出版,后者鼓励切比雪夫研究这些作品,从而对他的研究产生了深远影响。1848年,他提交了博士论文《同余理论》,并于1849年5月通过答辩\(^\text{[1]}\)。1850年他被选为圣彼得堡大学特聘教授,1860年晋升为正教授,1872年在任教25年后被授予功勋教授称号。1882年他离开大学,全心投入科研工作。

在大学任教期间(1852–1858年),切比雪夫还在位于圣彼得堡南郊的皇村(今普希金市)的亚历山大高级中学教授实用力学。

他在科研方面的成就使他于1856年被选为科学院青年院士。随后,他于1856年成为圣彼得堡科学院特聘院士,并于1858年成为正式院士。同年,他还被任命为莫斯科大学名誉院士。此后他接受了多项荣誉头衔,并多次获得勋章。1856年,切比雪夫成为国民教育部科学委员会成员。1859年,他被任命为科学院兵器部的正式成员,并领导与兵器和弹道学实验相关的数学问题委员会。1860年,法国科学院选他为通讯院士,1874年升为外籍正式院士。1878年,切比雪夫在法国科学促进会提交了一篇关于服装剪裁的论文,该灵感源于爱德华·吕卡斯(的一次讲座\(^\text{[6]}\)。

1893年,切比雪夫被选为新成立的圣彼得堡数学学会名誉会员。

1894年12月8日,切比雪夫在圣彼得堡去世\(^\text{[1][2]}\)。
\subsection{数学贡献}
\begin{figure}[ht]
\centering
\includegraphics[width=6cm]{./figures/3e3ba3c69826c40d.png}
\caption{帕夫努季·切比雪夫} \label{fig_PFNJ_2}
\end{figure}
切比雪夫以其在概率论、统计学、力学和数论方面的工作而著称。切比雪夫不等式指出:若 $X$ 是一个具有标准差 $\sigma > 0$ 的随机变量,那么 $X$ 的取值距离其数学期望至少 $d = k\sigma$ 的概率最多为:
$$
\Pr(|X - \mathbf{E}(X)| \geq d) \leq \frac{\sigma^2}{d^2} = \frac{1}{k^2}~
$$
切比雪夫不等式常被用于证明大数定律的弱形式。

伯特兰–切比雪夫定理(1845年、1852年)指出,对于任意 $n > 3$,都存在一个素数 $p$ 满足:$n < p < 2n$该结论可以由关于小于 $n$ 的素数个数 $\pi(n)$ 的切比雪夫不等式推出:

当 $x$ 足够大时,有:
$$
A \cdot \frac{x}{\log(x)} < \pi(x) < \frac{6A}{5} \cdot \frac{x}{\log(x)},\ \text{其中}A \approx 0.92129^\text{[7]}~
$$
五十年后,即1896年,素数定理被雅克·阿达玛\(^\text{[8]}\)与夏尔·让·德拉瓦莱-普桑\(^\text{[9]}\)分别独立地证明:
$$
\lim_{x \to \infty} \frac{\pi(x) \log x}{x} = 1~
$$
他们使用了贝恩哈德·黎曼提出的一些思想。切比雪夫还以切比雪夫多项式和切比雪夫偏差而著称。后者是指模4余3的素数数量与模4余1的素数数量之间的差异\(^\text{[10]}\)。

切比雪夫还是第一位系统性地以随机变量及其矩与期望为研究对象的数学家\(^\text{[11]}\)。
\subsection{遗产}
\begin{figure}[ht]
\centering
\includegraphics[width=6cm]{./figures/76a42cb3c1b324ee.png}
\caption{切比雪夫出现在2021年俄罗斯发行的邮票上。} \label{fig_PFNJ_3}
\end{figure}
切比雪夫被视为俄罗斯数学的奠基人之一。\(^\text{[1]}\)他著名的学生包括数学家德米特里·格拉韦、亚历山大·科尔金、亚历山大·柳apunov 和安德烈·马尔可夫。根据“数学世系计划”,截至2025年1月,切比雪夫拥有 17,533 位数学“后代”。\(^\text{[12]}\)

为了表彰他在数学领域的杰出成就,月球上的“切比雪夫陨石坑”以及小行星“2010 Chebyshev”均以他的名字命名。\(^\text{[13]}\)
\subsection{出版物}
\begin{itemize}
\item Tchebychef, P. L.(1899),Markov, Andrey Andreevich 和 Sonin, N.(编辑),《全集》第I卷,纽约:俄国皇家科学院出版委员会,MR 0147353,1962年由Chelsea重印。
\item Tchebychef, P. L.(1907),Markov, Andrey Andreevich 和 Sonin, N.(编辑),《全集》第II卷,纽约:俄国皇家科学院出版委员会,MR 0147353,1962年由Chelsea重印。
\item Butzer, Paul;Jongmans, Francois(1999),《P. L. Chebyshev(1821–1894):生平与工作的导引》,《近似理论杂志》,96: 111–138,doi:10.1006/jath.1998.3289
\end{itemize}
\subsection{另见}
\begin{itemize}
\item 以帕夫努季·切比雪夫命名的事物列表
\end{itemize}
\subsection{参考文献}
\begin{enumerate}
\item Pafnuty Chebyshev,《大英百科全书》
\item "Pafnuty Lvovich Chebyshev",MacTutor,检索于2024年11月22日。
\item Pafnuty Lvovich Chebyshev – 《大英百科全书》在线版
\item Chebyshev, Pafnutiĭ L'vovich,于 MathSciNet 上
\item MacTutor 档案馆中的传记
\item Tapia, Victor(2025),“Chebyshev 与服装剪裁:揭穿一些神话”,《数学通识》:1–5,doi:10.1007/s00283-025-10409-x。
\item Tchebichef(1852),“关于素数的论文”,《纯与应用数学杂志》(Journal de Mathématiques Pures et Appliquées,法文):366–390,ISSN 1776-3371,2024年11月26日检索。英文翻译:Mike Bertrand(2020年11月5日)。
\item Hadamard, Jacques(1896),“关于 ζ(s) 函数零点的分布及其算术后果”,《法国数学会会报》,第24卷,法国数学会:199–220,原文于2024年9月10日存档。
\item Charles-Jean de la Vallée Poussin(1896),“关于素数理论的分析研究”,《布鲁塞尔科学协会年鉴》,第20B、21B卷,比利时皇家科学院印刷部:183–256, 281–352, 363–397, 351–368。
\item Rubinstein, Michael;Sarnak, Peter(1994),“切比雪夫偏差”,《实验数学》,第3卷第3期:173–197,doi:10.1080/10586458.1994.10504289。
\item Mackey, George(1980年7月),“谐波分析作为对对称性的利用——历史回顾”,《美国数学会通报》(新系列),第3卷第1期:549,doi:10.1090/S0273-0979-1980-14783-7,hdl:1911/63317。
\item “切比雪夫”在“数学世系项目”(Mathematics Genealogy Project)中
\item Schmadel, Lutz D.(2007),“(2010) Chebyshev”,《小行星命名词典》,Springer Berlin Heidelberg,第163页,doi:10.1007/978-3-540-29925-7\_2011,ISBN 978-3-540-00238-3。
\end{enumerate}
\subsection{延伸阅读}
\begin{itemize}
\item Papadopoulos, Athanase(2019年5月14日),“帕夫努季·切比雪夫 (1821–1894)”,发表于《Bhāvanā – 数学杂志》,Bhavana 信托基金会,2024年11月25日检索。
\end{itemize}
\subsection{外部链接}
\begin{itemize}
\item 切比雪夫的机械装置 – 简短的 3D 动画,展现切比雪夫发明的实物模型
\item 切比雪夫在数学世系项目上的资料
\item O'Connor, John J. 与 Robertson, Edmund F. 撰写的“Pafnuty Chebyshev”传记,载于圣安德鲁斯大学的 MacTutor 数学史档案
\item 一篇传记、另一篇传记、还有一篇传记(以上均为俄文)
\item 一篇法文传记
\item 切比雪夫文集《Œuvres de P.L. Tchebychef》第一卷与第二卷(法文)
\item 《当数学家用几何来裁布》– Étienne Ghys 的演讲,介绍切比雪夫在微分几何中关于“切比雪夫网”的研究,重点讲述其裁剪理论的几何化成果
\end{itemize}
