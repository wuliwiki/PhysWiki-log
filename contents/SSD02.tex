% 首都师范大学 2002 年硕士入学考试试题
% keys 首都师范大学|考研|物理|2002
% license Copy
% type Tutor

\textbf{声明}:“该内容来源于网络公开资料,不保证真实性,如有侵权请联系管理员”
\begin{enumerate}
\item 如图,质量分别为$m_1$和$m_2$,的两木块用弹性系数为$k$的弹相,静止地放在光滑、水平地面上。质量为$m$的子弹以水平初速$V$射入木块$m_1$,并停留在$m_1$;内。设子弹射入过程的时间极短。\\
试求:(1)弹簧的最大压缩长度;\\
(2)木块$m_2$相对地面的最大速度和最小速度。
\begin{figure}[ht]
\centering
\includegraphics[width=12cm]{./figures/6d95d7f02c129416.png}
\caption{} \label{fig_SSD02_1}
\end{figure}
\item  已知一质点做简谐振动,其圆频率为$\omega$,位移为X,速度为 V。\\
(1)证明其在相图中的轨迹为椭圆(X-V 图)或者为圆($\displaystyle X-\frac{v}{\omega}$ 图);\\
(2)在$\displaystyle X-\frac{v}{\omega}$ 图上试画出阻尼振动的图线。
\item 一密绕螺线管,单位长度上的匝数为n,螺线管长度为L,横截面半径为 a,且L>>a,现给螺线管通上随时间增加的电流,其电流变化率为di/dt,求:\\
(1)螺线管内距轴线为r(0<r<a)处一点的感生电场E。、\\
(2)这一点上的玻印亭欠量S。作图并标出各矢量的方向。
\item 平行板电容器极板面积为S,板间距离为d。问:\\
(1)将电容接在电源上,插入厚度各为d/2的两均匀电介质板,板面积均为S,介电常数各为$\varepsilon_1,\varepsilon_2$,两介质中电场之比为多少?它们与未插入介质之前电场之比为多少?\\
(2)电容器充电后拆去电源,再插入上述介质板,两介质中电场之比为多少?它们与未插入介质之前电场之比为多少?
\item 已知氢原子光谱中,三条谱线的波长分别为$\lambda_1=1215.68A,\lambda_2=1025.73A,\lambda_3=972.54A$,试求:\\
(1)氢原子的里德伯常数(取五位有效数字);\\
(2)这三条谱线属于哪个线系?\\
(3)与这三条谱线有关的光谱项值;\\
(4)由以上数据还可以求出哪些谱线的波长?各属于哪个线系?\\
(5)画出相应的能级跃迁图。
\item 试证明核反应的反应能等于反应前与反应后体系静止能量之差,并说明此关系所反映的物理实质是什么?
\end{enumerate}