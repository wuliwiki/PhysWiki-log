% 欧拉函数(数论)
% keys 数论|欧拉函数
% license Usr
% type Tutor

\pentry{数论函数\nref{nod_NumFun},积性函数\nref{nod_MulFun}}{nod_24cb}

前面在数论函数中已经提到过欧拉函数,但欧拉函数相关内容很丰富,下面做一些展开。
首先回顾欧拉函数(Euler's Function/Euler's Totient Function)的定义:
\begin{definition}{欧拉函数}
欧拉函数 $\varphi(n)$ 表示 $n$ 以内的非零自然数中与 $n$ 互质的数的个数。也可以表示为,
$$\varphi(n) = \sum_{1 \le d \le n, \gcd(d, n) = 1} 1  ~.$$
其中 $\gcd$ 为求最大公约数,故可以用 $\gcd(d, n)=1$ 表示要求 $d$ 与 $n$ 互质。
\end{definition}

考察一个数论函数,我们经常按素数——素数幂次——非零自然数的顺序去考察。先探讨前两个的性质。

显然对于任意素数 $p$,其欧拉函数为 $\varphi(p) = p-1$。

对于一个素数幂次 $p^k$,可以想象在 $[1, p^k]$ 内每 $p$ 个数就有一个与 $p^k$ 不互素(有公因子 $p$)。也就是除去所有的 $p$ 的倍数,均与 $p^k$ 互质,故素数幂次的欧拉函数可以表示为 $\varphi(p^k) = p^k \left(1 - (1/p)\right)$。

\begin{corollary}{任意非零自然数 $n$ 的欧拉函数}\label{cor_EulFun_1}
\begin{equation}
\varphi(n) = n \cdot \prod_i \left( 1 - \frac1{p_i}\right)~.
\end{equation}
\end{corollary}
\textbf{证明}:可以利用容斥原理:
不妨设 $n = \prod {p_i^{k_i}}$,$p_i$ 是 $n$ 的各素因子,$k_i$ 是其幂次。则有:
\begin{enumerate}
\item 不超过 $n$ 的是某个 $p_i$ 的倍数的分别有 $n/p_i$ 个。
\item 不超过 $n$ 的是某两 $p_i$ 的倍数的,也即是 $p_i p_j$ 的倍数($i \neq j$)的,有 $n/(p_i p_j)$ 个。在上面重复计数,需要减去。
\item 不超过 $n$ 的是某三 $p_i$ 的倍数的,也即是 $p_i p_j p_k$ 的倍数($i, j, k$ 互不相等),有 $n/(p_i p_j p_k)$ 个。在上面被减去了,没有计数到,需要加上。
\item 依此类推...
\item 不超过 $n$ 的是所有 $p_i$ 的倍数的,即 $\prod p_i$ 的,有 $n/(\prod p_i)$ 个。根据 $i$ 的奇偶性决定是要加上或减去。
\end{enumerate}
我们发现这表达式可以将欧拉函数表示为:
$$\varphi(n) = n - \sum(n/p_i) + \sum_{i\le j} (n/(p_i p_j)) \cdots  ~, $$
恰好与下面的展开式对应:
$$\varphi(n) = n \cdot \prod_i \left( 1 - \frac1{p_i}\right)~.$$
这就是计算某任意数的欧拉函数的方法。

\subsection{积性函数}
\begin{theorem}{}
欧拉函数有非常良好的性质,他是一个积性函数。
\end{theorem}
\textbf{证明}:
回顾积性函数的定义,对于互质的 $n, m$,要求 $\varphi(n m ) = \varphi(n) \varphi(m)$。将 $n$、$m$ 的质因子列出,假设 $n$、$m$ 分别有质因子 $p_i$、$q_i$,则 $nm$ 的所有质因子为 $p_i, q_i$($p_i \neq q_j, \forall (i, j)$,因为 $n$ 与 $m$ 互质)。从而利用上面的\autoref{cor_EulFun_1},他们的积的欧拉函数可以表示为,
\begin{equation}
\begin{aligned}
\varphi(n m) &= nm \prod_i \left(1-\frac1{p_i}\right) \prod_i \left(1-\frac1{q_i}\right)\\
&= \left(n \prod_i \left(1 - \frac1{p_i}\right)\right) \left(m \prod_i \left(1 - \frac1{q_i}\right)\right)\\
&= \varphi(n) \varphi(m) ~.
\end{aligned}
\end{equation}
这就完成了证明。

\subsection{数论卷积性质}
\pentry{狄利克雷卷积(数论)\nref{nod_DirCon}}{nod_86ec}
\begin{theorem}{}
\begin{equation}
\varphi(n) * 1 = \sum_{d | n} \varphi(d) = \operatorname{id}(n) = n ~~
\end{equation}

\end{theorem}