% 受阻简谐振子
% keys 简谐振子|摩擦力|受阻振动

\pentry{二阶常系数非齐次微分方程\upref{Ode2N}}
\subsection{结论}
两个复数根($\alpha ^2 - 4km < 0$, 最常讨论和应用的情况)
\begin{equation}
y = C_1 \E^{rt} \cos(\omega t + C_2)~,
\end{equation}
其中
\begin{equation}
r =  - \frac{\alpha }{2m} ~,\qquad  \omega  = \sqrt{\omega_0^2 - \frac{\alpha ^2}{4 m^2}}~.
\end{equation}

其中 $\alpha$是阻力系数,$\omega_0 = \sqrt{k/m}$ 是振子的\textbf{固有振动频率}(无阻力$\alpha  = 0$ 时的振动频率)。$r$ 和 $\omega$ 还满足
\begin{equation}
r^2 + \omega ^2 = \omega_0^2~.
\end{equation}

\subsection{推导}
在弹簧振子的振动方程基础上,若振子还受到一个与速度成正比的阻力 $f =  - \alpha v$, 则振动方程如下($\alpha \ne 0$):
\begin{equation}
my'' =  - \alpha y' - ky~.
\end{equation}
之所以设为正比,是因为所得方程是线性方程,便于求解。根据“ 二阶常系数齐次微分方程\upref{Ode2N}”, 解特征方程 $m r^2 + \alpha r + k = 0$, 可得通解分为三种情况。
\begin{enumerate}
\item 有两个不同的实根 $r_1, r_2$ ($\alpha ^2 - 4km > 0$),方程的通解为
\begin{equation}
y = C_1 \E^{r_1 t} + C_2 \E^{r_2 t}~.
\end{equation}
这种情况下, 阻力系数太大以至于质点直接减速回到平衡位置而无法发生任何振动。

\item 有一个重根 $r$ ($\alpha ^2 - 4km = 0$),方程的通解为
\begin{equation}
y = C_1 \E^{rt} + C_2 t \E^{rt}~,
\end{equation}
这是质点振动与不振动的临界点。

\item 两个复数根 $r_1,r_2$ ($\alpha ^2 - 4km < 0$, 最常讨论和应用的情况)
\begin{equation}
y = C_1 \E^{rt}\cos(\omega t + C_2)~,
\end{equation}
其中
\begin{equation}
r =  - \frac{\alpha }{2m}~,
\qquad
\omega = \frac{1}{2m}\sqrt {4mk - \alpha ^2}  = \sqrt{\omega_0^2 - \frac{\alpha ^2}{4 m^2}}~.
\end{equation}
这种情况下, 质点做振幅不断衰减的振动, 衰减系数 $r$ 与阻力系数成正比。 若令(见简谐振子\upref{SHO}的振动频率)
\begin{equation}
\omega_0 = \sqrt{\frac{k}{m}}~, \qquad \gamma  = \frac{\alpha }{2\sqrt{mk}}~,
\end{equation}
则 
\begin{equation}
r =  - \omega_0 \gamma~,\qquad \omega  = \omega_0 \sqrt{1 - \gamma ^2}~.
\end{equation}
满足
$r^2 + \omega ^2 = \omega_0^2$。
\end{enumerate}


