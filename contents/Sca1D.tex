% 量子散射(一维)
% 散射|薛定谔方程|散射幅|正交归一|能量归一化|动量归一化

\begin{issues}
\issueTODO
\end{issues}

\pentry{平面波的的正交归一化\upref{EngNor}}

本文使用原子单位制\upref{AU}。 在量子力学中, 散射是一个十分重要的过程。 例如为了探索例如原子的结构, 一种重要的手段就是用其他粒子轰击原子, 并探测出射粒子在不同方向上的动量或能量分布等。 著名的卢瑟福散射就是用电子轰击原子的方法解开了原子结构之谜。 虽然卢瑟福最初使用经典力学来分析这类散射过程(当时量子力学还没有出现), 但原子尺度下量子力学比经典力学要精确得多。

在经典力学中, 散射\upref{Scater}过程考虑一个粒子从无穷远处入射, 经过一个势能 $V(x)$ 后发生偏折。 量子力学中, 这一过程可以用波包来描述: 一个波包 $\psi(x, t)$ 在初始时刻从远处入射, 我们想知道经过 $V(x)$ 散射后, $\psi(x, t)$ 会如何变化, 例如不同出射方向的概率流密度\upref{PrbJ}以及动量如何分布?

由于三维空间的量子散射所需的数学较为复杂, 我们先学习一维散射。 三维散射的许多性质都可以从一维情况类比, 但也有更丰富的特性。

我们已经学习了一维自由粒子\upref{FreeP1}如何随时间演化, 例如一维高斯波包\upref{GausWP}在自由演化过程中, 他的中心会像经典粒子一样以恒定速度移动, 但同时波包还会慢慢变宽。 在此基础上, 如果在波包前进方向上添加一个不为零的 $V(x)$ (例如一个势垒或势阱, \autoref{fig_QM0_3}~\upref{QM0}), 那么波包将会在各个方向上发生散射(一维运动只有左右两个方向)。 我们希望能计算该过程中波函数如何变化。 我们要求势能函数 $V(x)$ 只在局部不为零, 即满足 $V(\pm\infty) \to 0$。 

原则上, 我们可以直接把初始波函数和势能代入含时薛定谔方程\upref{TDSE}进行求解
\begin{equation}\label{eq_Sca1D_1}
-\frac{1}{2m}\pdv[2]{x}\psi(x, t) + V(x) \psi(x, t) = \I \pdv{t} \psi(x, t)~.
\end{equation}
即使没有解析解, 也可以通过数值方法, 依次求出每个时间步长 $t_n$ ($n = 1, 2, \dots$)的波函数 $\psi(x, t_n)$。

\subsection{解析分析}
即使能轻易得到数值解, 解析的分析也能帮我们更好地理解散射问题(或者任何问题)。 和 “一维自由粒子(量子)\upref{FreeP1}” 中的过程类似, 分离变量法可以得到\autoref{eq_Sca1D_1} 的通解(\autoref{eq_TDSE11_4}~\upref{TDSE11})。 首先看定态薛定谔方程定态薛定谔方程
\begin{equation}\label{eq_Sca1D_2}
-\frac{1}{2m}\dv[2]{x}\psi_E + V(x) \psi_E = E \psi_E ~,\qquad (E > 0)~.
\end{equation}
其中 $E$ 可以连续取值, 每个 $E$ 都对应不同的解。 当 $E< 0$ 时, 有可能存在束缚态, 但我们在散射问题中不考虑。 这是一个二阶常微分方程, 每个 $E$ 有两个线性无关的解, 记为 $\psi_{E,1}(x)$ 和 $\psi_{E,2}(x)$, 于是通解可以表示为
\begin{equation}
\Psi(x, t) = \int_0^{+\infty} [C_1(E) \psi_{E,1}(x) + C_2(E) \psi_{E,2}(x)] \E^{-\I E (t-t_0)} \dd{E}~.
\end{equation}
系数 $C_i(E)$ 由初始波函数决定, 满足
\begin{equation}
\Psi(x, t_0) = \int_0^{+\infty} [C_1(E) \psi_{E,1}(x) + C_2(E) \psi_{E,2}(x)] \dd{E}~.
\end{equation}
注意由于我们规定 $V(\pm\infty)\to 0$, 在无穷远处 $\psi_{E,i}(x)$ 显然都是平面波, 波数都是 $k = \sqrt{2mE}$。 另外由于\autoref{eq_Sca1D_2} 中所有的系数都是实数, $\psi_{E,1}, \psi_{E,2}$ 也可以都是实函数。

具体如何从 $\Psi(x, t_0)$ 得到系数 $C_i(E)$ 呢? 如果我们确保所有不同的基底 $\psi_{E,i}$ 都是正交归一\footnote{通常解二阶常微分方程只要求找到两个线性无关的解, 但正交归一的条件要更苛刻。}的(见 “傅里叶变换与连续正交归一基底\upref{COrNoB}”)
\begin{equation}
\braket{\psi_{E',i'}}{\psi_{E,i}} = \delta_{i',i}\delta(E' - E)~,
\end{equation}
那么就有
\begin{equation}
C_i(E) = \braket{\psi_{E,i}}{\Psi(t_0)}~, \qquad (i = 1,2)~,
\end{equation}
这就是\textbf{能量归一化}。 当然我们同样有不同的归一化方式(事实上有无数种), 例如类似于\autoref{eq_FreeP1_3}~\upref{FreeP1} 的\textbf{动量归一化}, 令\autoref{eq_Sca1D_2} 的本征函数为 $\psi_{k}(x)$($k \in \mathbb R$, $k^2/(2m) = E$), 满足归一化条件
\begin{equation}
\braket{\psi_{k'}}{\psi_{k}} = \delta(k' - k)~.
\end{equation}
那么波函数表示为
\begin{equation}\label{eq_Sca1D_3}
\Psi(x, t) = \int_{-\infty}^{+\infty} C(k) \psi_k(x)\E^{-\I E (t-t_0)} \dd{k}~,
\end{equation}
系数通过初始波函数的投影计算
\begin{equation}
C(k) = \braket{\psi_k}{\Psi(x, t_0)}~.
\end{equation}
归一化的具体过程要比平面波更复杂, 详见 “一维散射态的正交归一化\upref{ScaNrm}”。

一个基本的例子见 “方势垒散射数值计算\upref{FSBsct}”。
\addTODO{未完成,需要包含波包系数计算\upref{SqrPot}方势垒散射数值计算\upref{FSBsct}}

\subsection{散射幅}
\addTODO{未完成}
