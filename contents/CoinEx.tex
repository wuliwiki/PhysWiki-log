% 抛硬币实验进阶

\begin{issues}
\issueDraft
\end{issues}

\pentry{二项分布\upref{BiDist}, 中心极限定理\upref{CLT}}

我们都知道一枚公平的硬币, 若抛许多次, 那么正反的比例大约是 1:1. 那么若把某次实验的结果中, 前 $n$ 次中正面向上所占的比例记为 $P_n$, 那么对任意一次实验, 数列 $P_n$ 是否都满足以下极限呢?
\begin{equation}
\lim_{n\to\infty} P_n = \frac{1}{2} \qquad (\text{误})
\end{equation}
这是错误的. 根据数列极限的定义\upref{Lim0}, 对于任意给定的 $\epsilon > 0$, 都存在整数 $N$, 当 $n>N$ 时就必有 $\abs{P_n -1/2} < \epsilon$. 对于抛硬币而言, 这是无法做到的. 因为无论抛几次, 都存在不为零的概率使全部正面向上或反面向上(概率为 $1/2^n$). 所以只要令 $\epsilon = 0.5$ 或更小, 就不可能找到符合条件的 $N$.

当抛硬币次数较多时, 根据中心极限定理\upref{CLT}可知正面向上概率的方差为 $1/(4N)$. 该分布为二项分布, 本质上和高尔顿板\upref{Galton} 原理相同.

\begin{lstlisting}[language=matlab]
N = 60; % 每组投掷
M = 10000; % 实验组数
data = sum(rand(M, N) > 0.5, 2)/N;
format long;
disp('理论方差 =');
disp(1/(4*N));
disp('实验方差 =');
disp(std(data)^2);
hist(data, 20);
\end{lstlisting}
