% R-矩阵法(量子力学)
% license Xiao
% type Tutor

\pentry{球坐标系中的定态薛定谔方程\upref{RadSE}}

\footnote{本文参考 \cite{Bransden} 12.7 节。}本文使用原子单位制\upref{AU}。 $R$-矩阵法的中心思想是, 若要解
\begin{equation}\label{eq_Rmat_9}
H = -\frac{1}{2m}\dv[2]{x} + V(x)~.
\end{equation}
在 $[0, \infty)$ 的散射态波函数(边界条件 $\psi(0) = 0$), 把 $[0,a]$ 内的波函数用一组正交归一基底展开, 而在 $[a,\infty)$ 根据 $V(r)$ 的渐进形式写出近似的波函数, $a$ 越大, 该近似越精确。 最后在 $x=a$ 处, 匹配波函数。

在实际应用中, 我们往往是在球坐标系中解径向方程(\autoref{eq_RadSE_1}~\upref{RadSE}), 每个分波的径向哈密顿算符为
\begin{equation}
H_l = -\frac{1}{2m} \dv[2]{r} + V(r) + \frac{l(l + 1)}{2mr^2}~.
\end{equation}
这和\autoref{eq_Rmat_9} 在数学上是相同的, 所以为了简单起见下文还是使用前者。

\subsection{构建正交归一基底}
一个算符是否为厄米算符与边界条件有关。 例如
\begin{equation}
H = -\frac{1}{2m}\dv[2]{x} + V(x)~.
\end{equation}
要证明厄米性, 用分部积分法得
\begin{equation}\label{eq_Rmat_1}
\int_{0}^{\infty} uHv\dd{x} - \int_{0}^{\infty} vHu\dd{x}
= \eval{-\frac{1}{2m}[uv' - u'v]}_{0}^{+\infty}~.
\end{equation}
由于我们假设波函数在 $x=0$ 和无穷远处消失, 则该式为零, 说明 $H$ 是厄米的。 但如果在有限区间 $[0, a]$ 中, 则波函数的边界条件必须满足 $u(a)v'(a) - u'(a)v(a) = 0$ 才能保证厄米性。

但若边界条件不符合该要求, 为了在 $[0,a]$ 内构造一组离散的正交归一基底, 我们可以拼凑一个厄米算符。 把\autoref{eq_Rmat_1} 修改积分区间并移项得
\begin{equation}
\qty[\int_{0}^{a} uHv\dd{x} + \frac{1}{2m}u(a)v'(a)] - \qty[\int_{0}^{a} vHu\dd{x} + \frac{1}{2m}u'(a)v(a)]
= 0~.
\end{equation}
而又可以通过狄拉克 $\delta$ 函数\upref{Delta} 表示为
\begin{equation}
\int_{0}^{a} u\qty[H + \frac{\delta(x-a)}{2m}\dv{x}] v\dd{x} -
\int_{0}^{a} v\qty[H + \frac{\delta(x-a)}{2m}\dv{x}] u\dd{x} = 0~.
\end{equation}
所以无论波函数在 $x=a$ 端的边界条件如何, 方括号中的算符都是厄米的。 令\textbf{布洛赫算符(Bloch operator)}为
\begin{equation}\label{eq_Rmat_6}
\mathscr L = \delta(x-a)\dv{x}~,
\end{equation}
那么修正后的哈密顿算符(厄米算符)就是 $H + \mathscr L/(2m)$。 于是本征方程为
\begin{equation}\label{eq_Rmat_2}
\qty(H + \frac{\mathscr L}{2m})\chi_j = \frac{k_j^2}{2m} \chi_j~,
\end{equation}
本征值为 ${k_i^2}/{2m}$。 于是实函数 $\chi_i(x)$ 就构成一组 $[0, a]$ 内的正交归一基底, 满足 $\int_0^a \chi_i \chi_j \dd{x} = \delta_{ij}$。 为了明确\autoref{eq_Rmat_2} 的意义, 把\autoref{eq_Rmat_2} 左乘任意 $\chi_i$ 并在 $[0,a]$ 积分\footnote{该积分实际上是在区间 $[0,a+\epsilon]$ 积分然后令 $\epsilon\to 0^+$, 下同。}得
\begin{equation}\label{eq_Rmat_3}
\mel{\chi_i}{2mH + \mathscr L}{\chi_j} = k_i^2 \delta_{ij}~,
\end{equation}
这里使用了狄拉克符号\upref{braket}表示积分。 这组基底如何具体计算呢? 首先还是要明确 $x=a$ 处的边界条件。 一个简单的例子是令 $u(a) = 0$, 解 $H$ 的本征基底。 这相当于解 $[0,a]$ 中的无限深势阱加上势能 $V(x)$。 另一个简单的例子是把边界条件 $u(a) = 0$ 改成 $u'(a) = 0$, 也能得到一组基底。 更妙地, 也可以把这两组基底合并。 可以验证这三组基底都满足\autoref{eq_Rmat_3} 即\autoref{eq_Rmat_2}。

\subsection{散射态的展开}
求出区间 $[0,a]$ 的基底以后, 就可以在该区间展开任意的散射态, 散射态满足
\begin{equation}\label{eq_Rmat_4}
H\psi_k = \frac{k^2}{2m}\psi_k~.
\end{equation}
令
\begin{equation}\label{eq_Rmat_5}
\psi_k = \sum_j c_j(k)\chi_j~,
\end{equation}
代入并左乘 $2m\chi_i$ 并在 $[0,a]$ 积分得
\begin{equation}
\sum_j \qty(2mH_{ij} - \delta_{ij}k^2) c_j = 0~.
\end{equation}
这是一个其次线性方程组, 可以解出坐标 $c_j$。 其中
\begin{equation}
2mH_{ij} = \mel{\chi_i}{2mH}{\chi_j} = \mel{\chi_i}{2mH + \mathscr L}{\chi_j} - \mel{\chi_i}{\mathscr L}{\chi_j} 
= k_i^2\delta_{ij} - \chi_i(a)\chi'_j(a)~.
\end{equation}
代入得
\begin{equation}\label{eq_Rmat_7}
\sum_j \qty[(k_i^2 - k^2)\delta_{ij} - \chi_i(a)\chi'_j(a)] c_j = 0~,
\end{equation}

由此可以证明一个有用的关系: $x=a$ 处的对数导数为(留做习题)
\begin{equation}\label{eq_Rmat_8}
\frac{\psi_k'(a)}{\psi_k(a)} = \frac{1}{aR(k)}~,
\end{equation}
其中 $R(k^2)$ 就是 $R$-矩阵
\begin{equation}
R(k) = \frac{1}{a} \sum_{i=1}^\infty \frac{\chi_i^2(a)}{k_i^2 - k^2}~.
\end{equation}
事实上目前这只是一个数, 即 $1\times 1$ 的矩阵。 在多通道问题中才会成为真正的矩阵。 通过对数导数,我们就可以匹配 $[a,\infty)$ 区间的散射态波函数。

\subsection{势能修正项}
容易证明, 若给 $\mathscr L$ 的定义(\autoref{eq_Rmat_6}) 加上一个任意实函数 $U(x)$, 也可以使 $H+\mathscr L/(2m)$ 为厄米算符。 事实上这相当于修改了 $H$ 中的势能 $V(x)$。 有时候选取适当的 $U(x)$ 可以使基底 $\chi_i$ 变得更简单。 按照同样的推导, \autoref{eq_Rmat_7} 和\autoref{eq_Rmat_8} 变为
\begin{equation}
\sum_j \qty[(k_i^2 - k^2)\delta_{ij} - \chi_i(a)\chi'_j(a) - U(a)\chi_i(a)\chi_j(a)] c_j = 0~,
\end{equation}
\begin{equation}
\frac{\psi_k'(a)}{\psi_k(a)} = \frac{1}{aR(k)} - U(a)~.
\end{equation}
