% 多项式的可约性质
% keys 多项式|可约|不可约
% license Xiao
% type Tutor

\pentry{辗转相除法\nref{nod_SucDiv}}{nod_6517}

\footnote{吴群。矩阵分析[M].上海:同济大学出版社}这一节的内容是关于多项式的可约与不可约性质的,这些概念是因式分解的基础,也有助于理解其它数学分支中的“约化”概念。

\begin{definition}{}\label{def_RedPol_1}
设 $p(x)$ 是数域 $\mathbb{F}$ 上的多项式,若 $p(x)$ 在数域 $\mathbb{F}$ 上只有平凡因式\upref{ExDiv},则称 $p(x)$ 为域 $\mathbb{F}$ 上的\textbf{不可约多项式},否则,称 $p(x)$ 为域 $\mathbb{F}$ 上的\textbf{可约多项式}。
\end{definition}
按照定义,一次多项式总是不可约多项式。不可约多项式 $p(x)$ 与任一多项式 $f(x)$ 之间只可能有两种关系,或者 $p(x)|f(x)$,或者 $(p(x),f(x))=1$(该符号见\autoref{def_ExDiv_2}~\upref{ExDiv})。

\begin{theorem}{}
设 $p(x)$ 是数域 $\mathbb{F}$ 上的不可约多项式, $f(x),g(x)$ 是数域 $\mathbb{F}$ 上的 两个多项式,若 $p(x)|f(x)g(x)$ ,则一定有 $p(x)|f(x)$ 或者 $p(x)|g(x)$
\end{theorem}
\textbf{证明:} 若 $p(x)|f(x)$,那么结论以及成立。如果 $p(x)\nmid f(x)$,那么由 $p(x)$ 的不可约性,说明 $(p(x),f(x))=1$,由\autoref{the_SucDiv_4}~\upref{SucDiv}即得 $p(x)|g(x)$。

由数学归纳法很容易将这个定理推广
\begin{theorem}{}\label{the_RedPol_1}
若不可约多项式 $p(x)$ 整除对多项式 $f_1(x),f_2(x),\cdots,f_s(x)$ 的乘积
\begin{equation}
p(x)|f_1(x)f_2(x)\cdots f_s(x)~,
\end{equation}
则在多项式 $f_1(x),f_2(x),\cdots,f_s(x)$ 中,必有一多项式 $f_i(x)$ 存在,使得 $p(x)|f_i(x)$。
\end{theorem}
