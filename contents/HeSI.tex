% 氦原子单散射态的数值解

\begin{issues}
\issueDraft
\end{issues}

根据论文, 求本征态可以在每个 $L,M$ 空间中单独求.
\begin{equation}
\Psi_{l_1,l_2,k_2}^{L,M} = \frac{1}{r_1 r_2}\sum_{l'_1, l'_2}  \psi_{l'_1, l'_2, k_2}^{L, M}(r_1, r_2)\mathcal{Y}_{l'_1, l'_2}^{L, M}(\uvec r_1, \uvec r_2)
\end{equation}
非对称化的散射态满足边界条件
\begin{equation}
\psi_{l'_1,l'_2,k_2}^{L, M} \overset{r_2\to\infty}{\longrightarrow} \delta_{l_1,l'_1}\delta_{l_2,l'_2} r_1 R_{n_1,l_1}(r_1)
\sin\qty[k_2 r_2 - \frac{\pi l_2}{2} +\frac{1}{k_2}\ln(2k_2 r_2) + \sigma_{l_2} + \delta_{n_1,l_1,l_2}^{L,M}]
\end{equation}
那么接下来使用一组离散的基底来展开径向波函数
\begin{equation}
\psi_{l'_1, l'_2,k_2}^{L, M}(r_1, r_2) = \sum_{n_1,n_2} C_{n_1,n_2} \psi_{n_1,l_1}^{(1)}(r_1) \psi_{n_2,l_2}^{(2)}(r_2)
\end{equation}
其中 $\psi_{n_1,l_1}^{(1)}$ 是一系列符合当前 box 边界条件的基底, 对 $E < 0$, 可以取若干类氢原子束缚态径向波函数 $r_1 R_{n_1, l_1}$, 对 $E > 0$ 可以取氢原子散射态(库伦函数) $F_{l_1}(kr_1)$, 若要求散射态满足边界条件, $k$ 的取值同样也是离散的.

数值实验: 对 $r_\text{max} = 15.33$ 的球形 box (36FE), 取 7 个束缚态以及 23 个散射态, 可以得到基态能量为 -2.90159, 已经非常接近 imaginary time 的能量 -2.9037.

总共的基底个数为 $n_\text{max}^2 l_\text{max}/2$.
