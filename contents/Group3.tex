% 群作用
% keys 群作用|同态|凯莱定理|Cayley定理|群表示
% license Xiao
% type Tutor


\pentry{群的同态与同构\nref{nod_Group2}}{nod_52c8}
% Giacomo:是否需要引入 G-空间 和 等变映射 相关的概念。

\subsection{群在自身上的作用}\label{sub_Group3_1}

给定一个群 $G$,我们任意拿出一个元素 $a\in G$,用 $a$ 去左乘 $G$ 中的所有元素(包括 $a$ 自己),那么我们可以把 $a$ 看成一种 $G$ 到自身上的映射:$f_a:G\rightarrow G$,使得对于任意 $x\in G$,$f_a(x)=ax$。

群 $G$ 中的每一个元素都可以像这样生成一个映射,把这些映射全部放在一起,我们也可以整体上看成一个映射 $f:G\times G\rightarrow G$,满足:对于任意的 $a, x\in G$,有 $f(a,x)=f_a(x)=ax$。

% 这样的 $G\times X$ 到 $X$ 上的映射,被称为一个\textbf{群作用(group action)}。

% Giacomo:现在没法解释为什么右平移是右乘 a^{-1} 而不是 a,所以干脆之后一起讲。
% 
% 按照上述定义得到的作用,通常称为\textbf{左平移}作用;相应地,我们也可以让元素 $a$ 对 $x$ 的作用是 $g_a(x)=x a^{-1}$,这样的作用被称为\textbf{右平移}。

\subsection{群作用}

% Giacomo:群作用是一个映射 $G \times X \to G$,要指出等价的可以定义为 $G \to \Aut(G)$,(可以在备注里说明用Aut定义的版本使用范围更窄,没有办法拓展到,比如光滑的李群作用,一些更一般的情况);最后再说明可以简写成 g \cdot x,但本质上还是 $\rho(g, x)$ 或者 $\rho(g)(x)$.
% 额外备注,我会把他定义成$G \to \Aut(G)$,然后说它等价于$G \times X \to G$。

更一般地,对于任何集合 $X$,群 $G$ 中每个元素都可以代表 $X\rightarrow X$ 的一个映射。我们当然可以任意规定这些映射,但如果这些映射满足一定条件的话,就会构造出一个很有意思的结构:

\begin{definition}{群作用}\label{def_Group3_1}
设群 $G$ 和集合 $X$,$G$ 中每个元素都是 $X$ 到自身的映射,记 $g\in G$ 将 $x\in X$ 映射为 $g\cdot x\in X$。如果所有这些映射满足满足下面两条公理:
\begin{itemize}
\item \textbf{结合律}:对于 $g_1, g_2\in G, x\in X$,$(g_1 g_2)\cdot x=g_1\cdot (g_2\cdot x)$。
\item \textbf{单位元是恒等映射}:$G$ 的单位元 $e$ 将任何 $x\in X$ 映射到自身:$e\cdot x=x$。
\end{itemize}

那么我们称群 $G$ \textbf{作用(acts)}于集合 $X$ 上。

\end{definition}

群作用可以像定义里一样记为 $g\cdot x$,也可以记为 $X$ 到 $X$ 的若干映射 $f_g(x)=g\cdot x$,还可以整体上看成 $G\times X$ 到 $X$ 的一个映射 $f(g, x)=g\cdot x$。

\subsection{群作用的例子}

\begin{example}{平凡作用}\label{ex_Group3_5}
(不做任何变换的)群作用 $G \times X \to X, (g, x) \mapsto x$ 被称为\textbf{平凡作用}。
\end{example}

注:把trival翻译平凡并不贴切,平庸是更合适的翻译。% Giacomo注

\begin{example}{平移作用}\label{ex_Group3_1}
本条\autoref{sub_Group3_1} 定义的映射,就是群在自身集合上的作用,称为\textbf{左平移}作用。相应地,我们也可以让元素 $a$ 对 $x$ 的作用是 $x a^{-1}$,这样的作用被称为\textbf{右平移}。
\end{example}

\textbf{证明:} $G \times G \to G, (a, x) \mapsto x a^{-1}$ 满足对任意 $a, b \in G$,
\begin{equation}
(a b) \cdot x = x (a b)^{-1} = x b^{-1} a^{-1} = (b \cdot x) a^{-1} = a \cdot (b \cdot x) ~.
\end{equation}
\textbf{证明结束。}

可想而知,$(a, x) \mapsto x a$ 并不满足群作用的定义。


由群运算的唯一性(消去律),平移作用是群在自身上的(集合意义上的)双射。因此每个平移作用都可以看成一个\textbf{置换。}这么一来,我们还得到一个重要的性质:

\begin{theorem}{Cayley定理}\label{the_Group3_3}
任何群 $G$ 都同构于其自身的置换群 $S_G$ 的一个子群。
\end{theorem}

\begin{example}{伴随作用}\label{ex_Group3_2}
对于任意 $a\in G$,令 $f_a: G\rightarrow G$ 满足 $\forall x\in G, f_a(x)=axa^{-1}$,则这些映射定义了一个群 $G$ 在集合 $G$ 上的作用,称为\textbf{伴随作用}。$f_a(x)$ 称为 $x$ 的\textbf{共轭元素}。

在\textbf{群的同态与同构}\upref{Group2}文章中我们知道,全体伴随作用构成群的内自同构群,也称共轭自同构群。
\end{example}

\begin{example}{线性变换}\label{ex_Group3_3}
参考线性变换\upref{LTrans}。我们已经知道,$n$ 阶非奇异矩阵配上乘法可以构成一个群;相应地,满秩线性变换(可逆线性变换)配上映射的复合运算构成一个群。非奇异矩阵乘法是给定了基向量以后,满秩线性变换的复合的表示。

取 $n$ 维实数向量空间 $\mathbb{R}^n$,那么 $\mathbb{R}^n$ 是向量的集合;$\opn{GL}(n,\mathbb{R})$ 是 $\mathbb{R}^n$ 上可逆线性变换的群,显然 $\opn{GL}(n,\mathbb{R})$ 按照通常的线性变换定义,构成了在 $\mathbb{R}^n$ 上的一个作用。
\end{example}

\begin{example}{置换矩阵}\label{ex_Group3_6}
参考\autoref{def_Perm_1}~\upref{Perm},$\rho_\text{perm}(\pi): = R_\pi$ 定义了$\mathbb{R}^n$ 上的一个 $S_n$-作用。我们可以推广这个定义,考虑向量空间 $V$ 的 $n$ 次笛卡尔幂空间 $V^{\times n}$ 上存在一个 $n$ 阶对称群 $S_n$ \upref{Perm}的群作用:
\begin{equation}
\begin{aligned}
\rho_\text{perm}(\sigma): V^{\times n} &\to V^{\times n}~, \\
(v_1, \cdots, v_n) &\mapsto (v_{\sigma(1)}, \cdots, v_{\sigma(n)})~,
\end{aligned}
\end{equation}
被称为\textbf{自然置换作用}(或者\textbf{自然置换表示},参考群表示\upref{GrpRep})。此时 $\rho_\text{perm}(\sigma)$ 是一个 $n \times n$ 的分块矩阵,矩阵元为零矩阵或者恒等矩阵。

另一方面 $\pi \mapsto\det(\pi) \rho_\text{perm}(\pi)$ 也构成一个群作用,但它没有一个特定的名字。
\end{example}


\subsection{群作用的性质}


\begin{definition}{不动点}\label{def_Group3_2}
设群$G$作用在集合$X$上。如果$x\in X$在任意$g\in G$的作用下都不变,即$g\cdot x=x$,那么称$x$为该群作用的一个\textbf{不动点(fixed point)}。

定义不动点集为$X^G: = \opn{Fix}_G(X): = \{x\in X\mid g\cdot x=x, \forall g\in G\}$。
\end{definition}



% 当我们讨论群 $G$ 在集合 $X$ 上的作用时,一共有两个集合要关心。

\begin{definition}{轨道}
设群$G$作用在集合$X$上。

固定 $X$ 中的一个元素 $x$,那么每个 $G$ 中元素 $g$ 都把 $x$ 映射到某个 $f_g(x)\in X$ 上。所有能这样被映射到的元素 $f_g(x)$ 构成了 $X$ 的一个子集,称为元素 $x$ 的\textbf{轨道(orbit)},记为 $G \cdot x$\footnote{这里的 $\cdot$ 是群作用的 $\cdot$ }。
\end{definition}









\begin{exercise}{}\label{exe_Group3_1}
设群$G$作用在集合$X$上。给定$x\in X$,定义$F_x=\{g\in G\mid g\cdot x=x\}$。证明 $F_x$ 构成群(\autoref{ex_GroupP_2}~\upref{GroupP})。
\end{exercise}




\begin{definition}{迷向子群}
称\autoref{exe_Group3_1} 中的$F_x$为 $x$ 的在给定群作用下的\textbf{迷向子群(isotropy subgroup)}或者\textbf{稳定化子群(stablizer subgroup)},也可简称\textbf{稳定化子(stablizer)}。
\end{definition}






\begin{definition}{}
如果对于任何 $x\in X$,$x$ 的轨道都是整个 $X$,那么我们称这个作用是\textbf{可递的(transitive)},此时 $X$ 就是 $G$ 的\textbf{齐性空间}。如果对于任何 $x\in X$,$x$ 的轨道只是 $\{x\}$,那么这个作用就是\textbf{平凡(trivial)}的。
\end{definition}

\begin{example}{置换群的可递子群}\label{ex_Group3_4}
置换群$S_n$显然是在$n$元集合上的群作用,其子群$G$是可递的,当且仅当$G$对$n$元集合的作用是可递的。
\end{example}

如果对于任何 $x\in X$,任何 $g\in G-\{e\}$,都使得 $g\cdot x\not=x$,那么我们说这个作用是\textbf{有效的}。有效性等价于说任何 $x\in X$ 的迷向子群都是 $\{e\}$。

在之前的例子中,平移作用既是可递的,又是有效的。但是伴随作用不能保证有效性和可递性,具体情况要看群的结构性质。全体可逆线性变换构成的群作用在非零向量空间上,这个作用是可递的,也是有效的——注意一定得是非零向量空间,把零向量排除在外。

\begin{theorem}{}\label{the_Group3_1}
设群 $G$ 作用在 $X$ 上,在 $X$ 上定义关系 $\sim$ 如下:$\forall x, y\in X, x\sim y \iff \exists g\in G, g\cdot x=y$,或者说,$x\sim y$ 当且仅当 $y$ 在 $x$ 的轨道里。那么,$\sim$ 是一个等价关系。
\end{theorem}

由群 $G$ 的封闭性和逆元存在性分别可以证明\autoref{the_Group3_1} 中关系 $\sim$ 的传递性和对称性。这个定理说明,轨道划分是一种等价类划分。






\begin{definition}{}\label{def_Group3_3}
设群 $G$ 按照伴随作用,作用在自身上。
\begin{itemize}
\item 任取$g\in G$和$G$的子群$H$,则称$gHg^{-1}$为$H$的一个\textbf{共轭子群(conjugate subgroup)},或者说$H$和$gHg^{-1}$\textbf{彼此是共轭的(be conjugate to each other)}。
\item 对于 $g\in G$,记 $C_g$ 为 $g$ 在伴随作用下的轨道,称 $C_g$ 为 $g$ 的\textbf{共轭类(conjugate class)},每一个 $h\in C_g$ 都称为 $g$ 的\textbf{共轭元素(conjugate)}。

\item 对于 $g\in G$,记 $C_G(g)$ 是 $g$ 在伴随作用下的迷向子群, 称 $C_G(g)$ 为 $g$ 在 $G$ 中的\textbf{中心化子(centralizer)}。

\item 记 $C(G)=\bigcap_{g\in G} C_G(g)$,称为群 $G$ 的\textbf{中心(center)}。有的地方也记$C(G)=Z(G)$。
\end{itemize}
\end{definition}

群中心的定义也可以这么说:所有可以和 $G$ 中一切元素交换的元素构成的集合,就是 $C(G)$。








\begin{theorem}{共轭类等式(The Class Equation)}\label{the_Group3_4}

设\textbf{有限群}$G$按照伴随作用,作用在自身上,$C_x$是该作用下$x\in G$的轨道,即$x$的共轭类。全体$C_x$构成的集合(如果$C_x=C_y$则视为同一个)记为$O=\{C_i\}_{i=1}^n$。则有:
\begin{equation}\label{eq_Group3_1}
\abs{G} = \sum_{C\in O} \abs{C} = \abs{Z(G)} + \sum_{C\in O, \abs{C}>1} \abs{C}~.
\end{equation}

\end{theorem}


\textbf{证明}:

由于共轭类划分是$G$上的等价类划分,故$\abs{G} = \sum_{C\in O} \abs{C}$。

由$Z(G)$的定义,$x\in Z(G)$当且仅当$x$在伴随作用下不变,即$\abs{C_x}=1\iff x\in Z(G)$。故$\abs{Z(G)}=\sum_{C\in O, \abs{C}=1}$,从而得证\autoref{eq_Group3_1} 第二个等号。

\textbf{证毕}。




\begin{theorem}{轨道-迷向子群定理}\label{the_Group3_2}
设群 $G$ 作用在集合 $X$ 上。固定一个 $x\in X$,那么对于 $g, h\in G$,$g\cdot x= h\cdot x$ $\iff$ $g$ 和 $h$ 在迷向子群 $F_x$ 的同一个左陪集上。
\end{theorem}

\textbf{证明}:

 $g\cdot x=h\cdot x\iff g^{-1}h\cdot x=x\iff g^{-1}h\in F_x\iff h\in gF_x$。

\textbf{证毕}。






由\autoref{the_Group3_2} 直接可得如下推论:


\begin{corollary}{轨道元素数与迷向子群指数}\label{cor_Group3_2}
设群$G$作用在集合$X$上。任取$x\in X$,则其轨道中的元素数目,等于$x$的迷向子群的在$G$中的\textbf{指数}(\autoref{def_coset_2}~\upref{coset})。
\end{corollary}




\autoref{cor_Group3_2} 中取$X=G$、作用为伴随作用后,可得如下推论:

\begin{corollary}{}\label{cor_Group3_3}

设\textbf{有限群}$G$按照伴随作用,作用在自身上。任取$x\in G$,则$\abs{C_x}=\abs{G}/\abs{C_G(x)}$。

\end{corollary}




\begin{theorem}{}\label{the_Group3_5}
设群$G$作用在集合$X$上,$\abs{G}=p^k$,其中$p$是一个素数,$k$是一个\textbf{正整数}。则
\begin{equation}
\abs{X} \equiv \abs{\opn{Fix}_G(X)} \mod  p~,
\end{equation}
或者说,$p\mid \qty(\abs{X}-\abs{\opn{Fix}_G(X)})$。
\end{theorem}

\textbf{证明}:

由拉格朗日定理(\autoref{the_coset_2}~\upref{coset}),易知$G$的子群在$G$中的指数都形如$p^n$,其中各$n$是\textbf{非负整数}。又由\autoref{cor_Group3_2} ,$X$中各轨道的元素数量都形如$p^n$,其中元素数量是$1$的即不动点集合$\opn{Fix}_G(X)$。

于是$X-\opn{Fix}_G(X)$中各轨道的元素数量都形如$p^k$,其中各$k$为\textbf{正整数}。

\textbf{证毕}。









\begin{corollary}{}\label{cor_Group3_1}
由\autoref{the_Group3_2},$|G|/|F_x|=|O_x|$,其中 $O_x$ 是 $x$ 的轨道。
\end{corollary}

\begin{exercise}{Burnside引理}\label{exe_Group3_2}
设群 $G$ 作用在集合 $X$ 上。对于给定的 $g\in G$,记 $X^g=\{x\in X|g\cdot x=x\}$,$O_x$ 是 $x\in X$ 的轨道,那么 $|\{O_x|x\in X\}|=\frac{1}{|G|}\sum_{g\in G}|X^g|$。就是说,$X$ 上轨道的数目,等于每个 $g\in G$ 作用后不产生效果的元素数量之平均值。证明此引理(见\autoref{ex_GroupP_3}~\upref{GroupP})。
\end{exercise}
