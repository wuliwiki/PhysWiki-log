% 北京大学 1994 年 考研 固体物理
% license Usr
% type Note

\textbf{声明}:“该内容来源于网络公开资料,不保证真实性,如有侵权请联系管理员”

1.对于三维与二维自由中子气,分别导出能态密度作为穷能量的函数。(20)

2.设有晶格常数为$a$的一维晶格,异带底附近能量表示为;

价带项附近能量可以表示为。

是与$K$无关的常量,是求出:
(1)禁带宽度

(2)导带电子的有效质量

(3)电子由价带顶跃迁到导带底时准动量的改变量。(30)

3.(1)证明理想的六角密堆积(hcp)结构的轴比

(2)假设某固体从体心立方结构转变为hcp结构,后者具有理想的轴比,在相变过程中保持密度不变。求hcp结构的$a$与体心立方体结构的晶格常数$a$’之间的关系。(20)

4、(1)甲乙两种原子组成双原子链,质量分别为.$M$写$m$,间距为 $a$,只考虑临近的原子作用,求出光学波声学波的频率作为波数的函数并在简约布里渊区作图,驾出详细推导过程

(2)如$M$))$m$,讨论在长波极限和波接近于简约布里渊区边界的光学波与声学波的特性。(30)









