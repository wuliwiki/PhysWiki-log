% 二叉树
% 二叉树|树|数据结构|C++

每个节点最多含有两个子树的树被称为二叉树。

二叉树的特点:

\begin{enumerate}
\item 每个结点只有两棵子树,分别为左子树和右子树;
\item 二叉树的左右子树如果位置发生了改变,含义也会发生改变
\end{enumerate}

不包含任意结点的二叉树称为空树或零树,如果左子树非空,则它的根称为整棵树的根的左孩子,右子树也类似。在一棵二叉树中,如果一个结点仅有一个孩子,则它是左孩子还是右孩子是有关系的。而在有序树中,则是一样的。

二叉树的性质:

在第 $i$ 层上最多有 $2 ^ {i - 1}$ 个节点。$(i\geq1)$

二叉树中深度为$k$,那么最多有 $2 ^ {k - 1}$ 个节点。$(k\geq1)$,(根节点的深度为 $k_0 = 0$)。

\textbf{满二叉树:}

国内定义:满二叉树是指每层上的所有结点都有两个子结点的二叉树,最后一层的结点(叶结点)是满的,因此满二叉树形如一个三角形。一个层数为 $k$ 的满二叉树总结点数为 $2^k - 1$。$(k\geq1)$。

国外定义:如果二叉树的每个节点要么是叶结点,要么正好拥有两个子结点,那么就是一棵满二叉树。

\begin{figure}[ht]
\centering
\includegraphics[width=7cm]{./figures/5fcb08ad09645819.png}
\caption{示例图} \label{fig_tree_2}
\end{figure}

如上图在国内就不是一棵满二叉树,而在国外就是一棵满二叉树。

\textbf{完全二叉树:}

在一颗二叉树中,若除最后一层外的其余层都是满的,并且最后一层要么是满的,要么结点从左到右依次排列。