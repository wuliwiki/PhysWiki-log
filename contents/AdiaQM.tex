% 绝热近似(量子力学)
% license Xiao
% type Tutor

\begin{issues}
\issueTODO
\end{issues}

\pentry{薛定谔方程(单粒子一维)\nref{nod_TDSE11}}{nod_fa5c}

\subsection{绝热近似(非简并)}
\footnote{参考 Griffiths\cite{GriffE} 的章节 The Adiabatic Approximation; Shankar\cite{Shankar} Chap18-P478; Wikipedia \href{https://en.wikipedia.org/wiki/Adiabatic_theorem}{相关页面}。}量子力学中,\textbf{绝热近似(adiabatic approximation)}说的大概是: 若系统初始时处于某个离散非简并的本征态,那么当哈密顿量随时间缓慢改变时(改变的特征时间远大于本征态的), 那改变过程中波函数将仍然处于同一个本征态,但整体相位会发生某种改变。下面先给出定量结论,证明留到文末。本文只讨论离散束缚态张成的空间而不讨论散射态。

令含时薛定谔方程为(\autoref{eq_TDSE11_6})
\begin{equation}\label{eq_AdiaQM_13}
H(t)\Psi(t) = \I\hbar\dot\Psi(t)~.
\end{equation}
当系统不存在简并时, 绝热近似下含时薛定谔方程的通解可以表示为($C_n$ 为常数,由初始波函数决定)
\begin{equation}\label{eq_AdiaQM_2}
\Psi(t) \approx \sum_n C_n \psi_n(t) \E^{\I\theta_n(t)}~.
\end{equation}
其中相位函数定义为
\begin{equation}\label{eq_AdiaQM_8}
\theta_n(t) \equiv -\frac{1}{\hbar} \int_0^t E_n(t')\dd{t'}~.
\end{equation}
且 $\psi_n(t)$ 是 $H(t)$ 一组正交归一本征态,任意时刻都满足不含时薛定谔方程(时间看作数学参数)
\begin{equation}\label{eq_AdiaQM_3}
H(t)\psi_n(t) = E_n\psi_n(t)~.
\end{equation}
和正交归一化
\begin{equation}\label{eq_AdiaQM_6}
\braket*{\psi_m(t)}{\psi_n(t)} = \delta_{m,n}~.
\end{equation}
为了方便且不失一般性本文规定 $\psi_n(t)$ 始终是实值函数(否则有可能出现一个随时间变化的整体相位让事情更复杂)。 证明见下文。
\addTODO{这个规定可能并不那么方便,例如氢原子束缚态中的球谐函数是复值}

\subsubsection{缓慢条件}
如何判断绝热近似中的 “缓慢” 条件是否满足呢? 下文的证明中会看到当任意
\begin{equation}\label{eq_AdiaQM_11}
\braket*{\psi_m}{\dot\psi_n} = \frac{\mel*{\psi_m}{\dot H}{\psi_n}}{E_n-E_m} \qquad (E_m\ne E_n)~.
\end{equation}
可以忽略时\autoref{eq_AdiaQM_2} 成立。

\begin{example}{}
\begin{enumerate}
\item \enref{无限深势阱}{ISW}缓慢变长。
\item \enref{量子简谐振子}{QSHOop}的固有频率 $\omega$ 缓慢变化。
\end{enumerate}
\end{example}

容易看出若 $H(t)$ 不随时间变化时,通解就回到了熟悉的通解(\autoref{eq_TDSE11_5})
\begin{equation}
\Psi(t) = \sum_n C_n \psi_n \E^{-\I E_n t/\hbar}~.
\end{equation}

\autoref{eq_AdiaQM_2} 中 $C_n$ 为常数是一个很有力的结论。它告诉我们若开始时波函数处于某个(非简并)本征态,那么它将始终(近似)处于该本征态。

该理论在对分子的计算中有广泛的应用,且有一个响亮的名字,叫\textbf{波恩—奥本海默近似(Born–Oppenheimer approximation)}。 这是因为在分子运动中,原子核的运动速度通常要比电子慢得多,使绝热近似效果较好。

同为含时近似理论,绝热近似和\enref{含时微扰理论}{TDPTc}有什么区别呢?前者不要求 $H(t)$ 缓慢变化,例如用激光波包对原子光电离时,电场随时间的周期变化往往并不算慢。 那可以使用绝热近似的情况是否可以使用含时微扰理论呢? 理论上可以,但计算比较麻烦,因为含时微扰使用初始的本征态展开任意时刻的波函数。

\subsection{简单的简并}
先来看一个简单的简并含时哈密顿,也是量子力学中经常出现的。
\begin{theorem}{绝热近似(简单的简并)}\label{the_AdiaQM_2}
对形如
\begin{equation}\label{eq_AdiaQM_15}
\mat H(t) = \mat H^0 + \alpha(t)\mat H^1~
\end{equation}
的含时哈密顿($\mat H^0$ 和 $\mat H^1$ 本身不含时), 无论 $\mat H^0$ 是否简并, 当 $\alpha(t)$ 缓慢变化使得\autoref{eq_AdiaQM_11} 可以忽略时, 绝热近似\autoref{eq_AdiaQM_2} 仍然成立。
\end{theorem}
证明见下文。

若考虑的时间段内,只有初始的一瞬间存在简并, 那么可以认为这个瞬间波函数几乎不发生变化(毕竟 $H(t)$ 是缓慢变化),令 $\psi_n(0)$ 取好\enref{量子态}{TIPT},并假设系统始终是非简并的即可。

\pentry{一阶不含时微扰理论(量子力学)\nref{nod_TIPT}}{nod_5759}
\begin{example}{}
给氢原子的任意束缚态 $\psi_{n,l,m}$ 缓慢施加外电场或磁场(参考 “\enref{类氢原子斯塔克效应(微扰)}{HStark}”,以及“\enref{塞曼效应}{ZemEff}”)。注意 $\psi_{n,l,m}$ 并不是好本征态,需要先做投影。
\addTODO{推导}
\end{example}

\subsection{含时薛定谔方程的一种矩阵形式}
作为绝热近似证明的准备, 我们需要先采用某种基底把含时薛定谔方程变为矩阵形式。

在绝热近似中, 我们选择把含时波函数用瞬时本征态 $\psi_n(t)$ 展开(注意这里的系数是含时的)
\begin{equation}
\Psi(t) \equiv \sum_n c_n(t) \psi_n(t) \E^{\I \theta_n(t)}~,
\end{equation}
为了下面化简方便,不失一般性,令式中
\begin{equation}
\theta_n(t) \equiv -\frac{1}{\hbar} \int_0^t E_n(t')\dd{t'}~.
\end{equation}
代入含时薛定谔方程
\begin{equation}\label{eq_AdiaQM_1}
H(t)\Psi(t) = \I\hbar \dot \Psi(t)~,
\end{equation}
得
\begin{equation}\label{eq_AdiaQM_5}
\dot c_m(t) = -\sum_n c_n \braket*{\psi_m}{\dot\psi_n}\E^{\I(\theta_n-\theta_m)}~.
\end{equation}
这可以表示为矩阵乘法
\begin{equation}\label{eq_AdiaQM_9}
\dot{\bvec c}(t) = \frac{1}{\I\hbar}\tilde{\mat H}(t) \bvec c(t)~.
\end{equation}
其中矩阵 $\tilde{\mat H}$ 定义为
\begin{equation}\label{eq_AdiaQM_10}
\tilde H_{ij}(t) = -\I\hbar \braket*{\psi_m}{\dot\psi_n}\E^{\I(\theta_n-\theta_m)}~.
\end{equation}
注意\autoref{eq_AdiaQM_9} 可以看作含时薛定谔方程的一种矩阵形式,和\autoref{eq_AdiaQM_1} 完全等效。 这类似于\autoref{eq_TDPT_3}。

另外对\autoref{eq_AdiaQM_3} 求时间偏导得
\begin{equation}\label{eq_AdiaQM_4}
\mel*{\psi_m}{\dot H}{\psi_n} = (E_n-E_m)\braket*{\psi_m}{\dot\psi_n} + \delta_{m,n}\dot E_n~.
\end{equation}
对\autoref{eq_AdiaQM_6} 求导可以证明矩阵 $\braket*{\psi_m}{\dot\psi_n}$ 是一个反对称矩阵,即满足
\begin{equation}\label{eq_AdiaQM_7}
\braket*{\psi_m}{\dot\psi_n} = -\braket*{\psi_n}{\dot\psi_m}~.
\end{equation}
注意 $n=m$ 时该式两边恒为零,\autoref{eq_AdiaQM_10} 中 \textbf{$\tilde{\mat H}$ 的对角元也恒为零}。 且通过该式容易证明 \textbf{$\tilde{\mat H}$ 是厄米矩阵}。

要精确计算矩阵 $\tilde{\mat H}$,一般直接根据定义直接求解 $\psi_n(t)$(\autoref{eq_AdiaQM_3})再代入 \autoref{eq_AdiaQM_10} 即可。 但为了估计 $\tilde{\mat H}$ 矩阵元的大小,我们可以由\autoref{eq_AdiaQM_4} 得(\autoref{eq_AdiaQM_11})
\begin{equation}
\braket*{\psi_m}{\dot\psi_n} = \frac{\mel*{\psi_m}{\dot H}{\psi_n}}{E_n-E_m} \qquad (E_m\ne E_n)~.
\end{equation}

到现在为止,所有推导都是精确的。 绝热近似的关键就在于假设 $H$ 随时间变化缓慢,即 $\dot H$ 非常小,以至于如果两个能级 $E_n$ 和 $E_m$ 不是特别接近时,可以近似认为\autoref{eq_AdiaQM_11} 对应的矩阵元 $\tilde H_{m,n}$ 可以忽略不计。

\subsection{非简并情况的证明}
若在考虑的时间区间内, $H(t)$ 始终没有发生简并,不同的能级之间也没有太接近, 那么可以假设\autoref{eq_AdiaQM_11} 对全部 $m\ne n$ 为零,也就是 $\tilde{\mat H} = \bvec 0$。 此时\autoref{eq_AdiaQM_9} 直接变为
\begin{equation}
\dot{\bvec c}(t) = \bvec 0~.
\end{equation}
这说明所有系数都不随时间变化,令常数 $C_n = c_n(0)$,得到\autoref{eq_AdiaQM_2}。

\subsection{简并的情况的证明}
\pentry{块对角厄米矩阵的本征问题\nref{nod_BHeig}}{nod_a79d}
对 $\tilde{\mat H}$ 不为零的情况,\autoref{eq_AdiaQM_9} 形式上的通解为 % \addTODO{引用}
\begin{equation}
\bvec c(t) = \mat U(t)\bvec c(0)~.
\end{equation}
其中演化算符形式上为
\begin{equation}
\mat U(t) = \hat{\mathcal{T}}\exp[-\frac{\I}{\hbar}\int_0^t \tilde{\mat H}(t') \dd{t'}]~.
\end{equation}
但这无异于精确求解薛定谔方程,还没有告诉我们什么具体的结论。

考虑任意 $\mat H^0$ 的两个 $E_m\ne E_n$ 不同的简并空间在任何时间都具有不同能量且能使\autoref{eq_AdiaQM_11} 忽略的情况。 此时 $\tilde{\mat H}$ 是\enref{块对角厄米矩阵}{BHeig},每个对角块代表一个本征子空间(或者若干个可能在某时刻本征值相同的本征子空间张成的空间),不同子空间之间不存在耦合。 $\mat U$ 也是结构相同的分块酉矩阵,且每个对角块都分别是一个酉矩阵。每个本征子空间独立演化,投影概率保持不变。
\addTODO{链接: 一般含时薛定谔方程中 $\mat H(t)$ 是块对角矩阵的情况}

但是每个空间内部的波函数投影该如何随时间演化呢?该空间内部所有的态都是同一个能量的本征态,所以内部的正交归一基底可以任意选取。

\subsubsection{能级分裂与合并}
当每个子空间的本征能量随时间变化时,$\tilde{\mat H}(t)$ 的哪些矩阵元被忽略可能取决于时刻 $t$。 这就是说 $\tilde{\mat H}(t)$ 的对角块可能有些时候会发生拆分或合并(多个对角块发生耦合后合并为一个)。 但若这种合并的持续时间只有很短乃至一瞬间,那我们是否可以认为合并前后波函数不发生变化,也就是假设合并不存在呢?

我们先看一个经典的例子,或许会让你有些惊讶
\begin{example}{二阶系统}
令一个 $t=0$ 时简并的双态系统,哈密顿量为(为了书写方便本例中使用\enref{原子单位制}{AU},即 $\hbar=1$)
\begin{equation}
\mat H(t) = \pmat{1 & 0\\ 0 & 1}
+ \alpha t \pmat{0 & 1\\ 1 & 0}~.
\end{equation}
当 $t>0$ 时, 第二个哈密顿矩阵会导致能级分裂。

令 $\bvec\psi = (x,y)$, 代入含时薛定谔方程得
\begin{equation}
\leftgroup{
&\I\dot x = x + \alpha t y~,\\
&\I\dot y = \alpha t x + y~.
}\end{equation}
求解厄米矩阵 $\mat H(t)$ 的本征方程, 两个能级分别为
\begin{equation}\label{eq_AdiaQM_12}
E_\pm(t) = 1 \pm \alpha t~.
\end{equation}
正交归一的两个本征态始终为
\begin{equation}
\psi_\pm = \frac{1}{\sqrt{2}}\pmat{1\\ \pm1}~.
\end{equation}
其时间导数恒为零,根据\autoref{eq_AdiaQM_10} $\tilde{\mat H}$ 恒为零。 所以即使开始时存在简并,也能使\textbf{绝热近似精确成立}(\autoref{eq_AdiaQM_2})!

把\autoref{eq_AdiaQM_12} 代入\autoref{eq_AdiaQM_8} 得
\begin{equation}
\theta_\pm = -\qty(t \pm \frac{1}{2}at^2)~.
\end{equation}
所以方程的通解为:
\begin{equation}
\leftgroup{
&x(t) = C_+\E^{\I\theta_+(t)} + C_-\E^{\I\theta_-(t)}~,\\
&y(t) = C_+\E^{\I\theta_+(t)} + C_-\E^{\I\theta_-(t)}~.
}\end{equation}
\end{example}

更一般地,若我们只考虑 $\mat H^0$ 的某个能量为 $E^0$ 的简并子空间的演化,
\begin{theorem}{}\label{the_AdiaQM_1}
若令含时哈密顿为
\begin{equation}\label{eq_AdiaQM_14}
\mat H(t) = E^0\mat I + \alpha(t) \mat H^1~,
\end{equation}
其中 $\mat I$ 是单位矩阵, $\mat H^1$ 是厄米矩阵且与时间无关。 \textbf{那么绝热近似(\autoref{eq_AdiaQM_2})是精确的}。
\end{theorem}
注意定理中 $\mat H^1$ 允许存在简并,$\mat H^1$ 可能导致 $E^0$ 能级分裂。 注意定理不要求 $\alpha(t)$ 缓慢变化。

\textbf{证明}:记 $\mat H^1$ 的本征值和本征矢为
\begin{equation}
E_n^1~, \quad \bvec\psi_n \qquad (n=1,\dots,N)~,
\end{equation}
都\textbf{与时间无关}, 根据\autoref{eq_AdiaQM_10} $\tilde{\mat H}$ 恒为零。 这是证明的关键, 若\autoref{eq_AdiaQM_14} 使用更一般的含时哈密顿如 $\mat H(t) = \mat H^0 + \alpha(t) \mat H^1$ 则无法保证这点。

那么易证 $\mat H(t)$ 的本征值和本征矢为,
\begin{equation}\leftgroup{
&E_n(t) = E^0 + \alpha(t) E_n^1\\
&\bvec\psi_n
}\qquad (n=1,\dots,N)~,\end{equation}
且精确的含时波函数通解(\autoref{eq_AdiaQM_13})为
\begin{equation}
\bvec \Psi(t) = \sum_n C_n \bvec\psi_n \E^{\I\theta_n(t)}~.
\end{equation}
其中
\begin{equation}
\theta_n(t) = -\qty(E^0 t + E_n^1 \int_0^t\alpha(t')\dd{t'})~.
\end{equation}
\textbf{证毕}。

\subsubsection{证明\autoref{the_AdiaQM_2} }
当 $\mat H^0$ 非简并时, 直接使用本文开头的结论即可。 当 $\mat H^0$ 简并时, 上面已经得出, 在绝热近似下, 不同子空间的演化独立进行。 而\autoref{eq_AdiaQM_15} 在每个子空间内部则可化简为\autoref{the_AdiaQM_1} 中\autoref{eq_AdiaQM_14} 的形式! 这就证明了\autoref{the_AdiaQM_2}。
