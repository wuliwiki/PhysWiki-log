% ADK 电离率

\begin{issues}
\issueDraft
\end{issues}

\footnote{参考 \cite{Bransden} Chap.15.3} M. Ammosov, N. Delone, V. Krainov. 三个人给出了一般原子的瞬时的 tunnelling ionization rate, 叫做 \textbf{ADK rate}。
\begin{equation}
\dv{P}{t} = W_{ADK} = \omega_p \abs{C_{n^*l^*}}^2 G_{lm}\qty(\frac{4\omega_p}{\omega_T})^{2n^*-m-1}\exp(-\frac{4\omega_p}{3\omega_T})~.
\end{equation}
其中
\begin{equation}
\omega_p = I_p \qquad \omega_T = \frac{e\mathcal E_0}{\sqrt{2mI_p}} \qquad
n^* = \sqrt{\frac{I_p^H}{I_p}}~,
\end{equation}
\begin{equation}
\abs{C_{n^*l^*}}^2 = \frac{2^{2^{n^*}}}{n^* \Gamma(n^* + l^* + 1) \Gamma(n^* - l^*)}~,
\end{equation}
\begin{equation}
G_{lm} = \frac{(2l+1)(l+\abs{m})!}{2^{\abs{m}}\abs{m}!(l-\abs{m})!}~.
\end{equation}
其中 $I_p^H$ 是氢原子的电离能 $I_p$, $l,m$ 是角动量量子数, effective quantum number $l^* = 0$ 当 $l\ll n$, 否则 $l^* = n^*-1$。