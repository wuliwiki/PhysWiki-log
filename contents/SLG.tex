% 流体和固体
% 流体力学|剪应力|正应力

我们都知道物质的三态有液体、固体和气体\footnote{当然有一些物质无法归为这三类形态,比如果冻、凝胶等等,这些物质也会有复杂的性质},这三者在我们生活的世界里无处不在。我们脚底下的土地,我们呼吸的空气,我们喝的水、吹的风、踏过的河流…… 如何在物理定义上给出这三种物质形态的一个划分呢?

\subsection{应力的角度}
在流体力学中我们一般从应力的角度来划分这三种物质形态。\textbf{流体}是这样的一类物质,它能够在\textbf{剪应力}的作用下连续地发生形变。

\begin{figure}[ht]
\centering
\includegraphics[width=10cm]{./figures/acdc44e381b5a908.png}
\caption{几种不同的施加应力的方式} \label{fig_SLG_1}
\end{figure}

图中的“剪切”和“扭曲”过程都有剪应力的参与。剪应力是指施加在流体的某个单位面积元上的\textbf{平行于面积元方向的力},是应力张量的非对角元部分。它体现了剪应力与正应力的明显的区分。
\addTODO{应力张量词条}
图中的拉伸和压缩是\textbf{正应力},即力的方向是面积元的法线方向。在这种力的作用下,流体可能会发生\textbf{相变}——从液体到气体或从气体到液体。这是流体区分于固体的主要特点\footnote{当然,并不完全精确,总会有些特殊情况。}。不过要注意的是,\textbf{气体和液体}之间其实没有很强的分界线。在水的\textbf{临界温度}之上,水的液体与气体就不再有分界线了,也就是说在我们压缩和拉伸时,我们的肉眼将看不出液体与气体的特征,从应力和微观物理的层面上看,它们更没有分界线了。事实上,气体和液体的很多性质都是类似的,因此我们常用流体力学的手段去研究它们——例如我们的大气、我们的海洋,它们共有着许多规律——涡旋、对流等等。

\textbf{固体}是指在外力的影响下只会发生微小的形变(几乎没有),但撤去外力以后就会恢复原形。这意味着固体的形状不会轻易地发生变化(在一些特殊例子中,木头会发生断裂,金属会被拉伸而延展)。虽然固体不易形变,但总会有微小的形变,只是我们的肉眼很难观察的到。正因为它的形变及恢复力,固体才会发生振动,及声波在固体内部传播。于是我们才会听到鼓声敲响,才会感受到地震波的来临。

\subsection{微观角度}
从分子层面上看,固体是由一系列紧密排布的分子组成。它们在局部上或整体上是有序的,相互之间受分子间作用力或化学键的束缚,因此外力施加于固体之上时,它很难发生形变。有些固体的微观结构是长程有序的,可以找到它的晶胞(即重复排布的体积元),着被称为是“晶体”;否则被称为“非晶体”。铁、铜之类的金属就是晶体,氯化钠(食盐)、冰、金刚石也是晶体。

流体则不受到相互间作用力的束缚。所以微观图像是无序的,而且可以发生连续形变。其中,液体和气体可以由分子(或原子)的不同运动特征区分。液体的分子间距离较小,因此一个分子要跑到很远的外边去,必须要由充足的能量,而且要越过周围一圈的势垒,否则只能随周围的分子一起集体行动。气体分子则不同,分子的\textbf{平均自由程}较大,也就是说气体分子周围没有明显的势垒,其运动比液体要更加狂放。正因为气体的这一特点,才有著名的\textbf{麦克斯韦分布}麦克斯韦—玻尔兹曼分布\upref{MxwBzm}。