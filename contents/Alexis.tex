% 亚历克西·克洛德·克莱罗(综述)
% license CCBYSA3
% type Wiki

本文根据 CC-BY-SA 协议转载翻译自维基百科\href{https://en.wikipedia.org/wiki/Alexis_Clairaut}{相关文章}。

\begin{figure}[ht]
\centering
\includegraphics[width=6cm]{./figures/d23fa39a2d6e39da.png}
\caption{阿列克西·克劳德·克莱罗} \label{fig_Alexis_1}
\end{figure}
阿列克西·克劳德·克莱罗(Alexis Claude Clairaut,发音:/ˈklɛəroʊ/;法语:[alɛksi klod klɛʁo];1713年5月13日-1765年5月17日)是法国数学家、天文学家和地球物理学家。他是杰出的牛顿主义者,他的工作帮助确立了艾萨克·牛顿在1687年《自然哲学的数学原理》中所阐述的原则和结果的有效性。克莱罗是前往拉普兰的远征的关键人物之一,该远征帮助确认了牛顿关于地球形状的理论。在这一背景下,克莱罗推导出一个数学结果,现称为“克莱罗定理”。他还解决了引力三体问题,是第一个成功地得出月球轨道近日点进动的令人满意结果的人。在数学上,他还因提出克莱罗方程和克莱罗关系而闻名。
\subsection{传记}  
\subsubsection{童年与早期生活}  
克莱罗出生于法国巴黎,父母是让-巴蒂斯特·克莱罗和凯瑟琳·佩蒂。夫妇二人共育有20个孩子,但只有少数几个存活下来。他的父亲是数学教师。克莱罗自幼便表现出惊人的天赋——十岁时他便开始学习微积分。十二岁时,他撰写了一篇关于四种几何曲线的论文,并在父亲的指导下,在这一领域取得了飞速进展,到了十三岁时,他便在法国科学院前宣读了自己发现的四种曲线的性质。仅仅十六岁时,他便完成了一部关于曲折曲线的论文《关于双曲度曲线的研究》(Recherches sur les courbes à double courbure),该论文于1731年出版,使他得以被接纳为皇家科学院的会员,尽管当时他才十八岁,低于法定年龄。他还提出了一种突破性的公式——距离公式,用于计算笛卡尔坐标系或XY平面上任意两点之间的距离。
\subsubsection{个人生活与去世}  
克莱罗终身未婚,且以活跃的社交生活而著称。他在社会上的日益声望影响了他的科学工作:“他专注于宴会和夜晚, coupled with a lively taste for women,并试图将自己的享乐融入日常工作中,结果他失去了休息、健康,最终在五十二岁时丧命。”尽管他过着充实的社交生活,但他在推动年轻数学家的学术进步方面具有重要影响。

他于1737年10月27日当选为伦敦皇家学会会员。

克莱罗于1765年在巴黎去世。
\subsection{数学与科学著作}  
\subsubsection{地球的形状}  
1736年,克莱罗与皮埃尔·路易·莫佩尔图一起参加了前往拉普兰的远征,目的是估算经度弧度。[5] 这次旅行的目标是几何计算地球的形状,正如艾萨克·牛顿在其《自然哲学的数学原理》一书中所提出的,牛顿假设地球呈椭球形。他们试图验证牛顿的理论和计算是否正确。在远征队返回巴黎之前,克莱罗将他的计算结果送往伦敦皇家学会。该论文后来被学会刊登在1736-37年卷的《哲学汇刊》上。[6] 起初,克莱罗不同意牛顿关于地球形状的理论。在这篇文章中,他提出了几个关键问题,实际上推翻了牛顿的计算,并提供了一些解决这些问题的方法。讨论的内容包括计算引力、椭球体绕其轴旋转以及椭球体各轴的密度差异。[6] 在信的最后,克莱罗写道:

“即便是艾萨克·牛顿也认为,为了使地球在两极更扁平,地球必须在中心更为密集;并且,从这种更大的扁平性中可以推断出,重力从赤道向极地增加得更多。”[6]
\begin{figure}[ht]
\centering
\includegraphics[width=6cm]{./figures/3fa0a9bb0a54295e.png}
\caption{《地球形状的理论,基于流体静力学原理》,1743年} \label{fig_Alexis_2}
\end{figure}
这一结论不仅表明地球是一个扁球形椭球体,还表明它在两极处更为扁平,且在中心处较宽。他在《哲学汇刊》上的文章引发了许多争议,因为他讨论了牛顿理论中的问题,但并未提供太多关于如何修正计算的解决方案。返回后,他出版了他的论文《地球形状的理论》(1743年)。在这部著作中,他阐述了被称为“克莱罗定理”的定理,该定理将旋转椭球体表面点的重力与赤道的压缩力和离心力联系起来。这个地球形状的流体静力学模型是建立在苏格兰数学家科林·麦克劳林的论文基础上的,麦克劳林的论文表明,围绕其质心中心旋转的均匀流体团在粒子间的相互吸引作用下,会形成一个椭球体。假设地球由均匀密度的同心椭球壳组成,克莱罗定理可以应用于此,并允许从表面重力的测量中计算出地球的椭圆率。这证明了艾萨克·牛顿的理论,即地球的形状是一个扁球形椭球体。[2] 1849年,乔治·斯托克斯证明了无论地球内部的组成或密度如何,只要地球表面是一个小椭圆率的平衡球面,克莱罗的结果依然成立。
\subsubsection{几何学}  
1741年,克莱罗写了一本名为《几何学原理》(Éléments de Géométrie)的书。该书概述了几何学的基本概念。18世纪的几何学对普通学习者来说是复杂的,被认为是一个枯燥的学科。克莱罗看到了这一趋势,并写这本书试图让这一学科对普通学习者更加有趣。他认为,与其让学生反复做一些他们并不完全理解的题目,不如让他们通过主动的、体验式的学习,亲自去发现知识。[7] 他在书的开头通过将几何形状与土地测量进行比较,因为土地测量是大多数人都能理解的主题。他涵盖了从线条、形状到一些三维物体的内容。在全书中,他不断将物理学、天文学和其他数学分支与几何学进行联系。书中概述的一些理论和学习方法,今天仍然被教师们在几何学和其他学科中使用。[8]
\subsubsection{聚焦天体运动}  
18世纪最具争议的问题之一是三体问题,即地球、月球和太阳如何相互吸引。借助于新成立的莱布尼茨微积分,克莱罗能够通过四个微分方程解决这一问题。[9] 他还将牛顿的反比平方定律和引力定律纳入到他的解答中,虽然对其做了些许修改。然而,这些方程仅提供了近似的测量结果,并没有精确的计算。三体问题仍然存在一个未解的难题:月球在其近日点和远日点的旋转问题。即使是牛顿也只能解释月球近日点运动的一半。[9] 这个问题困扰了天文学家。实际上,克莱罗起初认为这个难题如此难以解释,以至于他几乎准备发表一个新的引力定律假设。
\begin{figure}[ht]
\centering
\includegraphics[width=6cm]{./figures/aa396e702b29b6c5.png}
\caption{} \label{fig_Alexis_3}
\end{figure}
关于近日点和远日点的问题在欧洲成为了一个激烈的辩论话题。除了克莱罗,还有两位数学家在争先恐后地提供三体问题的首个解释:莱昂哈德·欧拉和让·勒朗·达朗贝尔。[9] 欧拉和达朗贝尔反对使用牛顿定律来解决三体问题。特别是欧拉认为,反比平方定律需要修订,才能准确计算月球的近日点和远日点。

尽管在竞相提出正确解决方案的激烈竞争中,克莱罗还是成功地找到了三体问题的巧妙近似解。1750年,他凭借论文《月球理论》(Théorie de la lune)获得了圣彼得堡科学院的奖项;由克莱罗、热罗姆·拉朗德和尼科尔·雷娜·勒波特组成的团队成功计算出哈雷彗星1759年回归的日期。[10] 《月球理论》严格遵循牛顿学说,书中解释了近日点和远日点的运动。他想到将近似计算扩展到三阶,结果发现这一计算结果与观察数据一致。接着,在1754年,他使用离散傅里叶变换的形式计算了几个月球表的表格。[11]

解决三体问题的新发现不仅意味着证明牛顿定律是正确的,它还具有实际重要性。三体问题的破解对于航海有着重要的意义。它使得水手能够确定船只的经度方向,这不仅对航行到目的地至关重要,也对找到回家的路同样重要。[9] 这也具有经济意义,因为水手们可以更容易地根据经度测量找到贸易目的地。

此后,克莱罗撰写了多篇关于月球轨道和彗星运动的论文,尤其是关于哈雷彗星轨迹的研究,这些运动受行星的摄动影响。他还运用应用数学研究金星,精确测量了金星的大小和与地球的距离。这是首次对金星大小的精确计算。
\subsection{出版物}  
\begin{itemize}
\item 《地球形状理论,基于静力学原理》(法文)。巴黎:劳伦特·迪兰,1743年。  
\item 《地球形状理论,基于静力学原理》(法文)。巴黎:路易·库尔西耶,1808年
\end{itemize}。
\begin{figure}[ht]
\centering
\includegraphics[width=6cm]{./figures/8e19e8e94e7fcb14.png}
\caption{1743年版《地球形状理论,基于静力学原理》} \label{fig_Alexis_4}
\end{figure}
\begin{figure}[ht]
\centering
\includegraphics[width=6cm]{./figures/cfbb80c4ce7b3271.png}
\caption{《地球形状理论,基于静力学原理》导言} \label{fig_Alexis_5}
\end{figure}
\begin{figure}[ht]
\centering
\includegraphics[width=6cm]{./figures/3db48cd43ed0923e.png}
\caption{1765年版《月球理论与月球表》} \label{fig_Alexis_6}
\end{figure}
\begin{figure}[ht]
\centering
\includegraphics[width=6cm]{./figures/fafb7e2ec303cf03.png}
\caption{《月球理论与月球表》献词} \label{fig_Alexis_7}
\end{figure}
\begin{figure}[ht]
\centering
\includegraphics[width=6cm]{./figures/2cacff1821330d89.png}
\caption{《月球理论与月球表》献辞} \label{fig_Alexis_8}
\end{figure}
\begin{figure}[ht]
\centering
\includegraphics[width=6cm]{./figures/9382a4bd5e388b40.png}
\caption{《月球理论与月球表》第一页} \label{fig_Alexis_9}
\end{figure}
\subsection{参见}  
\begin{itemize}
\item 克莱罗方程  
\item 克莱罗关系  
\item 克莱罗定理  
\item 微分几何  
\item 人类计算机  
\item 分子间力  
\item 二阶导数的对称性
\end{itemize}
\subsection{注释}  
\begin{enumerate}
\item 其他日期也曾被提议过,比如5月7日,Judson Knight 和皇家学会的报告中提到过这个日期。这里有一段关于5月13日的讨论和辩论。  
\item Courcelle, Olivier (2007年3月17日)。 "13 mai 1713(1): Naissance de Clairaut"。*Chronologie de la vie de Clairaut (1713-1765)*(法文)。于2018年4月26日检索。  
\item Knight, Judson (2000年)。 "Alexis Claude Clairaut"。在Schlager, Neil;Lauer, Josh(编辑)。*Science and Its Times, Vol. 4: 1700-1799*。第247-248页。于2018年4月26日检索。  
\item Taner Kiral, Jonathan Murdock 和 Colin B. P. McKinney。 "The Four Curves of Alexis Clairaut"。MAA出版物。  
\item "Fellow Details: Clairaut; Alexis Claude (1713 - 1765)"。皇家学会。于2019年7月23日存档。于2018年4月26日检索。  
\item O'Connor 和 J. J.; E. F. Robertson(1998年10月)。"Alexis Clairaut"。*MacTutor History of Mathematics Archive*。圣安德鲁斯大学数学与统计系,苏格兰。于2009年3月12日检索。  
\item Claude, Alexis; Colson, John (1737年)。"An Inquiry concerning the Figure of Such Planets as Revolve about an Axis, Supposing the Density Continually to Vary, from the Centre towards the Surface"。*Philosophical Transactions*。40: 277–306。doi:10.1098/rstl.1737.0045。JSTOR 103921。  
Clairaut, Alexis Claude (1881年1月1日)。*Elements of geometry*,由J. Kaines翻译。  
\item Smith, David (1921年)。"Review of Èléments de Géométrie. 2 vols"。*The Mathematics Teacher*。  
\item Bodenmann, Siegfried (2010年1月)。 "The 18th century battle over lunar motion"。*Physics Today*。63(1):27–32。Bibcode:2010PhT....63a..27B。doi:10.1063/1.3293410。  
Grier, David Alan (2005年)。 "The First Anticipated Return: Halley's \item Comet 1758"。*When Computers Were Human*。普林斯顿:普林斯顿大学出版社。第11-25页。ISBN 0-691-09157-9。  
\item Terras, Audrey (1999年)。*Fourier analysis on finite groups and applications*。剑桥大学出版社。ISBN 978-0-521-45718-7,第30页。
\end{enumerate}
\subsection{参考文献}  
\begin{itemize}
\item Grier, David Alan,《When Computers Were Human》,普林斯顿大学出版社,2005年。ISBN 0-691-09157-9。  
\item Casey, J.,"Clairaut's Hydrostatics: A Study in Contrast",*American Journal of Physics*,第60卷,1992年,第549-554页。
\end{itemize}
\subsection{外部链接}
\begin{itemize}
\item 《克莱罗生平年表(1713-1765)》  
\item O'Connor, John J.; Robertson, Edmund F.,《Alexis Clairaut》,*MacTutor数学史档案*,圣安德鲁斯大学  
\item W.W. Rouse Ball,《数学史简述》
\end{itemize}