% x86-64 笔记
% license Xiao
% type Note

\begin{issues}
\issueDraft
\end{issues}

\footnote{参考 Wikipedia \href{https://en.wikipedia.org/wiki/X86-64}{相关页面}。}\textbf{x86 指令集(instruction set)}, 或者 \textbf{x86 指令集架构(ISA, instruction set architecture)}, 分为 32 位和 64 位两个版本。 现在最常见的是 64 位, 称为 \textbf{x86-64} (也叫 \textbf{x86_64}, \textbf{x64}, \textbf{amd64}, 和 \textbf{intel64})。 比较老的 32 位版本, 称为 \textbf{IA-32} (Intel Architecture, 32-bit, 也叫 \textbf{i386})。

另见\enref{汇编语言笔记(GAS, x86-64)}{AsmNt}”。

64 位 x86 CPU 可以运行为 32 位 x86 编译的二进制程序。 这是因为 64 位 x86 CPU 的兼容模式,允许它们运行 32 位软件而没有问题。 然而,操作系统也必须支持 32 位应用程序。大多数现代的 64 操作系统,包括Windows、Linux 和 macOS,都支持运行 32 位应用程序。在 Linux 中,这可能需要安装额外的 32 位支持库。

常见指令集包括
\begin{itemize}
\item \verb`MOV`
\item \verb`PUSH`, \verb`POP`
\item \verb`XCHG AX, BX`
\end{itemize}

\begin{itemize}
\item \verb`ADD`,\verb`SUB`
\item \verb`MUL`, \verb`IMUL`
\item \verb`DIV`,\verb`IDIV`
\end{itemize}

\begin{itemize}
\item \verb`AND`,\verb`OR`
\item \verb`XOR`,\verb`NOR`
\end{itemize}

\begin{itemize}
\item \verb`CMP`
\item \verb`JMP`,\verb`JZ`,\verb`JNZ`
\item \verb`CALL`
\item \verb`RET`
\end{itemize}

\begin{itemize}
\item \verb`MOVSB`,\verb`MOVSW`
CMPSB,CMPSW
LODSB,LODSW
STOSB,STOSW
\end{itemize}
