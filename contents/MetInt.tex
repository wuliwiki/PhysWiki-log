% 金属材料结构(科普)

\footnote{本文内容与图片主要参考自刘智恩的《材料科学基础》
(\autoref{MetInt_fig12} ),
Callister的 Material Science and Engineering An Introduction
(\autoref{MetInt_fig18}  \autoref{MetInt_fig3} \autoref{MetInt_fig4} \autoref{MetInt_fig6} \autoref{MetInt_fig7} \autoref{MetInt_fig9} \autoref{MetInt_fig11} \autoref{MetInt_fig16} ),
朱文涛的《简明物理化学》
(\autoref{MetInt_fig17} ),
以及小组课程作业
(\autoref{MetInt_fig13} ). 
部分文本、图片来自网络.本文无协议冲突的部分适用于CC-BY-SA,否则相应部分遵循原协定.}

“这个东西是怎么组成的?”这恐怕是每一个好奇的小朋友都会问的问题.本文将由小至大,从原子、晶胞、晶体,缺陷,微结构,外表面等方面,简要介绍金属材料的基本结构.
\begin{figure}[ht]
\centering
\includegraphics[width=14cm]{./figures/MetInt_18.png}
\caption{本文的简要目录} \label{MetInt_fig18}
\end{figure}

\subsection{原子、晶体、晶胞}
如果你的视力\footnote{可见光的波长(约为300-700nm)远大于此,因此凭光学显微镜是不可能看到如此细小的结构的.这也是为什么我们要发明电子显微镜)}好到可以看见纳米级别(大概$10^{-9}m =10^{-6} mm$)的金属结构,那么你会发现金属不仅是由大量原子组成的,而且好像是由大量原子层层叠叠、有序堆积而成的.堆积的具体方式与金属的种类\footnote{有些金属的堆积方式不止一种,与温度、压力等有关,互称为同素异形体.例如Fe,Ti等就具有多个同素异形体}有关.

\begin{figure}[ht]
\centering
\includegraphics[width=5cm]{./figures/MetInt_1.png}
\caption{晶体中原子的排列示意图} \label{MetInt_fig1}
\end{figure}

\begin{definition}{晶体}
原子(或分子、离子等)在三维空间按一定规律作周期性排列而形成的固体
\end{definition}

既然晶体中原子的排列是规律重复的,我们自然就能找出其中最小的一个重复单元,以反映这种排列方式的特征.这种最小单元被称为晶胞.
\begin{definition}{晶胞}
能够完全反应晶体几何特征的最小单元
\end{definition}
例如金属铁的晶胞是体心立方(BCC)结构.若把晶胞看作一个正方体,则其中心的一个原子与周围八个原子均相切,有点像这样:
\begin{figure}[ht]
\centering
\includegraphics[width=5cm]{./figures/MetInt_2.png}
\caption{铁的晶胞}} \label{MetInt_fig2}
\end{figure}

\subsection{缺陷}
如果金属中的原子都完全按这种理想的方式整整齐齐地排列,那材料科学也未免太无趣了(材料科学的书至少会薄半本!但同时,我们能用材料科学做的事也会少很多!).实际上,如果你的眼光放得长远一些,不只局限于几个原子的大小,就会发现真实的金属晶体往往存在原子排列偏离理想方式的区域,称为缺陷.

\begin{definition}{缺陷}
实际金属中原子偏离理想排列而出现的不完整区域
\end{definition}

\textbf{金属中缺陷的占比一般不高,但却对材料的性能起到了决定性影响}.“缺陷”这个词往往让人以为缺陷百害而无一利,但事实并非如此,有些缺陷反而能提升材料某一方面的性能(\textsl{但代价是什么呢?}).夸张点说,半个材料科学与工程的研究目的,就是理解并利用缺陷,以设计并制造符合需求的材料.

根据缺陷的空间尺度,缺陷一般被分为点缺陷、线缺陷与面缺陷.

\subsubsection{点缺陷}
点缺陷指的是单独少数原子的错误排列.
\begin{figure}[ht]
\centering
\includegraphics[width=8cm]{./figures/MetInt_3.png}
\caption{空位缺陷与间隙缺陷} \label{MetInt_fig3}
\end{figure}
空位:原子离开了自己的理想位置,留下了一个空位

间隙:原子插入了本不应存在原子的位置,一般是晶体的间隙

有时,一些其他种类的原子也会混入到金属晶体之中
\begin{figure}[ht]
\centering
\includegraphics[width=8cm]{./figures/MetInt_4.png}
\caption{杂原子的置换与间隙} \label{MetInt_fig4}
\end{figure}

(杂原子)置换:其他种类的原子替换了格点上的原子

(杂原子)间隙:其他种类的原子插入了间隙之中

\subsubsection{线缺陷}
线缺陷又称为位错,可细分为刃位错、螺位错、以及二者的组合 混合位错.位错理论源自于材料力学性能的理论值与实际值过大的差异.位错相关理论于1930年代被提出,并在1950年代因在电子显微镜下实际观察到位错而得到证实.(某种意义上说,位错是一个新鲜的事物,特别当考虑到狭义相对论在1905年就被发现了!)

\begin{exercise}{科学史测试}
列举一些早于位错理论的科学成就.

(好怪的题目)
\end{exercise}

刃位错可以理解为完整晶体中插入了(或失去了)半个额外的原子面,或者说上部分额外多滑移了一个原子间距.
\begin{figure}[ht]
\centering
\includegraphics[width=10cm]{./figures/MetInt_6.png}
\caption{刃位错示意图1} \label{MetInt_fig6}
\end{figure}
\begin{figure}[ht]
\centering
\includegraphics[width=10cm]{./figures/MetInt_5.png}
\caption{刃位错示意图2} \label{MetInt_fig5}
\end{figure}

螺位错也可以理解为上半部分相对于下半部分额外侧滑移的结果,但是滑移的方向与刃位错不一致.
\begin{figure}[ht]
\centering
\includegraphics[width=10cm]{./figures/MetInt_7.png}
\caption{螺位错示意图1} \label{MetInt_fig7}
\end{figure}
\begin{figure}[ht]
\centering
\includegraphics[width=10cm]{./figures/MetInt_8.png}
\caption{螺位错示意图2} \label{MetInt_fig8}
\end{figure}

总之,位错使材料的两部分并不完整对齐.更抽象、数学地描述位错时,我们往往使用位错线、滑移面、burgers矢量等概念,不过这部分的枯燥内容暂时按下不谈.

% \subsubsection{面缺陷}
% 面缺陷包括金属的外表面(就是你能看到的那部分)、晶界、相界等等,后者将在下一节简要讨论.

\subsubsection{缺陷与材料性质}
缺陷对材料的机械力学、热力学、化学乃至电学性能等都有深刻影响,可谓遇事不决缺陷背锅(?).

此处,我们先简要介绍一下缺陷对材料热力学性能的影响.以置换点缺陷为例.如果一个大的杂原子置换了原本的小原子,那会发生什么呢?这就好像你在一台电梯中,好巧不巧又挤进来一位善良的胖子.那么你会感觉到一股更大的压力.事实上原子也是如此,大原子挤压了相邻的其余原子,并造成了额外的压力.可见,点缺陷在其周围形成了额外的力场,·并提升了系统的总能量.
\begin{figure}[ht]
\centering
\includegraphics[width=5cm]{./figures/MetInt_9.png}
\caption{置换大原子造成了内压应力} \label{MetInt_fig9}
\end{figure}

这个结论可以推广至其他种类的缺陷(想想看,位错产生的力场是什么样的?),即\textbf{缺陷提升了系统的总能量};此外,大多数的缺陷(空位点缺陷是一个特例)还\textbf{提高了材料的自由能}\footnote{自由能是一个相对抽象的热力学概念\upref{GibbsG},你可以简单地认为自由能是系统能量与混乱度的综合考量.系统能量越高、混乱度越低,系统的自由能也就越高;系统偏向于发生降低自由能的过程,即降低能量、升高混乱度.}. 这似乎意味着\textsl{缺陷是热力学不稳定的}.现实中,材料经由(传统手段)缓慢冷却,确实能得到较为完美的晶体;然而由于材料中存在大量形成缺陷的机制等动力学因素,仍无法完全消除缺陷.

\begin{example}{鹤立鸡群的空位点缺陷}
根据热力学定律,一个稳定的系统需要同时满足熵最大与能量最低,即自由能最低.

当每个原子都安安静静乖乖巧巧地呆在自己的位置上时,尽管此时系统能量最低,但此时的系统非常有序,熵不高,自由能不处于最低.因此,总会有一些原子自发离去、留下空位,以升高系统的熵并降低自由能.这就是为什么空位点缺陷反而是热力学稳定的.
\end{example}

\begin{example}{冷加工金属不适用于高温环境}
冷加工(这是强化金属的一种方法,即在低温下变形材料,使金属的强度更高,但塑形更差)后,金属内的位错含量大幅升高,金属处于热力学不稳定状态.

这使金属的回复温度(在较高的温度下,金属自发减少缺陷并降低机械强度)更低、速度更快.在高温下,随着金属的回复,机械性能将下降.
\end{example}

可以预见,由于缺陷升高了金属的能量与自由能,金属的化学性质将更为活泼,也就更可能参与化学反应、被腐蚀等等等等.

如果你还了解电阻的经典微观模型\upref{Resist},就知道电阻来自于电子与金属中各类缺陷的碰撞.因此,缺陷数量上升,金属的导电性也就下降,金属的电阻也增大.

\subsection{微结构}
见完了缺陷的小打小闹,是时候继续调高眼界、往大的看了(大致是光学显微镜级别).在更广阔的视野下,你就会看见...更大的缺陷.这类在光学显微镜下可见的结构可以被称为微结构,有时也称“组织”.微结构的种类繁多,这里主要举一些典型的例子\footnote{这些晶界、孪晶界、相界等,以及下文所介绍的外表面,有时也被认为是面缺陷.}.

\subsubsection{晶粒与晶界}
或许你还对\autoref{MetInt_fig1} (提示:原子的规则排列)记忆犹新.在整块金属中,原子还是老老实实规规矩矩地沿同一个方向排列吗?答案当然是...否定的.实际中的原子排列可能更像这样:
\begin{figure}[ht]
\centering
\includegraphics[width=5cm]{./figures/MetInt_10.png}
\caption{Atomsk随机生成的晶粒} \label{MetInt_fig10}
\end{figure}
在一定区域内,金属原子的排列位向相同;但在不同区域内,原子的排列位向就不相同了.这样,天然存在一个边界划分这些区域.这些边界被称为晶界,而晶界所围成的区域称为晶胞.换句话说,金属整体可以看作是由一颗颗晶粒构成的.
\begin{figure}[ht]
\centering
\includegraphics[width=10cm]{./figures/MetInt_11.png}
\caption{晶界示意图,注意晶界两侧晶粒位向的不同} \label{MetInt_fig11}
\end{figure}

\begin{figure}[ht]
\centering
\includegraphics[width=10cm]{./figures/MetInt_13.png}
\caption{光学显微镜下的铁($0.45 wt\%$ C)} \label{MetInt_fig13}
\end{figure}

\subsubsection{孪晶界}
如果晶界两侧的晶体呈对称关系,那么这样的晶界被称为孪晶界.
\begin{figure}[ht]
\centering
\includegraphics[width=10cm]{./figures/MetInt_12.png}
\caption{孪晶界} \label{MetInt_fig12}
\end{figure}

\subsubsection{相界}
相界是一种更复杂的微结构,需要你对相的概念有所理解.不过简言之,相可以被如下定义:
\begin{definition}{相(基础材料科学)}
材料中结构、性质、成分相同或相近的一部分被称为一相;各相间有明显的边界.
\end{definition}

\begin{example}{图中有几个相含有水?}
\footnote{老师上课举的这个例子令我记忆犹深...}
\begin{figure}[ht]
\centering
\includegraphics[width=10cm]{./figures/MetInt_14.png}
\caption{(材料科学的)冰山一角} \label{MetInt_fig14}
\end{figure}
答案:3个

水的固相(冰川)、液相(大海)和气相(空气含有水蒸气)
\end{example}

\textbf{事实上,一个材料可以包括多个相},即一个宏观完整的材料可以由几种结构、成分、性质等都截然不同的部分组成.\textsl{听起来很令人震惊,是吧}.

\begin{example}{中、低碳钢}
常用的中、低碳钢\footnote{实际的钢铁往往包括其他元素,并且随热处理工艺不同,相组成可能变化.不过先暂时忽略这些有的没的...}实际上是两种组分的机械混合物,铁素体($\alpha$)与渗碳体($Fe_3C$)

铁素体的结构类似于\autoref{MetInt_fig1} ,而渗碳体的结构相当复杂,类似于下图
\begin{figure}[ht]
\centering
\includegraphics[width=5cm]{./figures/MetInt_15.png}
\caption{渗碳体示意图.原始结构数据来自materialsproject.org} \label{MetInt_fig15}
\end{figure}
二者往往形成具有特色的珠光体结构.\footnote{该结构的形成与铁水冷却过程中的热力学与动力学因素有关,在此按下不表.}
\begin{figure}[ht]
\centering
\includegraphics[width=8cm]{./figures/MetInt_16.png}
\caption{珠光体示意图} \label{MetInt_fig16}
\end{figure}
红蓝晶粒间的边界即为相界.
\end{example}

\subsection{宏观材料}
\subsubsection{外表面}
如果你还没有在材料科学的海洋里溺水的话,那么恭喜你,你现在回到了正常尺度的世界,并且...迎来了本篇文章的最“大”问题:材料的外表面.材料的外表面是材料与空气接触的地方,也是你目所能及之处.某种意义上说,材料的外表面有点像一类特殊的相界.

为了进一步分析表面性质,我们设计一个简单模型:
\begin{figure}[ht]
\centering
\includegraphics[width=6cm]{./figures/MetInt_17.png}
\caption{材料内部与表面原子的受力示意图} \label{MetInt_fig17}
\end{figure}
金属内部的原子,四周受到来自其他金属原子的作用力大致相同,因此内部的原子处于受力平衡状态;而表面附近的原子,情况变得微妙了起来:这部分原子的上方只有空气.显然空气对金属原子的作用力远远不及其他金属原子,因此可以认为,这部分原子受到一个向下的合力.

根据该模型稍加思索(或者看看表面张力\upref{sftens}找找灵感),我们得到表面的一些性质:表面存在表面张力,表面张力来自于表面原子的受力不均匀,且物质总是试图收缩他的表面;\textbf{表面提高了系统的能量与自由能}.

表面\textsl{也}对材料的性质起到重要影响.例如,由于表面的能量高,化学反应(腐蚀)、扩散与渗透等更容易在表面发生.这些结论同样适用于晶界与相界.

\begin{example}{表面处理}
表面处理是材料科学与工程的又一重要话题.

因为表面的高能,因此经常需要处理表面以保护材料,例如喷漆、镀膜等.

同时,表面的高能性质也允许我们设计更复杂的表面.例如齿轮需要坚硬耐磨的表面与有韧性的内部,而一般的高碳钢却过于硬而脆(韧的反义词).为了解决这一矛盾,我们便利用表面容易扩散的性质,热处理材料时将少量碳扩散进齿轮的表面(渗碳处理).这样既硬化了表面,又保持了内部的韧性.
\end{example}

现在,你对材料科学的理解,是不是比过去更深入了一些?(我希望你的答案是“是的”!:))
