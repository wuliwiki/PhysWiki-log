% Hartree-Fock 近似
% keys Hartree-Fock近似
% license Xiao
% type Tutor
\pentry{二次量子化\upref{SecQua}}

Hartree-Fock 近似理论可以用来近似描述相互作用电子气,其核心在于平均场思想。

写出库伦相互作用电子气的哈密顿量:

$$H=\sum\limits_{i}\langle i|H_0|i\rangle c_{i,\sigma}^\dagger c_{i}+\frac{1}{2}\sum\limits_{\mu ,\nu,\mu', \nu'}V_{\mu \nu,\mu' \nu'}c_{\mu'}^\dagger c_{\nu'}^\dagger c_{\nu} c_{\mu}~. $$
上式中的下表$\mu,\nu$等包括了粒子的自旋信息,值得注意的是我们考虑的是库伦相互作用,并不作用于自旋,所以相互作用项中的四算符部分里$\mu'$和$\mu$、$\nu'$和$\nu$表示的是同一个粒子散射前后的状态,其自选应该是相同的。

但四算符的难以处理的,我们使用平均场手段将其近似为二算符。我们需要关注四算符的期待值:
$$
\langle n_\mu, n_\nu |c_{\mu'}^\dagger c_{\nu'}^\dagger c_{\nu} c_{\mu} |n_\mu ,n_\nu \rangle = \langle n_\mu, n_\nu |c_{\mu'}^\dagger c_{\nu'}^\dagger |n_\mu-1 ,n_\nu-1 \rangle~.
$$
可以看到仅仅在$\mu=\mu',\nu=\nu'$或$\mu=\nu',\nu=\mu'$时上式非0:
$$\langle n_\mu, n_\nu |c_{\mu'}^\dagger c_{\nu'}^\dagger c_{\nu} c_{\mu} |n_\mu ,n_\nu \rangle = \left(\delta_{\mu\mu'}\delta_{\nu\nu'}\pm \delta_{\mu\nu'}\delta_{\nu\mu'}\right)n_\mu n_\nu~.$$
其中正负号来自于考虑的系统是费米子还是玻色子,费米子的反对易关系会带来一个负号,电子系统是费米子,所以我们需要的是负号的情况。所以在基态下四算符的平均值是:
\begin{equation}\label{eq_HFock_1}
\langle c_{\mu'}^\dagger c_{\nu'}^\dagger c_{\nu} c_{\mu} \rangle = \left(\delta_{\mu\mu'}\delta_{\nu\nu'}\pm \delta_{\mu\nu'}\delta_{\nu\mu'}\right)\langle n_\mu\rangle \langle n_\nu\rangle~.
\end{equation}
这其实是Wick 定理(标量场)\upref{wick}的一个结论。
考虑原式中的相互作用项期望值可以发现其变为两项:

$$\frac{1}{2}\sum\limits_{\mu ,\nu,\mu', \nu'}V_{\mu \nu,\mu' \nu'}\langle c_{\mu'}^\dagger c_{\nu'}^\dagger c_{\nu} c_{\mu}\rangle=\frac{1}{2}\sum\limits_{\mu,\nu}V_{\mu\nu,\mu\nu}\langle n_\mu\rangle \langle n_\nu\rangle-\frac{1}{2}\sum\limits_{\mu,\nu}V_{\mu\nu,\nu\mu}\langle n_\mu\rangle \langle n_\nu\rangle~.$$
上式中第一项被称为Hartree项,第二项被称为Fock项。

其中:
$$V_{\mu\nu,\mu\nu}=\iint \phi_\mu^*(r_1)\phi_\nu^*(r_2)\frac{e^2}{\abs{r_1-r_2}^2}\phi_\mu(r_1)\phi_\nu(r_2)dr_1dr_2=\iint \frac{e^2\abs{\phi_\mu(r_1)}^2\abs{\phi_\nu(r_2)}^2}{\abs{r_1-r_2}^2}dr_1dr_2~.$$
$$V_{\mu\nu,\nu\mu}=\iint \phi_\mu^*(r_1)\phi_\nu^*(r_2)\frac{e^2}{\abs{r_1-r_2}^2}\phi_\mu(r_2)\phi_\nu(r_1)dr_1dr_2~.$$

为了写出具体的哈密顿量形式,我们先不写出$\delta$项:
\begin{equation}\label{eq_HFock_2}
\langle c_{\mu'}^\dagger c_{\nu'}^\dagger c_{\nu} c_{\mu} \rangle = \langle c_{\mu'}^\dagger c_\mu\rangle\langle c_{\nu'}^\dagger c_\nu\rangle-\langle c_{\nu'}^\dagger c_\mu\rangle\langle c_{\mu'}^\dagger c_\nu\rangle~.
\end{equation}
可以验证\autoref{eq_HFock_1} 与\autoref{eq_HFock_2} 相同。

并且做如下变换:
$$\langle c_{\mu'}^\dagger c_\mu\rangle\langle c_{\nu'}^\dagger c_\nu\rangle\rightrightarrows c_{\mu'}^\dagger c_\mu\langle c_{\nu'}^\dagger c_\nu\rangle+\langle c_{\mu'}^\dagger c_\mu\rangle c_{\nu'}^\dagger c_\nu-\langle c_{\mu'}^\dagger c_\mu\rangle\langle c_{\nu'}^\dagger c_\nu\rangle ~.$$

可以验证上式左右期望值相同。
那么hartree项将变为:

\begin{equation}
\begin{aligned}
V^{int}_{hartree}&=\frac{1}{2}\sum\limits_{\mu,\nu,\mu',\nu'}V_{\mu\nu,\mu'\nu'}\left(c_{\mu'}^\dagger c_\mu\langle c_{\nu'}^\dagger c_\nu\rangle+\langle c_{\mu'}^\dagger c_\mu\rangle c_{\nu'}^\dagger c_\nu-\langle c_{\mu'}^\dagger c_\mu\rangle\langle c_{\nu'}^\dagger c_\nu\rangle\right)~.
\end{aligned}
\end{equation}

Fock项会变为:
\begin{equation}
\begin{aligned}
V^{int}_{hartree}&=-\frac{1}{2}\sum\limits_{\mu,\nu,\mu',\nu'}V_{\mu\nu,\nu'\mu'}\left(c_{\nu'}^\dagger c_\mu\langle c_{\mu'}^\dagger c_\nu\rangle+\langle c_{\nu'}^\dagger c_\mu\rangle c_{\mu'}^\dagger c_\nu-\langle c_{\nu'}^\dagger c_\mu\rangle\langle c_{\mu'}^\dagger c_\nu\rangle\right)~.
\end{aligned}
\end{equation}

这样$V^{int}$就从四算符项变成了二算符项。值得注意的是Hartree项和Fock项虽然看似相同,但存在根本性的区别。

Hartree项是直接的库仑相互作用,而Fock项则是交换相互作用,他体现了粒子“交换状态”这一过程,这是一个纯粹的量子力学带来的效应。

回想本节最初所说,$\mu$和$\mu'$是同一个粒子的两种状态,$\nu$和$\nu'$是另一个粒子的两种状态,而在库仑相互作用中散射前后并不改变粒子的自旋,所以$\mu'$和$\mu$的自旋相同,$\nu$和$\nu'$的自旋相同。但$\mu$和$\nu$的自旋可以不同,这使得Hartree项中产生湮灭算符对应的态的自旋是始终相同的,而Fock项的产生湮灭算符对应的自旋是可以不同的。
