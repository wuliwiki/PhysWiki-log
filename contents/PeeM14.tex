% 2014 年考研数学试题(数学一)
% keys 考研|数学
% license CCBYNCSA4
% type Tutor

\subsection{选择题}
\begin{enumerate}
\item 下列曲线中有渐近线的是 $(\quad)$\\
(A) $y=x+\sin x$\\
(B) $y=x^2+\sin x$\\
(C) $y=x+\sin \frac{1}{x}$\\
(D) $y=x^2+\sin \frac{1}{x}$
\item  设函数 $f(x)$ 具有2阶导数,$g(x)=f(0)(1-x)+f(1)x$  ,则在区间  [0,1] 上 $(\quad)$\\
(A)当 $f'(x) \ge 0$ 时, $f(x)\ge g(x)$\\
(B)当 $f'(x) \ge 0$ 时, $f(x)\le g(x)$\\
(C)当 $f''(x) \ge 0$ 时, $f(x)\ge g(x)$\\
(D)当 $f''(x) \ge 0$ 时, $f(x)\le g(x)$
\item 设 $f(x)$ 是连续函数,则 $\displaystyle \int_0^1 \dd{y}\int_{-\sqrt{1-y^2}}^{1-y}f(x,y)\dd{x}$=$(\quad)$\\
(A) $\displaystyle \int_0^1\dd{x}\int_0^{x-1}f(x,y)\dd{y}+\int_{-1}^0\dd{x}\int_0^{\sqrt{1-x^2}}f(x,y)\dd{y}$\\
(B)$\displaystyle \int_0^1\dd{x}\int_0^{1-x}f(x,y)\dd{y}+\int_{-1}^0\dd{x}\int_{-\sqrt{1-x^2}}^0f(x,y)\dd{y}$\\
(C) $\displaystyle \int_0^{\frac{\pi}{2}}\dd{\theta}\int_0^{\frac{1}{\cos \theta +\sin \theta}}f(r\cos \theta,r\sin \theta)\dd{r}+\int_{\frac{\pi}{2}}^\pi\dd{\theta}\int_0^1 f(r\cos \theta,r\sin \theta)\dd{r}$\\
(D) $\displaystyle \int_0^{\frac{\pi}{2}}\dd{\theta}\int_0^{\frac{1}{\cos \theta +\sin \theta}}f(r\cos \theta,r\sin \theta)r\dd{r}+\int_{\frac{\pi}{2}}^\pi\dd{\theta}\int_0^1 f(r\cos \theta,r\sin \theta)r\dd{r}$\\
\item 若 $\displaystyle \int_{-\pi}^\pi (x-a_1\cos x-b_1\sin x)^2\dd{x}=\min_{a,b \in R}\{\int_{-\pi}^\pi (x-a\cos x-b\sin x)^2\dd{x}\}$ ,则  $a_1\cos x+b_1\sin x$=$(\quad)$\\
(A) $2\sin x $\\
(B) $2\cos x$\\
(C) $2\pi \sin x$\\
(D)  $2\pi \cos x$

\item 行列式 $ \left|\begin{array}{llll}0 & a & b & 0 \\ a & 0 & 0 & b \\ 0 & c & d & 0 \\ c & 0 & 0 & d\end{array}\right|=(\quad)$\\
(A) $(ad-bc)^2$\\
(B) $-(ad-bc)^2$\\
(C) $a^2d^2-b^2c^2$\\
(D) $b^2c^2-a^2d^2$
\item 设 $a_1,a_2,a_3$ 均为3维向量,则对任意常数 $k,l$ 向量组 $a_1+ka_3,a_2+la_3$ 线性无关是向量组 $a_1,a_2,a_3$ 线性有关的 $(\quad)$\\
(A) 必要非充分条件\\
(B) 充分并非必要条件\\
(C) 充分必要条件\\
(D) 既非充分也非必要条件
\item 设随机事件 $A$ 与 $B$ 相互独立,且 $P(B)=0.5,P(A-B)=0.3$ ,则$P(B-A)=$ $(\quad)$\\
(A)0.1\\
(B)0.2\\
(C)0.3\\
(D)0.4
\item 设连续型随机变量 $X_1$ 与 $X_2$ 相互独立且方差均存在,$X_1$  与 $X_2$ 概率密度分别为 $f_1(x)$ 与 $f_2(x)$ 随机变量 $Y_1$ 的概率密度为 $f_{Y_1}(y)=\frac{1}{2}[f_1(y)+f_2(y)]$ ,随机变量 $Y_2=\frac{1}{2}(X_1+X_2)$ ,则$(\quad)$\\
(A) $E(Y_1)>E(Y_2),D(Y_1)>D(Y_2)$\\
(B) $E(Y_1)=E(Y_2),D(Y_1)=D(Y_2)$\\
(C) $E(Y_1)=E(Y_2),D(Y_1)<D(Y_2)$\\
(D) $E(Y_1)=E(Y_2),D(Y_1)>D(Y_2)$

\end{enumerate}
\subsection{填空题}
\begin{enumerate}
\item 曲面 $z=x^2(1-\sin y)+y^2(1-\sin x)$  在点 (1,0,1) 处的切平面方程为$(\quad)$
\item 设 $f(x)$ 是周期为4的可导奇函数,且 $f'(x)=2(x-1),x \in [0,2]$  则 $f(7)$=$(\quad)$
\item 微分方程 $xy'+y(\ln x-\ln y)=0$ 满足条件 $y(1)=e^3$ 的解为 $y=$ $(\quad)$
\item 设 $L$ 是柱面 $x^2+y^2=1$ 与平面 $y+z=0$ 的交线,从z轴正向往z轴负向看去为逆时针方向,则曲线积分 $\displaystyle \oint_L z\dd{x}+y\dd{z}$=$(\quad)$
\item 设二次型 $f(x_1,x_2,x_3)=x_1^2-x_2^2+2ax_1x_3+4x_2x_3$  的负惯性指数为1,则 $a$ 的取值范围是$(\quad)$。
\item 设总体 $X$ 的概率密度为$f(x;\theta)=\leftgroup{&\frac{2x}{3\theta^2},&\theta<x<2\theta \\ &0,&\text{其他}} $ ,其中 $\theta$  是未知参数,$X_1,X_2,\dots,X_n$  为来自总体 $X$的简单随机样本,若  $\displaystyle c\sum_{n=1}^\infty X_i^2$ 是 $\theta^2$ 的无偏差估计,则 $c=(\quad)$
\end{enumerate}
\subsection{解答题}
\begin{enumerate}
\item 求极限 $\displaystyle \lim \frac{\int_{1 }^{x} [t^2(e^{\frac{1}{t}-1})-t]\dd{t}}{x^2\ln(1+\frac{1}{x})}$.
\item 设函数 $y=f(x)$ 由方程 $y^3+xy+x^2y+6=0$ 确定,求 $f(x)$ 的极值。
\item 设函数 $f(u)$ 具有二阶连续倒数,$z=f(e^x\cos y)$满足 $\displaystyle \pdv[2]{z}{x}+\pdv[2]{z}{y}=(4z+e^x\cos y)e^{2x}$  。若 $f(0)=0,f'(0)=0$ ,求 $f(u)$ 的表达式。
\item  设 $\Sigma$为曲面 $z=x^2+y^2(z\le 1)$ 的上侧,计算曲面积分 $\displaystyle I=\iint_\Sigma(x-1)^3\dd{y}\dd{z}+(y-1)^3\dd{z}\dd{x}+(z-1)\dd{x}\dd{y}$
\item 设数列 $\{a_n\},\{b_n\}$满足 $\displaystyle 0<a_n<\frac{\pi}{2},0<b_n<\frac{\pi}{2},\cos a_n-a_n=\cos b_n$ ,且级数 $\displaystyle \sum_{n=1}^\infty b_n$  收敛。\\
(1)证明 $\displaystyle \lim_{n\to\infty} a_n=0;$\\
(2)证明级数 $\displaystyle \sum_{n=1}^\infty \frac{a_n}{b_n}$  收敛。
\item 设$A$= $\pmat{0&-2&3&-4\\0&1&-1&1\\1&2&0&-3}$ ,$ E$  为3阶单位矩阵。\\
(1)求方程组 $Ax=0$ 的一个基础解系;\\
(2)求满足 $AB=E$的所有矩阵 
\item 证明$n$阶矩阵 $\pmat{1&1&\dots&1\\1&1&\dots&1\\\vdots&\vdots&&\vdots\\1&1&\dots&1}$ 与 $\pmat{0&0&\dots&1\\0&0&\dots&2\\\vdots&\vdots&&\vdots\\0&0&\dots&n}$ 相似。
\item 设随机变量 $X$ 的概率分布为 $P{X=1}=P{X=2}=\frac{1}{2}$ 。在给定 $X=i$ 的条件下,随机变量 $Y$ 服从均匀分布 $U(0,i)(i=1,2)$ 。\\
(1)求 $Y$ 的分布函数 $F_Y(y)$;\\
(2)求 $E(Y)$.
\item 设总体 $X$ 的分布函数为 $\displaystyle F(x;\theta)=\leftgroup{&1-e^{-\frac{x^2}{\theta}},&x \ge 0\\&0, &x<0}$,其中 $\theta$ 是未知参数且大于零。$X_1,X_2,\dots,X_n$  为来自总体 $X$ 的简单随机样本。\\
(1)求 $E(X)$ 与$ E(X^2)$;\\
(2)求 $\theta$ 的最大似然估计量 $\Q \theta_n$;\\
(3)是否存在实数 $a$ 使得对任何$\varepsilon>0  $ ,都有 $\displaystyle \lim_{n\to\infty} P\{\abs{\Q \theta_n -a \ge \varepsilon} \}=0$?
\end{enumerate}
