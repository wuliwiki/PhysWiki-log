% 完备公理(戴德金分割)
% license Xiao
% type Tutor

\begin{issues}
\issueTODO
\issueOther{应当并入实数的完备公理\upref{RCompl}}
\end{issues}

\pentry{实数\upref{ReNum}}

从这里开始, 我们将不再区分有理数和由有理数确定的戴德金分割。 

通过戴德金分割, 我们得以填补有理数集中的"空隙", 从而得到了实数集。 纯粹从四则运算的角度, 还不能看出实数与有理数之间有何区别。 然而如果引入一些超出四则运算范围的操作, 即可看出实数与有理数的决定性区别: 实数集是\textbf{完备 (complete)} 的, 或者模糊地说, 实数集里"没有空隙".

\subsection{完备公理}

怎么刻画"不存在缝隙"这样的直观特性呢?

仍旧以数 $\sqrt{2}$ 为例。 我们记得这个数的下类是 $L=\{l\leq0\}\cup\{l>0:l^2<2\}$, 上类是 $R=\{r>0:r^2>2\}$. $L$ 和 $R$ 的并集就是有理数集 $\mathbb{Q}$, 而由于 2 并非任何有理数的平方, 因此 $L$ 和 $R$ 之间"空无一物", 分割 $L|R$ 并不是由一个实际存在的分点确定的。 但实数集却与此不同: 在区间 $(-\infty,\sqrt{2})$ 和 $(\sqrt{2},+\infty)$ 之间的确存在着一个分点 $\sqrt{2}$. 

一般来说, 如果像定义戴德金分割那样对实数集 $\mathbb{R}$ 进行操作, 那么得不到任何新的对象: \textbf{每一个这样的分割都必定是由一个分点确定的。} 实际上, 如果将实数集 $\mathbb{R}$ 分成不相交的两部分 $A\cup B$, 满足

\begin{enumerate}
\item 如果 $a\in A$, 那么任何小于 $a$ 的实数 $a'$ 都属于 $A$.
\item 如果 $b\in B$, 那么任何大于 $b$ 的实数 $b'$ 都属于 $B$.
\item 如果 $a\in A$, $b\in B$, 那么必有 $a\leq b$.
\item $A$ 不包含最大的元素;
\end{enumerate}

那么 $L=A\cap\mathbb{Q}$ 满足戴德金分割下类的定义, 从而 $R=B\cap\mathbb{Q}$ 自动满足分割上类的定义, 于是 $L|R$ 自动成为一个戴德金分割, 它自然就确定了一个实数 $x$. 这个实数 $x$ 自然就是"分割"$A\cup B$ 的分点, 它满足如下的性质:

\textbf{对于 $a\in A$, $b\in B$, 总有 $a\leq x\leq B$. 而且, 实际上更有 $A=(-\infty,x)$, $B=[x,+\infty)$.}

\begin{exercise}{}
证明这一点。 提示: 如果存在 $a\in A$ 使得 $a>x$, 那么开区间 $(x,a)$ 中存在有理数 $q$.
\end{exercise}

仿照这个思路, 我们便能证明如下的定理:

\begin{theorem}{实数的完备性}\label{the_Cmplt_1}
设 $A,B$ 是实数集的子集, 使得对于任意的 $a\in A$, $b\in B$ 都有 $a\leq b$. 那么存在实数 $x$ 使得对于任意的 $a\in A$, $b\in B$ 都有 $a\leq x\leq b$.
\end{theorem}

上面的这个性质"两个集合中间一定有一个元素"被称作\textbf{"完备公理" (the axiom of completeness)}.

\begin{exercise}{}
证明实数的完备性, 即定理 1. 提示: 考虑集合 $A$ 中所有元素的下类的并集, 它仍然还是一个下类。
\end{exercise}

\subsection{实数的公理刻画}

在得到了完备公理后, 我们便可以给出实数集合的公理刻画了。 

我们称一个带有二元运算 $+$, $\cdot$ 和序关系 $<$ 的集合 $\mathfrak{R}$ 为一个实数模型, 如果上述运算和序关系满足下列一组公理:

\subsubsection{阿贝尔群公理}
\begin{itemize}
\item 交换律: 对于任何 $x,y\in\mathfrak{R}$ 皆有 $x+y=y+x$.
\item 结合律: 对于任何 $x,y,z\in\mathfrak{R}$ 皆有 $x+(y+z)=(x+y)+z$.
\item 零元素: 存在一个元素 $0\in \mathfrak{R}$ 使得对于任何 $x,y\in\mathfrak{R}$ 皆有 $x+0=x$.
\item 负元素: 对于任何 $x\in\mathfrak{R}$, 都存在它的负元素 $-x\in\mathfrak{R}$ 使得 $x+(-x)=0$.
\end{itemize}

\subsubsection{域公理}
\begin{itemize}
\item 交换律: 对于任何 $x,y\in\mathfrak{R}$ 皆有 $x\cdot y=y\cdot x$.
\item 结合律: 对于任何 $x,y,z\in\mathfrak{R}$ 皆有 $x\cdot (y\cdot z)=(x\cdot y)\cdot z$.
\item 分配律: 对于任何 $x,y,z\in\mathfrak{R}$ 皆有 $x\cdot (y+z)=x\cdot y+x\cdot z$.
\item 单位元: 存在元素 $1\in\mathfrak{R}$, 使得对于任何 $x\in\mathfrak{R}$ 皆有 $x\cdot 1=x$.
\item 逆元素: 对于任何不等于0的 $x\in\mathfrak{R}$, 都存在 $x^{-1}\in\mathfrak{R}$ 使得 $x^{-1}\cdot x=1$.
\end{itemize}

\subsubsection{全序公理}
\begin{itemize}
\item 任意二元素 $x,y\subset\mathfrak{R}$ 之间有且仅有下列三种关系之一: $x<y$, $x=y$, $x>y$.
\item 如果 $x,y,z\in\mathfrak{R}$ 且有 $x<y$, $y<z$, 那么 $x<z$.
\item 如果 $x,y\in\mathfrak{R}$ 且 $x<y$, 那么对于任何 $z\in\mathfrak{R}$ 皆有 $x+z<y+z$.
\item 如果 $x,y\in\mathfrak{R}$ 且 $x<y$, 那么对于任何 $z>0$ 皆有 $x\cdot z<y\cdot z$.
\end{itemize}

\subsubsection{完备公理}
\begin{itemize}
\item 设 $A,B\subset\mathfrak{R}$, 使得对于任意 $a\in A$ 和任意 $b\in B$ 皆有 $a\leq b$, 那么存在元素 $c\in\mathfrak{R}$ 使得对于任意 $a\in A$ 和任意 $b\in B$ 皆有 $a\leq c\leq b$.
\end{itemize}

从中学课本已经知道, 可以用各种不同的方式来表示实数, 例如十进制小数, 但这些表示是相互等价的。 严格地来说, 这就是如下定理:
\begin{theorem}{实数的唯一性}
任意两个满足实数公理的实数模型都是同构的。
\end{theorem}
在这里, "同构" (isomorphic) 的意思是说: 如果 $\mathfrak{R}_1$ 和 $\mathfrak{R}_2$ 是两个实数模型, 那么存在一个双射 $f:\mathfrak{R}_1\to \mathfrak{R}_2$, 使得 $f$ 保持元素之间的加法和乘法运算 (即 $f$ 是域同构), 而且还保持序关系。

\begin{exercise}{}
试证明实数的唯一性定理。 提示: 若给定了一个实数模型, 那么可以从加法单位元0和乘法单位元1开始, 先在其中构造出整数和有理数, 然后利用完备公理说明所有的元素都可以被视为戴德金分割的分点; 这样一来, 所有的实数模型都同构于戴德金分割模型。
\end{exercise}
