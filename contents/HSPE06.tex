% 磁场(高中)
% 磁场|安培力|洛伦兹力|磁感应强度|磁通量

\begin{issues}
\issueDraft
\issueTODO
\end{issues}
\pentry{静电场\upref{HSPE01}}
\subsection{磁场}

具有磁性的物质叫做\textbf{磁体},能吸引铁、钴、镍等物质.磁体上磁性最强的部分叫做磁极,分\textbf{北极}($\mathrm N$)和\textbf{南极}($\mathrm S$),磁极不能单独存在.

磁极之间的作用规律为:同名磁极相互排斥,异名磁极相互吸引.

\textbf{磁场}是磁体或电流周围存在的一种看不见、摸不着的特殊物质.磁体与磁体、磁体与电流、电流与电流之间都存在相互作用,统称为磁相互作用,这种相互作用是通过磁场发生的.

磁场中某点磁场方向的表述:

\begin{enumerate}
\item 小磁针北极受磁场力的方向;
\item 小磁针静止时北极所指的方向;
\item 磁感线某点的切线方向;
\item 磁感应强度的方向.
\end{enumerate}

\subsubsection{磁感线}
与电场线类似,为了形象地描述磁场,在磁场中画出一系列有方向的假想曲线,曲线上每一点的切线方向都跟该点的磁场方向相同,这样的曲线叫做\textbf{磁感线}(也叫做\textbf{磁力线}).

在磁体的外部,磁感线从北极到南极,在磁体内部则从南极到北极.

磁感线的疏密程度反映了磁场的强弱,磁感线密的位置磁场强,磁感线疏的位置磁场强.

\subsubsection{地磁场}

地球是个巨大的磁体,在地球周围的空间存在着磁场,这个磁场叫做\textbf{地磁场}.地磁北极在地理南极附近,地磁南极在地理北极附近,地磁极和地理极并不重合,它们之间的夹角叫做\textbf{磁偏角}.