% 函数的极限
% keys 函数极限
% license Xiao
% type Tutor

\pentry{序列的极限\nref{nod_SeqLim},函数的极限(简明微积分)\nref{nod_FunLim}}{nod_737e}

回顾:函数\upref{functi}是一个非空集合 $A$ 到另一个集合 $B$ 的\textbf{对应法则}。

本文中,函数 $f$ 是指从 $\mathbb R$ 的某个非空子集 $X$ 到 $\mathbb{R}$ 的映射  

\subsection{函数的极限}
\begin{figure}[ht]
\centering
\includegraphics[width=12cm]{./figures/097039a199cabc65.png}
\caption{请添加图片描述} \label{fig_limfx_1}
\end{figure}
\begin{definition}{邻域和去心邻域}
定义\textbf{邻域}为: $U(x_0,\delta)=\{x\in \mathbb{R}:|x-x_0|<\delta\}$.

定义\textbf{去心邻域}为: $U_0(x_0,\delta)= U(x_0,\delta) \backslash \{x_0\}=\{x\in \mathbb{R}:0<|x-x_0|<\delta\}$.
\end{definition}
\begin{definition}{极限}
 设函数 $f(x)$ 在 $U_0(x_0,\delta_0)$($\delta_0>0$)内有定义。

  若存在实数 $A$ ,使得对任意 $\epsilon >0$,存在 $\delta>0$,使得当 $x\in U_0(x_0,\delta)$ 时,有 $|f(x)-A|<\epsilon$,则称\textbf{当 $x$ 趋于 $x_0$,函数 $f(x)$ 以 $A$ 为极限},记为 $\large \lim\limits_{x\rightarrow x_0}f(x)=A$ 或 $f(x)\rightarrow A\ (x\rightarrow x_0)$。
\end{definition}
补充:实际上有更宽泛的定义,只要 $x_0$ 是函数 $f(x)$ 定义域的\textbf{聚点},就可以定义在该点处的极限。

同序列极限的性质类似,函数极限也具有唯一性:
\begin{theorem}{}
  若函数 $f(x)$ 在 $x_0$ 处极限存在,证明在 $x_0$ 处极限唯一。
\end{theorem}
\begin{exercise}{}
\begin{enumerate}
\item  $f(x)=\left\{\begin{aligned} 0\ \ &(x<1)\\ 1\ \ &(x\ge 1) \end{aligned}\right.$ ,判断 $f(x)$ 在 $x_0=1$ 处极限是否存在。
\item $f(x)=\left\{\begin{aligned} 0\ \ &(x<1)\\ 1\ \ &(x= 1)\\2\ \ &(x>1) \end{aligned}\right.$,判断 $f(x)$ 在 $x_0=1$ 处极限是否存在。
\item $f(x)=x\cdot \sin(1/x)$,判断 $f(x)$ 在 $x_0=0$ 处极限是否存在。
\item $f(x)=\sin(1/x)$,判断 $f(x)$ 在 $x_0=0$ 处极限是否存在。
\item $f(x)=\E^x$,证明 $\lim\limits_{x\rightarrow a}=\E^a,\forall a\in\mathbb{R}$。
\end{enumerate}
\end{exercise}

\begin{figure}[ht]
\centering
\includegraphics[width=12cm]{./figures/2e581bb584574c4f.png}
\caption{黎曼函数} \label{fig_limfx_2}
\end{figure}

我们来看一个有趣的函数 $f(x)$,它的定义域为 $[0,1]$:
\begin{equation}
f(x)=\left\{
\begin{aligned} 
&1/q && (x=\frac{p}{q}\ (p,q\in \mathbb{N},\frac{p}{q}\text{为既约真分数}))\\
&0 && (\text{x=0或x=1或x}\notin \mathbb{Q})~.
\end{aligned}
\right.
\end{equation}
我们称它为\textbf{黎曼 (Riemann) 函数}。

虽然在定义域内有无穷多个点的函数值不为 $0$,但 $f(x)$ 的极限却处处为 $0$,我们之后还将看到,$f(x)$ 在无理点处处连续,但 $f(x)$ 处处不可导。

\begin{exercise}{}
\begin{enumerate}
\item 证明:若函数 $f(x)$ 在 $U(a,\delta_0)(\delta_0>0)$ 上有定义,且满足 $\large\lim\limits_{x\rightarrow a}f(x)=f(a)$,那么对任意极限为 $a$ 的序列 $\{x_n\}$ ,序列 $\{f(x_n)\}$ 的极限也为 $f(a)$。
     上述命题反过来也成立。
\item 对于任意给定的序列 $\{a_n\}(0<a_n<1)$,构造定义域为 $[0,1]$ 的函数 $f(x)$,满足 $\forall x\in \{a_n\},f(x)\neq 0;\ \forall x \in [0,1]\backslash \{a_n\},f(x)=0$,且 $f(x)$ 在定义域上极限处处为 $0$。
\end{enumerate}
\end{exercise}
\subsection{函数的左右极限}
\begin{figure}[ht]
\centering
\includegraphics[width=12cm]{./figures/3e87a1456060d6a0.png}
\caption{请添加图片描述} \label{fig_limfx_3}
\end{figure}
  如果把去心邻域 $U_0(x_0,\delta_0)$ 分成两块\textbf{单侧邻域}——

  左空心邻域:$U_0^+(x_0,\delta)=U_0(x_0,\delta)\cap (x_0,+\infty) \{x\in \mathbb{R} :x_0< x<x_0+\delta\}~.$

  右空心邻域:$U_0^-(x_0,\delta)=U_0(x_0,\delta)\cap (-\infty,x_0) \{x\in \mathbb{R} :x_0-\delta< x<x_0\}~.$

  那么就可以定义函数的左右极限:

  设 $f(x)$ 在 $U^+_0(x_0,\delta_0)(\delta_0>0)$ 上有定义。

  如果存在实数 $A$,使得对任意 $\epsilon >0$,存在 $\delta>0$,当 $x\in U_0^+(x_0,\delta)$ 时,有 $|f(x)-A|<\epsilon$,则称 $f(x)$ 在点 $x_0$ 的\textbf{右极限存在},而称 $A$ 为 $f(x)$ 在点 $x_0$ 的\textbf{右极限},记为 $\lim\limits_{x\rightarrow x_0^+}f(x)=A$ 或 $f(x_0^+)=A$。

  类似地可以定义\textbf{左极限存在}和\textbf{左极限}。

\begin{exercise}{}
\begin{enumerate}
\item $f(x)=\left\{\begin{aligned} 0\ \ &(x<1)\\ 1\ \ &(x= 1)\\2\ \ &(x>1) \end{aligned}\right.$,判断 $f(x)$ 在 $x_0=1$ 处的左极限与右极限。
\item  $f(x)=[x]$(取整函数),判断 $f(x)$ 在 $x_0=1$ 处的左极限与右极限。
\item  设函数 $f(x)$ 在 $U_0(x_0,\delta_0)$ 上有定义,证明: $\lim\limits_{x\rightarrow x_0} f(x)=A$ 当且仅当 $\lim\limits_{x\rightarrow x_0^-}f(x)=\lim\limits_{x\rightarrow x_0^+}f(x)=A$。
\end{enumerate}
\end{exercise}

\subsection{函数极限的各种情况}
称集合 $\{x:|x|>h\}(h>0)$ 为 $\infty$ 的邻域,记为 $U(\infty,h)$ (这时就没有必要定义去心邻域了)。

同样的可以分成两块\textbf{单侧邻域}:
\begin{equation}
\begin{aligned}
&U^+(\infty,h)=U(\infty,h)\cup (0,\infty)=\{x:x>h\}~,\\
&U^-(\infty,h)=U(\infty,h)\cup (-\infty,0)=\{x:x<-h\}~.
\end{aligned}
\end{equation}
  这样就可以函数在自变量趋向于无穷大时的极限:

  设函数 $f(x)$ 在 $U^+(\infty,h_0)$ 上有定义。若存在实数 $A$,使得 $\forall \epsilon >0, \exists X\in U^+(\infty,h_0)$,当 $x>X$ 时,有 $|f(x)-A|<\epsilon$,则称当 $x$ 趋于 $+\infty$ 时 $f(x)$ 的\textbf{极限存在},其极限为 $A$,记为 $\lim\limits_{x\rightarrow +\infty}f(x)=A$ 或 $f(x)\rightarrow A(x\rightarrow +\infty)$。

  类似地可以定义 $\lim\limits_{x\rightarrow -\infty}f(x)=A$ 和 $\lim\limits_{x\rightarrow \infty}f(x)=A$ 。
\begin{exercise}{}
\begin{enumerate}
\item 设函数 $f(x)$ 在 $U(\infty,h_0)$ 上有定义,证明: $\lim\limits_{x\rightarrow \infty} f(x)=A$ 当且仅当 $\lim\limits_{x\rightarrow -\infty}f(x)=\lim\limits_{x\rightarrow +\infty}f(x)=A$。
\item 设序列 $\{a_n\}$ 收敛于 $A$,定义函数 $f(x)=a_{|[x]+1|}$,证明:$\lim\limits_{x\rightarrow \infty} f(x)=A$。
\end{enumerate}
\end{exercise}

\begin{definition}{广义极限}
如果自变量趋向于一个值时,函数趋向于无穷大,则可以定义广义极限:

  设 $f(x)$ 在 $U_0(x_0,\delta_0)(\delta_0>0)$ 上有定义,若 $\forall M>0,\exists \delta>0$,使得当 $x\in U_0(x,\delta)$ 时,有 $f(x)>M$,则称当 $x$ 趋于 $x_0$,$f(x)$ 趋于 $+\infty$,或称当 $x$ 趋于 $x_0$ 时,$f(x)$ 的\textbf{广义极限}为 $+\infty$。记为 $\lim\limits_{x\rightarrow x_0}f(x)=+\infty$ 或 $f(x)\rightarrow +\infty(x\rightarrow x_0)$。此时也称 $f(x)$ 为当 $x$ 趋于 $x_0$ 时的\textbf{正无穷大量}。

  同样地可以定义\textbf{负无穷大量}和\textbf{无穷大量}。
\end{definition}

  

  对于函数极限而言,自变量有以下几种变化情况:
\begin{equation}
  x\rightarrow x_0;\ x\rightarrow x_0^+;\ x\rightarrow x_0^{-};\ x\rightarrow \infty;\ x\rightarrow +\infty;\ x\rightarrow -\infty~.
\end{equation}
  对于这六种情况可以定义各自的\textbf{广义极限}:
\begin{equation}
  f(x)\rightarrow A;\ f(x)\rightarrow +\infty;\ f(x)\rightarrow -\infty;\ f(x)\rightarrow \infty~,
\end{equation}
  因此一共有 $24$ 种可能的函数极限的情形。

  但由于它们之间有极大的相似之处,所以很容易进行记忆和想象。

\begin{exercise}{}
\begin{enumerate}
\item $f(x)=x^2\sin x$,判断它是不是当 $x\rightarrow \infty$ 时的无穷大量。
\item $f(x)=1/(x-1)\ (x\neq 1)$,求 $f(1^-),f(1^+),\lim\limits_{x\rightarrow 1}f(x),\lim\limits_{x\rightarrow \infty}f(x)$。
\end{enumerate}
\end{exercise}
