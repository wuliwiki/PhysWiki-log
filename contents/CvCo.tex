% 约定式提交
% keys 约定式提交
% license Usr
% type Wiki


约定式提交规范是一种在进行代码变更时的轻量级约定。约定式提交制定了一组简单的规则来创建清晰的提交历史,便于编写自动化工具。通过在提交信息中描述功能、修复和\textbf{破坏性变更}(Breaking Change)(下文会对破坏性变更做出解释),使这种惯例与语义化版本相互对应。

\subsection{简介}
对于每次提交的代码变更,说明结构:
\begin{lstlisting}[language=none]
<type>[optional scope]: <description>

[optional body]

[optional footer(s)]
\end{lstlisting}
可以对应翻译为,
\begin{lstlisting}[language=none]
<类型>[(可选)范围]: <描述>

[(可选)正文]

[(可选)脚注]
\end{lstlisting}
提交说明描述了下列内容,可以向使用者与协同编辑者说明本次变更的内容:
\begin{enumerate}
\item \verb`fix`:这代表本次提交修复了一些 bug(对于修复多个 bug,建议多次提交)。这对应语义化版本的 \verb`PATCH`。
\item \verb`feat`:这代表本次提交实现了新的功能(对于实现多个功能,也建议多次提交)。这对应语义化版本的 \verb`MINOR`。
\item \verb`BREAKING CHANGE`:一般在类型后面加感叹号表示本次提交是一个破坏性变更,表示引入了\textbf{破坏性 API 变更},破坏性变更可以是任意类型的提交的一部分。这对应语义化版本的 \verb`MAJOR`。可以将破坏性变更理解为,进行了某些重大更新,使旧版本\textbf{不再兼容}(有时重大更新,对用户使用有严重影响的,也可以使用破坏性变更表示)。\textbf{破坏性变更应在脚注中标注内容}。
\end{enumerate}
除了 \verb`fix(!)` 与 \verb`feat(!)` 以外,类型也可以是 \href{https://github.com/conventional-changelog/commitlint/tree/master/\%2540commitlint/config-conventional}{@commitlint/config-conventional} 中推荐的任意一个,下面给出一些另外常用的类型:
\begin{enumerate}
\item \verb`build`: 用于修改项目构建系统,例如修改依赖库、外部接口或者升级 Node 版本等;
\item \verb`chore`: 用于对非业务性代码进行修改,例如修改构建流程或者工具配置等;
\item \verb`ci`: 用于修改持续集成流程,例如修改 Travis、Jenkins 等工作流配置;
\item \verb`docs`: 用于修改文档,例如修改 README 文件、API 文档等;
\item \verb`style`: 用于修改代码的样式,例如调整缩进、空格、空行等;
\item \verb`refactor`: 用于重构代码,例如修改代码结构、变量名、函数名等但不修改功能逻辑;
\item \verb`perf`: 用于优化性能,例如提升代码的性能、减少内存占用等;
\item \verb`test`: 用于修改测试用例,例如添加、删除、修改代码的测试用例等。
\end{enumerate}
其它提交类型在约定式提交规范中并没有强制限制,并且在语义化版本中没有隐式影响(除非它们包含破坏性变更)。 可以为提交类型添加一个围在圆括号内的范围,以为其提供额外的上下文信息。例如 \verb`feat(console): adds ability to active console in game.`。当然,描述也可以是中文的,但是描述是必要的,且\textbf{破坏性变更的脚注也是必要的}。

如果一次变更符合多种类型,应当予以拆分,并多次提交。

\subsection{FAQ}
1. 类型打错了怎么办?写错了怎么办?
例如本身是 \verb`feat`,不小心打成了 \verb`faat` 或者 \verb`fix`。

在发布或合并之前,可以采用 \verb`git` 提供的 \verb`git rebase -i` 来修改历史功能。

2. 在约定式提交规范下,还原怎么办?
特别规定还原为 \verb`revert` 类型,并且建议在正文中进行描述。下面有一个例子:
\begin{lstlisting}[language=none]
revert: the cake is a lie.

Refs: 1298560a, c9012783
\end{lstlisting}

3. 变更范围涉及多个怎么办?
如果是全局,可以使用 \verb`*` 或 \verb`all`(推荐)来表示;如果是主体、主逻辑,可以使用 \verb`main` 来表示。
如果修改了多个文件导致范围涉及广,建议拆分为多次提交。对于单次修改的,可以在描述中指出变更范围而在括弧内写 \verb`*`。