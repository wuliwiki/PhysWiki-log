% 守恒量(量子力学)
% 量子力学|守恒|本征值|能量守恒

\pentry{平均值\upref{QMavg}, 无限深势阱\upref{ISW}, 薛定谔方程\upref{TDSE}}

本文使用原子单位制\upref{AU}. 在经典力学中, 某个时刻测量系统的某个量 $Q$ 可以得到唯一确定的值. 如果测量值不随时间变化, 就说它\textbf{守恒(conserved)}. 而在量子力学中, 若对一个系统单次测量, 测量值是不确定的, 那守恒量由该如何理解呢? 比如两个无限深势阱中的粒子具有相同的波函数——若干能量本征态的线性组合, 测得的能量却可能各不相同, 是否意味着能量不守恒?

答案是否定的, 在量子力学中, 我们定义:
\begin{definition}{守恒量(量子力学)}\label{QMcons_def1}
若对系统测量某个物理量得到的测量值的概率分布不随时间改变, 这个量就是守恒的.
\end{definition}

例如在无限深势阱中任意随时间变化的波包(未必是单个能量本征态, 而是任意能量本征态的线性组合, 例如 “无限深势阱中的高斯波包\upref{wvISW}”), 测得每个能量本征值的概率都不随时间变化, 这就说明能量守恒.

\begin{corollary}{平均值守恒}
守恒量的平均值也不随时间变化.
\end{corollary}
根据\autoref{QMcons_def1} 易证.

\subsection{能量守恒}

\begin{theorem}{能量守恒}
若哈密顿算符 $H$ 不随时间变化, 就有能量守恒.
\end{theorem}
证明: 令能量的所有本征态为(为了方便起见, 假设本征态是离散的) $\{\ket{u_i}\}$, 系统状态随时间的演化可以记为(见\autoref{TDSE_eq5}~\upref{TDSE}, 其中 $C_i$ 为常数, 由初始状态决定)
\begin{equation}
\ket{v(t)} = \sum_i C_i \E^{-\I E_i t} \ket{u_i}
\end{equation}
令能量 $E_j$ 对应的所有本征态的编号的集合为 $D_j$(如果 $E_j$ 是非简并的, 那么 $D_j = \qty{j}$),% 链接未完成
那么测得 $E_j$ 的概率为
\begin{equation}
P(E_j) = \sum_{i \in D_j} \abs{C_i}^2
\end{equation}
不随时间变化, 所以有能量守恒. 证毕.

\begin{example}{}
能量守恒的简单系统如: 无限深势阱\upref{ISW}, 简谐振子\upref{QSHOop}, 自由粒子\upref{FreeP1}, 有限深势阱\upref{FiSph}等. 这些系统的哈密顿算符都不随时间变化. 将任意时刻的波函数投影到能量的本征态上, 得到的系数的模方都是相同的.
\end{example}

\subsection{其他守恒量}

\pentry{对易厄米矩阵与共同本征矢\upref{Commut}}

\begin{theorem}{对易与守恒量}
令 $H$ 为哈密顿算符, $Q$ 是另一个物理量的算符. 以下两个命题互为充分必要条件\upref{SufCnd}
\begin{enumerate}
\item $Q$ 对应一个守恒量
\item $\comm{H}{Q} = 0$
\end{enumerate}
\end{theorem}

该定理在经典力学中的对应定理是: 若物理量 $\omega(q, p)$ 和哈密顿量 $H$ 的泊松括号\upref{poison} $\pb{\omega}{H}$ 为零, 则 $\omega$ 是一个守恒量.

证明:我们还是从离散本征态的简单情况来证明这个定理, 且假设哈密顿算符不含时\footnote{如果含时, 可以将时间分割成许多小区间, 假设每个小区间中不随时间变化.}. 首先由 2 证明 1, 若两算符对称, 则它们必有一组完备的共同本征矢 $\{\ket{u_i}\}$, 满足
\begin{equation}
\begin{cases}
H \ket{u_i} = E_i \ket{u_i}\\
Q \ket{u_i} = q_i \ket{u_i}
\end{cases}
\end{equation}

系统状态随时间的变化可以记为
\begin{equation}
\ket{v(t)} = \sum_i C_i \E^{-\I E_i t} \ket{u_i}
\end{equation}
其中 $C_i$ 为常数. 令集合 $D_j$ 包含 $q_j$ 对应的所有基底的角标. 于是, 任意时刻测到 $q_j$ 的概率等于相关系数的模方和
\begin{equation}
P(q_j) = \sum_{i \in D_j} \abs{C_i \E^{-\I E_i t}}^2 = \sum_{i \in D_j} \abs{C_i}^2
\end{equation}
也是常数. 证毕.

由 1 证明 2 要更麻烦一些. 我们新开一节, 不感兴趣的读者可以跳过.

\subsection{证明守恒量算符与哈密顿算符对易}
先假设能量的本征态基底为 $\{\ket{u_i}\}$, 系统状态随时间变化为
\begin{equation}
\ket{v(t)} = \sum_i C_i \E^{-\I E_i t} \ket{u_i}
\end{equation}
令算符 $Q$ 任意本征值 $q_m$ 对应的本征矢子空间的一组基底为 $\{\ket{v_k}\}$, 那么任意时刻测得 $q_m$ 的概率为
\begin{equation}
\begin{aligned}
P(q_m) &= \sum_k \abs{\bra{v_k} \sum_i C_i \E^{-\I E_i t} \ket{u_i}}^2\\
&= \sum_k \qty(\sum_i C_i \E^{\I E_i t} \braket{v_k}{u_i})\Her \qty(\sum_j C_j \E^{-\I E_j t} \braket{v_k}{u_j})\\
&= \sum_{i,j} \qty( C_i^* C_j \sum_k \braket{u_i}{v_k} \braket{v_k}{u_j}) \E^{\I (E_i - E_j) t}
\end{aligned}
\end{equation}
若 $Q$ 守恒, 则要求对于任何 $C_i$ 可能的取值, 该式的结果都与时间无关. 所以对任意 $i \ne j$ 都有
\begin{equation}
\sum_k \braket{u_i}{v_k} \braket{v_k}{u_j} = 0
\end{equation}
稍加思考可以得出要么 $\ket{u_j}$ 在 $q_m$ 子空间中, 要么 $\ket{u_j}$ 在 $q_m$ 子空间上的投影为 0. 所以任何 $q_m$ 子空间(假设是 $N_m$ 维)同样可以由 $N_m$ 个 $\ket{u_i}$ 张成. 所以可以找到 $H$ 和 $Q$ 的一套共同本征矢, 所以两算符对易. 证毕.
