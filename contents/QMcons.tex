% 守恒量(量子力学)
\pentry{平均值\upref{QMavg}}

在经典力学中, 某个时刻测量系统的某个量 $Q$ 可以得到唯一确定的值. 如果这个值不随时间变化, 就说它\bb{守恒}. 而在量子力学中, 这个值(算符的本征值)是不确定的, 那守恒量由该如何理解呢? 比如无限深势阱中两个相同的粒子处于相同的状态(相同的波函数), 测得的能量却可能各不相同, 是否意味着能量不守恒?

答案是否定的, 在量子力学中, 若测量某个物理量得到任意本征值的概率都不随时间改变, 这个量就是守恒的. 例如在无限深势阱中随时间变化的波包, 测得每个能量的概率都不随时间变化, 这就说明能量守恒. 由这个定义可知, 守恒量的平均值也不随时间变化.

\subsection{对易与守恒}
当 $H$ 不含时, 有能量守恒.

若 $Q$ 和哈密顿量 $H$ (无论是否含时)对易, $Q$ 就是一个守恒量.

证明: 将态矢拆分到 $Q$ 的各个本征值子空间, 由于对易, $H$ 在各个子空间中闭合, 使每个子空间中的状态可以独立根据薛定谔方程演化, 所以各个子空间中的概率不随时间变化.

另外 $\ev{Q}$ 的变化率可以用 $[H, Q]$ 表示(具体忘了).

当能量守恒时, 所有的守恒量就构成一组 CSCO (complete set of commutative operators). 例如氢原子的能量, 总角动量方 $L^2$, 以及一个分量 $L_z$(注意也有其他守恒量例如旋转对称和宇称是重复的起不到区分作用). 

话说氦原子的 CSCO 应该有 6 个, 然而我只知道 $L,M,l1,l2,n1,n2$, 如果用 $E,L,M$, 那么剩下三个是什么呢?
