% 凸函数
% 凸函数|凹凸性|琴生不等式|均值不等式

\pentry{导数(数学分析)\upref{Der2}}

设函数 $f(x)$ 在区间 $I$ 上有定义,若对于任意的 $x_1,x_2\in I$, $t\in (0,1)$,都有
\begin{equation}
f(tx_1+(1-t)x_2)\le tf(x_1)+(1-t)f(x_2)
\end{equation}
那么称 $f(x)$ 为 $I$ 上的\textbf{凸函数}。如果将 $\le$ 改为 $<$,那么称 $f(x)$ 为\textbf{严格凸函数}。类似地可以定义\textbf{凹函数}和\textbf{严格凹函数}。
\begin{theorem}{}\label{the_ConvFu_1}
设 $f(x)\in C[a,b]$。$f(x)$ 满足
\begin{enumerate}
\item 在 $(a,b)$ 上可导。则 $f(x)$ 是(严格)凸函数的充分必要条件是 $f'(x)$ 在 $(a,b)$ 内(严格)单调递增。
\item 在 $(a,b)$ 上二阶可导。则 $f(x)$ 是凸函数的充分必要条件是 $f''(x)\ge 0$;$f(x)$ 是严格凸函数的充分必要条件是 $f''(x)\ge 0$ 且 $f''(x)$ 在 $(a,b)$ 的任意子区间上都不恒等于 $0$。
\end{enumerate}
\end{theorem}
凸函数的以上几条充分必要条件可以从图像中直观地理解,事实上,画图常常有助于解决与凸函数性质相关的问题。
\begin{exercise}{}
证明两个 $I$ 上的凸函数相加仍然是 $I$ 上的凸函数。

\textbf{提示:}由于给定函数在区间 $I$ 上不一定可导,所以不能用  
\autoref{the_ConvFu_1} 证明,必须回归到定义。
\end{exercise}
\begin{exercise}{}
设 $f(x)$ 是区间 $(a,b)$ 上的凸函数,证明 $f(x)$ 在 $(a,b)$ 内的任意闭区间 $[\alpha,\beta]$ 上都满足 Lipschitz 条件(也就是说存在 $L>0$,对任意 $x_1,x_2\in [\alpha,\beta]$,都有 $|f(x_1)-f(x_2)|\le L|x_1-x_2|$)。

\textbf{提示:}如果将 $[\alpha,\beta]$ 改为开区间,则很容易找到反例,所以在证明这个命题时要充分利用好闭区间的特性,尝试将 $L$ 的上界量化。取充分小的 $h$,使 $[\alpha-h,\beta+h]\subset (a,b)$,令 $x_3=x_2+h$,根据凸函数的定义(事实上可以从图像上很快得出)$(f(x_2)-f(x_1))/(x_2-x_1)<(f(x_3)-f(x_2))/(x_3-x_2)\le (M-m)/h$,其中 $M,m$ 是区间 $[\alpha-h,\beta+h]$ 上函数的最大、最小值(可以证明,闭区间上的凸函数一定有最大最小值)。于是可以令 $L=(M-m)/h$。
\end{exercise}

函数的凹凸性可以通过\textbf{琴生不等式}引出各种不等式,函数的凹凸性的定义是琴生不等式的 $n=2$ 版本,而事实上两者作为定义是等价的。
\begin{theorem}{琴生不等式}
$f(x)$ 在区间 $[a,b]$ 上是凸函数的充分必要条件为:对任意 $x_1,\cdots,x_n\in [a,b],\sum_{i=1}^n t_i=1$,以下不等式总成立
\begin{equation}
f(t_1x_1+t_2x_2+\cdots+t_nx_n)\le t_1f(x_1)+t_2f(x_2)+\cdots+t_nf(x_n)
\end{equation}
对于严格凸函数,上面的充要条件应将 $\le$ 改为 $<$,并且 $x_1,\cdots,x_n$ 不全相等。
\end{theorem}
利用凸函数的性质,可以推出一系列不等式,例如调和-几何-算术平均不等式、柯西不等式、赫尔德不等式等。
\begin{exercise}{}
证明\textbf{幂平均不等式}:设 $x_1,\cdots,x_n>0$,构造以下函数:
\begin{equation}
F(p)=\left\{
\begin{aligned}
&\left(\frac{x_1^p+x_2^p+\cdots+x_n^p}{n}\right)^{1/p}, &p\neq 0\\
&(x_1\cdots x_n)^{1/n}, &p=0
\end{aligned}
\right.
\end{equation}
那么 $F(p)$ 是单调递增函数。
\end{exercise}
提示:幂平均不等式是调和-几何-算术平均不等式的自然推广,注意到 $F(-\infty)<F(-1)<F(0)<F(1)<F(\infty)$ 恰好对应着
\begin{equation}
\begin{aligned}
\min\{x_1,\cdots,x_n\}&<\frac{n}{\frac{1}{x_1}+\cdots+\frac{1}{x_n}}<(x_1\cdots x_n)^{1/n}\\
&<\frac{x_1+\cdots+x_n}{n}<\max\{x_1,\cdots,x_n\}
\end{aligned}
\end{equation}
先考虑证明 $0<p<q$ 时 $F(p)<F(q)$,其他情况的证明类似。为了能利用上琴生不等式,不等式一侧 $x_i$ 的指数最好为 $1$。所以进行换元,令 $a_i=x_i^p$,可以得到
\begin{equation}
\left(\frac{a_1+\cdots+a_n}{n}\right)^{q/p}\le \frac{a_1^{q/p}+\cdots+a_n^{q/p}}{n}
\end{equation}
函数 $x^{q/p}$ 在 $(0,\infty)$ 上是凸函数,因此可以利用琴生不等式证明这个不等式。

