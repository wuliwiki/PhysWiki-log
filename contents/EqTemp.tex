% 等温过程
% keys 等温过程|体积|压强|状态方程|做功
% license Xiao
% type Tutor

\pentry{压强体积图\nref{nod_PVgraf}}{nod_915e}

等温过程的特征是系统的温度保持不变,即 $\mathrm dT=0$。由于理想气体的内能只取决于温度,所以在等温过程中,理想气体的内能也保持不变,也就是说 $\mathrm dE=0$。

设想一汽缸壁是绝对不导热的,而底部则是绝对导热的(\autoref{fig_EqTemp_1})。现在将气缸的底部和一恒温热源相接触,当活塞上的外界压强无限缓慢地降低时,缸内气体也将随之逐渐膨胀,对外做功气体内能就随之缓慢减少,温度也将随之略微降低。然而,由于气体与恒温热源相接触,当气体温度比热源温度略低时,就有微小的热量传给气体,使气体温度维持原值不变。这一准静态过程就是一个\textbf{等温过程(isothermal process)}。
\begin{figure}[ht]
\centering
\includegraphics[width=8cm]{./figures/d9bc104ca5e75638.pdf}
\caption{气体的等温膨胀} \label{fig_EqTemp_1}
\end{figure}

在等温过程中,$P_1V_1=P_2V_2$,系统对外做的功为
\begin{equation}
W= \int_{V_{1}}^{V_{2}} p \mathrm{d} V=\int_{V_{1}}^{V_{2}} \frac{P_{1} V_{1}}{V} \mathrm{d} V=P_{1} V_{1} \ln \frac{V_{2}}{V_{1}}=P_{1} V_{1} \ln \frac{P_{1}}{P_{2}}~.
\end{equation}

根据理想气体状态方程\upref{PVnRT}可得
\begin{equation}
W=\frac{m}{M} R T \ln \frac{V_{2}}{V_{1}}=\frac{m}{M} R T \ln \frac{P_{1}}{P_{2}}~.
\end{equation}

又根据热力学第一定律,系统在等温过程中所吸收的热量应和它所做的功相等,即
\begin{equation}\label{eq_EqTemp_1}
Q_{T}=W=\frac{m}{M} R T \ln \frac{V_{2}}{V_{1}}=\frac{m}{M} R T \ln \frac{P_{1}}{P_{2}}~.
\end{equation}

那么等温过程在 $P$-$V$ 图上长什么样呢?当然是一条双曲线上的一段。这种双曲线就叫做\textbf{等温线(isotherm)}。如\autoref{fig_EqTemp_2} 所示,$\rm I\to II$ 就是一个等温膨胀过程。在等温膨胀过程中,理想气体所吸取的热量全部转化为对外所做的功;反之,在等温压缩时。外界对理想气体所做的功,将全部转化为传给恒温热源的热量。

\begin{figure}[ht]
\centering
\includegraphics[width=7cm]{./figures/86ba75e4bfd8f3b6.pdf}
\caption{等温过程中功的计算} \label{fig_EqTemp_2}
\end{figure}
