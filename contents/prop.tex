% 命题
% keys 命题
% license Usr
% type Tutor

\begin{definition}{真值}
真值有\textbf{真(True,$1$)}与\textbf{假(False,$0$)}两个值组成。
\end{definition}

\begin{definition}{命题}
\textbf{命题}是具有\textbf{确切的真值}(\textbf{唯一的真值})的无歧义的陈述句。命题无需知道真值具体是什么,但是要求真值一定唯一。
\end{definition}
例如,“$x>3$”不是命题,但“对于所有实数 $x$,$x>3$”是命题。有歧义的,例如“这句话是假的”不是命题。



\begin{definition}{命题常量}
对于在一个解释范围内,命题的真值是确定的,则称这命题是\textbf{命题常量}。
\end{definition}
\begin{definition}{命题变元}
\textbf{命题变元(又称句子变元)}是一个可真可假的变量。代表了一个在某解释范围内的、不确定的、泛指的命题,以真或假为取值范围。
\end{definition}
特别的,因为命题变元可真可假,所以其不是命题,但当确定了某个解释的时候,命题变元会确定,此时命题变元化为命题常元。


