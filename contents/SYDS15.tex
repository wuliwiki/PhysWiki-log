% 中山大学 2015 年918专业基础(数据结构)考研真题
% keys 中山大学|2015|数据结构|考研
% license Xiao
% type Tutor

\subsection{一、单项选择题(每小题3分,共45分)}
1. STL中的优先队列是采用什么数据结构来实现的? \\
A.堆  $\qquad$ B.队列 $\qquad$ C.栈 $\qquad$ D.图

2.数据结构中,与所使用的计算机无关的是数据的()结构。 \\
A.存储 $\qquad$ B.物理 $\qquad$ C.逻辑 $\qquad$ D.物理和存储

3.计算机算法指的是( )。 \\
A.计算方法 $\qquad$ B. 排序方法 \\
C.调度方法 $\qquad$ D. 解决问题的有限指令序列

4.给定一个有n个元素的数组(n为偶数)。如果要找出数组中的最大元素和最小元素,最少要进行( )次比较? \\
A.$2n$ $\qquad$ B.$3n/2-2$ $\qquad$ C.$2n-2$ $\qquad$ D.$4n/3$

5.给定一个包含250个整数的数组,该数组中的整数已按从小到大的顺序排好序。假设用二分查找从该数组中寻找某个给定的整数y,最多只需要做( )次比较。 \\
A.8 $\qquad$ B.9 $\qquad$ C.10 $\qquad$ D.7

6. $T(n)$表示某个算法的时间复杂度。假设$T(n)=2T(n/2)+O(m)$,则$T(n)$为( ) \\
A. $O(log3n)$ $\qquad$ B. $O(n)$ $\qquad$ C. $O(nlog_3n)$ $\qquad$ D. $O(n^2)$

7.假设整数n>0,下面的程序的时间复杂度是( ). \\
\begin{lstlisting}[language=cpp]
x=2;
while (x<n/3) x=2*x;
\end{lstlisting}
A. $O(log_2n)$ $\qquad$ B. $O(m)$ $\qquad$ C. $O(nlog_2n)$ $\qquad$ D.$O(n)$

8.下列排序算法中,哪个是稳定的排序算法? \\
A.选择排序 $\qquad$ B. 快速排序 $\qquad$ C. 归并排序 $\qquad$ D. 希尔排序

9.假设小明用某-排序算法对整数序列(82, 45, 25, 15, 21)进行排序。以下为排序过程中序列状态的变化过程: \\
输入:82 45 25 15 21 \\
第一步:45 82 25 15 21 \\
第二步:25 45 82 15 21 \\
第三步:15 25 45 82 21 \\
...... \\
请问小明用的是什么排序算法? \\
A.选择排序 $\qquad$ B.归并排序 $\qquad$ C.快速排序 $\qquad$ D.插入排序

10.以下的排序算法中,哪个算法在最坏情况下的时间复杂度是0(n3)? \\
A.堆排序  $\qquad$ B. 快速排序  $\qquad$ C. 归并排序 $\qquad$ D. 基数排序

11.给定一个算术表达式$X$。$X$的中缀形式是$A*B+C*D-E$,且X的前缀形式是$+*AB-*CDE$.那么,$x$的后缀形式是什么? \\
A. $ACD*E-B*+$ $\qquad$ B. $AB*CD*+E-$ $\qquad$ C. $CD*E-AB*+$ $\qquad$ D. $AB*CD*E-+$

12.给定一棵空的AVL 树,依次把13, 24, 37, 90, 53逐一插入该树, 在此过程中要保持该树为AVL树(假设左子树的元素要小于右子树) .则最终得到的AVL树的高度是( ), 树根是( ). \\
A.3,37 $\qquad$ B.3, 24 $\qquad$ C.4, 37 $\qquad$ D.3, 53

13.下面哪个函数随着n增大而增长得最快?( ). \\
A.$100n^2log_2n$ \\
B.$n(log_2n)^5$ \\
C.$n^{2.1}$ \\
D.$n^2+1000nlog_2n$

14.一个有$n$个顶点的无向图最多有( )条无向边(假设该图无自环)。 \\
A.$n$ $\qquad$ B. $n(n-1)$ $\qquad$ C. $n(n-1)/2$ $\qquad$ D.$2n$

15.一棵高度为$k$的二又树最多有( )个节点 \\
A.$2^{k+1}-1$ $\qquad$ B.$2^k-1$ \\
C.$2^{k+1}-1$ $\qquad$ D.$2^k+1$.

\subsection{二、简答题(共55分)}
1. (11 分)给定一个整数数组{4,6,3,2,1,5,7}.假设用选择排序算法对数组中的整数进行从小到大排序。请写出每次迭代后数组中的状态( 即每次迭代后,数组中的7个整数是如何排序的)

2. (11 分)从空的二叉树开始,根据字典顺序,严格按照二叉排序树(或称二叉搜索树)的插入算法,依次插入e,b, d, f, a, g, C。请画出构造二叉排序树的每一步骤。

3. (10分)假定一个堆为(56,38,42,30,25,40,35,20), 则依次从中删除两个元素后得到的堆是什么?要求画出过程。

4.给定一个空的哈希表,依次把键12, 34, 55, 54, 13, 21, 19, 70插入到哈希表中。假没采用的哈希函数是b(k)=k mod 11,采用线性探查(linear probing)来解决冲突 \\
(1)当上述键值全部插入后,请画出哈希表的状态(6分) \\
(2)假如每个键值被查找的概率均等,请计算出平均查找长度(average search length) (6 分)
\begin{table}[ht]
\centering
\caption{第二4题表}\label{tab_SYDS15_1}
\begin{tabular}{|c|c|c|c|c|c|c|c|c|c|c|c|}
\hline
下标 & 0 & 1 & 2 & 3 & 4 & 5 & 6 & 7 & 8 & 9 & 10 \\
\hline
键 &  &  &  &  &  &  &  &  &  &  & \\
\hline
\end{tabular}
\end{table}

5. 给定图$G$如下
\begin{figure}[ht]
\centering
\includegraphics[width=5cm]{./figures/864cedc16fe90ebc.png}
\caption{第二5题图} \label{fig_SYDS15_1}
\end{figure}
(1)请找出并画出图$G$的一棵最小代价生成树. (5分) \\
(2)请画出图$G$对应的邻接矩阵. (6分)

\subsection{三、编程题(共50分)}
1. (10分)以下是一个(不完整的)直接插入排序算法的代码,请根据注释的提示把缺少的代码补充完整。
\begin{lstlisting}[language=cpp]
void insertion_sort(int entrylI, int count)
{
    int first. unsorted;  // position of first unsorted entry
    int position;         // searches sorted part of list
    int current;          // holds the entry temporarily removed from list
    for(first_unsorted = 1; first_unsorted < count; first_unsorted++)
    {
        if(entry[first_unsorted] < entry[first_unsorted-1])
        {
            // please complete the code here
        }
    }
}
\end{lstlisting}

2. (15分)逆转链表:写一算法逆置带头结点的单链表L,要求逆置后的单链表利用L中的原结点,不可以重新申请结点空间。链表结点的声明如下:
\begin{lstlisting}[language=cpp]
template <class Entry>
struct Node{
    Entry data;
    Node<Entry> * next;
};
\end{lstlisting}
请实现下面的函数:
\begin{lstlisting}[language=cpp]
void reverse(Node<Entry>* L)
\end{lstlisting}

3. (25分)写一个算法,逐层遍历一棵二叉树(从上到下,从左到右)。以下是二叉树及二叉树中的节点的声明:
\begin{lstlisting}[language=cpp]
template <class Entry>
class Binary_tree {
public:
    Binary_tree();
    void Layer_order(vold(*visit)(Binary_node<Entry> &));
protected:
    Binary_node<Entry> * root;
};
template <class Entry>
struct Binary_node{
    Entry data;
    Binary_node<Entry> * left;
    Binary_node<Entry> * right;
    Binary_node();
    Binary_node(const Entry &x);
};
\end{lstlisting}
请实现下面的函数:
\begin{lstlisting}[language=cpp]
void Layer.order(void(*visit)(Entry &)).
\end{lstlisting}
