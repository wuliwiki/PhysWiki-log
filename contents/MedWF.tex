% 介质中的波动方程
% keys 波动方程|电磁场|介质|麦克斯韦
% license Xiao
% type Tutor

\begin{issues}
\issueDraft
\end{issues}

\pentry{真空中的平面电磁波\nref{nod_VcPlWv}}{nod_126f}

对非磁介质
\begin{equation}
\curl (\curl \bvec E) = -\mu_0 \pdv[2]{\bvec D}{t}~.
\end{equation}
平面波时
\begin{equation}
\laplacian \bvec E - \mu_0 \pdv[2]{\bvec D}{t} = 0~,
\end{equation}
\begin{equation}
\bvec D = \epsilon_0 \bvec E + \bvec P = \epsilon_0(1 + \tilde\chi)\bvec E + \bvec P^{NL} = \tilde \epsilon \bvec E + \bvec P^{NL}~,
\end{equation}
\begin{equation}
\laplacian\bvec E - \mu_0\tilde\epsilon \pdv[2]{\bvec E}{t} = \mu_0 \pdv[2]{\bvec P^{NL}}{t}~.
\end{equation}
 
波浪号表示复数,$\tilde\epsilon$ 和 $\tilde\chi$ 都只是 $\omega$ 而不是场强的函数。 先来看线性的情况( $\bvec P^{NL} = \bvec 0$ ), 例如洛伦兹模型。
\begin{equation}
\laplacian\bvec E - \mu_0 \tilde\epsilon \pdv[2]{\bvec E}{t} = \bvec 0~.
\end{equation}
(见齐次波动方程, 先做时间傅里叶变换, 再解齐次亥姆霍兹方程, 通解是所有平面波) 然而这里的 $\tilde k^2 = \mu_0 \tilde \epsilon \omega$ 是复数, 平面波变为指数衰减的单频单向波。 以 $z$ 方向传播 $x$ 方向极化为例, 令  $\tilde k = k + \kappa$
\begin{equation}
E_x(z,t) = \tilde E_{0x} \E^{\I(\tilde kz - \omega t)} = \tilde E_{0x} \E^{-\kappa z} \E^{\I(kz - \omega t)}~,
\end{equation}
\begin{equation}
\tilde k = \omega \sqrt{\mu_0\tilde\epsilon} = \frac{\omega }{c}\sqrt {1 + \chi ^{(1)}}~,
\end{equation}
通解仍然为所有可能的单频单向波的线性组合。 实折射率和吸收系数定义为
\begin{equation}
n = \frac{ck}{\omega } = \Re\qty[\sqrt{1 + \chi ^{(1)}}] \approx 1 + \frac12 \Re\qty[\chi^{(1)}]~,
\end{equation}
\begin{equation}
\alpha  = 2\kappa  = \frac{2\omega}{c} \Im\qty[\sqrt{1+\chi^{(1)}}] \approx \frac{\omega}{c}\Im\qty[\chi ^{(1)}]~,
\end{equation}
\begin{equation}
\tilde \epsilon  = \epsilon_0 (n + \I \alpha c/2\omega )^2~.
\end{equation}
 
注意区分 $\epsilon$ (permittivity), $\epsilon_r$ (dielectric constant) 和 $\chi$ (susceptivility)。 不同的书符号可能不一样,以名称和语境为准。
