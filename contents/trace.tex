% 矩阵的迹
% keys 矩阵|迹|相似变换|本征值

%基本完成。可能需要添加一些典型的实例。

\pentry{相似变换和相似矩阵\upref{MatSim}}

\begin{definition}{矩阵的迹}
令 域 $\mathbb{F}$ 上的 $N$ 维方阵 $\mat A$ 的矩阵元为 $a_{ij}\in\mathbb{F}$, 它的\textbf{迹(trace)}定义为对角线上矩阵元之和
\begin{equation}
\opn{tr}(A) = \sum_{i=0}^N a_{ii}~.
\end{equation}
\end{definition}

矩阵的迹是刻画矩阵性质的一个量,它的优点在于满足以下一些性质,从而成为了分析矩阵的变换的利器。


\subsection{性质}
\begin{theorem}{线性}
矩阵的求迹操作是\textbf{线性}的, 即对于域 $\mathbb{F}$ 上的方阵 $\mat A$,$\mat B$ 和域中元素 $c_1$,$c_2$,有
\begin{equation}
\opn{tr}(c_1\mat A+c_2\mat B) = c_1\opn{tr}(\mat A) + c_2\opn{tr}(\mat B)~,
\end{equation}
\end{theorem}
证明留作练习。

\begin{theorem}{交换性}\label{the_trace_1}
矩阵乘法的迹满足($\mat A$ 和 $\mat B$ 不必是方阵, 但要求乘积是方阵。 注意 $\mat A\mat B$ 的尺寸和 $\mat B\mat A$ 未必相同)
\begin{equation}\label{eq_trace_1}
\opn{tr}(\mat A\mat B) = \opn{tr}(\mat B\mat A)~.
\end{equation}
\end{theorem}
证明: 令\autoref{eq_trace_1} 中 $\mat A$ 为 $M\times N$ 的矩阵, $\mat B$ 为 $N\times M$ 的矩阵
\begin{equation}
\opn{tr}(\mat A\mat B) = \sum_{i=1}^M \sum_{k=1}^N a_{ik}b_{ki} = \sum_{i=1}^M \sum_{k=1}^N b_{ki}a_{ik} = \sum_{i=1}^N \sum_{k=1}^M b_{ik}a_{ki} = \opn{tr}(\mat B\mat A)~,
\end{equation}
证毕。

\begin{theorem}{}
相似矩阵的迹相等。
\end{theorem}
证明:根据交换性\autoref{eq_trace_1} 
$$\opn{tr}(\mat P^{-1}\mat A\mat P) = \opn{tr}(\mat A\mat P\mat P^{-1}) = \opn{tr}(\mat A)~,$$
证毕。

\begin{theorem}{}
矩阵的迹等于它的所有 $N$ 个本征值 $\lambda_i$ 相加, 如果某个本征值有 $n$ 重简并, 就视为 $n$ 个本征值
\begin{equation}
\opn{tr}(\mat A) = \sum_{i=0}^N \lambda_i~.
\end{equation}
\end{theorem}


\begin{exercise}{}\label{exe_trace_1}
对于矩阵 $\mat A$,若记其转置为 $\mat A\Tr$,共轭为 $\overline{\mat A}$,厄米共轭为 $\mat A^\dagger$,那么有:
\begin{equation}
\opn{tr}(A\Tr)=\opn{tr}(A), \qquad \opn{tr}(\overline{\mat A})=\overline{\opn{tr}(\mat A)}~.
\end{equation}
由此可得推论:$\opn{tr}(\mat A^\dagger)=\opn{tr}(\overline{\mat A\Tr})=\overline{\opn{tr}(\mat A)}$。
\end{exercise}

\begin{example}{}\label{ex_trace_1}
$\opn{tr}(AA^\dagger)=0$ 当且仅当 $\opn{tr}(A)=0$。
\end{example}

\subsection{矩阵的迹的应用实例}

\begin{exercise}{}
对于域 $\mathbb{C}$ 上的方阵 $\mat A$,如果 $\mat A^2=\mat A\mat A^\dagger$,求证 $\mat A=\mat A^\dagger$。

提示:使用\autoref{ex_trace_1} 的结论,证明 $\mat A-\mat A^\dagger=0$ 即可。
\end{exercise}
