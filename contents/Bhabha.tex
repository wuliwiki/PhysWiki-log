% Bhabha 散射
% keys bhabha 散射|正负电子对湮灭
% license Xiao
% type Tutor

Consider the process $e^+(p)e^+(k)\rightarrow \gamma(p')\gamma(k')$
\begin{equation}
\begin{aligned}
i\mathcal{M} =(-ie)^2 \varepsilon^{*\lambda}_\mu(p') \varepsilon^{*\lambda'}_\nu(k') \left( \bar v^r(k)\gamma^\nu \frac{i({\not p}-{\not p}'+m)}{(p-p')^2-m^2} \gamma^\mu u^s(p) 
+
 \bar v^r(k)\gamma^\mu \frac{i({\not p}-{\not k}'+m)}{(p-k')^2-m^2} \gamma^\nu u^s(p) 
\right)~.
\end{aligned}
\end{equation}
Now we calculate the scattering amplitude. When we sum over $\lambda,\lambda'$, we use
\begin{equation}
\begin{aligned}
\sum_{\lambda=1,2}\varepsilon_\mu^{*\lambda} \varepsilon_{\mu'}^{\lambda}=-g_{\mu\mu'} ~.
\end{aligned}
\end{equation}
\begin{equation}
\begin{aligned}
\overline{\sum}|\mathcal{M}|^2&=\frac{e^4}{4}\left[
	\frac{1}{[(p-p')^2-m^2]^2}\text{Tr}[({\not k}-m)\gamma^\nu ({\not p}-{\not p}'+m)\gamma^\mu ({\not p}+m)\gamma_{\mu}({\not p}-{\not p}'+m)\gamma_{\nu}]
	\right.
	\\
	&\quad +
	\frac{1}{[(p-k')^2-m^2]^2}\text{Tr}[({\not k}-m)\gamma^\nu ({\not p}-{\not k}'+m)\gamma^\mu ({\not p}+m)\gamma_{\mu}({\not p}-{\not k}'+m)\gamma_{\nu}]\\
	&\quad +
	\frac{1}{[(p-p')^2-m^2][(p-k')^2-m^2]} 
	\text{Tr}[({\not k}-m)\gamma^\nu ({\not p}-{\not p}'+m)\gamma^\mu ({\not p}+m)\gamma_\nu ({\not p}-{\not k}'+m)\gamma_\mu]\\
	&\quad \left.+
	\frac{1}{[(p-p')^2-m^2][(p-k')^2-m^2]}
	\text{Tr}[({\not k}-m)\gamma^\mu({\not p}-{\not k}'+m)\gamma^\nu({\not p}+m)\gamma_\mu ({\not p}-{\not p}'+m)\gamma_\nu]\right]~.
\end{aligned}
\end{equation}
Take the third line as an example:
\begin{equation}
\begin{aligned}
&\text{Tr}[({\not k}-m)\gamma^\nu ({\not p}-{\not p}'+m)\gamma^\mu ({\not p}+m)\gamma_\nu ({\not p}-{\not k}'+m)\gamma_\mu]
\\
&=\text{Tr}[({\not k}-m)\gamma^\nu (2p^\mu - {\not p}'\gamma^\mu)({\not p}+m)(2p_\nu-\gamma_\nu {\not k}')\gamma_\mu]\\
&=-32(p\cdot k)(p'\cdot k') -16m^2p'\cdot k'+[32(k\cdot p)(k'\cdot p)-16m^2k\cdot k']\\
&\quad +[32(k\cdot p)(p'\cdot p)-16m^2k\cdot p']
+32m^2k'\cdot p+32m^2p\cdot p'+16m^2k\cdot p-16m^4\\
&=8m^2(p+k)^2-32m^4~.
\end{aligned}
\end{equation}
Finally we have
\begin{equation}
\begin{aligned}
	\overline {\sum}|\mathcal{M}|^2&= 8e^4 \left[
	\frac{1}{4(p\cdot p')^2}((k\cdot p')(p\cdot p')
	+m^2(k\cdot p'+2p\cdot p'-k\cdot p)-2m^4)
	\right.\\
	&\quad +
	\frac{1}{4(p\cdot k')^2}
	((k\cdot k')(p\cdot k')
		+m^2(k\cdot k'+2p\cdot k'-k\cdot p)-2m^4)\\
	&\left.\quad +
	\frac{1}{4(p\cdot p')(p\cdot k')} 
	\left(
	m^2(p\cdot p'+p\cdot k')-2m^4
	\right)
	 \right]\\
	&= 2e^4 \left[
	\frac{k\cdot p'}{p\cdot p'}+\frac{m^2}{p\cdot p'}-\frac{m^4}{(p\cdot p')^2}\right.\\
	&\quad +\frac{p\cdot p'}{p\cdot k'}+\frac{m^2}{p\cdot k'}-\frac{m^4}{(p\cdot k')^2}\\
	&\left.\quad +
	m^2\left(\frac{1}{p\cdot k'}+\frac{1}{p\cdot p'}\right)-\frac{2m^4}{(p\cdot p')(p\cdot k')}
	\right]\\
	&=2e^4\left[\frac{p\cdot k'}{p\cdot p'}+\frac{p\cdot p'}{p\cdot k'}+2m^2\left(\frac{1}{p\cdot p'}+\frac{1}{p\cdot k'}\right)-m^4\left(\frac{1}{p\cdot p'}+\frac{1}{p\cdot k'}\right)^2\right]~.
\end{aligned}
\end{equation}
