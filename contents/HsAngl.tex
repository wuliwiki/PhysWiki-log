% 弧度制与任意角(高中)
% keys 高中|角|弧度|方向角|任意角|弧度制
% license Xiao
% type Tutor

\pentry{几何与解析几何初步\nref{nod_HsGeBa}}{nod_2571}

任意角和弧度制在高中数学中属于基础概念。类似于“实数”,尽管在后续课程中起着重要作用,但许多教材在讲解时往往只是定义角的正负方向,并给出它们与坐标系的关系,而缺乏直观的动机和背景,使得学生在初学时难以理解其意义。而关于弧度制,传统教科书在介绍时,通常直接抛出弧度与角度之间的转化公式,而没有详细解释引入弧度制的必要性,这让许多人在初次接触弧度制时感到困惑:“为什么需要弧度制?”。另外,日常生活中以“度”作为单位描述角度早已深入人心,且这些角度通常是简单的整数,而弧度值却常以分数乘以$\pi$的形式出现,其表达的复杂性在视觉上与角度形成了鲜明对比,这进一步加剧了初学者对弧度制的抗拒心理。

本文针对这些问题进行补充,不仅介绍弧度制和任意角的定义,还会探讨它们的必要性和直观理解,使读者不会有“空降知识”的感觉,而是能够循序渐进地理解这些概念为何被引入,以及它们如何在数学中发挥作用。

\subsection{为什么采用弧度制?}

在日常生活中,角的使用主要集中在两种情境:一是描述两条直线之间的倾斜关系,二是通过圆心角来分割圆周。在表达倾斜的角度时,“度分秒”系统十分方便,它精细且规整,足以满足日常使用的需求。对于圆的分割,由于这一操作在人们的生活中极为常见,通常并不需要精确量取角度值,因而人们往往忽略了这种分割实际上依赖于角度的概念。

早在约公元前2000年,古巴比伦人选择用$360^\circ$来表示圆周的完整度量,这一选择可能与360的独特性质有关——360是最小的、能够被除7以外所有一位数整除的整数。这一性质使得$360^\circ$在处理常见的“将圆分成几份”的问题时极为便利,既能满足分割的精细需求,又避免了度量数值过大而难以操作的问题。但这种基于便利性的选择,并不具有数学上的普遍性。例如,有人可能主张,为了使圆能够被分成7等份且每一份的度数为整数,可以采用 $2520^\circ$ 作为圆周的度量。这样的选择在理论上完全可行。此外,刚接触度分秒的初中生可能会感到困惑:既然现代社会普及了十进制,为何仍然沿用六十进制的度分秒?他们或许会想,若直接将圆周设定为 $100^\circ$,不是更简单直观吗?这种随意性与英制单位、市制单位,甚至曾经的某些公制单位类似,它们都是根据人们的习惯和实际需求设定的。

然而,在数学领域,这种依赖习惯的单位定义可能会带来问题。数学需要一种普适且逻辑严密的方式来定义角的度量,以避免随意性,同时在各种情境下保持简洁。此外,数学计算通常涉及无单位的纯数字,而非带有单位的量,由此问题在公元6世纪时被印度数学家\textbf{阿耶波多(Aryabhata)}发现。他在研究正弦函数时遇到了一个不知如何处理的情况\footnote{\href{https://www.hanspub.org/journal/PaperInformation?paperID=71063&utm_source=chatgpt.com}{马瑞芳, 库在强. 数学史融入弧度制的教学设计研究[J]. 教育进展, 2023, 13(8): 5911-5919.}}:
\begin{equation}
60^\circ+\sin30^\circ~.
\end{equation}
其中,$\sin 30^\circ$是一个无单位的实数,而$60^\circ$则是一个带有角度单位的量。这种单位的不一致使得它们无法直接进行代数运算(例如加法或乘法),类似于“$1\text{元} + 2$”,使人无法理解结果的意义。此外,随着数学研究的深入,人们还需要计算角$\alpha$的平方、立方等幂次以及它们的和。如果继续使用“度”为单位,这将引入诸如$\circ^2$和$\circ^3$等奇怪的单位。这些实际需求促使数学家思考如何定义一种新的角度度量方式,以满足以下要求:
\begin{enumerate}
\item 能够将角度转化为无单位的量,或者单位是“1”的量,从而便于代数运算。
\item 与传统的“度分秒”系统具有明确的换算关系。
\item 定义应独立于历史、习惯,具有普适性和数学上的严密性,并能在复杂分析、物理学等领域中表现出自然的适用性。
\end{enumerate}

在这个问题被发现之前,人们已经有了分割圆的经验。由于生活中容易量取长度,人们常将角度的分割转化为长度的分割,从而降低测量的难度。尽管,现在人们知道圆周的长度通常是一个无理数,这种方法事实上也是一种近似。然而,这一思路为角度测量问题的解决提供了重要启发。

阿耶波多通过选择用同一单位度量圆的半径和圆周,实际上提出了一个与现代弧度制几乎一致的概念。1748年,瑞士数学家\textbf{欧拉(Euler)}在其著作《无穷小分析概论(Introduction to the Analysis of the Infinite)》中,正式定义了以圆的半径作为弧长的度量单位。下面看看这样的定义可以推知些什么:

已知圆周长度 $C$ 与半径 $r$ 的比值是一个固定值,由于以圆的半径作为弧长的度量单位,这就得到了算数式:
\begin{equation}
C=2\pi r~.
\end{equation}
而一段圆弧$l$如果是周长的$\displaystyle\frac{n}{m}$,那么,它自然就可以表示为:
\begin{equation}
l=\frac{n}{m}C=\frac{2n}{m}\pi r~.
\end{equation}
因此,弧长 $l$ 与半径 $r$ 之间建立了一一对应的关系。在\aref{圆的相关概念}{sub_HsGeBa_1}中提到,每段弧长都对应着一个圆心角。

由于所有的圆都是相似的,对应同一圆心角的不同弧长之间的差异仅来源于圆的半径 $r$ 的不同。如果希望使新的圆心角度量方法不受半径 $r$ 的影响,需要消除半径 $r$ 的作用。将 $r$ 移到等式的左侧,可以得到:
\begin{equation}\label{eq_HsAngl_1}
\frac{l}{r}=\frac{2n\pi}{m}~.
\end{equation}
这样,对于占圆周相同比例的圆弧,不论圆的半径 $r$ 多大,右侧的值始终保持不变。

\subsection{弧度制}

如果将\autoref{eq_HsAngl_1} 右侧的值作为弧对应的圆心角的度量方式,那么通过定义“以圆的半径作为弧长的度量单位”,人们得到了一个新的角度度量方法,这种方法满足之前提出的所有要求:
\begin{enumerate}
\item 它是一个无单位的纯数值。
\item 由于“度”在定义时采用了类似的按比例等分的原则,若规定整圆的圆心角为一个固定值,这种新的方式可以实现与传统的度分秒系统的快速转换。
\item 它仅依赖于常数 $\pi$。
\end{enumerate}
这种新的角度度量方法也就是\textbf{弧度制(radian measure)}。

\begin{definition}{弧度}
对于任意圆,圆的弧长 $l$ 与半径 $r$ 存在固定比例关系,即:
\begin{equation}
\theta  = \frac{l}{r}~.
\end{equation}
称$\theta$为弧所对圆心角的\textbf{弧度(radian)}。
\end{definition}

尽管之前说过$\theta$是个数,但通常在物理上根据习惯,为了表示他是与角相关的量,会在后面加上单位$\Si{rad}$,读作弧度。注意,这只是为了在物理计算中与其他单位复合,表示它是角,本质上还是\enref{无量纲的}{USD}。在数学研究中,一般也不会写$\Si{rad}$。

根据圆的周长公式,圆周对应的圆心角为:
\begin{equation}
\frac{C}{r}=2\pi~.
\end{equation}



也就是说,弧度中与$360^\circ$对应的是$2\pi$。根据上面的原理可以得到$2\pi\Si{rad}$与$360^\circ$对应,而根据定义$0\Si{rad}$与$0^\circ$对应。若同一个角为$a^\circ$与$\theta\Si{rad}$通过对应关系化简可以得到二者的换算方法:
\begin{equation}
\theta=\frac{a}{180}\pi\qquad a=\frac{\theta}{\pi}\times180~.
\end{equation}

由于弧度制消除了不同半径带来的影响,因此一般只研究\textbf{单位圆(unit circle)},即半径为$1$的圆。可以认为,弧度就是圆心角在单位圆上所对的弧长。在单位圆中长度与半径相等的弧所对的圆心角就是$1 \Si{rad}$。

弧度制的引入使得圆的许多性质表达得更加简洁。例如,设半径为$r$的圆弧对应的圆心角为$\theta$(以弧度计),则有:
\begin{itemize}
\item 由弧度的定义可得弧长 $l = r\theta$;
\item 扇形面积为 $\displaystyle S = \frac{1}{2} r^2 \theta$,或用弧长表示为 $\displaystyle S = \frac{1}{2} l r$。
\end{itemize}

值得注意的是,扇形面积公式 $\displaystyle S = \frac{1}{2} l r$ 与三角形面积公式 $\displaystyle S = \frac{1}{2} bh$ 形式相同。这可以通过几何解释来直观理解。设想一个角度 $\theta$ 极小的扇形,其形状近似于一个三角形,其中弧长 $l$ 近似为底边,高为半径\footnote{因为半径与弧的切线垂直,而角度很小时,弧近似于切线。} $r$。更一般地,可以将扇形视为由无数个极窄的三角形组成,每个三角形的高均为 $r$,而底边依弧线分布。由于这一极限过程,扇形面积公式自然呈现出三角形面积的形式。需要注意的是,此处的分析是定性的,旨在帮助理解公式的直观来源。


\subsection{任意角}

在之前的学习中,角被定义为由一个公共端点引出的两条射线之间的夹角,并根据大小分类为锐角、直角、钝角、平角和周角。然而,当角度超过 $180^\circ$ 时,人们常采用等价的较小角度来表示,例如用 $360^\circ$ 减去该角度。例如,一个 $210^\circ$ 的角常被描述为 $360^\circ - 210^\circ = 150^\circ$,这样可以将角限制在 $180^\circ$ 以内,使其仍然符合常见的锐角、直角和钝角的分类。这种方法在仅关注角度大小时十分直观。

然而,在跳水运动员的空中翻转、机械部件的旋转等实际场景中,角度不仅用于测量静态夹角,还需要描述动态的旋转过程。这时,人们通过抽象这个过程,通过旋转的视角来理解角:设定一条\textbf{始边(initial side)},然后围绕其端点旋转,最终到达\textbf{终边(terminal side)}。旋转的角度即为从初始边到终边的变化量。这种定义方式不仅兼容原有的角度概念,还能直接地描述旋转现象。

这种角的定义解决了旋转的问题,但如果仍然使用 $0^\circ$ 到 $360^\circ$ 之间的角度来表示旋转,可能会导致信息丢失,主要体现在以下两种情况:
\begin{enumerate}
\item 当旋转角度超过 $180^\circ$ 时,尽管逆时针旋转 $210^\circ$ 和顺时针旋转 $150^\circ$ 后终边位置相同,但旋转路径不同。逆时针旋转 $210^\circ$ 经过了更大的角度,而顺时针旋转 $150^\circ$ 则是较短路径。因此,仅用终边位置描述角度,无法准确体现旋转的过程。
\item 设想一个车轮上的某个辐条在不断旋转。虽然辐条会多次回到相同的位置,但它实际经历了不同的旋转圈数,或者循环。如果仅用 $0^\circ$ 到 $360^\circ$ 之间的角度来表示,就无法区分它转动的圈数或循环的次数。
\end{enumerate}

由此可见,在修改角的定义后,需要扩展角的表示方法,使其不仅能表示一个固定的几何量,还能描述物体的连续运动。这类似于数系的拓展,例如,引入实数后,数学系统可以表示 $\sqrt{2}$ 这样的无理数,而不仅仅局限于有理数。下面分别针对每个问题讨论:
\begin{enumerate}
\item 区分旋转方向。这比较容易想,与整数区分正负的方式类似,把逆时针、顺时针对应成正数、负数,静止不动对应成$0$。至于该把逆时针对应为正,还是顺时针对应为正呢?通常,约定选择逆时针为正方向。这主要是由于笛卡尔坐标系的建立方式。在平面直角坐标系中,$x$ 轴向右,$y$ 轴向上,从 $x$ 轴正方向旋转到 $y$ 轴正方向的方向是逆时针,因此自然将其定义为正方向。这一约定更深层的原因是可以使得线性代数、向量运算、叉积、电磁学、计算机图形学等领域保持统一,也更符合许多自然现象的规律。
\item 区分旋转圈数。两种方案可以实现这个目的:一是引入新的单位“圈”,规定$1\text{圈}=360^\circ$;二是直接取消“角度必须在周角以内”的限制,让角度自由扩展。前者虽然符合日常语言习惯,比如“转了一圈”,但在数学上并不自然,就像角度制计算中额外转化分、秒一样,圈的引入会让计算变得更加离散,不利于运算的简化。此外,实际应用中角度往往是连续变化的,而“圈”作为单位太大,无法精确描述微小增量,因此并无额外优势。从数学的发展来看,更合理的做法是让角度可以无限扩展,以保持运算的封闭性和可计算性,使计算更加直接,无需额外拆分和归约,而不是人为增加一个新单位。
\end{enumerate}

综上,数学引入了“任意角”的概念。

\begin{definition}{任意角}
一条射线绕其端点旋转形成的图形称为角,旋转前的射线称为\textbf{始边(initial side)},旋转后的射线称为\textbf{终边(terminal side)}。规定:
\begin{itemize}
\item 按逆时针方向旋转形成的角称为正角;
\item 按顺时针方向旋转形成的角称为负角;
\item 若没有旋转,则称为零角。
\end{itemize}
这样定义的角称作\textbf{任意角}或\textbf{有向角(directed angle)}。
\end{definition}

这一扩展不仅让角度的表示方式更加统一,还让数学计算更自然,为后续的三角函数、旋转变换等概念奠定了基础。

早期的三角学发展是基于单位圆和弧长的概念,单位圆的思路使得角度可以与弧长直接对应,而解除角度的限制,使其能够自由扩展,与实数向两侧延伸的方式保持一致。这种视角下,角度不仅仅是几何图形中的静态量,而是可以连续变化的数值,因此弧度制成为了数学上描述任意角的自然选择。弧度制的本质是用单位圆上对应弧的长度来测量角度,使得角度的定义与数轴上的实数直接关联,而不依赖于人为设定的度数系统。

综上,角的概念推广后,在弧度制下,角的集合与实数集$\mathbb{R}$之间建立了一一对应关系。弧度制与任意角的结合,使得角的大小可以用十进制实数自然表示,使得实数不仅与线段上的点一一对应,同时也与角度大小一一对应,从而构建起更具一般性的数学框架。

为方便研究,通常将角的顶点放在坐标系原点,始边放在$x$轴正方向上。这时,角的终边(除端点外)所在的象限决定了该角的象限归属。例如,若终边落在第一象限,则称该角为第一象限角,其他象限类推。

另外,所有与某个角 $\alpha$ 终边相同的角,可构成一个集合:
\begin{equation}
S = \begin{Bmatrix} \beta|\beta=\alpha+2k\pi,k \in \mathbb{Z} \end{Bmatrix}~.
\end{equation}

这表示任何角 $\alpha$ 只要增加或减少整数倍的 $2\pi$(即完整旋转若干圈),其终边位置保持不变。这个集合反映了角的周期性,在三角函数、旋转变换等问题中具有重要作用。

