% 三角恒等变换(高中)
% keys 高中|三角恒等变换|三角函数|恒等式|勾股定理|三角恒等式
% license Xiao
% type Tutor

\begin{issues}
\issueDraft
\issueOther{正在与\enref{三角恒等式}{TriEqv}合并}
\addTODO{合并删除相同的部分}
\addTODO{介绍每部分的应用、推导、记忆及相关考量}
\addTODO{超出高中范围的知识写在最后}
\addTODO{移除原本三角恒等式的外部链接}
\end{issues}

\pentry{三角函数(高中)\nref{nod_HsTrFu}}{nod_e77a}

\subsection{两个基本公式}
\begin{equation}
\sin^2\alpha + \cos^2\alpha = 1~,
\end{equation}
\begin{equation}
\frac{\sin\alpha}{\cos\alpha} = \tan\alpha~.
\end{equation}

\subsection{两角和与两角差}
\begin{equation}\label{eq_HsAnTf_5}
\sin(\alpha + \beta) = \sin\alpha \cos\beta + \cos\alpha \sin\beta~,
\end{equation}
\begin{equation}\label{eq_HsAnTf_6}
\sin(\alpha - \beta) = \sin\alpha \cos\beta - \cos\alpha \sin\beta~,
\end{equation}
\begin{equation}\label{eq_HsAnTf_4}
\cos(\alpha + \beta) = \cos\alpha \cos\beta - \sin\alpha \sin\beta~,
\end{equation}
\begin{equation}\label{eq_HsAnTf_3}
\cos(\alpha - \beta) = \cos\alpha \cos\beta + \sin\alpha \sin\beta~,
\end{equation}
\begin{equation}\label{eq_HsAnTf_7}
\tan(\alpha + \beta) = \frac{\tan\alpha+\tan\beta}{1-\tan\alpha \tan\beta}~,
\end{equation}
\begin{equation}\label{eq_HsAnTf_8}
\tan(\alpha - \beta) = \frac{\tan\alpha - \tan\beta}{1+\tan\alpha \tan\beta}~.
\end{equation}

\subsection{二倍角公式}
\begin{equation}
\sin2\alpha = 2\sin\alpha \cos\alpha ~,
\end{equation}
\begin{equation}
\cos2\alpha = \cos^2\alpha - \sin^2\alpha = 1 - 2\sin^2\alpha = 2\cos^2\alpha -1~,
\end{equation}
\begin{equation}
\tan2\alpha = \frac{2\tan\alpha}{1-\tan^2\alpha}~.
\end{equation}

\subsection{半角公式}
\begin{equation}
\sin\frac{\alpha}{2} = \pm\sqrt{\frac{1-\cos\alpha}{2}}~,
\end{equation}
\begin{equation}
\cos\frac{\alpha}{2}= \pm\sqrt{\frac{1+\cos\alpha}{2}}~,
\end{equation}
\begin{equation}
\tan\frac{\alpha}{2} = \pm\sqrt{\frac{1-\cos\alpha}{1+\cos\alpha}} = \frac{\sin\alpha}{1+\cos\alpha} = \frac{1-\cos\alpha}{\sin\alpha}~.
\end{equation}
注意正负号的选择需要根据 $\alpha$ 的具体取值判断。

\subsection{升幂公式}
\begin{equation}
\cos2\alpha + 1 = 2\cos^2\alpha~,
\end{equation}
\begin{equation}
1-\cos2\alpha = 2\sin^2\alpha~.
\end{equation}

\subsection{降幂公式}
\begin{equation}
\cos\alpha = \pm\sqrt{\frac{1+\cos2\alpha}{2}}~,
\end{equation}
\begin{equation}
\sin\alpha = \pm\sqrt{\frac{1-\cos2\alpha}{2}}~.
\end{equation}

\subsection{万能公式}
\begin{equation}
\sin2\alpha = \frac{2\sin\alpha \cos\alpha}{\sin^2\alpha + \cos^2\alpha} = \frac{2\tan\alpha}{1+\tan^2\alpha}~,
\end{equation}
\begin{equation}
\cos2\alpha = \frac{\cos^2\alpha-\sin^2\alpha}{\sin^2\alpha+\cos^2\alpha} = \frac{1-\tan^2\alpha}{1+\tan^2\alpha}~.
\end{equation}

\subsection{辅助角公式}
\begin{equation}
a\sin\alpha + b\cos\alpha = \sqrt{a^2+b^2}\sin(\alpha + \phi)~.
\end{equation}
注: $\tan\phi = \frac{b}{a}$

\subsection{证明}

可参考与本节内容相似的\autoref{sub_TriEqv_1}。

\subsubsection{两角和与两角差(旋转法)}

这一证法是先通过旋转法求出余弦的加法公式\autoref{eq_HsAnTf_3},然后进行简单变换得到剩下的加法公式。思路来自П. М. Котельников的学位论文\footnote{据刘培杰工作室出版的《世界著名三角学经典著作钩沉 平面三角卷I》第22节。笔者按此名字搜索,并未找到出处,特此声明。}。

\begin{figure}[ht]
\centering
\includegraphics[width=14cm]{./figures/75ec09f8befeb803.pdf}
\caption{旋转法示意图。左图和右图表示的是同样的坐标系,圆都是同一个单位圆。右图中所有图形和点都围绕坐标系原点顺时针旋转了$\beta$,从而使得$B$点落在$x$轴上。} \label{fig_HsAnTf_2}
\end{figure}


如\autoref{fig_HsAnTf_2} 所示,在单位圆上取两个角(不一定是图示的锐角)$\alpha$和$\beta$,与单位元相交得交点$A$和$B$。由于是单位圆,故可知$A$的坐标为$\pmat{x_A, y_A}=\pmat{\cos\alpha, \sin\alpha}$,$B$的坐标为$\pmat{x_B, y_B}=\pmat{\cos\beta, \sin\beta}$。由此可计算线段$AB$的长度,或者准确来说,长度的平方:
\begin{equation}\label{eq_HsAnTf_1}
\begin{aligned}
\abs{AB}^2 =& (x_A-x_B)^2+(y_A-y_B)^2\\
=& (\cos^2\alpha+\cos^2\beta-2\cos\alpha\cos\beta)+\\&(\sin^2\alpha+\sin^2\beta-2\sin\alpha\sin\beta)\\
=& 2\qty(1-\cos\alpha\cos\beta-\sin\alpha\sin\beta)~.
\end{aligned}
\end{equation}

注意这里利用了$\cos^2x+\sin^2x=1$恒等式,下面也一样。

接下来,把所有点和图形都围绕坐标原点,顺时针旋转$\beta$,得到右图。此时$A$的坐标变成了$\pmat{x'_A, y'_A}=\pmat{\cos(\alpha-\beta), \sin(\alpha-\beta)}$,$B$的坐标变成了$\pmat{x'_B, y'_B}=\pmat{1, 0}$。

同样地,计算线段$AB$的长度平方:
\begin{equation}\label{eq_HsAnTf_2}
\begin{aligned}
\abs{AB}^2 =& (x'_A-x'_B)^2+(y'_A-y'_B)^2\\
=& \qty(\cos^2(\alpha-\beta)+1-2\cos(\alpha-\beta))+\sin^2(\alpha-\beta)\\
=& 2(1-\cos(\alpha-\beta))~.
\end{aligned}
\end{equation}

\autoref{eq_HsAnTf_1} 和\autoref{eq_HsAnTf_2} 应相等,比较它们的最后一步即可得\autoref{eq_HsAnTf_3}。

计算$\cos\qty(\alpha-(-\beta))$即可得\autoref{eq_HsAnTf_4}。将$\sin x=\cos(x-\pi/2)$代入这两个余弦加法公式,即可得到正弦加法公式\autoref{eq_HsAnTf_5} 与\autoref{eq_HsAnTf_6}。再代入$\tan x=\sin x/\cos x$即可得正切的加法公式\autoref{eq_HsAnTf_7} 与\autoref{eq_HsAnTf_8}。



\subsubsection{两角和与两角差(几何矢量证法)}
\begin{figure}[ht]
\centering
\includegraphics[width=7cm]{./figures/f6ec01fcb652c036.png}
\caption{图示} \label{fig_HsAnTf_1}
\end{figure}
设 $\alpha$、$\beta$ 对应的单位向量分别为 $a(\cos\alpha,\sin\alpha)$、$b(\cos(\alpha+\beta),\sin(\alpha+\beta))~.$\\
设 $a$ 与其垂直的单位向量 $m(-\sin\alpha,\cos\alpha)$ 为基向量,则
\begin{equation}
\begin{aligned}
b &= \cos\beta \cdot a + \sin\beta \cdot m \\
&= (\cos\beta \cos\alpha,\cos\beta \sin\alpha) + (-\sin\beta \sin\alpha,\sin\beta \cos\alpha) \\
&= (\cos\alpha \cos\beta-\sin\alpha \sin\beta,\sin\alpha \cos\beta + \cos\alpha \sin\beta)~.
\end{aligned}
\end{equation}
由此可得
\begin{equation}
\sin(\alpha+\beta) = \sin\alpha \cos\beta + \cos\alpha \sin\beta~,
\end{equation}
\begin{equation}
\cos(\alpha+\beta) = \cos\alpha \cos\beta - \sin\alpha \sin\beta~.
\end{equation}

用 $-\beta$ 代换 $\beta$ ,可得
\begin{equation}
\begin{aligned}
\sin(\alpha-\beta) &= \sin\alpha \cos(-\beta) + \cos\alpha \sin(-\beta)\\
&=\sin\alpha \cos\beta - \cos\alpha \sin\beta~,
\end{aligned}
\end{equation}
\begin{equation}
\begin{aligned}
\cos(\alpha-\beta) &= \cos\alpha \cos(-\beta) + \sin\alpha \sin(-\beta)\\
&=\cos\alpha \cos\beta + \sin\alpha \sin\beta~.
\end{aligned}
\end{equation}

由 $\tan\alpha = \frac{\sin\alpha}{\cos\alpha}$,得
\begin{equation}
\tan(\alpha+\beta) = \frac{\sin\alpha \cos\beta + \cos\alpha \sin\beta}{\cos\alpha \cos\beta - \sin\alpha \sin\beta}~.
\end{equation}
上下同除 $\cos\alpha\cos\beta$ ,得
\begin{equation}
\tan(\alpha+\beta) = \frac{\tan\alpha+\tan\beta}{1 - \tan\alpha\tan\beta}~.
\end{equation}

同理,可得
\begin{equation}
\tan(\alpha-\beta) = \frac{\tan\alpha - \tan\beta}{1 + \tan\alpha\tan\beta}~.
\end{equation}

\subsubsection{二倍角公式}
在两角和公式中,令 $\alpha = \beta$
\begin{equation}
\sin2\alpha = \sin\alpha \cos\alpha+\cos\alpha \sin\alpha = 2\sin\alpha \cos\alpha~,
\end{equation}
\begin{equation}
\cos2\alpha = \cos^2\alpha - \sin^2\alpha~,
\end{equation}
\begin{equation}
\tan2\alpha = \frac{2\tan\alpha}{1 - \tan^2\alpha}~.
\end{equation}

由 $\sin^2\alpha + \cos^2\alpha = 1$ ,可得
\begin{equation}
\cos^2\alpha = 1 - \sin^2\alpha~,
\end{equation}
\begin{equation}
\sin^2\alpha = 1 - \cos^2\alpha~.
\end{equation}

代入,可得
\begin{equation}
\cos2\alpha = 1 - \sin^2 - \sin^2 = 1 - 2\sin^2\alpha~,
\end{equation}
\begin{equation}
\cos2\alpha = \cos^2\alpha - (1 - cos^2\alpha) = 2\cos^2\alpha - 1~.
\end{equation}

\subsubsection{半角公式}
由余弦的二倍角公式,可得
\begin{equation}
2\sin^2\alpha = 1 - \cos2\alpha~,
\end{equation}
\begin{equation}
2\cos^2\alpha = 1 + \cos2\alpha~.
\end{equation}

用 $\frac{\alpha}{2}$ 代还 $\alpha$ ,可得
\begin{equation}
\begin{aligned}
2\sin^2\frac{\alpha}{2} &= 1 - \cos\alpha~,\\
\sin\frac{\alpha}{2}&= \pm\sqrt{\frac{1-\cos\alpha}{2}}~.
\end{aligned}
\end{equation}
\begin{equation}
\begin{aligned}
2\cos^2\frac{\alpha}{2} &= 1 + \cos\alpha~,\\
\cos\frac{\alpha}{2} &= \pm\sqrt{\frac{1+\cos\alpha}{2}}~.
\end{aligned}
\end{equation}





这里列出几个高中常见的三角函数恒等式。 以下用到的两个高中数学不常见的三角函数分别为 $\csc \alpha= 1/\sin \alpha$, $\sec \alpha = 1/\cos \alpha$, 分别读作 cosecant 和 secant。

\subsubsection{勾股定理}
\begin{equation}
\sin^2 \alpha + \cos^2 \alpha = 1~.
\end{equation}
等式两边同除 $\cos^2 \alpha$ 和 $\sin^2 \alpha$ 得
\begin{gather}
\tan^2 \alpha + 1 = \sec^2 \alpha~,\\
1 + \cot^2\alpha = \csc^2\alpha~.
\end{gather}

\subsubsection{两角和公式}
\begin{gather}
\sin(\alpha\pm \beta) = \sin \alpha\cos \beta \pm \cos \alpha\sin \beta~,\\
\cos(\alpha\pm \beta) = \cos \alpha\cos \beta \mp \sin \alpha\sin \beta~,\\
\tan(\alpha\pm \beta) = \frac{\tan \alpha \pm \tan \beta}{1 \mp \tan \alpha \tan \beta}~.
\end{gather}

\subsubsection{二倍角公式}

令\autoref{eq_TriEqv_1} 中 $\beta=\alpha$ 取上号得
\begin{gather}
\sin 2\alpha = 2\sin \alpha\cos \alpha~,\\
\cos 2\alpha = \cos^2 \alpha - \sin^2 \alpha~,\\
\tan 2\alpha = \frac{2\tan \alpha}{1 - \tan^2 \alpha}~.
\end{gather}

\subsubsection{降幂公式}
结合$\sin 2\alpha = 2\sin \alpha\cos \alpha$ 和 $\sin^2 \alpha + \cos^2 \alpha = 1$ 可以得到
\begin{gather}
\sin^2 \alpha = \frac12 (1- \cos 2\alpha) ~, \\
\cos^2 \alpha = \frac12 (1+\cos 2\alpha)~.
\end{gather}
由此可得半角公式
\begin{gather}
\sin\frac{ \alpha}{2} = \pm\sqrt{\frac{1-\cos \alpha}{2}}~,\\
\cos\frac{ \alpha}{2}= \pm\sqrt{\frac{1+\cos \alpha}{2}}~,\\
\tan\frac{ \alpha}{2} = \pm\sqrt{\frac{1-\cos \alpha}{1+\cos \alpha}} = \frac{\sin \alpha}{1+\cos \alpha} = \frac{1-\cos \alpha}{\sin \alpha}~.
\end{gather}
注意正负号的选择需要根据 $\alpha$ 所在的区间判断, 如果需要恒等式则两边取平方。

\subsubsection{和差化积公式}
\begin{gather}
\sin \alpha + \sin \beta = 2\sin\qty(\frac{\alpha + \beta}{2})\cos\qty(\frac{\alpha - \beta}{2})~,\\
\sin \alpha - \sin \beta = 2\sin\qty(\frac{\alpha - \beta}{2})\cos\qty(\frac{\alpha + \beta}{2})~,\\
\cos \alpha + \cos \beta = 2\cos\qty(\frac{\alpha+\beta}{2})\cos\qty(\frac{\alpha-\beta}{2})~,\\
\cos \alpha - \cos \beta = -2\sin\qty(\frac{\alpha+\beta}{2})\sin\qty(\frac{\alpha-\beta}{2})~.
\end{gather}

\subsubsection{积化和差公式}
根据上文的和差化积公式, 我们也可以直接写出积化和差公式
\begin{gather}
\sin \alpha\sin \beta = \frac12 [\cos(\alpha - \beta) - \cos
\cos \alpha\cos \beta = \frac12 [\cos(\alpha + \beta) + \cos(\alpha - \beta)]~,\\
\sin \alpha\cos \beta = \frac12 [\sin(\alpha + \beta) + \sin(\alpha - \beta)]~.
\end{gather}

\subsubsection{辅助角公式}
\begin{equation}
a\sin \alpha + b\cos \alpha = \sqrt{a^2+b^2}\sin(\alpha + \phi) \qquad \qty(\phi = \tan^{-1}\frac{b}{a})~.
\end{equation}

\subsection{证明}
\subsubsection{两角和公式}

如\autoref{fig_TriEqv_2}, 要证明\autoref{eq_TriEqv_1}, 令 $OB = 1$, 那么 $\sin(\alpha+\beta) = BD = AC + BE$, 而 $AC = OA \sin\alpha$, $OA = \cos\beta$; $BE = AB\cos\alpha$, $AB = \sin\beta$, 代入得 $\sin(\alpha+\beta) = \sin\alpha\cos\beta + \cos\alpha\sin\beta$。 注意当 $\alpha$ 或 $\beta$ 取其他任意值时, 重新画图同样可以证明该关系。 所以给 $\beta$ 取相反数, 就得到 $\sin(\alpha-\beta) = \sin\alpha\cos\beta - \cos\alpha\sin\beta$

要证明\autoref{eq_TriEqv_2}, $\cos(\alpha+\beta) = OD = OC - EA$, 而 $OC = OA\cos\alpha$, $OA = \cos\beta$; $EA = AB\sin\alpha$, $AB = \sin\beta$, 代入得 $\cos(\alpha+\beta) = \cos\alpha\cos\beta - \sin\alpha\sin\beta$, $\beta$ 取相反数得 $\cos(\alpha-\beta) = \cos\alpha\cos\beta + \sin\alpha\sin\beta$。 证毕。

\subsubsection{两角和公式(几何矢量)}
把以上过程用\enref{几何矢量}{GVec}语言可以表达得更自然。 令 $\uvec x, \uvec y$ 分别是\autoref{fig_TriEqv_2} 直角坐标的单位矢量, $OA$ 方向的单位矢量为 $\uvec a$, $AB$ 方向的单位矢量为 $\uvec b$。 易得
\begin{gather}
\uvec a = \cos\alpha\ \uvec x + \sin\alpha\ \uvec y~,\\
\uvec b = -\sin\alpha\ \uvec x + \cos\alpha\ \uvec y~.
\end{gather}
同样以 $O$ 为原点, $\uvec a, \uvec b$ 可以看成 $x$-$y$ 直角坐标系旋转后的坐标系中的单位矢量。 令 $OB$ 矢量为 $\uvec u$, 那么 $\uvec u = \cos\beta\ \uvec a + \sin\beta\ \uvec b$, 把以上两式代入得
\begin{equation}
\uvec u = (\cos\alpha\cos\beta - \sin\alpha\sin\beta)\uvec x + (\sin\alpha\cos\beta + \cos\alpha\sin\beta)\uvec y~,
\end{equation}
这就同时证明了两个两角和公式。 证毕。

\subsubsection{和差化积}
以\autoref{eq_TriEqv_9} 为例, $\cos \alpha, \cos \beta$ 和 $\cos \alpha + \cos \beta$ 分别等于\autoref{fig_TriEqv_1} 中矢量 $\bvec A, \bvec B$ (令它们的模长为 1) 和 $\bvec A + \bvec B$ 在水平方向的投影长度, 而 $\bvec A + \bvec B$ 在水平方向的投影长度等为 $\abs{\bvec A + \bvec B}\cos[(\alpha+\beta)/2]$, 其中 $\abs{\bvec A + \bvec B} = 2\cos [(\beta-\alpha)/2]$, 代入可得\autoref{eq_TriEqv_9}。 利用 $\bvec A + \bvec B$ 在竖直方向的投影可得\autoref{eq_TriEqv_7}, 把\autoref{eq_TriEqv_7} 和\autoref{eq_TriEqv_9} 中的 $\beta$ 分别替换成 $-\beta$ 和 $\beta+\pi$ 可推导出\autoref{eq_TriEqv_8} 和\autoref{eq_TriEqv_10}。
