% 中国科学院 2006 年考研普通物理 A 卷(甲)
% keys 中科院|物理|考研
% license Copy
% type Tutor

\textbf{声明}:“该内容来源于网络公开资料,不保证真实性,如有侵权请联系管理员”


\begin{enumerate}
\item 一带有电荷为$q$、质量为$m$的小球,悬于一不带电绝缘细丝线一端。线的另一端与一无穷大竖直带电导体平板相连,处于平衡状态时细线与平板成$30$°角,如图1所示。 试问此时带电平板的表面电荷密度 $\sigma$ 为多少?
\begin{figure}[ht]
\centering
\includegraphics[width=6cm]{./figures/74b710ea7da90c3f.png}
\caption{} \label{fig_ZKYP06_1}
\end{figure}
\item 如图2所示,电源电动势  $\varepsilon_1=3V,\varepsilon_2=12V$ ,其内阻均可忽略。 $R_1=8\Omega,R_2=4.4\Omega,R_3=2\Omega$  。求:\\
(1)$K$ 断开时,$A$  点的电势  $V_A$=?\\
(2)$K$合上后,$A$  点的电势又为多少?
\begin{figure}[ht]
\centering
\includegraphics[width=8cm]{./figures/a13c2fce2c72f7ec.png}
\caption{} \label{fig_ZKYP06_2}
\end{figure}
\item 延长线过圆心的两根长直导线与均匀金属圆环相接于 $A,B $ 两点,如图3所示。直导线上通有直流电 $I$。求:\\
(1)通过圆弧 $l_1$ 段的电流 $I_1 $与通过圆弧4 段的电流 $I_2$ 的比值;\\
(2)求环中心 $O$ 处的磁感应强度 $\vec B$ 。
\begin{figure}[ht]
\centering
\includegraphics[width=8cm]{./figures/867e38851f6d0799.png}
\caption{} \label{fig_ZKYP06_3}
\end{figure}
\item 如图4所示,长 $L$ 质量$M$的平板静止放在光滑水平面上,质量$m$的小木块以水平初速 $v_0$滑入平板上表面。已知小木块与平板的上表面间摩擦系数为 $\mu $。 试求:\\
(1)小木块不会从平板上表面滑离的条件。\\
(2)在满足(1)的条件下,平板速度的表达式。
\begin{figure}[ht]
\centering
\includegraphics[width=8cm]{./figures/f507397b0a4e40e1.png}
\caption{} \label{fig_ZKYP06_4}
\end{figure}
\item 假设:”神州六号“飞船的质量为 $m$ ,正在绕地球作圆周运动,圆半径为 $R_0$ ,速率为 $V_0$ 。某一时刻接到地面指令要求变轨飞行后火箭点火,给“神六”增加了向外的径向速度分量$v_r(<v_0)$,于是飞船的轨迹发生了变化。\\
(1)试推导出飞船在任意轨道上受到的引力表达式$\abs{F}=\frac{mv_0^2R_0}{r^2}$,其中$r$为“神州六号”飞船到地球球心的距离;\\
(2) 试用$ R_0$,$v_0$以及$v_r$,给出飞船变轨后的椭圆轨道近地点和远地点表达式。
\item 如图5所示,有一密度均匀的扇形板质量为$M$,半径为$R$,扇形角度为$\theta=\pi/3$,竖直悬挂在支架上,可绕垂直于纸面的轴无摩擦地摆动。现有一颗质量为$m$的子弹在支架下方距支点竖直距离为 $2R/3$的位置处以水平速度 $v_0$射向静止扇形板后,停留在板上$A$点处。求:\\
(1)子弹射入前扇形板的转动惯量;\\
(2)子弹射入后的瞬间,扇形板的转动角速度;\\
(3)要使扇形板的摆动角度超过$\pi/3$,$v_0$至少要多大?
\begin{figure}[ht]
\centering
\includegraphics[width=6cm]{./figures/75db0d797b579b17.png}
\caption{} \label{fig_ZKYP06_6}
\end{figure}
\item 使用增透膜可以减少玻璃的反射。在玻璃基片(折射率为$n_1$)上镀有一层折射率为$n_2$,的薄膜$(1<n_1<n_2)$,对特定波长$\lambda$的单色光由空气垂直入射到薄膜上。\\
(1)问:在薄膜两个界面上一次反射回空气的反射光中,哪一束存在半波损失?
(2)求:具有最佳透射效果的最小镀膜厚度$d$。
\item 能量 $12.9eV$ 的电子碰撞基态氢原子,已知氢原子基态电离能是$ 13.6eV$。不考虑精细结构,求:\\
(1)氢原子可能的激发态;\\
(2)受激发的氢原子向低能级跃迁时发出的光谱线能量.
\end{enumerate}