% Python 路径笔记
% license Usr
% type Note

\begin{issues}
\issueDraft
\end{issues}

\begin{itemize}
\item \verb`os.getcwd()` 获取当前路径
\item \verb`os.chdir('路径')` 改变当前路径 
\item \verb`for root, dirs, files in os.walk('文件夹'): ...` 可以在 \verb`文件夹` 中遍历整个文件树。 默认从顶到底遍历, 如果添加参数 \verb`topdown=False` 则从底到顶。
\item \verb`root,dirs,files` 分别是当前处理的路径、 该路径下的所有文件夹和所有文件。 其中 \verb`root` 开始的部分和 \verb`'文件夹'` 相同。 如果后者是相对路径前者也会是。
\item \verb`os.path.basename('路径')` 获取路径中的文件或最后的文件夹的名字。 \verb`os.path.dirname('路径')` 获取前面部分
\item \verb`os.path.abspath('路径')` 获取绝对路径
\item \verb`os.path.realname('路径')` 获取正则化的路径,也就是不含任何软链以及 \verb`.` 和 \verb`..` 的绝对路径。
\item \verb`os.path.relpath(路径2, 路径1)` 可以显示 \verb`路径2` 相对 \verb`路径1` 的相对路径
\item \verb`os.path.join(路径1, 路径2)` 可以把两个路径合并,后者必须是相对路径。
\item \verb`root, ext = os.path.splitext(字符串)` 可以把 \verb`字符串` 根据最后一个英文句号拆分成 \verb`root` 和 \verb`ext` 两部分(不含句号),如果没有英文句号,则 \verb`ext` 为空。
\end{itemize}
