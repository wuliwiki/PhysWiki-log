% 阻抗、电抗
% license Usr
% type Tutor

\begin{issues}
\issueDraft
\end{issues}

\pentry{振动的指数形式\nref{nod_VbExp}, 交流电, 交流电的复数形式, 电阻 欧姆定律\nref{nod_Resist}}{nod_434f} % \addTODO{链接}

\footnote{参考 Wikipedia \href{https://en.wikipedia.org/wiki/Electrical_impedance}{相关页面}。}\textbf{阻抗(electrical impedance)} 的定义和电阻\upref{Resist} 类似, 都使用电压 $V$ 除以电流 $I$, 不同的是, 这 $V, I$ 都是复数表示的正余弦交流电, 即
\begin{equation}
V = \abs{V} \E^{\I \phi_V - \I \omega t}~,
\qquad
I = \abs{V} \E^{\I \phi_I - \I \omega t}~.
\end{equation}
那么阻抗的定义为(通常用大写 $Z$ 表示)
\begin{equation}
Z = \frac{V}{I} = \frac{\abs{V}}{\abs{I}} \E^{\I (\phi_V - \phi_I)}~.
\end{equation}
这里的除法是复数相除\upref{CplxNo}, 几何意义上就是把两个复数的模长相除, 辐角相减。 注意电流和电压的方向按照被动符号规定(\autoref{sub_Resist_1}~\upref{Resist}), 即先规定一个正方向, 电流延该方向为正, 反之为负, 电势延正方向下降为正, 反之为负。

那么欧姆定律(\autoref{eq_Resist_5}~\upref{Resist})的复数拓展就是
\begin{equation}
V = IZ~.
\end{equation}
我们把阻抗的实部叫做\textbf{电阻(resistance)}, 把虚部叫做\textbf{电抗(reactance)}。

\begin{example}{电阻器}\label{ex_impeda_1}
根据定义, 普通电阻的阻抗就是其电阻值 $R$。 这是因为在交流电作用下, 电阻两端的电压和电流相位相同。
\end{example}

\begin{example}{电容器}
电容的电压和电容的相位差相差为 $\pi/2$, 所以\textbf{电容的阻抗是纯虚数}。 其虚部也叫做\textbf{容抗}\upref{CapRea}。
\addTODO{正还是负? 推导?}
\end{example}
