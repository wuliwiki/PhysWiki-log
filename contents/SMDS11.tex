% 上海海事大学 2011 年数据结构
% 上海海事大学 2011 年数据结构

\subsection{一。判断题(本题20分,每小题2分)}
1.为了很方便地插入和删除数据,可以使用双向链表存放数据。

2.两个栈共享一片连续内存空间时,为了提高内存利用率,减少溢出机会,应把两个栈的栈底分别设在这片内存空间的两端。

3.数组是同类型值的集合。

4.在查找树(二叉排序树)中插入一个新结点,总是插入到叶子结点的下面。

5.用邻接矩阵存储一个图时,在不考虑压缩存储的情况下,所占用的存储空间大小与图中顶点的个数有关,而与图的边数无关。

6.顺序存储方式只能用于存储线性结构,不能用于存储二叉树。

7.在执行某个排序算法过程中,出现了排序码朝着最终排序序列位置相反方向移动,则该算法是不稳定的。

8.数据的逻辑结构被分为集合结构、线性结构、树型结构、图结构四种,

9.将一棵树转换成二叉树后,根结点没有左子树。

10.哈夫曼树是带权路径长度最短的树,路径上权值较大的结点离根较近。


\subsection{二。填空题(本题30分,每空2分)}
1.分析下列程序段,其时间复杂度分别为:( (1) ),( (2) ).
\begin{lstlisting}[language=cpp]
i=1;
while(i<=n)
    i=i*3;

void test(int m) {
    int i=0, s=0;
    while (s<n) {
        i++;
        s=s+i;
    }
}
\end{lstlisting}

2.堆栈的插入和删除操作都是在栈顶位置进行,而队列的( (3) )操作在队尾进行,( (4) )操作在队头进行。

3.对具有n个结点的二叉树采用二叉链表存储结构,则该链表中有( (5) )个指针域,其中有( (6) )个指针域用于链接孩子结点,( (7) )个 指针域空闲存放着NULL.

4.对线性表采用折半查找方法,该线性表必须采用( (8) )存储结构, 并且数据元素按值( (9) ).

5.除了顺序存储结构与链式存储结构之外,数据的存储结构通常还有( (10) )结构和( (11) )结构。

6.已知具有4行6列的矩阵A采用行序为主序方式存储,每个元素占用4个存储单元,并且a[3][4]的存储地址为1234,元素a[1][1]的存储地址是( (12) ).

7.对于长度为n的线性表,采用顺序存储结构存储,插入或删除一个元素的时间复杂度为( (13) ).

8.若对线性表进行的操作主要不是插入和删除,则该线性表宜采用( (14) )存储结构,若频繁地对线性表进行插入和删除操作,则该线性表宜采用( (09) )存储结构。

\subsection{三、选择题(本题20分,每空2分)}
1.权值为{1.2.6.8}的四个结点构成的哈夫曼树的带权路径长度是(  )。 \\
A) 18 $\qquad$ B) 28 $\qquad$ C) 19 $\qquad$ D)29

2.在一个有向图中,所有顶点的入度之和等于所有顶点的出度之和的( )倍。 \\
A)1/2 $\qquad$ B) 1 $\qquad$ C)2 $\qquad$ D)4

3.无向图G=(V,E),其中: V={a,b,c.d.e.f). E={ab)(a,)(a.c)(b.c).(c.f), (f.d).(e.d)}, 对该图进行深度优先遍历,得到的顶点序列正确的是( )。 \\
A) a,b,e,c.d,f $\qquad$ B) a,c,f,e,b.d $\qquad$ C) a,e,b,c,fd $\qquad$ D) a,e.d.f.c,b

4.有8个结点的无向连通图最少有( ) 条边。 \\
A)5 $\qquad$ B) 6 $\qquad$ C) 7 $\qquad$ D) 8

5.在一棵度为了的树中,度为3的节点个数为2,度为2的节点个数为1,则度为0的节点个数为(  )。 \\
A)4 $\qquad$ B)5 $\qquad$ C)6 $\qquad$ D)7

6.对广义表L(a,b),(c,d.),(e,f)执行操作tail(tail(L))的结果是(  )。 \\
A)(e,f) $\qquad$ B) ((e,f)) $\qquad$ C)(f) $\qquad$ D)()

7.引起循环队列队头位置发生变化的操作是( ). \\
A)出队 $\qquad$ B)入队 $\qquad$ C)取队头元素 $\qquad$ D)取队尾元素

8.一个栈的入栈序列是a.b.c.d.e.则栈的不可能的输出序列是( )。 \\
A) edcba $\qquad$ B) decba $\qquad$ C) dceab $\qquad$ D) abcde

9.下列排序算法中,不稳定的排序是(  )。 \\
A)直接插入排序 $\qquad$ B)冒泡排序 $\qquad$ C)堆排序 $\qquad$ D)选择排序

10.顺序栈S中top为栈顶指针,指向栈顶元素所在的位置,elem为存放栈的数组,则元素e进栈操作的主要语句为(  ) \\
A)s.elem [ top] =e; s.top-s.top+1; $\qquad$ B)s.elem [ top+1] =e; s.top=s.top+1; \\
C) s.top-s.top+1; s.elem [ top+1 ] =e; $\qquad$ D) s.top=s.top+I; s.elem [ top ]=e;

\subsection{四。 (本题20分,每小题10分)}
1.某二叉树的结点数据采用顺序存储结构如下: \\
\begin{figure}[ht]
\centering
\includegraphics[width=14.25cm]{./figures/f477d3dcb0ad0c58.png}
\caption{第四1题图} \label{fig_SMDS11_1}
\end{figure}
试解答下列问题: \\
1)画出该二叉树; \\
2)画出把此二叉树还原成森林的图;

2.已知一棵二叉树的中序序列和后序序列分别为, \\
中序序列: CDBAFGEKHL \\
后序序列: DCBGFKLHEA \\
请根据画出该树的结构并写出其先序序列。

\subsection{五,(本题12分)}
设散列表为HT[0..12],即表的长度为13.散列函数为:H(key)=key\%13.采用线性探测再散列法解决冲突,若插入的关键码序列为(25, 9, 36, 43, 15, 28, 51, 67, 94). \\
1.试画出插入这9个关键码后的做列表。 \\
2.计算在等概率情况下查找成功的平均查找长度ASL. \\

\subsection{六,(本题12分,每小题6分)}
已知待排序记录的关键字序列为{435, 183, 506, 289,318, 705, 76,632, 826, 245),需要按关键字值递增的次序进行排序,请分别写出用下列两种排序方法进行第一趟扫描的过程和结果。 \\
1.以第一个元素为基准的快速排序 \\
2.冒泡排序

\subsection{七,(本题12分)}
右图是一个无向图,试用Prim算法从顶点2开始求该图的最小生成树,并给出依次产生的边,边用<i, j>的形式表示。
\begin{figure}[ht]
\centering
\includegraphics[width=5cm]{./figures/6371689c577998df.png}
\caption{第七题图} \label{fig_SMDS11_2}
\end{figure}

\subsection{八,编程题(本题24分,每小题12分)}
1.已知A, B为两个递增有序的线性表,试编写函数实现对A表的如下操作:删去其中那些在B表中出现的元素。 \\
2.已知T为一棵二叉排序树,设计算法按递减次序打印各节点的值。