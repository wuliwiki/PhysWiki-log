% 北京大学 2000 年 考研 普通物理
% license Usr
% type Note

\textbf{声明}:“该内容来源于网络公开资料,不保证真实性,如有侵权请联系管理员”

\subsection{(18分)}
1、(8分)理想气体的比热商(定压热容量与定体热容量之比)记为$Y$,试导出准静态绝热过程的 P-V方程。

2、(10 分)以理想气体为工作介质,将高温热源温度记为$T1$,低温热源温度记为$T2$.
试导出准静态卡诺循环的效率$\eta$。
\subsection{(16分)}
设太阳固定不动,某行星$P$围绕太阳在一椭圆轨道运动,如图所示,其中位置1为近太阳点,位置2为远太阳点。将太阳的质量记为$M$,圆半长轴、半短轴分别记为$A,B$.
\begin{figure}[ht]
\centering
\includegraphics[width=8cm]{./figures/8c28add43eee69c2.png}
\caption{} \label{fig_PKU200_1}
\end{figure}
利用能量守恒关系和以太阳为参考点的角动量守恒关系,导出$P$在位置1、2两处的运动速度大小 $V1,V2$;

已知椭圆面积为 $\pi AB$,导出$P$的轨道运动周期$T$。
\subsection{(16 分)}
半径为$R$,质量为$m$ 的匀质乒乓球,可处理为厚度可略的球壳。开始时以角速度$\omega0$围绕它的一条水平直径轴旋转,球心无水平方向速度,今将其轻放在水平地面上,乒乓球与地面之间的滑动摩擦处处相同。

试求乒乓球达到稳定运动状态时,它的转动角速度$\omega$;

计算从开始到最后达到稳定运动状态的全过程中,乒乓球动能的损失量$E$'

(已知半径为$R$,质量为$m$的匀质球壳相对其直径转轴的转动惯量为)
\subsection{(16 分)}
边长为$a$的正六边形分别有固定的点电荷,它们的电量或为$Q$,或为$-Q$,分布如图所示
\begin{figure}[ht]
\centering
\includegraphics[width=8cm]{./figures/788dad0cd57ae496.png}
\caption{} \label{fig_PKU200_2}
\end{figure}
\begin{enumerate}
\item 试求因点电荷间相互的静电作用而使系统具有的电势能$W$:
\item 若用外力将相邻的一对正、负电荷一起(即始终保持其间距不变)缓慢地移到无穷远处,其余固定的点电荷位置保持不变,试求外力作功量$A$。
\end{enumerate}
\subsection{16 分)}
半径为$r$的长直密绕空心螺线管,单位长度的绕线匝数为$n$,所加交变电流为$I = I_0 \sin \omega t$今在管的垂直平面上放置一个半径为$2r$,电阻为$R$的导线环,其圆心恰好在螺线管的轴线上。
\begin{figure}[ht]
\centering
\includegraphics[width=8cm]{./figures/551d7c8bf48a0497.png}
\caption{} \label{fig_PKU200_3}
\end{figure}
\begin{enumerate}
\item 计算导线环上涡旋电场$E$的值,并在图中画出其正方向;
\item 计算导线环上感应电流$I_1$;
\item 计算导线环与螺线管之间的互感系数$M$。
\end{enumerate}
\subsection{(18分)}
用钠黄光($\lambda =5893A$)观察迈克耳孙干涉仪的等倾圆条纹,开始时视场中共看到10个亮环,中心为亮斑,然后移动干涉仪一臂的平面镜,先后看到共有10个亮环缩进中央,而视场中除中心为亮斑外,还剩下5个亮环。试求:
\begin{enumerate}
\item 平面镜移动的距离:
\item 开始时中心亮斑的于涉级次;
\item 移动平面镜后最外一个亮环的干涉级次
\end{enumerate}
