% 轨道角动量升降算符归一化
% keys 量子力学|轨道角动量|升降算符|归一化
% license Xiao
% type Tutor

\pentry{轨道角动量\upref{QOrbAM}}{nod_27a7}

首先要提醒,一般来说,算符满足的一个条件是 $\braket*{g}{\Q Qf} = \braket*{\Q Q^*g}{f}$。 但是对于厄米算符, $\Q Q^* = \Q Q$, 所以有 $\braket*{g}{\Q Qf} = \braket*{\Q Qg}{f}$。

对于角动量升算符
\begin{equation}
L_+ L_- = (L_x + \I L_y)(L_x - \I L_y) = L_x^2 + L_y^2 - \I \comm{L_x}{L_y} = L^2 - L_z^2 + \hbar L_z~.
\end{equation} 
所以
\begin{equation}\ali{
L_+ L_- \psi_{l,m} &= \hbar^2 l(l + 1) \psi_{l,m} - \hbar^2 m^2 \psi_{l,m} + m\hbar^2 \psi_{l,m}\\
&= \hbar^2 [l(l + 1) - m(m - 1)] \psi_{l,m}~,
}\end{equation} 
所以 % 未完成: 第一个等号的依据?
\begin{equation}\label{eq_QLNorm_1}
\braket*{L_- \psi_{l,m}}{L_- \psi_{l,m}} = \braket*{\psi_{l,m}}{L_+L_-\psi_{l,m}} = \hbar^2 [l(l + 1) - m(m - 1)]~,
\end{equation} 
所以
\begin{equation}
L_- \psi_{l,m} = \hbar \sqrt{l(l + 1) - m(m - 1)} \psi_{l, m-1}~,
\end{equation}
同理可证
\begin{equation}
L_+ \psi_{l,m} = \hbar\sqrt{l(l + 1) - m(m + 1)} \psi_{l, m+1}~.
\end{equation} 
严格来说,归一化系数后面加上任意相位因子 $\E^{\I \theta}$ 后仍能满足\autoref{eq_QLNorm_1}, 但一般省略。
