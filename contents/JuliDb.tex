% Julia 调试笔记

\begin{issues}
\issueDraft
\end{issues}

\begin{itemize}
\item \href{https://github.com/JuliaDebug/Debugger.jl}{Debugger.jl} 是 julia 的常用的命令行 debugger。 如果想用图形界面, 用 Julia 的 \href{https://www.julia-vscode.org}{VScode 拓展} 比较好。
\item 安装: \verb|julia> import Pkg; Pkg.add("Debugger")|
\item 示例:
\begin{lstlisting}[language=none]
using Debugger
function foo(n)
    x = n+1
    ((BigInt[1 1; 1 0])^x)[2,1]
end
@enter foo(20)
\end{lstlisting}
\item 进入调试界面以后, 用 \verb|?| 可以看到常用操作。
\item \verb|st|: 显示当前函数, 运行位置。
\item \verb|+| 或 \verb|-|: 增减显示的代码行数。
\item \verb|o|: 用默认编辑器(如 vim)打开当前文件的当前行。
\item \verb|bt|: 显示调用顺序
\item \verb|n|: 运行到下一行
\item \verb|u 行号|: 运行到指定行
\item \verb|c|: 运行到下一个断点
\item \verb|s|: 进入下一个函数
\item \verb|so|: 退出当前函数
\item \verb|sl|: 进入当前行的最后一个函数调用例如进入 \verb|f(g(h(x)))| 中的 \verb|f|
\item \verb|bp add "file.jl":line [cond]| 添加断点。
\item \verb|bp| 显示所有断点
\item \verb|w add 表达式| 可以把表达式添加到 watch
\item \verb|w| 可以重新求 watch 中的所有表达式并显示。
\item \verb|`| 可以进入 eval 模式, 运行任何表达式, 用 \verb|Ctrl + C| 回到 debug 模式。
\item \verb|q| 退出 debugger。
\end{itemize}
