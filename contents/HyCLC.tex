% 氢原子的 streaking 计算

\begin{issues}
\issueDraft
\end{issues}

\pentry{光电离时间延迟\upref{HeAna2}}

根据 Pazourek 的延迟理论, 对氢原子, 总的 streaking 延迟为
\begin{equation}
t_s = t_{EWS} + t_{CLC}
\end{equation}
其中 $t_s$ 直接通过 streaking 谱得到。 $t_{EWS}$ 可以通过 XUV-only 的波包来测量, 或者等效地用 transition dipole\upref{HyIon2}来计算。

令 $\bvec{\mathcal E}$ 为电场常矢量, $C_{\bvec k}$ 为库仑平面波, $\ket{0}$ 为基态, 则
\begin{equation}\label{eq_HyCLC_1}
t_{EWS} = \pdv{E} \arg \mel{C_{\bvec k}}{\bvec{\mathcal E}\vdot \bvec r}{0} = \pdv{E} \arg \mel{C_{\bvec k}}{\uvec{k}\vdot \bvec r}{0}
\end{equation}
由于基态的对称性, 完全可以假设 $\uvec k = \uvec z$, 所以这个结果是和角度无关的。

现在可以进一步把\autoref{eq_HyCLC_1} 中的 $\ket{C_{\bvec k}}$ 展开为分波。 根据选择定则\upref{SelRul}, 只有 $l = 1$ 的分波有贡献。 最后得
\begin{equation}\label{eq_HyCLC_2}
t_{EWS} = \pdv{E} \arg \mel{C_{\bvec k}}{\bvec{\mathcal E}\vdot \bvec r}{0} = \pdv{\sigma_l}{E}
\end{equation}
