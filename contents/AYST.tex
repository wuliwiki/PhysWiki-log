% 阿尔伯特·爱因斯坦
% license CCBYSA3
% type Wiki

(本文根据 CC-BY-SA 协议转载自原搜狗科学百科对英文维基百科的翻译)

\begin{figure}[ht]
\centering
\includegraphics[width=8cm]{./figures/1c6df58255d4797a.png}
\caption{爱因斯坦} \label{fig_AYST_24}
\end{figure}

阿尔伯特·爱因斯坦(/ˈaɪnstaɪn/\textbf{EYEN}-styne;[1] 德语:[ˈalbɛɐ̯t ˈʔaɪnʃtaɪn]( 发音);1879年3月14日 ——1955年4月18日)是出生于德国的理论物理学家,[2]他发展了现代物理学的两大支柱之一(与量子力学一起)——相对论。[3][4]他的工作也因其对科学哲学的影响而闻名。[5][6]他最为人所知的质能等价公式 E = mc2被称为“世界上最著名的方程式”。[7]他获得了1921年诺贝尔物理学奖,“基于他对理论物理学的贡献,尤其是因为他发现了光电效应定律”,[8]这是量子理论发展的关键一步。

爱因斯坦在职业生涯初期认为牛顿力学已经不足以调和经典力学和电磁场的定律,这让他在伯尔尼瑞士专利局工作期间(1902-1909年)发展了狭义相对论。同时,他意识到相对论原理也可以扩展到引力场,于是在1916年发表了一篇关于广义相对论的论文,其中包含了他的引力理论。他继续处理统计力学和量子理论的问题,这使粒子理论和分子运动能够被解释。他还研究了光的热性质,这为光的光子理论奠定了基础。1917年,他应用广义相对论来模拟宇宙的结构。[9][10]

除了在布拉格的一年之外,爱因斯坦在1895年至1914年间都生活在瑞士。在此期间,于1896年放弃了德国国籍,于1900年在苏黎世获得了瑞士联邦理工学院(后来的瑞士联邦技术学院)的学术文凭。在无国籍五年多之后,他、于1901年获得了瑞士公民身份,并将其保留终身。1905年,被苏黎世大学授予博士学位。同年,在著名的奇迹年发表了四篇开创性的论文并引起了学术界的注意,此时他才26岁。爱因斯坦于1912年至1914年间在苏黎士教理论物理,然后前往柏林,在那里他被选为普鲁士科学院院士。

1933年,当爱因斯坦访问美国时,阿道夫·希特勒上台执政。由于爱因斯坦的犹太背景,他没有返回德国,[11]在美国定居,并于1940年成为美国公民。[12]在第二次世界大战前夕,他签署了一封致富兰克林·罗斯福总统的信,提醒他注意“新型超强炸弹”的潜在发展,并建议美国开始类似的研究,这最终导致了曼哈顿计划的诞生。爱因斯坦支持同盟国,但他谴责将核裂变作为武器的想法。他与英国哲学家伯特兰·罗素签署了《罗素—爱因斯坦宣言条约》,强调了核武器的危险性。他隶属于新泽西州的普林斯顿高等研究院,直到1955年去世。

爱因斯坦发表了300多篇科学论文和150多篇非科学著作。[9][13]他的智力成就和独创性使“爱因斯坦”这个词成为“天才”的同义词。[14]尤金·维格纳将爱因斯坦与他同时代的人相比写道,“爱因斯坦的理解甚至比扬西·冯·诺伊曼更深刻。他的思想比冯·诺伊曼的更具穿透力和独创性。这是一个非常了不起的声明。”[15]

\subsection{生活和事业}
\subsubsection{1.1 早期生活和教育}
\begin{figure}[ht]
\centering
\includegraphics[width=6cm]{./figures/28110856e7f46d2c.png}
\caption{1882年,爱因斯坦三岁时} \label{fig_AYST_1}
\end{figure}
阿尔伯特·爱因斯坦于1879年3月14日出生在德意志帝国符腾堡王国的乌尔姆。[2]他的父亲是赫尔曼·爱因斯坦,母亲是鲍林·科赫,分别是一名工程师和推销员。1880年,爱因斯坦一家人搬到了慕尼黑,他的父亲和叔叔雅各布在那里创立了这家基于直流电制造电气设备的公司。[2]

爱因斯坦一家是不遵守犹太教义的犹太人,他从5岁开始在慕尼黑的一所天主教小学就读了三年。8岁时,他被转到了路易斯波特体育馆(现称阿尔伯特·爱因斯坦体育馆),在那里接受了高级小学和中学教育,直到7年后离开德意志帝国。[16]
\begin{figure}[ht]
\centering
\includegraphics[width=6cm]{./figures/92a19247301e3ac0.png}
\caption{1893年,爱因斯坦14岁时} \label{fig_AYST_2}
\end{figure}
1894年,由于赫尔曼和雅各布缺乏资金将他们的设备从直流(DC)标准转换为更高效的交流(AC)标准,他们的公司失去了向慕尼黑提供电照明的投标。[17]损失迫使他们将慕尼黑工厂出售。为了寻找生意,爱因斯坦一家搬到了意大利,先是去了米兰,几个月后去了帕维亚。当全家搬到帕维亚时,15岁的爱因斯坦留在慕尼黑,在路易斯波特体育馆完成学业。他的父亲打算让他从事电气工程,但爱因斯坦与当局发生了冲突,并对学校的制度和教学方法表示不满。他后来写道,学习和创造性思维的精神在严格的死记硬背学习中丧失了。1894年12月底,用医生的证明说服学校让他离开后,爱因斯坦去意大利和他在帕维亚的家人团聚。[18]在意大利期间,他写了一篇题为“关于磁场中以太状态的研究”的短论文。[19][20]

爱因斯坦从小就擅长数学和物理,比同龄人早几年达到数学水平。十二岁的爱因斯坦在一个夏天自学了代数和欧几里得几何。同时爱因斯坦在12岁时也独立地发现了毕达哥拉斯定理的原始证明。[21]在给了12岁的爱因斯坦一本几何教科书后,家庭教师马克斯·塔木德说:“爱因斯坦在短时间内就完成了整本书。随后他致力于高等数学...很快,他的数学天赋就飞得如此之高,我都跟不上了。”[22]他对几何和代数的热情使这个12岁的孩子相信自然可以被理解为“数学结构”。[22]爱因斯坦12岁开始自学微积分,14岁时他说自己“掌握了积分和微分”。[23]

13岁时,爱因斯坦被介绍给康德学习纯粹理性批判,康德成为了他最喜欢的哲学家,他的导师说:“当时他还是个孩子,只有十三岁,但是康德的作品,常人无法理解,对他来说似乎很清楚。”[22]
\begin{figure}[ht]
\centering
\includegraphics[width=6cm]{./figures/b0289fde805edc8f.png}
\caption{爱因斯坦17岁时获得的入学证书,显示了他在阿戈维亚州学校的毕业成绩(Aargauische Kantonsschule,分数为1-6分,其中6分是最高分数)。他的分数:德语5分;法语3分;意大利语5分;历史6分;地理4分;代数6分;几何6分;画法几何6分;物理6分;化学5分;自然历史5分;艺术与技术制图4分} \label{fig_AYST_3}
\end{figure}
1895年,16岁的爱因斯坦参加了苏黎世瑞士联邦理工学院(后拉来的瑞士联邦科技学院)的入学考试。他在考试的一般课程上没有达到要求的标准,[24]但是在物理和数学上取得了优异的成绩。[25]根据理工学院校长的建议,他于1895年和1896年在瑞士阿劳的阿尔戈维亚州学校(体育馆)完成了中学教育。在寄宿于约斯特·温特尔教授的家庭时,他爱上了温特尔的女儿玛丽。爱因斯坦的姐姐玛嘉后来嫁给了温特尔的儿子保罗。[26]1896年1月,在他父亲的批准下,爱因斯坦放弃了他在德国符腾堡王国的国籍,以逃避服兵役。[27]1896年9月,他以优异的成绩通过了瑞士马图拉考试,包括物理和数学科目的6级最高分,分数范围为1-6。[28]17岁时,他报名参加了苏黎世理工学院为期四年的数学和物理教学文凭课程。玛丽·温特尔比他大一岁,她搬到了瑞士的奥尔森堡担任教师。

爱因斯坦未来的妻子,一位20岁的塞尔维亚女性米列娃·马利奇,也在那一年就读于理工学院。她是数学和物理教学文凭课程部分六名学生中唯一的女性。在接下来的几年里,爱因斯坦和马利奇的友谊发展成了爱情,他们一起阅读课外物理书籍,爱因斯坦对此越来越感兴趣。1900年,爱因斯坦通过了数学和物理考试,并被授予联邦理工学院教学文凭。[29]有人声称马利奇和爱因斯坦在他1905年的论文上合作过,[30][31]这篇论文被称为奇迹年 论文,但是研究过这个问题的物理历史学家没有发现任何证据表明马利奇做出了任何实质性的贡献。[32][33][34][35]

\subsubsection{1.2 婚姻和儿童}
\begin{figure}[ht]
\centering
\includegraphics[width=6cm]{./figures/c19fe20fe3b3fddf.png}
\caption{1904年,爱因斯坦25岁时} \label{fig_AYST_4}
\end{figure}
爱因斯坦和马利奇的早期通信于1987年被发现并发表,其中披露这对夫妇有一个女儿,名叫“丽莎尔”,1902年初出生在诺维萨德,马利奇和她的父母住在那里。马利奇没有带着孩子回到瑞士,孩子的真实姓名和命运均不明。1903年9月爱因斯坦来信的内容表明,这个女孩要么被送养,要么在婴儿期死于猩红热。[36][37]
\begin{figure}[ht]
\centering
\includegraphics[width=6cm]{./figures/1de872074fbaa68b.png}
\caption{1921年,爱因斯坦和他的第二任妻子埃尔莎} \label{fig_AYST_5}
\end{figure}
爱因斯坦和马利奇于1903年1月结婚。1904年5月,他们的儿子汉斯·爱因斯坦出生在瑞士伯尔尼。他们的小儿子爱德华于1910年7月出生在苏黎世。这对夫妇于1914年4月搬到柏林,但得知爱因斯坦认为他的表妹埃尔莎是最有吸引力的人后,马利奇和他们的儿子们回到了苏黎世。[38]他们分居五年,之后于1919年2月14日离婚。[39]爱德华大约20岁时被诊断患有精神分裂症。[40]他的母亲照顾他,他还被关在精神病院好几个时期,最后在马利奇死后被永久关押在精神病院。[41]

在2015年披露的信件中,爱因斯坦写信给他的初恋玛丽·温特尔,讲述了他的婚姻以及他对她的强烈感情。他在1910年写道,当时他的妻子怀上了他们的第二个孩子,他写道:“我每时每刻都在发自内心地爱着你,我是如此的不快乐,只有男人才能如此”。他谈到他对玛丽的爱是“执迷不悟的爱”和“错过的生活”。[42]

自从1912年和埃尔莎·温特哈尔交往后,爱因斯坦于1919年与她结婚[43][44][45]。[45]他们于1933年移民到美国。埃尔莎于1935年被诊断患有心脏和肾脏疾病,并于1936年12月去世。[46]

\subsubsection{1.3 朋友}
爱因斯坦的著名朋友有米给雷·贝索、保罗·埃伦费斯特、格罗斯曼·马塞尔、查诺斯·普莱西、丹尼尔·波辛、莫里斯·索洛文和斯蒂芬·怀斯。[47]

\subsubsection{1.4 专利局}
\begin{figure}[ht]
\centering
\includegraphics[width=6cm]{./figures/b72398c75f0c0e83.png}
\caption{奥林匹亚学院的创始人:康拉德·哈比奇、莫里斯·索洛文和爱因斯坦} \label{fig_AYST_6}
\end{figure}
1900年毕业后,爱因斯坦花了将近两年时间寻找一个教学职位。他于1901年2月获得瑞士公民身份,[48]但是由于医疗原因没有被征召入伍。在格罗斯曼·马塞尔父亲的帮助下,他在伯尔尼的联邦知识产权局,即专利局找到了一份工作,[49][50]作为三级助理审查员。[51][52]

爱因斯坦评估了各种设备的专利申请,包括砾石分类器和机电打字机。[52]1903年,他在瑞士专利局的职位成为永久职位,他在完全掌握机器技术之前一直被提拔。[53]

他在专利局的大部分工作都与电信号传输和时间的机电同步有关,这两个技术问题在思想实验中显露无疑,最终导致爱因斯坦得出了光的本质和空间与时间之间基本联系的激进结论。[53]

1902年,爱因斯坦和几个在伯尔尼认识的朋友成立了一个小型讨论组,自嘲地称为“奥林匹亚学院”,并会定期开会讨论科学和哲学。他们的读物包括儒勒·昂利·庞加莱、恩斯特·马赫和大卫·休谟的作品,这些著作影响了他的科学和哲学观点。[54]

\textbf{首批科学论文}

\begin{figure}[ht]
\centering
\includegraphics[width=6cm]{./figures/645995350a4ffe29.png}
\caption{1921年,爱因斯坦获得诺贝尔物理学奖后的官方画像} \label{fig_AYST_7}
\end{figure}
1900年,爱因斯坦的论文“毛细现象的结论”发表在杂志《物理学年鉴》上。[55][56]1905年4月30日,爱因斯坦完成了他的论文,[57]并让实验物理教授阿尔弗雷德·克莱纳担任形式顾问。爱因斯坦凭借他的论文”分子尺寸的新测定”获得了苏黎世大学的博士学位。[57][58]

1905年,这被称为爱因斯坦的奇迹年,他发表了四篇开创性的论文,分别是关于光电效应、布朗运动、狭义相对论以及质量和能量的等效性,这些论文使他在26岁时就引起了学术界的注意。

\subsubsection{1.5 学业生涯}
1908年,他被公认为一位杰出的科学家,并被任命为伯尔尼大学的讲师。第二年,在苏黎世大学做了一个关于电动力学和相对论原理的讲座后,阿尔弗雷德·克莱纳推荐他到新成立的理论物理系担任老师。爱因斯坦于1909年被任命为副教授。[59]

爱因斯坦于1911年4月成为布拉格的德国查尔斯-费迪南德大学的正教授,并因此获得奥匈帝国的奥地利公民身份。[60][61]在布拉格逗留期间,他写了11部科学著作,其中5部是关于辐射数学和固体量子理论的。1912年7月,他回到了他在苏黎世的母校。从1912年到1914年,他是苏黎世联邦理工学院担任理论物理教授,在那里他教授分析力学和热力学。他还研究了连续介质力学、分子热理论和万有引力问题,他与数学家兼朋友的马塞尔格罗斯曼一起工作。[62]

1913年7月3日,他在柏林被投票选为普鲁士科学院的成员。马克斯·普朗克和瓦尔特·能斯特第二周在苏黎士拜访了他,劝说他加入该学院,并为他提供了即将成立的凯撒·威廉物理研究所主任的职位。[63](学院的会员资格包括在柏林洪堡大学没有教学任务的带薪工资和教授职位。)他于7月24日正式当选为学院成员,并同意第二年搬到德意志帝国。他搬到柏林的决定也考虑到了可以住在他表妹埃尔莎附近,与她发展了一段浪漫的恋情。他于1914年4月1日加入学院,进而加入柏林大学。[64]随着那一年第一次世界大战的爆发,凯撒·威廉物理研究所的计划流产了。该研究所随后成立于1917年10月1日,由爱因斯坦担任主任。[65]1916年,爱因斯坦当选为德国物理学会主席(1916-1918)。[66]

根据爱因斯坦在1911年关于他的新广义相对论的计算,来自另一颗恒星的光应该被太阳的引力弯曲。1919年,亚瑟·爱丁顿爵士在1919年5月29日日食期间证实了这一预测。这些观察结果发表在国际媒体上,使爱因斯坦闻名于世。1919年11月7日,英国主要报纸 《泰晤士报》 印刷了标题为“科学革命” –新的宇宙理论 –牛顿思想被推翻”的横幅。[67]

1920年,他成为荷兰皇家艺术与科学学院的外籍成员。[68]1922年,他因“对理论物理的贡献,特别是对光电效应定律的发现”获得1921年诺贝尔物理学奖。[8]虽然广义相对论仍被认为是有争议的,但引用文献也没有将引用的光电工作视为说明 ,而是仅仅作为一个定律的发现,因为光子的概念被认为是古怪的,直到1924年安德拉·纳特·博斯推导出普朗克光谱才被普遍接受。爱因斯坦于1921年当选为英国皇家学会的外籍会员。[3]1925年,他还获得了皇家学会颁发的科普利奖章勋章。[3]

\subsubsection{1.6 1921-1922年:出国旅行}
\begin{figure}[ht]
\centering
\includegraphics[width=10cm]{./figures/bc63f7e1b8d3261a.png}
\caption{1922年至1932年,爱因斯坦在国际智力合作委员会(国际联盟)的一次会议上} \label{fig_AYST_8}
\end{figure}
1921年4月2日,爱因斯坦第一次访问纽约市,受到了市长约翰·弗朗西斯·海伦的正式欢迎,随后举行了为期三周的讲座和招待会。他接着在哥伦比亚大学和普林斯顿大学发表了几次演讲,并在华盛顿陪同国家科学院的代表访问了白宫。回到欧洲后,他成为英国政治家和哲学家霍尔丹子爵在伦敦的客人,在那里他会见了几位著名的科学、知识和政治人物,并在伦敦国王学院发表了演讲。[69][70]

1921年7月,他还发表了一篇题为《我对美国的第一印象》的文章,其中他试图简要描述美国人的一些特征,就像亚历西斯·德·托克维尔在1921年发表了自己对美国民主 的印象一样 (1835)。[71]对于他的一些观察,爱因斯坦显然很惊讶:“给游客留下印象的是对生活的快乐、积极的态度...美国人友好、自信、乐观,不嫉妒。”[72]

1922年,作为为期六个月的旅行和演讲之旅的一部分,他去了亚洲,后来又去了巴勒斯坦,访问了新加坡、锡兰和日本,在那里他给成千上万的日本人做了一系列讲座。第一次公开演讲后,他在皇宫会见了皇帝和皇后,成千上万的人前来观看。在给他儿子的一封信中,他描述了自己对日本人的印象,认为日本人谦虚、聪明、体贴,对艺术有真正的感受。[73]在1922-1923年访问亚洲期间,他在自己的旅行日记中表达了对中国人、日本人和印度人的一些看法,这些看法在2018年被重新发现时被描述为仇外心理和种族主义的判决。[74]

由于爱因斯坦的远东之行,他无法亲自在1922年12月的斯德哥尔摩颁奖典礼上接受诺贝尔物理学奖。一位德国外交官代替他发表了宴会演讲,他称赞爱因斯坦不仅是一位科学家,还是一位国际和平缔造者和活动家。[75]

在回程中,他访问了巴勒斯坦12天,这将成为他对该地区的唯一访问。他受到的欢迎就如同他是一位国家元首,而不是物理学家,其中包括抵达英国高级专员赫伯特·塞缪尔爵士家时的礼炮。在一次招待会上,大楼挤满了想看和想听他讲话的人。在爱因斯坦对观众的讲话中,他表达了犹太人开始被认为是世界上一股力量的喜悦。[76]

爱因斯坦在1923年访问了西班牙两个星期,在那里他与圣地亚哥·拉蒙-卡哈尔进行了短暂的会面,并获得了阿方索十三世国王授予他的文凭,任命他为西班牙科学院的成员。[77]

从1922年到1932年,爱因斯坦是日内瓦国际联盟智力合作国际委员会的成员(在1923-1924年间中断了几个月),[78]这是为促进科学家、研究人员、教师、艺术家和知识分子之间的国际交流而成立的机构。[79]秘书长埃里克·德拉蒙德最初被任命为瑞士代表,但天主教活动人士奥斯卡·哈利基和朱塞佩·莫塔说服他成为德国代表,从而让贡扎格·德·雷诺德获得瑞士席位,并借此宣扬传统天主教价值观。[80]爱因斯坦的前物理学教授亨德里克·洛伦兹和法国化学家玛丽·居里也是该委员会的成员。

\subsubsection{1.7 1930-1931年:去美国旅行}
1930年12月,爱因斯坦第二次访问美国,最初打算作为加州理工学院的研究员进行为期两个月的工作访问。在他第一次美国之行受到全国关注后,他和他的安排者们旨在保护他的隐私。尽管到处都是电报和邀请让他接受奖励或公开演讲,他还是拒绝了。[81]

到达纽约市后,爱因斯坦被带到了不同的地方和活动,包括唐人街,与《纽约时报》的编辑共进午餐以及在大都会歌剧院的表演卡门 ,他在到场时受到观众的欢呼。在接下来的几天里,市长吉米·沃克给了他这座城市的钥匙,并会见了哥伦比亚大学的校长,校长将爱因斯坦描述为“思想的统治者”。[82]纽约河畔教堂的牧师哈里·爱默生·福斯迪克带爱因斯坦参观了教堂,并在入口处展示了一尊根据爱因斯坦制作的全尺寸雕像。[82]在纽约逗留期间,他还和聚集在麦迪逊广场花园的15,000人一起参加了光明节的庆祝活动。[82]
\begin{figure}[ht]
\centering
\includegraphics[width=6cm]{./figures/133891c0fd9c3cb4.png}
\caption{1931年1月,爱因斯坦(左)和查理·卓别林在好莱坞城市灯光首映式上} \label{fig_AYST_9}
\end{figure}
爱因斯坦接着去了加利福尼亚,在那里他遇到了加州理工学院院长和诺贝尔奖获得者罗伯特·密立根。他和密立根的友谊比较“尴尬”,因为密立根“有爱国军国主义的倾向”,爱因斯坦是一个明显的和平主义者。[83]在对加州理工学院学生的演讲中,爱因斯坦指出科学往往弊大于利。[84]

这种对战争的厌恶也导致爱因斯坦与作家厄普顿·辛克莱和电影明星查理·卓别林交朋友,两人都以和平主义著称。环球影城的负责人卡尔·拉姆勒带爱因斯坦参观了他的工作室,并把他介绍给卓别林。他们很快就建立了融洽的关系,卓别林邀请爱因斯坦和他的妻子埃尔莎到他家吃饭。卓别林说爱因斯坦外表的平静和温柔似乎隐藏着一种“高度情绪化的气质”,从这种气质中他获得了“非凡的智力”。[85]

卓别林的电影,《城市之光》几天后将在好莱坞首映,卓别林邀请爱因斯坦和埃尔莎作为他的特邀嘉宾加入他的行列。爱因斯坦的传记作家沃尔特·伊萨克森将此描述为“名人新时代最难忘的场景之一”。[84]卓别林在后来的柏林之行中拜访了爱因斯坦的家,并回忆起他的“朴素的小公寓”和他开始写理论的钢琴。卓别林推测它“可能被纳粹分子用作引火柴”。[85]

\subsubsection{1.8 1933年:移民到美国}
\begin{figure}[ht]
\centering
\includegraphics[width=6cm]{./figures/195e3192e0cf58b6.png}
\caption{爱因斯坦“和平主义”翅膀脱落后的漫画(查尔斯R.麦考利,C.1933} \label{fig_AYST_10}
\end{figure}
1933年2月,当爱因斯坦访问美国时,他知道在德国新总理阿道夫·希特勒的领导下,随着纳粹政权的崛起,他无法回归德国。[86][87]

1933年初在美国大学期间,他在帕萨迪纳的加州理工学院担任了第三个为期两个月的客座教授。三月份,他和妻子埃尔莎乘船回到比利时,在旅途中,他们得知他们的小屋被纳粹袭击,他的私人帆船也被没收。3月28日抵达安特卫普后,他立即前往德国领事馆并交出护照,正式放弃德国国籍。[88]纳粹党后来卖掉了他的船,把他的小屋改造成了希特勒青年团营地。[89]

\textbf{难民身份}

\begin{figure}[ht]
\centering
\includegraphics[width=10cm]{./figures/806c6e148bfb8e08.png}
\caption{阿尔伯特·爱因斯坦的登陆卡(1933年5月26日),当时他从比利时奥斯坦德登陆多佛访问牛津大学} \label{fig_AYST_11}
\end{figure}
1933年4月,爱因斯坦发现德国新政府已经通过法律,禁止犹太人担任任何官方职位,包括在大学教书。[88]历史学家杰拉尔德·霍尔顿(Gerald Holton)描述了在“同事们几乎没有提出任何抗议”的情况下,成千上万的犹太科学家突然被迫放弃他们的大学职位,他们的名字被他们工作的机构名单中删除。[72]

一个月后,爱因斯坦的作品成为德国学生会在纳粹书籍燃烧中的目标之一,纳粹宣传部长约瑟夫·戈培尔宣称,“犹太人的知识分子已经死了。”[88] 一家德国杂志将他列入德国政权的敌人名单,上面写着“尚未绞死”,提供5000美元悬赏他的人头。[88][90]在随后给已经从德国移民到英国的物理学家朋友梅克斯·玻恩的信中,爱因斯坦写道:“... 我必须承认,他们的残暴和懦弱程度令人吃惊。"[88]移居美国后,他将这本书的烧毁描述为“回避大众启蒙运动”的人“自发的情感爆发”,并且“比世界上任何其他东西都更害怕知识独立人士的影响”。[91]

爱因斯坦现在仍然没有一个永久的家,不知道他将在哪里生活和工作,同样也担心仍在德国的无数其他科学家的命运。他在比利时德汉租了一栋房子,在那里住了几个月。1933年7月下旬,他应英国海军军官奥利弗·洛克·兰普森的个人邀请,前往英国约6周,此前几年他与爱因斯坦成为了朋友。为了保护爱因斯坦,洛克-兰普森让两名助手在他伦敦郊外僻静的小屋里照看他,在1933年7月24日出版的每日先驱报上刊登了一张他们拿着猎枪,守卫着爱因斯坦的照片。[92][93]

洛克-兰普森带着爱因斯坦在家里会见了温斯顿·丘吉尔,后来又去见了奥斯丁·张伯伦和前首相劳埃德·乔治。[94]爱因斯坦请求他们帮助犹太科学家离开德国。英国历史学家马丁·吉尔伯特指出,丘吉尔立即做出回应,并派他的朋友物理学家弗雷德里克·林德曼前往德国寻找犹太科学家,并将他们安置在英国的大学里。[95]丘吉尔后来观察到,由于德国驱逐了犹太人,他们降低了“技术标准”,让盟军的技术领先于他们的技术。[95]

爱因斯坦后来联系了包括土耳其总理伊斯梅尔(i̇smeti̇nönü)在内其他国家的领导人,他在1933年9月写信给他,要求安置失业的德国犹太科学家。由于爱因斯坦的信,被邀请到土耳其的犹太人最终总计超过“1000名获救者”。[96]

洛克-兰普森还向议会提交了一项法案,将英国公民身份授予爱因斯坦,在此期间爱因斯坦多次公开露面,描述了欧洲正在酝酿的危机。[97]在他的一次演讲中,他谴责了德国对待犹太人的方式,与此同时,他提出了一项在巴勒斯坦促进犹太公民身份的法案,因为他们在其他地方被剥夺了公民身份。[98]在他的演讲中,他将爱因斯坦描述为“世界公民”,应该在英国给爱因斯坦提供一个临时住所。[99][100]然而,这两项法案都失败了,爱因斯坦随后接受了美国新泽西州的普林斯顿高等研究院的邀请,成为一名常驻学者。[97]

\textbf{普林斯顿高等研究院常驻学者}

\begin{figure}[ht]
\centering
\includegraphics[width=6cm]{./figures/2157834f3983b52e.png}
\caption{1935年,在普林斯顿拍摄的爱因斯坦肖像} \label{fig_AYST_12}
\end{figure}
1933年10月,爱因斯坦回到美国,在普林斯顿高等研究院工作,[97][101]以成为逃离纳粹德国的科学家的避难所而闻名。[102]当时,大多数美国的大学,包括哈佛大学、普林斯顿大学和耶鲁大学,犹太教职员工或学生很少或根本没有,这是因为他们的犹太配额一直持续到20世纪40年代末。[102]

爱因斯坦仍未决定他的未来。他收到了几所欧洲大学的录取通知书,其中包括1931年5月至1933年6月间短暂停留三年的牛津大学基督堂学院,并获得了一个五年的学生奖学金,[103][104]但在1935年,他决定永久留在美国并申请公民身份。[97][105]

爱因斯坦与普林斯顿高等研究院的联系将持续到他1955年去世。[106] 他是新学院四个首批入选者之一(其他两个是约翰·冯·诺依曼和库尔特·哥德尔),在那里他很快与哥德尔建立了密切的友谊。两个人会一起散步,顺便讨论他们的工作。他的助手布鲁里亚·考夫曼后来成为一名物理学家。在此期间,爱因斯坦试图发展出一个统一场论,并驳斥公认的量子物理解释,但都失败了。

\textbf{第二次世界大战和曼哈顿计划}

1939年,一群匈牙利科学家,包括移民物理学家莱昂·萨尔德,试图提醒华盛顿注意正在进行的纳粹原子弹研究。该团体的警告没有得到重视。爱因斯坦和西格尔德以及其他难民,如爱德华·泰勒和尤金·维格纳,“认为提醒美国人注意德国科学家可能赢得制造原子弹的竞赛是他们的责任,并警告说希特勒将非常愿意使用这种武器。”[107][108]为了确保美国意识到这一危险,1939年7月,也就是第二次世界大战在欧洲爆发的前几个月,萨尔德和维格纳拜访了爱因斯坦,解释原子弹的可能性,和平主义者爱因斯坦说他从未考虑过原子弹的可能性。[109]他被要求通过与萨尔德一起写信给罗斯福总统以此表示支持,建议美国关注并参与自己的核武器研究。

这封信被认为是“可以说是美国在第二次世界大战前夕对核武器进行认真调查的关键刺激因素”。[110]除了这封信,爱因斯坦还利用了他与比利时王室[111]和比利时太后的联系,与一位私人特使一起进入白宫椭圆形办公室。有人说,由于爱因斯坦的来信和他与罗斯福的会晤,美国利用其“巨大的物质、财政和科学资源”发起了曼哈顿计划,开始了研制原子弹的“竞赛”。

对爱因斯坦来说,“战争是一种疾病 ...[并且]他呼吁抵抗战争。“在给罗斯福的信上签字,一些人认为他违背了他的和平主义原则。[112]1954年,也就是爱因斯坦去世前一年,他对他的老朋友莱纳斯·鲍林说,“我一生中犯了一个大错误——我在给罗斯福总统的信中签名,建议制造原子弹;但是却是正当的——因为德国人会给他们带来危险 ..."[113]

\textbf{美国公民身份}

\begin{figure}[ht]
\centering
\includegraphics[width=10cm]{./figures/acd5524a0670cd4c.png}
\caption{爱因斯坦接受菲利普福尔曼法官颁发的美国公民证书} \label{fig_AYST_13}
\end{figure}
爱因斯坦于1940年成为美国公民。在高等研究院(位于新泽西州普林斯顿)开始职业生涯后不久,他表达了对美国文化中与欧洲相比的精英制度的赞赏。他承认“个人有权畅所欲言,随心所欲地思考”,没有社会障碍,因此,他说,鼓励个人变得更有创造力,这是他从自己早期教育中珍视的一个特质。[114]

爱因斯坦在普林斯顿加入了有色人种全国进步协会(NAACP),在那里他为非裔美国人的公民权利而奋斗。他认为种族主义是美国“最严重的疾病”[90]认为它是“一代一代传下来的”。[115]作为参与的一部分,他与民权活动家W·E·B·杜波依斯通信,并准备在1951年审判期间代表他作证。[116]当爱因斯坦提出为杜波依斯做人格证人时,法官决定放弃这个案子。[117]

1946年,爱因斯坦访问了宾夕法尼亚州的林肯大学,这是一所历史上著名的黑人大学,在那里他被授予荣誉学位。(林肯大学是美国第一所授予非裔美国人大学学位的大学;校友包括兰斯顿·休斯和瑟古德·马歇尔。)爱因斯坦发表了一篇关于美国种族主义的演讲,他补充道,“我不想对此保持沉默。”[118]普林斯顿的一位居民回忆说,爱因斯坦曾经为一名黑人学生支付过大学学费。[117]

\subsubsection{1.9 个人生活}
\textbf{协助犹太复国主义事业}
\begin{figure}[ht]
\centering
\includegraphics[width=6cm]{./figures/0c60de58ced19560.png}
\caption{1947年,爱因斯坦} \label{fig_AYST_14}
\end{figure}
爱因斯坦是帮助建立1925年开放的耶路撒冷希伯来大学的傀儡领袖,也是它的首届董事会成员之一。此前,在1921年,世界犹太复国主义组织主席、生物化学家哈伊姆·魏茨曼邀请他为计划中的大学筹集资金。[119]他还就其初步方案提出了各种建议。

其中,他建议首先建立一个农业研究所,以解决未开发的土地问题。他建议,接下来应该由一个化学研究所和一个微生物研究所来对抗各种正在发生的流行病,如疟疾,他称之为破坏该国三分之一发展的“邪恶”。[120]建立一个包括希伯来语和阿拉伯语语言课程的东方学研究所,以便对该国及其历史遗迹进行科学探索,这一点也很重要。[120]

哈伊姆·魏茨曼后来成为以色列第一任总统。1952年11月,在埃兹瑞尔·卡吕巴赫的打压下,哈伊姆·魏茨曼去世了,大卫·本·古理安总理向爱因斯坦提供了以色列总统这个职位,这个职位主要是仪式上担任的。[121][122]这项提议是由以色列驻阿巴·埃班大使提出的,他解释说,这项提议“体现了犹太人民对其任何一个成员最深切的尊重”。[123]爱因斯坦拒绝了,并在回信中写道,他“深受感动”,并且“立刻感到悲伤和羞愧”,因为他不能接受。[123]

\textbf{热爱音乐}
\begin{figure}[ht]
\centering
\includegraphics[width=8cm]{./figures/c1803027c7c8161a.png}
\caption{1930年,爱因斯坦(右)与作家、音乐家和诺贝尔奖得主泰戈尔} \label{fig_AYST_15}
\end{figure}
爱因斯坦很小就开始欣赏音乐。在他最近的日记中,他写道:“如果我不是物理学家,我可能会成为一名音乐家。我经常在音乐中思考。我在音乐中度过我的白日梦。我从音乐的角度看待我的生活...我生活中最大的乐趣来自音乐。”[124][125]

他的母亲钢琴弹得相当好,希望她的儿子学小提琴,不仅是为了灌输他对音乐的热爱,也是为了帮助他融入德国文化。据指挥家莱昂·博特斯坦说,爱因斯坦5岁时就开始演奏了。然而,他在那个年龄并不喜欢它。[126]

当他13岁的时候,他发现了莫扎特的小提琴奏鸣曲,于是他开始迷恋莫扎特的作品,更愿意学习音乐。爱因斯坦自学玩耍,但“从来没有系统地练习过”。他说,“喜爱是比责任感更好的老师。”[126]17岁时,阿劳的一名学校考官在他演奏贝多芬的小提琴奏鸣曲时听到了他的声音。考官后来表示,他的演奏是“非凡的,揭示了‘伟大的洞察力’。”伯特斯坦写道,让考官震惊的是爱因斯坦“对音乐表现出了深深的热爱,这种品质过去和现在都很缺乏。音乐对这个学生来说有着不同寻常的意义。”[126]

从那时起,音乐在爱因斯坦的生活中起着举足轻重和永久的作用。虽然爱因斯坦本人从来没有想过要成为职业音乐家,但和他一起演奏室内乐的人中有几个都是专业人士,并且他为私人观众和朋友表演。在伯尔尼、苏黎世和柏林生活期间,室内乐也成为了他社交生活的一部分,在那里他和马克斯·普朗克和他的儿子以及其他人一起演奏。他有时被错误地认为是1937年版莫扎特作品《克歇尔目录》的编辑;这个版本其实是由阿尔弗雷德·爱因斯坦编写的,他可能是爱因斯坦的远亲。[127][128]

1931年,当他在加州理工学院从事研究时,他参观了洛杉矶的佐尔纳家庭音乐学院,在那里他与佐尔纳四重奏成员一起演奏了贝多芬和莫扎特的作品。[129][130]在他生命的最后,当年轻的朱利亚德四重奏在普林斯顿拜访他时,他和他们一起拉小提琴,朱利亚四重奏“对爱因斯坦的协调和语调水平印象深刻”。[126]

\textbf{政治和宗教观点}
\begin{figure}[ht]
\centering
\includegraphics[width=8cm]{./figures/d67b4e94e454e801.png}
\caption{1921年,阿尔伯特·爱因斯坦和他的妻子埃尔莎·爱因斯坦以及犹太复国主义领袖,包括以色列未来的总统柴姆·魏兹曼、妻子维拉·魏兹曼、门纳姆·乌西什金和本·锡安·摩森抵达纽约} \label{fig_AYST_16}
\end{figure}
爱因斯坦的政治观点是支持社会主义,批评资本主义,他在他的文章如《为什么是社会主义?》中详细阐述了这一点。[131][132]爱因斯坦提出并被要求对通常与理论物理或数学无关的问题给出判断和意见。[97]他强烈主张建立一个民主的全球政府,以在世界联邦的框架内检查民族国家的权力。[133]联邦调查局在1932年建立了一个关于爱因斯坦的秘密档案,在他去世时,他的联邦调查局档案长达1427页。[134]

圣雄甘地给爱因斯坦留下了深刻的印象。他与甘地交换了书面信件,并在一封关于他的信中称他为“后代的榜样”。[135]

爱因斯坦在许多原创作品和采访中谈到了他的精神面貌。[136]爱因斯坦表示他同情巴鲁赫·斯宾诺莎哲学中非个人的泛神论上帝。[137]他不相信一个关注人类命运和行为的个人上帝,他认为这种观点很天真。[138]然而,他澄清说,“我不是无神论者”,[139]他更愿称自己为不可知论者,[140] 或者是“一个虔诚的无信仰者”。[138]当被问及他是否相信来世时,爱因斯坦回答道:“不,一次生命对我来说就足够了。”[141]

爱因斯坦主要隶属于英国和美国的非宗教人文和伦理文化团体。他曾在纽约第一人文主义协会顾问委员会任职,[142]是英国出版物《新人文主义者 》理性主义协会的荣誉会员。在纽约伦理文化协会成立75周年之际,他表示,伦理文化的理念体现了他个人对宗教理想主义中最有价值和最持久的概念。他说,“没有‘道德文化’,人类就没有救赎。”[143]

\subsubsection{1.10 死亡}
1955年4月17日,爱因斯坦经历了由腹主动脉瘤破裂引起的内出血,这种内出血在1948年曾被鲁道夫·尼森通过手术治疗改善。[144]他带着准备在电视上露面纪念以色列建国七周年起草的演讲稿去了医院,但他活的时间不够长,无法完成。[145]

爱因斯坦拒绝手术,说:“我想去的时候就去。人为延长寿命是无味的。我已经尽了我的一份力量;该走了。我会做得优雅。”[146]第二天一早,他在普林斯顿医院去世,享年76岁,并且一直工作到接近尾声。[147]

在尸检过程中,普林斯顿医院的病理学家托马斯·斯托尔茨·哈维未经爱因斯坦的家人允许,将他的大脑取出保存,希望未来的神经科学能够发现是什么让爱因斯坦如此聪明。[148]爱因斯坦的遗体被火化,他的骨灰被撒在一个秘密的地方。[149][150]

1965年12月13日,在联合国教科文组织总部的核物理学家罗伯特·奥本海默发表了纪念演讲,总结了他对爱因斯坦的个人印象:“他几乎完全没有世故,完全没有俗气 ...他总是带着一种奇妙的纯洁,既天真又固执。"[151]

\subsection{科学事业}
爱因斯坦一生出版了数百本书和文章。[13][2]他发表了300多篇科学论文和150篇非科学论文。[9][13]2014年12月5日,各个大学和档案馆宣布发布爱因斯坦的论文,其中包括30,000多份独立的文件。[152][153]爱因斯坦的智力成就和独创性使“爱因斯坦”这个词成为“天才”的同义词[14]。除了他自己做的工作外,他还与其他科学家在其他项目上合作,包括玻色–爱因斯坦统计、爱因斯坦冰箱和其他项目。[154][155]

\subsubsection{2.1 1905年- 奇迹年 纸}
奇迹年 论文是关于光电效应(产生了量子理论)、布朗运动、狭义相对论和E = mc2的四篇文章,爱因斯坦于1905年发表在《物理学年鉴》的科学杂志上。这四部著作对现代物理学的发展做出了重大贡献,并改变了对空间、时间和物质的看法。这四篇论文是:
\begin{table}[ht]
\centering
\caption{辐射的类型}\label{AYST}
\begin{tabular}{|c|c}
\hline
\textbf{标题(已翻译)} & \textbf{重点领域} & \textbf{日期} & \textbf{出版} & \textbf{意义}\\
\hline
关于光的产生和转化的启发式观点 & 光电效应 & 3月18日  & 	6月9 日 & 通过提出能量只能以离散的量(量子)进行交换,解决了一个未解决的难题。[156]这个想法对量子理论的早期发展至关重要。[157]\\
\hline
分子热动力学理论要求的悬浮在静止液体中的小粒子的运动 & 布朗运动 & 5月11 日 & 7月18 日 & 	解释了原子理论的经验证据,支持统计物理的应用。\\
\hline
关于运动物体的电动力学 & 狭义相对论 & 6月30 日 & 9月26 日 & 通过引入力学的变化,使麦克斯韦的电和磁方程与力学定律相一致,这是基于经验证据的分析结果,即光速与观察者的运动无关。[158]质疑“发光醚”的概念。[159]\\
\hline
物体的惯性取决于它的能量含量吗? & 物质-能量等效 & 9月27 日 & 11月21 日 & 物质和能量的等价性,E = mc2(且言下之意,引力“弯曲”光的能力),“静止能量”的存在,核能的基础。\\
\hline
\end{tabular}
\end{table}

\subsubsection{2.2 统计力学}
\textbf{热力学波动和统计物理}
爱因斯坦在[160]1900年提交给《物理学年鉴》的第一篇论文是毛细现象。它于1901年出版,标题为"Folgerungen aus den Capillaritätserscheinungen", 翻译为“毛细血管现象的结论”。他在1902-1903年发表的两篇论文(热力学)试图从统计学的角度解释原子现象。这些论文是1905年布朗运动论文的基础,该论文表明布朗运动可以被解释为分子存在的确凿证据。他在1903年和1904年的研究主要涉及有限原子尺寸对扩散现象的影响。[160]

\textbf{批判乳光理论}

爱因斯坦回到热力学波动的问题,给出了流体在临界点的密度变化的处理方法。通常,密度波动由自由能被密度的二阶导数控制。在临界点,这个导数为零,就导致很大的波动。密度波动的影响是所有波长的光都被散射,使流体看起来像乳白色。爱因斯坦将此与瑞利散射联系起来,瑞利散射是当波动的大小远小于波长时发生的,这也解释了为什么天空是蓝色的。[161]爱因斯坦从对密度波动的处理中定量地推导出临界乳光,并证明了这种效应和瑞利散射是如何起源于物质的原子结构的。

\subsubsection{2.3 狭义相对论}
爱因斯坦的”zur Elektrodynik be wegter Krper"[162] (“论动体的电动力学”)于1905年6月30日收到,同年9月26日发表。它通过引入力学定律的变化来调和麦克斯韦方程组(电磁学定律)和牛顿力学定律之间的冲突。[163]从观测上看,这些变化的影响在高速下最为明显(物体以接近光速的速度运动)。这篇论文中提出的理论后来被称为爱因斯坦的狭义相对论。

这篇论文预测,当在一个相对运动的观察者的框架中测量时,一个运动的物体所携带的时钟会变慢,并且该物体本身会在它的运动方向上收缩。这篇论文还认为,光以太——当时物理学中的主要理论实体之一——这个概念是多余的。[164]

爱因斯坦在他关于质能公式的论文中提出的 E =mc 2 是以他的狭义相对论方程为基础。[165]爱因斯坦1905年在相对论方面的工作多年来一直备受争议,但从马克斯·普朗克开始,就被主要的物理学家所接受。[166][167]

爱因斯坦最初用运动学(对运动物体的研究)来构建狭义相对论。1908年,赫尔曼·闵可夫斯基用几何术语将狭义相对论重新解释为时空理论。爱因斯坦在1915年的广义相对论中采用了闵可夫斯基的形式主义。[168]

\subsubsection{2.4 广义相对论}
\textbf{广义相对论和等效原理}
\begin{figure}[ht]
\centering
\includegraphics[width=6cm]{./figures/1af2b2fc223631f7.png}
\caption{艾丁顿的日食照片} \label{fig_AYST_17}
\end{figure}
广义相对论是爱因斯坦在1907年至1915年间发展起来的一种引力理论。根据广义相对论,观察到的质量之间的引力是由这些质量对时间和空间的扭曲造成的。广义相对论已经发展成为现代天体物理学的一个重要工具。它为当前对黑洞的理解提供了基础,黑洞是一个引力强大到连光都无法逃脱的空间区域。

正如爱因斯坦后来所说,广义相对论发展的原因是,狭义相对论对惯性运动的偏好不令人满意,而从一开始就不偏好运动状态(即使是加速运动)的理论应该看起来更令人满意。[169]因此,在1907年,他发表了一篇关于狭义相对论下加速度的文章。在那篇题为《论相对论原理及其结论》的文章中,他认为自由落体实际上是惯性运动,对于自由落体的观察者来说,必须适用狭义相对论的规则。这个论点被称为等效原理。在同一篇文章中,爱因斯坦还预言了引力时间膨胀、引力红移和光偏转现象。[170][171]

1911年,爱因斯坦在1907年的文章里发表了另一篇文章《论引力对光传播的影响》,在这篇文章中,他估计了大质量物体对光的偏转量。因此,广义相对论的理论预测可以首次通过实验来检验。[172]

\textbf{引力波}

1916年,爱因斯坦预测了引力波在[173][174]时空曲率中的波动,它以波的形式传播,从源头向外传播,以引力辐射的形式传输能量。由于引力波的洛伦兹不变性,引力波的存在在广义相对论中是可能的。洛伦兹不变性带来了引力与它的物理相互作用的有限传播速度的概念。相比之下,引力波在牛顿万有引力理论中是不存在的,该理论假设引力的物理相互作用以无限的速度传播。

20世纪70年代,通过观测一对紧密环绕的中子星PSR B1913+16,对引力波进行了第一次间接探测。[175]对它们轨道周期衰变的解释是它们发出引力波。[175][176]由于激光干涉引力波天文台(LIGO)的研究人员发表了对引力波的首次观测,[177]于2015年9月14日在地球上发现了引力波,爱因斯坦的预测在2016年2月11日得到了证实,正好是其预测的一百年后。[175][178][179][180][181]

\textbf{空穴论证和恩图夫理论}

在发展广义相对论时,爱因斯坦对理论中的规范不变性感到困惑。他提出了一个论点,使他得出一般相对论性场论是不可能的结论。他放弃寻找完全一般协变的张量方程,而是寻找仅在一般线性变换下不变的方程。

1913年6月,恩特沃夫(“草稿”)理论是这些调查的结果。顾名思义,它是一个理论的草图,不如广义相对论优雅也更难理解,运动方程由附加的规范固定条件补充。经过两年多的深入研究,爱因斯坦意识到空穴的论点是错误的[182]。并于1915年11月放弃了这一理论。

\textbf{物理宇宙学}

1917年,爱因斯坦将广义相对论应用于整个宇宙的结构。[183]他发现一般的场方程预测了一个动态的宇宙,要么收缩要么扩张。由于当时还不知道动态宇宙的观测证据,爱因斯坦为场方程引入了一个新的术语——宇宙常数,以使理论能够预测静态宇宙。根据爱因斯坦近年来对马赫原理的理解,修正后的场方程预测了一个闭合曲率的静态宇宙。这个模型被称为爱因斯坦世界或爱因斯坦的静态宇宙。[184][185]

在1929年爱德文·哈勃发现星云的衰退之后,爱因斯坦放弃了他的静态宇宙模型,提出了两个宇宙动力学模型,1931年的弗里德曼-爱因斯坦宇宙[186][187]和1932年爱因斯坦-德西特宇宙。[188][189]在每一个模型中,爱因斯坦都抛弃了宇宙常数,声称它“在任何情况下理论上都不令人满意”。[186][187][190]

在爱因斯坦的许多传记中,有人声称爱因斯坦在晚年将宇宙常数称为他的“最大错误”。天体物理学家马里奥·利维奥最近对这一说法表示怀疑,认为这可能是夸大其词。[191]

在2013年末,由爱尔兰物理学家科马克·奥莱费雷塔领导的一个团队发现了证据,在得知哈勃对星云衰退的观测后不久,爱因斯坦认为这是一个宇宙的稳态模型。[192][193]在一份迄今为止被忽视的手稿中,显然是写于1931年初,爱因斯坦探索了一个膨胀宇宙的模型,在这个宇宙中物质的密度由于物质的不断创造而保持不变,这一过程他与宇宙常数联系在一起。[194][195] 正如他在论文中所说,“在下面的内容中,我想请大家注意方程(1)的一个解,它可以解释贺伯特(sic)的事实,其中密度随时间是恒定的" ... "如果考虑一个物理上有界的体积,物质粒子将不断地离开它。为了使密度保持不变,新的物质粒子必须从空间不断地在体积中形成。"

因此,爱因斯坦似乎比霍伊尔、邦迪和戈尔德早许多年就考虑了膨胀宇宙的稳态模型。[196][197]然而,爱因斯坦的稳态模型包含了一个根本性的缺陷,所以他很快放弃了这个想法。[194][195][198]

\textbf{能量动量伪张量}

广义相对论包含一个动态时空,因此很难看出如何确定守恒的能量和动量。诺特定理允许从具有平移不变性的拉格朗日量中确定这些量,但是一般的协方差使得平移不变性成为某种规范对称性。由于这个原因,诺特的公式在广义相对论中导出的能量和动量并不构成实张量。

爱因斯坦认为这是真的,有一个基本原因是:选择坐标可以使引力场消失。他认为非协变能量动量伪张量实际上是引力场中能量动量分布的最佳描述。这种方法得到了列夫·朗道、叶夫根尼·利夫希茨和其他人的响应,并已成为标准。

1917年,埃尔温·薛定谔和其他人对使用伪张量等非协变对象提出了严厉的批评。

\textbf{虫洞}

1935年,爱因斯坦与纳森·罗森合作制造了一个虫洞模型,通常被称为爱因斯坦-罗森桥。[199][200]他的动机是根据论文“引力场在基本粒子的构成中扮演重要角色吗?”中概述的程序,用电荷作为引力场方程的解来模拟基本粒子。这些解决方案通过剪切并粘贴史瓦西黑洞,并在两个补丁之间架起了一座桥梁。[201]

如果虫洞的一端带正电,另一端带负电。这些性质使爱因斯坦相信粒子对和反粒子对可以这样描述。

\textbf{爱因斯坦-嘉当理论}

\begin{figure}[ht]
\centering
\includegraphics[width=6cm]{./figures/6a04805b32b15202.png}
\caption{1920年,爱因斯坦在他的办公室,柏林大学} \label{fig_AYST_18}
\end{figure}
为了将旋转点粒子纳入广义相对论中,仿射联接要广义化以包含一个称为扭转的反对称部分。这一修改是爱因斯坦和卡坦在20世纪20年代做出的。

\textbf{运动方程}

广义相对论有一个基本定律—描述空间如何弯曲的爱因斯坦场方程。描述粒子如何运动的测地线方程可以从爱因斯坦场方程中推导出来。

由于广义相对论的方程是非线性的,一团由纯引力场构成的能量,像黑洞一样,会沿着由爱因斯坦场方程本身决定的轨迹运动,而不是由新的定律决定。所以爱因斯坦提出,奇异解的路径如像黑洞,将被确定为广义相对论本身的测地线。

这是由爱因斯坦、因菲尔德和霍夫曼为没有角动量的点状物体建立的,和由罗伊·克尔为旋转物体建立的。

\subsubsection{2.5 旧量子论}
\textbf{光子和能量量子}
\begin{figure}[ht]
\centering
\includegraphics[width=6cm]{./figures/7be4f8b720a26117.png}
\caption{光电效应;左边的入射光子撞击金属板(底部),并弹出电子,描绘为向右飞去} \label{fig_AYST_19}
\end{figure}
在1905年的一篇论文中,[202]爱因斯坦假设光本身由局部粒子(量子)组成的。爱因斯坦的光量子几乎被所有物理学家一致拒绝,包括马克斯·普朗克和尼尔斯·玻尔。直到1919年罗伯特·密立根详细的光电效应实验和康普顿散射的测量,这个想法才被普遍接受。

爱因斯坦得出结论,频率f的每一个波都与光子的集合有关,每个光子都有能量hf,其中h是普朗克常数。他没有多说什么,因为他不确定粒子与波有什么关系。但是他确实认为这个想法可以解释某些实验结果,特别是光电效应。[202]

\textbf{量子化原子振动}

1907年,爱因斯坦提出了一个物质模型,其中晶格结构中的每个原子都是一个独立的谐振子。在爱因斯坦模型中,每个原子都是独立振荡的——每个振荡器有一系列等间距的量子化态。爱因斯坦知道获得实际振荡的频率是困难的,但他还是提出了这个理论,因为它特别清楚地证明了量子力学可以解决经典力学中的比热问题。彼得·约瑟夫·威廉·德拜改进了这个模型。[203]

\textbf{绝热原理和作用角变量}

在整个1910年代,量子力学的范围扩大到涵盖了许多不同的系统。在欧内斯特·卢瑟福发现原子核并提出电子像行星一样运行之后,尼尔斯·玻尔证明了由普朗克引入并由爱因斯坦发展的同样的量子力学假设,可以解释电子在原子中的离散运动,以及元素周期表。

通过将这些发展与1898年威廉·维恩提出的论点联系起来,爱因斯坦为这些发展做出了贡献。维恩证明了热平衡状态绝热不变性的假设允许所有不同温度下的黑体曲线通过简单的移动过程相互导出。爱因斯坦在1911年指出,同样的绝热原理表明,在任何机械运动量化的量必须是绝热不变量。阿诺·索末菲将这个绝热不变量确定为经典力学的作用变量。

\textbf{玻色–爱因斯坦统计}

1924年,爱因斯坦从印度物理学家萨特延德拉·纳特·玻色那里得到了一个统计模型的描述,该模型基于一种假设光可以被理解为由无法分辨的粒子组成的气体的计数方法。爱因斯坦注意到玻色的统计适用于一些原子以及提出的轻粒子,并把他对玻色论文的翻译提交给了Zeitschrift für Physik。爱因斯坦还发表了自己的文章,描述了这个模型及其含义,其中包括玻色–爱因斯坦凝聚现象,即一些微粒在非常低的温度下会出现。[204]直到1995年,埃里克·康奈尔和卡尔·威曼才使用博尔德科罗拉多大学的NIST-JILA实验天体物理联合研究所实验室建造的超冷却设备,通过实验生产出第一批这样的冷凝液。[205]玻色–爱因斯坦统计现在被用来描述任何玻色子组合的行为。这个项目的爱因斯坦草图可以在莱顿大学图书馆的爱因斯坦档案中看到。[154]

\textbf{波粒二象性}

\begin{figure}[ht]
\centering
\includegraphics[width=6cm]{./figures/8c8c483fb14f602f.png}
\caption{爱因斯坦访问美国期间} \label{fig_AYST_20}
\end{figure}
尽管专利局在1906年将爱因斯坦提升为二等技术审查员,但他并没有放弃学术界。1908年,他成为了一名无薪大学教师 在伯尔尼大学。[206]在"Über die Entwicklung unserer Anschauungen über das Wesen und die Konstitution der Strahlung" ("关于辐射成分和本质观点的发展")中,关于光的量子化,在1909年早期的论文中,爱因斯坦表明,马克斯普朗克的能量量子必须具有明确定义的动量,并在某些方面作为独立的点状粒子。本文介绍了光子 概念(虽然光子 的名称后来由吉尔伯特·路易斯于1926年提出),并启发了量子力学中波粒二象性的概念。爱因斯坦认为辐射中的波粒二象性是他坚信物理学需要一个新的统一基础的具体证据。

\textbf{零点能量}

在1911年至1913年完成的一系列工作中,普朗克重新表述了他1900年的量子理论,并在他的“第二量子理论”中引入了零点能量的概念。很快,这个想法吸引了爱因斯坦和他的助手奥托·斯特恩的注意。假设旋转双原子分子的能量包含零点能量,然后他们将氢气的理论比热与实验数据进行比较。这些数字非常匹配。然而,在公布研究结果后,他们立即撤回了他们的支持,因为他们不再相信零点能量的想法是正确的。[207]

\textbf{受激发射}

1917年,在相对论研究的巅峰时期,爱因斯坦在Physikalische Zeitschrift上发表了一篇文章, 提出了受激发射的可能性,这是使微波激射和激光成为可能的物理过程。[208]本文表明,只有在光发射到具有n个光子的模式比光发射到空模式在统计上得到增强的情况下,光的吸收和发射的统计才符合普朗克分布定律。这篇论文对后来量子力学的发展产生了巨大的影响,因为它是第一篇表明原子跃迁统计有简单规律的论文。

\textbf{物质波}

爱因斯坦发现了路易·德布罗意的工作并支持他的想法,这些想法起初受到怀疑。在这个时代的另一篇重要论文中,爱因斯坦给出了德布罗意波的波动方程,爱因斯坦认为这就是力学中的哈密顿-雅可比方程式。这篇论文将激励薛定谔1926年的工作。

\subsubsection{2.6 量子力学}
\textbf{爱因斯坦对量子力学的反对}
\begin{figure}[ht]
\centering
\includegraphics[width=6cm]{./figures/9e92a3c1b8c51728.png}
\caption{1935年5月4日的报纸头条} \label{fig_AYST_21}
\end{figure}
爱因斯坦对现代量子力学不满意,因为它是在1925年以后发展起来的。与普遍的看法相反,他的怀疑不是因为相信上帝“不是在玩骰子”。[209]事实上,是爱因斯坦自己在他1917年的论文中提出了受激发射的可能性,[208]他首先提出了机会在解释量子过程中的基本作用。[210]相反,他反对量子力学对现实本质的暗示。爱因斯坦认为物理现实的存在与我们观察它的能力无关。相比之下,玻尔和他的追随者坚持认为,我们所能知道的只是测量和观察的结果,推测一个超越我们感知的终极现实是没有意义的。[211]

\textbf{玻尔对爱因斯坦}

\begin{figure}[ht]
\centering
\includegraphics[width=6cm]{./figures/1bd4239d95577865.png}
\caption{1925年,爱因斯坦和尼尔斯·玻尔} \label{fig_AYST_22}
\end{figure}
玻尔-爱因斯坦之争是爱因斯坦和尼尔斯·玻尔之间关于量子力学的一系列公开争论,尼尔斯玻尔是其创始人之一。他们的辩论因其对科学哲学的重要性而被铭记。[212][213][214]他们的辩论将影响到后人对量子力学的解释。

\textbf{爱因斯坦-波多尔斯基-罗森悖论}

1935年,爱因斯坦在“EPR论文”中回归量子力学,特别是其完备性问题。[214] 在一个思想实验中,他考虑了两个相互作用的粒子,使得它们的性质紧密相关。无论两个粒子相距多远,对一个粒子进行精确的位置测量都会同样精确的了解另一个粒子的位置;同样地,精确测量一个粒子的动量可以得到另一个粒子的动量,而不需要以任何方式干扰另一个粒子。[215]

考虑到爱因斯坦的局部实在论的概念,有两种可能性:(1)另一个粒子已经确定了这些性质,或(2)测量第一个粒子的过程瞬间影响了第二个粒子的位置和动量的真实性。爱因斯坦拒绝了第二种可能性(俗称“远距离的幽灵行动”)。[215]

爱因斯坦对局部实在论的信仰使他断言,虽然量子力学的正确性没有问题,但它一定是不完整的。但是作为一个物理原理,当1982年的Aspect实验证实了贝尔在1964年所描绘的贝尔定理时,局部实在论被证明是不正确的。这些实验和随后的实验结果表明,量子物理不能用任何版本的物理图像来表示,其中“粒子被认为是不相关的、独立的、类似经典的实体,每一个都不能在它们分离之后相互通信。”[216]

尽管爱因斯坦对局部实在论的看法是错误的,但他对其对立面的纠缠量子态的不寻常性质的清晰预测,使EPR论文成为《物理评论》发表的十大论文之一。它被认为是量子信息论发展的核心。[217]

\subsubsection{2.7 统一场论}
在对广义相对论进行研究之后,爱因斯坦进行了一系列的尝试,试图推广他的引力几何理论,将电磁学作为单个实体的另一个方面。1950年,他在一篇题为“广义引力理论”的科学美国文章中描述了他的“统一场论”。[218]尽管爱因斯坦的工作继续受到称赞,但他在研究中变得越来越孤立,他的努力最终没有成功。爱因斯坦在追求基本力的统一时,忽略了物理学中的一些主流发展,尤其是强核力和弱核力,直到他死后许多年,人们才很好地理解它们。反过来,主流物理学在很大程度上忽略了爱因斯坦的统一方法。爱因斯坦将其他物理定律与引力统一起来的梦想激发了现代对一切理论的探索,特别是弦理论,在弦理论中,几何场出现在统一的量子力学环境中

\subsubsection{2.8 其他调查}
爱因斯坦进行了其他不成功且被放弃的调查。这些调查与力、超导性和其他研究有关。

\subsubsection{2.9 与其他科学家的合作}
\begin{figure}[ht]
\centering
\includegraphics[width=10cm]{./figures/ae50ea25abacfeab.png}
\caption{1927年在布鲁塞尔举行的索尔维会议,是世界顶级物理学家的聚会。爱因斯坦在中心。} \label{fig_AYST_23}
\end{figure}
除了长期合作者利奥波德·英费尔德、纳森·罗森、彼得·博格曼和其他人之外,爱因斯坦还与许多科学家进行了一次性合作。

\textbf{爱因斯坦-德哈斯实验}

爱因斯坦和德哈斯证明了磁化是由于电子的运动引起的,现在被称为自旋。为了证明这一点,他们颠倒了悬挂在扭摆上的铁棒的磁化强度。他们证实这会导致棒的旋转,因为电子的角动量随着磁化强度的变化而变化。这个实验敏感度很高,因为与电子相关的角动量很小,但它确定了某种电子运动是磁化的原因。

\textbf{薛定谔气体模型}

爱因斯坦向埃尔温·薛定谔建议,他可以通过考虑一个盒子来重现玻色-爱因斯坦气体的统计数据。然后将一个独立的谐振子与盒子中粒子的每一个可能的量子运动联系起来。对这些振荡器量化后,每个能级都有一个整数占据数,也就是其中的粒子数。

这种表述是二次量子化的一种形式,但它早于现代量子力学。埃尔温·薛定谔应用这一理论导出了半经典理想气体的热力学性质。尽管爱因斯坦拒绝了邀请,但薛定谔还是敦促爱因斯坦加入他的合著者行列。[219]

\textbf{爱因斯坦冰箱}

1926年,爱因斯坦和他以前的学生莱昂斯扎德共同发明了爱因斯坦冰箱(并于1930年获得专利)。这种吸收式冰箱是革命性的,因为它没有运动部件,只使用热量作为输入。[220]1930年11月11日, 美国专利 1,781,541 被授予爱因斯坦和莱昂斯扎德的冰箱专利。他们的发明没有立即投入商业生产,他们最有希望的专利被瑞典公司伊莱克斯收购。[221]

\subsection{非科学遗产}
旅行时,爱因斯坦每天给他妻子埃尔莎写信,并收养了继女玛戈特和伊尔斯。这些信件包括在遗赠给希伯来大学的文件中。玛戈特·爱因斯坦允许向公众提供这些私人信件,但要求在她死后20年(她于1986年去世)才能使用[222]。爱因斯坦表达了他对水暖行业的兴趣,并被任命为水暖工和汽修工联合会的荣誉成员。[223][224]希伯来大学阿尔伯特·爱因斯坦档案馆的芭芭拉·沃尔夫告诉英国广播公司(BBC),在1912年到1955年间,爱因斯坦大约写了3500页的私人信件。[225]

作为该大学的代理人,罗杰·里奇曼机构的继任者Corbis,授权使用爱因斯坦的名字和相关图像。[226]

\subsection{在流行文化中}
在第二次世界大战之前,《纽约客》在他们的“城市谈话”专题中发表了一篇短文,说爱因斯坦在美国非常有名,以至于人们都想让他解释“那个理论”,他会在街上被拦住。他终于想出了处理不断询问的方法。他告诉他的询问者:“对不起,对不起!我总是被误认为爱因斯坦教授。”[227]

爱因斯坦是许多小说、电影、戏剧和音乐作品的主题或灵感来源。[228]他是描述疯狂科学家和心不在焉教授的最受欢迎的模型;他富有表情的脸和独特的发型被广泛复制和夸大。时代 杂志的弗雷德里克·戈尔登写道,爱因斯坦是“漫画家的梦想成真”。[229]

许多流行的引语经常被误认为是他的。[230][231]

\subsection{奖项和荣誉}
爱因斯坦获得了许多奖项和荣誉,1922年,他因“对理论物理学的贡献,特别是对光电效应定律的发现”获得了1921年诺贝尔物理学奖。1921年的提名没有一个符合阿尔弗雷德·诺贝尔设定的标准,因此1921年的奖项被推进并于1922年颁发给爱因斯坦。[8]

\subsection{出版物}
\textsl{本文引用了爱因斯坦的以下出版物。在阿尔伯特·爱因斯坦科学出版物列表可以找到更完整的清单。}
\begin{itemize}
\item Einstein, Albert (1901) [Manuscript received: 16 December 1900]. Written at Zurich, Switzerland. "Folgerungen aus den Capillaritätserscheinungen" [Conclusions Drawn from the Phenomena of Capillarity]. Annalen der Physik (Berlin) (in German). Hoboken, NJ (published 14 March 2006). \textbf{309} (3): 513–523. Bibcode:1901AnP...309..513E. doi:10.1002/andp.19013090306 – via Wiley Online Library.CS1 maint: Unrecognized language (link)
\item Einstein, Albert (1905a) [Manuscript received: 18 March 1905]. Written at Berne, Switzerland. "Über einen die Erzeugung und Verwandlung des Lichtes betreffenden heuristischen Gesichtspunkt" [On a Heuristic Viewpoint Concerning the Production and Transformation of Light] (PDF). Annalen der Physik (Berlin) (in German). Hoboken, NJ (published 10 March 2006). 322 (6): 132–148. Bibcode:1905AnP...322..132E. doi:10.1002/andp.19053220607 – via Wiley Online Library.CS1 maint: Unrecognized language (link)
\item Einstein, Albert (1905b) [Completed 30 April and submitted 20 July 1905]. Written at Berne, Switzerland, published by Wyss Buchdruckerei. Eine neue Bestimmung der Moleküldimensionen [A new determination of molecular dimensions] (PDF). Dissertationen Universität Zürich (PhD Thesis) (in German). Zurich, Switzerland: ETH Zürich (published 2008). doi:10.3929/ethz-a-000565688 – via ETH Bibliothek.CS1 maint: Unrecognized language (link)
\item Einstein, Albert (1905c) [Manuscript received: 11 May 1905]. Written at Berne, Switzerland. "Über die von der molekularkinetischen Theorie der Wärme geforderte Bewegung von in ruhenden Flüssigkeiten suspendierten Teilchen" [On the Motion – Required by the Molecular Kinetic Theory of Heat – of Small Particles Suspended in a Stationary Liquid]. Annalen der Physik (Berlin) (in German). Hoboken, NJ (published 10 March 2006). \textbf{322} (8): 549–560. Bibcode:1905AnP...322..549E. doi:10.1002/andp.19053220806. hdl:10915/2785 – via Wiley Online Library.CS1 maint: Unrecognized language (link)
\item Einstein, Albert (1905d) [Manuscript received: 30 June 1905]. Written at Berne, Switzerland. "Zur Elektrodynamik bewegter Körper" [On the Electrodynamics of Moving Bodies]. Annalen der Physik (Berlin) (Submitted manuscript) (in German). Hoboken, NJ (published 10 March 2006). \textbf{322} (10): 891–921. Bibcode:1905AnP...322..891E. doi:10.1002/andp.19053221004. hdl:10915/2786 – via Wiley Online Library.CS1 maint: Unrecognized language (link)
\item Einstein, Albert (1905e) [Manuscript received: 27 September 1905]. Written at Berne, Switzerland. "Ist die Trägheit eines Körpers von seinem Energieinhalt abhängig?" [Does the Inertia of a Body Depend Upon Its Energy Content?]. Annalen der Physik (Berlin) (in German). Hoboken, NJ (published 10 March 2006). 323 (13): 639–641. Bibcode:1905AnP...323..639E. doi:10.1002/andp.19053231314 – via Wiley Online Library.CS1 maint: Unrecognized language (link)
\item Einstein, Albert (1915) [Published 25 November 1915]. "Die Feldgleichungen der Gravitation" [The Field Equations of Gravitation] (Online page images) (in German). Berlin, Germany: Königlich Preussische Akademie der Wissenschaften: 844–847 – via ECHO, Cultural Heritage Online, Max Planck Institute for the History of Science.CS1 maint: Unrecognized language (link)
\item Einstein, Albert (1917a). "Kosmologische Betrachtungen zur allgemeinen Relativitätstheorie" [Cosmological Considerations in the General Theory of Relativity] (in German). Königlich Preussische Akademie der Wissenschaften, Berlin.CS1 maint: Unrecognized language (link)
\item Einstein, Albert (1917b). "Zur Quantentheorie der Strahlung" [On the Quantum Mechanics of Radiation]. Physikalische Zeitschrift (in German). \textbf{18}: 121–128. Bibcode:1917PhyZ...18..121E.CS1 maint: Unrecognized language (link)
\item Einstein, Albert (1923) [First published 1923, in English 1967]. Written at Gothenburg. Grundgedanken und Probleme der Relativitätstheorie [Fundamental Ideas and Problems of the Theory of Relativity] (PDF). Nobel Lectures, Physics 1901–1921 (in German and English). Stockholm: Nobelprice.org (published 3 February 2015) – via Nobel Media AB 2014.CS1 maint: Unrecognized language (link)
\item Einstein, Albert (1924) [Published 10 July 1924]. "Quantentheorie des einatomigen idealen Gases" [Quantum theory of monatomic ideal gases] (Online page images). Sitzungsberichte der Preussischen Akademie der Wissenschaften, Physikalisch-Mathematische Klasse (in German): 261–267 – via ECHO, Cultural Heritage Online, Max Planck Institute for the History of Science.CS1 maint: Unrecognized language (link)。关于这个主题的系列论文的第一篇。
\item Einstein, Albert (12 March 1926) [Cover Date 1 March 1926]. Written at Berlin. "Die Ursache der Mäanderbildung der Flußläufe und des sogenannten Baerschen Gesetzes" [On Baer's law and meanders in the courses of rivers]. Die Naturwissenschaften (in German). Heidelberg, Germany. \textbf{14 } (11): 223–224. Bibcode:1926NW.....14..223E. doi:10.1007/BF01510300. ISSN 1432-1904 – via SpringerLink.CS1 maint: Unrecognized language (link)
\item Einstein, Albert (1926b). Written at Berne, Switzerland. Fürth, R., ed. Investigations on the Theory of the Brownian Movement (PDF). Translated by Cowper, A. D. US: Dover Publications (published 1956). ISBN 978-1-60796-285-4. Retrieved 4 January 2015.
\item Einstein, Albert; Podolsky, Boris; Rosen, Nathan (15 May 1935) [Received 25 March 1935]. "Can Quantum-Mechanical Description of Physical Reality Be Considered Complete?". Physical Review (Submitted manuscript). 47 (10): 777–780. Bibcode:1935PhRv...47..777E. doi:10.1103/PhysRev.47.777 – via APS Journals.
\item Einstein, Albert (9 November 1940). On Science and Religion. Nature. \textbf{146}. pp. 605–607. Bibcode:1940Natur.146..605E. doi:10.1038/146605a0. ISBN 978-0-7073-0453-3.
\item Einstein, Albert; et al. (4 December 1948). "To the editors of The New York Times". The New York Times. Melville, New York: AIP, American Inst. of Physics. ISBN 0-7354-0359-7.
\item Einstein, Albert (May 1949). "Why Socialism? (Reprise)". Monthly Review. New York: Monthly Review Foundation (published May 2009). Archived from the original on 11 January 2006. Retrieved 16 January 2006 – via MonthlyReview.org.
\item Einstein, Albert (1950). "On the Generalized Theory of Gravitation". Scientific American. \textbf{CLXXXII} (4): 13–17. Bibcode:1950SciAm.182d..13E. doi:10.1038/scientificamerican0450-13.
\item Einstein, Albert (1954). Ideas and Opinions. New York: Random House. ISBN 978-0-517-00393-0.
\item Einstein, Albert (1969). Albert Einstein, Hedwig und Max Born: Briefwechsel 1916–1955 (in German). Munich: Nymphenburger Verlagshandlung. ISBN 978-3-88682-005-4.CS1 maint: Unrecognized language (link)
\item Einstein, Albert (1979). Autobiographical Notes. Paul Arthur Schilpp (Centennial ed.). Chicago: Open Court. ISBN 978-0-87548-352-8.。 这 追逐光束 48-51页描述了思维实验。
\item 收集的论文: Stachel, John; Martin J. Klein; A. J. Kox; Michel Janssen; R. Schulmann; Diana Komos Buchwald; et al., eds. (21 July 2008) [Published between 1987–2006]. The Collected Papers of Albert Einstein. Einstein's Writings. \textbf{1–10}. Princeton University Press.。有关迄今出版的书籍的更多信息,请访问 Einstein Papers Project 普林斯顿大学出版社 Einstein Page
\end{itemize}

\subsection{参考文献}
[1]
^Wells, John (3 April 2008). Longman Pronunciation Dictionary (3rd ed.). Pearson Longman. ISBN 978-1-4058-8118-0..

[2]
^"Albert Einstein – Biography". Nobel Foundation. Archived from the original on 6 March 2007. Retrieved 7 March 2007..

[3]
^Whittaker, E. (1 November 1955). "Albert Einstein. 1879–1955". Biographical Memoirs of Fellows of the Royal Society. 1: 37–67. doi:10.1098/rsbm.1955.0005. JSTOR 769242..

[4]
^Fujia Yang; Joseph H. Hamilton (2010). Modern Atomic and Nuclear Physics. World Scientific. ISBN 978-981-4277-16-7..

[5]
^Howard, Don A., ed. (2014) [First published 11 February 2004]. "Einstein's Philosophy of Science". Stanford Encyclopedia of Philosophy. The Metaphysics Research Lab, Center for the Study of Language and Information (CSLI), Stanford University. Retrieved 4 February 2015..

[6]
^Howard, Don A. (December 2005). "Albert Einstein as a Philosopher of Science" (PDF). Physics Today. 58 (12): 34–40. Bibcode:2005PhT....58l..34H. doi:10.1063/1.2169442. Retrieved 8 March 2015 – via University of Notre Dame, Notre Dame, IN, author's personal webpage..

[7]
^Bodanis, David (2000). E = mc2: A Biography of the World's Most Famous Equation. New York: Walker..

[8]
^"The Nobel Prize in Physics 1921". Nobel Prize. Retrieved 11 July 2016..

[9]
^"Scientific Background on the Nobel Prize in Physics 2011. The accelerating universe" (PDF). Nobel Media AB. p. 2. Archived from the original (PDF) on 29 May 2019. Retrieved 4 January 2015..

[10]
^Overbye, Dennis (24 November 2015). "A Century Ago, Einstein's Theory of Relativity Changed Everything". The New York Times. Retrieved 24 November 2015..

[11]
^Levenson, Thomas (9 June 2017). "The Scientist and the Fascist". The Atlantic..

[12]
^Paul S. Boyer; Melvyn Dubofsky (2001). The Oxford Companion to United States History. Oxford University Press. p. 218. ISBN 978-0-19-508209-8..

[13]
^Paul Arthur Schilpp, ed. (1951). Albert Einstein: Philosopher-Scientist. II. New York: Harper and Brothers Publishers (Harper Torchbook edition). pp. 730–746.. His non-scientific works include: About Zionism: Speeches and Lectures by Professor Albert Einstein (1930), "Why War?" (1933, co-authored by Sigmund Freud), The World As I See It (1934), Out of My Later Years (1950), and a book on science for the general reader, The Evolution of Physics (1938, co-authored by Leopold Infeld)..

[14]
^"Result of WordNet Search for Einstein". 3.1. The Trustees of Princeton University. Retrieved 4 January 2015..

[15]
^The Recollections of Eugene P. Wigner, By Eugene Paul Wigner, Andrew Szanton, (Springer, 2013), p. 170 [缺少ISBN].

[16]
^Stachel (2002), pp. 59–61..

[17]
^Barry R. Parker (2003). Einstein: The Passions of a Scientist, Prometheus Books, p. 31.

[18]
^Fölsing (1997), pp. 30–31..

[19]
^Stachel (2008),vol. 1 (1987), doc. 5..

[20]
^Mehra, Jagdish (2001). "Albert Einstein's first paper". The Golden Age of Physics. World Scientific. ISBN 978-981-02-4985-4..

[21]
^The Three-body Problem from Pythagoras to Hawking, Mauri Valtonen, Joanna Anosova, Konstantin Kholshevnikov, Aleksandr Mylläri, Victor Orlov, Kiyotaka Tanikawa, (Springer 2016), p. 43, Simon and Schuster, 2008.

[22]
^Einstein: His Life and Universe, By Walter Isaacson, p. 17.

[23]
^Einstein: His Life and Universe, By Walter Isaacson, p. 16.

[24]
^Stachel (2008),vol. 1 (1987), p. 11..

[25]
^Fölsing (1997), pp. 36–37..

[26]
^Highfield & Carter (1993), pp. 21, 31, 56–57..

[27]
^Fölsing (1997), p. 40..

[28]
^Stachel (2008),vol. 1 (1987), docs. 21–27..

[29]
^Stachel (2008),vol. 1 (1987), doc. 67..

[30]
^Troemel-Ploetz, D. (1990). "Mileva Einstein-Marić: The Woman Who Did Einstein's Mathematics". Women's Studies International Forum. 13 (5): 415–432. doi:10.1016/0277-5395(90)90094-e..

[31]
^Walker, Evan Harris (February 1989). "Did Einstein Espouse his Spouse's Ideas?" (PDF). Physics Today. Archived from the original (PDF) on 19 January 2012. Retrieved 19 October 2014..

[32]
^Pais (1994), pp. 1–29..

[33]
^Holton, G., Einstein, History, and Other Passions, Harvard University Press, 1996, pp. 177–193..

[34]
^Stachel (2002), pp. 49–56..

[35]
^Martinez, A. A., "Handling evidence in history: the case of Einstein's wife", School Science Review, 86 (316), March 2005, pp. 49–56. PDF Archived 11 8月 2011 at the Wayback Machine.

[36]
^J. Renn & R. Schulmann, Albert Einstein/Mileva Marić: The Love Letters, 1992, pp. 73–74, 78..

[37]
^A. Calaprice & T. Lipscombe, Albert Einstein: A Biography, 2005, pp. 22–23..

[38]
^Stachel (1966)..

[39]
^Stachel (2002), p. 50..

[40]
^Robinson, Andrew (2015). Einstein: A Hundred Years of Relativity. Princeton University Press. pp. 143–145. ISBN 978-0-691-16989-7..

[41]
^Neffe (2007), p. 203..

[42]
^Wüthrich, Urs (11 April 2015). "Die Liebesbriefe des untreuen Einstein" [The love letters of the unfaithful Einstein]. BZ Berner Zeitung (in German). Bern, Switzerland. Retrieved 11 April 2015. Ich denke in innigster Liebe an Dich in jeder freien Minute und bin so unglücklich, wie nur ein Mensch es sein kann.CS1 maint: Unrecognized language (link).

[43]
^Isaacson (2007)..

[44]
^Calaprice,Kennefick & Schulmann (2015), p. 62..

[45]
^Calaprice, Alice; Lipscombe, Trevor (2005). Albert Einstein: A Biography. Greenwood Publishing Group. p. 50. ISBN 978-0-313-33080-3..

[46]
^Highfield & Carter (1993), p. 216..

[47]
^An Einstein Encyclopedia, Alice Calaprice, Daniel Kennefick, Robert Schulmann, p. 65 et seq, Princeton University Press, 2015.

[48]
^Fölsing (1997), p. 82..

[49]
^J J O'Connor; E F Robertson, eds. (May 2010). "Grossmann biography" (scientific website?). School of Mathematics and Statistics, University of St Andrews, Scotland. Retrieved 27 March 2015..

[50]
^Isaacson (2007), p. 63..

[51]
^"Einstein at the patent office" (official website). Berne, Switzerland: Swiss Federal Institute of Intellectual Property, IGE/IPI. 6 February 2014. Retrieved 9 September 2016..

[52]
^"FAQ about Einstein and the Institute" (official website). Berne, Switzerland: Swiss Federal Institute of Intellectual Property, IGE/IPI. 27 May 2014. Retrieved 27 March 2015..

[53]
^Galison, Peter (2000). "Einstein's Clocks: The Question of Time". Critical Inquiry. 26 (2 Winter): 355–389. JSTOR 1344127..

[54]
^Isaacson (2007), p. 79-84..

[55]
^Einstein (1901)..

[56]
^Galison, Peter (2003). Einstein's Clocks, Poincaré's Maps: Empires of Time. New York: W. W. Norton. ISBN 978-0-393-02001-4..

[57]
^Einstein (1905b),"Meinem Freunde Herr Dr. Marcel Grossmann gewidmet (Dedicated to my friend, Dr. Marcel Grossmann)"..

[58]
^Einstein (1926b),chap. "A New Determination of Molecular Dimensions"..

[59]
^"Associate Professor at the University of Zurich und professor in Prague (1909–1912)" (digital library). Einstein Online (in German and English). Bern, Switzerland: ETH-Bibliothek Zurich, ETH Zürich, www.ethz.ch. 2014. Retrieved 17 August 2014.CS1 maint: Unrecognized language (link).

[60]
^Isaacson (2007), p. 164..

[61]
^von Hirschhausen, Ulrike (2007). "Von imperialer Inklusion zur nationalen Exklusion:Staatsbürgerschaft in Österreich- Ungarn 1867–1923" (PDF) (WZB Discussion Paper). ZKD – Veröffentlichungsreihe der Forschungsgruppe, „Zivilgesellschaft, Citizenship und politische Mobilisierung in Europa" Schwerpunkt Zivilgesellschaft, Konflikte und Demokratie, Wissenschaftszentrum Berlin für Sozialforschung. Berlin, Germany: WZB Social Science Research Center Berlin. p. 8. ISSN 1860-4315. Retrieved 4 August 2015. Eine weitere Diskontinuität bestand viertens darin, dass die Bestimmungen der österreichischen Staatsbürgerschaft, die in den ersten Dritteln des Jahrhunderts auch auf Ungarn angewandt worden waren, seit 1867 nur noch für die cisleithanische Reichshälfte galten. Ungarn entwickelte hingegen jetzt eine eige-ne Staatsbürgerschaft..

[62]
^"Professor at the ETH Zurich (1912–1914)" (digital library). Einstein Online (in German and English). Zurich, Switzerland: ETH-Bibliothek Zurich, ETH Zürich, www.ethz.ch. 2014. Retrieved 17 August 2014.CS1 maint: Unrecognized language (link).

[63]
^Stachel (2002), p. 534..

[64]
^Weinstein (2015), pp. 18–19..

[65]
^"Director in the attic". Max-Planck-Gesellschaft, München. Retrieved 9 July 2017..

[66]
^Calaprice, Alice; Lipscombe, Trevor (2005). Albert Einstein: a biography. Greenwood Publishing Group. p. xix. ISBN 978-0-313-33080-3., Timeline, p. xix.

[67]
^Andrzej, Stasiak (2003). "Myths in science". EMBO Reports. 4 (3): 236. doi:10.1038/sj.embor.embor779. PMC 1315907..

[68]
^"Albert Einstein (1879–1955)". Royal Netherlands Academy of Arts and Sciences. Retrieved 21 July 2015..

[69]
^Hoffmann (1972), pp. 145–148..

[70]
^Fölsing (1997), pp. 499–508..

[71]
^"As Einstein Sees American", Einstein's World, a 1931 reprint with minor changes, of his 1921 essay..

[72]
^Holton, Gerald. "The migration of physicists to the United States", Bulletin of the Atomic Scientists, Educational Foundation for Nuclear Science, Inc., April 1984 pp. 18–24.

[73]
^Isaacson (2007), pp. 307–308..

[74]
^Flood, Alison (12 June 2018). "Einstein's travel diaries reveal 'shocking' xenophobia". TheGuardian.com. Retrieved 13 June 2018..

[75]
^The Nobel Prize in Physics 1921: Albert Einstein. Banquet Speech by R. Nadolny (in German). Retrieved 9 December 2015 via Nobelprize.org.

[76]
^Isaacson (2007), p. 308..

[77]
^Montes-Santiago, J. (16 July 2017). "[The meeting of Einstein with Cajal (Madrid, 1923): a lost tide of fortune]". Revista de Neurologia. 43 (2): 113–117. ISSN 0210-0010. PMID 16838259..

[78]
^Grandjean, Martin (2018). Les réseaux de la coopération intellectuelle. La Société des Nations comme actrice des échanges scientifiques et culturels dans l'entre-deux-guerres [The Networks of Intellectual Cooperation. The League of Nations as an Actor of the Scientific and Cultural Exchanges in the Inter-War Period] (in 法语). Lausanne: Université de Lausanne. pp. 296–302.

[79]
^Grandjean, Martin (2017). "Analisi e visualizzazioni delle reti in storia. L'esempio della cooperazione intellettuale della Società delle Nazioni". Memoria e Ricerca (2): 371–393. doi:10.14647/87204. See also: French version (PDF) and English summary..

[80]
^Shine, Cormac (2018). "Papal Diplomacy by Proxy? Catholic Internationalism at the League of Nations' International Committee on Intellectual Cooperation". The Journal of Ecclesiastical History. 69 (4): 785–805. doi:10.1017/S0022046917002731..

[81]
^Isaacson (2007), p. 368..

[82]
^Isaacson (2007), p. 370..

[83]
^Isaacson (2007), p. 373..

[84]
^Isaacson (2007), p. 374..

[85]
^Chaplin, Charles. Charles Chaplin: My Autobiography, Simon and Schuster, N.Y. (1964).

[86]
^Fölsing (1997), p. 659..

[87]
^Isaacson (2007), p. 404..

[88]
^Isaacson (2007), pp. 407–410..

[89]
^Richard Kroehling (July 1991). "Albert Einstein: How I See the World". American Masters. PBS. Retrieved 29 May 2018..

[90]
^Fred Jerome; Rodger Taylor (2006). Einstein on Race and Racism. Rutgers University Press. p. 10. ISBN 978-0-8135-3952-2..

[91]
^Einstein (1954), p. 197..

[92]
^Isaacson (2007), p. 422..

[93]
^"Professor Einstein with Commander Locker-Lampson", ScienceMuseum.org, UK.

[94]
^Isaacson (2007), pp. 419–420..

[95]
^Gilbert, Martin. Churchill and the Jews, Henry Holt and Company, N.Y. (2007) pp. 101, 176.

[96]
^Reisman, Arnold (20 November 2006). "What a Freshly Discovered Einstein Letter Says About Turkey Today". History News Network, George Mason University..

[97]
^Clark (1971)..

[98]
^"Denunciation of German Policy is a Stirring Event", Associated Press, 27 July 1933.

[99]
^"Their leaders in Germany have not driven out her cut-throats and her blackguards. She has chosen the cream of her culture and has suppressed it. She has even turned upon her most glorious citizen, Albert Einstein, who is the supreme example of the selfless intellectual...The man, who, beyond all others, approximates a citizen of the world, is without a home. How proud we must be to offer him temporary shelter.".

[100]
^"Stateless Jews: The Exiles from Germany, Nationality Plan", The Guardian (UK) 27 July 1933.

[101]
^Fölsing (1997), pp. 649, 678..

[102]
^Arntzenius, Linda G. (2011). Institute for Advanced Study. Arcadia Publishing. p. 19. ISBN 978-0-7385-7409-7..

[103]
^"Oxford Jewish Personalities". Oxford Chabad Society. Retrieved 7 March 2015..

[104]
^"How Einstein fled from the Nazis to an Oxford college". The Oxford Times. 2012. Retrieved 7 March 2015..

[105]
^Fölsing (1997), pp. 686–687..

[106]
^"In Brief". Institute for Advanced Study. Archived from the original on 29 March 2010. Retrieved 4 March 2010..

[107]
^Isaacson (2007), p. 630..

[108]
^Gosling, F. G. (2010). "The Manhattan Project: Making the Atomic Bomb". U.S. Department of Energy, History Division. p. vii. Retrieved 7 June 2015..

[109]
^Lanouette, William; Silard, Bela (1992). Genius in the Shadows: A Biography of Leo Szilárd: The Man Behind The Bomb. New York: Charles Scribner's Sons. pp. 198–200. ISBN 978-0-684-19011-2..

[110]
^J. Diehl, Sarah; Moltz, James Clay (2008). Nuclear Weapons and Nonproliferation: A Reference Handbook. ABC-CLIO. p. 218. ISBN 978-1-59884-071-1. Retrieved 7 June 2015..

[111]
^Hewlett, Richard G.; Anderson, Oscar E. (1962). The New World, 1939–1946 (PDF). University Park: Pennsylvania State University Press. pp. 15–16. ISBN 978-0-520-07186-5. OCLC 637004643. Retrieved 7 June 2015..

[112]
^Einstein, Albert (1952). "On My Participation in the Atom Bomb Project". AJ Software & Multimedia. Retrieved 7 June 2015 – via atomicarchive.org..

[113]
^Clark (1971), p. ???..

[114]
^Isaacson (2007), p. 432..

[115]
^Calaprice, Alice (2005) The new quotable Einstein Archived 22 6月 2009 at the Wayback Machine. pp. 148–149 Princeton University Press, 2005..

[116]
^Robeson, Paul. Paul Robeson Speaks, Citadel (2002) p. 333.

[117]
^"Albert Einstein, Civil Rights activist", Harvard Gazette, 12 April 2007.

[118]
^Jerome, Fred (December 2004). "Einstein, Race, and the Myth of the Cultural Icon". Isis. 95 (4): 627–639. doi:10.1086/430653. JSTOR 10.1086/430653..

[119]
^Isaacson (2007), p. 290..

[120]
^Rowe, David E. and Schulmann, Robert, editors. Einstein on Politics, Princeton University Press (2007).

[121]
^"ISRAEL: Einstein Declines". Time. 1 December 1952. Retrieved 31 March 2010..

[122]
^Rosenkranz, Ze'ev (6 November 2002). The Einstein Scrapbook. Baltimore, Maryland: Johns Hopkins University Press. p. 103. ISBN 978-0-8018-7203-7..

[123]
^Isaacson (2007), p. 522..

[124]
^Duchen, Jessica (28 January 2011). "The relative beauty of the violin". The Independent..

[125]
^"Einstein and his love of music" (PDF). Physics World. January 2005. Archived from the original (PDF) on 28 August 2015..

[126]
^Peter Galison; Gerald James Holton; Silvan S. Schweber (2008). Einstein for the 21st Century: His Legacy in Science, Art, and Modern Culture. Princeton University Press. pp. 161–164. ISBN 978-0-691-13520-5..

[127]
^Article "Alfred Einstein", in The New Grove Dictionary of Music and Musicians, ed. Stanley Sadie. 20 vol. London, Macmillan Publishers Ltd., 1980. ISBN 1-56159-174-2.

[128]
^The Concise Edition of Baker's Biographical Dictionary of Musicians, 8th ed. Revised by Nicolas Slonimsky. New York, Schirmer Books, 1993. ISBN 0-02-872416-X.

[129]
^Cariaga, Daniel, "Not Taking It with You: A Tale of Two Estates," Los Angeles Times, 22 December 1985. Retrieved April 2012..

[130]
^Auction listing by RR Auction, auction closed 13 October 2010..

[131]
^Einstein, Albert (May 1949). Sweezy, Paul; Huberman, Leo, eds. "Why Socialism?". Monthly Review. 1 (1)..

[132]
^Rowe, David E. & Robert Schulmann (8 June 2007). David A., Walsh, ed. "What Were Einstein's Politics?". History News Network. Retrieved 29 July 2012..

[133]
^Isaacson (2007), p. 487, 494, 550..

[134]
^Waldrop, Mitch (19 April 2017). "Why the FBI Kept a 1,400-Page File on Einstein". National Geographic..

[135]
^"Einstein on Gandhi (Einstein's letter to Gandhi – Courtesy:Saraswati Albano-Müller & Notes by Einstein on Gandhi – Source: The Hebrew University of Jerusalem )". Gandhiserve.org. 18 October 1931. Archived from the original on 17 January 2012. Retrieved 24 January 2012..

[136]
^Hitchens, Christopher (6 November 2007). "Selected Writings on Religion: Albert Einstein Compiled by Miguel Chavez". The Portable Atheist: Essential Readings for the Nonbeliever. Da Capo Press. p. 155. ISBN 978-0-306-81608-6..

[137]
^Calaprice, Alice (2010). The Ultimate Quotable Einstein. Princeton: Princeton University Press, p. 325..

[138]
^Calaprice, Alice (2000). The Expanded Quotable Einstein. Princeton: Princeton University Press, p. 218..

[139]
^Isaacson, Walter (2008). Einstein: His Life and Universe. New York: Simon and Schuster, p. 390..

[140]
^Calaprice, Alice (2010). The Ultimate Quotable Einstein. Princeton NJ: Princeton University Press, p. 340. Letter to M. Berkowitz, 25 October 1950. Einstein Archive 59–215..

[141]
^Isaacson, Walter (2008). Einstein: His Life and Universe. New York: Simon and Schuster, p. 461..

[142]
^Dowbiggin, Ian (2003). A Merciful End. New York: Oxford University Press, p. 41..

[143]
^Einstein, Albert (1995). Ideas And Opinions. New York: Random House, p. 62..

[144]
^"The Case of the Scientist with a Pulsating Mass". 14 June 2002. Retrieved 11 June 2007..

[145]
^Albert Einstein Archives (April 1955). "Draft of projected Telecast Israel Independence Day, April 1955 (last statement ever written)". Einstein Archives Online. Archived from the original on 13 March 2007. Retrieved 14 March 2007..

[146]
^Cohen, J. R.; Graver, L. M. (November 1995). "The ruptured abdominal aortic aneurysm of Albert Einstein". Surgery, Gynecology & Obstetrics. 170 (5): 455–458. PMID 2183375..

[147]
^Cosgrove, Ben (14 March 2014). "The Day Albert Einstein Died: A Photographer's Story". Time. Retrieved 24 April 2018..

[148]
^"The Long, Strange Journey of Einstein's Brain". NPR. Retrieved 3 October 2007..

[149]
^O'Connor, J. J.; Robertson, E.F. (1997). "Albert Einstein". The MacTutor History of Mathematics archive. School of Mathematics and Statistics, University of St. Andrews..

[150]
^Late City, ed. (18 April 1955). Written at Princeton, NJ. "Dr. Albert Einstein Dies in Sleep at 76; World Mourns Loss of Great Scientist, Rupture of Aorta Causes Death, Body Cremated, Memorial Here Set". The New York Times. CIV (35, 514). New York (published 19 April 1955). p. 1. ISSN 0362-4331..

[151]
^Oppenheimer, J. Robert (March 1979). "Oppenheimer on Einstein". The Bulletin of the Atomic Scientists. 35 (3): 38. Retrieved 12 January 2017..

[152]
^Stachel (2008)..

[153]
^Overbye, Dennis (4 December 2014). "Thousands of Einstein Documents Are Now a Click Away". The New York Times. Retrieved 4 January 2015..

[154]
^"Einstein archive at the Instituut-Lorentz". Instituut-Lorentz. 2005. Retrieved on 21 November 2005..

[155]
^"Investigations into the origin of Einstein's Sink". Studium. 11: 260–268. 2019. doi:10.18352/studium.10183..

[156]
^Das, Ashok (2003). Lectures on quantum mechanics. Hindustan Book Agency. p. 59. ISBN 978-81-85931-41-8..

[157]
^Spielberg, Nathan; Anderson, Bryon D. (1995). Seven ideas that shook the universe (2nd ed.). John Wiley & Sons. p. 263. ISBN 978-0-471-30606-1..

[158]
^Major, Fouad G. (2007). The quantum beat: principles and applications of atomic clocks (2nd ed.). Springer. p. 142. ISBN 978-0-387-69533-4..

[159]
^Lindsay, Robert Bruce; Margenau, Henry (1981). Foundations of physics. Ox Bow Press. p. 330. ISBN 978-0-918024-17-6..

[160]
^Kuepper, Hans-Josef. "List of Scientific Publications of Albert Einstein". Einstein-website.de. Retrieved 3 April 2011..

[161]
^Levenson, Thomas (9 September 1997). "Einstein's Big Idea" (public broadcaster website). Boston: WBGH. Retrieved 20 June 2015 – via NOVA by Public Broadcasting Service (PBS)..

[162]
^Einstein (1905d)..

[163]
^Fölsing (1997), pp. 178–198..

[164]
^In his paper, Einstein wrote: "The introduction of a 'luminiferous æther' will be proved to be superfluous in so far, as according to the conceptions which will be developed, we shall introduce neither a 'space absolutely at rest' endowed with special properties, nor shall we associate a velocity-vector with a point in which electro-magnetic processes take place.".

[165]
^Stachel (2002), pp. vi, 15, 90, 131, 215..

[166]
^For a discussion of the reception of relativity theory around the world, and the different controversies it encountered, see the articles in Thomas F. Glick, ed., The Comparative Reception of Relativity (Kluwer Academic Publishers, 1987), ISBN 90-277-2498-9..

[167]
^Pais (1982), pp. 382–386..

[168]
^Pais (1982), pp. 151–152..

[169]
^Einstein (1923)..

[170]
^Pais (1982), pp. 179–183..

[171]
^Stachel (2008), pp. 273–274,vol. 2: The Swiss Years: Writings, 1900–1909..

[172]
^Pais (1982), pp. 194–195..

[173]
^Einstein, A (June 1916). "Näherungsweise Integration der Feldgleichungen der Gravitation". Sitzungsberichte der Königlich Preussischen Akademie der Wissenschaften Berlin. part 1: 688–696..

[174]
^Einstein, A (1918). "Über Gravitationswellen". Sitzungsberichte der Königlich Preussischen Akademie der Wissenschaften Berlin. part 1: 154–167..

[175]
^"Found! Gravitational Waves, or a Wrinkle in Spacetime". Nadia Drake. National Geographic. 11 February 2016. Archived from the original on 12 February 2016. Retrieved 6 July 2016..

[176]
^"Gravity investigated with a binary pulsar-Press Release: The 1993 Nobel Prize in Physics". Nobel Foundation. Retrieved 6 July 2016..

[177]
^Abbott, Benjamin P.; et al. (LIGO Scientific Collaboration and Virgo Collaboration) (2016). "Observation of Gravitational Waves from a Binary Black Hole Merger". Phys. Rev. Lett. 116 (6): 061102. arXiv:1602.03837. Bibcode:2016PhRvL.116f1102A. doi:10.1103/PhysRevLett.116.061102. PMID 26918975. Lay summary (PDF)..

[178]
^"Gravitational Waves: Ripples in the fabric of space-time". LIGO | MIT. 11 February 2016. Retrieved 12 February 2016..

[179]
^"Scientists make first direct detection of gravitational waves". Jennifer Chu. MIT News. Retrieved 12 February 2016..

[180]
^Ghosh, Pallab (11 February 2016). "Einstein's gravitational waves 'seen' from black holes" (in 英语). BBC News. Retrieved 12 February 2016..

[181]
^Overbye, Dennis (11 February 2016). "Gravitational Waves Detected, Confirming Einstein's Theory". The New York Times. ISSN 0362-4331. Retrieved 12 February 2016..

[182]
^van Dongen, Jeroen (2010) Einstein's Unification Cambridge University Press, p. 23..

[183]
^Einstein (1917a)..

[184]
^Pais (1994), pp. 285–286..

[185]
^North, J.D. (1965). The Measure of the Universe: A History of Modern Cosmology. New York: Dover. pp. 81–83..

[186]
^Einstein, A. 1931. Zum kosmologischen Problem der allgemeinen Relativitätstheorie Sitzungsb.König. Preuss. Akad. 235–237.

[187]
^O'Raifeartaigh, C; McCann, B (2014). "Einstein's cosmic model of 1931 revisited: An analysis and translation of a forgotten model of the universe". The European Physical Journal H. 39 (2014): 63–85. arXiv:1312.2192. Bibcode:2014EPJH...39...63O. doi:10.1140/epjh/e2013-40038-x..

[188]
^Einstein, A; de Sitter, W (1932). "On the relation between the expansion and the mean density of the universe". Proceedings of the National Academy of Sciences. 18 (3): 213–214. Bibcode:1932PNAS...18..213E. doi:10.1073/pnas.18.3.213. PMC 1076193..

[189]
^Nussbaumer, Harry (2014). "Einstein's conversion from his static to an expanding universe". Eur. Phys. J. H. 39 (1): 37–62. arXiv:1311.2763. Bibcode:2014EPJH...39...37N. doi:10.1140/epjh/e2013-40037-6..

[190]
^Nussbaumer and Bieri (2009). Discovering the Expanding Universe. Cambridge: Cambridge University Press. pp. 144–152..

[191]
^Zimmer, Carl (9 June 2013). "The Genius of Getting It Wrong". The New York Times..

[192]
^Castelvecchi, Davide (2014). "Einstein's lost theory uncovered". Nature News & Comment. 506 (7489): 418–419. doi:10.1038/506418a. PMID 24572403..

[193]
^"On His 135th Birthday, Einstein is Still Full of Surprises". Out There. 14 March 2014..

[194]
^O'Raifeartaigh, C.; McCann, B.; Nahm, W.; Mitton, S. (2014). "Einstein's steady-state theory: an abandoned model of the cosmos". Eur. Phys. J. H. 39 (3): 353–369. arXiv:1402.0132. Bibcode:2014EPJH...39..353O. doi:10.1140/epjh/e2014-50011-x..

[195]
^Nussbaumer, H. (2014). 'Einstein’s aborted attempt at a dynamic steady state universe’. Physics ArXiv preprint at https://arxiv.org/abs/1402.4099.

[196]
^Hoyle (1948). "A New Model for the Expanding Universe". MNRAS. 108 (5): 372. Bibcode:1948MNRAS.108..372H. doi:10.1093/mnras/108.5.372..

[197]
^Bondi; Gold (1948). "The Steady-State Theory of the Expanding Universe". MNRAS. 108 (3): 252. Bibcode:1948MNRAS.108..252B. doi:10.1093/mnras/108.3.252..

[198]
^"Einstein's Lost Theory Describes a Universe Without a Big Bang – The Crux". The Crux. 7 March 2014..

[199]
^Einstein, Albert & Rosen, Nathan (1935). "The Particle Problem in the General Theory of Relativity". Physical Review. 48 (1): 73. Bibcode:1935PhRv...48...73E. doi:10.1103/PhysRev.48.73..

[200]
^"2015 – General Relativity's Centennial". American Physical Society. 2015. Retrieved 7 April 2017..

[201]
^Lindley, David (25 March 2005). "Focus: The Birth of Wormholes". Physics. 15. Retrieved 7 April 2017..

[202]
^Einstein (1905a)..

[203]
^Celebrating Einstein "Solid Cold". U.S. DOE., Office of Scientific and Technical Information, 2011..

[204]
^Einstein (1924)..

[205]
^"Cornell and Wieman Share 2001 Nobel Prize in Physics". 9 October 2001. Archived from the original on 10 June 2007. Retrieved 11 June 2007..

[206]
^Pais (1982), p. 522..

[207]
^Stachel (2008), pp. 270ff,vol. 4: The Swiss Years: Writings, 1912–1914..

[208]
^Einstein (1917b)..

[209]
^Andrews, Robert (2003). The New Penguin Dictionary of Modern Quotations. Penguin UK. p. 499. ISBN 978-0-14-196531-4..

[210]
^Stone (2013), pp. 181–192..

[211]
^Isaacson (2007), pp. 460–461..

[212]
^Bohr N. "Discussions with Einstein on Epistemological Problems in Atomic Physics". The Value of Knowledge: A Miniature Library of Philosophy. Marxists Internet Archive. Archived from the original on 13 September 2010. Retrieved 30 August 2010. From Albert Einstein: Philosopher-Scientist (1949), publ. Cambridge University Press, 1949. Niels Bohr's report of conversations with Einstein..

[213]
^Einstein (1969),A reprint of this book was published by Edition Erbrich in 1982..

[214]
^Einstein,Podolsky & Rosen (1935)..

[215]
^Isaacson (2007),pp. 448–453..

[216]
^Penrose (2007), p. 583..

[217]
^Fine (2017)..

[218]
^Einstein (1950)..

[219]
^Moore, Walter (1989). Schrödinger: Life and Thought. Cambridge: Cambridge University Press. ISBN 978-0-521-43767-7..

[220]
^Goettling, Gary. Einstein's refrigerator Georgia Tech Alumni Magazine. 1998. Retrieved on 12 November 2014. Leó Szilárd, a Hungarian physicist who later worked on the Manhattan Project, is credited with the discovery of the chain reaction.

[221]
^In September 2008 it was reported that Malcolm McCulloch of Oxford University was heading a three-year project to develop more robust appliances that could be used in locales lacking electricity, and that his team had completed a prototype Einstein refrigerator. He was quoted as saying that improving the design and changing the types of gases used might allow the design's efficiency to be quadrupled.Alok, Jha (21 September 2008). "Einstein fridge design can help global cooling". The Guardian. UK. Archived from the original on 24 January 2011. Retrieved 22 February 2011..

[222]
^"Obituary". The New York Times. 12 July 1986. Retrieved 3 April 2011..

[223]
^"13 Plumbing Facts You Probably Didn't Know". Chicago Tribune. Archived from the original on 31 March 2016. Retrieved 10 April 2016..

[224]
^Sagan, Carl (14 March 2014). "Carl Sagan Explains Albert Einstein". The New Republic. Retrieved 10 April 2016..

[225]
^"Letters Reveal Einstein Love Life". BBC News. 11 July 2006. Retrieved 14 March 2007..

[226]
^"Einstein". Corbis Rights Representation. Archived from the original on 19 August 2008. Retrieved 8 August 2008..

[227]
^Libman, E. (14 January 1939). "Disguise". The New Yorker..

[228]
^McTee, Cindy. "Einstein's Dream for orchestra". Cindymctee.com..

[229]
^Golden, Frederic (3 January 2000). "Person of the Century: Albert Einstein". Time. Archived from the original on 21 February 2006. Retrieved 25 February 2006..

[230]
^Novak, Matt (16 May 2015). "9 Albert Einstein Quotes That Are Completely Fake". Gizmodo. Retrieved 4 May 2018..

[231]
^"Did Albert Einstein Humiliate an Atheist Professor?". Snopes. Retrieved 4 May 2018..