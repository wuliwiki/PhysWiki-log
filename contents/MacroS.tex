% 熵的宏观表达式
% keys 熵

\pentry{麦克斯韦关系\upref{MWRel},态函数\upref{statef}}

将 $S$ 写成 $p,T$ 的函数,则
\begin{equation}
\dd S=\left(\frac{\partial S}{\partial T}\right)_p \dd T+\left(\frac{\partial S}{\partial p}\right)_T \dd p~.
\end{equation}
由于 $T\dd S=\delta Q$,所以等压过程中熵随温度的变化乘以温度就是吸热,即 $C_p=T\left(\frac{\partial S}{\partial T}\right)_p$。再由麦克斯韦关系\upref{MWRel},可以将上式化简为
\begin{equation}
\dd S=\frac{C_p}{T}\dd T-\left(\frac{\partial V}{\partial T}\right)_p\dd p~,
\end{equation}

所以
\begin{equation}\label{eq_MacroS_1}
S=\int \left(\frac{C_p}{T}\dd T-\left(\frac{\partial V}{\partial T}\right)_p\dd p\right)+S_0~.
\end{equation}

\begin{example}{理想气体的熵}
由理想气体状态方程,$\left(\frac{\partial V}{\partial T}\right)_p=nR/p$。积分得
\begin{equation}\label{eq_MacroS_2}
S=\int \left(\frac{C_p}{T}\dd T-\frac{nR}{p}\dd p\right)+S_0~.
\end{equation}
如果把等压热容看作是常数,那么
\begin{equation}
S=C_p\ln T-nR\ln p+S_0~.
\end{equation}

由这个公式,我们可以轻松地求得在理想气体热机循环过程中系统熵的变化,通过熵的变化容易求得过程中吸热和放热的多少。
\end{example}
