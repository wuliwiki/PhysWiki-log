% 位移与路程

\pentry{曲线的长度\upref{CurLen}, 速度、加速度\upref{VnA}}

我们已经处理了质点位矢与位移\upref{Disp} , 的问题。接下来我们要问的是,质点从$t_0$至$t_1$这段时间之内运动的距离(或者说,路程)是多少?

\begin{figure}[ht]
\centering
\includegraphics[width=8cm]{./figures/8c3d4655a47ea909.pdf}
\caption{路程等于位移吗?} \label{fig_DPAS_1}
\end{figure}

如\autoref{fig_DPAS_1} ,首先可以\textbf{排除} $s = \abs{\bvec r(t_1) - \bvec r(t_0)}  $。这段时间内位移的模长显然不等于质点通过的路程长度。

\begin{figure}[ht]
\centering
\includegraphics[width=8cm]{./figures/64dffa303c998876.pdf}
\caption{划分曲线} \label{fig_DPAS_2}
\end{figure}

这就需要我们复习一下我们计算曲线长度时所用的套路“化曲为直”\upref{CurLen}。将这段时间分成一段段很小的时间间隔,在每一段小时间内,位移的模长$\abs{\Delta \bvec r}$就近似等于路程$\Delta s$。

如\autoref{fig_DPAS_2} 所示,应该有 $$s=\sum \abs{\Delta \bvec r}$$
亦即$$
\begin{aligned}
s&=\sum \abs{\bvec v \Delta t}\\
&=\sum \abs{\bvec v} \Delta t\\
&=\int _{t_0}^{t_1} \abs{\bvec v} \dd t\\
\end{aligned}
$$
展开为分量形式(假定$\bvec v = (x',y',z')^T$,其中$x',y',z'$都是关于$t$的函数):
$$
s = \int _{t_0}^{t_1} \sqrt{x'^2+y'^2+z'^2} \dd t
$$
