% 电荷守恒、电流连续性方程
% keys 电荷守恒|连续性方程|电流
% license Usr
% type Tutor

\pentry{散度定理\upref{Divgnc}, 电流密度\upref{Idens}}
\subsection{全局电荷守恒}
一个封闭系统的电荷总量保持不变\cite{GriffE}(可以成对创造或湮灭正负电荷,但总电荷量仍然保持不变)。

\subsection{局域电荷守恒}
全局电荷守恒定理并没有阻止一些电荷从一个地点凭空消失同时在另一个地点出现;事实上,电荷守恒定理可以更精确地表述为电荷的局域守恒:在空间中任意选取一个体积, 体积内总电荷的减少速率等于由内向外通过体积表面的电流。\cite{GriffE}
\begin{equation}
\oint \bvec j \vdot \dd{\bvec s}  =  - \dv{t} \int \rho \dd{V}~.
\end{equation} 

$\bvec j$: "电流密度"

$\rho$: "电荷密度"

左侧项由散度定理\upref{Divgnc}可写为
\begin{equation}
\oint \bvec j \vdot \dd{\bvec s}  =\int \div \bvec j \dd{V}~.
\end{equation} 

右侧项由于不同变量的积分和求导可以交换,可以写为%引用未完成
\begin{equation}
\dv{t} \int \rho  \dd{V}  = \int \pdv{\rho}{t} \dd{V}~,
\end{equation}

即
\begin{equation}
\int \div \bvec j \dd{V} + \int \pdv{\rho}{t} \dd{V} = 0~.
\end{equation} 

由于该体积分对任何体积都成立, 所以
\begin{equation}\label{eq_ChgCsv_4}
\div \bvec j + \pdv{\rho}{t} = 0~.
\end{equation}

在静电学条件下, 空间中的电荷密度以及电流密度都不随时间变化, 所以有
\begin{equation}
\div \bvec j = 0~,
\end{equation}

这可以类比基尔霍夫电流定律\upref{Kirch}。
