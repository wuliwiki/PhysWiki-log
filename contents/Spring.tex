% 弹簧的串联和并联
% keys 弹簧|弹性系数|串联|并联
% license Xiao
% type Tutor


\pentry{胡克定律}{nod_c1ce}
\begin{figure}[ht]
\centering
\includegraphics[width=9cm]{./figures/94bfdb48d5a9b540.pdf}
\caption{弹簧的串联(左)、并联(右)} \label{fig_Spring_2}
\end{figure}
两个劲度系数为 $k_1, k_2$ 的弹簧的串联后劲度系数为 $k$, 那么
\begin{equation}
\frac{1}{k} = \frac{1}{k_1} + \frac{1}{k_2}
\quad \text{或} \qquad
k = \frac{k_1 k_2}{k_1 + k_2}~.
\end{equation}
若并联后劲度系数为 $k$, 那么
\begin{equation}
k = k_1 + k_2~.
\end{equation}
可见弹簧的串联和并联分别类似于\enref{电阻的并联和串联}{Rcomb}。 类似地多个弹簧串联或并联有
\begin{equation}\label{eq_Spring_2}
\frac{1}{k} = \sum_i \frac{1}{k_i}~,
\end{equation}
\begin{equation}\label{eq_Spring_3}
k = \sum_i k_i~.
\end{equation}

\subsection{推导}
\subsubsection{弹簧的串联}
先考虑两弹簧串联的情况,
如\autoref{fig_Spring_2} 左,对于弹簧的串联,设弹簧系统两边受力 $F$ 达到平衡。对弹簧 $1$ 受力分析,由平衡条件,其受弹簧 $2$ 的力大小 $F_1$ 等于外力 $F$,由牛顿第三定律,弹簧 $2$ 受弹簧 $1$ 作用力大小 $F_2$ 也为 $F$。即
\begin{equation}
F_1=F_2=F~.
\end{equation}
由胡克定律 $F=-k\Delta x$,上式可写为
\begin{equation}
k_1\Delta x_1=k_2\Delta x_2=-F~.
\end{equation}
那么对于弹簧系统,其劲度系数 $k$ 为
\begin{equation}
k=\frac{-F}{\Delta x}=\frac{-F}{\Delta x_1+\Delta x_2}=\frac{-F}{\frac{-F}{k_1}+\frac{-F}{k_2}}=\frac{k_1k_2}{k_1+k_2}~,
\end{equation}
或者
\begin{equation}
\frac{1}{k}=\frac{1}{k_1}+\frac{1}{k_2}~.
\end{equation}

现在,我们用数学归纳法推导 $n(n\in \mathbb N)$ 个弹簧的串联公式\autoref{eq_Spring_2}。

1)对于 $n=2$ 的情形,已由上面所证明;

2)假设对任意给定的 $n=l$,\autoref{eq_Spring_2} 成立,那么对 $n=l+1$ 情形,可看成前 $l$ 个弹簧串联的弹簧系统(劲度系数 $k_l$ )与第 $l+1$ 个弹簧(劲度系数 $ k_{l+1}$ )串联的情形。即对于 $n=l+1$ 个弹簧串联的系统,其劲度系数 $k$ 满足
\begin{equation}\label{eq_Spring_1}
\frac{1}{k}=\frac{1}{k_l}+\frac{1}{k_{l+1}}~.
\end{equation}

由假设
\begin{equation}
\frac{1}{k_l}=\sum\limits_{i=1}^{l}\frac{1}{k_i}~,
\end{equation}
代入\autoref{eq_Spring_1},得
\begin{equation}
\frac{1}{k_l}=\sum\limits_{i=1}^{l+1}\frac{1}{k_i}~.
\end{equation}

由数学归纳法原理,\autoref{eq_Spring_2} 成立
\subsubsection{弹簧的并联}
同样,先考虑两弹簧并联的情况,
如\autoref{fig_Spring_2} 右,对于弹簧的并联,设弹簧系统两边受力 $F$ 达到平衡。设弹簧1和弹簧2受力分别为 $F_1$ 和 $F_2$,对整个系统受力分析,由平衡条件
\begin{equation}
F_1+F_2=F~,
\end{equation}
而对并联情形,明显有整个系统伸长量 $\Delta x$、两弹簧伸长量 $\Delta x_1$ 和 $\Delta x_2$ 三者相同
\begin{equation}\label{eq_Spring_5}
\Delta x=\Delta x_1=\Delta x_2~.
\end{equation}
由胡克定律 $F=-k\Delta x$,\autoref{eq_Spring_4} 可写为
\begin{equation}\label{eq_Spring_6}
k_1\Delta x_1+k_2\Delta x_2=k\Delta x~.
\end{equation}
\autoref{eq_Spring_5} 代入\autoref{eq_Spring_6},得整个弹簧系统劲度系数 $k$
\begin{equation}
k=k_1+k_2~.
\end{equation}

与弹簧串联中证明完全类似,我们得到 $n$ 个弹簧并联时的公式\autoref{eq_Spring_3} 

\subsection{弹簧的切割}
如果把一根均匀弹簧切割成原长的 $\lambda$ ($\lambda < 1$)倍, 那么它的劲度系数变为
\begin{equation}\label{eq_Spring_4}
k' = \frac{k}{\lambda}~.
\end{equation}

证明: 我们可以把弹簧原长分割成 $n$ 等分, 由于弹簧是均匀的, 每份的劲度系数都为 $k_0$, 那么根据\autoref{eq_Spring_2} 有
\begin{equation}
k_0 = nk~,
\end{equation}
然后再把其中连续的 $m$ ($m < n$)等分串联, 有
\begin{equation}
k' = \frac{n}{m}k~.
\end{equation}
由于以上的 $m, n$ 可以任取, 我们可以使 $m/n \to x$ (当 $x$ 是有理数时取等号)。所以有\autoref{eq_Spring_4}。

\begin{example}{}\label{ex_Spring_1}
一根弹性绳劲度系数为 $k$, 固定在水平相距为 $L$ 的两点之间, 绳子原长远小于 $L$。 在距离绳一端 $x$ 处固定一个质点, 质点受重力下沉后使其平衡静止, 求下沉的深度 $h$。
\begin{figure}[ht]
\centering
\includegraphics[width=10cm]{./figures/a042122da69b57cd.pdf}
\caption{受力分析} \label{fig_Spring_1}
\end{figure}
假设质点左边部分的原长占总原长的比例为 $\lambda$, 右边部分的原长占 $1-\lambda$, 则有
\begin{equation}
\leftgroup{
&\frac{k_1}{k_2} = \frac{1-\lambda}{\lambda}\\
&\frac{1}{k} = \frac{1}{k_1} + \frac{1}{k_2}~.
}
\end{equation}
根据\autoref{eq_Spring_4} 
\begin{equation}
k_1 = \frac{k}{\lambda} ~,\qquad
k_2 = \frac{k}{1-\lambda}~.
\end{equation}
受力分析
\begin{equation}
\leftgroup{
&T_1\sin\theta_1 + T_2\sin\theta_2 = mg\\
&T_1\cos\theta_1 = T_2\cos\theta_2~,
}
\end{equation}
其中
\begin{equation}
\tan\theta_1 = \frac{h}{x}~,
\qquad
\tan\theta_2 = \frac{h}{L-x}~,
\end{equation}
\begin{equation}
T_1 = \frac{x}{\cos\theta_1} k_1 ~,\qquad
T_2 = \frac{L-x}{\cos\theta_2} k_2~.
\end{equation}
解得
\begin{equation}
h = \frac{mg}{Lk\qty(\frac{1}{x} + \frac{1}{L-x})}~.
\end{equation}
\end{example}
