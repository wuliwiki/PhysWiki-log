% 南京大学 2017 年考研普通物理
% keys 南大|南京大学|普物|普通物理
% license Copy
% type Tutor
\subsection{力学}
1. 一质量为 $m$ 的杂技演员,从蹦床上沿竖直方向跳起,当他上升到某一高度时,迅速抱起旁边栖木上质量为 $m/5$ 的猴子,结果他又上升了相同的高度后落下。试求此人达到的最大高度 $H$ 与他不抱猴子所能达到的最大高度 $H_{0}$ 之比。

2. 两个质量为 $m$ 的质点之间只有万有引力作用,假定在初始时刻,两质点相距为 $r$ ,质点 1 具有沿两质点连线方向的速度,质点 2 具有垂直于两质点连线方向的速度,两质点速度大小都是 $v_{0}$ 。试问,若两质点在以后的运动中能相距无穷远,速度 $v_0$ 需要满足什么条件?

3. 一质量为 $m$ 半径为 $r$ 的匀质实心小球沿圆弧形导轨自静止开始无滑滚下,圆弧形导轨在铅直面内,半径为 $R$ 。开始时,小球质心与圆弧的中心同高度,求小球运动到最低点时\\ 
($a$) 它的质心速率;\\
($b$) 它的角速度;\\
($c$) 它作用于导轨的正压力。
\begin{figure}[ht]
\centering
\includegraphics[width=5.5cm]{./figures/1e3bf2c8f74216d9.pdf}
\caption{力学第三题图} \label{fig_NJU17_1}
\end{figure}
\subsection{热学}
1. 一摩尔范德瓦尔斯气体的物态方程为 
\begin{equation}
\left(p+\frac{a}{V^{2}}\right)(V-b)=k_{B} T~,
\end{equation}
其中 $a, b$ 是常数。假定该气体经历了从 $V_{1}$ 到 $V_{2}$ 的等温膨胀,计算熵的变化。
\subsection{电磁学}
1. 在半径为 $a$ 的金属球和半径为 $b$ ($b>a$) 的金属球壳间填充有电导率为 $\sigma$ 的介质。设在 $t=0$ 时刻,内球上突然出现了电量为 $q_0$ 的电荷。\\
($a$) 计算 $t$ 时刻介质中的电流;\\
($b$) 问此电流一共产生多少焦耳热。
\begin{figure}[ht]
\centering
\includegraphics[width=5cm]{./figures/39e3b6abf6cfb4bb.pdf}
\caption{电磁学第一题图} \label{fig_NJU17_2}
\end{figure}
2. 半径为 $R$ 的圆形截流导线中通有电流强度为 $I$ 的稳恒电流,求圆形截流导线轴线上与圆心距离为 $x$ 的一点的磁感应强度。
\begin{figure}[ht]
\centering
\includegraphics[width=6cm]{./figures/d54a673e0c5385f6.pdf}
\caption{电磁学第二题图} \label{fig_NJU17_3}
\end{figure}
\subsection{光学}
1. 在玻璃板(折射率 $n_{1}=1.5$ ) 上表面镀一层折射率为 $n_{2}=2.5$ 的透明介质膜以增强反射。假设在镀膜过程中,用一束波长 $\lambda=6000 \opn{\mathring{A}}$ 的单色光从上方垂直照射到介质膜上,并用照度表测量透射光的强度。当介质膜的厚度逐渐增大时,透射光的强度发生时强时弱的变化。求当观察到透射光的强度第三次出现最弱时,介质膜已经镀了多厚?
