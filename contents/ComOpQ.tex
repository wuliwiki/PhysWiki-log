% 算符对易性(量子力学)
% keys 海森堡对易性|动量算符|能量算符|哈密顿算符|角动量算符
% license Xiao
% type Tutor
\addTODO{加入目录。}


\pentry{量子力学中的基本算符\nref{nod_OprQM}}{nod_1a62}

\subsection{符号介绍}

\begin{definition}{对易子}\label{def_ComOpQ_13}
对于算符$X$和$Y$,定义$[X, Y]=XY-YX$,称为算符$X$和$Y$的\textbf{对易子(commutator)},或译作\textbf{对易关系}、\textbf{交换子}。

定义$\{X, Y\}=XY+YX$,称为算符$X$和$Y$的\textbf{反对易子}。
\end{definition}


\begin{definition}{相容}\label{def_ComOpQ_17}
若算符$X, Y$满足$[X, Y]=0$,则称它们是\textbf{相容(compatible)}的,或者彼此\textbf{对易(commute)}。
\end{definition}



\begin{definition}{克罗内克函数}
\begin{equation}
\delta_{ij} = \leftgroup{
    0, \qquad i\neq j\\
    1, \qquad i = j
}~.
\end{equation}
\end{definition}




\begin{definition}{Levi-Civita符号\footnote{另见\textbf{列维—奇维塔符号}\upref{LeviCi}}。}
设$\sigma\in S_n$,即$\sigma$是正整数$1, 2, \cdots, n$之间的一个\textbf{置换},或者说集合$\{1, 2, \cdots n\}$到自身的双射。$\opn{sgn}\sigma$是$\sigma$的\textbf{逆序数}\upref{InvNum}。
\begin{equation}
\epsilon_{\sigma(1) \sigma(2) \cdots \sigma(n)}=\opn{sgn}\sigma~.
\end{equation}

而对于$i_k\in\{1, 2, \cdots, n\}$,若存在$a<b\leq n$使得$i_a=i_b$,则
\begin{equation}
\epsilon_{i_1 i_2 \cdots i_n} = 0~.
\end{equation}

\end{definition}


\subsection{对易关系}


\textbf{量子力学中的基本算符}\upref{OprQM}的对易关系列举如下\footnote{计算过程中不要忘了,要代入辅助函数,利用映射的复合来定义算子的乘法。}:

\begin{theorem}{海森堡对易关系}\label{the_ComOpQ_5}

\begin{equation}
[\hat{x}, \hat{p}_x] = \I\hbar~.
\end{equation}

\end{theorem}



海森堡对易关系是量子力学中最基本的对易关系。



\begin{corollary}{}\label{cor_ComOpQ_1}

\begin{equation}
[\hat{i}, \hat{p}_j]=\I\hbar\delta_{ij}~,
\end{equation}
其中$i, j\in\{x, y, z\}$,$\delta_{ij}$是\textbf{克罗内克 delta 函数}\upref{Kronec}。
\end{corollary}

\begin{theorem}{空间角动量算符的对易关系}\label{the_ComOpQ_6}

定义空间角动量算符$\hat{\bvec{L}}$的$x$分量为$\hat{L}_x=-\I\hbar(y\frac{\partial}{\partial_z}-z\frac{\partial}{\partial _y})$。定义$\hat{\bvec{L}}^2=\hat{\bvec{L}}\cdot \hat{\bvec{L}}=\hat{L}_x^2+\hat{L}_y^2+\hat{L}_z^2$。


\begin{equation}
[\hat{p}_i, \hat{L}_j]=\epsilon_{ijk}\I\hbar\hat{p}_k~,
\end{equation}


\begin{equation}
[\hat{i}, \hat{L}_j]=\epsilon_{ijk}\I\hbar\hat{k}~,
\end{equation}



\begin{equation}
[\hat{L}_i, \hat{L}_j]=\epsilon_{ijk}\I\hbar\hat{L}_k~,
\end{equation}

\begin{equation}
[\hat{\bvec{L}}^2, \hat{L}_i]=0~,
\end{equation}

\begin{equation}
\{\hat{L}_i, \hat{L}_j\}=\delta_{ij}\frac{\hbar^2}{2}~,
\end{equation}

\begin{equation}
\hat{\bvec{L}}\times\hat{\bvec{L}}=\I\hbar \hat{\bvec{L}}~.
\end{equation}

\end{theorem}




\addTODO{引用自旋算符内容为证明吧}
\begin{theorem}{自旋算符的对易关系}\label{the_ComOpQ_8}



\begin{equation}
[\hat{S}_i, \hat{S}_j]=\epsilon_{ijk}\I\hbar S_k~,
\end{equation}

\begin{equation}
\{\hat{S}_i, \hat{S}_j\}=\delta_{ij}\frac{\hbar^2}{2}~.
\end{equation}


\end{theorem}






