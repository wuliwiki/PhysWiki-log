% 字长
% 字长|机器字长

\begin{issues}
\issueDraft
\end{issues}

\footnote{参考 Wikipedia \href{https://en.wikipedia.org/wiki/Word_(computer_architecture)}{相关页面} 以及 \cite{唐计}.}字长或机器字长是指中央处理器(CPU)一次操作所能处理的数据的二进制位数.字长通常是与处理器寄存器的位数有关.

字长度在计算机结构和操作的多个方面均有体现.计算机中大多数寄存器的大小是一个字长.机器处理的典型数值也可能是以字长为单位.CPU和内存之间的数据传送单位也通常是一个字长.还有在内存中用于指明一个存储位置的地址也经常是以字长为单位的.

机器字长会影响机器的运算速度.若字长较短,对于相同数量的数据,可能需要较多计算次数.
