% 流体力学守恒方程
% 流体力学|守恒方程

\pentry{流体运动的描述方法\upref{fluid1},物质导数(实质导数)\upref{fluid2},重积分、面积分、体积分(简明微积分)\upref{IntN}}

人们一般用密度 $\rho$,速度 $\bvec v$ 等物理量描述流体中每个位置流体微元的性质,并列出所谓的流体力学方程来描述每个流体微元随时间的演化。它们之间应当满足一定的守恒方程,例如\textbf{连续性方程,动量守恒方程,能量守恒方程}等。在这一节中,我们将对流体力学的守恒方程进行简要的推导和介绍。
\subsection{连续性方程(流守恒方程)}
让我们对物质体(Material Volumn)进行分析,取一个随流体一起运动的被闭曲面 $S$ 包围的体积 $V$,对其中的密度进行积分得到这一个物质体的质量。物质体在随流体一起运动的过程中,其质量应当是不变的,那么我们有
\begin{equation}
\frac{\dd}{\dd t}\int_{V(t)} \rho \dd V=0
\end{equation}
进一步将它拆成两部分,一部分是每个位置密度的偏导数随时间的变化,另一部分是闭合曲面 $S$ 的变化带来的质量变化。我们得到
\begin{equation}
\int \pdv{\rho}{t}\dd V+\int_S \rho \bvec u\cdot \hat{\bvec n}\dd S=0\Rightarrow \int \qty[\pdv{\rho}{t}+\nabla\cdot(\rho\bvec u)]\dd V=0
\end{equation}
如上,我们得到了连续性方程的微分形式
\begin{equation}\label{fluidC_eq6}
\pdv{\rho}{t}+\nabla\cdot(\rho\bvec u)=0
\end{equation}
利用 $\nabla\cdot (\rho\bvec u)=\rho \nabla\cdot u+\bvec u\cdot \nabla \rho$,上式可以改写为
\begin{equation}\label{fluidC_eq1}
\frac{\dd \rho}{\dd t}+\rho \nabla\cdot u=0
\end{equation}
我们也可以用另一种方式来考察这一结果。直接从 $\dv{t} (\rho \delta V)=0$ 出发(这表明一个流体微元随时间演化的过程中质量是不变的),得到
\begin{equation}
\rho \dv{\delta V}{t}+\delta V\dv{\rho}{t}=0\Rightarrow \dv{\delta V}{t}=-\delta V \frac{1}{\rho}\frac{\dd \rho}{\dd t}
\end{equation}
利用 \autoref{fluidC_eq1},上式得到了以下结果:
\begin{equation}
\frac{1}{\delta V}\dv{\delta V}{t}=\nabla\cdot \bvec u
\end{equation}
这个结果是不难理解的。取 $\delta V$ 为一个无限小立方体,计算六个面流体运动的流进流出,可以轻易得到这个表达式,然后事实上不论 $\delta V$ 的形状如何,上式都是成立的,因为我们总能把 $\delta V$ 分割成许许多多个小立方体,而每个小立方体都满足上式,进而利用 $\nabla\cdot u$ 在这一小区域的连续性,就可以得到这个结果。
\subsection{动量守恒方程}
类似上面的推导,我们先对物质体列出积分方程:
\begin{equation}\label{fluidC_eq2}
\dv{t}\int_{V(t)} \rho \bvec u \dd V=\int \rho \bvec g\dd V+\int_S \bvec f\dd S
\end{equation}
其中 $\bvec g$ 表示作用于每个流体微元上的体力,例如重力就是体力的一种。对上式左侧进行化简,可以得到
\begin{equation}
\int \pdv{(\rho \bvec u)}{t}\dd V+\int_S\rho \bvec u (\bvec u\cdot \hat{\bvec n})\dd S=\int \rho\bvec g\dd V+\int_S \hat n\cdot \overleftrightarrow {\bvec T}\dd S
\end{equation}
其中 $T$ 是流体的应力张量,$T_{ij}$ 表示在法线为 $x_i$ 方向的单位面元上,面外对面内的面力的 $x_j$ 分量。可以证明,为了保证角动量守恒,应力张量是二阶对称的张量。
最终我们有矢量表达式
\begin{equation}\label{fluidC_eq3}
\rho \frac{\partial \bvec u}{\partial t}+\rho(\bvec u\cdot \nabla)\bvec u=\rho \bvec g+\nabla\cdot \overleftrightarrow {\bvec T}
\end{equation}

或者我们有一种更简单的推导方法。注意到 $\dd(\rho \bvec u \delta V)/\dd t=\rho \delta V \dd {\bvec u}/\dd t+\bvec u \dd (\rho \delta V)/\dd t=\rho \delta V \dd {\bvec u}/\dd t$。所以从\autoref{fluidC_eq2} 出发立刻可以得到
\begin{equation}
\rho \dv{\bvec u}{t}=\rho \bvec g+\nabla\cdot \overleftrightarrow {\bvec T}
\end{equation}
这和\autoref{fluidC_eq3} 是等价的。

应力张量 $T_{ij}$ 一般表示为 $-p+\tau_{ij}$,$p$ 表示压强,$\tau_{ij}$ 则是由流体的粘性带来的项,$\tau_{ij}$ 是对称张量。

\subsection{能量守恒方程}
最后让我们讨论能量守恒方程。流体微元的能量一般分为内能 $e$ 和动能 $\frac{1}{2}\rho u^2$,前者和流体的热力学状态(压强、温度或密度)有关,而后者和流体微元的运动有关。下面我们来具体分析能量守恒方程的形式。

\begin{equation}
\dv{t} \int_{V(t)} \rho\qty(e+\frac{1}{2}|\bvec u|^2)\dd V=\int \rho \bvec g\cdot \bvec u \dd V+\int_S \bvec f\cdot \bvec u \dd S -\int_S \bvec q\cdot \bvec n\dd A
\end{equation}
$e$ 为单位质量的内能。面力 $\bvec f$ 在速度方向上作用一段距离导致做功,$\bvec q$ 表示面元上的热流方向。

那么经过一系列的化简我们可以得到
\begin{equation}\label{fluidC_eq4}
\begin{aligned}
\rho \dv{t} \qty(e+\frac{1}{2}|\bvec u|^2)&=\rho \bvec g\cdot \bvec u+\pdv{x_i} (T_{ij}u_j)-\nabla\cdot \bvec q\\
&=\rho g_iu_i+\qty(-p\pdv{u_j}{x_j}+\tau_{ij}\pdv{u_j}{x_i})+u_j\qty(-\pdv{p}{x_j}+\pdv{\tau_{ij}}{x_i})-\pdv{q_i}{x_i}
\end{aligned}
\end{equation}
以上就是我们得到的能量守恒方程。下面我们对动量守恒方程点乘 $\bvec u$,来考察能量守恒方程中的\textbf{动能守恒}
\begin{equation}\label{fluidC_eq5}
\rho \dv{}{t}\qty(\frac{1}{2}|\bvec u|^2)=\rho g_iu_i+u_j\pdv{T_{ij}}{x_i}=\rho g_iu_i+u_j\qty(-\pdv{p}{x_j}+\pdv{\tau_{ij}}{x_i})
\end{equation}
将 \autoref{fluidC_eq4} 和 \autoref{fluidC_eq5} 相减,我们得到了\textbf{内能守恒}:
\begin{equation}
\begin{aligned}
\rho \dv{e}{t}&=T_{ij}\pdv{u_j}{x_i}-\pdv{q_i}{x_i}
\\&=-p\pdv{u_j}{x_j}+\tau_{ij}\pdv{u_j}{x_i}-\pdv{q_i}{x_i}=-\frac{p}{\rho}\dv{\rho}{t}-\nabla\cdot \bvec q\\
\Rightarrow \dv{e}{t}&=p\dv{\nu}{t}+\tau_{ij}S_{ij}-\nabla\cdot \bvec q,\ (\nu=1/\rho)\\
\end{aligned}
\end{equation}

如果将应力张量的表达式\autoref{NSeq_eq1}~\upref{NSeq} 代入,并利用热传导定律\upref{Heatco}将 $\bvec q$ 表达为 $-k\nabla T$,上述方程可以改写为
\begin{equation}\label{fluidC_eq7}
\rho \dv{e}{t}=-p\pdv{u_m}{x_m}+2\mu\qty(S_{ij}-\frac{1}{3}\pdv{u_m}{x_m}\delta_{ij})^2+\mu_\nu\qty(\pdv{u_m}{x_m})^2+\pdv{x_i}\qty(k\pdv{T}{x_i})
\end{equation}
