% 南京大学 2016 年考研普通物理
% keys 南大|南京大学|普物|普通物理
% license Copy
% type Tutor
\subsection{力学}
1. 一束速率为 $V_{0}$ 的水流水平地射向质量为 $m$ 的木块,使原来静止的木块获得水平速度。水与木块撞击后即附着于木块,随即自行消失。设水流横截面积为 $A$,密度为 $\rho$,木块与水平面的摩擦系数为 $\mu$。求木块获得的最终速率。

2. 一质量为 $m$ 的质点在保守力作用下沿 $x$ 方向运动,其势能为 $V=a x^{2}(b-x)$,其中 $a$ 和 $b$ 均为大于 0 的常量\\
(1) 求质点所受力的表达式;\\
(2) 确定质点的平衡位置,并讨论其稳定性;\\
(3) 若质点从原点 $x=0$ 处以 $v_{0}$ 开始运动,试问,$v_{0}$ 在什么范围内质点不可能到达无穷远处?
\subsection{热学}
1. 两个长圆筒共轴套在一起,两筒的长度均为 $L$,内筒和外筒的半径分别是 $R_{1}$ 和 $R_{2}$,内筒和外筒分别保持在恒定的温度 $T_{1}$ 和 $T_{2}$,且 $T_{1}>T_{2}$,已知两筒间的热导率为 $\kappa$,求每秒钟由内筒通过空气传到外筒的热量。
\begin{figure}[ht]
\centering
\includegraphics[width=6.5cm]{./figures/bd3099b8b0251e64.pdf}
\caption{热学第一题图} \label{fig_NJU16_1}
\end{figure}
\subsection{电磁学}
1. 设空间被两种不同的均匀线性电介质充满,两种介质的交界面是一个平面。在交界面处有一个电量为 $q$ 的点电荷。两种介质的相对介电常数分别为 $\varepsilon_{r 1}$ 和 $\varepsilon_{r 2}$。试求空间各点的电场强度和电位移矢量。
\begin{figure}[ht]
\centering
\includegraphics[width=6.5cm]{./figures/58ba6cb1fb01a678.pdf}
\caption{电磁学第一题图} \label{fig_NJU16_2}
\end{figure}
2. 两根长而平行的直导线,间距为 $d$,在导线中维持反向的稳恒电流 $I$\\
(1) 将这两根导线分开到相距 $2 d$,求此过程中磁场对单位长度的导线所做的功是正功还是负功;\\
(2) 在上述过程中,此两导线组成的系统单位长度的磁能变化是多少?是增加还是减少?
\subsection{光学}
1. 有一杨氏双缝干涉装置,一缝宽为 $a$,另一缝宽为 $na$,两缝中心之间的距离为 $d$ ($a$ 和 $na \ll d$)。今有一单色平行光垂直射到双缝上,接收屏与双缝所在平面的距离为 $L$ ($L \gg d$),试求\\
(1) 条纹的间距;\\
(2) 条纹对比度。
\begin{figure}[ht]
\centering
\includegraphics[width=9.5cm]{./figures/b23a011d91bfa5fa.pdf}
\caption{光学第一题图} \label{fig_NJU16_3}
\end{figure}
2. 一潜水员从水面下目测其上方空气中的物体距水面高 $2 \mathrm{~m}$,求此物体距水面的实际高度。设空气的折射率为 1,水的折射率为 1.33.
