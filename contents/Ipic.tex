% 相互作用表象
% 相互作用表象|相互作用场论
\pentry{海森堡绘景\upref{HsbPic},}

一个相互作用系统中,哈密顿量一般可以写为自由哈密顿量与相互作用哈密顿量之和 $H=H_0(m_0)+H_{int}$。我们已经知道如何在海森堡表象中讨论自由场论,但当相互作用引入时,原先的能量本征态不再是完全哈密顿量 $H$ 的本征态,系统的真空态(基态)也不再是原来的真空态,这造成了很大的困难。为此我们引入相互作用表象,试图在微扰论的框架下解决相应的问题。我们约定上下标中的 $S$ 代表薛定谔表象,$I$ 代表相互作用表象。
\subsection{相互作用场}
\pentry{标量场的量子化\upref{quanti}}
我们知道在海森堡表象下态矢量是不变的,场算符随时间的演化方程为
\begin{equation}
\phi(t,\bvec x)=e^{H(t-t_0)}\phi_S(\bvec x)e^{-H(t-t_0)}~.
\end{equation}
在 $t=t_0$ 时刻,它完全等同于薛定谔表象下的算符,但在 $t\neq t_0$ 时刻,由于相互作用的存在,它将是十分复杂的。因此,我们定义相互作用表象中的场算符为
\begin{equation}
\phi_I(t,\bvec x)=e^{H_0(t-t_0)}\phi_S(\bvec x)e^{-H_0(t-t_0)}~.
\end{equation}


相互作用表象中,场算符的时空演化算符与自由场完全相同。除此以外,它可以用时间演化算符 $U(t,t_0)$ 与海森堡表象的场 $\phi(x)$ 联系起来
\begin{equation}
\begin{aligned}
&\phi_I(x)=U_I(t,t_0)\phi(x)U_I^\dagger(t,t_0),|\psi(t)\rangle^I=U_I(t,t_0)|\psi\rangle~,\\
&U_I(t,t_0)=e^{iH_0(t-t_0)}e^{-iH(t-t_0)}~.
\end{aligned}
\end{equation}
相互作用表象下的时间演化算符 $U_I(t,t_0)$ 满足下列微分方程和初条件
\begin{equation}
i\frac{\partial}{\partial t}U_I(t,t_0)=V_I(t)U_I(t,t_0), U_I(t_0,t_0)=\mathbb{1}~,
\end{equation}
其中 $V_I(t)=e^{i(t-t_0)}H_{int} e^{-i(t-t_0)}$ 为相互作用表象下的相互作用哈密顿量。求解该微分方程,可以得到
\begin{equation}
\begin{aligned}
U_I(t,t_0)=&T\left[\exp\left(-i\int_{t_0}^{t}V_I(t')\dd t'\right)\right]\\
=&\mathbb{1}-i\int_{t_0}^t \dd t' V_I(t')+i^2\int_{t_0}^t \dd t'  \int_{t_0}^{t'} \dd t'' V_I(t')V_I(t'') \\
 &-i^3\int_{t_0}^t \dd t'  \int_{t_0}^{t'} \dd t'' \int_{t_0}^{t''}\dd t''' V_I(t')V_I(t'')V_I(t''')+\cdots
\end{aligned}~
\end{equation}
以上解的形式是编时指数函数,对编时指数函数进行幂级数展开后得到 Dyson 级数。事实上,相互作用表象下的时间演化算符可以推广到任两个时刻 $t,t'$ 之间
\begin{equation}\label{eq_Ipic_1}
\begin{aligned}
& |\psi(t)\rangle^I=U_I(t,t')|\psi(t')\rangle^I~,\\
& U_I(t,t')=e^{iH_0(t-t_0)}e^{-iH(t-t')}e^{-iH_0(t'-t_0)}~.
\end{aligned}
\end{equation}
容易验证它是一个幺正算符,并且满足
\begin{equation}
U_I(t_1,t_2)U_I(t_2,t_3)=U_I(t_1,t_3), U_I(t_1,t_3)[U_I(t_2,t_3)]^\dagger=U_I(t_1,t_2)~.
\end{equation}

