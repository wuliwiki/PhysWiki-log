% 文件 Advisory Lock 笔记(Linux)
% license Usr
% type Note

如果希望一个文件一次之被一个进程写入,且进程因任何原因中断时锁会自动释放,那么可以用 Advisory Lock (咨询锁/顾问锁),该锁由系统管理,所以即使进程崩溃也会解锁。

\begin{lstlisting}[language=cpp]
#include <iostream>
#include <fcntl.h>
#include <unistd.h>
#include <cstring>
#include <sys/types.h>
#include <sys/stat.h>

int main() {
    const char* filename = "example.txt";
    
    // Open the file for reading and writing; create it if it doesn't exist
    int fd = open(filename, O_RDWR | O_CREAT, 0666);
    if (fd == -1) {
        std::cerr << "Error opening file: " << strerror(errno) << std::endl;
        return 1;
    }

    // Set up the file lock structure
    struct flock lock;
    memset(&lock, 0, sizeof(lock));
    lock.l_type = F_WRLCK; // Write lock
    lock.l_whence = SEEK_SET;
    lock.l_start = 0;      // Start of the file
    lock.l_len = 0;        // Lock the whole file (0 means entire file)

    // Try to acquire the write lock
    if (fcntl(fd, F_SETLK, &lock) == -1) {
        std::cerr << "Unable to lock file: " << strerror(errno) << std::endl;
        close(fd);
        return 1;
    }
    
    std::cout << "File locked successfully. You can now read and write." << std::endl;
    
    // Write to the file
    const char* message = "Hello, world!\\n";
    if (write(fd, message, strlen(message)) == -1) {
        std::cerr << "Error writing to file: " << strerror(errno) << std::endl;
    }

    // Reading from the file (rewind to the beginning)
    lseek(fd, 0, SEEK_SET);
    char buffer[256];
    ssize_t bytesRead = read(fd, buffer, sizeof(buffer) - 1);
    if (bytesRead == -1) {
        std::cerr << "Error reading from file: " << strerror(errno) << std::endl;
    } else {
        buffer[bytesRead] = '\\0';
        std::cout << "Content of file:\\n" << buffer;
    }

    // Unlock the file
    lock.l_type = F_UNLCK;
    if (fcntl(fd, F_SETLK, &lock) == -1) {
        std::cerr << "Unable to unlock file: " << strerror(errno) << std::endl;
    } else {
        std::cout << "File unlocked successfully." << std::endl;
    }

    // Close the file descriptor
    close(fd);
    return 0;
}

\end{lstlisting}