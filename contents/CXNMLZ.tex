% 磁性纳米粒子
% license CCBYSA3
% type Wiki

(本文根据 CC-BY-SA 协议转载自原搜狗科学百科对英文维基百科的翻译)

磁性纳米粒子是一类可以利用磁场操纵的纳米粒子。这种颗粒通常由两种成分组成,一种是磁性材料,通常是铁、镍和钴,另一种是具有功能性的化学成分。纳米粒子的直径小于1微米(通常为1-100纳米),而较大的微珠的直径为0.5-500微米。由许多单个磁性纳米粒子组成的磁性纳米粒子簇被称为直径为50-200纳米的磁性纳米珠。[1]磁性纳米粒子簇是它们进一步磁性组装成磁性纳米链的基础。磁性纳米粒子最近成为许多研究的焦点,因为它们具有吸引人的性质,可以在催化领域看到潜在的用途,包括纳米材料基催化剂、[2] 生物医学 [3]和组织特异性靶向,[4]磁性可调胶体光子晶体,[5] 微流体、[6] 磁共振成像,[7] 磁性粒子成像、[8] 数据存储,[9] 环境修复、[10] 纳米流体,[11][12]光学过滤器、[13]缺陷传感器[14],磁冷却[15][16]和阳离子。[17]

\subsection{性能}
磁性纳米粒子的物理化学性质很大程度上取决于合成方法和化学结构。在大多数情况下,颗粒的尺寸在1至100纳米的范围内,并且可能显示出超顺磁性。

\subsection{磁性纳米粒子的类型}
\subsubsection{2.1 氧化物:铁氧体}
\begin{figure}[ht]
\centering
\includegraphics[width=8cm]{./figures/5b7d98a62fae6427.png}
\caption{具有二氧化硅外壳的磁赤铁矿磁性纳米粒子团簇的透射电镜图像。[4][5]} \label{fig_CXNMLZ_1}
\end{figure}

\subsubsection{2.2 带壳铁氧体}
磁赤铁矿或磁铁矿磁性纳米粒子的表面相对惰性,通常不允许与功能化分子形成强共价键。然而,磁性纳米粒子的反应活性可以通过在其表面涂覆一层二氧化硅来提高。[20]二氧化硅壳可以通过有机硅烷分子和二氧化硅壳之间的共价键容易地用各种表面官能团改性。[21]此外,一些荧光染料分子可以共价键合到功能化的二氧化硅壳上。[22]

由涂覆有二氧化硅壳的超顺磁性氧化物纳米粒子(每珠约80个磁赤铁矿超顺磁性纳米粒子)组成的具有窄尺寸分布的铁氧体纳米粒子簇比金属纳米粒子有几个优点:[19]
\begin{itemize}
\item 更高的化学稳定性(对生物医学应用至关重要)
\item 窄尺寸分布(对生物医学应用至关重要)
\item 更高的胶体稳定性,因为它们不会磁性聚集
\item 磁矩可以根据纳米粒子簇的大小来调节
\item 保留的超顺磁性(与纳米粒子簇的大小无关)
\item 二氧化硅表面能够直接共价功能化
\end{itemize}

\subsubsection{2.3 金属的}
\begin{figure}[ht]
\centering
\includegraphics[width=8cm]{./figures/5ccf99e5e11601a6.png}
\caption{具有石墨烯壳的钴纳米粒子(注意:单个石墨烯层是可见的)[9]} \label{fig_CXNMLZ_2}
\end{figure}
金属纳米颗粒由于其较高的磁矩而可能有利于某些技术应用,而氧化物(磁赤铁矿、磁铁矿)将有利于生物医学应用。这也意味着在同一时刻,金属纳米粒子可以做得比它们相对应的氧化物更小。另一方面,金属纳米颗粒的最大缺点是容易自燃,和对氧化剂有不同程度的反应。该缺点使对它们的处理变得困难,并导致不必要的副反应,这使得它们不太适合生物医学应用。金属颗粒的胶体形成也更具挑战性。

\subsubsection{2.4 带壳的金属}
磁性纳米粒子的金属核可以通过温和氧化、表面活性剂、聚合物和贵金属进行钝化。[23]在氧环境中,钴纳米粒子在钴纳米粒子的表面形成反铁磁CoO层。最近,人们研究了具有金外壳的钴核氧化钴壳纳米粒子的合成和交换偏置效应。[23]最近合成了具有由单质铁或钴组成的磁芯和由石墨烯制成的无活性壳的纳米粒子。与铁氧体或单质纳米粒子相比,优势在于:
\begin{itemize}
\item 更高的磁化强度
\item 在酸性和碱性溶液以及有机溶剂中具有更高的稳定性
\item 通过已知的碳纳米管方法在石墨烯表面进行化学反应[24]
\end{itemize}

\subsection{综合}
合成存在的几种制备磁性纳米粒子的方法。

\subsubsection{3.1 共沉淀}
共沉淀是在室温或高温惰性气氛下通过加入碱从$Fe^{2+}/Fe^{3+}$盐水溶液合成氧化铁($Fe_3O_4\text{或}\gamma-Fe_2O_3$)的简便方法。磁性纳米粒子的大小、形状和组成在很大程度上取决于所用盐的类型(例如氯化物、硫酸盐、硝酸盐)、$Fe^{2+}/Fe^{3+}$比率、反应温度、介质的酸碱度和离子强度,[25]以及与用于引发沉淀的基础溶液的混合速率。[25]共沉淀方法被广泛用于生产尺寸和磁性可控的铁氧体纳米粒子。[26][27][28][29]已经报道了多种实验安排,以通过快速混合促进磁性粒子的连续和大规模共沉淀。[30][31]最近,在反应物混合区内,通过集成交流磁化率计在磁铁矿纳米颗粒沉淀过程中实时测量磁性纳米颗粒的生长率。[32]

\subsubsection{3.2 热解}
较小尺寸的磁性纳米晶体基本上可以通过碱性有机金属化合物在含有稳定表面活性剂的高沸点有机溶剂中的热分解来合成。[25][33][34]

\subsubsection{3.3 微乳液}
采用微乳液技术,以1-丁醇为助表面活性剂,辛烷为油相,在十六烷基三甲基溴化铵反胶束中合成了金属钴、钴/铂合金和镀金钴/铂纳米粒子。[25][35]

\subsubsection{3.4 火焰喷雾合成}
使用火焰喷雾热解[36][36]并改变反应条件,以大于30 g/h的速率生产氧化物、金属或碳包覆的纳米颗粒。
\begin{figure}[ht]
\centering
\includegraphics[width=10cm]{./figures/a0e7c9e9fdd218db.png}
\caption{各种火焰喷涂条件及其对所得纳米粒子的影响} \label{fig_CXNMLZ_3}
\end{figure}
\begin{figure}[ht]
\centering
\includegraphics[width=10cm]{./figures/3ccbc8aae63db745.png}
\caption{常规火焰喷涂合成和还原火焰喷涂合成的操作布局差异} \label{fig_CXNMLZ_4}
\end{figure}

\subsection{潜在应用}
人们已经设想了各种各样的潜在应用。由于磁性纳米粒子的生产成本很高,人们对其回收利用或高度专业化的应用很感兴趣。它们只用于科学研究中,工业用途尚未确定。

磁性化学的潜力和多功能性源于磁性纳米粒子的快速和容易分离,消除了化学中通常应用的繁琐和昂贵的分离过程。此外,磁性纳米粒子可以通过磁场被引导到期望的位置,例如可以在抗癌中实现精确定位。

\subsubsection{4.1 医学诊断和治疗}
磁性纳米粒子已经被检测用于一种被称为磁热疗 [37]的实验癌症治疗,其中交变磁场(AMF)被用来加热纳米粒子。为了实现足够的磁性纳米粒子加热,AMF的频率通常在100-500千赫之间,尽管已经在低频和高达10兆赫的频率下进行了大量研究,磁场的振幅通常在$8-16 kAm^-1$之间。[38]

亲和配体,如表皮生长因子、叶酸、适体、凝集素等,可以通过使用各种化学物质附着到磁性纳米粒子表面。这使得磁性纳米粒子能够靶向特定的组织或细胞。[39]该策略用于癌症研究,结合磁热疗或纳米粒子递送的癌症药物靶向治疗肿瘤。然而,尽管进行了大量研究,但所有类型的癌症肿瘤中纳米颗粒的积累都不是最佳的,即使有亲和配体也是如此。威廉等人对纳米粒子输送到肿瘤进行了广泛的分析,并得出结论,到达实体肿瘤的注射剂量的中值仅为0.7\%。[40]在在肿瘤内部积累大量纳米粒子的挑战可以说是纳米医学面临的最大障碍。虽然在某些情况下使用直接注射,但静脉注射最常用于在整个肿瘤中获得良好的颗粒分布。磁性纳米粒子具有明显的优势,因为它们可以通过磁导向递送在期望的区域中累积,尽管这种技术仍需要进一步发展以实现对实体肿瘤的最佳递送。癌症的另一种潜在治疗方法包括将磁性纳米粒子附着到自由漂浮的癌细胞上,让它们被捕获并带出体外。这种疗法已经在实验室对小鼠进行了测试,并将在生存研究中进行观察。[41][42]

磁性纳米粒子可用于癌症的检测。血液中可以植入含磁性纳米颗粒的微流控芯片。当血液自由流动时,由于外加磁场,这些磁性纳米粒子被截留在血液内部。磁性纳米粒子涂有靶向癌细胞或蛋白质的抗体。磁性纳米粒子可以被回收,并且附着的癌症相关分子可以被分析以测试它们的存在。

磁性纳米粒子可以与碳水化合物结合,用于检测细菌。氧化铁颗粒已用于检测革兰氏阴性菌如大肠杆菌和革兰氏阳性菌如猪链球菌。[43][44]

\subsubsection{4.2 磁性免疫测定}
磁性免疫测定[45](MIA)是一种新型的诊断免疫测定法,利用磁性纳米珠作为标记,代替传统的酶、放射性同位素或荧光部分。这种检测包括抗体与其抗原的特异性结合,其中磁性标记与配对的一种元素相结合。然后,磁性纳米珠的存在由磁性读取器(磁力计)检测,该读取器可以测量由纳米珠引起的磁场变化。磁力计测量的信号与初始样本中分析物(病毒、毒素、细菌、心脏标志物等)的数量成比例。

\subsubsection{4.3 废水处理}
由于磁性纳米粒子通过施加磁场容易分离,并且具有非常大的表面体积比,因此磁性纳米粒子具有处理污染水的潜力。[46]在这种方法中,将乙二胺四乙酸类螯合剂附着到碳包覆的金属纳米磁体上,产生一种磁性试剂,用于从溶液或污染水中快速去除重金属,去除量级为三个数量级,浓度低至微克/升。由美国食品和药物管理局批准的氧化物超顺磁性纳米粒子(例如磁赤铁矿、磁铁矿)组成的磁性纳米珠或纳米粒子簇在废水处理方面具有很大潜力,因为它们表现出优异的生物相容性,这与金属纳米粒子相比,在材料的环境影响方面是一个优势。

\subsubsection{4.4 酶和肽的载体}
酶、蛋白质和其他生物和化学活性物质已经固定在磁性纳米粒子上。[47]它们是固相合成的可能载体而受到人们的关注。[48]

这项技术可能与细胞标记/细胞分离、生物液体解毒、组织修复、药物输送、磁共振成像、热疗和磁转染有关。[49]

\subsubsection{4.5 催化剂载体}
磁性纳米粒子有用作催化剂或催化剂载体的潜力。[50]在化学中,催化剂载体是附着催化剂的材料,通常是具有高表面积的固体。多相催化剂的反应性发生在表面原子上。因此,通过将催化剂分布在载体上,尽最大努力使其表面积最大化。载体可以是惰性的或参与催化反应。典型的载体包括各种碳、氧化铝和二氧化硅。将催化中心固定在具有大表面积/体积比的纳米颗粒上解决了这个问题。就磁性纳米粒子而言,它增加了易于分离的特性。一个早期的例子涉及附着在磁性纳米粒子上的铑催化作用。[51]
\begin{figure}[ht]
\centering
\includegraphics[width=14.25cm]{./figures/33aabbe04a4de1a9.png}
\caption\label{fig_CXNMLZ_5}
\end{figure}
在另一个例子中,稳定的自由基TEMPO通过重氮反应附着到石墨烯涂覆的钴纳米粒子上。然后将所得催化剂用于伯醇和仲醇的化学选择性氧化。[52]
\begin{figure}[ht]
\centering
\includegraphics[width=14.25cm]{./figures/d389b9510bfc7870.png}
\caption\label{fig_CXNMLZ_6}
\end{figure}
催化反应可以在连续流反应器中进行,而不是在最终产物中没有催化剂残留的间歇反应器中进行。石墨烯包覆的钴纳米粒子已经用于该实验,因为它们显示出比铁氧体纳米粒子更高的磁化强度,这对于通过外部磁场进行快速和干净的分离是必不可少的。[53]
\begin{figure}[ht]
\centering
\includegraphics[width=14.25cm]{./figures/5eaa5603075523bc.png}
\caption\label{fig_CXNMLZ_7}
\end{figure}

\subsubsection{4.6 生物医学成像}
氧化铁基纳米粒子与磁共振成像有许多应用。[54]磁性CoPt纳米粒子正被用作移植神经干细胞检测的磁共振造影剂。[55]

\subsubsection{4.7 癌症治疗}
在磁流体热疗[56]中,将不同类型的纳米粒子如氧化铁、磁铁矿、磁赤铁矿或甚至金注射到肿瘤中,然后在高频磁场下进行处理。这些纳米粒子产生的热量通常会将肿瘤温度提高到40-46℃,从而杀死癌细胞。[57][58][59]磁性纳米粒子的另一个主要潜力是将热(热疗)和药物释放结合起来用于癌症治疗的能力。大量研究表明,粒子结构可以装载药物和磁性纳米粒子。[60]最普遍的构造是“磁性脂质体”,这是一种磁性纳米粒子通常嵌入脂质双层的脂质体。在交变磁场下,磁性纳米粒子被加热,这种加热使膜透化。这导致装载药物的释放。这种治疗方案有很大的潜力,因为热疗和药物释放相结合可能比单独使用这两种方案更好地治疗肿瘤,但仍在开发中。

\subsubsection{4.8 信息存储}
面心四方相FePt合金是一种有前途的高密度存储候选材料。粒度可以小到3纳米。如果有可能以如此小的规模修改多核处理器,使用这种介质可以实现的信息密度很容易超过每平方英寸1tb。[61]

\subsubsection{4.9 遗传工程}
磁性纳米粒子可用于多种遗传学应用。一个应用是快速分离mRNA。在一种应用中,磁珠附着在聚T字尾上。当与mRNA混合时,mRNA的多聚腺苷酸尾将附着在珠子的多聚腺苷酸尾上,只需在试管的侧面放置一个磁铁并倒出液体即可进行分离。磁珠也已用于质粒组装。利用纳米珠作为锚,通过在生长的遗传链上连续添加基因,实现了快速的遗传回路构建。这种方法已被证明比以前的方法快得多,在体外用不到一个小时就能产生功能性多基因构建体。[61]

\subsection{物理建模}
有多种数学模型来描述磁性纳米粒子的旋转动力学。[62][63]简单模型包括朗之万函数和斯通-沃尔法思模型,它们描述了纳米粒子在平衡状态下的磁化强度。德拜/罗森斯威格模型可用于粒子的低振幅或高频振荡,该模型假设磁化强度对振荡磁场的线性响应。[64]非平衡方法包括朗之万方程形式和福克-普朗克方程形式,这些方法已被广泛开发用于模拟应用,例如磁性纳米粒子热疗、磁性纳米粒子成像(MPI)、[65]磁性光谱学[66]和生物传感[67]等。

\subsection{参考文献}
[1]
^磁性纳米材料,编辑:S . H . Bossmann,H . Wang,英国皇家化学学会,剑桥2017,https://pubs.rsc.org/en/content/ebook/978-1-78801-037-5.

[2]
^A.-H. Lu; W. Schmidt; N. Matoussevitch; H. Bönnemann; B. Spliethoff; B. Tesche; E. Bill; W. Kiefer; F. Schüth (August 2004). "Nanoengineering of a Magnetically Separable Hydrogenation Catalyst". Angewandte Chemie International Edition. 43 (33): 4303–4306. doi:10.1002/anie.200454222. PMID 15368378..

[3]
^A. K. Gupta; M. Gupta (June 2005). "Synthesis and surface engineering of iron oxide nanoparticles for biomedical applications". Biomaterials. 26 (18): 3995–4021. doi:10.1016/j.biomaterials.2004.10.012. PMID 15626447..

[4]
^Ramaswamy, B; Kulkarni, SD; Villar, PS; Smith, RS; Eberly, C; Araneda, RC; Depireux, DA; Shapiro, B (24 June 2015). "Movement of magnetic nanoparticles in brain tissue: mechanisms and safety". Nanomedicine: Nanotechnology, Biology and Medicine. 11 (7): 1821–9. doi:10.1016/j.nano.2015.06.003. PMC 4586396. PMID 26115639..

[5]
^He, Le; Wang, Mingsheng; Ge, Jianping; Yin, Yadong (18 September 2012). "Magnetic Assembly Route to Colloidal Responsive Photonic Nanostructures". Accounts of Chemical Research. 45 (9): 1431–1440. doi:10.1021/ar200276t. PMID 22578015..

[6]
^Kavre, Ivna; Kostevc, Gregor; Kralj, Slavko; Vilfan, Andrej; Babič, Dušan (13 August 2014). "Fabrication of magneto-responsive microgears based on magnetic nanoparticle embedded PDMS". RSC Advances. 4 (72): 38316–38322. doi:10.1039/C4RA05602G..

[7]
^Mornet, S.; Vasseur, S.; Grasset, F.; Veverka, P.; Goglio, G.; Demourgues, A.; Portier, J.; Pollert, E.; Duguet, E. (July 2006). "Magnetic nanoparticle design for medical applications". Progress in Solid State Chemistry. 34 (2–4): 237–247. doi:10.1016/j.progsolidstchem.2005.11.010..

[8]
^B. Gleich; J. Weizenecker (2005). "Tomographic imaging using the nonlinear response of magnetic particles". Nature. 435 (7046): 1214–1217. Bibcode:2005Natur.435.1214G. doi:10.1038/nature03808. PMID 15988521..

[9]
^Hyeon, Taeghwan (3 April 2003). "Chemical synthesis of magnetic nanoparticles". Chemical Communications (8): 927–934. doi:10.1039/B207789B..

[10]
^Elliott, Daniel W.; Zhang, Wei-xian (December 2001). "Field Assessment of Nanoscale Bimetallic Particles for Groundwater Treatment". Environmental Science & Technology. 35 (24): 4922–4926. Bibcode:2001EnST...35.4922E. doi:10.1021/es0108584..

[11]
^J. Philip; Shima.P.D. B. Raj (2006). "Nanofluid with tunable thermal properties". Applied Physics Letters. 92 (4): 043108. Bibcode:2008ApPhL..92d3108P. doi:10.1063/1.2838304..

[12]
^乔德里五世,王z,雷A,斯里达尔一世,拉马努金RV,自泵磁致冷,J. Phys D: Appl. Phys. 50 (3),03LT03 (2017),https://IOP science . IOP . org/article/10.1088/1361-6463/aa4f 92/meta.

[13]
^J.Philip, T.J.Kumar, P.Kalyanasundaram, B.Raj (2003). "Tunable Optical Filter". Measurement Science & Technology. 14 (8): 1289–1294. Bibcode:2003MeScT..14.1289P. doi:10.1088/0957-0233/14/8/314.CS1 maint: Multiple names: authors list (link).

[14]
^Mahendran, V. (2012). "Nanofluid based opticalsensor for rapid visual inspection of defects in ferromagnetic materials". Appl. Phys. Lett. 100 (7): 073104. Bibcode:2012ApPhL.100g3104M. doi:10.1063/1.3684969..

[15]
^乔德里五世,拉马努金五世。用于主动冷却的铁-镍-铬纳米粒子的磁热特性。科学报告。2016年;6:35156,https://www.nature.com/articles/srep35156.

[16]
^乔德里五世,陈十世,拉马努金RV。用于近室温热管理的铁和锰基磁热材料。材料科学进展。2019年;100:64-98,https://doi.org/10.1016/j.pmatsci.2018.09.005.

[17]
^Philip, V. Mahendran; Felicia, Leona J. (2013). "A Simple, In-Expensive and Ultrasensitive Magnetic Nanofluid Based Sensor for Detection of Cations, Ethanol and Ammonia". J. Nanofluids. 2 (2): 112–119. doi:10.1166/jon.2013.1050..

[18]
^An-Hui Lu, An-Hui; E. L. Salabas; Ferdi Schüth (2007). "Magnetic Nanoparticles: Synthesis, Protection, Functionalization, and Application". Angew. Chem. Int. Ed. 46 (8): 1222–1244. doi:10.1002/anie.200602866. PMID 17278160..

[19]
^Tadic, Marin; Kralj, Slavko; Jagodic, Marko; Hanzel, Darko; Makovec, Darko (December 2014). "Magnetic properties of novel superparamagnetic iron oxide nanoclusters and their peculiarity under annealing treatment". Applied Surface Science. 322: 255–264. Bibcode:2014ApSS..322..255T. doi:10.1016/j.apsusc.2014.09.181..

[20]
^Kralj, Slavko; Makovec, Darko; Čampelj, Stanislav; Drofenik, Miha (July 2010). "Producing ultra-thin silica coatings on iron-oxide nanoparticles to improve their surface reactivity". Journal of Magnetism and Magnetic Materials. 322 (13): 1847–1853. Bibcode:2010JMMM..322.1847K. doi:10.1016/j.jmmm.2009.12.038..

[21]
^Kralj, Slavko; Drofenik, Miha; Makovec, Darko (16 December 2010). "Controlled surface functionalization of silica-coated magnetic nanoparticles with terminal amino and carboxyl groups". Journal of Nanoparticle Research. 13 (7): 2829–2841. Bibcode:2011JNR....13.2829K. doi:10.1007/s11051-010-0171-4..

[22]
^Kralj, Slavko; Rojnik, Matija; Romih, Rok; Jagodič, Marko; Kos, Janko; Makovec, Darko (7 September 2012). "Effect of surface charge on the cellular uptake of fluorescent magnetic nanoparticles". Journal of Nanoparticle Research. 14 (10): 1151. Bibcode:2012JNR....14.1151K. doi:10.1007/s11051-012-1151-7..

[23]
^Johnson, Stephanie H.; C.L. Johnson; S.J. May; S. Hirsch; M.W. Cole; J.E. Spanier (2010). "Co@CoO@Au core-multi-shell nanocrystals". Journal of Materials Chemistry. 20 (3): 439–443. doi:10.1039/b919610b..

[24]
^R.N. Grass, Robert N.; E.K. Athanassiou; W.J. Stark (2007). "Covalently Functionalized Cobalt Nanoparticles as a Platform for Magnetic Separations in Organic Synthesis". Angew. Chem. Int. Ed. 46 (26): 4909–12. doi:10.1002/anie.200700613. PMID 17516598..

[25]
^A.-H. Lu; E. L. Salabas; F. Schüth (2007). "Magnetic Nanoparticles: Synthesis, Protection, Functionalization, and Application". Angew. Chem. Int. Ed. 46 (8): 1222–1244. doi:10.1002/anie.200602866. PMID 17278160..

[26]
^G.Gnanaprakash, S.Ayyappan, T.Jayakumar, John Philip & Baldev Raj (2006). "A simple method to produce magnetic nanoparticles with enhanced alpha to gamma-Fe2O3 phase transition temperature". Nanotechnology. 17 (23): 5851–5857. Bibcode:2006Nanot..17.5851G. doi:10.1088/0957-4484/17/23/023.CS1 maint: Multiple names: authors list (link).

[27]
^G. Gnanaprakash; John Philip; T. Jayakumar; Baldev Raj (2007). "Effect of Digestion Time and Alkali Addition Rate on the Physical Properties of Magnetite Nanoparticles". J. Phys. Chem. B. 111 (28): 7978–7986. doi:10.1021/jp071299b. PMID 17580856..

[28]
^S.Ayyappan, John Philip & Baldev Raj (2009). "Solvent polarity effect on physical properties of CoFe2O3 nanoparticles". J. Phys. Chem. C. 113 (2): 590–596. doi:10.1021/jp8083875..

[29]
^S. Ayyappan; S. Mahadevan; P. Chandramohan; M. P.Srinivasan; John Philip; Baldev Raj (2010). "Influence of Co2 Ion Concentration on the Size, Magnetic Properties, and Purity of CoFe2O4 Spinel Ferrite Nanoparticles". J. Phys. Chem. C. 114 (14): 6334–6341. doi:10.1021/jp911966p..

[30]
^Fun Chin, Suk; Iyer, K. Swaminathan; Raston, Colin L.; Saunders, Martin (2008). "Size Selective Synthesis of Superparamagnetic Nanoparticles in Thin Fluids under Continuous Flow Conditions". Adv. Funct. Mater. 18 (6): 922–927. doi:10.1002/adfm.200701101..

[31]
^奈杰尔·史密斯、科林·拉斯顿、马丁·桑德斯、罗伯特·伍德沃德;利用旋转圆盘法合成磁性纳米粒子.

[32]
^Ström, Valter; Olsson, Richard T.; Rao, K. V. (2010). "Real-time monitoring of the evolution of magnetism during precipitation of superparamagnetic nanoparticles for bioscience applications". J. Mater. Chem. 20 (20): 4168–4175. doi:10.1039/c0jm00043d..

[33]
^Sharifi, Ibrahim; Zamanian, Ali; Behnamghader, Aliasghar (2016-08-15). "Synthesis and characterization of Fe0.6Zn0.4Fe2O4 ferrite magnetic nanoclusters using simple thermal decomposition method". Journal of Magnetism and Magnetic Materials. 412: 107–113. doi:10.1016/j.jmmm.2016.03.091. ISSN 0304-8853..

[34]
^Monfared, A. H.; Zamanian, A.; Beygzadeh, M.; Sharifi, I.; Mozafari, M. (2017-02-05). "A rapid and efficient thermal decomposition approach for the synthesis of manganese-zinc/oleylamine core/shell ferrite nanoparticles". Journal of Alloys and Compounds. 693: 1090–1095. doi:10.1016/j.jallcom.2016.09.253. ISSN 0925-8388..

[35]
^S S.Rana, J. Philip, B.Raj (2010). "Micelle based synthesis of Cobalt Ferrite nanoparticles and its characterization using Fourier Transform Infrared Transmission Spectrometry and Thermogravimetry". Materials Chemistry and Physics. 124: 264–269. doi:10.1016/j.matchemphys.2010.06.029.CS1 maint: Multiple names: authors list (link).

[36]
^R. N. Grass, Robert N.; W. J. Stark (2006). "Gas phase synthesis of fcc-cobalt nanoparticles". J. Mater. Chem. 16 (19): 1825. doi:10.1039/B601013J..

[37]
^Rabias, I.; et al. (2010). "Rapid magnetic heating treatment by highly charged maghemite nanoparticles on Wistar rats exocranial glioma tumors at microliter volume". Biomicrofluidics. 4 (2): 024111. doi:10.1063/1.3449089. PMC 2917883. PMID 20697578..

[38]
^Kumar, CS; Mohammad, F (2011). "Magnetic nanomaterials for hyperthermia-based therapy and controlled drug delivery". Adv. Drug Deliv. Rev. 63 (9): 789–808. doi:10.1016/j.addr.2011.03.008. PMC 3138885. PMID 21447363..

[39]
^Kralj, Slavko; Rojnik, Matija; Kos, Janko; Makovec, Darko (26 April 2013). "Targeting EGFR-overexpressed A431 cells with EGF-labeled silica-coated magnetic nanoparticles". Journal of Nanoparticle Research. 15 (5): 1666. Bibcode:2013JNR....15.1666K. doi:10.1007/s11051-013-1666-6..

[40]
^Wilhelm, Stefan; Tavares, Anthony J.; Dai, Qin; Ohta, Seiichi; Audet, Julie; Dvorak, Harold F.; Chan, Warren C. W. (2016). "Analysis of nanoparticle delivery to tumours". Nature Reviews Materials. 1 (5): 16014. Bibcode:2016NatRM...116014W. doi:10.1038/natrevmats.2016.14..

[41]
^Scarberry KE, Dickerson EB, McDonald JF, Zhang ZJ (2008). "Magnetic Nanoparticle-Peptide Conjugates for in Vitro and in Vivo Targeting and Extraction of Cancer Cells". Journal of the American Chemical Society. 130 (31): 10258–62. doi:10.1021/ja801969b. PMID 18611005..

[42]
^利用磁性纳米粒子抗癌新闻方面,检索于2008年7月17日。.

[43]
^Parera Pera N; Kouki A.; Finne J.; Pieters R. J. (2010). "Detection of pathogenic Streptococcus suis bacteria using magnetic glycoparticles". Organic & Biomolecular Chemi. 8 (10): 2425–2429. doi:10.1039/C000819B. PMID 20448902..

[44]
^化学生物学亮点。Rsc.org(2007-06-13)。检索于2011年10月7日。.

[45]
^磁性免疫测定:POCT的新范例,IVDt,2008年7月/8月。.

[46]
^F.M. Koehler, Fabian M.; M. Rossier; M. Waelle; E.K. Athanassiou; L.K. Limbach; R.N. Grass; D. Günther; W.J. Stark (2009). "Magnetic EDTA: Coupling heavy metal chelators to metal nanomagnets for rapid removal of cadmium, lead and copper from contaminated water". Chem. Commun. 32 (32): 4862–4. doi:10.1039/B909447D. PMID 19652806..

[47]
^Huang-Hao Yang, Huang-Hao; Shu-Qiong Zhang; Xiao-Lan Chen; Zhi-Xia Zhuang; Jin-Gou Xu; Xiao-Ru Wang (2004). "Magnetite-Containing Spherical Silica Nanoparticles for Biocatalysis and Bioseparations". Analytical Chemistry. 76 (5): 1316–1321. doi:10.1021/ac034920m. PMID 14987087..

[48]
^K.Norén, Katarina; M. Kempe (2009). "Multilayered Magnetic Nanoparticles as a Support in Solid-Phase Peptide Synthesis". International Journal of Peptide Research and Therapeutics. 15 (4): 287–292. doi:10.1007/s10989-009-9190-3..

[49]
^Gupta AK, Ajay Kumar; Gupta M (2005). "Synthesis and surface engineering of iron oxide nanoparticles for biomedical applications". Biomaterials. 26 (18): 3995–4021. doi:10.1016/j.biomaterials.2004.10.012. PMID 15626447..

[50]
^A. Schätz, Alexander; O. Reiser; W.J. Stark (2010). "Nanoparticles as Semi-Heterogeneous Catalyst Supports". Chem. Eur. J. 16 (30): 8950–67. doi:10.1002/chem.200903462. PMID 20645330..

[51]
^Tae-Jong Yoon, Tae-Jong; Woo Lee; Yoon-Seuk Oh; Jin-Kyu Lee (2003). "Magnetic nanoparticles as a catalyst vehicle for simple and easy recyclingElectronic supplementary information (ESI) available: XRD and FT-IR data, as well as the detailed experimental conditions for the catalytic hydroformylation reactions. See http://www.rsc.org/suppdata/nj/b2/b209391j/". New Journal of Chemistry. 27 (2): 227.229. doi:10.1039/B209391J. External link in |title= (help).

[52]
^A. Schätz, Alexander; R. N. Grass; W. J. Stark; O. Reiser (2008). "TEMPO Supported on Magnetic C/Co-Nanoparticles: A Highly Active and Recyclable Organocatalyst". Chemistry: A European Journal. 14 (27): 8262–8266. doi:10.1002/chem.200801001. PMID 18666291..

[53]
^A. Schätz, Alexander; R. N. Grass; Q. Kainz; W. J. Stark; O. Reiser (2010). "Cu(II)−Azabis(oxazoline) Complexes Immobilized on Magnetic Co/C Nanoparticles: Kinetic Resolution of 1,2-Diphenylethane-1,2-diol under Batch and Continuous-Flow Conditions". Chemistry of Materials. 22 (2): 305–310. doi:10.1021/cm9019099..

[54]
^Colombo, M; et al. (2012). "Biological Applications of Magnetic Nanoparticles". Chem Soc Rev. 41 (11): 4306–34. doi:10.1039/c2cs15337h. PMID 22481569..

[55]
^Xiaoting Meng, Xiaoting; Hugh C. Seton; Le T. Lu; Ian A. Prior; Nguyen T. K. Thanh; Bing Song (2011). "Magnetic CoPt nanoparticles as MRI contrast agent for transplanted neural stem cells detection". Nanoscale. 3 (3): 977–984. Bibcode:2011Nanos...3..977M. doi:10.1039/C0NR00846J. PMID 21293831..

[56]
^Sharifi, Ibrahim; Shokrollahi, H.; Amiri, S. (2012-03-01). "Ferrite-based magnetic nanofluids used in hyperthermia applications". Journal of Magnetism and Magnetic Materials. 324 (6): 903–915. Bibcode:2012JMMM..324..903S. doi:10.1016/j.jmmm.2011.10.017. ISSN 0304-8853..

[57]
^Javidi, Mehrdad; Heydari, Morteza; Attar, Mohammad Mahdi; Haghpanahi, Mohammad; Karimi, Alireza; Navidbakhsh, Mahdi; Amanpour, Saeid (2014). "Cylindrical agar gel with fluid flow subjected to an alternating magnetic field during hyperthermia". International Journal of Hyperthermia. 31 (1): 33–39. doi:10.3109/02656736.2014.988661. PMID 25523967..

[58]
^Javidi, M; Heydari, M; Karimi, A; Haghpanahi, M; Navidbakhsh, M; Razmkon, A (2014). "Evaluation of the effects of injection velocity and different gel concentrations on nanoparticles in hyperthermia therapy". J Biomed Phys Eng. 4 (4): 151–62. PMC 4289522. PMID 25599061..

[59]
^Heydari, Morteza; Javidi, Mehrdad; Attar, Mohammad Mahdi; Karimi, Alireza; Navidbakhsh, Mahdi; Haghpanahi, Mohammad; Amanpour, Saeid (2015). "Magnetic Fluid Hyperthermia in a Cylindrical Gel Contains Water Flow". Journal of Mechanics in Medicine and Biology. 15 (5): 1550088. doi:10.1142/S0219519415500888..

[60]
^Estelrich, Joan; et al. (2015). "Iron Oxide Nanoparticles for Magnetically-Guided and Magnetically-Responsive Drug Delivery". Int. J. Mol. Sci. 16 (12): 8070–8101. doi:10.3390/ijms16048070. PMC 4425068. PMID 25867479..

[61]
^娜塔莉·弗雷和孙寿恒用于信息存储应用的磁性纳米粒子.

[62]
^Reeves, Daniel B. (2017). "Nonlinear Nonequilibrium Simulations of Magnetic Nanoparticles". Magnetic Characterization Techniques for Nanomaterials (in 英语). Springer, Berlin, Heidelberg. pp. 121–156. doi:10.1007/978-3-662-52780-1_4. ISBN 978-3-662-52779-5..

[63]
^Reeves, Daniel B.; Weaver, John B. (2014). "Approaches for Modeling Magnetic Nanoparticle Dynamics". Critical Reviews in Biomedical Engineering (in English). 42 (1): 85–93. arXiv:1505.02450. doi:10.1615/CritRevBiomedEng.2014010845. ISSN 0278-940X. PMC 4183932. PMID 25271360.CS1 maint: Unrecognized language (link).

[64]
^Carrey, J.; Mehdaoui, B.; Respaud, M. (15 April 2011). "Simple models for dynamic hysteresis loop calculations of magnetic single-domain nanoparticles: Application to magnetic hyperthermia optimization". Journal of Applied Physics. 109 (8): 083921–083921–17. Bibcode:2011JAP...109h3921C. doi:10.1063/1.3551582. ISSN 0021-8979..

[65]
^Weizenecker, J.; Gleich, B.; Rahmer, J.; Dahnke, H.; Borgert, J. (2009). "Three-dimensional real-time in vivo magnetic particle imaging". Physics in Medicine and Biology (in 英语). 54 (5): L1–L10. Bibcode:2009PMB....54L...1W. doi:10.1088/0031-9155/54/5/L01. ISSN 0031-9155. PMID 19204385..

[66]
^Reeves, Daniel B.; Weaver, John B. (15 December 2012). "Simulations of magnetic nanoparticle Brownian motion". Journal of Applied Physics. 112 (12): 124311. Bibcode:1998JChPh.109.4281T. doi:10.1063/1.477032. ISSN 0021-8979. PMC 3537703. PMID 23319830..

[67]
^Zhang, Xiaojuan; Reeves, Daniel B.; Perreard, Irina M.; Kett, Warren C.; Griswold, Karl E.; Gimi, Barjor; Weaver, John B. (15 December 2013). "Molecular sensing with magnetic nanoparticles using magnetic spectroscopy of nanoparticle Brownian motion". Biosensors and Bioelectronics. 50: 441–446. doi:10.1016/j.bios.2013.06.049. PMC 3844855. PMID 23896525..