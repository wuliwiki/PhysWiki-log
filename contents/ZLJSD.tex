% 重力加速度
% license CCBYSA3
% type Wiki

(本文根据 CC-BY-SA 协议转载自原搜狗科学百科对英文维基百科的翻译)


重力加速度(又名自由落体加速度)是一个物体受重力作用的情况下所具有的加速度。

通常指地面附近物体受地球引力作用在真空中下落的加速度,记为g。为了便于计算,其近似标准值通常取为980cm/s²或9.8m/s²。在月球、其他行星或星体表面附近物体的下落加速度,则分别称月球重力加速度、某行星或星体重力加速度。

\subsection{对于点质量}

牛顿万有引力定律指出,任何两个质点之间都有一个引力,该引力对每个质点的大小相等,并且两个质点处于同一直线,使它们相互靠近。万有引力公式为:

$F = G \frac{m_1 m_2}{r^2}$

其中 $m_1$ 和 $m_2$ 是两个质点的质量,$G$ 是万有引力常数,并且 $r$ 是两个质点之间的距离。这个公式是由行星运动推导出来的,在行星运动中,行星和太阳之间的距离使得把天体看作质点是合理的。(对于在轨卫星来说,“距离”指的是距质心的距离,而不是行星表面的高度。)

如果其中一个质点的质量比另一个大得多,就可以方便地在较大质量的质点周围定义一个引力场,如下所示:

$\mathbf{g} = -\frac{GM}{r^2} \hat{\mathbf{r}}$

其中 $M$ 是较大物体的质量,且 $\hat{\mathbf{r}}$ 是从质量较大质点指向质量较小质点的单位向量。负号表示力是一种吸引力。

这样,作用在较小质量质点上的力可以计算为:

$\mathbf{F} = m\mathbf{g}$

其中 $\mathbf{F}$ 是力矢量,$m$ 是较小质量,并且 $\mathbf{g}$ 是指向较大质量物体的向量。请注意    这个模型代表了与大质量物体相关的“远场”重力加速度。当物体的尺寸与测量距离相比并不微不足道时,叠加原理可以用于微分质量,假设整个物体的密度分布,从而获得一个更详细的“近场”重力加速度模型。对于在轨卫星,远场模型足以粗略计算高度与周期的关系,但不适用于多轨道后未来位置的精确估计。

更详细的模型包括(除其他外)地球赤道膨胀和月球不规则质量浓度(由于流星撞击)。重力恢复和气候实验(GRACE)任务于2002年发射,由两个昵称为“汤姆”和“杰瑞”的探测器组成,在环绕地球的极地轨道上测量两个探测器之间的距离差异,以便更精确地确定环绕地球的重力场,并跟踪随时间发生的变化。同样,2011-2012年的重力恢复和内部实验室(GRAIL)任务包括围绕月球的极地轨道上的两个探测器(“落潮”和“流动”),以更精确地确定重力场,用于未来的导航目的,并推断有关月球物理组成的信息。

具有加速度单位,是相对于较大物体位置的矢量函数,与较小质量的大小(甚至存在与否)无关。

\subsection{地球重力模型}

用于地球的重力模型的类型取决于给定问题所需的保真度。对于许多问题,例如飞机模拟,将重力视为常数就足够了,定义为:[2]

$\mathbf{g} =9.80665(32.1740\text{英尺}) m/s^2$

基于1984年世界大地测量系统(WGS-84)的数据,其中$\mathbf{g}$被理解为在当前参照系中指向“下”。

如果希望将地球上物体的重量作为纬度的函数进行建模,可以使用以下方法([2] 第41页):

$g = g_{45} - \frac{1}{2} \left( g_{\text{poles}} - g_{\text{equator}} \right) \cos \left( 2 \varphi \cdot \frac{\pi}{180} \right)$

其中

\begin{itemize}
\item $g_\text{poles}$=9.832 (32.26{英尺})$m/s^2$
\item $g_{45}$= 9.806米(32.17英尺)$m/s^2$
\item $g_\text{equator}$=9.780米(32.09英尺)$m/s^2$
\item $\varphi$=纬度,介于90度和90度之间。
\end{itemize}

这两种方法都不能解释重力随海拔高度的变化,但是余弦函数模型确实考虑了地球自转产生的离心释放。就质量吸引效应本身而言,由于赤道离质量中心更远赤道处的重力加速度比两极处的重力加速度小约0.18\%。当包含旋转分量时(如上所述),赤道处的重力比两极处的重力小约0.53\%,而两极处的重力不受旋转的影响。因此,由纬度引起的旋转分量的变化(0.35\%)大约是由纬度引起的质量引力变化(0.18\%)的两倍,但是两者都降低了赤道重力相对于两极重力的强度。

请注意,对于卫星来说,轨道与地球的旋转是分离的,因此轨道周期不一定是一天,而且误差会在多个轨道上累积,因此精度很重要。对于这样的问题,除非模拟经度的变化,否则地球的自转是无关紧要的。此外,重力随高度的变化变得很重要,特别是对于高度椭圆的轨道。

1996年的地球重力模型(EGM96)包含130,676个系数,这些系数完善了地球重力场模型([2] 第40页)。最重要的修正项大约比第二大的修正项重要两个数量级([2] 第40页)。该系数被称$J_2$项,用于解释地球极点的变平或扁率。(一个在对称轴上拉长的形状,像美式足球,被称为拉长(prolate)。)引力势函数可以被写成单位质量从无限接近地球时势能的变化。取该函数相对于坐标系的偏导数,随后将解析重力加速度矢量的方向分量,作为位置的函数。如果合适的话,可以根据恒星的恒星日(约366.24天/年)而不是太阳日(约365.24天/年)来计算地球自转的分量。这个分量垂直于旋转轴,而不是地球表面。

根据火星的几何形状和引力场调整的类似模型可以在美国国家航空航天局(NASA)SP-8010出版物中找到。[3]

空间某一点的重心重力加速度由下式给出:

$\mathbf{g} = -\frac{GM}{r^2} \hat{\mathbf{r}}$

其中:

M 是吸引对象的质量,$\hat{\mathbf{r}}$是从吸引物体的质心到被加速物体的质心的单位矢量, r 是两个物体之间的距离, G 是万有引力常数。

当对地球表面的物体或随地球旋转的飞机进行这种计算时,必须考虑到地球正在旋转的事实,并且必须从中减去离心加速度。例如,上式给出了当$GM = 3.986\times10^{14} m^3/s^2$,R=6.371×106 m时,$9.820 m/s^2$处的加速度。向心半径为r = R cos($\varphi$),向心时间单位约为$(day / 2\pi)$,对于R = 5×106 m,减小到$9.79379 m/s^2$,这更接近观测值。

\subsection{广义相对论}

在爱因斯坦的广义相对论中,引力是弯曲时空的一个属性,而不是由于物体间传播的力。在爱因斯坦的理论中,质量扭曲了附近的时空,其他粒子沿着时空几何形状决定的轨迹运动。重力是一种虚构的力。 没有重力加速度,因为物体在自由落体中的适当加速度和四加速度为零。自由落体中的物体不会经历加速度,而是沿着弯曲时空上的直线(测地线)运动。

\subsection{参考文献}

[1]
^Fredrick J. Bueche (1975). Introduction to Physics for Scientists and Engineers, 2nd Ed. USA: Von Hoffmann Press. ISBN 978-0-07-008836-8..

[2]
^Brian L. Stevens; Frank L. Lewis (2003). Aircraft Control And Simulation, 2nd Ed. Hoboken, New Jersey: John Wiley & Sons, Inc. ISBN 978-0-471-37145-8..

[3]
^Richard B. Noll; Michael B. McElroy (1974), Models of Mars' Atmosphere [1974], Greenbelt, Maryland: NASA Goddard Space Flight Center, SP-8010..