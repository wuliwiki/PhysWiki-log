% 平面波的的正交归一化
% 波函数|归一化|delta 函数|完备性|薛定谔方程

\pentry{一维自由粒子(量子)\upref{FreeP1}}

本文使用原子单位制\upref{AU}. 回顾一维自由粒子\upref{FreeP1}中的分析, 我们需要把能量本征态即平面波进行\textbf{动量归一化}或者\textbf{能量归一化}% 链接未完成
\begin{equation}\label{EngNor_eq1}
\int_{-\infty}^{+\infty} \psi_{k'}(x)^* \psi_{k}(x) \dd{x} = \delta(k' - k) \qquad (k \in \mathbb R)
\end{equation}
\begin{equation}\label{EngNor_eq4}
\int_{0}^{+\infty} \psi_{E',i'}(x)^* \psi_{E,i}(x) \dd{x} = \delta_{i',i}\delta(E' - E) \qquad (E > 0)
\end{equation}
其中 $*$ 表示复共轭, 满足该式的 $\psi_k$ 就是动量归一化的平面波:% 链接未完成
\begin{equation}\label{EngNor_eq2}
\psi_k(x) = \frac{\E^{\I kx}}{\sqrt{2\pi}}
\end{equation}
但什么样的归一化系数能满足能量归一化条件呢?

我们可以令定态薛定谔方程的两个线性无关解 $\psi_{E,1}(x), \psi_{E,2}(x)$ 分别为 $-k$ 和 $k$ 的平面波(暂时规定 $k = \sqrt{2mE} > 0$),只是归一化系数和\autoref{EngNor_eq2} 不同. 只需要使用 $\delta$ 函数的性质\autoref{Delta_eq12}~\upref{Delta}, 把 $E'-E$ 看成 $k$ 的函数($k'$ 视为常数)\footnote{反之亦然: $k'$ 看成变量, $k$ 看成常数}, 唯一的正根为 $k'$
\begin{equation}
f(k) = \frac{k'^2}{2m} - \frac{k^2}{2m}
\qquad
\abs{f'(k')} = \frac{k'}{m}
\end{equation}
所以
\begin{equation}
\delta(E'-E) = \delta[f(k)] = \frac{m}{k'} \delta(k'-k) = \frac{m}{k} \delta(k'-k)
\end{equation}
代入\autoref{EngNor_eq4} 得\textbf{能量归一化}后的平面波
\begin{equation}
\psi_{E,1}(x) = \sqrt{\frac{m}{2\pi k}}\E^{\I kx}
\qquad
\psi_{E,2}(x) = \sqrt{\frac{m}{2\pi k}}\E^{-\I kx}
\end{equation}

现在, 任意波函数 $\psi(x)$ 就可以用能量归一化的基底展开为% 链接未完成
\begin{equation}
\begin{aligned}
\psi(x) &= \int_0^{+\infty} [A(k) \psi_{E,1}(x) + A(-k) \psi_{E,2}(x)] \dd{E}\\
&= \int_0^{+\infty} \sqrt{\frac{k}{m}}\qty[A(k) \frac{\E^{\I kx}}{\sqrt{2\pi}} + A(-k)\frac{\E^{-\I kx}}{\sqrt{2\pi}}]\dd{k}\\
&= \int_{-\infty}^{+\infty} \sqrt{\frac{\abs{k}}{m}}A(k) \psi_k(x) \dd{k}
\end{aligned}
\end{equation}
对比用动量归一化展开
\begin{equation}
\psi(x) = \int_{-\infty}^{+\infty} C(k)\psi_k(x) \dd{k}
\end{equation}
两种系数间满足关系
\begin{equation}
C(k) = \sqrt{\frac{\abs{k}}{m}}A(k)
\end{equation}

\subsection{其他正交归一本征态}
事实上归一化其实有无数种, 因为每个能量 $E$ 的本征波函数空间是一个二维简并空间, 我们可以把以上正交归一的 $\psi_{E,1}, \psi_{E,2}$ 做一个任意幺正变换后仍然得到两个正交归一的解. 最常见的方法例如令(角标 $o$ 代表 odd, $e$ 代表 even)
\begin{equation}
\begin{aligned}
\psi_{E,e}(x) &= \frac{1}{\sqrt{2}}[\psi_{E,1}(x) + \psi_{E,2}(x)] = \sqrt{\frac{m}{\pi k}}\cos(kx)\\
\psi_{E,o}(x) &= \frac{1}{\sqrt{2}\,\I}[\psi_{E,1}(x) - \psi_{E,2}(x)] = \sqrt{\frac{m}{\pi k}}\sin(kx)
\end{aligned}
\end{equation}
他们满足\autoref{EngNor_eq4} (习题). 这个变换使用了矩阵 $\pmat{1 & 1\\ -\I & \I}/\sqrt{2}$, 可以证明使用任何 2 乘 2 的酉矩阵\upref{UniMat}做变换都可以.

令 $k > 0$, 把 $\psi_{\pm k}$ 做类似的变换, 我们也可以写出动量归一化的一种变体\footnote{注意这样得到的并不是动量算符的本征函数, 所以严格来说动量归一化只有一种可能.}
\begin{equation}\label{EngNor_eq5}
\begin{aligned}
\psi_{k,e}(x) &= \frac{1}{\sqrt{2}}[\psi_{k}(x) + \psi_{-k}(x)] = \frac{1}{\sqrt{\pi}}\cos(kx)\\
\psi_{k,o}(x) &= \frac{1}{\sqrt{2}\,\I}[\psi_{k}(x) - \psi_{-k}(x)] = \frac{1}{\sqrt{\pi}}\sin(kx)
\end{aligned}
\qquad (k > 0)
\end{equation}
这就是三角傅里叶变换\upref{FTTri}中使用的基底.他们满足归一化条件(\autoref{Delta_exe2}~\upref{Delta})
\begin{equation}\label{EngNor_eq3}
\int_{-\infty}^{+\infty} \psi_{k',i}(x)^* \psi_{k,i}(x) \dd{x} = \delta(k' - k) \qquad (k > 0, i = e, o)
\end{equation}
