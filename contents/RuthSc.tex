% 卢瑟福散射
% 碰撞参量|双曲线|经典散射

\begin{issues}
\issueDraft
\end{issues}

\pentry{开普勒问题\upref{CelBd}, 散射\upref{Scater}}

定义\textbf{碰撞参量}为双曲线的渐近线到焦点的距离,若轻质点一直做匀速直线运动,则碰撞参量就是两质点的最近距离.由双曲线的性质,%链接未完成
碰撞参量等于双曲线的参数 $b$.

看做经典散射,微分截面等于
\begin{equation}
\dv{\sigma}{\Omega} = \frac{b \dd{b}\dd{\phi} }{\sin \theta \dd{\theta} \dd{\phi} } = \frac{b}{\sin \theta }\dv{b}{\theta}
\end{equation}
由双曲线性质,偏射角满足
\begin{equation}
\cot{\frac{\theta }{2}}= \frac{b}{a}
\end{equation}
由反开普勒问题  $E = kQq/(2a)$,消去 $a$ 得
\begin{equation}
b = \frac{kQq}{2E}\cot {\frac{\theta }{2}}
\end{equation}
求导代入微分截面得
\begin{equation}\label{RuthSc_eq1}
\dv{\sigma}{\Omega} = \qty[ \frac{kQq}{4E \sin[2](\theta /2)} ]^2
\end{equation}
