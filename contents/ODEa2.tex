% 一阶常微分方程解法:常数变易法
% keys 微分方程|differential equation|考研

\pentry{一阶常微分方程解法:变量可分离方程\upref{ODEa1}}

观察以下方程:

\begin{equation}\label{eq_ODEa2_1}
\frac{\dd y}{\dd x}=P(x)y+Q(x)~,
\end{equation}
其中 $P$ 和 $Q$ 都是所考虑区间上的连续函数。

这样的方程被称作“一阶线性方程”,其中如果 $Q(x)=0$,则还可以称之为“齐次”的方程,否则便是“非齐次”的。

容易看出,一阶齐次线性方程 $\frac{\dd y}{\dd x}=P(x)y$ 就是变量可分离方程,我们已经在预备知识\textbf{一阶常微分方程解法:变量可分离方程}\upref{ODEa1}中详细讨论过了。

\begin{exercise}{}\label{exe_ODEa2_1}
证明:$\frac{\dd y}{\dd x}=P(x)y$ 的通解是 $y=C\E^{\int P(x)\dd x}$。
\end{exercise}

\autoref{exe_ODEa2_1} 给出了齐次方程的解,其中含一个待定常数 $C$。非齐次方程是齐次方程的拓展,可以用常数变易法来解。

\textbf{常数变易法}就是将齐次方程解中的 $C$ 拓展为一个待定函数 $C(x)$,再代回非齐次方程,看看能不能解出这个 $C(x)$;如果能,那么非齐次方程也就得解了。

\subsubsection{常数变易法的推导}
考虑方程\autoref{eq_ODEa2_1} 。根据齐次方程的解\autoref{exe_ODEa2_1} ,假设\autoref{eq_ODEa2_1} 的解是 $y=C(x)\E^{\int P(x)\dd x}$。

则
\begin{equation}
\begin{aligned}
\frac{\dd y}{\dd x}=C'(x)\E^{\int P(x)\dd x}+C(x)P(x)\E^{\int P(x)\dd x}~.
\end{aligned}
\end{equation}

比较\autoref{eq_ODEa2_1} 和\autoref{eq_ODEa2_1} ,发现
\begin{equation}
Q(x)=C'(x)\E^{\int P(x)\dd x}~,
\end{equation}
或改写为
\begin{equation}
\frac{\dd C}{\dd x}=\frac{Q(x)}{\E^{\int P(x)\dd x}}~,
\end{equation}

这是一个变量可分离方程。移项、积分后,得到其通解
\begin{equation}
C(x)=\int\frac{Q(x)}{\E^{\int P(x)\dd x}}\dd x+K~,
\end{equation}
其中 $K$ 为积分常数。

因此,\autoref{eq_ODEa2_1} 的解就是
\begin{equation}\label{eq_ODEa2_2}
\begin{aligned}
y&=C(x)\E^{\int P(x)\dd x}\\
&=\qty(\int\frac{Q(x)}{\E^{\int P(x)\dd x}}\dd x+K)\E^{\int P(x)\dd x}~.
\end{aligned}
\end{equation}

\begin{example}{}
考虑方程
\begin{equation}
\frac{\dd y}{\dd x}=2xy+x~
\end{equation}
和\autoref{eq_ODEa2_1} 比较可得,$P(x)=2x$,$Q(x)=x$。

因此,根据\autoref{eq_ODEa2_2} ,该方程的通解为
\begin{equation}
\begin{aligned}
y&=\qty(\int\frac{Q(x)}{\E^{\int P(x)\dd x}}\dd x+K)\E^{\int P(x)\dd x}\\
&=\qty(\int\frac{x}{\E^{x^2}}\dd x+K)\E^{x^2}\\
&=\qty(-\frac{1}{2}\E^{-x^2}+K)\E^{x^2}\\
&=K\E^{x^2}-\frac{1}{2}~.
\end{aligned}
\end{equation}
\end{example}

\subsubsection{伯努利微分方程}
形如
\begin{equation}
\frac{\dd y}{\dd x}=P(x)y+Q(x)y^n~
\end{equation}
的方程,称为\textbf{伯努利微分方程(Bernoulli differential equation)}。它可以通过变量代换,化为一阶线性微分方程。



















