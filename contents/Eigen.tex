% Eigen (C++ 线性代数库)笔记

\subsection{Visual Studio中使用 Eigen}
Eigen 的\href{http://eigen.tuxfamily.org/index.php?title=Main_Page}{主页}.
\begin{enumerate}
\item 首先\href{http://eigen.tuxfamily.org/index.php?title=Main_Page#Documentation}{下载} Eigen, 不需要任何安装.
\item 在 Visual Studio 中创建新 Project, 在 Project 的属性中的 Include Directories 加入解压文件夹的目录.
\item 在使用 Eigen 的代码中加入 \verb|#include <Eigen/Dense>| 即可.
\end{enumerate}

\subsection{基础}
\begin{itemize}
\item 与 Matlab 的\href{https://eigen.tuxfamily.org/dox/AsciiQuickReference.txt}{语法对照表}.
\item 所有的 Eigen 名字都有 namespace Eigen
\end{itemize}

\begin{lstlisting}[language=cpp]
// Matrix Class
Matrix<double, 3, Dynamic> A
Matrix<double, Dynamic, Dynamic, Aligned | RowMajor> A
typedef Matrix<int, 4, 4> Matrix4i;
typedef Matrix<double, 4, 4> Matrix4d;
typedef Matrix<double, Dynamic, Dynamic> MatrixXd;
typedef Matrix<double, Dynamic, 1> VectorXd;
typedef Matrix<double, 1, Dynamic> RowVectorXd;

declaration & initialization
MatrixXd a(10,15); // No initialization
a.setRandom(m,n); 随机赋值
Matrix<>::setZero(); 把矩阵赋值为 0;
Matrix<>::setZero(m,n); 把矩阵赋值为 m*n 的 0 矩阵;
\end{lstlisting}

以下函数用法类似
\verb|setOnes(); setConstant(); setIdentity(); setRandom(); setLinSpaced();|
注意其中 \verb|setConstant()| 输入 1 个变量时与 \verb|fill()| 功能相同, 输入 3 个变量时最后一个为常数. \verb|setLinSpaced(size, val1, val2)| 只能对 Vector 或者 RowVector 使用. 如果想赋值给 \verb|MatrixXd|, 可以用 \verb|MatrixXd a = VectorXd::LinSpaced(3, 3, 4)|; 对于整型, \verb|setLinSpaced| 不保证最后一个值等于 val2, 而是保证间隔相等. 这时候只能先生成 Xd, 然后 .cast<int>.
* 逗号赋值都是 RowMaor 的, 无论 a 是什么 major.
a << 1,2,3,4;

\subsection{operations}
获取信息
\begin{itemize}
\item \verb|Matrix<>::size()| 相当于 Matlab 的 numel(), 另外, rows() 和 cols() 分别是行数和列数.
\item \verb|Matrix<>::data()| 可以获得 Matrix 数据的指针, 用于直接读写矩阵数据. 注意 rowwise 需要专门声明. 也可以用 \verb|&| 来获取矩阵元的地址.
\item 矩阵转置用 \verb|Matrix<>::transpose()| 复数矩阵共轭用 conjugate(), 共轭转置用 adjoint()
\item 矩阵元求和, \verb|Matrix<>::sum(); Matrix<>::colwise().sum(); Matrix<>::rowwise().sum()|
\item 点乘和叉乘如 \verb|v.dot(w), v.cross(w)|
类似的有 \verb|mean(), prod(), all(), any(), maxCoeff(), minCoeff()|
获取子矩阵
对 Vector, 有 \verb|Matrix<>::head(n), tail(n), segment(n)|. 对 Matrix, 有 \verb|block(i,j,rows,cols)|, 有 \verb|row(i), col(j), topRows(n), middleRows(i, rows), bottomRows(n), leftCols(n), middleCols(j, cols), rightCols(n), topLeftCorner(rows,cols), bottomRightCorner(rows,cols)| 等.
\end{itemize}


\subsubsection{逐个矩阵元的运算}
\begin{itemize}
\item \verb|Matrix<>::cwiseAbs()|
\item 类似的有 \verb|cwiseInverse(); cwiseMax(); cwiseMin(); cwiseSign(); cwiseSqrt(); Matrix<>::array()| 用于把 Matrix 转换为 array, 逐个元素运算, 或加减一个常数. 如 \verb|Matrix<>::array().square()|
\item 类似的有 \verb|round(); pow(); sqrt();| 注意 array 可以直接赋值给 matrix.
\item 复制 vector 给矩阵的每一列 \verb|mat.colwise() = v;| 类似地, \verb|+=, -=| 等也可以使用.
\end{itemize}


\subsubsection{Aliasing}
如果矩阵同时出现在等号两边, 就有可能出现 Aliasing. 例如 \verb|MatrixXi mat(3,3);  mat.bottomRightCorner(2,2) = mat.topLeftCorner(2,2);| 这时, 要在等号右边的加上 \verb|.eval()|. 特殊地, 对于转置等, 可以直接用 \verb|mat = mat.transposeInPlace()|. 类似地有 \verb|adjointInPlace()|.

\subsubsection{SVD (支持 complex)}
要用 SVD, 要 \verb|#include<Eigen/SVD>|, 例程:
\begin{lstlisting}[language=cpp]
MatrixXd  A(3,3), U, V, X;
A << 1, 2, 3, 4, 5, 6, 7, 8, 9;
BDCSVD<MatrixXd> svd(A, ComputeThinU | ComputeThinV);
U = svd.matrixU(); V = svd.matrixV(); X = svd.singularValues();
\end{lstlisting}

\subsection{外部接口}
这里介绍如何直接操作 \verb|Matrix<>| 底层的数据, 以及如何将内存中已有的数据直接给 Eigen 使用而无需复制.

\subsubsection{访问底层数据}
\begin{itemize}
\item \verb|Matrix<>::data()| 可以返回第一个矩阵元的指针. 然后就可以当做矩阵来用了.
\item 查看了一下源码, \verb|Matrix<>::innerStride()| 一律返回 1, \verb|outerStride()| 返回 \verb|innerSize()|, 而 \verb|innerSize| 对于 c-major 是 \verb|rows()|, 对于 r-major 是 \verb|cols()|. 这就说明矩阵的内存显然是连续的(无论怎么优化), 可以放心使用.
\item 要注意 \verb|Matrix<> |到底是 col-major 还是 row-major, 默认是 col-major. 不推荐用 \verb|#define EIGEN_DEFAULT_TO_ROW_MAJOR| 将默认改为 row-major. 建议用例如 \verb`typedef Matrix<double, Dynamic, Dynamic, Aligned | RowMajor> RMatrixXd.`
\end{itemize}

\subsubsection{使用内存数据}
\begin{itemize}
\item 创建 \verb|Map<>| 类即可, 所有接受 \verb|Matrix<>| 的函数同样也接受 \verb|Map<>| (除了自己写的函数)
\item \verb|Map<>| 有三个模板参数, 但是只需要用一个, 即 \verb|Map<Matrix<>>|.
\item 创建 Map object 如 \verb|Map<MatrixXf> a(duuble* pa, rows,cols);|
\end{itemize}
