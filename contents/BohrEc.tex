% Bohr-Sommerfeld 原子模型
% keys 玻尔原子模型|半经典|量子|角动量|量子化
% license Xiao
% type Tutor

\pentry{玻尔原子模型\nref{nod_BohrMd}, 环积分\nref{nod_IntL}}{nod_6dcd}

\footnote{参考 Wikipedia \href{https://en.wikipedia.org/wiki/Old_quantum_theory}{相关页面}
。}本文使用原子单位制\upref{AU}。 Bohr-Sommerfeld 模型(以下简称 Sommerfeld 模型)和玻尔原子模型一样属于量子力学发展早期的半经典原子模型, 它对玻尔模型进行了改进, 能更好地符合一些实验结果(如塞曼效应\upref{ZemEff})。 玻尔模型假设电子以圆形轨道绕原子核旋转, 而 Sommerfeld 模型使用椭圆轨道。 为了简单我们同样先假设原子核固定不动, 要考虑原子核运动使用相对坐标和约化质量\upref{TwoBD}即可。

Sommerfeld 模型中, 轨道量子化的条件有两个
\begin{equation}\label{eq_BohrEc_4}
L = l~,
\end{equation}
\begin{equation}\label{eq_BohrEc_3}
\oint m\dot r \dd{r} = kh~.
\end{equation}
其中 $L$ 是轨道角动量的模长, $l, k$ 是正整数, $h$ 是普朗克常数(原子单位制下等于 $2\pi$), $r$ 是电子到原子核的距离, $\dot r$ 表示 $r$ 的时间导数, $\oint$ 表示延椭圆轨道的环积分\upref{IntL}。

根据这两个量子化条件, 可以由量子数 $k, l$ 唯一地确定椭圆轨道的形状和大小, 可以证明轨道总能量为
\begin{equation}\label{eq_BohrEc_5}
E = -\frac{Z^2}{2n^2}~.
\end{equation}
其中
\begin{equation}\label{eq_BohrEc_2}
n = l + k~.
\end{equation}
注意由于 $k$ 是正整数, $l$ 必须满足 $0 < l < n$。

\subsection{数值验证}

下面来验证\autoref{eq_BohrEc_5}, \autoref{eq_BohrEc_2} 符合量子化条件\autoref{eq_BohrEc_3}。 这是一个中心力场问题, 把\autoref{eq_CenFrc_10}~\upref{CenFrc} 和\autoref{eq_BohrEc_4}  代入\autoref{eq_BohrEc_3} 得
\begin{equation}\label{eq_BohrEc_1}
2\int_{a-c}^{a+c} \sqrt{2m\qty(E + \frac{Z}{r} - \frac{l^2}{2mr^2})} \dd{r} = kh~.
\end{equation}
注意\autoref{eq_BohrEc_3} 中的环积分可以被替换为两个相等的积分, 即从椭圆轨道的近日点 $a-c$ 积分到远日点 $a+c$(详见 “椭圆的\upref{Elips3}”), 在从远日点积分到近日点\footnote{“近日点” 和 “远日点” 是开普勒问题中的习惯叫法, 无论中心是天体还是原子核。}。 在开普勒问题中, 椭圆的离心率 $e$ 和参数 $a, c$ 可以用能量和角动量表示(\autoref{eq_CelBd_7}~\upref{CelBd},\autoref{eq_CelBd_8}~\upref{CelBd})
\begin{equation}\label{eq_BohrEc_6}
e = \frac{c}{a} = \sqrt{1 + \frac{2EL^2}{mZ^2}}~,
\qquad
a = \frac{Z}{2\abs{E}}~.
\end{equation}
把\autoref{eq_BohrEc_5}, \autoref{eq_BohrEc_2} 和\autoref{eq_BohrEc_6} 代入\autoref{eq_BohrEc_1}, 若等式两边对所有的 $l,k > 0$ 成立则说明满足量子化条件。 由于该积分较为复杂, 我们姑且用 Matlab 进行数值积分(连接未完成)。 注意原子单位中 $h = 2\pi$。
\begin{lstlisting}[language=matlab]
l = 2; k = 1; n = l + k;
Z = 1; h = 2*pi;
E = -Z^2/(2*n^2);
me = 1;
a = abs(k)/(2*abs(E)); b = l/sqrt(2*me*abs(E));
c = sqrt(a^2 - b^2);
f = @(r)real(sqrt(2*me*(E + Z./r - l^2./(2*me*r.^2))));
r = linspace(a-c, a+c, 500);
figure; plot(r, f(r));
I = 2*integral(f, a - c, a + c);
rel_err = (I - h*k)/(h*k);
disp('relative error =');
disp(rel_err);
\end{lstlisting}
运行结果: 
\begin{lstlisting}[language=matlabC]
relative error =
  1.4136e-16
\end{lstlisting}
读者也可以把 $k, l$ 替换成其他正整数进行验证。
