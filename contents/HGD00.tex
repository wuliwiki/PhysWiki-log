% 哈尔滨工业大学 2000 年硕士物理考试试题
% keys 哈尔滨工业大学|考研|物理|2000年
% license Copy
% type Tutor

\textbf{声明}:“该内容来源于网络公开资料,不保证真实性,如有侵权请联系管理员”

\begin{enumerate}
\item 设光纤芯线与外套的折射率$n_g>n_c$,垂直端面外介质的折射率为$n_a$(如图1),试求能使光在光纤内发生全反射的入射光束的最大孔径角$i_0$。若将光纤制成圆锥腔,试问有否“聚光”作用,并说明理由。
\begin{figure}[ht]
\centering
\includegraphics[width=8cm]{./figures/5885197456b5d97b.png}
\caption{} \label{fig_HGD00_1}
\end{figure}
\item 如图二所示望远镜,物镜$L_1$的焦距$f_1$,镜框内径$D_1$;目镜$L_2$的焦距$f_2$,镜框内径$D_2$。在重合焦平面上的光阑$A$的孔径为$d$。试求该望远镜的孔径光阑、入射瞳、出射瞳和视场光阑的位置和大小。又若当你用此望远镜观测时上述量有何不同。
\begin{figure}[ht]
\centering
\includegraphics[width=8cm]{./figures/80fcbd27be813432.png}
\caption{} \label{fig_HGD00_2}
\end{figure}
\item 将焦距为的柱透镜沿轴线对半剖开,分成两片半透镜$L_A$和$L_B$,按图三位置安放。$P$点为波长$\lambda$的单色线光源。在两束光交叠区域放置观察屏$Q$。\\
(1)写出明暗条纹的条件;\\
(2)指出中央条纹的位置及明或暗;\\
(3)讨论若屏$Q$向右移动,条纹为何变化。
\begin{figure}[ht]
\centering
\includegraphics[width=8cm]{./figures/7984dafcd5e05874.png}
\caption{} \label{fig_HGD00_3}
\end{figure}
\item 一凹面镜上放一平晶,为图4所示,以单色光垂直照射观察干涉现象。当波长$\lambda_1=500nm$时,中心处为暗纹,连续改变波长直至$\lambda_2=600nm$时,中心处才又变为暗纹。\\
(1)请定性描述零级条纹的位置及明暗、条纹宽窄、条纹间距和条纹形状;\\
(2)计算空气膜的最大厚度。
\begin{figure}[ht]
\centering
\includegraphics[width=8cm]{./figures/35b0881fab605ec9.png}
\caption{} \label{fig_HGD00_4}
\end{figure}
\item 一平面透射光栅置于两介质之界面上,光栅常数为$d$。两介质的折射率$n_1<n_2$。入射光在$n_!$介质中波长为$\lambda$,入射角为$\theta$,照亮光栅的$N$个缝。试求\\
(1)光栅零级条纹的衍射角$\varphi$_0;\\
(2)该零级条纹的半角宽度。
\item 两正交尼科耳棱镜之间插一方解石晶片,它的光轴与表面平行,并与尼科耳棱镜的主截面成45°角。设光通过第一个尼科耳棱镜后的振幅为1,通过晶片时引起o光e光的位相差$\Delta \varphi$。试求\\
(1)通过晶片后o光和e光的振幅和光强;\\
(2)通过第二个尼科耳棱镜后光的振幅和光强。
\item 一平薄圆环内外半径为$R_1$、$R_2$,若以圆环中心为原点,在圆环平面上$r$为矢径。面电荷密度$\sigma_e=\frac{\sigma}{r}$,其中$R_1\le r \le R_2$,常量$\sigma>0$。试求其轴线上距原点$x$处的电场强度。
\item 上题中,若圆环绕轴线以匀角速ω转动,试求轴线上距原点$x$处的磁感应强度。
\end{enumerate}