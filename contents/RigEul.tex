% 刚体定点旋转的运动方程(欧拉角)

\begin{issues}
\issueDraft
\end{issues}

\pentry{刚体的运动方程\upref{RBEqM}, 欧拉角\upref{EulerA}}

首先摆放刚体, 使得三个主轴分别与 $x,y,z$ 轴重合。 然后使用 $z$-$y$-$z$ 欧拉角描述刚体每个时刻的位置。 若已知角速度, 欧拉角的时间导数为(\autoref{EulerA_eq2}~\upref{EulerA})
\begin{equation}\label{RigEul_eq1}
\dot\psi = \omega_r + \omega_\theta \cot\theta,\qquad
\dot\theta = \omega_\phi,\qquad
\dot\phi = -\frac{\omega_\theta}{\sin\theta}
\end{equation}
球坐标中三个方向构建的坐标系与刚体三个主轴 $\uvec e_1, \uvec e_2, \uvec e_3$ ($\uvec e_1 = \uvec r$)所构建的体坐标系之间还需要经过绕 $\uvec r$ 轴的 $\psi$ 角旋转变换。
\begin{equation}\label{RigEul_eq3}
\pmat{\omega_r\\\omega_\theta\\\omega_\phi} = \pmat{1 & 0 & 0\\ 0 & C_\psi & -S_\psi\\ 0 & S_\psi & C_\psi}\pmat{\omega_{01}\\\omega_{02}\\\omega_{03}}
\end{equation}
在体坐标系中有欧拉方程(\autoref{RBEqM_eq9}~\upref{RBEqM})
\begin{equation}\label{RigEul_eq2}
\dot{\bvec \omega_0} = \mat I_0^{-1} (\bvec \tau  - \bvec\omega_0\cross \mat I_0 \bvec\omega)
\end{equation}
分量形式
\begin{equation}
\leftgroup{
&\dot\omega_{01} = [\tau_{01} - (I_3-I_2)\omega_{03}\omega_{02}]/I_{1}\\
&\dot\omega_{02} = [\tau_{02} - (I_1-I_3)\omega_{01}\omega_{03}]/I_{2}\\
&\dot\omega_{03} = [\tau_{03} - (I_2-I_1)\omega_{01}\omega_{02}]/I_{3}
}\end{equation}
特殊地, 若两个主转动惯量\upref{PrncAx}相等 $I_2 = I_3$, 那么则无需进行\autoref{RigEul_eq3} 的变换, 直接在 $\uvec r, \uvec \theta, \uvec\phi$ 系中列出\autoref{RigEul_eq2} 即可。

现在, \autoref{RigEul_eq1} 和\autoref{RigEul_eq2} 就组成了一个六元一阶常微分方程组。 另外体坐标系到实验室坐标系之间的变换矩阵为(\autoref{EulerA_eq3}~\upref{EulerA})
\begin{equation}
\mat R = \pmat{
C_\phi C_\theta C_\psi-S_\phi S_\psi & -C_\phi C_\theta S_\psi - S_\phi C_\psi & C_\phi S_\theta\\
S_\phi C_\theta C_\psi + C_\phi S_\psi & -S_\phi C_\theta S_\psi + C_\phi C_\psi & S_\phi S_\theta\\
-S_\theta C_\psi & S_\theta S_\psi & C_\theta}
\end{equation}
