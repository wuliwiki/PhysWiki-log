% 矢量场(矢量分析)
% keys 多元微积分|球坐标系|矢量|矢量场|线性组合
% license Xiao
% type Tutor

\pentry{球坐标系的定义\nref{nod_Sph}, 矢量的求导法则\nref{nod_DerV}, 几何矢量的内积\nref{nod_Dot}, 几何矢量的基底和坐标\nref{nod_Gvec2}}{nod_32e8}

对空间中指定范围的每一点 $P$ 赋予一个矢量 $\bvec v$, 就在该空间中形成了一个\textbf{矢量场}。例如,电荷附近的任意一点都存在一个电场矢量,这就构成了一个矢量场。 管道中任意一点的水流都存在一个速度矢量,它们也构成一个矢量场。其他例子如力场\upref{V}, 电场\upref{Efield}, 磁场\upref{MagneF}。

\subsubsection{直角坐标系}
矢量场在不同的参考系中有不同的表示方法。在空间直角坐标系中,矢量场可以用矢量的三个分量关于 $x,y,z$ 三个坐标的函数表示。点 $P(x,y,z)$ 处的矢量分量为($\vdot$ 表示\enref{点乘}{Dot})
\begin{equation}
\begin{cases}
v_x(x,y,z) = \bvec v \vdot \uvec x\\
v_y(x,y,z) = \bvec v \vdot \uvec y\\
v_z(x,y,z) = \bvec v \vdot \uvec z
\end{cases}~,
\end{equation}
也可以作为单位正交基\upref{OrNrB} 的线性组合写成一个整体
\begin{equation}
\ali{
\bvec v &= (\bvec v \vdot \uvec x)\,\uvec x + (\bvec v \vdot \uvec y)\,\uvec y + (\bvec v \vdot \uvec z)\,\uvec z\\
&= v_x(x,y,z)\,\uvec x + v_y(x,y,z)\,\uvec y + v_z(x,y,z)\,\uvec z~.
}\end{equation}

\subsubsection{球坐标系}
在球坐标系\upref{Sph}中,也可以把每个点的矢量根据该点处的三个单位矢量 $\uvec r$,  $\uvec \theta$,  $\uvec \phi$ 分解为三个分量。 基底的线性组合为
\begin{equation}
\bvec v = v_r(r,\theta ,\phi)\,\uvec r + v_\theta(r,\theta ,\phi) \,\uvec\theta  + v_\phi(r,\theta ,\phi)\,\uvec \phi ~. 
\end{equation} 

需要特别注意,$\uvec r$,  $\uvec \theta$,  $\uvec \phi$ 也是关于 $(r,\theta ,\phi )$ 的函数,所以对 $\bvec v$ 求导(或偏导)时必须根据矢量的求导法则\upref{DerV} 进行。

\subsection{场线}
场线是矢量场的一种可视化工具。 我们可以顺着矢量场的方向画出多条有方向的不相交曲线, 使得曲线上任意一点的切线方向都等于该点处矢量场的方向。 对于无散场, 场线没有起点和终点; 对于无旋场, 场线不会闭合。
\addTODO{链接,说明}
