% Volkov 波函数
% keys 波函数|本征波函数|偶极子近似|速度规范|库仑势
% license Xiao
% type Tutor

\pentry{速度规范\nref{nod_LVgaug}, 加速度规范\nref{nod_AccGau}}{nod_f8ad}

本文使用原子单位制\upref{AU}。 若初始时刻 $t_0$ 一个自由粒子的波函数是平面波, 波数为 $\bvec k_0$。 接下来空间中出现了随时间变化的电场 $\bvec {\mathcal E}(t)$ (我们采用偶极子近似\upref{DipApr}, 即假设电场不随位置变化, 感生磁场可以忽略), 那么我们把该电场中的含时波函数称为 \textbf{Volkov 波函数}。

以下我们分别在长度规范\upref{LenGau}、速度规范\upref{LVgaug}、加速度规范\upref{AccGau}中求解 Volkov 波函数。 在 $t \le t_0$ 时, 令 $\bvec {\mathcal E}(t) = \bvec A(t) = \bvec 0$。 令三种规范下 $t \le t_0$ 的波函数都是平面波(不同规范仅当非零电场出现后才有区别)
\begin{equation}\label{eq_Volkov_3}
\Psi_{\bvec k_0}(\bvec r, t \le t_0) = (2\pi)^{-3/2} \exp\qty(\I\bvec k_0 \vdot \bvec r - \I E_0 t)~,
\end{equation}
其中 $E_0 = \bvec k_0^2/(2m)$ 是初始动能(下同)。 注意以下所有规范中, 算符 $\bvec p = -\I \grad$ 都是广义动量\upref{QMEM}。

\subsection{加速度规范}
求解 Volkov 波函数最容易的方法就是使用所谓加速度规范, 由于空间中没有净电荷, 薛定谔方程(\autoref{eq_AccGau_4}~\upref{AccGau})中 $V(\bvec r) = 0$
\begin{equation}
\frac{\bvec p^2}{2m}\Psi_{\bvec k_0}^A = \I \pdv{t} \Psi_{\bvec k_0}^A~.
\end{equation}
显然\autoref{eq_Volkov_3} 就是该方程在 $t > t_0$ 的解
\begin{equation}\label{eq_Volkov_2}
\Psi_{\bvec k_0}^A(\bvec r, t) = (2\pi)^{-3/2} \exp\qty(\I\bvec k_0 \vdot \bvec r - \I E_0 t)~.
\end{equation}
可见在 K-H 参考系中, 波函数始终保持平面波的形式。

\begin{example}{高斯波包与电磁场}
在 K-H 参考系中, 若使用偶极子近似且令势能函数 $V(\bvec r) = 0$, 我们会发现高斯始终是高斯波包。 电磁波消失以后, K-H 系变为原来的惯性系, 这是因为电场不能含有直流分量。 (未完成:讲详细点?)
\end{example}

\subsection{速度规范}
使用\autoref{eq_AccGau_1}~\upref{AccGau}对\autoref{eq_Volkov_2} 做规范变换, 得速度规范下的 Volkov 波函数为
\begin{equation}\label{eq_Volkov_1}
% 已代入薛定谔方程验证
\Psi_{\bvec k_0}^V(\bvec r, t) = (2\pi)^{-3/2} \exp\qty{\I \bvec k_0 \vdot [\bvec r - \bvec \alpha(t)] - \I E_0 t}~,
\end{equation}
其中 $\bvec \alpha(t)$ 对应的是一个经典点电荷在电场中的位移(\autoref{eq_AccGau_3}~\upref{AccGau})
\begin{equation}
\bvec \alpha(t) = -\frac{q}{m} \int_{t_0}^t \bvec A(t') \dd{t'}~.
\end{equation}
该波函数空间频率不发生改变而只是以经典粒子的方式进行平移。 这是因为速度规范中,由外电场产生的粒子速度变化不会体现在波函数中。

容易验证\autoref{eq_Volkov_1} 是速度规范薛定谔方程(见\autoref{eq_LVgaug_7}~\upref{LVgaug})的解
\begin{equation}
\I \pdv{t} \Psi^V = \qty[\frac{\bvec p^2}{2m} - \frac{q}{m}\bvec A(t) \vdot \bvec p] \Psi^V~.
\end{equation}
注意 $E_0$ 只是初始时间的动能 $E(t_0)$, 电场中的粒子能量不守恒。 任意时刻, 波函数都是 $m\bvec v(t)$ 和动能 $E(t)$ 的本征矢, 本征值和经典粒子的动量动能相同。
\begin{equation}\label{eq_Volkov_5}
m\bvec v(t) = \bvec k_0 - q\bvec A(t)~,
\end{equation}
\begin{equation}
E(t) = \frac{\qty[\bvec k_0 - q\bvec A(t)]^2}{2m}~.
\end{equation}
初始时刻有 $\bvec A(0) = \bvec 0$, $m\bvec v(t_0) = \bvec k_0$。

\subsection{长度规范}
要求长度规范长度规范下的 Volkov 波函数只需要对\autoref{eq_Volkov_1} 再次做规范变换即可(\autoref{eq_LVgaug_6}~\upref{LVgaug}), 得\footnote{如果不在乎波函数的全局相位, 可以把\autoref{eq_Volkov_4} 方括号最后的常数项去掉。}
\begin{equation}\label{eq_Volkov_4}
% 已代入薛定谔方程验证
\Psi_{\bvec k_0}^L(\bvec r, t) = (2\pi)^{-3/2} \exp\qty[\I m\bvec v(t)\vdot \bvec r - \I\int_{t_0}^t E(t') \dd{t'} - \I E t_0]~.
\end{equation}
注意本文中的 $\bvec A(t)$ 都是指速度规范中的矢势, 长度规范下的矢势恒为零(\autoref{eq_LenGau_4}~\upref{LenGau}),广义动量就是 $m\bvec v$(\autoref{eq_LenGau_6}~\upref{LenGau})。

容易验证\autoref{eq_Volkov_4} 是长度规范薛定谔方程的解(\autoref{eq_LenGau_3}~\upref{LenGau})
\begin{equation}
\qty[\frac{\bvec p^2}{2m} - q \bvec {\mathcal E}(t) \vdot \bvec r] \Psi^L = \I \pdv{t} \Psi^L~.
\end{equation}
