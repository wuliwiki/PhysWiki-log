% 个体词、谓词与量词(数理逻辑)
% keys 个体词|谓词
% license Usr
% type Tutor

\pentry{原子命题,量词的连接词\nref{nod_propco}}{nod_d33c}

\subsection{定义}
\begin{definition}{个体词}
在原子命题中,可以独立存在的客体(句子中的主语、宾语等),称为个体词,可以是\textbf{个体常量}或\textbf{个体变量}。
\begin{itemize}
\item \textbf{个体常量}表示特定的或具体的个体,是不变的,一般用带或不带下标的小写字母 $a, b, c, \dots, a_1, a_2, \dots$ 表示。
\item \textbf{个体变量}表示泛指的或抽象的、不定的个体,是变化的,一般用带或不带下标的小写字母 $x, y, z, \dots, x_1, x_2, \dots$ 表示。
\end{itemize}

\end{definition}
\begin{definition}{论域}
所有个体词的取值范围称为\textbf{论域}(或\textbf{取值域}),常用字母 $D$ 表示。
\end{definition}


\begin{definition}{谓词}
用以刻画客体的属性或心智(类似于一元函数),或多个客体之间的关系(类似于多元函数)的词称为\textbf{谓词}。
\end{definition}

\begin{definition}{量词}
表达个体词的数量或数量关系的称为量词。
\end{definition}

\begin{definition}{全称量词}
$\forall x$ 表示\textbf{全称量词},即“任意 $x$”、“每个 $x$”、“一切的 $x$”......

符号化时,全称量词 $\forall$ 一般都接 $\to$ 连接词。
\end{definition}
\begin{definition}{存在量词}
$\exists x$ 表示\textbf{存在量词},即“存在 $x$”、“至少有一个 $x$”......

符号化时,存在量词 $\exists$ 一般都接 $\land$ (合取、并且)连接词。
\end{definition}

\begin{definition}{作用变量与辖域}
对于全称量词 $\forall x$ 与存在量词 $\exists x$,对应的符号 $\forall$ 与 $\exists$ 紧接一个变量 $x$,这变量就称作\textbf{作用变量}。

一般将量词加到其谓词之前,描述这个句子或句子的一部分的个体词的数量或数量关系,此时就称这个部分的谓词是这量词的\textbf{辖域}。例如 $(\forall x) F(x)$,则 $F(x)$ 就是这个全称量词的辖域。
\end{definition}

\begin{definition}{自由变元与约束变元}
有量词约束而在对应量词的辖域范围内的称为\textbf{约束变元}。

无量词约束的称为\textbf{自由变元}。
\end{definition}


\subsection{例子}
个体词、谓词与量词的定义比较抽象,下面借助几个例子来理解。

\begin{example}{谓词的例子}
用 $P(x)$ 表示 $x$ 是北方人、$Q(x)$ 表示 $x$ 怕冷,$c$ 表示李华,符号化下面这个句子:

除非李华是北方人,否则李华一定怕冷。
\end{example}
首先表示为命题逻辑的形式,先用 $p$ 表示“李华怕冷”,用 $q$ 表示“李华是北方人”,原来的句子就可以化作
\begin{equation}
\neg p \to  q ~.
\end{equation}
显然 $p$ 就是 $Q(c)$,而 $q$ 就是 $P(c)$,所以这句子就可以化为
\begin{equation}
\neg Q(c) \to  P(c) ~.
\end{equation}

\begin{example}{全称量词的例子}
符号化下面这个句子:

所有人的头发都是黑色的。
\end{example}
显然这是一个全称量词的句子,我们用 $P(x)$ 表示 $x$ 是人、$Q(x)$ 表示 $x$ 长有黑头发,则原句可以表示为:
\begin{equation}
(\forall x)(P(x) \to Q(x)) ~.
\end{equation}

\begin{example}{存在量词的例子}
符号化:有的人登上过月球。
\end{example}
这里“有的”就是一个存在量词的例子。我们仍用 $P(x)$ 表示 $x$ 是人、$Q(x)$ 表示 $x$ 上过月球,则原句可以表示为:
\begin{equation}
(\exists x)(P(x) \land Q(x)) ~.
\end{equation}

\begin{example}{}
符号化:所有人都不喜欢杂草。
\end{example}
仍用 $P(x)$ 表示 $x$ 是人、$Q(y)$ 表示 $y$ 是杂草、$L(x, y)$ 表示 $x$ 喜欢 $y$,则原句可以表示为:
\begin{equation}
\forall x(P(x) \to \forall y(Q(y) \to \neg L(x, y))) ~,
\end{equation}
或
\begin{equation}
\forall x \forall y (P(x) \land Q(y) \to \neg L(x, y)) ~.
\end{equation}





