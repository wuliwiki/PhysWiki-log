% 理想气体(微正则系综法)
% keys 理想气体|微正则系综|相空间|熵

\begin{issues}
\issueDraft
\end{issues}

\pentry{理想气体的熵:纯微观分析\upref{IdeaS},热力学关系式\upref{MWRel}}

在微正则系综中,计算体系的熵的方式是,计算能壳 $E\sim E+\Delta E$ 内系统能级的个数,也就是微观状态数,再利用玻尔兹曼公式 $S=k\ln \Omega$ 求解体系的熵。可以将微观状态数除以 $\Delta E$ 来忽略能壳厚度对计算结果的影响。经过一系列计算\footnote{具体计算过程可以参考 \upref{IdeaS}。},能量为 $E$,粒子数为 $N$,体积为 $V$ 的理想气体的熵的公式(\autoref{eq_IdeaS_2}~\upref{IdeaS})为
\begin{equation}\label{eq_IdNCE_1}
S(E, V, N) = k\ln \Omega  = Nk \qty(\ln \frac{V}{N\lambda^3} + \frac52)~,
\end{equation}
该式被称为 \textbf{Sackur-Tetrode 公式}。 其中
\begin{equation}\label{eq_IdNCE_2}
\lambda = \frac{h}{\sqrt{4\pi mE/(3N)}} = \frac{h}{\sqrt{2\pi mkT}}~,
\end{equation}

\subsection{理想气体的热力学量}
微正则系综的精神是,从熵公式出发,利用熵的微分关系得到各个热力学量——温度、压强、化学势,从而可以利用麦克斯韦关系计算其他的热力学势,这样理论所就能推出平衡态热力学系统的一切信息。

根据熵的微分关系
% 链接未完成
\begin{equation}
\dd{S} = \frac{1}{T} \dd{E} + \frac{P}{T} \dd{V} - \frac{\mu}{T} \dd{N}~.
\end{equation}
所以对 $S$ 求三个偏导得
\begin{equation}
T = \frac{2E}{3Nk}~,
\end{equation}
\begin{equation}
P = NkT/V~,
\end{equation}
\begin{equation}
\mu = kT \ln \frac{N\lambda^3}{V}~.
\end{equation}
