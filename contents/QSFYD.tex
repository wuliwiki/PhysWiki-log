% 轻水反应堆
% license CCBYSA3
% type Wiki

(本文根据 CC-BY-SA 协议转载自原搜狗科学百科对英文维基百科的翻译)

\begin{figure}[ht]
\centering
\includegraphics[width=6cm]{./figures/a03f6ef172639559.png}
\caption{一个简单的轻水反应堆} \label{fig_QSFYD_7}
\end{figure}

\textbf{轻水反应堆(LWR)}是一种热中子反应堆,它使用普通的水,而不是重水,作为它的冷却剂和中子慢化剂——此外,一种固体形式的裂变元素被用作燃料。热中子反应堆是最常见的核反应堆,轻水反应堆则是最常见的热中子反应堆。

轻水反应堆有三种类型:压水堆(PWR)、沸水堆(BWR)和超临界水堆 (SCWR)。

\subsection{历史}
\subsubsection{1.1 早期概念和实验}
在发现了裂变、慢化以及核链式反应的理论可能性后,早期的实验结果迅速表明,天然铀只能在石墨或重水作为慢化剂的情况下发生持续的链式反应。而世界上第一批反应堆( CP-1 、 X10 等)成功达到临界状态后,铀浓缩开始从理论概念发展到实际应用,以满足曼哈顿计划的目标,即制造核爆炸。

1944年5月,在洛斯阿拉莫斯的低功率(LOPO) 反应堆中,第一批生产出来的浓缩铀达到了临界质量,该反应堆被用来估计制造原子弹的U235的临界质量。[1]LOPO不能被认为是第一座轻水反应堆,因为它的燃料不是固体铀化物,且包裹着耐腐蚀材料中,而是由溶解在水中的硫酸铀酰盐组成。[2]但它是第一个含水均相反应堆,也是第一个以浓缩铀为燃料,以普通水为慢化剂的反应堆。[1]

第二次世界大战末期,根据阿尔文·温伯格的想法,天然铀燃料元素被安置在X10反应堆顶部普通水中的晶格中,以评估中子倍增系数。[3]本实验的目的是确定一个以轻水为慢化剂和冷却剂,并以固体铀为燃料的核反应堆的可行性。结果表明,用低浓缩铀可以达到临界状态。[4]这个实验是迈向轻水反应堆的第一步。

第二次世界大战后,随着浓缩铀的出现,新的反应堆概念变得可行。1946年,尤金·维格纳(Eugene Wigner)和阿尔文·温伯格(Alvin Weinberg)提出并发展了使用浓缩铀作为燃料,并以轻水作为慢化剂和冷却剂的反应堆概念。[3]这个概念是为一个反应堆提出的,其目的是测试材料在中子通量下的行为。这个反应堆——材料测试反应堆(MTR) 建于爱达荷州的 INL ,于1952年3月31日达到临界状态。[5]对于该反应堆的设计,实验是必要的,因此在 ORNL 建造了实物模型(MTR),以评估一回路的液压性能,然后测试其中子学特性。这个MTR模型,后来被称为低强度试验反应堆(LITR),于1950年2月4日达到临界状态,[6]是世界上第一个轻水反应堆。[7]

\subsubsection{1.2 第一座压水堆}
第二次世界大战结束后,美国海军在首长海曼·里科弗(后来的海军上将)的指导下,开始了一项计划,目标是为船舶的提供核推进。它在20世纪50年代早期开发了第一个压水堆,并成功部署了第一艘核潜艇“鹦鹉螺号”(USS Nautilus (SSN-571)。

苏联在20世纪50年代末独立研制了一种名为VVER的压水堆反应堆。虽然在功能上与美国的设计非常相似,但它与西方压水堆也有一定的设计区别。

\subsubsection{1.3 第一座沸水堆}
研究员塞缪尔·恩特尔梅尔二世 (Samuel Untermyer II)在美国国家反应堆试验站(现在的爱达荷国家实验室Idaho National Laboratory )领导了一项名为硼砂实验的系列试验,以开发沸水堆(BWR)。

\subsection{概论}
\begin{figure}[ht]
\centering
\includegraphics[width=8cm]{./figures/ede476a7a9f25af4.png}
\caption{Koeberg 核电站由两个以铀为燃料的压水堆组成} \label{fig_QSFYD_1}
\end{figure}
被称为轻水反应堆(LWR)的核反应堆系列,使用普通水进行冷却和慢化,建造起来往往比其他类型的核反应堆更简单、更便宜;由于这些因素,截至2009年,它们在全世界服役的民用核反应堆和海军推进反应堆中占绝大多数。轻水堆可分为三类——压水堆(PWRs)、沸水堆(BWRs)和超临界水堆(SCWR )。截至2009年,超临界水堆(SCWR) 仍处于假设状态;它属于第四代核电设计技术;但它只是由轻水部分慢化,并表现出快中子反应堆的某些特性。

许多国家在压水堆方面掌握着领先的运行经验,提供反应堆出口的领导国家有几个,美国(提供具有固有安全的AP1000 ,以及一些小规模、模块化、具有固有安全性的压水堆,如 Babcock & Wilcox MPower以及 NuScale MASLWR)、俄罗斯(提供VVER-1000和VVER-1200供出口)、法国(提供阿海珐EPR出口),日本(提供三菱公司设计制造的先进压水堆出口);此外,中国和韩国两者都迅速地上升到压水堆建造国前列,中国正在进行大规模的核电扩张计划,韩国目前正在设计和建造他们的第二代自主核电设计。美国和日本与通用电气(General Electric)和日立(Hitachi)结成联盟,提供先进的沸水反应堆(ABWR)和经济简化的沸水反应堆(ESBWR),用于建设和出口;此外,东芝还为在日本的建筑提供了ABWR改型。西德也曾是BWR的主要建设、运营者。用于发电的其他类型的核反应堆是重水慢化反应堆,加拿大建造(CANDU)和印度建造的(AHWR),英国建造了先进气体冷却反应堆(AGCR),俄罗斯,法国和日本分别建造了液态金属冷却反应器堆(LMFBR),由石墨慢化的水冷反应堆(RBMK or LWGR)仅在俄罗斯和前苏联国家被建造过。

尽管所有这些类型的反应堆的发电能力是相当的,由于上述特点,以及LWR在运行方面的丰富经验,它在绝大多数新的核电厂中受到青睐,并被建设。此外,轻水反应堆构成了为海军核动力船舶提供动力的反应堆的绝大多数。拥有核海军推进能力的五个大国中,有四个国家专门使用轻水反应堆:英国皇家海军,中国人民解放军海军,法国国家海军陆战队和美国海军。只有俄罗斯的海军使用了相对较少的液态金属冷却反应器堆,特别是705型核潜艇,它使用铅铋共晶作为反应堆慢化剂和冷却剂,但绝大多数俄罗斯核动力船舶只使用轻水反应堆。LWR在核动力海军舰艇上几乎专用的原因是这些类型反应堆的固有安全水平。由于轻水在这些反应堆中既用作冷却剂又用作中子慢化剂,如果其中一个反应堆因军事行动而受损,导致反应堆堆芯完整性受损,轻水慢化剂的释放将起到停止核反应和关闭反应堆的作用。这种能力被称为反应性负反应系数。

目前已有的的轻水反应堆堆型包括以下几类:
\begin{itemize}
\item ABWR
\item AP1000
\item APR-1400
\item CPR-1000
\item EPR
\item VVER
\end{itemize}

\subsubsection{2.1 轻水反应堆的统计数据}

国际原子能机构 2009年的数据:[8]
\begin{table}[ht]
\centering
\caption\label{QSFYD}
\begin{tabular}{|c|c}
\hline
正在运行的反应堆数量 & 359\\
\hline
正在建设的反应堆数量 & 27\\
\hline
拥有轻水反应堆的国家数量 & 27\\
\hline
总发电量(千兆瓦)。 & 328.4\\
\hline
\end{tabular}
\end{table}

\subsection{反应堆设计}
轻水反应堆通过受控核裂变产生热量。核反应堆堆芯是核反应堆发生核反应的部分。它主要由核燃料和控制元件组成。这些铅笔般细的核燃料棒,每根大约12英尺(3.7米)长,按数百个分组,被称为燃料组件。在每个燃料棒内部,含有铀或者更常见的是氧化铀,是首尾相连堆叠的。控制元件被称为控制棒,里面充满了像铪或镉这样的元素,它们很容易捕捉中子。当控制棒下降到核心时,它们吸收中子,中子因此不能参与链式反应。相反,当控制棒被提起时,更多的中子会撞击附近燃料棒中可裂变的铀-235或钚-239原子核,链式反应就会加剧。所有这些材料都被封闭在一个充水的钢制压力容器中,叫做反应堆容器。

在沸水堆,裂变产生的热量将水转化为蒸汽,蒸汽直接驱动发电涡轮机。但是在压水堆中,裂变产生的热量通过热交换器传递到次级回路。蒸汽在次级回路中产生,次级回路驱动发电涡轮机。在这两种情况下,蒸汽流过涡轮机后,在冷凝器中会被冷却成水。[9]
\begin{itemize}
\item 沸水堆的动画图
\begin{figure}[ht]
\centering
\includegraphics[width=10cm]{./figures/a7ddc1f49be2598b.png}
\caption\label{fig_QSFYD_2}
\end{figure}

\item 压水堆的动画图
\begin{figure}[ht]
\centering
\includegraphics[width=10cm]{./figures/0e97f52ce566be66.png}
\caption\label{fig_QSFYD_3}
\end{figure}
\end{itemize}
冷却冷凝器所需的水取自附近的河流或海洋。经过冷凝器加热后,温度上升,并被泵重新打入河流或海洋中。另一方面,反应堆冷却水的热量也可以通过冷却塔散发到大气中。美国主要使用轻水反应堆发电,而加拿大使用的是重水反应堆。[10]

\subsubsection{3.1 控制}
\begin{figure}[ht]
\centering
\includegraphics[width=6cm]{./figures/b97b363c915ba553.png}
\caption{压水堆的顶部,控制棒在顶部可见} \label{fig_QSFYD_4}
\end{figure}
控制棒通常组合成控制棒组件——典型的商业压水堆组件有20根控制棒——并插入燃料元件内的导管中。控制棒从堆芯的中心取出或插入,以控制将进一步分裂铀原子的中子数。这反过来又会影响反应堆生成的热能、产生的蒸汽量,从而影响发电量。控制棒从堆芯中部分移除,以允许堆芯发生链式反应。插入控制棒的数量和插入距离可以变化,从而控制反应堆的反应性。

通常还有其他控制反应性的方法。在压水堆设计中,将可溶性的中子吸收剂(通常为硼酸)添加到反应堆冷却剂中,使控制棒在固定功率运行时完全抽出,确保整个堆芯的功率和通量分布均匀。沸水堆设计的操作人员通过改变反应堆循环泵的流量,利用通过堆芯的冷却剂流来控制反应性。通过堆芯的冷却剂流量的增加有利于蒸汽气泡的去除,进而增加了冷却剂/慢化剂的密度,以增加了堆芯功率。

\subsubsection{3.2 冷却剂}
轻水反应堆也使用普通水来保持反应堆冷却。冷却源,即轻水,在反应堆核心周围循环,以吸收其产生的热量。热量从反应对堆中带走,然后用于产生蒸汽。大多数反应堆系统采用与水物理分离的冷却系统,该冷却系统将被加热后产生用于涡轮机的加压蒸汽,如压水反应堆。但是在一些反应堆中,汽轮机的水直接由反应堆堆芯沸腾,例如沸水反应堆。

许多其他反应堆也是轻水冷却的,特别是压力管式石墨慢化沸水反应堆和一些军用钚生产反应堆。这些不被视为轻水堆,因为它们由石墨慢化的,因此它们的一些核特性非常不同。尽管商用压水堆中的冷却剂流量是恒定的,但美国海军舰艇上使用的核反应堆中的冷却剂流量却不是恒定的。

\subsubsection{3.3 燃料}
\begin{figure}[ht]
\centering
\includegraphics[width=8cm]{./figures/fed01f740af05341.png}
\caption{核燃料颗粒} \label{fig_QSFYD_5}
\end{figure}

\begin{figure}[ht]
\centering
\includegraphics[width=8cm]{./figures/3eaace414b85048f.png}
\caption{准备进行燃料组装的核燃料芯块} \label{fig_QSFYD_6}
\end{figure}
必须对由燃料进行一定量的浓缩过程,才能使得在燃料被轻水冷却的过程中能够达到临界状态。轻水反应堆使用铀235 作为燃料,浓度约为3\%。虽然这是它的主要燃料,但铀238 原子也通过转化为钚239 而促进裂变过程;其中大约一半在反应堆中被消耗掉。轻水反应堆通常每12到18个月更换一次燃料,届时大约有25\%的燃料被替换。

将富集的UF6转化为二氧化铀粉末,再将二氧化铀粉末加工成球状。然后这些球团在高温烧结炉中被烧制成坚硬的陶瓷铀球团。然后圆柱形球团经过一个研磨过程,以达到统一的球团大小。氧化铀在插入管道之前会被烘干,以消除陶瓷燃料中可能导致腐蚀和氢脆的水分。根据每个核芯的设计规格,小球被堆叠到耐腐蚀的金属合金管中。这些被称为燃料棒的管是密封的,以容纳燃料球。

完成的燃料棒被分组在特殊的燃料组件中,然后被用来建立动力反应堆的核燃料堆芯。管道所用的金属取决于反应堆的设计——过去使用的是不锈钢,但现在大多数反应器使用的是锆合金。对于最常见的反应堆,管道被组装成束,管道之间有精确的距离。然后,给这些捆一个唯一的标识号,这使它们能够从制造到使用和后处理过程中被跟踪。

压水堆燃料是由圆柱形的燃料棒组成的。氧化铀陶瓷被制成球团,然后插入锆合金管中,并被捆绑在一起。锆合金管直径约为1cm,燃料包壳间隙填充氦气,以改善燃料向包壳的导热。每个燃料包大约有179-264个燃料棒,大约121到193个燃料包被装入反应堆堆芯。通常,燃料束由14x14至17x17捆扎的燃料棒组成。压水堆燃料束大约4米长。锆合金管用氦加压,以尽量减少可能导致燃料棒长期失效的颗粒包层相互作用。

在沸水堆中,燃料与压水堆燃料相似,只是管束是“罐装”的;也就是说,每个束周围都有一个细管。这主要是为了防止局部密度的变化影响到核芯的中子学和热工水力学。在现代沸水堆燃料棒束中,根据制造商的不同,每个组件有91根、92根或96根燃料棒。美国有最小的368个组件和最大的800个组件构成了反应堆堆芯。每根沸水堆燃料棒都要回填氦气,压力约为3个大气压(300千帕)。

\subsubsection{3.4 慢化剂}
中子慢化剂是一种降低快中子速度的介质,从而将它们转化为能够维持铀-235核链式反应的热中子。良好的中子慢化剂是一种充满轻核原子的材料,不容易吸收中子。中子撞击原子核并被反弹。经过充分的撞击后,中子的速度将与原子核的热速度相当;这个中子被称为热中子。

轻水反应堆使用普通的水,也称为轻水,作为中子慢化剂。轻水吸收了太多的中子,不能与未经浓缩的天然铀一起使用,因此铀浓缩或核后处理成为操作此类反应堆的必要条件,从而增加了总成本。这与使用重水作为中子慢化剂的重水堆不同。虽然普通水含有一些重水分子,但所占比例太小,无法起到作用。在压水堆中,冷却水被用作慢化剂,让中子与水中的轻氢原子发生多次碰撞,中子在这个过程中失去速度。当水的密度更大时,中子的减速会发生得更频繁,因为会发生更多的碰撞。

使用水作为慢化剂是压水堆的一个重要安全特征,因为温度的升高会导致水膨胀,密度降低;从而降低了中子减速的程度,从而降低了反应堆的反应性。因此,如果反应性增加超过正常水平,减少的中子慢化将导致连锁反应减慢,产生更少的热量。这一特性被称为反应性的负温度系数,使得压水堆非常稳定。一旦发生冷却剂损失事故,慢化剂也会损失,活跃的裂变反应也会停止。当核裂变的放射性副产物停止后,仍然会产生大约5\%的额定功率的热量。这个“衰变热”将在关闭后持续1至3年,届时反应堆最终达到“完全冷停堆”。衰变热虽然危险而强烈,足以使核心熔化,但不如裂变反应那样强烈。在关闭后的时期,反应堆需要冷却水泵或反应堆将过热。如果温度超过2200摄氏度,冷却水就会分解成氢和氧,形成(化学)爆炸性混合物。衰变热是轻水反应堆安全记录中的一个主要危险因素。

\subsection{PIUS反应堆}
PIUS,代表过程固有的最终安全性,是由瑞典通用电机公司原子设计的瑞典设计。这是轻水反应堆系统的概念。[11]与安全反应堆一样[12],它依靠被动措施,不需要操作员的动作或外部能源供应来提供安全操作。从未建造过任何单位。

\subsection{参考文献}
[1]
^"Federation of American Scientists - Early reactor" (PDF). Retrieved 2012-12-30..

[2]
^还可以注意到,由于LOPO被设计成零功率运行,并且不需要冷却装置,所以普通水仅用作减速剂。.

[3]
^"ORNL - An Account of Oak Ridge National Laboratory's Thirteen Nuclear Reactors" (PDF). p. 7. Retrieved 2012-12-28. ... Afterwards, responding to Weinberg’s interest, the fuel elements were arranged in lattices in water and the multiplication factors determined. ....

[4]
^"ORNL - History of the X10 Graphite Reactor". Archived from the original on 2012-12-11. Retrieved 2012-12-30..

[5]
^"INEEL - Proving the principle" (PDF). Archived from the original (PDF) on 2012-03-05. Retrieved 2012-12-28..

[6]
^"INEL - MTR handbook Appendix F (historical backgroup)" (PDF). p. 222. Retrieved 2012-12-31..

[7]
^"DOE oral history presentation program - Interview of LITR operator transcript" (PDF). p. 4. Archived from the original (PDF) on 2013-05-14. ... We were so nervous because there had never been a reactor fueled with enriched uranium go critical before. ....

[8]
^"IAEA - LWR". Archived from the original on 2009-02-25. Retrieved 2009-01-18..

[9]
^"European Nuclear Society - Light water reactor". Retrieved 2009-01-18..

[10]
^"Light Water Reactors". Retrieved 2009-01-18..

[11]
^美国国家研究委员会。未来核能委员会,核能:未来的技术和体制选择国家科学院出版社,1992年,ISBN 0-309-04395-6第122页.

[12]
^https://web.archive.org/web/20221028222241/http://www.gdm-marketing.se/en/gdm-marketing.