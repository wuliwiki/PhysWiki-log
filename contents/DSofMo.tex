% 模的直和
% keys 模论|module|direct sum|decomposition|分解|模
% license Usr
% type Tutor


\pentry{模\upref{Module}}




\begin{definition}{内直和}
给定一个$R$-模$M$,若它有子模$M_1, M_2, \cdots M_s$,使得任取$m\in M$,都存在$m_i\in M_i$使得
\begin{equation}
m = \sum_{i=1}^{s} m_i~, 
\end{equation}
那么称$M$是$M_i$的\textbf{和(sum)},记为
\begin{equation}
M=M_1+M_2+\cdots+M_s=\sum_{i=1}^s M_i~.
\end{equation}

进一步,如果对于任意的$m\in M$,上述表示$\sum_i m_i$是\textbf{唯一}的,则称$M$是$M_i$的\textbf{内直和(inner direct sum)},记为
\begin{equation}
M=M_1\oplus M_2\oplus \cdots \oplus M_s= \bigoplus_{i=1}^s M_i~.
\end{equation}
\end{definition}


\begin{definition}{外直和}

给定$R$-模$M_1, M_2, \cdots, M_s$,在笛卡尔积集合$M=M_1\times M_2\times \cdots \times M_s$上定义加法运算为
\begin{equation}
(a_1, a_2, \cdots, a_s)+(b_1, b_2, \cdots, b_s) = (a_1+b_1, a_2+b_2, \cdots, a_s+b_s)~, 
\end{equation}
定义$R$对$M$的数乘运算为
\begin{equation}
r(a_1, a_2, \cdots, a_s) = (ra_1, ra_2, \cdots, ra_s)~, 
\end{equation}
则$M$也构成一个$R$-模,称为$M_i$的\textbf{外直和(outer direct sum)},记为
\begin{equation}
M=M_1\oplus M_2\oplus \cdots \oplus M_s= \bigoplus_{i=1}^s M_i~.
\end{equation}

\end{definition}




外直和与内直和本质上是一回事。如果$M$是其子模$M_i$的内直和,那么计算各$M_i$的外直和,其结果和$M$同构。如果给定$R$-模$M_i$,求它们的外直和,为方便同样记为$M$,那么各$M_i$也可以同构于$M$的子模,且$M$是这些子模的内直和。












