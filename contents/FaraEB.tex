% 法拉第电磁感应定律
% keys 法拉第|电磁感应|感生电动势|楞次定律
% license Xiao
% type Tutor

\begin{issues}
\issueAbstract
\issueTODO
\end{issues}

\pentry{磁通量\upref{BFlux}}{nod_a977}

\subsection{电磁感应定律的积分形式}

我们在高中物理学过, 闭合线圈产生的感生电动势 % 未完成:链接
等于线圈内磁通量随时间的变化率。方向由右手定则决定。 即
\begin{equation}\label{eq_FaraEB_1}
\varepsilon  =  -\dv{\Phi}{t} =  - \dv{t} \int \bvec B \vdot \dd{\bvec a} =  - \int \pdv{\bvec B}{t} \vdot \dd{\bvec a}~.
\end{equation} 
这里的 $\bvec a$ 表示面积, 积分的曲面是以线圈为边界的任意曲面。 另一方面,感生电动势是由感生电场产生的。 
\begin{equation}\label{eq_FaraEB_2}
\varepsilon  = \oint \bvec E \vdot \dd{\bvec r}~.
\end{equation}
这里的路径积分是沿着线圈进行的。 规定线圈正方向以后, \autoref{eq_FaraEB_1} 中曲面的正方向由右手定则\upref{RHRul}决定。

根据麦克斯韦方程组, 电场产生的原因有两种, 一种是电荷产生电场(电场的高斯定律\upref{EGauss}), 另一种是变化的磁场产生电场(法拉第电磁感应)。 前者在\autoref{eq_FaraEB_2} 中的环路积分为零, 对电动势没有贡献。 所以\autoref{eq_FaraEB_2} 中的 $\bvec E$ 既可以只包含感生电场, 也可以是总电场。 我们一般理解为总电场。

对比上面两式,得
\begin{equation}\label{eq_FaraEB_4}
\oint \bvec E \vdot \dd{\bvec r}  =  - \int \pdv{\bvec B}{t} \vdot \dd{\bvec a} ~.
\end{equation} 
如果我们假设感生电场只与电场的分布和变化率有关,则这个公式对空间中任何假想中的回路都成立,而不需要有真正的线圈存在。注意上式中的磁场是空间中的所有磁场。

\subsection{电磁感应定律的微分形式}
\pentry{斯托克斯定理\upref{Stokes}}{nod_faa7}

对电场项应用斯托克斯定理\upref{Stokes}, 
\begin{equation}
\oint \bvec E \vdot \dd{\bvec r}  = \int \curl \bvec E \vdot \dd{\bvec a}~.
\end{equation}
代回积分方程\autoref{eq_FaraEB_4},得
\begin{equation}
\int \curl \bvec E \vdot\dd{\bvec a}  =  - \int \pdv{\bvec B}{t} \vdot \dd{\bvec a} ~.
\end{equation} 

由于该公式对于所有回路均成立,所以
\begin{equation}\label{eq_FaraEB_3}
\curl \bvec E =  - \pdv{\bvec B}{t}~.
\end{equation} 

\subsection{楞次定律}
由电磁感应定律可以得出\textbf{楞次定律(Lenz's law)}。 它能够确定由电磁感应产生的电动势的方向, 即感应电流产生的磁场总是与外磁场变化的方向相反。 
