% 拓扑空间的收敛序列
% keys 收敛序列|拓扑空间
% license Xiao
% type Tutor

\pentry{拓扑空间\nref{nod_Topol}}{nod_a6e9}

\cite{Ke1}度量空间中,收敛序列起着奠基性的作用,例如数列的极限就是重要的例子,而数列的极限是微积分的基石。收敛序列的概念很容易搬到拓扑空间中来,但是它在拓扑空间中的地位不像度量空间中那样。主要的原因在于在度量空间中,序列收敛到 $x$ 是 $x$ 为接触点的充要条件,而在拓扑空间中,这一论断失效了。

\subsection{定义}
本文将始终假设构造拓扑空间 $\mathcal T$ 的基础集合为 $X$\upref{Topol},而 $x\in\mathcal T$ 表示 $x\in X$。

拓扑空间 $\mathcal T$ 中序列的定义本质是不变的,仅在于每一点 $x_n$ 都属于 $\mathcal T$ 的基础集合 $X$。
\begin{definition}{收敛序列}
设 $\{x_n\}$ 是拓扑空间 $\mathcal T$ 上的点列,若 $x\in\mathcal T$ 的任一邻域都包含从某一 $N$ 开始的所有点 $x_n,n\geq N$,则称 $\{x_n\}$ \textbf{收敛}于 $x$,而 $\{x_n\}$ 则称为\textbf{收敛的}。
\end{definition}
\subsection{性质}
 $M\subset\mathcal X$ 的\textbf{接触点} $x$ 是指 $x\in X$ 的任一邻域都含有 $M$ 的点(即 $x\in[M]$\footnote{$[M]$ 指 $M$ 的闭包})。
 
 \begin{lemma}{}\label{lem_ConvTp_1}
 取 $[0,1]$ 为基本集合构造如下拓扑 $\mathcal T$:$O$ 是 $\mathcal T$ 的开集,当且将当 $O$ 是从 $[0,1]$ 中去掉任意有限或可数个点得到的。那么 $\mathcal T$ 的序列 $\{x_n\}$ 是收敛的,当且仅当 $\{x_n\}$ 是定常序列,即从某一下标开时,所有元素相同,即当且将当存在整数 $N>0$,使得$x_n=x_{n+1}=\ldots,n\geq N$。
 \end{lemma}
 \textbf{证明:}
设 $\{x_n\}$ 是 $\mathcal T$ 中任意这样的一个序列:对任意整数 $N>0$,都有 $n,m\geq N$,使得 $x_n\neq x_m$。下面用反证法证明。 

假设 $\{x_n\}$ 收敛到 $x\in\mathcal T$,则对任一 $x$ 的邻域 $O$,存在整数 $N>0$,使得 $x_n\in O,n\geq N$。然而,$[0,1]$ 中删掉了一切点 $x_i\neq x$ 得到的子集 $O'$ 满足 $\mathcal T$ 上开集的定义,因此 $O'$ 是 $x$ 的邻域。然而任一 $N>0$,存在 $x_i\neq x,i\geq N$,因此邻域 $O'$ 不能包含从某一项开始的序列中的点。  这一矛盾证明了定理。

 \textbf{证毕!}

\begin{corollary}{}\label{cor_ConvTp_1}
设拓扑空间 $\mathcal T$ 如\autoref{lem_ConvTp_1} 定义。则 $0$ 是 $(0,1]$ 的接触点,但不存在 $(0,1]$ 中收敛于 $0$ 的点列。
\end{corollary}
\textbf{证明:}第二部分可以由\autoref{lem_ConvTp_1} 直接证得:$[0,1]$ 中收敛于 $0$ 的点列只能是从某一项开始全为0的点列 $\{x_n\}$。而这在 $(0,1]$ 中是不可能的。

第一部分证明如下:由开集的定义,任一 $0$ 的邻域,都包含有包含 $0$ 的开集 $O$,其中 $O$ 至多去掉可数个 $[0,1]$ 中的点,这表明 $O$ 是不可数的,因此必定包含有 $(0,1]$ 中的点。即任意 $0$ 的邻域都含有 $(0,1]$ 中的点,因此 $0$ 是 $(0,1]$ 的接触点。 

\textbf{证毕!}


 \begin{theorem}{}
 一般而言,在拓扑空间中,序列收敛到 $x$ 不是 $x$ 为接触点的充要条件。
 \end{theorem}
 \textbf{证明}:\autoref{cor_ConvTp_1} 直接表明了这一论断。

 \textbf{证毕!}

既然度量空间是拓扑空间的特殊情形,而在度量空间中,序列收敛到 $x$ 是 $x$ 为接触点的充要条件。那么什么情况下,即给拓扑空间加上什么条件,就能使得这一论断成立呢?下面定理给出了保证。


\begin{theorem}{}
设拓扑空间 $\mathcal T$ 满足第一可数性公理(即空间 $\mathcal T$ 的每一点 $x$ 都存在可数的确定领域族\footnote{即对每一包含 $x$ 的开集,都能在该族中找到一个邻域,使它完全包含在该开集中}),那么任意集 $M\subset\mathcal T$ 的每一接触点都可看作某一点列的极限。
\end{theorem}

 \textbf{证明}:设 $\{O_n\}$ 是 $x$ 的可数确定邻域族。可以认为 $O_{n+1}\subset O_n$(否则用 $\bigcap_{k=1}^{n+1} O_{k}$ 代替 $O_{n+1}$)。设 $x_k\subset O_k\cap M$,则 $\{x_n\}$ 收敛到 $x$(因为任一 $x$ 的邻域必然包含某一 $O_i$,而 $O_{i+k}\subset O_{i},k=1,\cdots$ )。这样的 $x_k$ 必然存在,否则 $x$ 就不是接触点了。 

 
 \textbf{证毕!}

因为收敛序列的极限点显然就是接触点(有接触点的定义),因此上面定理也表明在第一可数性拓扑空间,序列收敛到 $x$ 是 $x$ 为接触点的充要条件。任一度量空间都是满足第一可数性公理的拓扑空间。因为每一点 $x$,开球族 $\{B(x,1/n)|n=1,2,\cdots\}$ 是它的可数确定邻域族。






