% QCD 费曼规则
% license Usr
% type Tutor


\begin{issues}
\issueMissDepend
\issueAbstract
\end{issues}

QCD 主要考虑夸克内线、胶子(传播子)内线、鬼粒子内线;夸克-胶子相互作用顶点、三胶子顶点、四胶子顶点与胶子-鬼粒子顶点。鬼粒子仅在内线中出现,不会有外线鬼粒子。$f_{abc}$ 为结构常数。考虑耦合常数为 $g$,度规则是有指标的 $g_{\mu\nu}$。

\subsection{内线}
\subsubsection{夸克}
\begin{figure}[ht]
\centering
\includegraphics[width=6cm]{./figures/802e961bc222fa42.png}
\caption{夸克内线} \label{fig_qcdfey_1}
\end{figure}
\begin{equation}
\mathrm i \delta^{mn} \frac{1}{ p\not ~- m} = \mathrm i S^{mn}_F(p)~.
\end{equation}
其中分子处的 $m$ 非指标而是质量。

\subsubsection{胶子}
\begin{figure}[ht]
\centering
\includegraphics[width=6cm]{./figures/72568c83b4b0f177.png}
\caption{胶子内线} \label{fig_qcdfey_2}
\end{figure}
\begin{equation}
\mathrm i D_{F, \mu\nu}^{ab}(p) = -\mathrm i \delta^{ab} \frac{1}{p^2} \left[ g_{\mu\nu} - \left(1-\frac{1}{\xi}\right) \frac{p_\mu p_\nu}{p^2} \right] ~.
\end{equation}
其中 $\xi$ 是规范选择项。

\subsubsection{鬼粒子}
\begin{figure}[ht]
\centering
\includegraphics[width=6cm]{./figures/0222c47fda024c63.png}
\caption{鬼粒子内线} \label{fig_qcdfey_3}
\end{figure}
\begin{equation}
\mathrm i \widetilde{\Delta}_F(q) = \mathrm i \delta^{ab} \frac{1}{q^2} ~.
\end{equation}

\subsection{外线}
外线没有鬼粒子,胶子的规则类似于 QED 的光子,夸克的则类似于电子。

\subsection{相互作用顶角}
\subsubsection{夸克-胶子顶点}
\begin{figure}[ht]
\centering
\includegraphics[width=6cm]{./figures/6e138c550bfa68c4.png}
\caption{夸克-胶子顶点} \label{fig_qcdfey_4}
\end{figure}
\begin{equation}
-\mathrm i g \gamma^\mu T_a ~.
\end{equation}

\subsubsection{三胶子顶点}
\begin{figure}[ht]
\centering
\includegraphics[width=6cm]{./figures/ac9dc6b83e165f12.png}
\caption{三胶子顶点} \label{fig_qcdfey_5}
\end{figure}
\begin{equation}
    \begin{aligned}
		-g f^{abc}  [&g^{\mu\nu}(p-q)^\sigma \\
		+& g^{\nu\sigma} (q-k)^\mu \\
		+& g^{\sigma\mu}(k-p)^\nu] ~.
	\end{aligned}
\end{equation}

\subsubsection{四胶子顶点}
\begin{figure}[ht]
\centering
\includegraphics[width=6cm]{./figures/310f072b7a34445c.png}
\caption{四胶子顶点} \label{fig_qcdfey_6}
\end{figure}
\begin{equation}
	\begin{aligned}
		-\mathrm i g^2 [& f^{abe} f^{cde}(g^{\mu\sigma}g^{\nu\rho} - g^{\mu\rho}g^{\nu\sigma}) \\
		+& f^{ace}f^{bde}(g^{\mu\nu} g^{\sigma\rho} - g^{\mu\rho}g^{\sigma\nu}) \\
		+& f^{ade}f^{bce} (g^{\mu\nu} g^{\rho\sigma} - g^{\mu\sigma}g^{\rho\nu})] ~.
	\end{aligned}
\end{equation}

\subsubsection{胶子-鬼粒子顶点}
\begin{figure}[ht]
\centering
\includegraphics[width=6cm]{./figures/67a8f60a0fc0f113.png}
\caption{胶子-鬼粒子顶点} \label{fig_qcdfey_7}
\end{figure}
\begin{equation}
g f^{cab} p^\mu ~.
\end{equation}
