% 正交矩阵、酉矩阵
% 线性代数|矩阵|复数|正交矩阵|酉矩阵|幺正矩阵|正交归一基底|逆矩阵|单位正交阵

\begin{issues}
\issueDraft
\issueOther{需要整合}
\end{issues}

\textbf{酉矩阵(unitary matrix)}也叫\textbf{幺正矩阵}。 当矩阵元为实数时也叫\textbf{正交矩阵(orthogonal matrix)}, 是正交矩阵的复数拓展, 即矩阵元可以是复数。 酉矩阵 $\mat U$ 的定义同样为
\begin{equation}
\sum_k U_{ki}^* U_{kj} = \delta_{ij}
\end{equation}
但由于复数的列矢量没有对应的几何矢量\upref{GVec}, 所以这里的正交完全是复线性空间中的广义正交。

\subsection{实数的情况}

\pentry{正交归一基底\upref{OrNrB}, 逆矩阵\upref{InvMat}}

若一个实数方阵的每一列的模长都等于 $1$, 且任意两列都正交, 那么这个矩阵就是一个\textbf{单位正交阵}。 若 $\mat U$ 为单位正交阵, 则其矩阵元满足
\begin{equation}\label{UniMat_eq1}
\sum_k U_{ki} U_{kj} = \delta_{ij}
\end{equation}
其中 $\delta_{ij}$ 是克罗内克 $\delta$ 函数。 所以若把 $N$ 阶单位正交阵的每一列看做 $N$ 维空间中的一个单位矢量的直角坐标, 那么这些单位矢量就组成该空间的一组正交归一基底。

\autoref{UniMat_eq1} 也可以用矩阵转置和矩阵乘法表示为
\begin{equation}
\mat U\Tr \mat U = \mat I
\end{equation}
其中 $\mat I$ 是单位矩阵。 根据逆矩阵的定义, 我们得到单位正交矩阵的一个重要性质, 即其逆矩阵等于转置矩阵。
\begin{equation}\label{UniMat_eq2}
\mat U^{-1} = \mat U\Tr
\end{equation}
由逆矩阵的性质(\autoref{InvMat_eq2}~\upref{InvMat})得
\begin{equation}
\mat U \mat U\Tr = \mat U \mat U^{-1} = \mat I
\end{equation}
表示为矩阵元的运算就是
\begin{equation}\label{UniMat_eq5}
\sum_k U_{ik} U_{jk} = \delta_{ij}
\end{equation}
所以单位正交矩阵的所有行同样正交归一。 易证\autoref{UniMat_eq1} 和\autoref{UniMat_eq5} 互为充分必要条件, 都可以作为单位正交阵的定义。

\subsection{几何理解}
为了更形象地理解单位正交阵的上述性质, 我们以二维几何矢量和二维实数方阵为例讨论, 三维的情况同理可得。

任意的二维实数酉矩阵可以看做是两组正交归一基底 $\uvec x, \uvec y$ 和 $\uvec u, \uvec v$  之间的变换(“见平面旋转矩阵\upref{Rot2D}” 中的 “被动理解”)。
\begin{equation}
\mat U = \pmat{\uvec u \vdot \uvec x & \uvec v \vdot \uvec x \\
\uvec u \vdot \uvec y & \uvec v \vdot \uvec y}
\end{equation}
显然矩阵的两个列矢量满足正交归一。 由内积的交换律, 我们同样可以把矩阵的两行分别看做单位矢量 $\uvec x, \uvec y$ 在正交归一基底 $\uvec u, \uvec v$ 上的坐标, 所以矩阵的两个行矢量同样正交归一。

现在我们令任意几何矢量以 $\uvec x, \uvec y$ 为基底的坐标为 $(x, y)$, 以 $\uvec u, \uvec v$ 为基底的坐标为 $(u, v)$, 即
\begin{equation}
\uvec v = u\uvec u + v\uvec v = x\uvec x + y\uvec y
\end{equation}

由正交归一基底的性质,
\begin{equation}
\begin{cases}
x = (u\uvec u + v\uvec v)\vdot\uvec x\\
y = (u\uvec u + v\uvec v)\vdot\uvec y
\end{cases}
\qquad
\begin{cases}
u = (x\uvec x + y\uvec y)\vdot\uvec u\\
v = (x\uvec x + y\uvec y)\vdot\uvec v
\end{cases}
\end{equation}
写成矩阵的形式, 就是
\begin{equation}
\pmat{x\\y} = \pmat{\uvec u \vdot \uvec x & \uvec v \vdot \uvec x \\
\uvec u \vdot \uvec y & \uvec v \vdot \uvec y} \pmat{u\\v}
\qquad
\pmat{u\\v} = \pmat{\uvec x \vdot \uvec u & \uvec y \vdot \uvec u \\
\uvec x \vdot \uvec v & \uvec y \vdot \uvec v} \pmat{x\\y}
\end{equation}
注意上式左边的矩阵是 $\mat U$, 右边的矩阵是 $\mat U\Tr$, 而右边的变换是左边的逆变换, 所以 $\mat U\Tr = \mat U^{-1}$。 
