% Python 输入和输出
% keys Python|输入|输出|print|input
% license Usr
% type Tutor

\begin{issues}
\issueTODO
\issueDraft
\end{issues}

\pentry{Python 注释\nref{nod_Pynot1}, Python 基本数据类型与转换\nref{nod_PyData}}{nod_1237}

\subsection{print 和 input}

\begin{itemize}
\item Python两种输出值的方式: 表达式语句和 print() 函数。
\item 第三种方式是使用文件对象的 \verb|write()| 方法,标准输出文件可以用 \verb|sys.stdout| 引用。
\item 如果你希望输出的形式更加多样,可以使用 \verb|str.format()| 函数来格式化输出值。
\item 如果你希望将输出的值转成字符串,可以使用 \verb|repr()| 或 \verb|str()| 函数来实现。
\end{itemize}

\begin{itemize}
\item \verb|str()|: 函数返回一个用户易读的表达形式
\item \verb|repr()|: 产生一个解释器易读的表达形式。
\end{itemize}

注意:在 python2.x 版本中,\verb|print| 是一个关键字,输出内容不需要加括号, python 3.x 中它是一个函数, 可以介绍任意多个用逗号隔开的变量, 他们甚至可以是不同类型的。 例如
\begin{lstlisting}[language=python]
print volumn  # 2.x
print(volumn) # 3.x
\end{lstlisting}
