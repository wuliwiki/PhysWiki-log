% 古斯塔夫·赫兹(综述)
% license CCBYSA3
% type Wiki

本文根据 CC-BY-SA 协议转载翻译自维基百科\href{https://en.wikipedia.org/wiki/Gustav_Ludwig_Hertz}{相关文章}。

\begin{figure}[ht]
\centering
\includegraphics[width=6cm]{./figures/b43bf9ebf56875f5.png}
\caption{赫兹在1925年} \label{fig_GSTFhz_1}
\end{figure}
古斯塔夫·路德维希·赫兹(德语:[ˈɡʊs.taf ˈluːt.vɪç hɛʁt͡s] ⓘ;1887年7月22日-1975年10月30日)\(^\text{[2]}\)是德国的原子物理学家,他与詹姆斯·弗兰克共同获得了1925年诺贝尔物理学奖,“因其发现了电子与原子碰撞的规律”。
\subsection{生平}
赫兹出生于汉堡,父母分别是奥古斯特(婚前姓阿尔宁)和律师古斯塔夫·西奥多·赫兹(1858-1904)\(^\text{[2]}\),他是海因里希·鲁道夫·赫兹的兄弟。他在汉堡的约翰修道院学校接受教育,之后在以下大学学习:1906年至1907年间,哥廷根的乔治-奥古斯特大学;1907年至1908年间,慕尼黑的路德维希·马克西米利安大学;以及1908年至1911年间,柏林的洪堡大学。1911年,他在海因里希·鲁本斯教授指导下获得博士学位。\(^\text{[3][4]}\)

从1911年到1914年,赫兹是鲁本斯教授在柏林大学的助手。在这段时间里,赫兹与詹姆斯·弗兰克一起进行了关于气体中非弹性电子碰撞的实验,通常称为弗兰克-赫兹实验,\(^\text{[5]}\)因这一实验,他们共同获得了1925年诺贝尔物理学奖。\(^\text{[6]}\)

第一次世界大战期间,赫兹自1914年起服役。1915年,他加入了弗里茨·哈伯的部队,该部队负责使用有毒的氯气作为武器。\(^\text{[7]}\)赫兹在1915年受重伤。1917年,他返回柏林大学担任私人讲师。1920年,他在埃因霍温的飞利浦白炽灯厂担任研究物理学家,直到1925年。\(^\text{[2]}\)
\subsubsection{职业生涯}
1925年,赫兹成为马丁·路德大学哈雷-威滕贝格分校物理学研究所的正教授和所长。1928年,他成为柏林工业大学(现为柏林工业大学)实验物理学的正教授及物理学研究所所长。在此期间,他开发了一种通过气体扩散分离同位素的技术。

由于赫兹在第一次世界大战期间曾服役,他曾一度免受纳粹政策和《恢复公务员法》的影响,但随着政策和法律变得更加严格,到1934年底,他被迫辞去柏林工业大学的职务,因为他被归类为“二级部分犹太人”(他的祖父古斯塔夫·费尔迪南德·赫兹(原名大卫·古斯塔夫·赫兹)(1827-1914)童年时期曾是犹太人,直到全家在1834年皈依路德教)。随后,他在西门子公司担任第二研究实验室的所长。在那里,他继续从事原子物理学和超声波的研究,但最终停止了同位素分离方面的工作。他一直担任这一职务,直到1945年前往苏联。\(^\text{[4][2][9]}\)
\subsubsection{在苏联}
\textbf{“叛逃协议”}

赫兹对自己的安全感到担忧,与他同为诺贝尔奖得主的詹姆斯·弗兰克一样,他正在寻求前往美国或其他任何德国以外的地方。因此,他与三位同事达成了协议:曼弗雷德·冯·阿尔登,私人实验室“电子物理学研究实验室”主任;彼得·阿道夫·蒂森,柏林洪堡大学的正教授及凯瑟·威廉姆物理化学与电化学研究所(KWIPC)在柏林达赫姆的所长;以及马克斯·福尔默,柏林工业大学物理化学研究所的正教授和所长。\(^\text{[10]}\)该协议的内容是,谁先与苏联取得联系,谁就代表其他人发声。协议的目标有三个:(1)防止他们的研究所遭受掠夺,(2)尽量减少工作中断,(3)保护他们免受因过去政治行为的起诉。\(^\text{[11]}\)在第二次世界大战结束前,蒂森是纳粹党成员,并有共产主义联系。\(^\text{[12]}\)

\textbf{参与苏联核武器计划}

1945年4月27日,蒂森与一名苏联军队上校共同乘装甲车到达冯·阿尔登的研究所,该上校也是一名苏联著名化学家。\(^\text{[13]}\)所有四位协议成员都被带到了苏联。赫兹被任命为阿古杰里研究所的负责人,阿古杰里位于苏胡米东南约10公里,是古尔里普斯的郊区。\(^\text{[13][14]}\)赫兹领导的G研究所负责的课题包括:(1)通过惰性气体流中的扩散分离同位素,赫兹为此项目的负责人;(2)开发冷凝泵,尤斯图斯·缪伦福特为负责人;(3)设计和制造用于确定铀同位素组成的质谱仪,维尔纳·舒策为负责人;(4)开发无框(陶瓷)扩散隔板用于过滤器,赖因霍尔德·赖希曼为负责人;(5)开发扩散级联的稳定性和控制理论,海因茨·巴尔维希为负责人。\(^\text{[13][15]}\)

巴尔维希曾是赫兹在西门子公司的副手。\(^\text{[16]}\)G研究所的其他成员包括维尔纳·哈特曼和卡尔-弗朗茨·兹尔克。\(^\text{[17]}\)曼弗雷德·冯·阿尔登被任命为A研究所的负责人。冯·阿尔登的A研究所的目标包括:(1)同位素的电磁分离,冯·阿尔登为负责人;(2)制造用于同位素分离的多孔屏障的技术,彼得·阿道夫·蒂森为负责人;(3)铀同位素分离的分子技术,马克斯·斯廷贝克为负责人。

在与拉夫连季·贝利亚的第一次会议中,冯·阿尔登被要求参与制造原子弹,但他很快意识到,参与此项目将禁止他返回德国,于是他提出同位素浓缩作为目标,最终获得同意。

\textbf{苏胡米的研究}

到1940年代末,几乎有300名德国人在该研究所工作,但这还不是全部工作人员。\(^\text{[13][14]}\)A研究所被用作苏胡米物理技术研究所的基础,该所位于苏胡米的一个郊区——西诺普。冯尔默前往莫斯科的第九科学研究所(NII-9)\(^\text{[18]}\),并获得了一个设计局,负责重水生产的工作。在A研究所,蒂森成为了制造用于同位素分离的多孔屏障技术的负责人。\(^\text{[13]}\)

1949年,包括赫兹、蒂森和巴尔维希在内的六位德国科学家受邀到斯维尔德洛夫斯克-44(负责铀浓缩)进行咨询。该工厂比美国的橡树岭气体扩散工厂规模小,铀浓缩度只有预期90\%以上的一半多一点。\(^\text{[19]}\)

1950年后,赫兹搬到了莫斯科。1951年,赫兹与巴尔维希一起获得了斯大林奖二等奖。\(^\text{[13]}\) 同年,詹姆斯·弗兰克与赫兹共同获得了德国物理学会的马克斯·普朗克奖章。赫兹一直留在苏联,直到1955年。\(^\text{[4]}\)

\textbf{返回东德}

从苏联回国后,赫兹成为了莱比锡大学的正教授。从1955年到1967年,他还担任德国民主共和国(GDR)物理学会主席;1967年到1975年,他担任荣誉主席。\(^\text{[6]}\)
\subsubsection{个人生活}
古斯塔夫·赫兹是海因里希·鲁道夫·赫兹的侄子,马蒂尔德·卡门·赫兹的表亲。1919年,赫兹与埃伦(原姓迪尔曼)结婚,埃伦于1941年去世。他们有两个儿子,卡尔·赫尔穆特·赫兹和约翰内斯·海因里希·赫兹;两人都成为了物理学家。\(^\text{[2]}\)

他于1975年10月30日在东柏林去世,享年88岁。
\subsection{科学会员身份}
赫兹是柏林德国科学院的成员,哥廷根科学院的通讯会员,匈牙利科学院的荣誉会员,捷克斯洛伐克科学院的会员,以及苏联科学院的外籍会员。\(^\text{[2]}\)

\subsection{出版物}
\begin{itemize}
\item Franck, J.; Hertz, G. (1914). "Über Zusammenstöße zwischen Elektronen und Molekülen des Quecksilberdampfes und die Ionisierungsspannung desselben". Verh. Dtsch. Phys. Ges. 16: 457–467.
\item Gustav Hertz, Über das ultrarote Adsorptionsspektrum der Kohlensäure in seiner Abhängigkeit von Druck und Partialdruck。(博士论文)。(Vieweg Braunschweig, 1911)
\item Gustav Hertz (编辑) Lehrbuch der Kernphysik I-III (Teubner, 1961–1966)
\item Gustav Hertz (编辑) Grundlagen und Arbeitsmethoden der Kernphysik(Akademie Verlag, 1957)
\item Gustav Hertz Gustav Hertz in der Entwicklung der modernen Physik*(Akademie Verlag, 1967)
\end{itemize}
\subsection{参见}
\begin{itemize}
\item 电子衍射
\item 电辉放电
\item 弗兰克–赫兹实验
\item 等离子窗
\item 真空管
\item 散射
\item 俄罗斯Alsos
\end{itemize}
\subsection{参考文献}
\begin{enumerate}
\item "Gustav Hertz". Mathematics Genealogy Project.
\item Hertz – Nobel Biography.
\item Gustav Hertz, Über das ultrarote Adsorptionsspektrum der Kohlensäure in seiner Abhängigkeit von Druck und Partialdruck*。(博士论文)。(Vieweg Braunschweig, 1911)
\item Mehra 和 Rechenberg, 2001, 197。
\item Franck, J.; Hertz, G. (1914). "Über Zusammenstöße zwischen Elektronen und Molekülen des Quecksilberdampfes und die Ionisierungsspannung desselben". Verh. Dtsch. Phys. Ges. 16: 457–467。
\item Hentschel, 1996, 附录 F;见赫兹条目。
\item Van der Kloot, W. (2004). "April 1918: Five Future Nobel prize-winners inaugurate weapons of mass destruction and the academic-industrial-military complex". Notes Rec. R. Soc. Lond. 58 (2): 149–160. doi:10.1098/rsnr.2004.0053. S2CID 145243958。
\item Wolff, Stefan L. (2008年1月4日). "Juden wider Willen – Wie es den Nachkommen des Physikers Heinrich Hertz im NS-Wissenschaftsbetrieb erging". Jüdische Allgemeine。
\item Hentschel, 1996, 23 和附录 F – 见蒂森条目。
\item sachen.de 归档于2008年3月25日 – Zur Ehrung von Manfred von Ardenne。
\item Heinemann-Grüder, 2002, 44。
\item Hentschel, 1996, 附录 F;见蒂森条目。
\item Oleynikov, 2000, 第5、10–13、18、21页。
\item Naimark, 1995, 213。
\item Kruglov, 2002, 131。
\item Naimark, 1995, 209。
\item Maddrell, 2006, 179–180。
\item 今天,NII-9是波赫瓦尔全俄无机材料科学研究所。见 Oleynikov, 2000, 第4页。
\item Holloway, 1994, 191–192。
\end{enumerate}
\subsection{进一步阅读}
\begin{itemize}
\item Albrecht, Ulrich, Andreas Heinemann-Grüder, 和 Arend Wellmann, Die Spezialisten: Deutsche Naturwissenschaftler und Techniker in der Sowjetunion nach 1945* (Dietz, 1992, 2001) ISBN 3-320-01788-8
\item Barwich, Heinz 和 Elfi Barwich, *Das rote Atom* (Fischer-TB.-Vlg., 1984)
vBeneke, Klaus, *Die Kolloidwissenschaftler Peter Adolf Thiessen, Gerhart Jander, Robert Havemann, Hans Witzmann und ihre Zeit* (Knof, 2000)
\item Heinemann-Grüder, Andreas, *Die sowjetische Atombombe* (Westfaelisches Dampfboot, 1992)
\item Heinemann-Grüder, Andreas, *Keinerlei Untergang: German Armaments Engineers during the Second World War and in the Service of the Victorious Powers* in Monika Renneberg 和 Mark Walker (编辑) *Science, Technology and National Socialism* 30–50 (Cambridge, 2002 paperback edition) ISBN 0-521-52860-7
\item Hentschel, Klaus (编辑) 和 Ann M. Hentschel (编辑助理和翻译), Physics and National Socialism: An Anthology of Primary Sources (Birkhäuser, 1996) ISBN 0-8176-5312-0
\item Holloway, David, *Stalin and the Bomb: The Soviet Union and Atomic Energy 1939–1956* (Yale, 1994) ISBN 0-300-06056-4
\item Kruglov, Arkadii, *The History of the Soviet Atomic Industry* (Taylor and Francis, 2002)
\item Maddrell, Paul, *Spying on Science: Western Intelligence in Divided Germany 1945–1961* (Oxford, 2006) ISBN 0-19-926750-2
\item Mehra, Jagdish 和 Helmut Rechenberg, *The Historical Development of Quantum Theory. Volume 1 Part 1 The Quantum Theory of Planck, Einstein, Bohr and Sommerfeld 1900–1925: Its Foundation and the Rise of Its Difficulties* (Springer, 2001) ISBN 0-387-95174-1
\item Naimark, Norman M., *The Russians in Germany: A History of the Soviet Zone of Occupation, 1945–1949* (Belknap, 1995)
\item Oleynikov, Pavel V. 2000. *German Scientists in the Soviet Atomic Project*. *The Nonproliferation Review*, Volume 7, Number 2, 1–30. 作者曾是俄罗斯联邦核中心技术物理研究所的组长,位于斯涅津斯克(Chelyabinsk-70)。
\end{itemize}
\subsection{外部链接}
\begin{itemize}
\item [Gustav Hertz 的媒体相关内容](https://commons.wikimedia.org/wiki/Category:Gustav_Hertz) 在 Wikimedia Commons
\item [Gustav Hertz 在诺贝尔奖官网](https://www.nobelprize.org/prizes/physics/1925/hertz/facts/)
\item [SIPT – 苏胡米物理技术研究所](http://www.sipt.su) — 网站上发布了曾为苏联核计划工作的德国核物理学家的照片
\end{itemize}
