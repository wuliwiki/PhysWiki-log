% 群的自由积
\pentry{自由群\upref{FreGrp}}

将自由群的概念推广,即可得到两个群之间的自由积的概念。

\subsection{自由积的构造}

给定两个群 $G$ 和 $H$,取集合 $G\cup H$ 上的自由群 $F(G\cup H)$,则 $F(G\cup H)$ 的元素形如 $x_1x_2\cdots x_k$ 的有限长字符串,其中 $k$ 是某个正整数,各 $x_i$ 都是 $G\cup H$ 的元素。

在 $F(G\cup H)$ 上定义一个等价关系:如果字符串 $g_1g_2\cdots g_k$ 中各 $g_i\in G$,那么令 $g_1g_2\cdots g_k\sim g_1\cdot g_2\cdot\cdots\cdot g_k$,即把该字符串等同于各字母在群 $G$ 中运算的结果;同样地把字母都是 $H$ 中元素的字符串等同于这些元素在群 $H$ 中的运算结果;把 $G$ 和 $H$ 的单位元等同于空词。比如说,在整数加法群 $\mathbb{Z}$ 中,把字符串 $123$ 等同于数字 $1+2+3$ 所代表的字符串,即只有一个字母 $6$ 的字符串。

这样一来,商群 $F(G\cup H)/\sim$ 中的字符串就形如 $g_1h_1g_2h_2\cdots g_kh_k$、$g_1h_1g_2h_2\cdots g_k$、$h_1g_1h_2g_2\cdots h_kg_k$ 或 $h_1g_1h_2g_2\cdots h_k$ 的字符串,或者简单来说,有限长的 $g$ 和 $h$ 的交替字符串,其中 $g, g_i\in G$,$h, h_i\in H$。

称商群 $F(G\cup H)/\sim$ 为群 $G$ 和群 $H$ 的\textbf{自由积(free product)},记为 $G*H$。

\subsection{共合积}
\begin{definition}{共合积}
设有三个群 $F, G, H$,且有群同态 $\phi:F\rightarrow G$ 和 $\varphi:F\rightarrow H$,那么可以定义 $G$ 和 $H$ 关于 $F$ 的\textbf{共合积} $G*_FH$ 如下:

$G*_FH$ 是 $G*H$ 的商集,$G*H/\sim$,其中等价关系为:$\forall f\in F, \phi(f)\sim\varphi(f)$。
\end{definition}

简单来说,$G*_FH$ 的元素依然是 $H$ 和 $G$ 中元素交替排列的字符串,但是在给定同态 $\phi:F\rightarrow G$ 和 $\varphi:F\rightarrow H$ 时,把所有 $\phi(f)\sim\varphi(f)$ 都看成等价元素。




既然是自由群上规定等价关系得来的,共合积也可以用描述为自由群的商群。




\begin{theorem}{}

给定互不相交的群$G$和$H$,再给定群$F$和\textbf{群同态}$\varphi:F\to G$和$\phi: F\to H$。

令$S$为$G*H$上全体形如$A\varphi(x)\phi(x^{-1})A^{-1}$和$A\phi(x)\varphi(x^{-1})A^{-1}$的元素构成的集合,其中$A\in G*H$,$x\in F$,则由$S$生成的$G*H$的子群$\langle S \rangle$是$G*H$的\textbf{正规子群}。

进一步有$G*_FH=G*H/\langle S \rangle$。

\end{theorem}




\textbf{证明}:

$\langle S \rangle \lhd G*H$很好证,从形式上就可以,此处只举一例权作证明思路:对于$A\varphi(x)\phi(x^{-1})A^{-1}B\phi(y)\varphi(y^{-1})B^{-1}\in \langle S \rangle$,任取$C\in G*H$,则
\begin{equation}
\begin{aligned}
    &CA\varphi(x)\phi(x^{-1})A^{-1}B\phi(y)\varphi(y^{-1})B^{-1}C^{-1} \\
    ={}& 
    CAC^{-1}C\varphi(x)\phi(x^{-1})C^{-1}CA^{-1}C^{-1}CBC^{-1}C\phi(y)\varphi(y^{-1})C^{-1}CB^{-1}C^{-1}\\
    ={}& 
    (CAC^{-1}C)\varphi(x)\phi(x^{-1})(CAC^{-1}C)^{-1}(CBC^{-1}C)\phi(y)\varphi(y^{-1})(CBC^{-1}C)^{-1}\\
    \in{}& \langle S \rangle~.
\end{aligned}
\end{equation}

定义满同态$h:G*H\to G*_FH$,使得$h$限制在$G$或$H$上都是恒等映射。

$G*H$中的一个不是$1$的词要被$h$映射为$1$,那么词中必有一对相邻字母互相抵消,按照共合积的定义,这对相邻字母必形如$\varphi(x)\phi(x^{-1})$或$\phi(x)\varphi(x^{-1})$。抵消后,剩下的词(按$G*H$中的方式抵消后)如果不是$G*H$的$1$,则还可以继续找互相抵消的相邻字母。以此类推,可知$\opn{ker}h\subseteq \langle S \rangle$。

又显然有$\langle S \rangle \subseteq \opn{ker}h$,从而$\opn{ker}h=\langle S \rangle$。由群同态基本定理(\autoref{exe_Group2_1}~\upref{Group2})即可得证。

\textbf{证毕}。



















