% 无穷 Galois 扩张与 Krull 定理
% keys 域论|域扩张|Galois群|伽罗华|伽罗瓦|Krull拓扑|Krull定理|无限Galois扩张|伽罗华扩张
% license Xiao
% type Tutor

% \addTODO{}

\pentry{Galois扩张\nref{nod_GExt},连通性\nref{nod_Topo3},分离性\nref{nod_Topo5}}{nod_db35}

\enref{Galois扩张}{GExt}中除了Galois扩张和Galois群的基本性质,剩下的重点内容全是\textbf{有限}Galois扩张的情况,见\autoref{sub_GExt_1}。作为提醒,再总结一次:有限Galois扩张都是单代数扩张,且为分裂域。

本节介绍的是无限Galois扩张中的性质,将有限扩张的\textbf{Galois理论基本定理}(\autoref{the_GExt_10})进行拓展,得到\textbf{Krull}定理。Krull的工作亮点,在于给Galois群赋予了一个拓扑结构。


为了得到Krull拓扑,我们要先观察Galois扩域的一些性质。注意,接下来我们不再限定为有限扩张了。


由\autoref{the_GExt_6},$\mathbb{K}/\mathbb{M}$是Galois扩张,因此$\opn{Gal}(\mathbb{K}/\mathbb{M})$存在。有了这一点,我们就可以讨论下面两条引理:

\begin{lemma}{}\label{lem_GExInf_2}
设$\mathbb{K}/\mathbb{F}$是Galois扩张,且存在中间域$\mathbb{M}$。

则$\mathbb{M}/\mathbb{F}$是Galois扩张 $\iff$ $\mathbb{M}/\mathbb{F}$是正规扩张 $\iff$ $\opn{Gal}(\mathbb{K}/\mathbb{M})\vartriangleleft \opn{Gal}(\mathbb{K}/\mathbb{F})$ 。
\end{lemma}

\autoref{lem_GExInf_2} 实际上就是\autoref{the_GExt_8},因此证明参见该定理。


\begin{lemma}{}\label{lem_GExInf_1}
设$\mathbb{K}/\mathbb{F}$是Galois扩张,且存在中间域$\mathbb{M}$。则$[\opn{Gal}(\mathbb{K}/\mathbb{F}):\opn{Gal}(\mathbb{K}/\mathbb{M})]=[\mathbb{M}:\mathbb{F}]$。


\end{lemma}


\textbf{证明}:

取$f, g\in\opn{Gal}(\mathbb{K}/\mathbb{F})$,则$f$和$g$模$\opn{Gal}(\mathbb{K}/\mathbb{M})$同余\textbf{当且仅当}$f^{-1}g\in\opn{Gal}(\mathbb{K}/\mathbb{M})$,或者说$f$和$g$限制在$\mathbb{M}$上是相同的。

因此,$\opn{Gal}(\mathbb{K}/\mathbb{M})$的每个左陪集对应一个$\mathbb{M}/\mathbb{F}$的映射。

\textbf{证毕}。


这条引理很像\autoref{the_GExt_9},只不过不再要求是有限扩张了。这显得\autoref{the_GExt_9} 似乎没有存在的必要,然而我们依然将它保留了,体现“有限Galois扩张就是单扩张”的思路。




\begin{theorem}{}\label{the_GExInf_1}
设$\mathbb{K}/\mathbb{F}$是Galois扩域,$\mathcal{M}=\{\opn{Gal}(\mathbb{K}/\mathbb{M})\mid \mathbb{M}/\mathbb{F}\text{是有限扩张}\}$\footnote{注意,由\autoref{the_GExt_6},$\mathbb{K}/\mathbb{M}$必是Galois扩张。但$\mathbb{M}/\mathbb{F}$则不一定,取决于它是否正规。}。则有:

1。$\forall H\in\mathcal{M}$,$[\opn{Gal}(\mathbb{K}/\mathbb{F}): H]$\footnote{即群指数,子群$H$在$\opn{Gal}(\mathbb{E}/\mathbb{K})$中陪集的数量。}是有限的;

2。$\bigcap_{H\in\mathcal{M}}=\{e\}$。这里$e$是群的单位元,即$\mathbb{K}$上的恒等映射$\opn{id}_{\mathbb{K}}$。

3。$\forall H_1, H_2\in\mathcal{M}$,有$H_1\cap H_2\in\mathcal{M}$;

4。$\forall H\in\mathcal{M}$,$\exists N\vartriangleleft\opn{Gal}(\mathbb{K}/\mathbb{F})$,且$N\subseteq H$;

\end{theorem}

\textbf{证明}:

1。

由\autoref{lem_GExInf_1} 直接可得。

2。

任取$\sigma\in\opn{Gal}(\mathbb{K}/\mathbb{F})$,只要$\sigma\not=\opn{id}_{\mathbb{K}}$,那么就存在$\alpha\in\mathbb{K}-\mathbb{F}$使得$\sigma(\alpha)\neq\alpha$。取$\mathbb{M}=\mathbb{F}(\alpha)$,则$\sigma\not\in\opn{Gal}(\mathbb{K}/\mathbb{F})$。故$\sigma\not\in \bigcap_{H\in\mathcal{M}}$。

3。

由\autoref{lem_GExt_1},$H_1\cap H_2$是$\mathbb{M}_1\mathbb{M}_2$的Galois群。

有限域的合成可以看成$\mathbb{M}_1$用$\mathbb{M}_2$的元素反复进行有限次单扩张的结果。由于是Galois扩张,故这些单扩张全都是代数扩张,从而是有限扩张,从而$\mathbb{M}_1\mathbb{M}_2/\mathbb{F}$是有限扩张。

4。

由\autoref{lem_GExInf_2},只需要证明存在$\mathbb{M}'$,使得$\mathbb{M}'\supseteq\mathbb{M}$且$\mathbb{M}'/\mathbb{F}$是正规扩张。

取$\mathbb{M}'$为$\mathbb{M}$关于$\mathbb{F}$的所有共轭域之合成即可。

\textbf{证毕}。

你可能会想到,\autoref{the_GExInf_1} 的第4条完全可以取$N=\{e\}$来证明,也就是取$\mathbb{M}'=\mathbb{K}$,这就导致情况过于平凡,似乎定理第4条没有存在的必要。但我们实际采用的证明过程说明非平凡的情况也是存在的。





\subsection{Krull 定理}\label{sub_GExInf_1}


\subsubsection{Krull拓扑}

考虑任意集合$X$和$Y$,令$M\subseteq X^Y$。任取$f\in M$,以及$X$的\textbf{有限}子集$S$,令
\begin{equation}
V(f, S) = \{g\in M\mid g(s)=f(s), \forall s\in S\}~,
\end{equation}
即$V(f, S)$是全体属于$M$且限制在$S$上与$f$相等的映射的集合。

任取$h\in V(f, S)\cap V(g, T)$,则易得$V(f, S)\cap V(g, T) = V(h, S\cup T)$\footnote{这是因为,任取$V(f, S)\cap V(g, T)$中的元素$c$,则$c$和$f$在$S$上相等,故和$h$在$S$上相等;同理可得$c$和$h$在$T$上相等,从而由$c$的任意性知,$V(f, S)\cap V(g, T)\subseteq V(h, S\cup T)$。反过来,任取$c\in V(h, S\cup T)$,则也可以推知$c$和$f$在$S$上相等、和$g$在$T$上相等,从而$V(h, S\cup T)\subseteq V(f, S)\cap V(g, T)$。}。于是,全体$V(f, S)$的集合对于有限交封闭,从而据\autoref{sub_Topol_1} 的讨论知,全体$V(f, S)$的集合是一个\textbf{拓扑基}(\autoref{def_Topol_2})。

这样全体$V(f, S)$的集合就定义了$M$上的一个拓扑,称之为$M$上的\textbf{有限拓扑(finite topology)}。



\begin{theorem}{}\label{the_GExInf_2}
考虑Galois扩张$\mathbb{K}/\mathbb{F}$。令
\begin{equation}\label{eq_GExInf_1}
\begin{aligned}
\mathcal{N} &= \{\sigma N\mid N\in\mathcal{M}, \sigma\in\opn{Gal}(\mathbb{K}/\mathbb{F}), N\vartriangleleft\opn{Gal}(\mathbb{K}/\mathbb{F})\}\\
\mathcal{L} &= \{\sigma H\mid H\in\mathcal{M}, \sigma\in\opn{Gal}(\mathbb{K}/\mathbb{F})\}\\
\mathcal{R} &= \{H\sigma \mid H\in\mathcal{M}, \sigma\in\opn{Gal}(\mathbb{K}/\mathbb{F})\}~,
\end{aligned}
\end{equation}
其中$\mathcal{M}d$的定义见\autoref{the_GExInf_1}。

则$\mathcal{N}$、$\mathcal{L}$和$\mathcal{R}$都是$\opn{Gal}(\mathbb{K}/\mathbb{F})$上\textbf{有限拓扑}的拓扑基。
\end{theorem}

\textbf{证明}:

只需证明$\mathcal{N}$的情况即可,因为$\mathcal{N}\subseteq \mathcal{L}\cap\mathcal{R}$。

任取$\mathbb{K}/\mathbb{M}$的中间域$\mathbb{M}$,使得$\mathbb{M}/\mathbb{F}$是有限扩张。考虑$\mathcal{M}$的定义,以及“有限可分扩张都是单扩张”(\autoref{cor_PrmtEl_2}),可知存在$\alpha$使得$\mathbb{M}=\mathbb{F}(\alpha)$。因此,如果$\sigma_i\in\opn{Gal}(\mathbb{K}/\mathbb{F})$使得$\sigma_1(\alpha)=\sigma_2(\alpha)$,那么$\sigma_1\sigma_2^{-1}\in\opn{Gal}(\mathbb{K}/\mathbb{M})$。

换句话说,$\mathbb{K}$的两个自同构模$\opn{Gal}(\mathbb{K}/\mathbb{M})$同余(在$\opn{Gal}(\mathbb{K}/\mathbb{M})$的同一个左陪集里),当且仅当它们把$\alpha$映射为同一个元素。于是,取$\mathbb{K}$的\textbf{有限子集}$\{\alpha\}$,有
\begin{equation}\label{eq_GExInf_2}
V(\sigma, \{\alpha\}) = \sigma \opn{Gal}(\mathbb{K}/\mathbb{M})~.
\end{equation}

因此,$\mathcal{N}$、$\mathcal{L}$和$\mathcal{R}$都是$\opn{Gal}(\mathbb{K}/\mathbb{F})$上\textbf{有限拓扑}的\textbf{子族}。

下证任意$V(\sigma, S)$总能由$\mathcal{N}$求并得到。

任取$\sigma\in\opn{Gal}(\mathbb{K}/\mathbb{F})$,以及$\mathbb{K}$的有限子集$S$。则$\mathbb{F}(S)/\mathbb{F}$是有限扩张。由\autoref{the_GExInf_1} 第4条,可以取$\mathbb{F}(S)$关于$\mathbb{F}$的全体共轭域之合成$\mathbb{M}\subseteq\mathbb{F}(S)$,使得$\mathbb{K}/\mathbb{M}$是Galois扩张。同样由于“有限可分扩张都是单扩张”,存在$\alpha$使得$\mathbb{M}=\mathbb{F}(\alpha)$。

于是,$V(\sigma, S)=V(\sigma, \{\alpha\})=\sigma \opn{Gal}(\mathbb{K}/\mathbb{M})$。

\textbf{证毕}。

\begin{definition}{Krull拓扑}\label{def_GExInf_1}
\autoref{the_GExInf_1} 中所描述的$\opn{Gal}(\mathbb{K}/\mathbb{F})$上的拓扑,又被称为\textbf{Krull 拓扑}。
\end{definition}

任取$H\in\mathcal{M}$,根据\autoref{the_GExInf_2},它是个开集。同时它的补集$H^C$是各$\sigma H$的并,其中$\sigma\not\in H$,而$\sigma H$也都是开集,因此$H^C$是开集——于是$H$还是闭集。同理,$\sigma H$也是既开又闭的。

如果$\opn{Gal}(\mathbb{K}/\mathbb{F})$本身就是有限群,那么可以类比\autoref{eq_GExInf_2} 的原理,推知Krull拓扑是个离散拓扑。


\begin{theorem}{}
设$\mathbb{K}/\mathbb{F}$是Galois扩张,则$\opn{Gal}(\mathbb{K}/\mathbb{F})$的Krull拓扑是紧致的、Hausdorff的和完全不连通的。
\end{theorem}

\textbf{证明}:

任取$\sigma_i\in\opn{Gal}(\mathbb{K}/\mathbb{F})$,且$\sigma_1\neq \sigma_2$。则存在$\alpha\in\mathbb{K}-\mathbb{F}$,使得$\sigma_1(\alpha)\neq\sigma_2(\alpha)$。于是取开集$V(\sigma_i, \{\alpha\})$,即可分离这两个点$\sigma_i$,得证Hausdorff性。

令$H=\opn{Fix}_\mathbb{K}(\mathbb{F}(\alpha))$,则$\sigma_1\in\sigma_1H$。由于$\sigma_1^{-1}\sigma_2(\alpha)\neq \alpha$,故$\sigma_1H\neq\sigma_2H$。而我们已经知道,$\sigma_iH$是既开又闭的,也就是连通分支。由于$\sigma_i$的任意性,这意味着任意两个不同的自同构总在不同的连通分支中,故任意连通分支只有一个元素,得证完全不连通性。

\addTODO{紧致性有点复杂,之后补充。可能需要用到Tychonoff定理。}%GTM242 Proposition 4.4, P202

\textbf{证毕}。





\subsubsection{Krull定理}

























