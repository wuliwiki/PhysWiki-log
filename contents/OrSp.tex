% 正交空间与辛空间
% license Usr
% type Tutor

\pentry{双线性函数\nref{nod_BiFunc}}{nod_eae4}
\begin{definition}{}
设$V$是域$\mathbb F$上的线性空间,$f$是$V$上的双线性函数,
\begin{itemize}
\item 若$f$是\textbf{对称的},称$(V,f)$是\textbf{正交空间};
\item 若$f$是\textbf{反对称的},称$(V,f)$是\textbf{辛空间};
\end{itemize}
$f$则为\textbf{广义内积},$(V,f)$为广义内积空间。对任意$\bvec x,\bvec y\in V$,简记广义内积为$(\bvec x,\bvec y)\equiv f(\bvec x,\bvec y)$
\end{definition}
引入广义内积的概念后,向量关系不再是直观的几何关系。在欧几里得空间下,由于内积是正定对称双线性型,我们可以定义两个点之间的距离,向量长度和角度。然而在广义内积下,这些概念无法再定义。譬如一般定义长度$l_{\bvec x}=\sqrt{\bvec x^2}\ge 0$,但在辛空间内向量内积都为0,不再具有区分性,所以没有这个概念。
同理,正交性不再由角度定义,而是采取内积为$0$的定义。由于正交关系是对称的,左根和右根等同,统称为根,以$\opn{Rad}$表示。

需要注意的是,在平面几何里,两向量正交则线性无关。但在广义内积空间内并不一定成立,比如退化的正交空间内总有与自身内积为0的向量,因此“正交基”要严格保证既是线性无关也是相互正交。

一般称与自身正交为$0$的向量为\textbf{迷向向量}。

\begin{definition}{}
设$V$是域$F$的内积空间,$S,T$为其子空间。若对于任意$\bvec x\in S,\bvec  y\in T$有$(\bvec x.\bvec y)$,则称这两个空间互相垂直。若任意与$S$垂直的向量都在$T$内,则可以表示$T=S^{\bot}$。
\end{definition}
\begin{theorem}{}
设特征不为2的域$F$上有$n$维正交空间$(V,f)$,则该正交空间一定存在正交基。
\end{theorem}
\textbf{证明:}

由于$f$是对称的,则度量矩阵是对称矩阵,设为$A$。$A$可以通过正交矩阵来对角化,每一列是$A$的两两正交的特征向量。

若$V$有非退化子空间,则该子空间上的正交基可以扩展为全空间的正交基。或者说:

\begin{theorem}{}
$(V,f)$是正交空间,A是$f$的度量矩阵表示,则有子空间$W$非退化$\Longleftrightarrow  W^{\bot}$非退化$\Longleftrightarrow A|_{W}$满秩$\Longleftrightarrow V=W\oplus W^{\bot} $
\end{theorem}
\textbf{证明:}

$W$非退化$\Longleftrightarrow \opn{Rad}W=W\cap W^{\bot}=0\Longleftrightarrow  W^{\bot}$非退化。前两条得证。

设任意$\bvec x\in \opn{Rad} W,\bvec y\in W$,$W$的退化性意味着$(\bvec x^{T}A|_{W})\bvec  y=0$只有$\bvec 0$解,因此$A|_{W}$是单射,满秩。

接下来证明最后一条。$W\cap W^{\bot}=0$,只需证明$V=W+W^{\bot}$即可。正交空间必有正交基,设$\{\bvec e_i\}_{i=1}^k$是$W$上的一组由非迷向向量组成的正交基,任意$\bvec z\in V$,$\bvec z'$为任意向量在$W$上的分量。剩余分量为
\begin{equation}
\bvec z-\bvec z'=\bvec z-\sum\limits_{i=1}^k\frac{f(\bvec z,\bvec e_i)}{f(\bvec e_i,\bvec e_i)}\bvec e_i~.
\end{equation}
若剩余分量属于$W$的正交补,则必然与$W$上的基向量内积为$0$。验证有:
\begin{equation}
f(\bvec z-\bvec z',\bvec e_j)=f(\bvec z,\bvec e_j)-\sum\limits_{i=1}^k\frac{f(\bvec z,\bvec e_i)}{f(\bvec e_i,\bvec e_i)}f(\bvec e_i,\bvec e_j)=0~,
\end{equation}
得证。

实际上,该定理也适用于辛空间:
\begin{theorem}{}
$(V,f)$是有限维辛空间,$W$是其非退化的子空间,则$V=W\oplus W^{\bot}$
\end{theorem}
\textbf{证明:}

由于$W$非退化,所以$\opn{Rad} W=W\cap W^{\bot}={\bvec 0}$且$W$是偶数维的辛空间。设$\{\bvec \theta_i,\bvec \theta_{-i}\}_{i=1}^k$是$W$的一组基并满足$f(\bvec \theta_i,\bvec\theta_{-i})=1$,若$i\neq t$则有$f(\bvec\theta_i,\bvec\theta_{\pm t})= 0$\footnote{有限维辛空间总可以找到这么一组基。}。扩展其为$V$上的一组基:$\{\bvec \theta_i,\bvec \theta_{-i}\}_{i=1}^k\cup\{\bvec \eta_i\}_{i=1}^{t}$。

设$\bvec \eta'_i=\bvec\eta_i-\sum\limits_{t=1}^kf(\bvec \eta_i,\bvec \theta_{-t})\bvec\theta_t+\sum\limits_{t=1}^kf(\bvec\eta_i,\bvec\theta_t)\bvec\theta_{-t}$,则有:
\begin{equation}
\begin{aligned}
f(\bvec \eta'_j,\bvec  \theta_i)&=f(\bvec \eta_j,\bvec  \theta_i)-f(\bvec\eta_j,\bvec  \theta_{-i})\bvec f(\bvec \theta_i,\bvec \theta_i)+f(\bvec\eta_j,\bvec \theta_i)\bvec f(\bvec \theta_{-i},\bvec \theta_{i})=0\\
f(\bvec \eta'_j,\bvec  \theta_{-i})&=f(\bvec \eta_j,\bvec  \theta_{-i})-f(\bvec\eta_j,\bvec  \theta_{-i})\bvec f(\bvec \theta_i,\bvec \theta_{-i})+f(\bvec\eta_j,\bvec \theta_i)\bvec f(\bvec \theta_{-i},\bvec \theta_{-i})=0~.
\end{aligned}
\end{equation}
因而$\{\bvec \eta'_i\}_{i=1}^t\in W^{\bot}$,可以证明$\{\bvec \theta_i,\bvec \theta_{-i}\}_{i=1}^k\cup\{\bvec \eta'_i\}_{i=1}^t$是$V$上的一组基,得证。


由\autoref{the_BiFunc_1}~\upref{BiFunc}可知,总可以找到一组基使得辛空间的度量矩阵为:
\begin{equation}
\operatorname{diag}\left\{\left(\begin{array}{rr}
0 & 1 \\
-1 & 0
\end{array}\right), \cdots,\left(\begin{array}{rr}
0 & 1 \\
-1 & 0
\end{array}\right), 0, \cdots, 0\right\} \text {, }
~.\end{equation}
这组基又称为\textbf{辛基}。改变基的顺序,从满足上述形式的${\bvec \theta_1,\bvec\theta_{-1},\bvec\theta_2,\bvec\theta_{-2}...\bvec \theta_k,\bvec\theta_{-k}...\bvec\eta_1...\bvec\eta_t}$变为${\bvec \theta_1,\bvec\theta_2,\bvec \theta_k,\bvec\theta_{-k}...\bvec\theta_{-1}\bvec\theta_{-2}\bvec\theta_{-k}...\bvec\eta_1...\bvec\eta_t}$。则可以证明在这组基下度量矩阵的形式为:
\begin{equation}
\begin{pmatrix}
 0 & \boldsymbol E_k & 0\\
  -\boldsymbol E_k&  0& 0\\
 0&  0&0
\end{pmatrix}~.
\end{equation}