% 北京大学 2011 年 考研 普通物理
% license Usr
% type Note

\textbf{声明}:“该内容来源于网络公开资料,不保证真实性,如有侵权请联系管理员”

\subsection{力学:}
第一题

地面上有一斜面木块 $3m$,斜面上有一小木块",地面与斜面之间无摩擦力,斜面与小木块之间无摩擦力,一外力$F$推斜面竖直面,是小木块和斜面保持相对静止,求$F$

第二题

一台阶上有一竖直细杆,细杆刚好竖直放在台阶一端,台阶平面距离地面高为日,细杆长为$L$,质量为M,现有一质点$m$。速度为$V$,以平行于台阶的方向,垂直射入细杆距离台阶$H$处的一端,并粘覆其上,然后一起向地面运动,当运动到地面时,恰好转过360度,即此时细杆到达地面且呈竖直状态,求质点初速$V$

第三题地球有一卫星$MV$,其近地点和远地点距地心分别为$r1$和$r2$,求卫星最大速度
\subsection{电磁学:}
第四题

无限大空间中,一半为真空,一般为介质,介质相对介电常数为$e$(常用的我打不出来,用这个代替一下),分界面上有一点电荷$Q$,距离$Q$为$d$的界面上有一点电荷$q$,求$q$受到的静电力

第五题

求无限长圆柱面电流$j$和无限大面电流$j$共同在空间产生的磁场分布,圆面电流沿圆周方向,面电流沿圆柱母线方向

第六题

一斜双轨置于均匀磁场$B$中,磁场方向竖直向上,斜双轨与地面夹角为$a$,双轨距离为$L$,铁杆位于双轨之上,由静止向下滑行,求细杆在以下两种情况下的运动情况,(1)双轨,铁杆,与一些导线,和一电阻$R$著称回路(2)双轨,铁杆,与一些导线,和一电容$C$构成回路.
\subsection{光学:}
第七题

使用一种激光向月球照射,2.5秒接到反射光(1)在月球上,地球发出波长为$633nm$的光,问在地面多大的范围内可视为平面光(2)使用一种激光器,从口径为$2mm$的出口向月球发出波长为$633nm$的激光,在欲求形成多大光斑
(3)使用上述激光器测量地月距离合不合理,若不合理怎么改进

第八题

是关于光栅相干涉的问题,具体记不清了。时求两种波长的光波垂直入射时第二级的长波与第三级的短波是否发生重何等
第九题。这道题我没做。当时没时间了。是求$O$光和$E$光方面的。