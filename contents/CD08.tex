% 四川大学 2008 年硕士物理考试试题(933)
% license Usr
% type Tutor

\textbf{声明}:“该内容来源于网络公开资料,不保证真实性,如有侵权请联系管理员”

\subsection{计算题.(90分)}
\begin{enumerate}
\item (10分)把一个电荷为$q$的粒子从无穷远处被移到一个半径为$R$厚度为$t$的空心导体球壳中心(此电荷通过球亮上一个小孔移入),在此过程中需做多少功?
\item ( 12分)有一内径为$a$、外径为b的空气柱形电容器,电容器的长度较$(b-a)$大得多,在电容器一端两极之间加上直流电压$U$,另一端两极之间接上负载电阻$R$,忽略电容器极板的电阻,求电容器中能流密度矢量的分布。
\begin{figure}[ht]
\centering
\includegraphics[width=6cm]{./figures/b738f2e85b50d098.png}
\caption{} \label{fig_CD08_1}
\end{figure}
\item (10分)两根无限长导线互相平行,间距为$2a$,均载有电流为$I$(方向如图)。在两导线组成的平面内,有一边长为$2a$的等边三角形,等边三角形的一条边垂直于导线,且与导线绝缘。求导线与该等边三角形之间的互感系数。
\item (10分)一电荷为$q$的点电荷,以匀角速度$\omega$作圆周运动,圆周的半径为$R$.设$t=0$时$q$所在点的坐标为$x_0=R,y_0=0$以$\vec{i},\vec{j}$分别表示$x$轴和$y$轴上的单位矢量,求圆心处的位移电流密度$\vec{j}$.
\begin{figure}[ht]
\centering
\includegraphics[width=6cm]{./figures/a872c2c89ea1f508.png}
\caption{} \label{fig_CD08_2}
\end{figure}
\item (12 分)一块透明片的振幅透射率为$t(x)=\exp(-\pi x^2)$(高斯分布),将其置于透镜的前焦面上,并用单位振幅的单色光垂直照明,求透镜后焦面上的振幅分布。
\[
\left( \int_0^{\infty} e^{-a^2 x^2} \, dx = \frac{\sqrt{\pi}}{2a} \right)~
\]
\item (12 分)平行光照明如图所示衍射屏,图中标出的是该处到场点的光程$r_0$是中心到场点的光程,用矢量图解法求轴上场点的光强比自由传播时小多少倍(忽略距离和倾斜因子对振幅的影响)?
\begin{figure}[ht]
\centering
\includegraphics[width=6cm]{./figures/d85f7110cd03e6da.png}
\caption{} \label{fig_CD08_3}
\end{figure}
\end{enumerate}