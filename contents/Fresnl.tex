% 菲涅尔公式、布儒斯特角、临界角、内反射与外反射
% keys 菲涅尔|折射|反射|偏振|布儒斯特角
% license Xiao
% type Tutor

\begin{issues}
\issueTODO
\end{issues}

\pentry{麦克斯韦方程组(介质)\nref{nod_MWEq1}}{nod_907f}

\subsection{菲涅尔公式}
\begin{figure}[ht]
\centering
\includegraphics[width=14cm]{./figures/630b71d6f16d0f65.pdf}
\caption{菲涅尔公式} \label{fig_Fresnl_1}
\end{figure}
 
利用具体的电磁场的边界条件\upref{mbdy} % \addTODO{链接}
\begin{itemize}
\item $\div \bvec D = 0$ 和$\div \bvec B = 0$  分别对应 $\epsilon \bvec E_\bot = \epsilon' \bvec E'_\bot$ 和 $\epsilon \bvec B_\bot = \epsilon' \bvec B'_\bot$。

\item $\div \bvec E = 0$ 和 $\div \bvec H = 0$ 分别对应 $\bvec E_{//} = \bvec E'_{//}$ 和 $\bvec B_{//}/\mu = \bvec B'_{//}/\mu'$。
\end{itemize}

\begin{theorem}{菲涅尔定律}
现在分两种情况讨论
\begin{enumerate}
\item 极化方向垂直于入射面(\autoref{fig_Fresnl_1} 右)

\begin{equation}
\frac{E_R^{(s)}}{E_I^{(s)}} =  \frac{m_1\cos{\theta_i} - m_2\cos\theta_t}{m_1\cos\theta_i + m_2\cos\theta_t}~,
\qquad
\frac{E_T^{(s)}}{E_I^{(s)}} = \frac{2 m_1\cos\theta_i}{m_1\cos\theta_i + m_2\cos\theta_t}~.
\end{equation}

\item 极化方向平行于入射面(\autoref{fig_Fresnl_1} 左)
\begin{equation}\label{eq_Fresnl_2}
\frac{E_R^{(p)}}{E_I^{(p)}} =  \frac{m_2\cos\theta_i - m_1\cos\theta_t}{m_2 \cos\theta_i + m_1\cos\theta_t}~,
\qquad
\frac{E_T^{(p)}}{E_I^{(p)}} =  \frac{2 m_1\cos\theta_i}{m_2\cos\theta_i + m_1\cos\theta_t}~.
\end{equation}
\end{enumerate}
其中 $m_i=n_i/\mu_i = c\sqrt{\epsilon_i/\mu_i}$ ($i=1,2$)。这两个表达式是菲涅尔方程中的两个,它们是极其普遍的陈述,适用于任何线性、各向同性的均匀介质。

另外注意菲涅尔公式包含相位信息,即以上的 $E$ 可以是复振幅$\tilde E=E \E^{\I \varphi_0}$。\upref{VcPlWv} \upref{PWave}
\end{theorem}

% \addTODO{均匀介质的定义}
一般情况下介质的磁导率与真空区磁导率的区别可忽略:$\mu_1\approx\mu_2\approx\mu_0=1$。此时$m_i=n_i/\mu_i \approx n_i$, 菲涅尔公式可以简化为:

\begin{equation}\label{eq_Fresnl_3}
r_s \equiv \left(\frac{E_R}{E_I}\right)_s = \frac{n_1\cos{\theta_i} - n_2\cos\theta_t}{n_1\cos\theta_i + n_2\cos\theta_t}~,
\end{equation}
\begin{equation}\label{eq_Fresnl_4}
t_s \equiv \left(\frac{E_T}{E_I}\right)_s =  \frac{2 n_1\cos\theta_i}{n_1\cos\theta_i + n_2\cos\theta_t}~.
\end{equation}
其中,$r_s$表示\textbf{振幅反射系数},$t_s$表示\textbf{振幅透射系数}。
同理:
\begin{equation}\label{eq_Fresnl_5}
r_p \equiv \left(\frac{E_R}{E_I}\right)_p = \frac{n_2\cos{\theta_i} - n_1\cos\theta_t}{n_1\cos\theta_t + n_2\cos\theta_i}~,
\end{equation}
\begin{equation}\label{eq_Fresnl_6}
t_p \equiv \left(\frac{E_T}{E_I}\right)_p =  \frac{2 n_1\cos\theta_i}{n_1\cos\theta_t + n_2\cos\theta_i}~.
\end{equation}

应用斯涅尔定律\upref{Snel},可以进一步使记号简化。别害怕,它们只是看起来有些复杂:
\begin{equation}\label{eq_Fresnl_7}
r_s = -\frac{\sin(\theta_i - \theta_t)}{\sin(\theta_i + \theta_t)}~,
\end{equation}
\begin{equation}\label{eq_Fresnl_8}
r_p = +\frac{tan(\theta_i - \theta_t)}{tan(\theta_i + \theta_t)}~,
\end{equation}
\begin{equation}\label{eq_Fresnl_9}
t_s = -\frac{2\sin\theta_t\cos\theta_i}{\sin(\theta_i + \theta_t)}~,
\end{equation}
\begin{equation}\label{eq_Fresnl_10}
t_p = +\frac{2\sin\theta_t\cos\theta_i}{\sin(\theta_i + \theta_t)\cos(\theta_i - \theta_t)}~.
\end{equation}
这里请大家注意,推导菲涅尔方程时,场的方向(更精确地说是相位)是相当任意地选择的。因此,为了避免混乱,必须把菲涅尔方程与导出它们的特定的场的方向联系起来。

\subsection{内反射与外反射}
\begin{definition}{外反射}
设入射介质与出射介质的折射系数分别为 $n_1$ 和 $n_2$,若 $n_1<n_2$,则称为\textbf{外反射(external reflection)};反之,为\textbf{内反射(internal reflection)}。
\end{definition}

\subsection{布儒斯特角}
\begin{figure}[ht]
\centering
\includegraphics[width=8cm]{./figures/2afaccd613ed159e.pdf}
\caption{布儒斯特角示意图} \label{fig_Fresnl_2}
\end{figure}
我们这里考虑常见的 $n_2>n_1$ 且 $\mu_1 = \mu_2$ 情况。由\autoref{eq_Fresnl_8} 容易证明当 $\theta_i + \theta_t = \pi/2$时, $r_p = 0$, 反射光的平行分量消失,反射光为线偏振光。 此时,入射角称为\textbf{布儒斯特角(Brewster's angle)} ,记为 $\theta_B$。 代入斯涅尔公式 $n_i\sin\theta_i = n_t\sin\theta_t$可得
\begin{equation}
\theta_B = \arctan (n_2/n_1)~.
\end{equation}

\subsection{临界角}
\begin{figure}[ht]
\centering
\includegraphics[width=8cm]{./figures/7dbed095f72d505b.pdf}
\caption{临界角示意图} \label{fig_Fresnl_3}
\end{figure}
\begin{definition}{}
当 $\theta_t  = \frac{\pi}{2}$ 时,此时的入射角 $\theta_i$ 称为 \textbf{临界角(critical angle)},记为 $\theta_c$。
\end{definition}

\begin{exercise}{}
类比 $\theta_B$ 的求法,给出 $\theta_c$ 表达式。
\end{exercise}

易知, $\theta_c = \arcsin(n_2/n_1)~.$
% \addTODO{画图!}
\addTODO{线偏振光} 


