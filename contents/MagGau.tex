% 磁场的高斯定律
% 磁场|磁感应强度|散度|高斯定律

\begin{issues}
\issueDraft
\end{issues}

\pentry{电场的高斯定律\upref{EGauss}}

\footnote{本文参考 Wikipedia \href{https://en.wikipedia.org/wiki/Gauss's_law_for_magnetism}{相关页面}.}\textbf{磁场的高斯定律(Gauss's law for magnetism)}是麦克斯韦方程组\upref{MWEq}中的一条方程.

磁场的高斯定律: 对于任意磁场 $\bvec B(\bvec r)$ 和任意闭合曲面, 曲面上的磁通量\upref{BFlux}为零.
\begin{equation}
\oint \bvec B(\bvec r) \vdot \dd{\bvec s} = 0
\end{equation}

也就是说空间任意一点的电场散度等于电荷密度除以真空介电常数.可以写成微分形式:
\begin{equation}
\nabla \cdot \bvec B = 0
\end{equation}

磁场的高斯定律实际上是电场的高斯定律\upref{EGauss}在磁学中的对应,它反映了自然界没有孤立的磁单极.形象地看,任意一条磁感线都不会起始或终止于空间中的某一点,它要么是闭合的回路,要么从无穷远来延伸到无穷远去.正因为磁场的这条性质,我们可以将磁感应强度 $\bvec B$ 写成某个矢量场 $\bvec A$ 的旋度,其中 $\bvec A$ 称为矢量势(矢势)\upref{BvecA}.