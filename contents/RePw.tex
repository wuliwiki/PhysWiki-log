% 幂函数(数学分析)
% license Xiao
% type Tutor

\pentry{极限存在的判据、柯西序列\nref{nod_CauSeq}}{nod_525c}

在中学数学中, 我们已经学习过正实数的幂的定义。 按照此定义, 给定实数\upref{ReNum}$x>0$ 和任意实数 $\alpha,\beta$, 幂 $x^\alpha$ 满足如下性质:

\begin{itemize}
\item 对于任意的 $x>0$, 均有 $x^0=1$.
\item $x^{\alpha+\beta}=x^\alpha\cdot x^\beta$.
\item $(x^\alpha)^\beta=x^{\alpha\beta}$.
\item 给定 $x_1,x_2>0$, 那么 $x_1^\alpha\cdot x_2^\alpha=(x_1x_2)^\alpha$.
\item 如果 $x>1$, $\alpha<\beta$, 那么 $x^\alpha<x^\beta$.
\item 如果 $0<x_1<x_2$, $\alpha>0$, 那么 $x_1^\alpha<x_2^\alpha$.
\end{itemize}

特别地, 对于正整数 $n$, $x^n$ 就是将 $x$ 自乘 $n$ 次, $x^{-n}$ 就是 $1/x^n$, 而 $x^{1/n}$ 就是 $x$ 的 $n$ 次算术根。 

一个自然的问题是, 满足以上六条性质的运算是否唯一? 答案是肯定的。 严格的实数理论给出了一个构造幂的方法。

\subsection{有理数次幂的构造}

首先, 正实数的整数次幂可以通过乘法来直接定义。 它当然满足以上六个要求。 现在要由此出发来构造正实数的有理数次幂。

给定正整数 $n$. 根据实数理论, 我们可以严格地构造出任意正实数 $x$ 的 $n$ 次算术根, 也就是满足 $y^n=x$ 的唯一一个正实数 $y$. 由于正整数次幂的保序性质, 可见算术根若存在则必然只有唯一一个。 至于实际的构造, 则可以确定一个戴德金分割如下: 分割的下类 $L$ 包含所有非正的有理数, 以及满足 $l^n<x$ 的正有理数 $l$; 分割的上类 $R$ 包含所有满足 $r^n\geq x$ 的正有理数。 容易验证它满足戴德金分割的定义, 从而它确定了一个实数 $y$. 

为了说明 $y^n=x$, 只需要注意到, 实数 $y$ 是下类 $L_y$ 的上确界, 同时也是上类 $R_y$ 的下确界。 所以给定 $\varepsilon>0$, 都存在 $l\in L_y$ 和 $r\in R_y$ 使得
\[
y-\varepsilon<l<y~,\quad y\leq r<y+\varepsilon~.
\]
于是
\[ 
r^n-l^n<(r-l)(r^{n-1}+...+l^{n-1})
<2n(y+\varepsilon)^{n-1}\varepsilon~.
\]
但另一方面当然有 $l^n<y^n\leq r^n$ 和 $l^n<x\leq r^n$, 从而 $y^n$ 同 $x$ 的差可以小于任意预先指定的正数, 于是只能有 $y^n=x$.

由此出发, 就可以构造正实数的有理数次幂了。 给定 $x>0$, 正整数 $n$ 和任意整数 $m$, 定义 $x^{1/n}$ 为 $x$ 的 $n$ 次算术根, 而 $x^{m/n}=(x^{1/n})^m$. 由此得到的方幂满足开头所提到的六条性质。

\begin{exercise}{幂的连续性}
给定实数 $x>0$. 试证明: 如果 $r_k$ 是收敛到零的有理数序列, 那么序列 $x^{r_k}$ 的极限是1. 提示: 利用伯努利不等式的如下变形: 如果 $x>1$, $N$ 为正整数, 则
\[
0<x^{1/N}-1<\frac{x-1}{N}~.
\]
\end{exercise}

\subsection{实数次幂的构造}
有了有理数次幂作为基础, 就可以根据六条性质中的最后两条 (保序性) 来构造任意实数次幂了。 

给定 $x>0$ 和任意实数 $\alpha$. 记实数 $\alpha$ 的戴德金分割为 $L_\alpha|R_\alpha$. 于是可得到两个数集
\[
A=\{x^l:l\in L_\alpha\}~,\quad B=\{x^r:r\in R_\alpha\}~.
\]
由于前面已经定义了有理数次幂, 所以这两个数集都是良好定义的, 而且根据保序性质也可看出 $A$ 中的元素总是小于 $B$ 中的元素。 根据实数的完备性, 这两个数集之间存在着一个实数。 另一方面, 由于上类和下类之间的元素可以任意接近, 所以根据习题1., 可见数集 $A$ 和 $B$ 之间只能有一个实数。 我们定义这唯一的一个实数为 $x^\alpha$.

从这个定义出发, 可以验证幂的一切性质。 例如, 设实数 $\alpha,\beta$ 的戴德金分割分别为 $L_\alpha|R_\alpha$ 和 $L_\beta|R_\beta$, 那么实数 $\alpha+\beta$ 的戴德金分割就是 $(L_\alpha+L_\beta)|(R_\alpha+R_\beta)$. 根据有理数次幂的性质就能够得到
\[
x^{\alpha+\beta}=x^\alpha\cdot x^\beta~.
\]
幂的其余运算性质均可由类似的方法导出。

幂还具有如下的连续性:

\begin{theorem}{幂的连续性}
\begin{itemize}
\item 设正数序列 $x_k$ 收敛到正实数 $x$. 那么
\[
\lim_{k\to\infty}x_k^\alpha=x^\alpha~.
\]
\item 设实数序列 $\alpha_k$ 收敛到 $\alpha$. 那么
\[
\lim_{k\to\infty}x^{\alpha_k}=x^\alpha~.
\]
\end{itemize}
\end{theorem}

\begin{exercise}{}
证明上述定理。 提示: 对于第一条, 试说明对于 $y>x>1$, 成立
\[
0<y^\alpha-x^\alpha<x^\alpha\cdot\left(\left(\frac{y}{x}\right)^{[\alpha]+1}-1\right)~.
\]
这样一来, 当 $y$ 接近 $x$ 时, $y^\alpha$ 便接近 $x^\alpha$. 对于第二条, 参考习题 1.
\end{exercise}

