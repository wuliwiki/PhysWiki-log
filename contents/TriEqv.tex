% 三角恒等式
% keys 三角函数|三角恒等式|勾股定理
% license Xiao
% type Tutor

\pentry{三角函数(高中)\nref{nod_HsTrFu}}{nod_931f}

这里列出几个高中常见的三角函数恒等式。 以下用到的两个高中数学不常见的三角函数分别为 $\csc \alpha= 1/\sin \alpha$, $\sec \alpha = 1/\cos \alpha$, 分别读作 cosecant 和 secant。

\subsubsection{勾股定理}
\begin{equation}
\sin^2 \alpha + \cos^2 \alpha = 1~.
\end{equation}
等式两边同除 $\cos^2 \alpha$ 和 $\sin^2 \alpha$ 得
\begin{gather}
\label{eq_TriEqv_13}
\tan^2 \alpha + 1 = \sec^2 \alpha~,\\
1 + \cot^2\alpha = \csc^2\alpha~.
\end{gather}

\subsubsection{两角和公式}
\begin{gather}
\label{eq_TriEqv_1}
\sin(\alpha\pm \beta) = \sin \alpha\cos \beta \pm \cos \alpha\sin \beta~,\\
\label{eq_TriEqv_2}
\cos(\alpha\pm \beta) = \cos \alpha\cos \beta \mp \sin \alpha\sin \beta~,\\
\tan(\alpha\pm \beta) = \frac{\tan \alpha \pm \tan \beta}{1 \mp \tan \alpha \tan \beta}~.
\end{gather}

\subsubsection{二倍角公式}

令\autoref{eq_TriEqv_1} 中 $\beta=\alpha$ 取上号得
\begin{gather}
\sin 2\alpha = 2\sin \alpha\cos \alpha~,\\
\label{eq_TriEqv_4}
\cos 2\alpha = \cos^2 \alpha - \sin^2 \alpha~,\\
\tan 2\alpha = \frac{2\tan \alpha}{1 - \tan^2 \alpha}~.
\end{gather}

\subsubsection{降幂公式}
结合\autoref{eq_TriEqv_4} 和 $\sin^2 \alpha + \cos^2 \alpha = 1$ 可以得到
\begin{gather}
\sin^2 \alpha = \frac12 (1- \cos 2\alpha) \label{eq_TriEqv_5}~, \\
\cos^2 \alpha = \frac12 (1+\cos 2\alpha) \label{eq_TriEqv_6}~.
\end{gather}
由此可得半角公式
\begin{gather}
\sin\frac{ \alpha}{2} = \pm\sqrt{\frac{1-\cos \alpha}{2}}~,\\
\cos\frac{ \alpha}{2}= \pm\sqrt{\frac{1+\cos \alpha}{2}}~,\\
\tan\frac{ \alpha}{2} = \pm\sqrt{\frac{1-\cos \alpha}{1+\cos \alpha}} = \frac{\sin \alpha}{1+\cos \alpha} = \frac{1-\cos \alpha}{\sin \alpha}~.
\end{gather}
注意正负号的选择需要根据 $\alpha$ 所在的区间判断, 如果需要恒等式则两边取平方。

\subsubsection{和差化积公式}
\begin{gather}
\sin \alpha + \sin \beta = 2\sin\qty(\frac{\alpha + \beta}{2})\cos\qty(\frac{\alpha - \beta}{2})\label{eq_TriEqv_7}~,\\
\sin \alpha - \sin \beta = 2\sin\qty(\frac{\alpha - \beta}{2})\cos\qty(\frac{\alpha + \beta}{2})\label{eq_TriEqv_8}~,\\
\cos \alpha + \cos \beta = 2\cos\qty(\frac{\alpha+\beta}{2})\cos\qty(\frac{\alpha-\beta}{2})\label{eq_TriEqv_9}~,\\
\cos \alpha - \cos \beta = -2\sin\qty(\frac{\alpha+\beta}{2})\sin\qty(\frac{\alpha-\beta}{2})\label{eq_TriEqv_10}~.
\end{gather}

\subsubsection{积化和差公式}
根据上文的和差化积公式, 我们也可以直接写出积化和差公式
\begin{gather}
\label{eq_TriEqv_11}
\sin \alpha\sin \beta = \frac12 [\cos(\alpha - \beta) - \cos(\alpha + \beta)]~,\\
\label{eq_TriEqv_12}
\cos \alpha\cos \beta = \frac12 [\cos(\alpha + \beta) + \cos(\alpha - \beta)]~,\\
\label{eq_TriEqv_14}
\sin \alpha\cos \beta = \frac12 [\sin(\alpha + \beta) + \sin(\alpha - \beta)]~.
\end{gather}

\subsubsection{辅助角公式}
\begin{equation}
a\sin \alpha + b\cos \alpha = \sqrt{a^2+b^2}\sin(\alpha + \phi) \qquad \qty(\phi = \tan^{-1}\frac{b}{a})~.
\end{equation}

\subsection{证明}\label{sub_TriEqv_1}
\subsubsection{两角和公式}
\begin{figure}[ht]
\centering
\includegraphics[width=7cm]{./figures/79c816a338c20765.pdf}
\caption{两角和公式} \label{fig_TriEqv_2}
\end{figure}
如\autoref{fig_TriEqv_2}, 要证明\autoref{eq_TriEqv_1}, 令 $OB = 1$, 那么 $\sin(\alpha+\beta) = BD = AC + BE$, 而 $AC = OA \sin\alpha$, $OA = \cos\beta$; $BE = AB\cos\alpha$, $AB = \sin\beta$, 代入得 $\sin(\alpha+\beta) = \sin\alpha\cos\beta + \cos\alpha\sin\beta$。 注意当 $\alpha$ 或 $\beta$ 取其他任意值时, 重新画图同样可以证明该关系。 所以给 $\beta$ 取相反数, 就得到 $\sin(\alpha-\beta) = \sin\alpha\cos\beta - \cos\alpha\sin\beta$

要证明\autoref{eq_TriEqv_2}, $\cos(\alpha+\beta) = OD = OC - EA$, 而 $OC = OA\cos\alpha$, $OA = \cos\beta$; $EA = AB\sin\alpha$, $AB = \sin\beta$, 代入得 $\cos(\alpha+\beta) = \cos\alpha\cos\beta - \sin\alpha\sin\beta$, $\beta$ 取相反数得 $\cos(\alpha-\beta) = \cos\alpha\cos\beta + \sin\alpha\sin\beta$。 证毕。

\subsubsection{两角和公式(几何矢量)}
把以上过程用几何矢量\upref{GVec} 语言可以表达得更自然。 令 $\uvec x, \uvec y$ 分别是\autoref{fig_TriEqv_2} 直角坐标的单位矢量, $OA$ 方向的单位矢量为 $\uvec a$, $AB$ 方向的单位矢量为 $\uvec b$。 易得
\begin{gather}
\uvec a = \cos\alpha\ \uvec x + \sin\alpha\ \uvec y~,\\
\uvec b = -\sin\alpha\ \uvec x + \cos\alpha\ \uvec y~.
\end{gather}
同样以 $O$ 为原点, $\uvec a, \uvec b$ 可以看成 $x$-$y$ 直角坐标系旋转后的坐标系中的单位矢量。 令 $OB$ 矢量为 $\uvec u$, 那么 $\uvec u = \cos\beta\ \uvec a + \sin\beta\ \uvec b$, 把以上两式代入得
\begin{equation}
\uvec u = (\cos\alpha\cos\beta - \sin\alpha\sin\beta)\uvec x + (\sin\alpha\cos\beta + \cos\alpha\sin\beta)\uvec y~,
\end{equation}
这就同时证明了两个两角和公式。 证毕。

\subsubsection{和差化积}
以\autoref{eq_TriEqv_9} 为例, $\cos \alpha, \cos \beta$ 和 $\cos \alpha + \cos \beta$ 分别等于\autoref{fig_TriEqv_1} 中矢量 $\bvec A, \bvec B$ (令它们的模长为 1) 和 $\bvec A + \bvec B$ 在水平方向的投影长度, 而 $\bvec A + \bvec B$ 在水平方向的投影长度等为 $\abs{\bvec A + \bvec B}\cos[(\alpha+\beta)/2]$, 其中 $\abs{\bvec A + \bvec B} = 2\cos [(\beta-\alpha)/2]$, 代入可得\autoref{eq_TriEqv_9}。 利用 $\bvec A + \bvec B$ 在竖直方向的投影可得\autoref{eq_TriEqv_7}, 把\autoref{eq_TriEqv_7} 和\autoref{eq_TriEqv_9} 中的 $\beta$ 分别替换成 $-\beta$ 和 $\beta+\pi$ 可推导出\autoref{eq_TriEqv_8} 和\autoref{eq_TriEqv_10}。
\begin{figure}[ht]
\centering
\includegraphics[width=5.5cm]{./figures/da4b6ee15cdf94d1.pdf}
\caption{和差化积公式推导} \label{fig_TriEqv_1}
\end{figure}
\addTODO{图中 $x,y$ 改为 $\alpha,\beta$}
