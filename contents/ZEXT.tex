% 正则系综(综述)
% license CCBYSA3
% type Wiki

本文根据 CC-BY-SA 协议转载翻译自维基百科\href{https://en.wikipedia.org/wiki/Canonical_ensemble}{相关文章}。

在统计力学中,正则系综是表示与在固定温度下与热库处于热平衡的机械系统可能状态的统计系综。\(^\text{[1]}\)系统可以与热库交换能量,因此系统的状态将在总能量上有所不同。

正则系综的主要热力学变量,决定状态的概率分布,是绝对温度(符号:\(T\))。该系综通常还依赖于机械变量,如系统中的粒子数(符号:\(N\))和系统的体积(符号:\(V\)),这些变量都会影响系统内部状态的性质。具有这三个参数的系综,假设这些参数对于被认为是正则的系综是恒定的,有时被称为\(NVT\)系综。

正则系综为每一个不同的微观状态分配一个概率\(P\),其公式为以下指数形式:
\[
P = e^{(F - E) / (kT)},~
\]
其中\(E\)是微观状态的总能量,\(k\)是玻尔兹曼常数。

数值\(F\)是自由能(具体来说是赫尔姆霍兹自由能),并假设对于特定的系综,\(F \)是常数,才认为该系综是正则的。然而,如果选择不同的\(N\)、\(V\)、\( T\),概率和\(F\)将发生变化。自由能\(F\)承担两个角色:首先,它为概率分布提供了一个归一化因子(所有微观状态的概率总和必须为1);其次,许多重要的系综平均值可以直接通过函数\(F(N, V, T)\)计算。

对于同一概念,另一种等效的表述将概率写为
\[
P = \frac{1}{Z} e^{-E / (kT)},~
\]
其中使用了正则系综的分配函数
\[
Z = e^{-F / (kT)},~
\]
而不是自由能。下面的方程(以自由能为单位)可以通过简单的数学操作转化为正则系综分配函数的形式。

从历史上看,正则系综最早由玻尔兹曼于1884年在一篇相对不为人知的论文中描述(他称之为全像体)。\(^\text{[2]}\)该理论后来由吉布斯在1902年重新表述并进行了广泛研究。\(^\text{[1]}\)
\subsection{正则系综的适用性}  
正则系综是描述与热库处于热平衡的系统可能状态的系综(这一事实的推导可以在吉布斯的著作中找到\(^\text{[1]}\))。

正则系综适用于任何规模的系统;尽管需要假设热库非常大(即,取宏观极限),但系统本身可以是小型的或大型的。

系统必须是机械隔离的,以确保它不与热库以外的任何外部物体交换能量。\(^\text{[1]}\)一般来说,正则系综适用于直接与热库接触的系统,因为正是这种接触确保了平衡。在实际情况中,通常通过以下两种方式之一来证明正则系综的适用性:1)假设接触是机械弱的,或者 2)将热库连接的适当部分纳入所分析的系统,从而在系统内建模接触对系统的机械影响。

当总能量固定但系统的内部状态未知时,适当的描述不是正则系综,而是微正则系综。对于粒子数可变的系统(由于与粒子库的接触),正确的描述是巨正则系综。在涉及相互作用粒子系统的统计物理教材中,通常假设三种系综在热力学上是等价的:宏观量围绕其平均值的波动变得很小,并且随着粒子数趋向于无穷大,这些波动趋于消失。在这种极限下,称为热力学极限,平均约束实际上变为硬约束。系综等价性的假设可以追溯到吉布斯,并且已经在一些具有短程相互作用且受到少量宏观约束的物理系统模型中得到了验证。尽管许多教材仍然传达着系综等价性适用于所有物理系统的信息,但在过去几十年中,已经发现了一些物理系统的例子,在这些系统中,系综等价性被打破了。\(^\text{[3][4][5][6][7][8]}\)
\subsection{性质}  
\begin{itemize}
\item 唯一性:对于给定的物理系统和给定的温度,正则系综是唯一确定的,并且不依赖于任意选择,如坐标系的选择(经典力学)、基的选择(量子力学)或能量的零点选择。\(^\text{[1]}\)正则系综是唯一一个具有恒定的\( N \)、\( V \)和\( T \),并且能够再现基本热力学关系的系综。\(^\text{[9]}\)
\item 统计平衡(稳态):正则系综不会随着时间演化,尽管底层系统在不断运动。这是因为正则系综仅是系统某个守恒量(能量)的函数。\(^\text{[1]}\)
\item 与其他系统的热平衡:两个系统,每个系统都由具有相同温度的正则系综描述,若它们被热接触(注释1),则每个系统将保持相同的系综,结果组合系统由相同温度的正则系综描述。\(^\text{[1]}\)
\item 最大熵:对于给定的机械系统(固定\( N \)、\( V \)),正则系综平均值\( -\langle \log P \rangle \)(熵)是任何具有相同\( \langle E \rangle \)的系综中可能的最大值。\(^\text{[1]}\)
\item 最小自由能:对于给定的机械系统(固定\( N \)、\( V \)和给定的\( T \)值,正则系综平均值\( \langle E + kT \log P \rangle \)(赫尔姆霍兹自由能)是任何系综中可能的最小值。\(^\text{[1]}\)这显然等价于最大化熵。
\end{itemize}
\subsection{自由能、系综平均值和精确微分}  
\begin{itemize}
\item 函数\( F(N, V, T) \)的偏导数给出了重要的正则系综平均量:  
\item 平均压强是\(^\text{[1]}\)  
\[
\langle p \rangle = -\frac{\partial F}{\partial V},~
\]  
\item 吉布斯熵是\(^\text{[1]}\)  
\[
S = -k \langle \log P \rangle = -\frac{\partial F}{\partial T},~
\]  
\item 偏导数\(\partial F/\partial N\)与化学势近似相关,尽管化学平衡的概念并不完全适用于小系统的正则系综。[注释2]  
\item 平均能量是\(^\text{[1]}\)  
\[
\langle E \rangle = F + S T.~
\]  
\item 精确微分:从上述表达式可以看出,函数\( F(V, T) \)对于给定的\( N \)具有精确的微分\(^\text{[1]}\)  
\[
dF = -S \, dT - \langle p \rangle \, dV.~
\]  
\item 热力学第一定律:将上述关系代入\( \langle E \rangle \)的精确微分中,得到一个类似于热力学第一定律的方程,只不过一些量上带有平均符号:\(^\text{[1]}\)  
\[
d \langle E \rangle = T \, dS - \langle p \rangle \, dV.~
\]  
\item 能量波动:系统中的能量在正则系综中具有不确定性。能量的方差是\(^\text{[1]}\)  
\[
\langle E^2 \rangle - \langle E \rangle^2 = k T^2 \frac{\partial \langle E \rangle}{\partial T}.~
\]”
\end{itemize}
\subsection{例子系综}  
“我们可以想象大量相同性质的系统,但它们在给定时刻的配置和速度不同,且不仅仅是微小的不同,甚至可以是包括每一种可能的配置和速度组合…” —— J. W. 吉布斯(1903年)\(^\text{[10]}\) 
\subsubsection{玻尔兹曼分布(可分系统)}  
如果由正则系综描述的系统可以分解为独立的部分(当不同部分不相互作用时),且每个部分具有固定的物质组成,那么每个部分都可以看作是一个独立的系统,并且由与整体相同温度的正则系综描述。此外,如果系统由多个相似的部分组成,那么每个部分的分布与其他部分完全相同。

通过这种方式,正则系综为任何数量粒子的系统提供了精确的玻尔兹曼分布(也称为麦克斯韦-玻尔兹曼统计)。相比之下,从微正则系综推导玻尔兹曼分布仅适用于具有大量部分的系统(即在热力学极限下)。

玻尔兹曼分布本身是将统计力学应用于实际系统的最重要工具之一,因为它大大简化了对可以分解为独立部分的系统的研究(例如气体中的粒子、腔体中的电磁模式、聚合物中的分子键)。
\subsubsection{伊辛模型(强相互作用系统)}  
在由彼此相互作用的部分组成的系统中,通常无法像玻尔兹曼分布那样将系统分解为独立的子系统。在这些系统中,必须使用正则系综的完整表达式,以便在系统与热库相连时描述其热力学性质。正则系综通常是研究统计力学的最直接框架,甚至允许在某些相互作用模型系统中得到精确解。\(^\text{[11]}\)

一个经典的例子是伊辛模型,它是一个广泛讨论的玩具模型,用于研究铁磁现象和自组装单分子层的形成,是最简单的展示相变的模型之一。拉尔斯·温萨格著名地精确计算了零磁场下无限大方格晶格伊辛模型的自由能,该计算是在正则系综下进行的。\(^\text{[12]}\)
\subsection{系综的精确表达式}
统计系综的精确数学表达式取决于所考虑的力学类型——量子力学或经典力学——因为在这两种情况下,“微观状态”的概念有很大的不同。在量子力学中,正则系综提供了一个简单的描述,因为对角化过程提供了具有特定能量的离散微观状态集合。经典力学的情况则更为复杂,因为它涉及到在正则相空间中的积分,且相空间中微观状态的大小可以根据需要进行某种程度的任意选择。
\subsubsection{量子力学的}
量子力学中的统计系综由密度矩阵表示,记作\(\hat{\rho}\)。在无基表示法中,正则系综是密度矩阵:
\[
\hat{\rho} = \exp \left( \frac{1}{kT} (F - \hat{H}) \right),~
\]
其中\(\hat{H}\)是系统的总能量算符(哈密顿算符),而\(\exp()\)是矩阵指数算符。自由能\(F\)由概率归一化条件决定,即密度矩阵的迹为1,\(\operatorname{Tr} \hat{\rho} = 1:\)
\[
e^{- \frac{F}{kT}} = \operatorname{Tr} \exp \left( - \frac{1}{kT} \hat{H} \right).~
\]
如果已知系统的能量本征态和能量本征值,正则系综可以用简洁的形式写出,采用布拉-凯特表示法。给定一组完整的能量本征态\(|\psi_i\rangle\),由\(i\)索引,正则系综为:
\[
\hat{\rho} = \sum_{i} e^{\frac{F - E_i}{kT}} |\psi_i \rangle \langle \psi_i |,~
\]
\[
e^{- \frac{F}{kT}} = \sum_{i} e^{- \frac{E_i}{kT}}.~
\]
其中\( E_i \)是由\(\hat{H} |\psi_i\rangle = E_i |\psi_i\rangle \) 确定的能量本征值。换句话说,量子力学中的一组微观状态由一组完整的驻态给出。密度矩阵在这个基底中是对角的,对角线上的每个元素直接给出一个概率。