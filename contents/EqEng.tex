% 能均分定理
% 热力学|能均分|动能|理想气体

若一个系统有若干个自由度($x$ 或 $p$) 对能量的贡献是其平方的函数, 例如动能 $p_x^2/(2m)$, 势能 $kx^2/2$, $ky^2/2$, 等。 那么其对热容的贡献就是 $kT/2$。

用巨正则系宗证明
\begin{equation}
\begin{aligned}
Q &= \frac{1}{h^N} \int \exp(-\beta \sum_i \alpha_i x_i^2 - \beta \sum_i \gamma_i p_i^2) \dd[N]{x} \dd[N]{p} = \frac{1}{h} \int \exp(-\alpha_1 x_1^2 \beta) \dd{x}\\
& \times \int \dots \times \int \dots \propto \sqrt{\frac{1}{\beta}} \times \dots
\end{aligned}
\end{equation}
这里只分析第一个自由度。 系统能量为
\begin{equation}
E = -\pdv{\beta} \ln Q = -\pdv{\beta} \ln \sqrt{\frac{1}{\beta}} - \pdv{\beta} \ln \dots = \frac{1}{2\beta} + \dots = \frac{1}{2}kT + \dots
\end{equation}
