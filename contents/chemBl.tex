% 化学反应平衡
% license Usr
% type Tutor

\begin{issues}
\issueDraft
\end{issues}

\pentry{态函数\nref{nod_statef}, 吉布斯自由能\nref{nod_GibbsG}}{nod_b19e}
\footnote{参考书目:朱文涛等人的《简明物理化学》}
\subsection{化学反应平衡判据}
什么是化学反应?唯象地说,化学反应前后系统中各物质的含量改变,系统状态也随之改变。

我们假定一个始终保持等温等压的系统(这要求化学反应全程,系统能与环境交互)。系统反应前的Gibbs能是 
$$G_0 = \sum_i n_i \mu_i~,$$
现在,我们想象系统中发生了一点化学反应、系统物质的组成轻微变化\footnote{或许我们应该使用代表虚变动的$\delta$而不是$\mathrm{d}$。}:
$$G_1 = \sum_i (n_i + \dd n_i) \mu_i~,$$
这个微小过程前后,Gibbs能变化自然是:
\begin{equation}\label{eq_chemBl_2}
\dd G = G_1 - G_0 = \sum_i \mu_i \dd n_i~,
\end{equation}
根据我们对\enref{Gibbs判据 }{GibbsG} 的理解,
\begin{equation}\label{eq_chemBl_1}
\begin{aligned}
\text{反应能继续发生} &\Longleftrightarrow \dd G < 0~,\\
\text{反应达到平衡} &\Longleftrightarrow \dd G = 0~.\\
\end{aligned}
\end{equation}
至此,我们已经原则上解决了化学平衡问题。复读一遍高中课本:化学反应达到平衡,不是反应物被完全消耗,而是反应物与生成物按特定比例共处一室。

\subsection{化学反应的物质守恒;化学反应进度}
\pentry{物质的量与摩尔(高中)\nref{nod_MOLE}}{nod_8d89}
我们上文中直接描述了每个物质的变化量$\dd n_i$。然而故事没那么简单,\textsl{或者说故事其实更简单}:根据物质守恒,每个物质的物质的量变动不是独立的。比如说,对于一个最简单的分解反应$A\to2B$,我们都知道消耗$1 \Si{mol} A$后,总会生成$2 \Si{mol} B$,而不能生成$3 \Si{mol} B$,或者只生成$1 \Si{mol} B$。

因此,化学反应中各个物质的物质的量变化是高度关联的,一个化学反应只有一个自由度,我们称之为化学反应进度 $\dd \xi$。系统中各个物质的物质的量变化$\dd n_i$都可由$\dd \xi$确定:
\begin{equation} \label{eq_chemBl_7}
\dd n_i = \nu_i \dd \xi~,
\end{equation}
其中$\nu_i$是化学计量数,即化学方程式中物质前的数字,生成物记为正、反应物记为负。

例如,在$A\to2B$中,若$\dd \xi=1 \Si{mol}$(读作“发生$1 \Si{mol}$反应后”),意味着$\dd n_A = \nu_A \dd \xi = - 1 \Si{mol}$以及$\dd n_B = \nu_i \dd \xi = 2 \Si{mol}$。

化用化学反应进度的语言 \autoref{eq_chemBl_7} ,我们惊奇地发现Gibbs变 \autoref{eq_chemBl_2} 变得非常简单:
\begin{equation}\label{eq_chemBl_3}
\dd G = \sum_i \mu_i \dd n_i = \sum_i \mu_i \nu_i \dd \xi =  \dd \xi \sum_i \mu_i \nu_i~,
\end{equation}
在化学学科,可将 \autoref{eq_chemBl_3} 的$\dd \xi$ "除"到左边、令其变为偏微分形式,从而定义反应的摩尔Gibbs变$\Delta_r G_M$:
\begin{equation} \label{eq_chemBl_5}
\Delta_r G_M = \left(\pdv{G}{\xi}\right)_{p,T} = \sum_i \mu_i \nu_i~.
\end{equation}
根据定义,$\Delta_r G_M$与物质组成、\textbf{物质浓度}、系统的温度压力等有关。$\Delta_r G_M \dd \xi$的含义可以不严谨地理解为,发生少量反应$\dd \xi$后,系统Gibbs能的变化。

由于$\Delta_r G_M$ 只是化学反应进度版的$\Delta G$,因此Gibbs判据\autoref{eq_chemBl_1} 对$\Delta_r G_M$依旧成立:
\begin{equation}\label{eq_chemBl_6}
\begin{aligned}
\text{反应能继续发生} &\Longleftrightarrow \Delta_r G_M < 0~,\\
\text{反应达到平衡} &\Longleftrightarrow \Delta_r G_M = 0~.\\
\end{aligned}
\end{equation}



\subsection{化学平衡常数}
\pentry{理想混合物的热力学量\nref{nod_IMCPTV}}{nod_bc35}
如果你对理想混合物稍有理解,你就知道
\begin{equation}\label{eq_chemBl_4}
\mu_i = \mu_i^* + RT \ln x_i~,
\end{equation}
其中$\mu_i^*$是纯物质的化学势,$x_i$是物质的物质的量浓度。

将\autoref{eq_chemBl_4} 代入 \autoref{eq_chemBl_5} ,
\begin{equation}
\begin{aligned}
\Delta_r G_M &= \sum_i \mu_i \nu_i = \sum_i (\mu_i^* + RT \ln x_i) \nu_i\\
&= \sum_i \mu_i^* \nu_i + RT \sum_i \nu_i \ln x_i
\end{aligned}~.
\end{equation}

$\Delta_r G_M$神奇地分为了相对独立的两项:
\begin{itemize}
\item $\sum_i \mu_i^* \nu_i$项与物质的浓度无关,与物质种类(以及温度、压力、化学方程式的书写)有关,称为标准摩尔Gibbs变:$\Delta_r G_M^* = \sum_i \mu_i^* \nu_i$
\item $RT \sum_i \nu_i \ln x_i$项与物质的浓度有关。按化学学科惯例,一般将$\ln$提到最前,使累加变累乘:$RT \ln \Pi_i x_i^{\nu_i}$。$\Pi_i x_i^{\nu_i}$称为活度积 $J=\Pi_i x_i^{\nu_i}$。
\end{itemize}
因此,\begin{equation} \label{eq_chemBl_9}
\Delta_r G_M = \Delta_r G_M^* + RT \ln J~.
\end{equation}

\subsubsection{标准平衡常数}
对\autoref{eq_chemBl_9} 运用Gibbs判据 \autoref{eq_chemBl_6}:
\begin{equation} \label{eq_chemBl_10}
\begin{aligned}
\text{反应达到平衡} &\Rightarrow \Delta_r G_M = 0\\
&\Rightarrow \Delta_r G_M^* + RT \ln J_{eq} = 0\\
&\Rightarrow \Delta_r G_M^* = - RT \ln J_{eq}\\
&\Rightarrow J_{eq} = e^{-\frac{\Delta_r G_M^*}{RT}}\\
\end{aligned}~.
\end{equation}
$J_{eq}$也称标准平衡常数 $K^*$,是不少化学书的标准术语。

\subsubsection{活度积判据}
在化学中,相比于Gibbs判据,更直观、易用的是活度积判据,这也是我们在高中时就\textsl{做烂的题}。根据 \autoref{eq_chemBl_9} ,\autoref{eq_chemBl_10} 与Gibbs判据 \autoref{eq_chemBl_6} ,
$$
\begin{aligned}
\Delta_r G_M & = \Delta_r G_M^* + RT \ln J_{eq}\\
& = - RT \ln K^* + RT \ln J_{eq}\\
& = - RT \ln \frac{K^*}{J} \\
\end{aligned}~.
$$
即$- RT \ln \frac{K^*}{J}<0 \Rightarrow \ln \frac{K^*}{J} >0 \Rightarrow \frac{K^*}{J} >1 \Rightarrow K^*>J, J<K^*$时,反应正向进行;$J=K^*$时,反应达到平衡。此为活度积判据。

\begin{example}{电池、电池、还是电池}
我们知道,Gibbs变还体现了系统做非体积功的能力 \upref{Td2Law}。系统发生$1 \Si{mol}$反应后\footnote{这么表述是不严谨的,因为$1 \Si{mol}$反应期间,系统的$J$也在变化。我们这里假定系统很大,在$1 \Si{mol}$反应期间系统的$J$几乎不变。当然如果你高兴,也可以取为$\dd \xi$,对于下述的分析没有本质影响,但是会让书写麻烦不少。},能做的最多非体积功是:
$$w = - \Delta_r G_M~,$$
如果系统是一个电池,那么非体积功用来驱动电荷:
$$w = NqE~,$$
其中$N$代表$1\Si{mol}$反应中转移的电子数量,$q$代表电子电荷,$E$是电池电动势。%在电化学中,更常用的方案是,$n$代表载流子物质的量,$F = q \cdot N_A$是法拉第常数。
因此,
$$E = -\frac{\Delta_r G_M}{Nq} = -\frac{\Delta_r G_M^*}{Nq} - \frac{RT}{Nq} \ln J ~,$$
有时定义标准电势
$$E^* = -\frac{\Delta_r G_M^*}{Nq}~,$$
那么,
$$E = E^* - \frac{RT}{Nq} \ln J ~.$$
可见,电池电动势与物质种类、活度积等有关,史称能斯特Nernst方程。能斯特方程部分解释了不少电池现象:电压与电极材料的关联、电压与电池剩余电量的联系、温度对电压的影响...
\end{example}

至此,我们基于经典热力学理论,论证了为什么化学反应往往不会完全进行:因为Gibbs判据,或者更基本的,热力学第二定律。

必须注意的是,化学反应平衡仅仅表明反应的热力学倾向,而并不代表这个反应能真正发生。你可能查阅各种数据手册后发现,从单质生成某些蛋白质的反应有 $\Delta_r G_M <0$,然而这并不意味着你把单质元素堆在一块就能“合成”出蛋白质!你还得考虑更为复杂的动力学因素。更老生常谈的例子是钻石与石墨:常温下钻石到石墨的反应其实是$\Delta G<0$的,然而由于动力学因素(钻石结构十分紧凑,同时常温下原子活性低,原子没有动力克服能垒以形成热力学上更稳定的石墨结构),几乎不可能观察到钻石转变为石墨。


% --以下是本文章的旧版本,按需删除:--

% 对于一个多元系统,可以写出
% \begin{equation}
% G=G(p,T,n_1,n_2,...)~.
% \end{equation}
% 那么,
% \begin{equation}
% \dd G=-S \dd T +V \dd P + \sum \mu_B \dd n_B~.
% \end{equation}
% 在等温等压下,p,T均为定值
% \begin{equation}
% \dd G=\sum \mu_B \dd n_B~,
% \end{equation}

% 代入化学反应进度 $\dd \xi=\dd n_B/\nu_b$,得
% \begin{equation}
% \dd G=\sum \nu_B \mu_B \dd \xi~.
% \end{equation}
% $\nu$: 化学反应系数

% 或写为导数形式
% \begin{equation}
% \dv{G}{\xi}=\sum \nu_B \mu_B~.
% \end{equation}

% 定义摩尔Gibbs变
% \begin{equation}
% \Delta _r G_M = \dv{G}{\xi}=\sum \nu_B \mu_B~.
% \end{equation}
% 含义为“化学反应进度再多进行一点后,系统的Gibbs能变”。

% \subsection{化学势判据}
% 根据吉布斯自由能的含义\upref{GibbsG},
% 若 $\Delta _r G_M<0$,则反应会继续进行;若$\Delta _r G_M = 0$,则反应达到平衡。

% \subsection{活度积、标准平衡常数、活度积判据}
% 注意到 
% $\mu_B=\mu_B^*+RT \ln x_B$,那么
% \begin{equation} \label{eq_chemBl_1}
% \Delta _r G_M =\sum \nu_B \mu_B = \sum \nu_B (\mu_B^*+RT\ln _B)=\sum \nu_B \mu_B^* + RT \sum \nu_B \ln x_B~.
% \end{equation}

% 定义标准摩尔Gibbs焓变 $\Delta _r G_M^*=\sum \nu_B \mu_B^*$与活度积 $J=\prod x_B^{\nu_B}~,$

% 那么
% \begin{equation}
% \Delta _r G_M = \Delta _r G_M^* + RT \ln J~.
% \end{equation}

% 再定义标准平衡常数 $K = e^{\frac{-\Delta _r G_M^*}{RT}}$,即$\Delta _r G_M^* = -RT \ln K~.$

% 那么 
% \begin{equation}
% \Delta _r G_M = -RT \ln K + RT \ln J = RT \ln \frac{J}{K}~.
% \end{equation}

% 根据化学势判据,当 
% $RT \ln \frac{J}{K} < 0$,即$J<K$时,反应正向进行;$J=K$时,反应达到平衡。此为活度积判据。
