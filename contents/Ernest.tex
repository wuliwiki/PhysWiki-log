% 欧内斯特·卢瑟福
% license CCBYSA3
% type Wiki

(本文根据 CC-BY-SA 协议转载自原搜狗科学百科对英文维基百科的翻译)

\textbf{欧内斯特·卢瑟福}, 尼尔森第一鲁瑟福德男爵, OM, FRS,[1] HFRSE ,LLD(1871年8月30日 –1937年10月19日),是新西兰物理学家,后来被称为核子物理之父。 大英百科全书 认为他是自迈克尔·法拉第 (1791-1867) 以来最伟大的实验主义者。

在早期工作中,卢瑟福提出了放射性半衰期的概念,并发现了放射性元素氡,[2] 区分并命名为α和β辐射。[3] 这项工作是在加拿大的麦吉尔大学完成的。这是他于1908年“因对元素分解和放射性物质化学的研究”而获得诺贝尔化学奖的基础。[4] 为此,他是第一位加拿大和大洋洲的诺贝尔奖获得者。

卢瑟福于1907年搬到英国的曼彻斯特维多利亚大学 (今天的曼彻斯特大学),在那里他和托马斯·罗伊茨证明了α辐射是氦核。[5][6] 卢瑟福成为诺贝尔奖获得者后,完成了他最著名的工作。[4] 在1911年,尽管他不能证明究竟是正电荷还是负电荷, 他推测原子的电荷集中在一个非常小的原子核中,[7] 因此,他通过汉斯·盖格(Hans Geiger)和欧内斯特·马斯登(Ernest Marsden)的金箔实验发现并解释了卢瑟福散射,从而开创了原子的卢瑟福模型。他进行的研究推进了1917年在氮和α粒子之间的核反应中原子的第一次“分裂”,在其中他还发现了(并命名为)质子。[8]

卢瑟福于1919年成为剑桥大学卡文迪许实验室学院的院长。在他的领导下,詹姆斯·查德威克(James Chadwick)于1932年发现了中子。同年,在他的指导下,他的学生约翰·考克饶夫和欧内斯特·瓦耳顿进行了第一次完全受控分裂细胞核的实验。1937年去世后,他有幸与英国最伟大的科学家艾萨克·牛顿爵士安葬在同一地点,坟墓在威斯敏斯特教堂附近。化学元素rutherfordium(第104号元素) 于1997年以他的名字命名。

\subsection{ 传记}
\subsubsection{1.1 早期生活和教育}
\begin{figure}[ht]
\centering
\includegraphics[width=6cm]{./figures/0020fd0308247deb.png}
\caption{卢瑟福,21岁} \label{fig_Ernest_1}
\end{figure}
欧内斯特·卢瑟福是农民詹姆斯·卢瑟福和他的妻子玛莎·汤普森的儿子,玛莎·汤普森最初来自英国埃塞克斯郡的霍恩彻奇。[9] 詹姆斯于1960年从苏格兰的珀斯移民到新西兰,“为了种植一点亚麻并抚养许多孩子”。欧内斯特出生于亮水,靠近纳尔逊,新西兰。他的名字在出生登记时被错误地拼写为“欧内斯特”。[10]卢瑟福的母亲玛莎·汤普森是一名教师。[11]

他先后在哈夫洛克学校和纳尔逊学院学习,并在纽西兰大学的澳洲坎特伯雷中学获得了奖学金,在那里他参加了辩论协会,并打了橄榄球。[12] 在获得文学学士,文学硕士学位和理学学士学位后,经过两年的研究,他发明了一种新型的无线电接收机,于1895年,卢瑟福被授予了1851年皇家展览委员会的研究奖学金[13] 并去英国,剑桥大学的卡文迪许实验室读研究生。[14] 在约瑟夫·汤姆孙鼓舞人心的领导下,他是第一批被允许在大学进行研究“外星人”(没有剑桥学位的人)的学生之一,这引起了卡文迪许兄弟会更保守成员的嫉妒。在汤姆森的鼓励下,卢瑟福设法在半英里处探测到了无线电波,并短暂地保持了电磁波探测距离的世界记录,尽管当他在1896年的英国协会会议上介绍自己的研究结果时,他发现自己已经被另一位讲师马可尼(Marconi)的表现超越了。

1898年,汤姆森推荐卢瑟福在加拿大蒙特利尔的麦吉尔大学工作。他将接替休·朗伯恩·卡伦达,后者曾担任麦克唐纳物理学教授的主席,即将来到剑桥。[15] 卢瑟福被如期接受,这意味着他可以在1900年和玛丽·乔治娜·牛顿结婚(1876-1954),[16][17] 卢瑟福在离开新西兰之前已经和她订婚;后来他们在基督城的圣保罗圣公会结婚,[18][19] 他们有一个女儿艾琳·玛丽(1901-1930),嫁给了拉尔夫·福勒。1901年,他获得了新西兰大学的文学学士学位 。[14] 1907年,卢瑟福回到英国,在曼彻斯特维多利亚大学担任物理学院的主席。

\subsubsection{1.2 晚年和荣誉}
卢瑟福在1914年被封爵。[20] 第一次世界大战期间,他致力于一个绝密项目,以解决利用声纳探测潜艇的实际问题。[21] 1916年,他被授予赫克托纪念奖章。1919年,他回到卡文迪许,接替约瑟夫·汤姆孙成为卡文迪许教授和主任。在他的领导下,詹姆斯·查德威克因发现中子(1932年),约翰·考克饶夫和欧内斯特·瓦耳顿使用粒子加速器完成了一项被称为分裂原子的实验,爱德华·阿普尔顿证明了电离层的存在。1925年,卢瑟福呼吁新西兰政府支持教育和研究,导致第二年成立了科学和工业研究部(DSIR) 。[22] 1925年至1930年间,他担任皇家学会主席,后来担任学术援助委员会主席,该委员会帮助了近1000名来自德国的大学难民。[23]在1925年,他被授予功绩勋章 (英国)新年荣誉[23] 并于1931年晋升为剑桥郡的纳尔逊·卢瑟福男爵,[24] 这个头衔在他1937年意外去世后消失了。1933年,卢瑟福是 T. K. Sidey奖章的两位首任获得者之一,该奖章由新西兰皇家学会设立,作为对有杰出科学研究贡献人的奖励。[25][26]
\begin{figure}[ht]
\centering
\includegraphics[width=6cm]{./figures/a1c7cde616085bb4.png}
\caption{卢瑟福勋爵在威斯敏斯特教堂的坟墓} \label{fig_Ernest_2}
\end{figure}
在他死前的一段时间,卢瑟福患有不太严重的疝气,他忽略了对它的治疗,结果它变得绞窄,导致他严重生病。尽管在伦敦进行了紧急手术,四天后他还是在剑桥死于医生所说的“肠麻痹”症状。[27] 在格德斯绿地火葬场火化后,[27] 他被荣誉安葬在威斯敏斯特教堂,靠近艾萨克·牛顿和其他英国杰出科学家。[28]

\subsection{科学研究}
\begin{figure}[ht]
\centering
\includegraphics[width=8cm]{./figures/4f761be204fbfda8.png}
\caption{欧内斯特·卢瑟福1905年在麦吉尔大学} \label{fig_Ernest_4}
\end{figure}

在剑桥大学,卢瑟福开始与约瑟夫·汤姆孙研究关于X射线对气体的传导效应,这项工作导致了电子的发现,汤姆森在1897年向世界展示了它。听到贝克雷尔在铀方面的经验后,卢瑟福开始探索其放射性,发现了两种与X射线不同的穿透能力的射线。 他继续在加拿大进行研究,于1899年创造了“α射线”和“β射线”这两个术语来描述两种不同类型的辐射。然后他发现钍释放出一种气体,这种气体产生的辐射本身具有放射性,会覆盖其他物质。他发现,任何大小的这种放射性物质的样品都不可避免地需要相同的时间来衰变一半的样本——它的“半衰期" (本例中为11分钟)。

从1900年到1903年,年轻的化学家弗雷德里克·索迪 ( 诺贝尔化学奖,1921年)加入了卢瑟福在麦吉尔的工作,卢瑟福为索迪提出了识别钍射气的问题。在索迪消除了所有正常的化学反应之后,他建议这一定是一种惰性气体,他们将其命名为索伦(后来发现是氡的同位素)。他们还发现了另一种钍,他们称之为钍X,并继续寻找氦的痕迹。他们还研究了来自威廉·克鲁克斯的“铀X”样品和来自玛丽·居里的镭。

1903年,他们发表了“放射性变化定律”,以解释他们所有的实验。在那之前,原子被认为是所有物质不可摧毁的基础,尽管居里曾提出放射性是一种原子现象,但放射性物质原子分裂的想法是一个全新的想法。卢瑟福和索迪证明了放射性涉及原子自发分解成其他尚未识别的物质。1908年诺贝尔化学奖授予欧内斯特·卢瑟福“因为他对元素分解和放射性物质化学的研究”。

1903年,卢瑟福认为法国化学家保罗·维拉德在1900年发现了一种辐射(但没有命名),它来自镭的辐射,他意识到这一观察结果一定代表了与他自己的α和β射线不同的东西,因为这种辐射的穿透力要大得多。卢瑟福因此将这第三种辐射命名为γ射线。卢瑟福的所有三个术语今天都在标准使用——其他类型的放射性衰变也已经被发现,但是卢瑟福的这三种类型是最常见的。

在曼彻斯特,他继续研究α辐射。与汉斯·盖革一起,他开发了硫化锌闪烁屏和电离室来计数α。通过将它们产生的总电荷除以计数的数字,卢瑟福确定α上的电荷是2。1907年底,欧内斯特·卢瑟福和托马斯·罗伊茨允许α通过一个非常薄的窗口进入真空管。当它们激发管子放电时,随着α粒子在管子中积累,从中获得的光谱发生了变化。最终,清晰的氦气光谱出现了,证明α粒子至少是电离的氦原子,也可能是氦核。

\subsubsection{2.1 金箔实验}
\begin{figure}[ht]
\centering
\includegraphics[width=6cm]{./figures/fbafd84f5eca4492.png}
\caption{顶部:预期结果:$\alpha$粒子不受干扰地穿过原子的梅子布丁模型。 底部:观察结果:一小部分粒子偏转,表明小而集中的电荷。请注意,图像不是按比例绘制的;实际上,原子核比电子亚层小得多。} \label{fig_Ernest_5}
\end{figure}
卢瑟福完成了他最著名的工作在1908年获得诺贝尔奖之后。1909年,他与汉斯·盖革和欧内斯特·马斯登一起进行了盖革-马士登实验,通过偏转穿过薄金箔的$\alpha$粒子,证明了原子的核性质。卢瑟福受到启发,在这个实验中要求盖革和马斯登寻找具有非常高偏转角的α粒子,这是当时任何物质理论都没有想到的类型。尽管这种偏转很少,但还是被发现了,并被证明是平滑偏转角度但高阶的函数。卢瑟福对这些数据的解释导致他在1911年提出了原子的拉塞福模型 – 一个非常小的带电核,[29] 包含大部分原子质量的原子由低质量电子环绕。

在1919-1920年,卢瑟福发现氮和其他轻元素粒子被$\alpha$粒子撞击时弹出一个质子(卢瑟福说“氢原子”,而不是“质子”)。[29] 这个结果表明卢瑟福认为氢核是氮核的一部分(根据推断,可能也是其他核)。基于普鲁特假设,这种结构多年来一直被怀疑是基于氢的原子量的整数。氢是已知最轻的元素,它的原子核大概是最轻的原子核。现在,由于所有这些考虑,卢瑟福认为氢核可能是所有原子核的基本组成部分,也可能是一种新的基本粒子,因为从原子核中没有比氢更轻的物质。 因此,为证实并扩展威廉·维恩的工作,他在1898年在电离气体流中发现了质子。[30] 卢瑟福在1920年假设氢核是一种新粒子,他称之为质子。

卢瑟福对中子的存在进行了理论化分析(他在他的1920年贝克演讲中将其命名为中子),该中子可以通过产生引诱的核力以某种方式补偿质子正电荷的排斥作用,从而使原子核远离质子之间的排斥。 中子的唯一替代方法是存在“核电子”,它可以抵消原子核中的某些质子电荷,因为到那时,人们已经知道,如果原子核是简单地由氢组装而成的,那么核的质量大约是其两倍。 但是这些核电子是如何被困在原子核中的,是一个谜。

卢瑟福的中子理论于1932年被他的助手詹姆士·查德威克(James Chadwick)证明,他在其他科学家和后来的他本人用$\alpha$粒子轰击铍时,立即认出了中子。 1935年,查德威克因这项发现而获得了诺贝尔物理学奖。

\subsection{遗产}
\begin{figure}[ht]
\centering
\includegraphics[width=6cm]{./figures/291e575116939f95.png}
\caption{纪念卢瑟福出现在曼彻斯特大学} \label{fig_Ernest_6}
\end{figure}

\subsubsection{3.1 核子物理}
卢瑟福的研究工作以及在他的带领下完成的工作,确立了原子的核结构以及放射性衰变作为核过程的本质。 在卢瑟福(Rutherford)领导下工作的研究员帕特里克·布莱克特(Patrick Blackett)使用天然α粒子证明了诱导的核嬗变。卢瑟福的团队后来使用加速器中的质子演示了人工诱导的核反应和嬗变。他被认为是核物理之父。卢瑟福过早去世,以至于看不到莱奥·西拉德(LeóSzilárd)提出的控制核链反应的想法。然而,塞兹拉德(Szilárd)于1933年9月12日在伦敦发表的《泰晤士报》(The Times)上发表了有关卢瑟福(Rutherford)关于锂的人工诱发嬗变的演讲,这是他思考可控产能核链式反应可能性的灵感来源。同一天,西拉德在伦敦散步时有了这个想法。

卢瑟福的演讲涉及到他的学生约翰·考克饶夫和欧内斯特·瓦耳顿 1932年的工作,他们用他们建造的粒子加速器中的质子轰击锂,将锂“分裂”成α粒子。卢瑟福意识到从分裂的锂原子中释放的能量是巨大的,但他也意识到加速器所需的能量,以及它以这种方式分裂原子的本质上的低效,使得该项目不可能作为实际的能源(轻元素的加速器-诱发裂变仍然效率太低,以至于今天也不能以这种方式使用)。卢瑟福的演讲部分内容如下:

在这些过程中,我们获得的能量可能比提供的质子多得多,但平均而言,我们无法期望以这种方式获得能量。 这是一种非常效率低下的能源生产方式,任何在原子转化中寻找动力源的人都是杯水车薪。 但是该主题在科学上很有趣,因为它可以洞悉原子。

\subsubsection{3.2 为纪念卢瑟福的生活和工作而命名的物品}
\begin{figure}[ht]
\centering
\includegraphics[width=8m]{./figures/bfef5208c3094b0c.png}
\caption{新西兰布莱特沃特一座年轻欧内斯特·卢瑟福的雕像。} \label{fig_Ernest_7}
\end{figure}
科学发现
\begin{itemize}
\item 元素钌原子,$Rf,Z=104$。(1997年)[31]
\item 卢瑟福(Rd),一种相当于一个兆贝克尔的过时放射性单位。
\end{itemize}
机构
\begin{itemize}
\item 卢瑟福·阿普尔顿实验室,牛津郡迪德科特附近的科学研究实验室
\item 奥克兰卢瑟福学院,新西兰奥克兰学校
\item 肯特卢瑟福学院,位于英国坎特伯雷肯特大学,
\item 卢瑟福创新研究所,剑桥大学
\item 新西兰旺加尼卢瑟福中学
\item 卢瑟福大厅,拉夫堡大学的一个住宅大厅
\end{itemize}
授予
\begin{itemize}
\item 卢瑟福奖章,由新西兰皇家学会颁发的最高科学奖章
\item 托马斯卡尔学院(Thomas Carr College)的卢瑟福奖(Rutherford Award),澳大利亚维多利亚州教育化学证书
\item 卢瑟福纪念奖章,由加拿大皇家学会颁发的物理和化学领域研究奖
\item 卢瑟福奖章和奖金,由物理研究所每两年颁发一次,以表彰“核物理或核技术方面的杰出研究”
\item 卢瑟福纪念讲座,由1952年卢瑟福计划下创建的皇家学会赞助的国际巡回讲座
\end{itemize}
建筑物
\begin{itemize}
\item 卢瑟福大厦,尼尔森学院的寄宿公寓[32]
\item 卢瑟福酒店 ,尼尔森最大的酒店,包括卢瑟福咖啡厅和酒吧
\item 新西兰坎特伯雷大学的物理和化学大楼
\item 新西兰坎特伯雷大学的罗彻斯特和卢瑟福大厅
\item 卢瑟福大厦,惠灵顿维多利亚大学,美国皮皮茶校区,最初是新西兰电力部门的总部,位于惠灵顿,新西兰。
\item 贝德福德现代学校的卢瑟福大楼。
\item 现代建筑卡文迪许实验室,剑桥大学
\item 欧内斯特·卢瑟福物理大楼,麦吉尔大学,蒙特利尔[33]
\item 曼彻斯特大学卢瑟福工作的地方,于2006年更名为“卢瑟福大楼”。
\item 舒斯特实验室的卢瑟福演讲厅,曼彻斯特大学
\end{itemize}
主要街道
\begin{itemize}
\item 卢瑟福勋爵路(他的出生地在新西兰布赖特沃特)
\item 卢瑟福街,尼尔逊中部的一条主要街道
\item 卢瑟福·克洛斯,牛津郡阿宾登的居住区道路
\item 卡尔斯巴德(加利福尼亚州) 生物技术区卢瑟福路
\item 加拿大安大略沃恩商业区/居住区道路卢瑟福路
\end{itemize}
其他的
\begin{itemize}
\item 卢瑟福公园,尼尔逊的一个运动场
\item 在新西兰布莱特沃特的卢瑟福纪念馆,卢瑟福的出生地
\item 他的形象出现在新西兰100美元纸币的正面(自1992年以来)。
\item 卢瑟福基金会,由新西兰皇家学会设立的慈善信托基金,旨在支持科学技术研究。[34]
\item 卢瑟福大厦,地址麦克莱恩学院, 奥克兰,新西兰
\item 卢瑟福大厦,哈密尔顿(新西兰) 希尔克雷斯特高中
\item 新西兰罗托路亚罗托路亚中学卢瑟福楼
\item 卢瑟福之家,在兰吉拉高中
\item 火山口卢瑟福月球和卢瑟福陨石坑火星
\item 欧内斯特·卢瑟福是斯图尔特·霍尔戏剧的主题。
\item 在剑桥原始卡文迪许实验室所在地的蒙德实验室一侧,卢瑟福的记忆中有一个以鳄鱼形式的雕刻,这是它的专员,他的同事彼得·卡皮萨给他的昵称。
\item 卢瑟福火箭发动机,由Rocket Lab 在新西兰开发的发动机,是第一个使用电动泵进给循环的发动机。
\item 他的肖像被描绘在长老会教堂的彩色玻璃窗上林迪斯法恩学院在新西兰黑斯廷斯。该窗口于2007年揭幕,致力于该学院关于具有最高品格的人的推广,并以了卢瑟福和查尔斯·厄本,埃德蒙·希拉里,和John Rangihau作为标志性的例子。[35]
\end{itemize}

\subsection{卢瑟福以前实验室的癌症发病率}
卢瑟福在曼彻斯特大学的库普兰大楼进行了许多实验,该大楼一直是癌症群集研究的主题。据统计,卢瑟福以前的实验室内和周围发生的胰腺癌、脑癌和运动神经元疾病的发病率很高,自1984年以来,共有6名工人患有这些疾病。2009年,一个独立委员会得出结论,与卢瑟福几十年前的实验相关的各种辐射水平略微升高并不是这种癌症的可能原因,并裁定这些疾病是巧合。[36]

\subsection{出版物}
\begin{itemize}
\item 无线电活动(1904),第2版。(1905年),ISBN 978-1-60355-058-1
\item 放射性转变(1906年),ISBN 978-1-60355-054-3
\item 放射性物质及其辐射(1913年)[37]
\item 物质的电结构(1926)
\item 元素的人工蜕变(1933年)
\item 新炼金术(1937年)
\end{itemize}

\subsubsection{5.1 文章}
\begin{itemize}
\item “放射性元素的分解” 哈珀月刊,1904年1月,第279至284页。
\end{itemize}

\subsection{称呼和称呼的风格}
\subsubsection{6.1 地址样式}
\begin{itemize}
\item 1871-1903年:欧内斯特·卢瑟福先生
\item 1903-1914年:欧内斯特·卢瑟福先生皇家学会会员
\item 1914-1925年:欧内斯特·卢瑟福爵士皇家学会会员
\item 1925年:欧内斯特·卢瑟福爵士功绩勋章 FRS
\item 1925-1930年:欧内斯特·卢瑟福爵士功绩勋章 PRS
\item 1930-1931年:欧内斯特·卢瑟福爵士功绩勋章 FRS
\item 1931-1937年:尊敬的阁下纳尔逊·奥姆的卢瑟福勋爵 FRS
\end{itemize}

\subsubsection{6.2 武器}
\subsection{参考文献}
[1]
^Eve, A. S.; Chadwick, J. (1938). "Lord Rutherford 1871–1937". Obituary Notices of Fellows of the Royal Society. 2 (6): 394. doi:10.1098/rsbm.1938.0025..

[2]
^Nicholas P. Cheremisinoff (20 April 2016). Pollution Control Handbook for Oil and Gas Engineering. Wiley. pp. 886–. ISBN 978-1-119-11788-9..

[3]
^"The Discovery of Radioactivity". lbl.gov. 9 August 2000..

[4]
^"Ernest Rutherford – Biography". NobelPrize.org. Retrieved 21 February 2013..

[5]
^Campbell, John. "Rutherford – A Brief Biography". Rutherford.org.nz. Retrieved 4 March 2013..

[6]
^Rutherford, E.; Royds, T. (1908). "XXIV.Spectrum of the radium emanation". Philosophical Magazine. Series 6. 16 (92): 313. doi:10.1080/14786440808636511..

[7]
^Longair, M. S. (2003). Theoretical concepts in physics: an alternative view of theoretical reasoning in physics. Cambridge University Press. pp. 377–378. ISBN 978-0-521-52878-8..

[8]
^欧内斯特·卢瑟福·|·新西兰历史在线。NZ history . net . NZ(1937年10月19日)。检索于2011-01-26。.

[9]
^McLintock, A.H. (18 September 2007). "Rutherford, Sir Ernest (Baron Rutherford of Nelson, O.M., F.R.S.)". An Encyclopaedia of New Zealand (1966 ed.). Te Ara – The Encyclopedia of New Zealand. ISBN 978-0-478-18451-8. Retrieved 2 April 2008..

[10]
^Campbell, John. "Rutherford, Ernest 1871–1937". Dictionary of New Zealand Biography. Ministry for Culture and Heritage. Retrieved 4 April 2011..

[11]
^作者:J.L .海尔布隆-欧内斯特·卢瑟福和原子爆炸牛津大学出版社ISBN 0-19-512378-6.

[12]
^Campbell, John (30 October 2012). "Rutherford, Ernest". An Encyclopaedia of New Zealand. Te Ara – The Encyclopaedia of New Zealand. Retrieved 1 October 2013..

[13]
^1851年皇家委员会档案馆.

[14]
^"Rutherford, Ernest (RTRT895E)". A Cambridge Alumni Database. University of Cambridge..

[15]
^McKown, Robin (1962). Giant of the Atom, Ernest Rutherford. Julian Messner Inc, New York. p. 57..

[16]
^TEARA:新西兰百科全书故事:卢瑟福,欧内斯特.

[17]
^新西兰政府出生、死亡和婚姻历史记录注册号1954/19483.

[18]
^"Family history in from the cold". March 18, 2009..

[19]
^Summerfield, Fiona (9 November 2012). "Historic St Paul's Church in the Christchurch suburb of Papanui is being fully restored". Anglican Taonga..

[20]
^"No. 12647". Edinburgh Gazette. 27 February 1914..

[21]
^Alan Selby (2014-11-09). "Manchester scientist Ernest Rutherford revealed as top secret mastermind behind sonar technology". Manchester Evening News. Retrieved 2014-11-13..

[22]
^Brewerton, Emma (2014-12-15). "Ernest Rutherford". Ministry for Culture and Heritage..

[23]
^"Ernest Rutherford, Baron Rutherford of Nelson". Encyclopædia Britannica..

[24]
^"No. 33683". London Gazette. 23 January 1931..

[25]
^"Background of the Medal". Royal Society of New Zealand. Retrieved 7 August 2015..

[26]
^"Recipients". Royal Society of New Zealand. Retrieved 7 August 2015..

[27]
^The Complete Peerage, Volume XIII – Peerage Creations, 1901–1938. St Catherine's Press. 1949. p. 495..

[28]
^海尔布隆,j . l .(2003)《欧内斯特·卢瑟福与原子爆炸》。牛津:牛津大学出版社。第123-124页。ISBN 0-19-512378-6。.

[29]
^Rutherford, Ernest (1911). "The scattering of alpha and beta particles by matter and the structure of the atom". Philosophical Magazine. 21: 669..

[30]
^威廉·维恩:12月,积极的环境和存在的环境。物理年鉴。乐队318 (4),1904,s . 669–677。.
[31]
^Freemantle, Michael (2003). "ACS Article on Rutherfordium". Chemical & Engineering News. American Chemical Society. Retrieved 2 April 2008..

[32]
^"Rutherford House History". Nelson College. Nelson College. Retrieved 1 April 2018..

[33]
^"ErnestRutherford Physics Building". Virtual McGill. McGill University. 24 January 2000. Retrieved 2 April 2008..

[34]
^卢瑟福勋爵可能留下了致命的遗产卢瑟福勋爵可能留下了致命的遗产新西兰皇家新闻学会 Archived 14 10月 2008 at the Wayback Machine。Royalsociety.org.nz。2011年1月26日检索。.

[35]
^"'Good Kiwi men' reflected in chapel window". Hawke's Bay Today. NZPA. 25 July 2007. Retrieved 6 December 2016..

[36]
^Brumfiel, Geoff (30 September 2009). "Rutherford Building cancers a 'coincidence'". Nature. doi:10.1038/news.2009.965..

[37]
^Carmichael, R. D. (1916). "Review" Radioactive Substances and their Radiations, by E. Rutherford" (PDF). Bull. Amer. Math. Soc. 22 (4): 200. doi:10.1090/s0002-9904-1916-02762-5..