% 一致收敛与极限换序
% 一致收敛|换序|limit|微积分|数学分析

\pentry{一致收敛\upref{UniCnv}}

\subsection{极限换序}

函数、数列等有多个自变量存在时,我们有可能会需要考虑多重极限。比如说对于一个二元函数 $f(x, y)$,设它的定义域是 $x>0, y>0$,那么我们如何计算 $x\to 0, y\to 0$ 时 $f$ 的极限值呢?我们可以用以下式子来计算:
\begin{equation}\label{eq_UniCo2_1}
\lim\limits_{y\to 0}\lim\limits_{x\to 0}f(x, y)
\end{equation}

其中,对于任意 $y>0$,$f(x, y)$ 可以认为是关于 $x$ 的一元函数,这样我们就可以计算出 $\lim\limits_{x\to 0}f(x, y)$。对于每个 $y>0$,可以定义 $g(y)=\lim\limits_{x\to 0}f(x, y)$,因此对二元函数 $f$ 进行第一个求极限后,得到的是一个一元函数 $g$。这样,我们同样也能计算出 $\lim\limits_{y\to 0}g(y)$,这也就是\autoref{eq_UniCo2_1} .

从直观的几何角度来理解,$\lim\limits_{x\to 0}f(x, y)$ 就像是求出了 $y$ 轴上的一个函数,然后 $\lim\limits_{y\to 0}\lim\limits_{x\to 0}f(x, y)$ 就是这个函数的一个极限。

我们也可以反过来,用
\begin{equation}\label{eq_UniCo2_2}
\lim\limits_{x\to 0}\lim\limits_{y\to 0}f(x, y)
\end{equation}
来计算 $f$ 的二重极限,这个时候就相当于先求了 $x$ 轴上的一个函数,再求它的极限。

随之而来的问题是,\autoref{eq_UniCo2_1} 和\autoref{eq_UniCo2_2} 的值一样吗?换个说法就是,$f(x, y)$ 的二重极限可以交换次序吗?答案是“不一定”,取决于函数的性质。\autoref{ex_UniCo2_1} 就是一个反例。

\begin{example}{}\label{ex_UniCo2_1}
考虑函数 $f(x, y)=x^y$。我们有:
\begin{equation}
\lim\limits_{x\to 0}f(x, y)=0
\end{equation}
和
\begin{equation}
\lim\limits_{y\to 0}f(x, y)=1
\end{equation}

这就导致
\begin{equation}
\begin{aligned}
\lim\limits_{y\to 0}\lim\limits_{x\to 0}f(x, y)&=0\\
&\not = 1\\
&=\lim\limits_{x\to 0}\lim\limits_{y\to 0}f(x, y)
\end{aligned}
\end{equation}

因此,对于这个 $f(x, y)$,极限是不可以随便换序的。

\end{example}



\subsection{一致收敛的极限换序}

可以极限换序的函数,性质非常良好。一致收敛的函数列就拥有这个良好的性质。

\begin{theorem}{}\label{the_UniCo2_1}
设在开区间 $(a, b)$ 上,函数列 $\{f_n(x)\}$ 一致收敛。如果对于每个 $n$,右极限 $\lim\limits_{x\to 0^+}f_n(x)$ 都存在,那么就有
\begin{equation}\label{eq_UniCo2_3}
\lim\limits_{n\to\infty}\lim\limits_{x\to 0^+}f_n(x)=\lim\limits_{x\to 0^+}\lim\limits_{n\to\infty}f_n(x)
\end{equation}
\end{theorem}

\textbf{证明}:

回顾\textbf{一致收敛}\upref{UniCnv}中提到的技巧,在研究和一致收敛相关的问题时,我们可以只考虑函数列收敛到 $f(x)=0$ 的情况。这是因为 $\{f_n(x)\}$(一致)收敛到 $f(x)$,等价于说 $\{f_n(x)-f(x)\}$ 一致收敛到 $0$。

$\lim\limits_{x\to 0^+}f_n(x)$ 关于编号 $n$ 构成一个数列;$\lim\limits_{n\to\infty}f_n(x)$ 是一个函数。我们首先要证明 $\lim\limits_{x\to 0^+}f_n(x)$ 收敛,这样才能保证 $\lim\limits_{n\to\infty}\lim\limits_{x\to 0^+}f_n(x)$ 是有意义的。

首先定义 $f_n(a)=\lim\limits_{x\to 0^+}f_n(x)$。考虑一致收敛的柯西收敛原理\autoref{the_UniCnv_6}~\upref{UniCnv},$f_n(x)$ 一致收敛意味着对于任意 $\epsilon>0$,存在 $N_\epsilon$,使得对于任意 $m, n>\epsilon$,恒有 $\abs{f_n(x)-f_m(x)}<\epsilon$。把每个 $\abs{f_m(x)-f_n(x)}$ 看作关于 $x$ 的一个函数,那么我们对它求极限 $\lim\limits_{x\to 0^+}$,就可以得到 $\abs{f_n(a)-f_m(a)}<\epsilon$。这么一来,$f_n(a)$ 就是收敛的数列,进而\autoref{eq_UniCo2_3} 左边是有意义的。

假设 $\{f_n(x)\}$ 一致收敛到 $f(x)=0$,那么\autoref{eq_UniCo2_3} 的右边就等于 $0$,接下来我们要证明左边也等于 $0$。

同样地,$f_n(x)$ 一致收敛意味着对于任意 $\epsilon>0$,存在 $N_\epsilon$,使得对于任意 $mn>\epsilon$,恒有 $\abs{f_n(x)-f(x)}<\epsilon$,因此 $\abs{f_n(x)}<\epsilon$。这也就是 $\lim\limits_{n\to\infty}f_n(x)=0$ 的定义,即\autoref{eq_UniCo2_3} 左边也等于 $0$。

% 设 $\abs{\lim\limits_{n\to\infty}\lim\limits_{x\to 0^+}f_n(x)-\lim\limits_{x\to 0^+}\lim\limits_{n\to\infty}f_n(x)}=A\geq 0$。我们接下来就证明 $A=0$。

% 同样地,由于 $f_n$ 一致收敛,对于任意 $\epsilon>0$,存在 $N_\epsilon$,使得对于任意 $m, n>\epsilon$,恒有 $\abs{f_n(x)-f_m(x)}<\epsilon$,因此 $\abs{f_n(a)-f_m(a)}<\epsilon$,因此 $\abs{\lim\limits_{n\to\infty}f_n(a)-f_m(a)}<\epsilon$。

\textbf{证毕}。



\begin{example}{极限不可以换序的反例}
我们还是用
\begin{equation}
f_n(x)=x^n
\end{equation}
在区间 $(0, 1)$ 上做例子。记其逐点连续为 $\lim\limits_{n\to\infty}f_n(x)=f(x)$。
\begin{equation}
\begin{aligned}
\lim\limits_{n\to\infty}\lim\limits_{x\to 1^-}f_n(x)&=\lim\limits_{n\to\infty}1\\
&=1\\
&\not=0\\
&=\lim\limits_{x\to 1^-} 0\\
&=\lim\limits_{x\to 1^-}f(x)\\
&=\lim\limits_{x\to 1^-}\lim\limits_{n\to\infty}f_n(x)
\end{aligned}
\end{equation}
\end{example}

\autoref{the_UniCo2_1} 的一个推论是:
\begin{corollary}{}
设 $\{f_n\}$ 在区间 $I$ 上一致收敛到 $f$。如果各 $f_n$ 连续,那么 $f$ 也连续。
\end{corollary}

\begin{theorem}{}
在闭区间 $[a, b]$ 上,如果函数列 $\{f_n(x)\}$ 一致收敛到 $f(x)$,且各 $f_n(x)$ 在 $[a, b]$ 上可积,那么 $f(x)$ 在 $[a, b]$ 上也可积,且有
\begin{equation}
\int_a^b f(x) \dd x=\lim_{n\to\infty}\int_a^b f_n(x)\dd x
\end{equation}

即此时积分和极限也可以换序。

另外,如果令
\begin{equation}
\leftgroup{
    F_n(x)&=\int_a^x f_n(t)\dd t\\
    F(x)&=\int_a^x f(t)\dd t
}
\end{equation}
那么 $F_n(x)$ 在 $[a, b]$ 上也一致收敛到 $F(x)$。

\end{theorem}

\begin{theorem}{}
设 $\{f_n(x)\}$ 在某点 $x_0\in [a, b]$ 处收敛。如果各 $f_n$ 在 $[a, b]$ 上都处处可导,且导函数列 $\{f'_n(x)\}$ 在 $[a, b]$ 上一致收敛,那么 $\{f_n(x)\}$ 也在 $[a, b]$ 上一致收敛。

如果记 $\lim f_n(x)=f(x)$,那么还有
\begin{equation}
f'(x)=\lim\limits_{n\to\infty}f'_n(x)
\end{equation}
\end{theorem}

















