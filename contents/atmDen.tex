% 大气密度
% 大气|积分方程|理想气体|气压|摩尔质量

\pentry{一阶线性微分方程\upref{ODE1}, 理想气体分压定律\upref{PartiP}}

假设大气是理想气体, 密度随高度变化为 $\rho(z)$. 所以高度 $z$ 处压强为
\begin{equation}\label{atmDen_eq1}
P(z) = \int_{z}^\infty \rho(z') g \dd{z'}
\end{equation}
而根据理想气体状态方程\upref{PVnRT},
\begin{equation}
PV = n R T
\end{equation}
先假设大气只是由一种分子构成, 摩尔质量为 $\mu$, 即 $m = n\mu$, 代入有
\begin{equation}
P = \frac{m}{\mu V} RT = \frac{R}{\mu} \rho T
\end{equation}
其中 $P, T, \rho$ 都是高度的函数. 代入\autoref{atmDen_eq1} 得关于 $\rho(z)$ 的积分方程
\begin{equation}
\frac{R}{\mu} \rho(z) T(z) = \int_{z}^\infty \rho(z') g \dd{z'}
\end{equation}
通常来说海拔越高的地方气温越低, 如果 $T(z)$ 是已知的, 就可以解出 $\rho(z)$. 方程两边对 $z$ 求导, 整理得
\begin{equation}
\rho'(z)  +  \frac{1}{T(z)}\qty[T'(z) + \frac{\mu g}{R}]\rho(z) = 0
\end{equation}
这是一个一阶线性微分方程, 可以直接用公式求解\upref{ODE1}.

作为一种简单情况, 假设温度不随高度变化, 那么方程变为常系数的
\begin{equation}
\rho'(z)  +  \frac{\mu g}{RT}\rho(z) = 0
\end{equation}
容易解得
\begin{equation}\label{atmDen_eq2}
\rho(z) = P_0\exp(-\frac{\mu g}{RT} z)
\end{equation}
其中 $P_0$ 是某个高度 $z_0$ 处的气压. 这说明恒温条件下气压随海拔升高呈指数下降, 且温度越低下降越快.

当大气中有多种气体时, 可以对每种气体分别使用\autoref{atmDen_eq2}, 把 $P_0$ 替换为改气体在 $z_0$ 处的分压\upref{PartiP}. 总密度就是每种气体的密度之和.
