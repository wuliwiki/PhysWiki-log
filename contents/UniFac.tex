% 因式分解唯一性定理
% keys 因式分解|唯一性
% license Xiao
% type Tutor

\pentry{多项式的可约性质\upref{RedPol}}{nod_10bb}
\footnote{吴群。矩阵分析[M].上海:同济大学出版社}把一个多项式分解成几个多项式乘积的形式叫做这个多项式的\textbf{因式分解}。在数论中,我们知道任何大于1的整数都可以分解成素数的乘积,并且这种分解除素数的顺序之外是唯一确定的。对于多项式,也有类似的性质。
\begin{theorem}{因式分解唯一性定理}
数域 $\mathbb{F}$ 上任意一个次数大于1的多项式 $f(x)$ 都可以分解成数域 $\mathbb{F}$ 上有限个不可约多项式的乘积,并且这个分解式是唯一的。即若有两个分解式
 \begin{equation}
 \begin{aligned}
 &f(x)=p_1(x)p_2(x)\cdots p_s(x)~,\\
 &f(x)=q_1(x)q_2(x)\cdots q_t(x)~.
 \end{aligned}
 \end{equation}
 则必有 $s=t$,并且适当调整因式的次序后有
 \begin{equation}
 p_i(x)=c_iq_i(x)\quad (i=1,2,\cdots,s)~,
 \end{equation}
 其中,$c_i(i=1,2,\cdots,s)$ 都是非零常数。
\end{theorem}


附注:对多项式更广泛的讨论,可以参考小时百科《代数学进阶》部分的内容,尤其是环论开始的部分。这里推荐几例值得作为拓展阅读的词条:\textbf{爱森斯坦判别式}\upref{EsstCr}(\textbf{整环}\upref{Domain}上判定多项式不可约的条件),\textbf{唯一析因环}\upref{UFD}(满足因式分解唯一性的环,比如本词条涉及的整数环),\textbf{欧几里得环}\upref{EuRing},\textbf{多项式环}\upref{RPlynm}。

在域论和伽罗华理论部分,则会对多项式进行更深入的讨论。


\subsection{证明}此处用数学归纳法来证明

1.存在性证明:因为1次多项式都是不可约的,所以当多项式 $f(x)$ 的次数 $\mathrm{deg}\;f(x)=1$ 时分解式必存在。

假设 $\mathrm{deg}\;f(x)\leq n$ 时分解式存在,则在 $\mathrm{deg}\;f(x)=n+1$ 时,只考虑 $f(x)$ 为可约的情形,否则存在性就成立了。即
\begin{equation}
f(x)=f_1(x)f_2(x)~,
\end{equation}
其中 $f_1(x),f_2(x)$ 的次数都小于 $n$。由归纳假定 $f_1(x),f_2(x)$ 都可分解为数域 $\mathbb{F}$ 上有限个不可约多项式的乘积。把 $f_1(x),f_2(x)$ 的分解式合起来就得到 $f(x)$ 的一个分解式。由数学归纳法原理,次数大于1的多项式分解式必存在。

 2.唯一性证明:设 $f(x)$ 可分解为两种不可约多项式的乘积
 \begin{equation}
 \begin{aligned}
 &f(x)=p_1(x)p_2(x)\cdots p_s(x)~,\\
 &f(x)=q_1(x)q_2(x)\cdots q_t(x)~,
 \end{aligned}
 \end{equation}
 于是
 \begin{equation}\label{eq_UniFac_1}
 p_1(x)p_2(x)\cdots p_s(x)=q_1(x)q_2(x)\cdots q_t(x)~.
 \end{equation}
 对 $s$ 作归纳法。当 $s=1$ 时, $f(x)$ 是不可约多项式,由不可约多项式定义\autoref{def_RedPol_1}~\upref{RedPol}\upref{RedPol} 得,$s=t=1$,且
 \begin{equation}
 p_1(x)=q_1(x)~.
 \end{equation}
 
 假设当 $s\leq n$ 时,唯一性成立,则由\autoref{eq_UniFac_1} 
 \begin{equation}
 p_1(x)|q_1(x)q_2(x)\cdots q_t(x)~.
 \end{equation}
 由\autoref{the_RedPol_1}~\upref{RedPol}, $p_1(x)$ 必能整除其中一个,不妨设
 \begin{equation}
 p_1(x)|q_1(x)~,
 \end{equation}
 因为 $q_1(x)$ 也是不可约多项式,所以
 \begin{equation}\label{eq_UniFac_2}
 p_1(x)=c_1q_1(x)~,
 \end{equation}
 利用\autoref{eq_UniFac_2} 消去\autoref{eq_UniFac_1} ,就有
 \begin{equation}\label{eq_UniFac_3}
 p_2(x)\cdots p_s(x)=c_1q_2(x)\cdots q_t(x)~,
 \end{equation}
 由归纳假定,有 $s-1=t-1$,或者
 \begin{equation}\label{eq_UniFac_4}
 s=t~,
 \end{equation}
 并经适当排列后就有
 \begin{equation}\label{eq_UniFac_5}
 p_i(x)=c_iq_i(x)\quad (i=2,\cdots,s)~.
 \end{equation}
 联立\autoref{eq_UniFac_3} ,\autoref{eq_UniFac_4} ,\autoref{eq_UniFac_5} 就是所需证的结论,唯一性得证。

注意,因式分解定理并没有给出一个具体的将多项式分解为不可约乘积的方法。事实上,对一般数域上的多项式,不存在通用的因式分解的方法,甚至判定一个多项式是否可约都是非常困难的。
\subsection{标准分解式}
在多项式 $f(x)$ 分解式中,把不可约因式的高次项系数提出来,使它成为高次项系数为 1 的多项式,再把相同的不可约因式合并,这样得到的分解式称为 $f(x)$ 在数域 $\mathbb{F}$ 上的\textbf{标准分解式}。于是 $f(x)$ 的标准分解式成为
\begin{equation}
f(x)=ap_1^{r_1}(x)p_2^{r_2}(x)\cdots p_s^{r_s}(x)~.
\end{equation}
其中, $a$ 是 $f(x)$ 的高次项系数, $p_1(x),\cdots,p_s(x)$ 是两两不同的且最高次项系数为1的不可约多项式,而 $r_1,\cdots ,r_s$ 为正整数。

在带余除法\upref{DivAlg}一节中,我们知道 $c$ 是 $f(x)$ 的根的充要条件是 $(x-c)|f(x)$ \autoref{cor_DivAlg_2}~\upref{DivAlg} 。若 $x-c$ 是 $f(x)$ 的 $k$ 重根(即 $x-c$ 出现在因式中的次数),则当 $k=1$ ,称 $c$ 为 $f(x)$ 的\textbf{单根}, 当 $k>1$ ,称 $c$ 为 $f(x)$ 的\textbf{$k$ 重根}。显然,$\mathbb{F}[x]$ 中 $n$ 次多项式在数域 $\mathbb{F}$ 上的根不会多于 $n$ 个。

在复数中有个定理,即对任一复系数 $n$ 次多项式,在复数域 $\mathbb{C}$ 上恰有 $n$ 个根,由此可知,次数大于1的多项式在复数域上都是可约的,即复数域上不可约的多项式只能是1次多项式。
