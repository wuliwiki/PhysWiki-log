% 巴拿赫空间
% 巴拿赫空间|范数|收敛|柯西序列

\pentry{柯西序列 完备度量空间\upref{cauchy}}

\begin{definition}{巴拿赫空间}
完备的赋范空间叫做\textbf{巴拿赫空间(Banach space)}。
\end{definition}

\subsection{巴拿赫空间的例子}
\begin{example}{$\mathbb C^N$ 空间}
$N$ 维实空间 $\mathbb R^N$ 或者复空间 $\mathbb C^N$ 在任何范数之下都是完备的, 因此在任何范数之下都是巴拿赫空间。
\end{example}

\begin{example}{连续函数空间}
有限闭区间上的所有连续函数构成的空间记为 $X := C[a, b]$, 这是一个不可数维的空间。 若令范数为
$$
\|{f}\|:= \max_{a \leqslant x \leqslant b} \abs{f(x)}~,
$$
则它成为一个巴拿赫空间。 这范数下收敛的连续函数序列恰为一致收敛的连续函数序列。 而若赋予 $X$ 以 $L^p$ 范数
$$
\|f\|_{L^p}:=\left(\int_{\mathbb{R^N}}|f(x)|^pdx\right)^{1/p}~,
$$
则在此范数下它不是完备的。

更一般地, 对于任何紧 Hausforff 空间 $K$, 连续函数空间 $C(K)$ 在范数
$$
\|f\|:=\sup_{x\in K}|f(x)|~
$$
之下也是巴拿赫空间。
\end{example}

\begin{example}{$L^p$ 空间}
对于测度空间 $(\Omega,\mathcal{A},\mu)$, 定义可测函数的 $L^p$ 范数为
$$
\|f\|_{L^p(\mu)}=\left(\int_\Omega |f(x)|^pd\mu(x)\right)^{1/p}~,
$$
$$
\|f\|_{L^\infty(\mu)}=\text{ess sup}_{\Omega}|f|~.
$$
如果将几乎处处相等的函数视为相同, 则当 $1\leq p\leq\infty$ 时 $\|\cdot\|_{L^p(\mu)}$ 便是一个范数, 而满足 $\|f\|_{L^p(\mu)}$ 的可测函数的线性空间就是 $L^p(\mu)$。 它是完备的。 当 $\Omega=\mathbb{N}$, 而测度 $\mu$ 为普通的计数测度时, 可测函数就是通常的序列, 此时将空间记为 $l^p$, 而序列的范数是
$$
\|x\|_p=\left(\sum_{n=1}^\infty|x(n)|^p\right)^{1/p}~.
$$
本例内容可参考Bogachev, V. I. (2007). Measure theory (Vol. 1). Springer Science \& Business Media., 第三章。
\end{example}

\subsection{巴拿赫空间上的线性算子}
巴拿赫空间上有界算子的定义与赋范空间中一样。 不过, 在分析数学中, 也常常需要考虑不一定有界的线性算子。 一般来说, 对于两个巴拿赫空间 $X,Y$, 分析数学中最经常考虑的是\textbf{稠定算子 (densely defined operator)}, 即定义在 $X$ 的某个稠密子空间上的线性算子 $T$。 这个稠密子空间称为算子的定义域, 常记为 $\text{Dom}(T)$。 稠定算子不一定可以延拓为全空间上的有界算子。 如果算子 $T_1$ 是算子 $T$ 的延拓, 即 $\text{Dom}(T)\subset \text{Dom}(T_1)$ 且在 $\text{Dom}(T)$ 上有 $T_1=T$, 则写 $T\subset T_1$。

对于巴拿赫空间 $X,Y$ 之间的 (不一定有界的) 线性算子 $T$, 其图像 (graph) 定义为 $X\times Y$ 的子空间
$$
\text{G}(T):=\{(x,Tx):x\in \text{Dom}(T)\}~.
$$
在 $\text{Dom}(T)$ 上定义的范数 $[|x|]_T:=\|x\|_X+\|Tx\|_Y$ 称为图范数 (graph norm)。 如果 $\text{G}(T)$ 是闭子空间, 则称算子 $T$ 为\textbf{闭算子 (closed operator)}。 如果存在延拓 $T_1\supset T$ 使得 $T_1$ 为闭算子, 则 $T$ 称为\textbf{可闭化 (closable)} 的。 闭图像定理 (Closed graph theorem) \upref{BanThm}说明: 在巴拿赫空间上, 定义于全空间的闭算子是有界的。 谱理论就是针对闭算子展开的。

\subsection{巴拿赫空间的基本性质}
赋范线性空间在度量空间的意义下经过完备化之后成为巴拿赫空间。 完备的赋范空间在赋以等价范数后还是完备的。 可分的赋范线性空间的完备化空间还是可分的。 

一个赋范空间 $(X,\|\cdot\|)$ 是完备的, 当且仅当由 $\sum _{n=1}^{\infty }\|x_{n}\|<\infty $ 总可以推出 $\sum _{n=1}^{\infty }x_{n}$ 按照范数收敛。

如果 $(X,\|\cdot\|_X),(Y,\|\cdot\|_Y)$ 是巴拿赫空间, 那么其直积 $X\times Y$ 在范数 $\|(x,y)\|_{p}$ 之下也是巴拿赫空间。 有界线性算子空间 $\mathfrak{B}(X,Y)$ 在算子范数下也是巴拿赫空间。 如果 $X$ 巴拿赫空间, $M\subset X$ 是其闭子空间, 则商空间 $X/M$ 在商范数之下也是巴拿赫空间。 这里用到的定义见\upref{NormV}。

无穷维巴拿赫空间的代数维数 (即其中极大线性无关向量组的势) 一定是不可数的。 这是因为, 如果 $\{x_k\}\subset X$ 是任何序列, 则若命 $X_n$ 是 $\{x_1,...,x_n\}$ 张成的子空间, 那么每个 $X_n$ 都是真闭子空间。 按照贝尔纲定理, 它们的并集是第一纲集, 但巴拿赫空间作为完备度量空间是第二纲集, 所以不可能有 $X=\bigcup_n X_n$。

无穷维巴拿赫空间中的闭单位球 $\{\|x\|\leq1\}$ 在其范数拓扑下一定不是紧的。 详见里斯引理\upref{RiLem}。
