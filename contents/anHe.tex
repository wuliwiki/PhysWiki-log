% 反常霍尔效应
% 晶体|电子|霍尔效应
\pentry{电子运动的准经典模型\upref{cryele}}
\begin{issues}
\issueDraft
\end{issues}
一般霍尔效应的产生需要磁场,并且满带不出现霍尔效应。但是反常霍尔效应不需要这些条件。
\subsection{介绍}
我们知道一个布洛赫态可以写成:
\begin{equation}
\psi_{n,\boldsymbol{k}}=e^{i\boldsymbol{k}\cdot\boldsymbol{r}}u_{n,\boldsymbol{k}}
\end{equation}
其哈密顿量为:$\widehat{H_0}=-\frac{\hbar^2}{2m}\nabla^2+V(\boldsymbol{r})$,对应的能量是$E_{n,\boldsymbol{k}}$。其中$V(\boldsymbol{r})$是一个周期函数,有$V(\boldsymbol{r}+\boldsymbol{R})=V(\boldsymbol{r})$,$\boldsymbol{R}$是任意一个格矢。

外力作用下哈密顿量变成$\widehat{H}=\widehat{H_0}-\boldsymbol{F}\cdot\boldsymbol{r}$,则dt时间后,布洛赫态变成:
\begin{equation}
\psi(\boldsymbol{r},dt)=e^{-\frac{i\,\widehat{H}\,dt}{\hbar}}\psi_{n,\boldsymbol{k}} \approx \psi_{n,\boldsymbol{k}}-\frac{idt}{\hbar}(\widehat{H_0}-\boldsymbol{F}\cdot\boldsymbol{r})\psi_{n,\boldsymbol{k}}
=(1-i\frac{E_{n,\boldsymbol{k}}dt}{\hbar}+i\frac{\boldsymbol{F}\cdot\boldsymbol{r}dt}{\hbar})e^{i\boldsymbol{k}\cdot\boldsymbol{r}}u_{n,\boldsymbol{k}}
\approx e^{-\frac{i}{\hbar}E_{n,\boldsymbol{k}}dt}e^{i(\boldsymbol{k}+\frac{\boldsymbol{F}dt}{\hbar})\cdot \boldsymbol{r}}u_{n,\boldsymbol{k}}
\end{equation}
在推导电子运动的准经典模型时,我们忽略了$u_{n,\boldsymbol{k}}$到$u_{n,\boldsymbol{k}+\frac{\boldsymbol{F}dt}{\hbar}}$之间的变化(包络近似),从而得出了$d\boldsymbol{k}=\frac{\boldsymbol{F}dt}{\hbar}$的结论。现在我们不忽略它的变化,从而推导出反常霍尔效应来。

有:
\begin{equation}
\psi(\boldsymbol{r},dt)\approx e^{-\frac{i}{\hbar}E_{n,\boldsymbol{k}}dt}e^{i(\boldsymbol{k}+\frac{\boldsymbol{F}dt}{\hbar})\cdot \boldsymbol{r}}u_{n,\boldsymbol{k}}
=e^{-\frac{i}{\hbar}E_{n,\boldsymbol{k}}dt}e^{i(\boldsymbol{k}+\frac{\boldsymbol{F}dt}{\hbar})\cdot \boldsymbol{r}}(u_{n,\boldsymbol{k}}-u_{n,\boldsymbol{k}+\frac{\boldsymbol{F}}{\hbar}dt}+u_{n,\boldsymbol{k}+\frac{\boldsymbol{F}}{\hbar}dt})
\end{equation}
\begin{equation}
=e^{-\frac{i}{\hbar}E_{n,\boldsymbol{k}}dt}e^{i(\boldsymbol{k}+\frac{\boldsymbol{F}dt}{\hbar})\cdot \boldsymbol{r}}u_{n,\boldsymbol{k}+\frac{\boldsymbol{F}}{\hbar}dt}-e^{-\frac{i}{\hbar}E_{n,\boldsymbol{k}}dt}e^{i(\boldsymbol{k}+\frac{\boldsymbol{F}dt}{\hbar})\cdot \boldsymbol{r}}(u_{n,\boldsymbol{k}+\frac{\boldsymbol{F}}{\hbar}dt}-u_{n,\boldsymbol{k}})
=e^{-\frac{i}{\hbar}E_{n,\boldsymbol{k}}dt}\psi_{n,\boldsymbol{k}+\frac{\boldsymbol{F}dt}{\hbar}}-e^{-\frac{i}{\hbar}E_{n,\boldsymbol{k}}dt}e^{i(\boldsymbol{k}+\frac{\boldsymbol{F}dt}{\hbar})\cdot \boldsymbol{r}}(\nabla_{\boldsymbol{k}} u\cdot \frac{\boldsymbol{F}}{\hbar}dt)
\approx e^{-\frac{i}{\hbar}E_{n,\boldsymbol{k}}dt}\psi_{n,\boldsymbol{k}+\frac{\boldsymbol{F}dt}{\hbar}}-e^{i\boldsymbol{k}\cdot\boldsymbol{r}}\nabla_{\boldsymbol{k}} u\cdot \frac{\boldsymbol{F}}{\hbar}dt
\end{equation}
其中$\nabla_{\boldsymbol{k}} u$表示函数$u$在$\boldsymbol{k}$空间的梯度。最后一条等式的前一项即为经典理论的结果,后一项的导出只保留了dt的一阶小量。
\subsection{深入}
现在我们来研究一下$\nabla_{\boldsymbol{k}} u$。由上可知,有:
\begin{equation}
\widehat{H_0}e^{i\boldsymbol{k}\cdot\boldsymbol{r}}u_{n,\boldsymbol{k}}(\boldsymbol{r})
=E_{n,\boldsymbol{k}}e^{i\boldsymbol{k}\cdot\boldsymbol{r}}u_{n,\boldsymbol{k}}(\boldsymbol{r})
\end{equation}
所以有:
\begin{equation}\label{eq_anHe_1}
e^{-i\boldsymbol{k}\cdot\boldsymbol{r}}\widehat{H_0}e^{i\boldsymbol{k}\cdot\boldsymbol{r}}u_{n,\boldsymbol{k}}(\boldsymbol{r})
=E_{n,\boldsymbol{k}}u_{n,\boldsymbol{k}}(\boldsymbol{r})
\end{equation}
即$\widehat{H_\boldsymbol{k}}=e^{-i\boldsymbol{k}\cdot\boldsymbol{r}}\widehat{H_0}e^{i\boldsymbol{k}\cdot\boldsymbol{r}}$的本征函数是$u_{n,\boldsymbol{k}}(\boldsymbol{r})$。代入$\widehat{H_0}=-\frac{\hbar^2}{2m}\nabla^2+V(\boldsymbol{r})$,对于任意一个函数$f(\boldsymbol{r})$,即有:
\begin{equation}
\widehat{H_\boldsymbol{k}}f(\boldsymbol{r})=e^{-i\boldsymbol{k}\cdot\boldsymbol{r}}(-\frac{h^2}{2m}\nabla^2+V(\boldsymbol{r}))e^{i\boldsymbol{k}\cdot\boldsymbol{r}}f(\boldsymbol{r})
=V(\boldsymbol{r})f(\boldsymbol{r})+e^{-i\boldsymbol{k}\cdot\boldsymbol{r}}(-\frac{\hbar^2}{2m})\nabla(i\boldsymbol{k}e^{i\boldsymbol{k}\cdot\boldsymbol{r}}f(\boldsymbol{r})+e^{i\boldsymbol{k}\cdot\boldsymbol{r}}\nabla f(\boldsymbol{r}))
\end{equation}
\begin{equation}
=V(\boldsymbol{r})f(\boldsymbol{r})+e^{-i\boldsymbol{k}\cdot\boldsymbol{r}}(-\frac{\hbar^2}{2m})(-k^2e^{i\boldsymbol{k}\cdot\boldsymbol{r}}+2i\boldsymbol{k}e^{i\boldsymbol{k}\cdot\boldsymbol{r}}\nabla f(\boldsymbol{r})+e^{i\boldsymbol{k}\cdot\boldsymbol{r}}\nabla^2 f(\boldsymbol{r}))
=V(\boldsymbol{r})f(\boldsymbol{r})-\frac{\hbar^2}{2m}(-k^2+2i\boldsymbol{k}+\nabla^2)f(\boldsymbol{r})
\end{equation}
即$\widehat{H_\boldsymbol{k}}$的具体形式为
\begin{equation}
\widehat{H_\boldsymbol{k}}=\frac{\hbar^2}{2m}(-i\nabla+\boldsymbol{k})^2+V(\boldsymbol{r})
\end{equation}
当$\boldsymbol{k}$变化$\delta\boldsymbol{k}$时,即有:
\begin{equation}
\hat{H_{\boldsymbol{k}+\delta\boldsymbol{k}}}=\frac{\hbar^2}{2m}(-i\nabla+\boldsymbol{k})^2+V(\boldsymbol{r})+\frac{\hbar^2}{2m}((\boldsymbol{k}+\delta{\boldsymbol{k}})^2-k^2)+\frac{\hbar^2}{m}\delta\boldsymbol{k}\cdot(-i\nabla)
\end{equation}
等式右边第1、2项即为原先的$\widehat{H_\boldsymbol{k}}$,第3、4项可以视为微扰。其中第3项是能量微扰项,不作用在波函数上,第四项为$\boldsymbol{k}\cdot\boldsymbol{p}$微扰项,与动量相关(此处还说明了晶格动量$\hbar\boldsymbol{k}$与电子动量$\boldsymbol{p}$不是同一个东西)。