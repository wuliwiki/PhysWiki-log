% 天津大学 2011 年考研量子力学
% 考研|天津大学|量子力学|2011

\subsection{ }
\begin{enumerate}
\item 质量为 $m$,频率为 $\omega$ 的谐振子,初始时刻处于状态 $Ax\varPsi_{n}(x)$,求归一化系数 $A$,任意时刻的波函数以及坐标的平均值.
\item 写出历史上确定光的波粒二象性的主要实验,指出至少两个可以验证实物粒子具有波动性的实验.
\end{enumerate}
\subsection{ }
\begin{enumerate}
\item 设 $\hat{F}$ 是任意一个算符,尝试用其构造两个厄米算符.
\item 利用角动量各分量 $\hat{L}_{x}$,$\hat{L}_{y}$,$\hat{L}_{z}$ 之间的对易关系,证明:在任意分量的本征态下,其他两个分量的平均值为0.
\item 将指数 $e^{i\alpha \sigma_{y}}$ 算符写成 $A+i\sigma_{y}B$ 的形式,求出常数 $A$,$B$ 与 $\alpha$ 的关系.
\end{enumerate}
\subsection{ }
质量为 $m$ 的粒子在一维势场 $V_{x}=\leftgroup{
    & 0 \quad \frac{a}{2}\leq x \leq a\\
    & bx \quad 0\leq x\leq\frac{a}{2}\\
    &\infty \quad x<0,x>a
    }$ 中运动.
\begin{enumerate}
\item 写出微扰哈密顿量.
\item 求能量至二级修正,波函数至一级修正.
\end{enumerate}
\subsection{ }
质量为 $m$ 的粒子被限制在两个无限大平行板之间运动,设两个无限大平板间距为 $d$ 求体系的能级和波函数.
\subsection{ }
中子的自旋是 $\frac{1}{2}$,磁矩 $\vec{\mu}=g\cdot\vec{S}$,其中 $\vec{S}$ 是自旋算符,$g$ 是常数.
\begin{enumerate}
\item 若两个中子间有自旋相互作用 $a\vec{S_1}\cdot\vec{S_2}$,其中 $a$ 是常数,$\vec{S_1}$ 和 $\vec{S_2}$ 分别是两个中子的自旋角动量,求体系的能级和波函数.
\item 若三个中子的相互作用哈密顿为 $a(\vec{S_{1}}\cdot\vec{S_{2}}+\vec{S_{2}}\cdot\vec{S_{3}}+\vec{S_{3}}\cdot\vec{S_{1}})$,其中 $a$ 是常数,$\vec{S_{3}}$ 是第三个中子的自旋角动量,求体系的能级(只在自旋空间运算).
\end{enumerate}