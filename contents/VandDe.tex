% 范德蒙矩阵、范德蒙行列式
% license Xiao
% type Tutor


\pentry{秩\nref{nod_MatRnk},行列式的性质\nref{nod_DetPro}}{nod_7359}

\footnote{参考 Wikipedia \href{https://en.wikipedia.org/wiki/Vandermonde_matrix}{相关页面}。}\textbf{范德蒙矩阵(Vandermonde matrix)}是一种特殊的行列式和多项式相关。

\begin{definition}{}
范德蒙矩阵是一个 $n\times m$ 的\enref{矩阵}{Mat}, 定义为
\begin{equation}\label{eq_VandDe_1}
\mat V = 
\pmat{1 & x_1 & x_1^2 & \dots & x_1^{m-1}\\
1 & x_2 & x_2^2 & \dots & x_2^{m-1}\\
1 & x_3 & x_3^2 & \dots & x_3^{m-1}\\
\vdots & \vdots & \vdots & \ddots & \vdots\\
1 & x_n & x_n^2 & \dots & x_n^{m-1}}~.
\end{equation}
若 $\mat{\mat V}$ 是方阵($m = n$), 其\enref{行列式}{Deter}称为\textbf{范德蒙行列式(Vandermonde determinant)}。

一些文献中也把\autoref{eq_VandDe_1} 中的各列左右翻转, 即按照幂从大到小排列。
\end{definition}

可应用于多项式最小二乘法拟合(\autoref{sub_LstSqr_1}~\upref{LstSqr}) 以及多项式插值。


\subsection{性质}
当 $m \le n$ 时, \enref{矩阵的秩}{MatRnk}为 $m$ 当且仅当所有的 $x_i$ 各不相等。

当 $m \ge n$ 时, 矩阵的秩为 $n$ 当且仅当至少 $n$ 个 $x_i$ 各不相等。
\subsubsection{证明}
先证明 $m = n$ 时范德蒙矩阵满秩,即行列式不为零。

大小为 $n \times n$ 的范德蒙矩阵 $\mat V_n $的行列式:
\begin{equation}
\vmat {\mat V_n} = \vmat{1 & x_1 & x_1^2 & \dots & x_1^{m-1}\\
1 & x_2 & x_2^2 & \dots & x_2^{m-1}\\
1 & x_3 & x_3^2 & \dots & x_3^{m-1}\\
\vdots & \vdots & \vdots & \ddots & \vdots\\
1 & x_n & x_n^2 & \dots & x_n^{m-1}}~.  
\end{equation}
将 $\mat V_n$ 的每一列记成 $V_j $。

根据\enref{行列式的性质}{DetPro},从第 $n$ 列开始,对每一列依次进行列变换 $V_{j}\rightarrow \ V_{j}-x_1V_{j-1}  $ :
\begin{equation}
\vmat{\mat{V}_n} =
\vmat{1 & x_1-x_1 & x_1^2-x_1^2 & \dots & x_1^{m-1}-x_1^{m-1}\\
1 & x_2-x_1 & x_2^2-x_1x_2 & \dots & x_2^{m-1}-x_1x_2^{m-2}\\
\vdots & \vdots & \vdots & \ddots & \vdots \\
1 & x_n-x_1 & x_n^2-x_1x_n & \dots & x_n^{m-1}-x_1x_n^{m-2}}
=
\vmat{1 & 0 & 0 & \dots & 0\\
1 & x_2-x_1 & x_2(x_2-x_1) & \dots & x_2^{m-2}(x_2-x_1)\\
\vdots & \vdots & \vdots & \ddots & \vdots \\
1 & x_n-x_1 & x_n(x_n-x_1) & \dots & x_n^{m-2}(x_n-x_1)}
=\vmat{\mat{V^\prime}_{n-1}}\prod_\limits{j=2}^n (x_j-x_1) ~.
\end{equation}
其中,
\begin{equation}
\mat V^\prime_{n-1} = 
\pmat{
    1 & x_2 & x_2^2 & \dots & x_2^{m-2}\\
1 & x_3 & x_3^2 & \dots & x_3^{m-2}\\
\vdots & \vdots & \vdots & \ddots & \vdots\\
1 & x_n & x_n^2 & \dots & x_n^{m-2}
}~.
\end{equation}
也是一个范德蒙矩阵,只不过 $ x$ 的下标是从 $2$ 到 $n$。

由于 $x_i \neq x_j(i \neq j)$,我们知道只要 $\vmat{\mat V^\prime_{n-1}}\neq 0$ ,就有 $\vmat {\mat V_n}\neq 0$ 。因此,要证明 $\vmat {\mat V_n} \neq 0$,只需要证明 $\vmat {\mat V^\prime_2} \neq 0$。
而
\begin{equation}
\vmat{\mat V^\prime_2}=\vmat {
    1 & x_{n-1} \\
    1 & x_n
}= x_n - x_{n-1} \neq 0 ~.
\end{equation}
因此,我们证明了 $\vmat{\mat V_n}\neq 0$,甚至得到了其表达式:
\begin{equation}
\vmat{\mat V_n}=\prod_\limits{1\le i < j \le n }(x_j-x_i)~.
\end{equation}

当 $m \neq n$ 时,容易知道,矩阵的秩 $= \min \{m,n\}$。因此原命题得证。




