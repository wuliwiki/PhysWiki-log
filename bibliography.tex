% 每个部分请根据从入门到进阶排序
\begin{thebibliography}{99}
% 微积分
%=========================================
\bibitem{Thomax}
J. Hass, C. Heil, M. Weir, \textsl{Thomas' Cauculus} 14ed
\bibitem{同济高}
同济大学数学系. \textsl{高等数学} 第六版
% 线性代数
%=========================================
\bibitem{Axler}
Sheldon Axler. \textsl{Linear Algebra Done Right} 3ed
\bibitem{同济线}
同济大学数学系. \textsl{线性代数} 第五版
% 抽象代数
%=========================================
% 微分几何
%=========================================
\bibitem{梁书}
梁灿杉, 周杉. \textsl{微分几何与广义相对论} 第二版
% 概率与统计
%=========================================
% 偏微分方程和特殊函数
%=========================================
\bibitem{Arfken}
Arfken, Weber, Harris. \textsl{Mathematical Methods for Physicists - A Comprehensive Guide} 7ed
% 数学分析
%=========================================
\bibitem{Rudin}
Walter Rudin. \textsl{Principle of Mathematical Analysis}
% 泛函分析
%=========================================
\bibitem{Zeidler}
Eberhard Zeidler. \textsl{Applied Functional Analysis - Applications to Mathematical Physics}
% 力学
%=========================================
\bibitem{Goldstein}
Herbert Goldstein. \textsl{Classical Mechanics} 3ed
\bibitem{新力}
赵凯华, 罗蔚茵. \textsl{新概念物理教程 力学} 第二版
% 电动力学
%=========================================
\bibitem{GriffE}
David Griffiths, \textsl{Introduction to Electrodynamics}, 4ed
\bibitem{新电}
赵凯华, 陈熙谋. \textsl{新概念物理教程 电磁学} 第二版
% 量子力学
%=========================================
\bibitem{GriffQ}
David Griffiths, \textsl{Introduction to Quantum Mechanics}, 4ed
\bibitem{新量}
赵凯华, 罗蔚茵. \textsl{新概念物理教程 量子物理} 第二版
\bibitem{Shankar}
R. Shankar. \textsl{Principles of Quantum Mechanics} 2ed
\bibitem{Merzbacher}
Eugen Merzbacher. \textsl{Quantum  Mechanics} 3ed
\bibitem{Sakurai}
J.J. Sakurai. \textsl{Modern Quantum Mechanics} Revised Edition
% 统计力学
%=========================================
\bibitem{Schroeder}
Daniel V. Schroeder, \textsl{An Introduction to Thermal Physics}
% 宇宙学
%=========================================
\bibitem{Peebles}
P. Peebles \textsl{Principles of Physical Cosmology}
% 量子场论
%=========================================
\bibitem{Peskin}
Peskin. \textsl{An introduction To Quantum Field Theory}
% 其他
\bibitem{PhysWiki}
小时物理百科志愿者. \textsl{小时物理百科}. \href{http://wuli.wiki}{http://wuli.wiki}. 
\end{thebibliography}
