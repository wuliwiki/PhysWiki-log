% 平面波的球谐展开

\pentry{球谐函数\upref{SphHar}}

复数形式的平面波可以展开为\footnote{易证这里的复共轭可以加在任意一个球谐函数上.}
\begin{equation}
\E^{\I \vec k \vdot \vec r} = 4\pi \sum_{l=0}^{\infty} \sum_{m=-l}^l \I^l j_l(kr) Y_{lm}(\uvec k) Y_{lm}^*(\uvec r)
\end{equation}
\begin{equation}
\E^{-\I \vec k \vdot \vec r} = 4\pi \sum_{l=0}^{\infty} \sum_{m=-l}^l \I^{-l} j_l(kr) Y_{lm}(\uvec k) Y_{lm}^*(\uvec r)
\end{equation}

\subsection{球谐展开函数的傅里叶变换}

任意函数可以表示为
\begin{equation}
f(\vec r) = \sum_{l,m} R_{lm}(r) Y_l^m(\uvec r)
\end{equation}
则傅里叶变换为
\begin{equation}
\frac{1}{(2\pi)^{3/2}} \int f(\vec r) \E^{-\I \vec k \vec r} \dd[3]{r}
= \sqrt{\frac{2}{\pi}} \sum_{l,m} \I^{-l} Y_l^m(\uvec k) \int_0^{+\infty} R_{lm}(r) j_l(kr) r^2 \dd{r}
\end{equation}
