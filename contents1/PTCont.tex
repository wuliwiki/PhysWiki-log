%含连续态的微扰理论%图未完成

\pentry{不含时微扰理论}% 未完成

一般的束缚+连续微扰理论. 假设我们有两个束缚态和连续态, 总的波函数可以写成
 \begin{equation}
\left| \psi  \right\rangle  = {C_1}\ket{1}  + {C_2}\ket{2}  + \int {\phi (\vec k)\ket{\vec k}{\D ^3}k} 
\end{equation}
令归一化条件为 $\bra{1}\ket{1} = \bra{2}\ket{2} = 1$,  $\bra{\vec k'}\ket{\vec k}  = {\delta ^3}(\vec k' - \vec k)$,  $otherwise = 0$. 

 $\mat H'$  矩阵可以想象成是这个样子的
\begin{figure}[h]
\centering
\includegraphics[width=5cm]{./figures/PTCont.pdf}
\caption{$\mat H'$ 矩阵的结构} 
\end{figure}

方格子代表 ${C_{ij}} = \left\langle i \right|H'\left| j \right\rangle $,  横条代表 ${H_{i\vec k'}} = \left\langle i \right|H'\left| {k'} \right\rangle $,  纵条代表 ${H_{\vec kj}} = \left\langle {\vec k} \right|H'\left| j \right\rangle $. 

与离散的情况相似, 微扰理论的推导方法是先把 $\left| \psi  \right\rangle $ 代入含时薛定谔方程, 然后两边分别左乘基底 $\left\langle i \right|$ 和 $\left\langle {\vec k} \right|$,  注意后者这里要使用动量归一化条件把对 $\vec k$ 的积分消去. 微扰递推公式为
 \begin{equation}
C_i^{(n + 1)}(t) = \frac{1}{{i\hbar }}\int {\D t'\left( {\sum\limits_{j \ne i} {{{H'}_{ij}}C_j^{(n)}}  + \int {{H_{i\vec k'}}{\phi ^{(n)}}(\vec k'){\D ^3}k'} } \right)} 
\end{equation}
 \begin{equation}
 {\phi ^{(n + 1)}}(\vec k) = \frac{1}{{i\hbar }}\int {\D t'\left( {\sum\limits_j {{{H'}_{\vec kj}}C_j^{(n)}}  + \int {{H_{\vec k\vec k'}}{\phi ^{(n)}}(\vec k'){\D ^3}k'} } \right)} 
\end{equation}
关于 $\ket{\vec k}$  的定义, 若势能函数是局部的, 那么在无穷远处波函数是平面波, 由此来定义 $\vec k$.  这样, 在计算 $\bra{\vec k'}\ket{\vec k}$ 时, 由于积分范围是无穷, 可以忽略局部势能对波函数的影响, 所以归一化系数就是 ${\E^{\I \vec k\vec r}} = {\E^{\I{k_x}x}}{\E^{\I{k_y}y}}{\E^{\I{k_z}z}}$ 的归一化系数即 $1/{(2\pi )^{3/2}}$. 
 
