% 动量定理 动量守恒

\pentry{动量\ 动量定理(单个质点)\upref{PLaw1}, 质点系\upref{PSys}}
\subsection{结论}
系统总动量的变化率等于合外力,所以合外力为零时系统总动量守恒.

\subsection{推导}
任何系统都可以看做质点系,质点系中第 $i$ 个质点可能受到系统内力 $\vec F_i^{in}$ 或系统外力 $\vec F_i^{out}$. 由单个质点的动量定理\upref{PLaw1},
\begin{equation}
\dv{t} \vec p_i = \vec F_i^{in} + \vec F_i^{out}
\end{equation}
总动量的变化率为
\begin{equation}
\dv{\vec P}{t} = \sum_i \dv{t} \vec p_i  = \sum_i \vec F_i^{in}  + \sum_i \vec F_i^{out}
\end{equation}
由“质点系\upref{PSys}” 中的结论, 上式右边第一项求和是系统合内力, 恒为零. 于是我们得到系统的动量定理
\begin{equation}
\dv{\vec P}{t} = \sum_i \vec F_i^{out}
\end{equation}

