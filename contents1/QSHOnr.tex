% 简谐振子升降算符归一化
%25 min
\pentry{简谐振子(量子)}
首先要提醒的是,一般算符满足的一个条件是 $\braket{g}{\hat Q f}=\braket{\hat Q^\dagger g}{ f}$ . 但是对于厄米算符, ${\hat Q^\dagger } = \hat Q$,  所以有 $\braket{g}{\hat Q f} = \braket{\hat Q g}{f}$ .

对于谐振子的升降算符 ${\uvec a_ \pm } = {1}/{{\sqrt {2m\omega \hbar } }}\left( {m\omega \uvec x \mp i\uvec p} \right)$, 有
\begin{equation}\begin{aligned}
{\uvec a_ - }{\uvec a_ + } &= \frac{1}{{2m\omega \hbar }}\left( {{m^2}{\omega ^2}{{\uvec x}^2} + {{\uvec p}^2} - im\omega {\kern 1pt} {\kern 1pt} [\uvec x,\uvec p]} \right) \\
&= \frac{1}{{\omega \hbar }}\left( {\frac{1}{{2m}}\left( {{m^2}{\omega ^2}{{\uvec x}^2} + {{\uvec p}^2}} \right) + \frac{{\omega \hbar }}{2}} \right) \\
&= \frac{1}{{\omega \hbar }}\uvec H + \frac{1}{2}
\end{aligned}\end{equation}
\begin{equation}\begin{aligned}
{\left| {{{\uvec a}_ + }{\varphi _n}} \right|^2} &= \left\langle {{{{\uvec a}_ + }{\varphi _n}}}
\mathrel{\left | {\vphantom {{{{\uvec a}_ + }{\varphi _n}} {{{\uvec a}_ + }{\varphi _n}}}}
\right. \kern-\nulldelimiterspace}
{{{{\uvec a}_ + }{\varphi _n}}} \right\rangle = \left\langle {{{\varphi _n}}}
\mathrel{\left | {\vphantom {{{\varphi _n}} {{{\uvec a}_ - }{{\uvec a}_ + }{\varphi _n}}}}
\right. \kern-\nulldelimiterspace}
{{{{\uvec a}_ - }{{\uvec a}_ + }{\varphi _n}}} \right\rangle  \\
&= \left\langle {{{\varphi _n}}}
\mathrel{\left | {\vphantom {{{\varphi _n}} {\left( {\frac{1}{{\omega \hbar }}\uvec H + \frac{1}{2}} \right){\varphi _n}}}}
\right. \kern-\nulldelimiterspace}
{{\left( {\frac{1}{{\omega \hbar }}\uvec H + \frac{1}{2}} \right){\varphi _n}}} \right\rangle \\
&= \left( {n + \frac{1}{2}} \right) + \frac{1}{2} = n + 1
\end{aligned}\end{equation}
所以有 ${\uvec a_ + }{\varphi _n} = \sqrt {n + 1}\, {\varphi _{n + 1}}$ (同理 ${\uvec a_ - }{\varphi _n} = \sqrt n \,{\varphi _{n - 1}}$ ).\\
再次提醒,归一化系数后面可以加上任意相位因子 ${e^{i\theta }}$, 同样能满足归一化条件,但一般省略.\\