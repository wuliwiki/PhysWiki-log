% \Gamma 函数
% Gamma 函数
% 25 min

\pentry{定积分\upref{DefInt} }

\subsection{结论}
当 $x$ 取实数且 $x>-1$ 时,可以定义连续的阶乘函数为
\begin{equation}
x! \equiv \Gamma (x + 1) = \int_0^{ + \infty } {{t^x}{e^{ - t}}\D t} 
\end{equation}
递推关系仍为
\begin{equation}\label{Gamma_eq2}
x!=x(x-1)! \qquad (x>0)
\end{equation}
且 $(-1/2)!=\sqrt{\pi}$,  $0! = 1$.

\subsection{推导}

首先定义 $\Gamma$ \textbf{(Gamma)} 函数为
\begin{equation}
\Gamma \left( x \right) = \int_0^{ + \infty } {{t^{x - 1}}{e^{ - t}}\D t} 
\end{equation}
当 $x \le 0$ 时该积分在 $x=0$ 处不收敛,以下仅讨论 $x$ 为正实数的情况\footnote{事实上,自变量为负实数(非整数)时,$\Gamma$ 函数有另一种定义,这里不讨论.}.

我们现在验证当 $x$ 取正整数时,新定义的阶乘 $x! = \Gamma(x+1)$ 与原来的定义 $x! = x(x-1)..1$ 相同.首先
\begin{equation}\label{Gamma_eq4}
0! = \Gamma \left( 1 \right) = \int_0^{ + \infty } {e^{ - t}\D t} = 0-(-1) = 1
\end{equation}

使用分部积分法\upref{IntBP},令 ${t^x}$ 为“求导项”, ${e^{ - t}}$ 为积分项,可得递推公式\footnote{该证明仅对 $x>0$ 适用,这有这样才有 $0^x e^{-0}=0$, 使第三个等号成立.}(\autoref{Gamma_eq2})
\begin{equation}\label{Gamma_eq5}
\begin{aligned}
x! &= \Gamma ( x+1) = \int_0^{ + \infty } {{t^x}{e^{ - t}}\D t} =  - {t^x}{e^{ - t}}|_{t = 0}^{t =  + \infty } + \int_0^{ + \infty } {x{t^{x - 1}}{e^{ - t}}\D t} \\
&= x\int_0^{ + \infty } {{t^{x - 1}}{e^{ - t}}\D t} = x\Gamma (x) = x(x-1)!
\end{aligned} \end{equation} 
由递推\autoref{Gamma_eq5} 和初值\autoref{Gamma_eq4}, 对任意正整数 $n$ 有
\begin{equation}
n! = n(n-1)! = n(n-1)(n-2)!... = n(n-1)...1
\end{equation}

再来看半整数的阶乘,我们讨论范围内的最小半整数为 $-1/2$ 
\begin{equation}
(-\frac 12) ! = \int_0^{ + \infty } {\frac{{{e^{ - x}}}}{{\sqrt x }}dx} = \sqrt{\pi}
\end{equation}
该积分的计算超出本书范围,可使用 Wolfram Alpha 或 Mathematica 计算.%未完成
对任意大于零的半整数 $n/2$,有
\begin{equation}
\frac{n}{2}! = \frac{n}{2} \left(\frac{n}{2}-1 \right)! = \frac{n}{2} \left(\frac{n}{2}-1\right) \dots \frac 12 \left(-\frac 12\right) ! = \frac{n}{2} \left(\frac{n}{2}-1\right) \dots \frac 12 \sqrt{\pi}
\end{equation}














