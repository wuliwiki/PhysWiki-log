%柱坐标系中的拉普拉斯方程

\pentry{分离变量法简介,柱坐标的拉普拉斯算符}

\subsection{结论}

柱坐标中的径向方程为贝赛尔方程
 \begin{equation}
x\frac{\D}{{\D x}}\left( {x\frac{{\D y}}{{\D x}}} \right) + \left( {{x^2} - {m^2}} \right)y = 0
\end{equation}
其中 $x = lr$ 

\subsection{分离变量法}
柱坐标系中的拉普拉斯方程为
\begin{equation}
\frac{1}{r} \pdv{r} \left( {r\pdv{u}{r}} \right) + \frac{1}{{{r^2}}} \pdv[2]{u}{\theta} + \pdv[2]{u}{z} = 0
\end{equation}
令 $u = R\left( r \right)\Phi \left( \theta  \right)Z\left( z \right)$,   代入方程得
\begin{equation}
\frac{1}{{rR}}\pdv{r} \left( {r\pdv{R}{r}} \right) + \frac{1}{{{r^2}\Phi }} \pdv[2]{\Phi}{\theta} + \frac{1}{Z} \pdv[2]{Z}{z} = 0
\end{equation}
前两项只是 $r$ 和 $\theta $ 的函数, 第三项只是 $z$ 的函数, 所以它们分别为常数. 令
\begin{equation}\label{CylLap_4}
\frac{1}{{\rm Z}} \pdv[2]{Z}{z} = {l^2}
\end{equation}
\begin{equation}
\text{ 则前两项为} \frac{1}{{rR}} \pdv{r} \left( {r\pdv{R}{r}} \right) + \frac{1}{{{r^2}\Phi }} \pdv[2]{\Phi}{\theta} =  - {l^2} \text{. } 
\end{equation}

为了继续分离 $r$ 和 $\theta$, 两边乘以 ${r^2}$,   则左边第二项只是关于 $\theta $  的函数, 剩下的部分只是关于r的函数. 令
\begin{equation}\label{CylLap_6}
\frac{1}{\Phi }\frac{{{\D ^2}\Phi }}{{\D {\theta ^2}}} =  - {m^2}
\end{equation}
则剩下的部分为 ${m^2}$,   即
  \begin{equation}\label{CylLap_7}
r\frac{\D }{{\D r}}\left( {r\frac{{\D R}}{{\D r}}} \right) + \left( {{l^2}{r^2} - {m^2}} \right)R = 0
\end{equation}

令 $x = lr$,   $y\left( x \right) = R\left( r \right)$ 则
\begin{equation}\label{CylLap_8} 
x\frac{\D }{{\D x}}\left( {x\frac{{\D y}}{{\D x}}} \right) + \left( {{x^2} - {m^2}} \right)y = 0
\end{equation}
到此为止, 三个变量已经完全分离, 各自的微分方程为\autoref{CylLap_4}, \autoref{CylLap_6},\autoref{CylLap_7}.

$Z\left( z \right)$ 的通解为 ${C_1}{\E^{lz}} + {C_2}{\E^{ - lz}}$,   $\Phi \left( \theta  \right)$ 的通解为 ${\E^{\I m\theta }}$.   \autoref{CylLap_7} 的解不能用有限的初等函数表示, \autoref{CylLap_8} 为贝赛尔方程的标准形式(见贝赛尔方程).

需要注意的是, 贝赛尔函数的阶数 $m$ 是角向方程 $\frac{1}{\Phi }{{{\D^2}\Phi }}/{{\D{\theta ^2}}} =  - {m^2}$ 的参数, 而不是径向方程的参数l. 参数l被包含在自变量 $x$ 中.
 
