% 电容

\subsubsection{单个导体的电容量}

一个导体的电容等于电荷量除以电势

\begin{exam}{导体球的电容}
若规定无穷远处为零势点, 由高斯定理, 半径为 $R$, 带电为 $Q$ 的导体球的电势为 $V = Q/(4\pi\epsilon_0 R)$, 所以其电容为
\begin{equation}
C = \frac{Q}{V} = 4\pi\epsilon_0 R
\end{equation}
\end{exam}


\subsubsection{两导体之间的电容量}
若两导体带等量异种电荷 $\pm Q$, 电势差为 $V$, 则两导体间的电容量同样被定义为
\begin{equation}
C = \frac{Q}{V}
\end{equation}
现在来证明两导体间的电容只与他们的形状, 相对位置以及空间中的电介质分布有关, 而与电荷量无关. 这就要求证明电势差始终与电荷 $Q$ 成正比, 如果我们假设量导体表面的电荷面密度 $\sigma$ 始终与 $Q$ 成正比(证明见%未完成, 参考 Griffiths
, 那么由库伦定律% 引用公式未完成,引用对 \rho 三重积分的那条
可知空间中任意一点的场强也与 $Q$ 成正比. 而电势差
\begin{equation}
V = \int_+^- \vec E(r) \cdot \dd{\vec l}
\end{equation}
(式中的 “$\pm$” 分别代表带电荷量为 $\pm Q$ 的导体表面上的一点, 积分路径任意选取)和场强成正比也就是和 $Q$ 成正比. 证毕.

\begin{exam}{平行板电容器}
电容器中最常见也是最基本的模型就是平行板电容器, 我们假设空间中存在均匀的电介质, 介电常数为 $\epsilon$ 空间中两块面积为 $S$ 的方形导体板相距为 $d$ 平行放置, 带电量分别是 $\pm Q$, 如果忽略边缘效应(即假设只有两板之间的长方体空间中存在匀强电场), 由高斯定理,%引用例题未完成
两板之间的电场为 $E = {\sigma}/{\epsilon} = Q/(s\epsilon)$, 电势差为 $V = Ed = Qd/(s\epsilon)$, 所以电容等于
\begin{equation}
C = \frac{Q}{V} = \epsilon \frac sd
\end{equation}
可见在相同介质中, 平行板电容器的电容量与板的面积成正比, 与距离成反比, 比例系数为 $\epsilon$. 这就是为什么 $\epsilon$ 也被称为电容率. 当不存在电介质时, $\epsilon = \epsilon_0$, 所以 $\epsilon_0$ 被称为真空中的电容率.
\end{exam}